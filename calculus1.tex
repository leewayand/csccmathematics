\documentclass[10pt,handout,twocolumn,twoside,wordchoicegiven]{xourse}


\graphicspath{
  {./}
  {ximeraTutorial/}
  {basicPhilosophy/}
}

\newcommand{\mooculus}{\textsf{\textbf{MOOC}\textnormal{\textsf{ULUS}}}}


\usepackage{tkz-euclide}\usepackage{tikz}
\usepackage{tikz-cd}
\usetikzlibrary{arrows}
\tikzset{>=stealth,commutative diagrams/.cd,
  arrow style=tikz,diagrams={>=stealth}} %% cool arrow head
\tikzset{shorten <>/.style={ shorten >=#1, shorten <=#1 } } %% allows shorter vectors

\usetikzlibrary{backgrounds} %% for boxes around graphs
\usetikzlibrary{shapes,positioning}  %% Clouds and stars
\usetikzlibrary{matrix} %% for matrix
\usepgfplotslibrary{polar} %% for polar plots
\usepgfplotslibrary{fillbetween} %% to shade area between curves in TikZ
\usetkzobj{all}
\usepackage[makeroom]{cancel} %% for strike outs
%\usepackage{mathtools} %% for pretty underbrace % Breaks Ximera
%\usepackage{multicol}
\usepackage{pgffor} %% required for integral for loops



%% http://tex.stackexchange.com/questions/66490/drawing-a-tikz-arc-specifying-the-center
%% Draws beach ball
\tikzset{pics/carc/.style args={#1:#2:#3}{code={\draw[pic actions] (#1:#3) arc(#1:#2:#3);}}}



\usepackage{array}
\setlength{\extrarowheight}{+.1cm}
\newdimen\digitwidth
\settowidth\digitwidth{9}
\def\divrule#1#2{
\noalign{\moveright#1\digitwidth
\vbox{\hrule width#2\digitwidth}}}
























%%This is to help with formatting on future title pages.
\newenvironment{sectionOutcomes}{}{}


\pdfOnly{\usepackage{printStyles/lulu1}}

\logo{logos/calculus1Logo/logo.png}

\title{CSCC Calculus 1}
\begin{document}
\maketitle

\setcounter{tocdepth}{2}
\begin{onlineOnly}
%% How to use
%\chapterstyle
\activity{ximeraTutorial/titlePage.tex}
%\sectionstyle
\activity{ximeraTutorial/howToUseXimera.tex}
\activity{ximeraTutorial/howIsMyWorkScored.tex}
\end{onlineOnly}

%\part{Functions, limits, and continuity}
%\part{Content for the First Exam}


%% Understanding Functions
\chapterstyle
\activity{understandingFunctions/titlePage.tex}
\sectionstyle
\activity{understandingFunctions/breakGround.tex}
\activity{understandingFunctions/digInForEachInputExactlyOneOutput.tex}
\activity{understandingFunctions/digInCompositionOfFunctions.tex}
\activity{understandingFunctions/digInInversesOfFunctions.tex}




%% Review of famous functions
\chapterstyle
\activity{reviewOfFamousFunctions/titlePage.tex}
\sectionstyle
\activity{reviewOfFamousFunctions/breakGround.tex}
\activity{reviewOfFamousFunctions/digInPolynomialFunctions.tex}
\activity{reviewOfFamousFunctions/digInRationalFunctions.tex}
\activity{reviewOfFamousFunctions/digInTrigonometricFunctions.tex}
\activity{reviewOfFamousFunctions/digInExponentialAndLogarithmeticFunctions.tex}




%% What is a limit
\chapterstyle
\activity{whatIsALimit/titlePage.tex}
\sectionstyle
\activity{whatIsALimit/breakGround.tex}
\activity{whatIsALimit/digInWhatIsALimit.tex}
\activity{whatIsALimit/digInContinuity.tex}





%% Limit Laws
\chapterstyle
\activity{limitLaws/titlePage.tex}
\sectionstyle
\activity{limitLaws/breakGround.tex}
\activity{limitLaws/digInLimitLaws.tex}
\activity{limitLaws/digInTheSqueezeTheorem.tex}





%% Indeterminant forms
\chapterstyle
\activity{indeterminateForms/titlePage.tex}
\sectionstyle
\activity{indeterminateForms/breakGround.tex}
\activity{indeterminateForms/digInLimitsOfTheFormZeroOverZero.tex}
\activity{indeterminateForms/digInLimitsOfTheFormNonZeroOverZero.tex}




%%%%%%%%%%%%%%%%%%%%%%%%%%%%%%%%%%%%%%%%%%%%%%%%%%%%%%%%%%%%%%%%%%%%%%%%%%%%%%
%\activity{samplePractice/samplePractice.tex}
%%%%%%%%%%%%%%%%%%%%%%%%%%%%%%%%%%%%%%%%%%%%%%%%%%%%%%%%%%%%%%%%%%%%%%%%%%%%%%






%% Using limits to detect asymptotes
\chapterstyle
\activity{asymptotesAsLimits/titlePage.tex}
\sectionstyle
\activity{asymptotesAsLimits/breakGround.tex}
\activity{asymptotesAsLimits/digInVerticalAsymptotes.tex}
\activity{asymptotesAsLimits/digInHorizontalAsymptotes.tex}
\activity{asymptotesAsLimits/digInSlantAsymptotes.tex}





%% The intermediate value theorem
\chapterstyle
\activity{continuity/titlePage.tex}
\sectionstyle
\activity{continuity/breakGround.tex}
\activity{continuity/digInContinuityOfPiecewiseFunctions.tex}
\activity{continuity/digInTheIntermediateValueTheorem.tex}


%%%%%%%%%%%%%%%%%%%%%%%%%%%%%%%%%%%%%%%%%%%%%%%%%%%%%%%%%%%%%%%%%%%%%%%%%%%%%%
%\activity{exams/calculus1/examReviews/examOneReview.tex}
%%%%%%%%%%%%%%%%%%%%%%%%%%%%%%%%%%%%%%%%%%%%%%%%%%%%%%%%%%%%%%%%%%%%%%%%%%%%%%

%\part{Content for the Second Exam}






%% An application of limits
\chapterstyle
\activity{anApplicationOfLimits/titlePage.tex}
\sectionstyle
\activity{anApplicationOfLimits/breakGround.tex}
\activity{anApplicationOfLimits/digInInstantaneousVelocity.tex}

%\part{Derivatives}

















\pdfOnly{\printindex}
\end{document}
