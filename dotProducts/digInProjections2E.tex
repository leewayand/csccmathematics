\documentclass{ximera}


\graphicspath{
  {./}
  {ximeraTutorial/}
  {basicPhilosophy/}
}

\newcommand{\mooculus}{\textsf{\textbf{MOOC}\textnormal{\textsf{ULUS}}}}


\usepackage{tkz-euclide}\usepackage{tikz}
\usepackage{tikz-cd}
\usetikzlibrary{arrows}
\tikzset{>=stealth,commutative diagrams/.cd,
  arrow style=tikz,diagrams={>=stealth}} %% cool arrow head
\tikzset{shorten <>/.style={ shorten >=#1, shorten <=#1 } } %% allows shorter vectors

\usetikzlibrary{backgrounds} %% for boxes around graphs
\usetikzlibrary{shapes,positioning}  %% Clouds and stars
\usetikzlibrary{matrix} %% for matrix
\usepgfplotslibrary{polar} %% for polar plots
\usepgfplotslibrary{fillbetween} %% to shade area between curves in TikZ
\usetkzobj{all}
\usepackage[makeroom]{cancel} %% for strike outs
%\usepackage{mathtools} %% for pretty underbrace % Breaks Ximera
%\usepackage{multicol}
\usepackage{pgffor} %% required for integral for loops



%% http://tex.stackexchange.com/questions/66490/drawing-a-tikz-arc-specifying-the-center
%% Draws beach ball
\tikzset{pics/carc/.style args={#1:#2:#3}{code={\draw[pic actions] (#1:#3) arc(#1:#2:#3);}}}



\usepackage{array}
\setlength{\extrarowheight}{+.1cm}
\newdimen\digitwidth
\settowidth\digitwidth{9}
\def\divrule#1#2{
\noalign{\moveright#1\digitwidth
\vbox{\hrule width#2\digitwidth}}}
























%%This is to help with formatting on future title pages.
\newenvironment{sectionOutcomes}{}{}

\author{Bart Snapp and Jim Talamo}


\outcome{Define orthogonal decomposition.}
\outcome{Define orthogonal and scalar projections.}
\outcome{Use the dot product in applied settings.}

\title[Dig-In:]{Projections and orthogonal decomposition}

\begin{document}
\begin{abstract}
 Projections tell us how much of one vector lies in the direction of another and are important in physical applications.
\end{abstract}
\maketitle


\section{Projections and components}

\subsection{Projections}
One of the major uses of the dot product is to let us \textit{project}
one vector in the direction of another. Conceptually, we are looking
at the ``shadow'' of one vector projected onto another, sort of like
in the case of a sundial.
\begin{image}%%https://commons.wikimedia.org/wiki/File:Perceton_sundial_-_detail.JPG
  \includegraphics{sundial.jpg}
\end{image}
In essence we imagine the ``sun'' directly over a vector, casting a shadow onto another vector.
\begin{image}
  \begin{tikzpicture}
    \foreach \angle in { 270,280,...,450 }{
      \draw [ultra thick, yellow!70!orange, rotate around={\angle:(3,4)}]
      (3,3.5)--(3,3);
    };
    \draw[very thick, penColor,->] (0,0) -- (5,0);
    \draw[ultra thick,penColor4,->] (0,0) -- (3,2);
    \draw[ultra thick,penColor2,->] (0,0) -- (3,0);
    \draw[ultra thick, yellow!50!orange] (2.5,4) arc (180:360:.5);

    \draw[decoration={brace,mirror,raise=.2cm},decorate,thin] (0,0)--(3,0);
    \node at (1.5,-.5) {projection};

  \end{tikzpicture}
\end{image}

















































\end{document}
