\documentclass{ximera}


\graphicspath{
  {./}
  {ximeraTutorial/}
  {basicPhilosophy/}
}

\newcommand{\mooculus}{\textsf{\textbf{MOOC}\textnormal{\textsf{ULUS}}}}


\usepackage{tkz-euclide}\usepackage{tikz}
\usepackage{tikz-cd}
\usetikzlibrary{arrows}
\tikzset{>=stealth,commutative diagrams/.cd,
  arrow style=tikz,diagrams={>=stealth}} %% cool arrow head
\tikzset{shorten <>/.style={ shorten >=#1, shorten <=#1 } } %% allows shorter vectors

\usetikzlibrary{backgrounds} %% for boxes around graphs
\usetikzlibrary{shapes,positioning}  %% Clouds and stars
\usetikzlibrary{matrix} %% for matrix
\usepgfplotslibrary{polar} %% for polar plots
\usepgfplotslibrary{fillbetween} %% to shade area between curves in TikZ
\usetkzobj{all}
\usepackage[makeroom]{cancel} %% for strike outs
%\usepackage{mathtools} %% for pretty underbrace % Breaks Ximera
%\usepackage{multicol}
\usepackage{pgffor} %% required for integral for loops



%% http://tex.stackexchange.com/questions/66490/drawing-a-tikz-arc-specifying-the-center
%% Draws beach ball
\tikzset{pics/carc/.style args={#1:#2:#3}{code={\draw[pic actions] (#1:#3) arc(#1:#2:#3);}}}



\usepackage{array}
\setlength{\extrarowheight}{+.1cm}
\newdimen\digitwidth
\settowidth\digitwidth{9}
\def\divrule#1#2{
\noalign{\moveright#1\digitwidth
\vbox{\hrule width#2\digitwidth}}}
























%%This is to help with formatting on future title pages.
\newenvironment{sectionOutcomes}{}{}




\author{Bart Snapp}


\outcome{State the definition of a vector.}
\outcome{Work with vectors in two or three dimensions. }
\outcome{Multiply vectors by scalars.}
\outcome{Add and subtract vectors.}
\outcome{Calculate the magnitude of a vector.}
\outcome{Find unit vectors.}
\outcome{Use vectors in applied settings.}


\title[Dig-In:]{Vectors}

\begin{document}
\begin{abstract}
  Vectors are lists of numbers that denote direction and magnitude.
\end{abstract}
\maketitle


\section{The idea of vectors}

The most successful textbook that was ever written was
\link[\textit{Euclid's
    Elements}]{https://mathcs.clarku.edu/~djoyce/java/elements/elements.html}. While
you are surely skeptical of this claim, and it is \textit{good} to be
skeptical, consider this: \textit{Euclid's Elements} was used (in
various editions) as a primary mathematics textbook for nearly 2000
years. There are few textbooks (if any) that can share this
claim. However, \textit{Euclid's Elements} does have its shortcomings. 
Euclid defines a point as ``that which has no part.'' Many people 
(including this author) find this to be 
a pretty confusing definition. What does Euclid mean by this statement? 
However, from our modern viewpoint, a point is an
ordered list of numbers, like
\[
(1,1)\quad\text{or}\quad(4,2).
\]
We have grown to see that a point should be thought of as
\textit{location}, and nothing but location. With this definition in mind, it
doesn't really make sense to have operations between points like
addition or subtraction.

When trying to understand the world around us, we are often concerned
with quantities that denote both \textit{direction} and
\textit{magnitude}. We can do this by starting with two points




\begin{image}
  \begin{tikzpicture}
  \begin{axis}[
            xmin=-1,xmax=5,ymin=-1,ymax=3,
            clip=false,
            axis lines=center,
            ticks=none,
            unit vector ratio*=1 1 1,
            xlabel=$x$, ylabel=$y$,
            %ytick={-2,-1,...,7},
      %xtick={-2,-1,...,10},
      %grid = major,
            every axis y label/.style={at=(current axis.above origin),anchor=south},
            every axis x label/.style={at=(current axis.right of origin),anchor=west},
          ]
          \addplot[mark=*,penColor2] coordinates {(1,1)};
          \addplot[mark=*,penColor2] coordinates {(4,2)};
          \node[above] at (axis cs:4, 2) [penColor2] {$(a,b)$};
          \node[below] at (axis cs:1, 1) [penColor2] {$(c,d)$};
        \end{axis}
\end{tikzpicture}
\end{image}
and thinking of the differences of their coordinates.
\begin{image}
  \begin{tikzpicture}
  \begin{axis}[
            xmin=-1,xmax=5,ymin=-1,ymax=3,
            clip=false,
            axis lines=center,
            ticks=none,
            unit vector ratio*=1 1 1,
            xlabel=$x$, ylabel=$y$,
            %ytick={-2,-1,...,7},
      %xtick={-2,-1,...,10},
      %grid = major,
            every axis y label/.style={at=(current axis.above origin),anchor=south},
            every axis x label/.style={at=(current axis.right of origin),anchor=west},
          ]
          \addplot[mark=*,penColor2] coordinates {(1,1)};
          \addplot[mark=*,penColor2] coordinates {(4,2)};
          
          \addplot[very thick,penColor,->] plot coordinates {(1,1) (4,2)};
          \node[above] at (axis cs:4, 2) [penColor2] {$\mathrm{tip}=(a,b)$};
          \node[below] at (axis cs:1, 1) [penColor2] {$\mathrm{tail}=(c,d)$};
          \node[below right] at (axis cs:2.5, 1.5) [penColor] {$\overset{\rightharpoonup}{\mathbf{v}}=\left\langle a-c,b-d \right\rangle$};
        \end{axis}
\end{tikzpicture}
\end{image}





This object formed by the differences in the values of the coordinates of the points is 
called a \textit{vector}. In the graph above, the vector is
$\overset{\rightharpoonup}{\mathbf{v}}=\left\langle a-c,b-d \right\rangle$. We write vectors typographically in
boldface, decorated with a harpoon (like $\overset{\rightharpoonup}{\mathbf{v}}$ or
$\overset{\rightharpoonup}{\mathbf{w}}$). Other authors may simply use a boldface (like
$\mathbf{v}$ or $\mathbf{w}$) or just a harpoon (like ${\overset{\rightharpoonup}{v}}$
or ${\overset{\rightharpoonup}{w}}$). We often visualize a vector (at least in two and
three dimensions) as an arrow to explicitly show its direction and
magnitude. This visualization leads us to our definition of a vector.





\begin{definition}
  A \textbf{vector} is something that can be ascribed the qualities of
  direction and magnitude. 
\end{definition}

\begin{question}
  What vector has its tip at $(1,2)$ and its tail at $(4,3)$?
  \begin{prompt}
    \[
    \overset{\rightharpoonup}{\mathbf{v}} = \left\langle \answer{-3},\answer{-1} \right\rangle
    \]
  \end{prompt}
  
  \begin{feedback}[correct]
    Note that since we are denoting the vector by a single pair of
    numbers, this pair of numbers represents the tip of the vector,
    and we assume that the tail of the vector is at the origin.
     \begin{image}
       \begin{tikzpicture}
        \begin{axis}[
            xmin=-4,xmax=5,ymin=-2,ymax=4,
            clip=false,
            axis lines=center,
            %ticks=none,
            unit vector ratio*=1 1 1,
            xlabel=$x$, ylabel=$y$,
            ytick={-2,-1,...,4},
      xtick={-4,-3,...,5},
      grid = major,
            every axis y label/.style={at=(current axis.above origin),anchor=south},
            every axis x label/.style={at=(current axis.right of origin),anchor=west},
          ]
          \addplot[very thick,penColor,->] plot coordinates {(4,3) (1,2)};
          \addplot[very thick,penColor2,->] plot coordinates {(0,0) (-3,-1)};
          \node[above] at (axis cs:2.5, 2.5) [penColor] {$\overset{\rightharpoonup}{\mathbf{v}}$};
          \node[above] at (axis cs:-1.5, -.5) [penColor2] {$\overset{\rightharpoonup}{\mathbf{v}}$};

         \end{axis}
       \end{tikzpicture}
     \end{image}
  \end{feedback}
\end{question}
Two vectors are equal when they have the same direction and magnitude.
\begin{question}
  True or False: Given vectors $\overset{\rightharpoonup}{\mathbf{v}}$ and $\overset{\rightharpoonup}{\mathbf{w}}$ in the diagram
  below
  \begin{image}
  \begin{tikzpicture}
  \begin{axis}[
            xmin=-1,xmax=5,ymin=-1,ymax=4,
            clip=false,
            axis lines=center,
            %ticks=none,
            unit vector ratio*=1 1 1,
            xlabel=$x$, ylabel=$y$,
            %ytick={-2,-1,...,7},
      %xtick={-2,-1,...,10},
      grid = major,
            every axis y label/.style={at=(current axis.above origin),anchor=south},
            every axis x label/.style={at=(current axis.right of origin),anchor=west},
          ]
          \addplot[very thick,penColor,->] plot coordinates {(1,2) (4,3)};
          \addplot[very thick,penColor2,->] plot coordinates {(0,0) (3,1)};
          \node[above] at (axis cs:2.5, 2.5) [penColor] {$\overset{\rightharpoonup}{\mathbf{v}}$};
          \node[above] at (axis cs:1.5, .5) [penColor2] {$\overset{\rightharpoonup}{\mathbf{w}}$};

        \end{axis}
\end{tikzpicture}
\end{image}
  we have that $\overset{\rightharpoonup}{\mathbf{v}}=\overset{\rightharpoonup}{\mathbf{w}}$.
  \begin{prompt}
  \begin{multipleChoice}
    \choice[correct]{true}
    \choice{false}
  \end{multipleChoice}
  \end{prompt}
\end{question}

Vectors need not be limited to the $(x,y)$-plane. They can have any \textit{dimension}.


\begin{definition}
The \textbf{dimension} of a vector is the number of entries. Each
individual entry of a vector is called a \textbf{component}.
\end{definition}

In $\mathbb{R}^2$ we usually label the first component the ``$x$-component,''
and the second component the ``$y$-component.'' In $\mathbb{R}^3$ we usually
label the components ``$x$,'' ``$y$,'' and ``$z$.''


\begin{question}
  What is the dimension of the vector 
  \[
  \left\langle 3,4,1,-4 \right\rangle?
  \]
  \begin{prompt}
  \[
  \text{Dimension} = \answer{4}
  \]
  \end{prompt}
  \begin{question}
    What are the components of the vector $\left\langle 1,2,3 \right\rangle$?
    \begin{prompt}
      \begin{itemize}
      \item The $x$-component is $\answer{1}$.
      \item The $y$-component is $\answer{2}$.
      \item The $z$-component is $\answer{3}$.
      \end{itemize}
    \end{prompt}
  \end{question}
\end{question}


So far, we have mostly studied functions which take single numbers as
their inputs and output either individual numbers or ordered pairs (as
in the case of parametric functions).  Now, we set the stage for the
study of functions that accept lists of numbers as inputs and give lists of
numbers as outputs. When we want to keep track of more than one number
at a time, especially when we have more than one output depending on the 
same input, we often use a vector.





\subsection{Computing the direction and magnitude of vectors}


Since vectors are determined only by their direction and magnitude,
notation such as
\[
\left\langle a,b,c \right\rangle
\]
completely describes a vector, since we assume the tail is at the
origin. We should point out that the following are other types of notation for vectors.
\[
\begin{bmatrix}
  a\\
  b\\
  c
\end{bmatrix}, \quad
\begin{bmatrix}
  a & b & c
\end{bmatrix},
\quad
(a,b,c).
\]
When dealing with a vector in $1$, $2$, or $3$ dimensions, we can
visualize the vector as a directed arrow, where the \textit{magnitude} of the vector 
is the length of the arrow.

\begin{question}
  What is the magnitude of the vector $\left\langle 1,1 \right\rangle$?
  \begin{prompt}
    \[
    \text{Magnitude}  = \answer{\sqrt{2}}
    \]
  \end{prompt}
\end{question}

You were able to find the answer to the question above because you are
used to working with $2$ dimensional objects.  We make the following
definition in $n$ dimensions.

\begin{definition}
  Let $\overset{\rightharpoonup}{\mathbf{v}} = \left\langle v_1, v_2, v_3, \dots, v_n \right\rangle$ in $\mathbb{R}^n$
        be an $n$-dimensional vector.  Then the \textbf{magnitude} of
        $\overset{\rightharpoonup}{\mathbf{v}}$ is denoted by $|\overset{\rightharpoonup}{\mathbf{v}}|$ and is defined by:
  \[
  |\overset{\rightharpoonup}{\mathbf{v}}| = \sqrt{v_1^2+v_2^2+v_3^2+\dots+v_n^2}
  \]
\end{definition}

Notice that the magnitude of the vector is just the distance between 
the origin and the point determined by the components of our vector!

\begin{question}
  What is the magnitude of the vector
  \[
  \overset{\rightharpoonup}{\mathbf{v}}=\left\langle 2,-1,4,-2 \right\rangle?
  \]
  \begin{prompt}
    \[
    |\overset{\rightharpoonup}{\mathbf{v}}|=\answer{5}
    \]
  \end{prompt}
\end{question}








\section{Operations on vectors}


We can add vectors of the same dimension together by component-wise
addition. Here, it is useful to write vectors vertically.
\[
\begin{bmatrix}
  a\\
  b\\
  c
\end{bmatrix}
+
\begin{bmatrix}
  d\\
  e\\
  f
\end{bmatrix}
=
\begin{bmatrix}
  a+d\\
  b+e\\
  c+f
\end{bmatrix}.
\]

\begin{question}
  \[
  \left\langle 1,2,3 \right\langle + \left\langle -1,2,2 \right\rangle =
  \left\langle \answer{0},\answer{4},\answer{5} \right\rangle
  \]
\end{question}

Now, let us investigate the geometry of addition of vectors. Let
$\overset{\rightharpoonup}{\mathbf{v}} = \left\langle 1,2 \right\rangle$ and $\overset{\rightharpoonup}{\mathbf{w}} = \left\langle 3,1 \right\rangle$.  If we
place the tail of the vector $\overset{\rightharpoonup}{\mathbf{w}}$ at the tip of the vector
$\overset{\rightharpoonup}{\mathbf{v}}$, like this:
\begin{image}
  \begin{tikzpicture}
    \begin{axis}[
        xmin=-1,xmax=5,ymin=-1,ymax=4,
            axis lines=center,
            %ticks=none,
            unit vector ratio*=1 1 1,
            xlabel=$x$, ylabel=$y$,
            ytick={-2,-1,...,7},
      %yticklabels={$0.5$,$1$,$1.5$,$2$},
      xtick={-2,-1,...,10},
      %xticklabels={$0.5$,$1$,$1.5$,$2$},
      grid = major,
            every axis y label/.style={at=(current axis.above origin),anchor=south},
            every axis x label/.style={at=(current axis.right of origin),anchor=west},
          ]
          \addplot[very thick,penColor,->] plot coordinates {(1,2) (4,3)};
          \addplot[very thick,penColor2,->] plot coordinates {(0,0) (1,2)};
          \addplot[ultra thick,penColor3,->] plot coordinates {(0,0) (4,3)};

           \node[left] at (axis cs:.5, 1) [penColor2] {$\overset{\rightharpoonup}{\mathbf{v}}$};
           \node[above] at (axis cs:2.5, 2.5 ) [penColor] {$\overset{\rightharpoonup}{\mathbf{w}}$};
           \node[below right] at (axis cs:2, 1.5 ) [penColor3] {$\overset{\rightharpoonup}{\mathbf{v}}+\overset{\rightharpoonup}{\mathbf{w}}$};
    \end{axis}
\end{tikzpicture}
\end{image}
or like this:
\begin{image}
  \begin{tikzpicture}
    \begin{axis}[
        xmin=-1,xmax=5,ymin=-1,ymax=4,
            axis lines=center,
            %ticks=none,
            unit vector ratio*=1 1 1,
            xlabel=$x$, ylabel=$y$,
            ytick={-2,-1,...,7},
      %yticklabels={$0.5$,$1$,$1.5$,$2$},
      xtick={-2,-1,...,10},
      %xticklabels={$0.5$,$1$,$1.5$,$2$},
      grid = major,
            every axis y label/.style={at=(current axis.above origin),anchor=south},
            every axis x label/.style={at=(current axis.right of origin),anchor=west},
          ]
          \addplot[very thick,penColor2,->] plot coordinates {(3,1) (4,3)};
          \addplot[very thick,penColor3,->] plot coordinates {(0,0) (4,3)};
          \addplot[ultra thick,penColor,->] plot coordinates {(0,0) (3,1)};

           \node[right] at (axis cs:3.5, 2) [penColor2] {$\overset{\rightharpoonup}{\mathbf{v}}$};
           \node[below] at (axis cs:1.5, .5 ) [penColor] {$\overset{\rightharpoonup}{\mathbf{w}}$};
           \node[above left] at (axis cs:2, 1.5 ) [penColor3] {$\overset{\rightharpoonup}{\mathbf{v}}+\overset{\rightharpoonup}{\mathbf{w}}$};
    \end{axis}
\end{tikzpicture}
\end{image}
then the sum $\overset{\rightharpoonup}{\mathbf{v}}+\overset{\rightharpoonup}{\mathbf{w}}$ connects the tail of $\overset{\rightharpoonup}{\mathbf{v}}$ to the
tip of $\overset{\rightharpoonup}{\mathbf{w}}$. In fact, you can think of the sum of two vectors as
being the diagonal of the parallelogram formed by the two vectors.
\begin{image}
  \begin{tikzpicture}
    \begin{axis}[
        xmin=-1,xmax=5,ymin=-1,ymax=4,
            axis lines=center,
            %ticks=none,
            unit vector ratio*=1 1 1,
            xlabel=$x$, ylabel=$y$,
            ytick={-2,-1,...,7},
      %yticklabels={$0.5$,$1$,$1.5$,$2$},
      xtick={-2,-1,...,10},
      %xticklabels={$0.5$,$1$,$1.5$,$2$},
      grid = major,
            every axis y label/.style={at=(current axis.above origin),anchor=south},
            every axis x label/.style={at=(current axis.right of origin),anchor=west},
          ]
          \addplot[very thick,penColor,->] plot coordinates {(1,2) (4,3)};
          \addplot[very thick,penColor2,->] plot coordinates {(0,0) (1,2)};
          \addplot[ultra thick,penColor3,->] plot coordinates {(0,0) (4,3)};

          \node[left] at (axis cs:.5, 1) [penColor2] {$\overset{\rightharpoonup}{\mathbf{v}}$};
          \node[above] at (axis cs:2.5, 2.5 ) [penColor] {$\overset{\rightharpoonup}{\mathbf{w}}$};
          \node[below right] at (axis cs:1.9, 1.6 ) [penColor3] {$\overset{\rightharpoonup}{\mathbf{v}}+\overset{\rightharpoonup}{\mathbf{w}}$};
          
          \addplot[very thick,penColor2,->] plot coordinates {(3,1) (4,3)};
          \addplot[very thick,penColor3,->] plot coordinates {(0,0) (4,3)};
          \addplot[ultra thick,penColor,->] plot coordinates {(0,0) (3,1)};
          
          \node[right] at (axis cs:3.5, 2) [penColor2] {$\overset{\rightharpoonup}{\mathbf{v}}$};
          \node[below] at (axis cs:1.5, .5 ) [penColor] {$\overset{\rightharpoonup}{\mathbf{w}}$};
          %\node[above left] at (axis cs:2, 1.5 ) [penColor3] {$\overset{\rightharpoonup}{\mathbf{v}}+\overset{\rightharpoonup}{\mathbf{w}}$};
    \end{axis}
\end{tikzpicture}
\end{image}



Hence,
\[
\overset{\rightharpoonup}{\mathbf{v}}+\overset{\rightharpoonup}{\mathbf{w}} = \left\langle 4 , 3 \right\rangle.
\]
\begin{question}
  Consider the following diagram.
  \begin{image}
  \begin{tikzpicture}
    \begin{axis}[
        xmin=0,xmax=5,ymin=-1,ymax=4,
        axis lines=center,
            %ticks=none,
            unit vector ratio*=1 1 1,
            xlabel=$x$, ylabel=$y$,
            ytick={-2,-1,...,7},
      %yticklabels={$0.5$,$1$,$1.5$,$2$},
      xtick={-2,-1,...,10},
      %xticklabels={$0.5$,$1$,$1.5$,$2$},
      grid = major,
            every axis y label/.style={at=(current axis.above origin),anchor=south},
            every axis x label/.style={at=(current axis.right of origin),anchor=west},
          ]
          \addplot[very thick,penColor,->] plot coordinates {(2,3) (1,0)};
          \addplot[very thick,penColor2,->] plot coordinates {(4,1) (1,0)};
          \addplot[very thick,penColor3,->] plot coordinates {(4,1) (2,3)};
          
          \node[left] at (axis cs:1.5, 1.5 ) [penColor] {$\overset{\rightharpoonup}{\mathbf{a}}$};
          \node[below] at (axis cs:2.5, .5) [penColor2] {$\overset{\rightharpoonup}{\mathbf{b}}$};
          \node[above right] at (axis cs:3, 2 ) [penColor3] {$\overset{\rightharpoonup}{\mathbf{c}}$};
    \end{axis}
  \end{tikzpicture}
  \end{image}
  Which equation is represented by the diagram above?
  \begin{multipleChoice}
    \choice{$\overset{\rightharpoonup}{\mathbf{a}} + \overset{\rightharpoonup}{\mathbf{b}} = \overset{\rightharpoonup}{\mathbf{c}}$}
    \choice[correct]{$\overset{\rightharpoonup}{\mathbf{a}} + \overset{\rightharpoonup}{\mathbf{c}} = \overset{\rightharpoonup}{\mathbf{b}}$}
    \choice{$\overset{\rightharpoonup}{\mathbf{b}} + \overset{\rightharpoonup}{\mathbf{c}} = \overset{\rightharpoonup}{\mathbf{a}}$}
  \end{multipleChoice}
  \begin{feedback}[correct]
    Notice that we also could have drawn the diagram above like this.
      \begin{image}
        \begin{tikzpicture}
          \begin{axis}[
              xmin=0,xmax=5,ymin=-1,ymax=4,
              axis lines=center,
              %ticks=none,
              unit vector ratio*=1 1 1,
              xlabel=$x$, ylabel=$y$,
              ytick={-2,-1,...,7},
        %yticklabels={$0.5$,$1$,$1.5$,$2$},
        xtick={-2,-1,...,10},
        %xticklabels={$0.5$,$1$,$1.5$,$2$},
        grid = major,
              every axis y label/.style={at=(current axis.above origin),anchor=south},
              every axis x label/.style={at=(current axis.right of origin),anchor=west},
            ]
            \addplot[very thick,penColor,->] plot coordinates {(4,3) (3,0)};
            \addplot[very thick,penColor2,->] plot coordinates {(4,3) (1,2)};
            \addplot[very thick,penColor3,->] plot coordinates {(3,0) (1,2)};
          
            \node[right] at (axis cs:3.5, 1.5 ) [penColor] {$\overset{\rightharpoonup}{\mathbf{a}}$};
            \node[above] at (axis cs:2.5, 2.5) [penColor2] {$\overset{\rightharpoonup}{\mathbf{b}}$};
            \node[below left] at (axis cs:2, 1 ) [penColor3] {$\overset{\rightharpoonup}{\mathbf{c}}$};
          \end{axis}
        \end{tikzpicture}
      \end{image}
  \end{feedback}
\end{question}








We can also multiply vectors by a \textbf{scalar} (a number), by
multiplying each component by the scalar:

\begin{question}
  \[
  4\cdot \left\langle 2 , 4 , 0 , 1 \right\rangle = \left\langle \answer{8}, \answer{16} , \answer{0} , \answer{4} \right\rangle
  \]  
\end{question}

\begin{question}
  True or False: Multiplying a vector by a nonzero scalar will not
  change the direction of the vector.
  \begin{prompt}
  \begin{multipleChoice}
    \choice{true}
    \choice[correct]{false}
  \end{multipleChoice}
  \end{prompt}
  \begin{feedback}
    Multiplying a vector by a positive scalar $s$ will not change the
    direction of the vector.
    \begin{image}
      \begin{tikzpicture}
   \begin{axis}[
            xmin=-1,xmax=7,ymin=-1,ymax=3,
            clip=false,
            axis lines=center,
            %ticks=none,
            unit vector ratio*=1 1 1,
            xlabel=$x$, ylabel=$y$,
            ytick={-1,0,...,3},
      xtick={-1,0,...,7},
      grid = major,
            every axis y label/.style={at=(current axis.above origin),anchor=south},
            every axis x label/.style={at=(current axis.right of origin),anchor=west},
          ]
          \addplot[ultra thick,penColor,->] plot coordinates {(0,0) (3,1)};
          \addplot[very thick,penColor2,->] plot coordinates {(0,0) (6,2)};
          \node[above] at (axis cs:1.5, .5) [penColor] {$\overset{\rightharpoonup}{\mathbf{v}}$};
          \node[below] at (axis cs:4, 1.25) [penColor2] {$s\cdot\overset{\rightharpoonup}{\mathbf{v}}$};

         \end{axis}
       \end{tikzpicture}
    \end{image}
    However, if we multiply a vector by a \textit{negative} scalar $-s$, 
    then the direction will change.
    \begin{image}
      \begin{tikzpicture}
   \begin{axis}[
            xmin=-7,xmax=4,ymin=-3,ymax=2,
            clip=false,
            axis lines=center,
            %ticks=none,
            unit vector ratio*=1 1 1,
            xlabel=$x$, ylabel=$y$,
            ytick={-3,-2,...,2},
      xtick={-7,-6,...,4},
      grid = major,
            every axis y label/.style={at=(current axis.above origin),anchor=south},
            every axis x label/.style={at=(current axis.right of origin),anchor=west},
          ]
          \addplot[very thick,penColor,->] plot coordinates {(0,0) (3,1)};
          \addplot[very thick,penColor2,->] plot coordinates {(0,0) (-6,-2)};
          \node[above] at (axis cs:1.5, .5) [penColor] {$\overset{\rightharpoonup}{\mathbf{v}}$};
          \node[below] at (axis cs:-3, -1) [penColor2] {$-s\cdot\overset{\rightharpoonup}{\mathbf{v}}$};

         \end{axis}
       \end{tikzpicture}
    \end{image}
  \end{feedback}
\end{question}

Thinking about how the magnitude of a vector changes when we multiply by a scalar 
reveals why scalars are called \textit{scalars}.

\begin{onlineOnly}
  You can use this interactive to see how scalars affect vectors.
  \begin{center}
    \geogebra{WYNdzcGP}{800}{600}%https://www.geogebra.org/m/WYNdzcGP
  \end{center}
\end{onlineOnly}

\begin{question}
  Consider a vector
  \[
  \overset{\rightharpoonup}{\mathbf{v}} = \left\langle a,b,c \right\rangle.
  \]
  What is the magnitude of $\overset{\rightharpoonup}{\mathbf{v}}$?
  \begin{prompt}
    \[
    |\overset{\rightharpoonup}{\mathbf{v}}| = \answer{\sqrt{a^2+b^2+c^2}}
    \]
  \end{prompt}
  \begin{question}
    What is the magnitude of $6\cdot\overset{\rightharpoonup}{\mathbf{v}}$?
    \begin{prompt}
      \[
      |6\cdot\overset{\rightharpoonup}{\mathbf{v}}| = \answer{6}\cdot\sqrt{a^2+b^2+c^2}
      \]
    \end{prompt}
    \begin{question}
    What is the magnitude of $-6\cdot\overset{\rightharpoonup}{\mathbf{v}}$?
    \begin{prompt}
      \[
      |-6\cdot\overset{\rightharpoonup}{\mathbf{v}}| = \answer{6}\cdot\sqrt{a^2+b^2+c^2}
      \]
    \end{prompt}
    \begin{feedback}
      If $s$ is a positive constant, and $\overset{\rightharpoonup}{\mathbf{v}}$ is a vector, then
      vector $s\cdot\overset{\rightharpoonup}{\mathbf{v}}$ points in the same direction as
      $\overset{\rightharpoonup}{\mathbf{v}}$, but its length is scaled by a factor of $s$.  If $s$
      is negative, then $s\cdot\overset{\rightharpoonup}{\mathbf{v}}$ points in the opposite
      direction of $\overset{\rightharpoonup}{\mathbf{v}}$, and its length is scaled by a factor of
      $|s|$.
    \end{feedback}
  \end{question}
  \end{question}
\end{question}






\section{Unit vectors}

Vectors with magnitude $1$ are particularly important.

\begin{definition}
  A \textbf{unit vector} is a vector of magnitude $1$. In this text, unit
  vectors will wear hats: $|\mathbf{\hat{u}}|=1$.
\end{definition}

\begin{theorem}
  If $\overset{\rightharpoonup}{\mathbf{v}}$ is a nonzero vector, then the unit vector which points
  in the same direction as $\overset{\rightharpoonup}{\mathbf{v}}$ is $\frac{\overset{\rightharpoonup}{\mathbf{v}}}{|\overset{\rightharpoonup}{\mathbf{v}}|}$.
\end{theorem}


\begin{question}
  Find a unit vector $\mathbf{\hat{u}}$ which points in the same direction as the vector $\overset{\rightharpoonup}{\mathbf{v}} = \left\langle 2,1,3,7,1 \right\rangle$.
  \begin{prompt}
  \[
  \mathbf{\hat{u}} = \left\langle 
    \answer{2/8},
    \answer{1/8},
    \answer{3/8},
    \answer{7/8},
    \answer{1/8} \right\rangle
  \]
  \end{prompt}
  \begin{hint}
    Scaling the vector $\overset{\rightharpoonup}{\mathbf{v}}$ by the reciprocal of its magnitude should result in a magnitude $1$ vector which points in the same direction.
  \end{hint}
  \begin{hint}
    $|\overset{\rightharpoonup}{\mathbf{v}}| = \sqrt{2^2+1^2+3^2+7^2+1^2} = \sqrt{64} = 8$
  \end{hint}
\end{question}

Now consider any vector $\overset{\rightharpoonup}{\mathbf{v}}$. We can  extract its direction
and magnitude in the following way.
\[
\overset{\rightharpoonup}{\mathbf{v}} = \underbrace{|\overset{\rightharpoonup}{\mathbf{v}}|}_{\text{magnitude}} \cdot \underbrace{\frac{\overset{\rightharpoonup}{\mathbf{v}}}{|\overset{\rightharpoonup}{\mathbf{v}}|}}_{\text{direction}}
\]
This equation illustrates the fact that a vector has both magnitude
and direction, where we view a unit vector as supplying \textit{only}
direction information. Identifying unit vectors with direction allows
us to define \textit{parallel vectors}.
\begin{definition}
Unit vectors $\mathbf{\hat{u}}$ and $\mathbf{\hat{v}}$ are \textbf{parallel} if
\[
\mathbf{\hat{u}} = \pm \mathbf{\hat{v}}.
\]
Nonzero vectors $\overset{\rightharpoonup}{\mathbf{a}}$ and $\overset{\rightharpoonup}{\mathbf{b}}$ are \textbf{parallel} if their
respective unit vectors are parallel.
\end{definition}
It is equivalent to say that vectors $\overset{\rightharpoonup}{\mathbf{a}}$ and $\overset{\rightharpoonup}{\mathbf{b}}$ are
parallel if there is a scalar $s\neq 0$ such that $\overset{\rightharpoonup}{\mathbf{a}} =
s\cdot\overset{\rightharpoonup}{\mathbf{b}}$.

\begin{question}
  Let $\overset{\rightharpoonup}{\mathbf{v}} = \left\langle 1,-4,2 \right\rangle$. Find all unit vectors parallel to $\overset{\rightharpoonup}{\mathbf{v}}$.
  \begin{prompt}
    Write your answers in the order of increasing $x$-coordinates:
    \[
    \mathbf{\hat{u}}_1 = \left\langle \answer{-1/\sqrt{21}},\answer{4/\sqrt{21}},\answer{-2/\sqrt{21}} \right\rangle \quad \mathbf{\hat{u}}_2 = \left\langle \answer{1/\sqrt{21}},\answer{-4/\sqrt{21}},\answer{2/\sqrt{21}} \right\rangle
    \]
  \end{prompt}
\end{question}


Note that the zero vector $\overset{\rightharpoonup}{\mathbf{0}}$ is directionless, because
there is no unit vector in the ``direction'' of $\overset{\rightharpoonup}{\mathbf{0}}$. Different
authors have different conventions regarding the zero vector. Some
even say the zero vector is ``parallel to every vector.'' We prefer to simply say
that the zero vector has no direction, as this statement is grounded in the fact
that \textbf{unit vectors provide direction information}.  So, in our case, 
the zero vector is not parallel to any vector.  Check for yourself 
using our definition of parallel vectors!

\begin{question}
  True or False: If two vectors are parallel, then they point in the same direction.
  \begin{prompt}
    \begin{multipleChoice}
      \choice{true}
      \choice[correct]{false}
    \end{multipleChoice}
  \end{prompt}
\end{question}






















\end{document}
