\documentclass{ximera}


\graphicspath{
  {./}
  {ximeraTutorial/}
  {basicPhilosophy/}
}

\newcommand{\mooculus}{\textsf{\textbf{MOOC}\textnormal{\textsf{ULUS}}}}


\usepackage{tkz-euclide}\usepackage{tikz}
\usepackage{tikz-cd}
\usetikzlibrary{arrows}
\tikzset{>=stealth,commutative diagrams/.cd,
  arrow style=tikz,diagrams={>=stealth}} %% cool arrow head
\tikzset{shorten <>/.style={ shorten >=#1, shorten <=#1 } } %% allows shorter vectors

\usetikzlibrary{backgrounds} %% for boxes around graphs
\usetikzlibrary{shapes,positioning}  %% Clouds and stars
\usetikzlibrary{matrix} %% for matrix
\usepgfplotslibrary{polar} %% for polar plots
\usepgfplotslibrary{fillbetween} %% to shade area between curves in TikZ
\usetkzobj{all}
\usepackage[makeroom]{cancel} %% for strike outs
%\usepackage{mathtools} %% for pretty underbrace % Breaks Ximera
%\usepackage{multicol}
\usepackage{pgffor} %% required for integral for loops



%% http://tex.stackexchange.com/questions/66490/drawing-a-tikz-arc-specifying-the-center
%% Draws beach ball
\tikzset{pics/carc/.style args={#1:#2:#3}{code={\draw[pic actions] (#1:#3) arc(#1:#2:#3);}}}



\usepackage{array}
\setlength{\extrarowheight}{+.1cm}
\newdimen\digitwidth
\settowidth\digitwidth{9}
\def\divrule#1#2{
\noalign{\moveright#1\digitwidth
\vbox{\hrule width#2\digitwidth}}}
























%%This is to help with formatting on future title pages.
\newenvironment{sectionOutcomes}{}{}


\title{Exponential Rules}

\begin{document}

\begin{abstract}
rules
\end{abstract}
\maketitle





The algebra of exponents are grouped into those with the same base...




$\bigstar$  $A^n \cdot A^m = A^{n+m}$


$\bigstar$  $\frac{A^n}{A^m} = A^{n-m}$


$\bigstar$  $(A^n)^m = A^{n \cdot m}$ \\


...and those with the same exponent. 


$\blacksquare$  $A^n \cdot B^n = (A \cdot B)^n$


$\blacksquare$  $\frac{A^n}{B^n} = \left(\frac{A}{B}\right)^n$ \\





These lead to equalities such as




$\blacklozenge$ $A^0 = 1$


$\blacklozenge$ $A^1 = A$


$\blacklozenge$ $A^{-n} = \frac{1}{A^n}$





We have also seen that exponential functions are \textbf{one-to-one}. One-to-one means that each range number is paired with a unique domain number.





\begin{image}
\begin{tikzpicture}
  \begin{axis}[
            domain=-10:10, ymax=10, xmax=10, ymin=-10, xmin=-10,
            axis lines =center, xlabel=$x$, ylabel=$y$, grid = major,
            ytick={-10,-8,-6,-4,-2,2,4,6,8,10},
          	xtick={-10,-8,-6,-4,-2,2,4,6,8,10},
          	ticklabel style={font=\scriptsize},
            every axis y label/.style={at=(current axis.above origin),anchor=south},
            every axis x label/.style={at=(current axis.right of origin),anchor=west},
            axis on top
          ]
          
      		\addplot [line width=2, penColor, smooth,samples=200,domain=(-10:0.6),<->] {0.33 * 2^(x+5)-7};

          	\addplot [line width=1, gray, dashed,samples=200,domain=(-10:10),<->] {-7};


      		\addplot[color=penColor,fill=penColor,only marks,mark=*] coordinates{(-5,-6.66)};

   

  \end{axis}
\end{tikzpicture}
\end{image}



Our definition of functions includes a similar rule for domain numbers.  Each domain number is in exactly one pair.  One-to-one puts the same restrictions on range numbers.  Each range number is in exactly one pair.

In other words, each function value occurs exactly once.


If you know that $A^r = A^t$, then $r=t$. \\


This is an important rule.


\begin{fact} Uniqueness


\[     \text{If } \,  A^r = A^t, \,  \text{ then }  \, r=t    \]


\end{fact}




\begin{example} Solving Equations


Solve $5^m = 25$


\textbf{\textcolor{purple!50!blue!90!black}{SOLUTION}}


$5^m = 25$

$5^m = 5^2$

$m = 2$

\end{example}










\begin{example} Solving Equations


Solve $2^{2k+1} = 32$


\textbf{\textcolor{purple!50!blue!90!black}{SOLUTION}}


$2^{2k+1} = 32$

$2^{2k+1} = 2^5$

$2k + 1 = 5$

$k = 2$

\end{example}






\begin{example} Solving Equations


Solve $9^{-x + 2} = 27^{x-1}$


\textbf{\textcolor{purple!50!blue!90!black}{SOLUTION}}


$9^{-x + 2} = 27^{x-1}$

$(3^2)^{-x + 2} = (3^3)^{x-1}$

$3^{-2x+4} = 3^{3x-3}$

$-2x + 4 = 3x - 3$

$7 = 5x$


$\frac{7}{5} = x$

\end{example}














\section{e}


You know the number $\pi$ (pronounced "pie").  The definition of $\pi$ is kind of weird.  For any circle, $\pi$ is the ratio of the circumference to the diameter.

$\pi$ is approximately $3.1415926$.





There is another number that comes up quite frequently when dealing with naturla growth.  This number called "e" and its symbol is $e$.

$e$ is approximately $2.718281828$.

The definition of $e$ is even weirder. \\


Consider the function $e(x) = \left(1 + \frac{1}{x}\right)^x$



\begin{center}
\desmos{cmurxtv0dc}{400}{300}
\end{center}


This function is an increasing function, but it increases at a smaller and smaller rate.  It starts to level off and approaches a horizontal asymptote.  This asymptote is at a height of $e$.  And that is the definition of $e$


\[   e = \lim_{x \to \infty}  \left(1 + \frac{1}{x}\right)^x      \]
























\end{document}
