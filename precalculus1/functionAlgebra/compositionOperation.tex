\documentclass{ximera}


\graphicspath{
  {./}
  {ximeraTutorial/}
  {basicPhilosophy/}
}

\newcommand{\mooculus}{\textsf{\textbf{MOOC}\textnormal{\textsf{ULUS}}}}


\usepackage{tkz-euclide}\usepackage{tikz}
\usepackage{tikz-cd}
\usetikzlibrary{arrows}
\tikzset{>=stealth,commutative diagrams/.cd,
  arrow style=tikz,diagrams={>=stealth}} %% cool arrow head
\tikzset{shorten <>/.style={ shorten >=#1, shorten <=#1 } } %% allows shorter vectors

\usetikzlibrary{backgrounds} %% for boxes around graphs
\usetikzlibrary{shapes,positioning}  %% Clouds and stars
\usetikzlibrary{matrix} %% for matrix
\usepgfplotslibrary{polar} %% for polar plots
\usepgfplotslibrary{fillbetween} %% to shade area between curves in TikZ
\usetkzobj{all}
\usepackage[makeroom]{cancel} %% for strike outs
%\usepackage{mathtools} %% for pretty underbrace % Breaks Ximera
%\usepackage{multicol}
\usepackage{pgffor} %% required for integral for loops



%% http://tex.stackexchange.com/questions/66490/drawing-a-tikz-arc-specifying-the-center
%% Draws beach ball
\tikzset{pics/carc/.style args={#1:#2:#3}{code={\draw[pic actions] (#1:#3) arc(#1:#2:#3);}}}



\usepackage{array}
\setlength{\extrarowheight}{+.1cm}
\newdimen\digitwidth
\settowidth\digitwidth{9}
\def\divrule#1#2{
\noalign{\moveright#1\digitwidth
\vbox{\hrule width#2\digitwidth}}}
























%%This is to help with formatting on future title pages.
\newenvironment{sectionOutcomes}{}{}


\title{A New Operation}

\begin{document}

\begin{abstract}
composition
\end{abstract}
\maketitle





{\Huge !} With the caveat that domains need to be aligned, we have many operations on functions that make up an arithmetic on functions. \\




$\blacktriangleright$ \textbf{\textcolor{blue!75!black}{Addition}}  

The sum of two functions is again a function. 

The additive identity is the zero function, $Zero(x) = 0$ for all real numbers.

For each function, $f$, there is another function $-f$, such that $f + (-f) = 0$.   $f$ and $-f$ are inverses of each other with respect to addition.




$\blacktriangleright$ \textbf{\textcolor{blue!75!black}{Multiplication}} 

The product of two functions is again a function.  

The multiplicative identity is $One(x) = 1$ for all real numbers.

For each function, $f$, there is another function $\frac{1}{f}$, such that $f \cdot \left( \frac{1}{f} \right) = 1$, again domain restrictions might be needed for problems.   $f$ and $\frac{1}{f}$ are inverses of each other with respect to multiplication.  (If we were talking about numbers, then we might also use exponential notation $\frac{1}{n} = n^{-1}$.


$\blacktriangleright$ \textbf{\textcolor{blue!75!black}{Composition}} 

Composition is a new operation on functions.

The composition of two functions is again a function. 

The identity function, $Id(x) = x$ for all $x$, is the identity element with respect to composition.



\[   f \circ Id = f    \, \text{ and } \, Id \circ f = f        \]



The inverse of $f$ with respect to composition is another function whose composition with $f$ produces the identity function.

Stealing the exponential notation from numbers, the symbol for the inverse of $f$ is $f^{-1}$.

\[   f \circ f^{-1} = Id    \, \text{ and } \, f^{-1} \circ f = Id       \]










\begin{example} Composition


\begin{itemize}
\item Let $K(u) = \frac{u}{u-1}$ with its natural or implied domain: $(-\infty, 1) \cup (1, \infty)$. \\

\item Let $T(w) = w^2 + 5w + 7$ with its natural or implied domain: \textbf{$\mathbb{R}$}.
\end{itemize}


Form the composition $K \circ T$.

\[        (K \circ T)(w) =     \frac{w^2 + 5w + 7}{(w^2 + 5w + 7)-1}   =    \frac{w^2 + 5w + 7}{w^2 + 5w + 6}  \]


$K$ can accept any number, except $1$.  Therefore we need to find out when $T = 1$ and take out the domain numbers where $1$ occurs.



\begin{align*}
T(w) & = 1   \\
w^2 + 5w + 7 & = 1 \\
w^2 + 5w + 6 & = 0   \\
(w+2)\left( \answer{w+3} \right) & = 0
\end{align*}


We need to remove $-2$ and $\answer{-3}$ from the domain of $K \circ T$, because $T(-2)=1$ and $T(-3)=1$ and $1$ cannot be an input into $K$.


The domain for $K \circ T$ is $(-\infty, -3) \cup (-3, -2) \cup (-2, \infty)$




\[        (K \circ T)(w)  =    \frac{w^2 + 5w + 7}{w^2 + 5w + 6}  =    \frac{w^2 + 5w + 7}{(w+2)(w+3)} \]










The composition is a function. \\


This is a good reminder that a formula is not a function. There are always domain considerations.  In the case of a composition, we need to restrict the domain of the inner function to avoid inner function values that are not in the domain of the outer function.










\end{example}










\begin{example} Composition



\begin{itemize}
\item Let $g(x) = \frac{x-3}{x+1}$ with its natural or implied domain: $(-\infty, -1) \cup (-1. \infty)$. \\

\item Let $H(t) = \frac{t+3}{1-t}$ with its natural or implied domain: $(-\infty, 1) \cup (1. \infty)$.
\end{itemize}



\[
(g \circ H)(y) = \frac{\left( \frac{y+3}{1-y} \right) - 3}{\left(  \frac{y+3}{1-y}\right) + 1} = \frac{y+3-3(1-y)}{y+3+(1-y)} = \frac{4y}{y} = y
\]



$g$ and $H$ are inverse functions.




\end{example}

Of course, that isn't the whole story.  As we said earlier, formulas are not functions. \\

$(g \circ H)(y) = y$ and so $g \circ H = Id$, the identity function.  But remember our caveat.  There are always domain issues.  In this case $(g \circ H)(y) = y$ for almost all of the real numbers.

Actually, 


\[
(g \circ H)(y) = \frac{\left( \frac{y+3}{1-y} \right) - 3}{\left(  \frac{y+3}{1-y}\right) + 1} 
\]


This is equivalent to $y$ for almost all real numbers.


In this case, $(g \circ H)(y) = g(H(y))$, which means that $y \ne 1$, since $1$ is not in the domain of $H$.  Secondly, $-1$ is not in the domain of $g$.  We also cannot have values of $y$ that make $H(y) = -1$.




\[
H(r) = \frac{r+3}{1-r} = -1
\]

\[
r + 3 = -(1-r) = -1 + r
\]


\[
3 = -1
\]


There are no such real numbers.



Therefore, $g$ and $H$ are inverse functions on $(-\infty, 1) \cup (1,\infty)$.




















\section{The Other Half of the Story}


We would like our function arithmetic to mimic our arithmetic for numbers.  For numbers, the inverses are commutative.

\[
4 + (-4) = (-4) + 4 = 0
\]


\[
4 \cdot \frac{1}{4} = \frac{1}{4} \cdot 4 = 1
\]



In our example, we also want $(H \circ g)(y) = H(g(y)) = y$. It does.


\[
(H \circ g)(y) = \frac{\left( \frac{y-3}{y+1} \right) + 3}{1 - \left(  \frac{y-3}{y+1}\right)} = \frac{y-3 + 3(y+1)}{y+1-(y-3)} = \frac{4y}{4} = y
\]


In this case, we cannot have $y = -1$






$\blacktriangleright$  Therefore, $g$ and $H$ are inverse functions on $(-\infty, -1) \cup (-1, 1) \cup (1, \infty)$.


























\end{document}
