\documentclass{ximera}


\graphicspath{
  {./}
  {ximeraTutorial/}
  {basicPhilosophy/}
}

\newcommand{\mooculus}{\textsf{\textbf{MOOC}\textnormal{\textsf{ULUS}}}}


\usepackage{tkz-euclide}\usepackage{tikz}
\usepackage{tikz-cd}
\usetikzlibrary{arrows}
\tikzset{>=stealth,commutative diagrams/.cd,
  arrow style=tikz,diagrams={>=stealth}} %% cool arrow head
\tikzset{shorten <>/.style={ shorten >=#1, shorten <=#1 } } %% allows shorter vectors

\usetikzlibrary{backgrounds} %% for boxes around graphs
\usetikzlibrary{shapes,positioning}  %% Clouds and stars
\usetikzlibrary{matrix} %% for matrix
\usepgfplotslibrary{polar} %% for polar plots
\usepgfplotslibrary{fillbetween} %% to shade area between curves in TikZ
\usetkzobj{all}
\usepackage[makeroom]{cancel} %% for strike outs
%\usepackage{mathtools} %% for pretty underbrace % Breaks Ximera
%\usepackage{multicol}
\usepackage{pgffor} %% required for integral for loops



%% http://tex.stackexchange.com/questions/66490/drawing-a-tikz-arc-specifying-the-center
%% Draws beach ball
\tikzset{pics/carc/.style args={#1:#2:#3}{code={\draw[pic actions] (#1:#3) arc(#1:#2:#3);}}}



\usepackage{array}
\setlength{\extrarowheight}{+.1cm}
\newdimen\digitwidth
\settowidth\digitwidth{9}
\def\divrule#1#2{
\noalign{\moveright#1\digitwidth
\vbox{\hrule width#2\digitwidth}}}
























%%This is to help with formatting on future title pages.
\newenvironment{sectionOutcomes}{}{}


\title{Factoring}

\begin{document}

\begin{abstract}
Rational Roots Theorem
\end{abstract}
\maketitle







We have investigated polynomial functions from several viewpoints.  Time to collect all of our thoughts and charcterize polynomial functions.











\begin{example} Polynomial


Completely analyze $p(w) = -\frac{1}{5}(w+4)(w-3)(w-3)$

First let's collect like factors: $p(w) = -\frac{1}{5}(w+4)(w-3)^2$

We have a polynomial of degree $3$.  It has two roots.

\begin{itemize}
\item $-4$ is a root of multiplicity $\answer{1}$.  Since this multipicity is odd, $p$ will \wordChoice{\choice[correct]{change sign} \choice{not change sign}} through $-4$ and the graph will cross at $(-4,0)$.
\item $3$ is a root of multiplicity $\answer{2}$.  Since this multipicity is even, $p$ will \wordChoice{\choice{change sign} \choice[correct]{not change sign}} sign through $3$ and the graph will not cross at $(2,0)$.  The graph will touch and then bounce back.
\end{itemize}


The end-behavior of $p$ is $-\frac{1}{5} w^3$.  Therefore, $\lim\limits_{w \to -\infty}p(w) = \infty$ and $\lim\limits_{w \to \infty}p(w) = -\infty$.





The graph is very suggestive that there is a local minimum somewhere around $-1$.  $3$ is a zero and also a critical number, since the function is negative around that zero.






With some technology, we can approximate the other critical number to be $-1.67$ and the local minimum to be $-10.163$.


\begin{itemize}
\item $p$ \wordChoice{\choice{increasing} \choice[correct]{decreasing}}  on $(-\infty, -1.67]$.
\item $p$ \wordChoice{\choice[correct]{increasing} \choice{decreasing}}  on $[-1.67, 3]$.
\item $p$ \wordChoice{\choice{increasing} \choice[correct]{decreasing}}  on $[3, \infty)$.
\end{itemize}



There is no global maximum or minimum.  There is no local maximum.



\end{example}




If we were also given that $p'(w) = \frac{1}{5}(3w^2 - 4w - 15)$.  This is a quadratic.  We can obtain its zeros via the quadratic formula.


\[  \frac{4 \pm \sqrt{(-4)^2 - 4 \cdot 3 \cdot (-15)}}{2 \cdot 3} =    \frac{4 \pm \sqrt{196}}{6}  = \frac{4 \pm 14}{6}       \]

We get two real roots: $\frac{4 + 14}{6} = \frac{18}{6} = 3$  and $\frac{4 - 14}{6} = \frac{-10}{6} = \frac{-5}{3} \approx -1.67$



\begin{itemize}
\item $p$ decreases on $\left(-\infty, \frac{-5}{3}\right]$.
\item $p$ increases on $\left[\frac{-5}{3}, 3\right]$.
\item $p$ decreases on $[3, \infty)$.
\end{itemize}



Another approach to obtain the roots of $p'(w) = \frac{1}{5}(3w^2 - 4w - 15)$ would be to notice that we already know that $3$ is a root. $w+3$ must be a factor.



\begin{align*}
3w^2 - 4w - 15     & = 0        \\
3w^2 - 4w - 15     & = (w+3)(???)        \\
3w^2 - 4w - 15     & = (w+3)(3w + ?)        \\
3w^2 - 4w - 15     & = (w+3)(\answer{3w + 5})        
\end{align*}














\end{document}
