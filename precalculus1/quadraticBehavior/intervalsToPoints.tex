\documentclass{ximera}


\graphicspath{
  {./}
  {ximeraTutorial/}
  {basicPhilosophy/}
}

\newcommand{\mooculus}{\textsf{\textbf{MOOC}\textnormal{\textsf{ULUS}}}}


\usepackage{tkz-euclide}\usepackage{tikz}
\usepackage{tikz-cd}
\usetikzlibrary{arrows}
\tikzset{>=stealth,commutative diagrams/.cd,
  arrow style=tikz,diagrams={>=stealth}} %% cool arrow head
\tikzset{shorten <>/.style={ shorten >=#1, shorten <=#1 } } %% allows shorter vectors

\usetikzlibrary{backgrounds} %% for boxes around graphs
\usetikzlibrary{shapes,positioning}  %% Clouds and stars
\usetikzlibrary{matrix} %% for matrix
\usepgfplotslibrary{polar} %% for polar plots
\usepgfplotslibrary{fillbetween} %% to shade area between curves in TikZ
\usetkzobj{all}
\usepackage[makeroom]{cancel} %% for strike outs
%\usepackage{mathtools} %% for pretty underbrace % Breaks Ximera
%\usepackage{multicol}
\usepackage{pgffor} %% required for integral for loops



%% http://tex.stackexchange.com/questions/66490/drawing-a-tikz-arc-specifying-the-center
%% Draws beach ball
\tikzset{pics/carc/.style args={#1:#2:#3}{code={\draw[pic actions] (#1:#3) arc(#1:#2:#3);}}}



\usepackage{array}
\setlength{\extrarowheight}{+.1cm}
\newdimen\digitwidth
\settowidth\digitwidth{9}
\def\divrule#1#2{
\noalign{\moveright#1\digitwidth
\vbox{\hrule width#2\digitwidth}}}
























%%This is to help with formatting on future title pages.
\newenvironment{sectionOutcomes}{}{}


\title{Tangents}

\begin{document}

\begin{abstract}
intervals to points
\end{abstract}
\maketitle





\section{Intervals and Secants}



Suppose we have a qudratic function $Q(x) = a (x - h)^2 + k$, with $a > 0$. 


Then its graph is a parabola.







\begin{image}
\begin{tikzpicture}
     \begin{axis}[
                domain=-5:10, ymax=10, xmax=10, ymin=-5, xmin=-5,
                axis lines =center, xlabel=$x$, ylabel=$y$,
                ytick={-4,-2,2,4,6,8,10},
                xtick={-4,-2,2,4,6,8,10},
                ticklabel style={font=\scriptsize},
                every axis y label/.style={at=(current axis.above origin),anchor=south},
                every axis x label/.style={at=(current axis.right of origin),anchor=west},
                axis on top,
                ]


        \addplot [draw=penColor, very thick, smooth, domain=(-1:7),<->] {0.5*(x-3)^2 + 2};
        \addplot [line width=1, gray, dashed,samples=100,domain=(-9.5:9.5)] ({3},{x});
        


        \addplot [color=penColor,only marks,mark=*] coordinates{(3,2)};
        \node[penColor] at (axis cs:4,1.5) {$(h, k)$};
        %\node[penColor] at (axis cs:5,-9) {$-0.5 x^2 - 5 x + 15.5$};



    \end{axis}
\end{tikzpicture}
\end{image}
We can examine the rate of change over any interval.


Let's examine the interval $[2, 5]$.


$\blacktriangleright$ \textbf{Algebraically}, the rate of change of $Q(x)$ over the interval $[2,5]$ is given by 

\[
\frac{Q(5) - Q(2)}{5 - 2} 
\]



$\blacktriangleright$ \textbf{Geometrically}, this is the slope of the \textbf{secant} line running through the points $(2, Q(2))$ and $(5, Q(5))$.














\begin{image}
\begin{tikzpicture}
     \begin{axis}[
                domain=-5:10, ymax=10, xmax=10, ymin=-5, xmin=-5,
                axis lines =center, xlabel=$x$, ylabel=$y$,
                ytick={-4,-2,2,4,6,8,10},
                xtick={-4,-2,2,4,6,8,10},
                ticklabel style={font=\scriptsize},
                every axis y label/.style={at=(current axis.above origin),anchor=south},
                every axis x label/.style={at=(current axis.right of origin),anchor=west},
                axis on top,
                ]


        \addplot [draw=penColor, very thick, smooth, domain=(-1:7),<->] {0.5*(x-3)^2 + 2};
        \addplot [line width=1, gray, dashed,samples=100,domain=(-9.5:9.5)] ({3},{x});
        

        \addplot [color=penColor2,only marks,mark=*] coordinates{(5,4)};
        \addplot [color=penColor2,only marks,mark=*] coordinates{(2,2.5)};
        
        \addplot [draw=penColor2, very thick, smooth, domain=(-2:9),<->] {0.5*(x-5) + 4};

        \addplot [color=penColor,only marks,mark=*] coordinates{(3,2)};
        \node[penColor] at (axis cs:4,1.5) {$(h, k)$};
        %\node[penColor] at (axis cs:5,-9) {$-0.5 x^2 - 5 x + 15.5$};



    \end{axis}
\end{tikzpicture}
\end{image}

Intervals and secants are algebraic and geometric partners. Secants give a picture of rates of change over an interval.



What if we push the secant over a little bit until it becomes a tangent line?







\begin{image}
\begin{tikzpicture}
     \begin{axis}[
                domain=-5:10, ymax=10, xmax=10, ymin=-5, xmin=-5,
                axis lines =center, xlabel=$x$, ylabel=$y$,
                ytick={-4,-2,2,4,6,8,10},
                xtick={-4,-2,2,4,6,8,10},
                ticklabel style={font=\scriptsize},
                every axis y label/.style={at=(current axis.above origin),anchor=south},
                every axis x label/.style={at=(current axis.right of origin),anchor=west},
                axis on top,
                ]


        \addplot [draw=penColor, very thick, smooth, domain=(-1:7),<->] {0.5*(x-3)^2 + 2};
        \addplot [line width=1, gray, dashed,samples=100,domain=(-9.5:9.5)] ({3},{x});
        

        \addplot [color=penColor2,only marks,mark=*] coordinates{(3.5,2.125)};
        
        \addplot [draw=penColor2, very thick, smooth, domain=(-2:9),<->] {0.5*(x-3.5) + 2.125};

        %\addplot [color=penColor,only marks,mark=*] coordinates{(3,2)};
        %\node[penColor] at (axis cs:4,1.5) {$(h, k)$};
        %\node[penColor] at (axis cs:5,-9) {$-0.5 x^2 - 5 x + 15.5$};



    \end{axis}
\end{tikzpicture}
\end{image}

Now it is a picture of the rate of change over the interval $\left[ \tfrac{7}{2}, \tfrac{7}{2} \right]$. \\


How should we interpret this? \\


What is the rate of change \textbf{\textcolor{red!90!darkgray}{AT}} a point? \\


You cannot calculate the rate of change over an interval with $0$ length.  Therefore, we will invent an interpretation for this rate of change at a point, i.e. \textit{instantaneous rate of change}.















\section{Tangent Lines}




Tangent lines are degenerate secants. Secant lines need two points.  A tangent line is a secant line where the two points are the same point. Tangent lines span an interval of length $0$.  But, tangent lines are still lines.  They have a constant rate of change. We'll call this the \textbf{instantaneous rate of change} at the domain number corresponding to the tangent point.

We'll being our investigation into the instantaneous with quadratic functions. \\


\begin{definition} \textbf{\textcolor{green!50!black}{Instantaneous Rate of Change}}  


Let $f$ be a quadratic function. Let $a$ be a number in the domain of $f$.

If the graph of $y = f(x)$ has a non-vertical tangent line at the point $(a, f(a))$, then the slope of this tangent line is the \textbf{instantaneous rate of change} of $f$ \textbf{at} a.


\end{definition}

We need a method of obtaining the slope of tangent lines to parabolas.




Let's consider the graph of $Q(x) = a (x - h)^2 + k$, with $a > 0$. We are investigating the graph of $y = a (x - h)^2 + k$. \\

Let's select a domain number, $x_0$, number of $Q$, which corresponds to the point $(x_0, y_0)$.

Let's add to our picture the tangent line at $(x_0, y_0)$.


\begin{image}
\begin{tikzpicture}
     \begin{axis}[
                domain=-5:10, ymax=10, xmax=10, ymin=-5, xmin=-5,
                axis lines =center, xlabel=$x$, ylabel=$y$,
                ytick={-4,-2,2,4,6,8,10},
                xtick={-4,-2,2,4,6,8,10},
                ticklabel style={font=\scriptsize},
                every axis y label/.style={at=(current axis.above origin),anchor=south},
                every axis x label/.style={at=(current axis.right of origin),anchor=west},
                axis on top,
                ]


        \addplot [draw=penColor, very thick, smooth, domain=(-1:7),<->] {0.5*(x-3)^2 + 2};
        \addplot [line width=1, gray, dashed,samples=100,domain=(-9.5:9.5)] ({3},{x});
        

        \addplot [color=penColor2,only marks,mark=*] coordinates{(5,4)};
        \node[penColor2] at (axis cs:6.5,4) {$(x_0, y_0)$};
        \addplot [draw=penColor2, very thick, smooth, domain=(1:7),<->] {2*(x-5) + 4};
        


        \addplot [color=penColor,only marks,mark=*] coordinates{(3,2)};
        \node[penColor] at (axis cs:2,1.5) {$(h, k)$};




    \end{axis}
\end{tikzpicture}
\end{image}

The tangent line is the graph of a linear function. Let's call it $T$.


T is a linear function. Therefore, the formula for $T$, would look like $T(x) = m(x - x_0) + y_0$, for some $m$.  $m$ is the instantaneous rate of change of $Q$ at $x_0$.  How do we determine its exact value?



$\blacktriangleright$ We would like a way to obtain the value of $m$. \\


To do this we need to note something about formulas for quadratic functions.


$\blacktriangleright$ \textbf{Double Roots}

Our general formula for a quadratic function looks like  $f(x) = a (x - h)^2 + k$ and the graph is a parabola with vertex at $(h, k)$. If we happen to have $k = 0$, then the vertex of the parabola sits on the horizontal axis and the formula looks like $f(x) = a (x - h)^2$.


In this case, $h$ is a double zero or double root of $f(x)$.  $(x - h)$ is a factor, twice.


Another way of looking at this situation is that the function


\[
\frac{f(x)}{x-h} = \frac{a (x - h)^2}{x - h} = a (x - h)
\]


also, has $h$ as a root.

This is obvious with the way we have written it here.  However, no matter how we write the formula for our quadratic function, if we divide by the factor correspoinding to the zero, then the result again has the same root.


With this in mind, we can get a formula for $m$, the slope of out tangent line at $(x_0, y_0)$ and the instanteous rate of change of $Q$ at $x_0$.






The story so far...

$Q(x) = a (x - h)^2 + k$ \\

$T(x) = m(x - x_0) + y_0$ \\





\begin{image}
\begin{tikzpicture}
     \begin{axis}[
                domain=-5:10, ymax=10, xmax=10, ymin=-5, xmin=-5,
                axis lines =center, xlabel=$x$, ylabel=$y$,
                ytick={-4,-2,2,4,6,8,10},
                xtick={-4,-2,2,4,6,8,10},
                ticklabel style={font=\scriptsize},
                every axis y label/.style={at=(current axis.above origin),anchor=south},
                every axis x label/.style={at=(current axis.right of origin),anchor=west},
                axis on top,
                ]


        \addplot [draw=penColor, very thick, smooth, domain=(-1:7),<->] {0.5*(x-3)^2 + 2};
        \addplot [line width=1, gray, dashed,samples=100,domain=(-9.5:9.5)] ({3},{x});
        

        \addplot [color=penColor2,only marks,mark=*] coordinates{(5,4)};
        \node[penColor2] at (axis cs:6.5,4) {$(x_0, y_0)$};
        \addplot [draw=penColor2, very thick, smooth, domain=(1:7),<->] {2*(x-5) + 4};
        


        \addplot [color=penColor,only marks,mark=*] coordinates{(3,2)};
        \node[penColor] at (axis cs:2,1.5) {$(h, k)$};




    \end{axis}
\end{tikzpicture}
\end{image}
The point of tangency is an intersection point of the two graphs.  In fact, it is more like a point of double intersection and will be the source of a double root.






Let's consider a new function $P(x) = Q(x) - T(x)$. \\


\[
P(x) = Q(x) - T(x) = a (x - h)^2 + k - (m(x - x_0) + y_0)
\]

This formula looks complicated, but it is still a quadratic.  We could multiply everything out and collect like terms.




\[
P(x) = Q(x) - T(x) = a\, x^2 - 2ah \, x + ah^2 + k - m \, x + m \, x_0 - y_0
\]


\[
P(x) = Q(x) - T(x) = a\, x^2 - (2ah - m) \, x + (ah^2 + k  + m \, x_0 - y_0)
\]



$P(x)$ is a quadratic function.  And, we know some characteristics of $P(x)$



\begin{itemize}
\item $P(x) = Q(x) - T(x) \geq 0$, since $Q(x) \geq T(x)$
\item $P(x_0) = Q(x_0) - T(x_0) = 0$
\end{itemize}




$P(x)$ is a quadratic function.  Its graph is a parabola opening up with $(x_0, 0)$ as its vertex. \\


$\blacktriangleright$   $x_0$ is a double root of $P(x)$. \\


$(x - x_0)$ will divide evenly into $P(x)$ and after dividing, the resulting function will also have $x_0$ as a zero.



We can use long division to divide $(x - x_0)$ into $P(x)$



\begin{image}
\includegraphics{longDivision.png}
\end{image}


If $(x - x_0)$ divides evenly into $P(x)$, which means this remainer should equal $0$.  Let's check.



We'll substitute $a (x_0 - h)^2 + k$ in for $y_0$.

\[
a h^2 + k - (a (x_0 - h)^2 + k) + a x_0^2 + m x_0 - x_0 (2 a h + m)
\]


\[
a h^2 + k - a x_0^2 + 2 a h x_0 - a h^2 - k + a x_0^2 + m x_0 - 2 a h x_0 - m x_0 = 0
\]


It is $0$.   \\


$(x - x_0)$ did divide evenly into $P(x)$, like it was supposed to.

The quotient is 


\[
a x + (a x_0 - (2 a h + m))
\]

And, we said $x_0$ must be a root, since $x_0$ was a double root of $P(x)$.  Therefore, if we substitute $x_0$ in for $x$, the expression must equal $0$.


\[
a x_0 + (a x_0 - (2 a h + m)) = 0
\]


\[
a x_0 + a x_0 - 2 a h - m = 0
\] 


We can solve this for $m$.



\[
m = a x_0 + a x_0 - 2 a h = 2 a (x_0 - h)
\] 



\begin{conclusion} Slope of Tangent Line


The graph of $y = a (x - h)^2 + k$ is a parabola.

Let $(x_0, y_0)$ be a point on this parabola.

Then the slope of the tangent line to the parabola at $(x_0, y_0)$ is given by 



\[ 2 a (x_0 - h) \]


\end{conclusion}
This for any point on the parabola.


\textbf{Note:} If we look at the vertex $(x_0, y_0) = (h, k)$, then $2 a (x_0 - h) = 2 a (h - h) = 0$ and the slope of the tangent line is $0$, just like we had reasoned before.





\section{iRoC}

Let $Q(x) = a (x - h)^2 + k$ be any quadratic function.

Every point, $(x_0, y_0)$, on the graph of $y = Q(x)$ has a tangent line.

Each tangent line has a slope, which we are calling the instantaneous rate of change of $Q$ at $x_0$.


$\blacktriangleright$ We can create a new function from this.



\begin{definition} \textbf{\textcolor{green!50!black}{iRoC}}  


Given a quadratic function, $Q(x) = a (x - h)^2 + k$, we define \textbf{the instantaneous rate of change of Q} to be the slope of the tangent line on the graph of $y = Q(x)$ at the point $(x, y)$.

$iRoC_Q(x)$ is a linear function given by 

\[  iRoC_Q(x) = 2 a (x-h) \]

\end{definition}


























\end{document}




