\documentclass{ximera}


\graphicspath{
  {./}
  {ximeraTutorial/}
  {basicPhilosophy/}
}

\newcommand{\mooculus}{\textsf{\textbf{MOOC}\textnormal{\textsf{ULUS}}}}


\usepackage{tkz-euclide}\usepackage{tikz}
\usepackage{tikz-cd}
\usetikzlibrary{arrows}
\tikzset{>=stealth,commutative diagrams/.cd,
  arrow style=tikz,diagrams={>=stealth}} %% cool arrow head
\tikzset{shorten <>/.style={ shorten >=#1, shorten <=#1 } } %% allows shorter vectors

\usetikzlibrary{backgrounds} %% for boxes around graphs
\usetikzlibrary{shapes,positioning}  %% Clouds and stars
\usetikzlibrary{matrix} %% for matrix
\usepgfplotslibrary{polar} %% for polar plots
\usepgfplotslibrary{fillbetween} %% to shade area between curves in TikZ
\usetkzobj{all}
\usepackage[makeroom]{cancel} %% for strike outs
%\usepackage{mathtools} %% for pretty underbrace % Breaks Ximera
%\usepackage{multicol}
\usepackage{pgffor} %% required for integral for loops



%% http://tex.stackexchange.com/questions/66490/drawing-a-tikz-arc-specifying-the-center
%% Draws beach ball
\tikzset{pics/carc/.style args={#1:#2:#3}{code={\draw[pic actions] (#1:#3) arc(#1:#2:#3);}}}



\usepackage{array}
\setlength{\extrarowheight}{+.1cm}
\newdimen\digitwidth
\settowidth\digitwidth{9}
\def\divrule#1#2{
\noalign{\moveright#1\digitwidth
\vbox{\hrule width#2\digitwidth}}}
























%%This is to help with formatting on future title pages.
\newenvironment{sectionOutcomes}{}{}


\title{Tangent Lines}

\begin{document}

\begin{abstract}
zooming in
\end{abstract}
\maketitle





\subsection*{Tangent Lines}



Suppose we have a quadratic function $Q(x) = \frac{1}{2} (x - 2)^2 + 1$. \\


Then its graph is a parabola. The point $\left( 3, \frac{3}{2} \right)$ is on the parabola.


The line $y=x-\frac{3}{2}$  goes through the point $\left( 3, \frac{3}{2} \right)$.

The parabola and line have a special relationship, which we can see by zooming in.



\begin{center}
\desmos{166ph20ncb}{400}{300}
\end{center}



\begin{center}
\desmos{nir27ultzi}{400}{300}
\end{center}



\begin{center}
\desmos{nzmyqtiy64}{400}{300}
\end{center}


\begin{center}
\desmos{52vfow5d2y}{400}{300}
\end{center}



As you zoom in, the graph slowly looks more and more like the line. \\


The line does the best job of approximating the graph at the point $\left( 3, \frac{3}{2} \right)$. \\


And, this is the only line that will work for the point $\left( 3, \frac{3}{2} \right)$.  Any other line, besides this one, will have a permanant angle between the graph and the line.  The graph and the line have the same ``slope'' at the point. \\


This line is called the tangent line for $Q(x) = \frac{1}{2} (x - 2)^2 + 1$ at the point $\left( 3, \frac{3}{2} \right)$.




\begin{definition} \textbf{\textcolor{green!50!black}{Tangent Line}}


Suppose $C$ is a curve, like the graph of a function. \\

Let $(a, b)$ be a point on the curve.

The line tangent to the curve, $C$, at the point $(a, b)$ is the line that goes through the point $(a, b)$ and does the best job of approximating the curve at the point $(a, b)$.

This line is called a \textbf{tangent line} or a \textbf{tangent}.


\end{definition}


Since ``slope'' is a characteristic of a line and a curve is not a line, then curves do not have a slope as we know that word. \\

However, it is easy to see that curves have a slope at points. They might have different slopes at different points, but our eyes definitely see curves moving in different directions.\\

We want to quantize this idea of a slope for a curve at a point. \\

If we can quantize it, then we can build a function for it and use that to measure a rate of change for our curve. \\

We will first investigate this idea with quadratic functions and parabolas.













\begin{center}
\textbf{\textcolor{green!50!black}{ooooo=-=-=-=-=-=-=-=-=-=-=-=-=ooOoo=-=-=-=-=-=-=-=-=-=-=-=-=ooooo}} \\

more examples can be found by following this link\\ \link[More Examples of Quadratic Behavior]{https://ximera.osu.edu/csccmathematics/precalculus1/precalculus1/quadraticBehavior/examples/exampleList}

\end{center}






\end{document}




