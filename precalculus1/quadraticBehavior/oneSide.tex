\documentclass{ximera}


\graphicspath{
  {./}
  {ximeraTutorial/}
  {basicPhilosophy/}
}

\newcommand{\mooculus}{\textsf{\textbf{MOOC}\textnormal{\textsf{ULUS}}}}


\usepackage{tkz-euclide}\usepackage{tikz}
\usepackage{tikz-cd}
\usetikzlibrary{arrows}
\tikzset{>=stealth,commutative diagrams/.cd,
  arrow style=tikz,diagrams={>=stealth}} %% cool arrow head
\tikzset{shorten <>/.style={ shorten >=#1, shorten <=#1 } } %% allows shorter vectors

\usetikzlibrary{backgrounds} %% for boxes around graphs
\usetikzlibrary{shapes,positioning}  %% Clouds and stars
\usetikzlibrary{matrix} %% for matrix
\usepgfplotslibrary{polar} %% for polar plots
\usepgfplotslibrary{fillbetween} %% to shade area between curves in TikZ
\usetkzobj{all}
\usepackage[makeroom]{cancel} %% for strike outs
%\usepackage{mathtools} %% for pretty underbrace % Breaks Ximera
%\usepackage{multicol}
\usepackage{pgffor} %% required for integral for loops



%% http://tex.stackexchange.com/questions/66490/drawing-a-tikz-arc-specifying-the-center
%% Draws beach ball
\tikzset{pics/carc/.style args={#1:#2:#3}{code={\draw[pic actions] (#1:#3) arc(#1:#2:#3);}}}



\usepackage{array}
\setlength{\extrarowheight}{+.1cm}
\newdimen\digitwidth
\settowidth\digitwidth{9}
\def\divrule#1#2{
\noalign{\moveright#1\digitwidth
\vbox{\hrule width#2\digitwidth}}}
























%%This is to help with formatting on future title pages.
\newenvironment{sectionOutcomes}{}{}


\title{Two Sides}

\begin{document}

\begin{abstract}
one side
\end{abstract}
\maketitle




Tangent lines are tangent to a curve or graph at a tangent point. There must be a point on the curve. At that tangent point, the curve or graph is behaving in some manner, which the tangent line is modeling.  \\



For quadratics, we have a formula for the tangent line. But, for a random function, how do you get a tangent line without know its slope ahead of time? \\


\textbf{\textcolor{red!90!darkgray}{$\blacktriangleright$}} There are two behaviors happening. \\



\begin{itemize}
\item \textbf{First}, the curve or graph is approaching that point. The graph is connecting up to the point.\\
\item \textbf{Second}, the tangent lines at approaching points are smoothly turning into the tangent line at the tangent point. \\
\end{itemize}


There are three situations to consider. \\










\section*{One Side}

Suppose our function doesn't have a discontinuity at $t$, but $t$ is just an endpoint of a domain interval. \\

The number is a domain number and there is a corresponding point on the graph.  The point $(t, f(t))$ is on the graph. \\










\begin{image}
\begin{tikzpicture}
     \begin{axis}[
                domain=-10:10, ymax=10, xmax=10, ymin=-6, xmin=-6,
                axis lines =center, xlabel=$x$, ylabel=$y$,
                ytick={-6,-4,-2,2,4,6,8,10},
                xtick={-6,-4,-2,2,4,6,8,10},
                ticklabel style={font=\scriptsize},
                every axis y label/.style={at=(current axis.above origin),anchor=south},
                every axis x label/.style={at=(current axis.right of origin),anchor=west},
                axis on top,
                ]


        \addplot [draw=penColor, very thick, smooth, domain=(0:4),<-] {(x-3)^2 - 4};


        \addplot [color=penColor,only marks,mark=*] coordinates{(4,-3)};


          \addplot [draw=penColor2, very thick, smooth, domain=(3:8),<->] {2*(x-4)-3};




    \end{axis}
\end{tikzpicture}
\end{image}


This point is an endpoint on the graph, in which case there might be a line that models one side of the graph. \\

A tangent line, well, a one-sided tangent line. \\







The approaching tangent lines, approach the tangent line, on one side only.



\begin{image}
\begin{tikzpicture}
     \begin{axis}[
                domain=-10:10, ymax=10, xmax=10, ymin=-6, xmin=-6,
                axis lines =center, xlabel=$x$, ylabel=$y$,
                ytick={-6,-4,-2,2,4,6,8,10},
                xtick={-6,-4,-2,2,4,6,8,10},
                ticklabel style={font=\scriptsize},
                every axis y label/.style={at=(current axis.above origin),anchor=south},
                every axis x label/.style={at=(current axis.right of origin),anchor=west},
                axis on top,
                ]


        \addplot [draw=penColor, very thick, smooth, domain=(0:4),<-] {(x-3)^2 - 4};


        \addplot [color=penColor,only marks,mark=*] coordinates{(4,-3)};

        


        \addplot [draw=penColor5, very thick, smooth, domain=(1:5)] {-4};
        \addplot [draw=penColor5, very thick, smooth, domain=(1:5)] {0.6*(x-3.3)-3.82};
        \addplot [draw=penColor5, very thick, smooth, domain=(1:5)] {1.2*(x-3.6)-3.64};


          \addplot [draw=penColor2, very thick, smooth, domain=(3:8),<->] {2*(x-4)-3};





    \end{axis}
\end{tikzpicture}
\end{image}

In this case, we have a one-side tangent line.  \\

We have a one-side derivative. \\


For the function above, we have a \textbf{left derivative}.











Or, the other side. \\

\begin{image}
\begin{tikzpicture}
     \begin{axis}[
                domain=-10:10, ymax=10, xmax=10, ymin=-6, xmin=-6,
                axis lines =center, xlabel=$x$, ylabel=$y$,
                ytick={-6,-4,-2,2,4,6,8,10},
                xtick={-6,-4,-2,2,4,6,8,10},
                ticklabel style={font=\scriptsize},
                every axis y label/.style={at=(current axis.above origin),anchor=south},
                every axis x label/.style={at=(current axis.right of origin),anchor=west},
                axis on top,
                ]


  
        \addplot [draw=penColor, very thick, smooth, domain=(4:6),->] {(x-3)^2 + 1};

        \addplot [color=penColor,only marks,mark=*] coordinates{(4,2)};

        
       


        \addplot [draw=penColor4, very thick, smooth, domain=(3:8)] {4*(x-5)+5};
        \addplot [draw=penColor4, very thick, smooth, domain=(3:8)] {3.2*(x-4.6)+3.56};
        \addplot [draw=penColor4, very thick, smooth, domain=(3:8)] {2.6*(x-4.3)+2.69};




          \addplot [draw=penColor2, very thick, smooth, domain=(3:8),<->] {2*(x-4)+2};





    \end{axis}
\end{tikzpicture}
\end{image}
In this case, we have a one-side tangent line.  \\

We have a one-side derivative. \\


For the function above, we have a \textbf{right derivative}.









\begin{definition} \textbf{\textcolor{green!50!black}{Derivative}}  a.k.a. two-sided 


Suppose we have a really nice situation.


\begin{itemize}
\item We have a function, $f$, 
\item We have and a domain number, $t$, 
\item This domain number is inside an open interval inside the domain.  $t \in (a, b) \subset Domain$. 
\item The tangent lines on both sides of $(t, f(t))$ are smootlhy approach the same line from both sides.
\end{itemize}

Then, there is a tangent line to the graph of $f$ and the tangent lines on both sides are smootly turning into this tangent line at the tangent point $(t, f(t))$. \\

The slope of this tangent line is the value of the \textbf{derivative} of $f$ at $t$.

\[
iRoC(t) =f'(t) = \text{slope of tangent line}
\]


On the other hand, if the tangent lines and both sides are not smootly agreeing, then there is no tangent line at $(t, f(t))$ and the derivative does not exist at $t$.   

\end{definition}

The derivative implies a two-sided derivative \\




\begin{definition} \textbf{\textcolor{green!50!black}{Left Derivative}}  a.k.a. one-sided 


Suppose we have a really nice situation.


\begin{itemize}
\item We have a function, $f$, 
\item We have and a domain number, $t$, 
\item This domain number is inside an  interval inside the domain of the form  $t \in (a, t] \subset Domain$. 
\item The tangent lines on the left side of $(t, f(t))$ are smootlhy approach the same line.
\end{itemize}

Then, there is a tangent line to the graph of $f$ and the tangent lines on the left side are smootly turning into this tangent line at the tangent point $(t, f(t))$. \\

The slope of this tangent line is the value of the \textbf{left derivative of $f$ at $t$}.

\[
iRoC_{f_{-}}(t) =f'_{-}(t) = \text{slope of tangent line}
\]


 

\end{definition}
The left derivative is symbolized with a subscript of ``-'' with the derivative name. \\





\begin{definition} \textbf{\textcolor{green!50!black}{Right Derivative}}  a.k.a. one-sided 


Suppose we have a really nice situation.


\begin{itemize}
\item We have a function, $f$, 
\item We have and a domain number, $t$, 
\item This domain number is inside an  interval inside the domain of the form  $t \in [t, b) \subset Domain$. 
\item The tangent lines on the right side of $(t, f(t))$ are smootlhy approach the same line.
\end{itemize}

Then, there is a tangent line to the graph of $f$ and the tangent lines on the right side are smootly turning into this tangent line at the tangent point $(t, f(t))$. \\

The slope of this tangent line is the value of the \textbf{right derivative of $f$ at $t$}.

\[
iRoC_{f_{+}}(t) =f'_{+}(t) = \text{slope of tangent line}
\]


 

\end{definition}

The right derivative is symbolized with a subscript of ``+'' with the derivative name. \\







\section*{No Side}


If the domain number is isolated, meaning no domain numbers immediately to the left or right, then there just is no derivative there.




















\begin{center}
\textbf{\textcolor{green!50!black}{ooooo=-=-=-=-=-=-=-=-=-=-=-=-=ooOoo=-=-=-=-=-=-=-=-=-=-=-=-=ooooo}} \\

more examples can be found by following this link\\ \link[More Examples of Quadratic Behavior]{https://ximera.osu.edu/csccmathematics/precalculus1/precalculus1/quadraticBehavior/examples/exampleList}

\end{center}




\end{document}




