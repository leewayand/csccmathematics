\documentclass{ximera}


\graphicspath{
  {./}
  {ximeraTutorial/}
  {basicPhilosophy/}
}

\newcommand{\mooculus}{\textsf{\textbf{MOOC}\textnormal{\textsf{ULUS}}}}


\usepackage{tkz-euclide}\usepackage{tikz}
\usepackage{tikz-cd}
\usetikzlibrary{arrows}
\tikzset{>=stealth,commutative diagrams/.cd,
  arrow style=tikz,diagrams={>=stealth}} %% cool arrow head
\tikzset{shorten <>/.style={ shorten >=#1, shorten <=#1 } } %% allows shorter vectors

\usetikzlibrary{backgrounds} %% for boxes around graphs
\usetikzlibrary{shapes,positioning}  %% Clouds and stars
\usetikzlibrary{matrix} %% for matrix
\usepgfplotslibrary{polar} %% for polar plots
\usepgfplotslibrary{fillbetween} %% to shade area between curves in TikZ
\usetkzobj{all}
\usepackage[makeroom]{cancel} %% for strike outs
%\usepackage{mathtools} %% for pretty underbrace % Breaks Ximera
%\usepackage{multicol}
\usepackage{pgffor} %% required for integral for loops



%% http://tex.stackexchange.com/questions/66490/drawing-a-tikz-arc-specifying-the-center
%% Draws beach ball
\tikzset{pics/carc/.style args={#1:#2:#3}{code={\draw[pic actions] (#1:#3) arc(#1:#2:#3);}}}



\usepackage{array}
\setlength{\extrarowheight}{+.1cm}
\newdimen\digitwidth
\settowidth\digitwidth{9}
\def\divrule#1#2{
\noalign{\moveright#1\digitwidth
\vbox{\hrule width#2\digitwidth}}}
























%%This is to help with formatting on future title pages.
\newenvironment{sectionOutcomes}{}{}


\title{Two Sides}

\begin{document}

\begin{abstract}
one side
\end{abstract}
\maketitle



Tangent lines are lines that are tangent to a curve or graph at a tangent point. There must be a point on the curve. At that tangent point, the curve or graph is behaving in some manner, which the tangent line is modeling.  \\

\begin{idea} \textbf{\textcolor{blue!55!black}{Tangent}}
A tangent line is a line that does the best job of pretending to be the curve at a single point.
\end{idea}
\textbf{Best job} means the line shares the tangent point with the curve and the slope of the line is the same as the ``slope'' of the curve at the tangent point. \\


For quadratics, we have a formula for the tangent line. But, for a random function, how do you get a tangent line without know its slope ahead of time? \\


\textbf{\textcolor{red!90!darkgray}{$\blacktriangleright$}} There are two behaviors happening. \\



\begin{itemize}
\item \textbf{First}, the curve or graph is approaching that point. The graph is connecting up to the point. The funciton is continuous at the corresponding domain number.\\
\item \textbf{Second}, the secant lines through the tangent point are smoothly turning into the tangent line at the tangent point. \\
\end{itemize}



\begin{warning} \textbf{\textcolor{red!80!black}{Both Sides}}

This means both sides. \\

The points on both sides are approaching the tangent point. \\

The secant lines on both sides are approaching the tangent line.

\end{warning}
We will also consider the situation where the graph only has one side. \\





\section{Two Sides Points and Secants Agree}


Suppose our tangent point is $(t_0 , f(t_0))$ and $t$ is inside an open interval in the domain, $t_0 \in (a, b)$. \\


Suppose the points on teh graph appraoch the point $(t_0 , f(t_0))$ and the secant lines approach the same line.



\textbf{\textcolor{blue!55!black}{Example}}  \\




Here is the graph of the function $f(x) = (x - 3)^2 - 4$ and the point $(4, f(4))= (4, -3)$. \\

\begin{image}
\begin{tikzpicture}
     \begin{axis}[
                domain=-10:10, ymax=10, xmax=10, ymin=-6, xmin=-6,
                axis lines =center, xlabel=$x$, ylabel=$y$,
                ytick={-6,-4,-2,2,4,6,8,10},
                xtick={-6,-4,-2,2,4,6,8,10},
                ticklabel style={font=\scriptsize},
                every axis y label/.style={at=(current axis.above origin),anchor=south},
                every axis x label/.style={at=(current axis.right of origin),anchor=west},
                axis on top,
                ]


        \addplot [draw=penColor, very thick, smooth, domain=(0:6),<->] {(x-3)^2 - 4};

        \addplot [color=penColor2,only marks,mark=*] coordinates{(4,-3)};
        


        %\node[penColor] at (axis cs:5,-4) {$(h, k)$};
        %\node[penColor] at (axis cs:5,-9) {$-0.5 x^2 - 5 x + 15.5$};



    \end{axis}
\end{tikzpicture}
\end{image}

A parabola and a point. \\


\textbf{First}, the points on the graph are approaching the point on both sides.  There is no break in the curve. $f$ is continuous at $4$.\\


The soon to be tangent point is $(4, 3)$ \\







\begin{image}
\begin{tikzpicture}
     \begin{axis}[
                domain=-10:10, ymax=10, xmax=10, ymin=-6, xmin=-6,
                axis lines =center, xlabel=$x$, ylabel=$y$,
                ytick={-6,-4,-2,2,4,6,8,10},
                xtick={-6,-4,-2,2,4,6,8,10},
                ticklabel style={font=\scriptsize},
                every axis y label/.style={at=(current axis.above origin),anchor=south},
                every axis x label/.style={at=(current axis.right of origin),anchor=west},
                axis on top,
                ]


        \addplot [draw=penColor, very thick, smooth, domain=(0:6),<->] {(x-3)^2 - 4};

        \addplot [color=penColor,only marks,mark=*] coordinates{(4,-3)};
        
        \addplot [draw=penColor2, very thick, smooth, domain=(3:8),<->] {2*(x-4)-3};

        %\node[penColor] at (axis cs:5,-4) {$(h, k)$};
        %\node[penColor] at (axis cs:5,-9) {$-0.5 x^2 - 5 x + 15.5$};



    \end{axis}
\end{tikzpicture}
\end{image}



The tangent line does the best job of modeling the curve right at the tangent point. \\


As it turns out, we know how to get the slope of lines tangent to a parabola.  Use the derivative or $iRoC_f$. We know from the derivative or $f'(x)=iRoC_f(x) = 2 (x-3)^2$, that the slope of this tangent line is $f'(3) = 2 (4-3)^2 = 2$ \\

That gives us a tangent line described by the equation $y = 2 (x - 4) - 3$.











\textbf{Second}, the secant lines lines through approaching points smoothly turn into the tangent line at the tangent point...on both sides.


Here are the secant lines through the tangent point and the approaching points $(3, -4)$, $(3.3, -3.91)$, $(3.7, -3.51)$, $(4.3, -2.31)$, $(4.7, --1.11)$, and $(5, 0)$.



\begin{image}
\begin{tikzpicture}
     \begin{axis}[
                domain=-10:10, ymax=10, xmax=10, ymin=-6, xmin=-6,
                axis lines =center, xlabel=$x$, ylabel=$y$,
                ytick={-6,-4,-2,2,4,6,8,10},
                xtick={-6,-4,-2,2,4,6,8,10},
                ticklabel style={font=\scriptsize},
                every axis y label/.style={at=(current axis.above origin),anchor=south},
                every axis x label/.style={at=(current axis.right of origin),anchor=west},
                axis on top,
                ]


        \addplot [draw=penColor, very thick, smooth, domain=(0:6),<->] {(x-3)^2 - 4};

        \addplot [color=penColor,only marks,mark=*] coordinates{(4,-3)};
        
       


        \addplot [draw=penColor4, very thick, smooth, domain=(3:8)] {(x-4)-3};
        \addplot [draw=penColor4, very thick, smooth, domain=(3:8)] {1.3*(x-4)-3};
        \addplot [draw=penColor4, very thick, smooth, domain=(3:8)] {1.7*(x-4)-3};

        \addplot [draw=penColor5, very thick, smooth, domain=(1:5)] {2.3*(x-4)-3};
        \addplot [draw=penColor5, very thick, smooth, domain=(1:5)] {2.7*(x-4)-3};
        \addplot [draw=penColor5, very thick, smooth, domain=(1:5)] {3*(x-4)-3};


          \addplot [draw=penColor2, very thick, smooth, domain=(3:8),<->] {2*(x-4)-3};



        %\node[penColor] at (axis cs:5,-4) {$(h, k)$};
        %\node[penColor] at (axis cs:5,-9) {$-0.5 x^2 - 5 x + 15.5$};



    \end{axis}
\end{tikzpicture}
\end{image}


The secant lines on the left smootly turn into the tangent line as you move to the right. The secant lines on the right smoothly turn into the tangent line as you move to the left. \\



































A function with a discontiuity has a break in the graph.



\begin{image}
\begin{tikzpicture}
     \begin{axis}[
                domain=-10:10, ymax=10, xmax=10, ymin=-6, xmin=-6,
                axis lines =center, xlabel=$x$, ylabel=$y$,
                ytick={-6,-4,-2,2,4,6,8,10},
                xtick={-6,-4,-2,2,4,6,8,10},
                ticklabel style={font=\scriptsize},
                every axis y label/.style={at=(current axis.above origin),anchor=south},
                every axis x label/.style={at=(current axis.right of origin),anchor=west},
                axis on top,
                ]


        \addplot [draw=penColor, very thick, smooth, domain=(0:4),<-] {(x-3)^2 - 4};
        \addplot [draw=penColor, very thick, smooth, domain=(4:6),->] {(x-3)^2 + 1};

        \addplot [color=penColor,only marks,mark=*] coordinates{(4,-3)};
        \addplot [color=penColor,fill=white,only marks,mark=*] coordinates{(4,2)};
        
       



          %\addplot [draw=penColor2, very thick, smooth, domain=(3:8),<->] {2*(x-4)-3};



        %\node[penColor] at (axis cs:5,-4) {$(h, k)$};
        %\node[penColor] at (axis cs:5,-9) {$-0.5 x^2 - 5 x + 15.5$};



    \end{axis}
\end{tikzpicture}
\end{image}













However, if there is a break in the graph, then the two sides might not agree. \\






\begin{image}
\begin{tikzpicture}
     \begin{axis}[
                domain=-10:10, ymax=10, xmax=10, ymin=-6, xmin=-6,
                axis lines =center, xlabel=$x$, ylabel=$y$,
                ytick={-6,-4,-2,2,4,6,8,10},
                xtick={-6,-4,-2,2,4,6,8,10},
                ticklabel style={font=\scriptsize},
                every axis y label/.style={at=(current axis.above origin),anchor=south},
                every axis x label/.style={at=(current axis.right of origin),anchor=west},
                axis on top,
                ]


        \addplot [draw=penColor, very thick, smooth, domain=(0:4),<-] {(x-3)^2 - 4};
        \addplot [draw=penColor, very thick, smooth, domain=(4:6),->] {(x-3)^2 + 1};

        \addplot [color=penColor,only marks,mark=*] coordinates{(4,-3)};
        \addplot [color=penColor,fill=white,only marks,mark=*] coordinates{(4,2)};
        
       


        \addplot [draw=penColor4, very thick, smooth, domain=(3:8)] {4*(x-5)+5};
        \addplot [draw=penColor4, very thick, smooth, domain=(3:8)] {3.2*(x-4.6)+3.56};
        \addplot [draw=penColor4, very thick, smooth, domain=(3:8)] {2.6*(x-4.3)+2.69};




        \addplot [draw=penColor5, very thick, smooth, domain=(1:5)] {-4};
        \addplot [draw=penColor5, very thick, smooth, domain=(1:5)] {0.6*(x-3.3)-3.82};
        \addplot [draw=penColor5, very thick, smooth, domain=(1:5)] {1.2*(x-3.6)-3.64};


       



    \end{axis}
\end{tikzpicture}
\end{image}

The two sides have tangent lines, but they are not smoothly turning to agree. \\


In this case, we say that there is not tangent line. \\

And, if there is no tangent line, then there is no slope of the tangent line, then the derivative has no value here. \\







So, a derivative implies that there are two sides and the two sides are agreeing. \\



But, there is no need to just throw everything else away.  We can extend our idea of derivative to include just one side. \\ 





































\begin{center}
\textbf{\textcolor{green!50!black}{ooooo=-=-=-=-=-=-=-=-=-=-=-=-=ooOoo=-=-=-=-=-=-=-=-=-=-=-=-=ooooo}} \\

more examples can be found by following this link\\ \link[More Examples of Quadratic Behavior]{https://ximera.osu.edu/csccmathematics/precalculus1/precalculus1/quadraticBehavior/examples/exampleList}

\end{center}




\end{document}




