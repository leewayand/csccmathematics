\documentclass{ximera}


\graphicspath{
  {./}
  {ximeraTutorial/}
  {basicPhilosophy/}
}

\newcommand{\mooculus}{\textsf{\textbf{MOOC}\textnormal{\textsf{ULUS}}}}


\usepackage{tkz-euclide}\usepackage{tikz}
\usepackage{tikz-cd}
\usetikzlibrary{arrows}
\tikzset{>=stealth,commutative diagrams/.cd,
  arrow style=tikz,diagrams={>=stealth}} %% cool arrow head
\tikzset{shorten <>/.style={ shorten >=#1, shorten <=#1 } } %% allows shorter vectors

\usetikzlibrary{backgrounds} %% for boxes around graphs
\usetikzlibrary{shapes,positioning}  %% Clouds and stars
\usetikzlibrary{matrix} %% for matrix
\usepgfplotslibrary{polar} %% for polar plots
\usepgfplotslibrary{fillbetween} %% to shade area between curves in TikZ
\usetkzobj{all}
\usepackage[makeroom]{cancel} %% for strike outs
%\usepackage{mathtools} %% for pretty underbrace % Breaks Ximera
%\usepackage{multicol}
\usepackage{pgffor} %% required for integral for loops



%% http://tex.stackexchange.com/questions/66490/drawing-a-tikz-arc-specifying-the-center
%% Draws beach ball
\tikzset{pics/carc/.style args={#1:#2:#3}{code={\draw[pic actions] (#1:#3) arc(#1:#2:#3);}}}



\usepackage{array}
\setlength{\extrarowheight}{+.1cm}
\newdimen\digitwidth
\settowidth\digitwidth{9}
\def\divrule#1#2{
\noalign{\moveright#1\digitwidth
\vbox{\hrule width#2\digitwidth}}}
























%%This is to help with formatting on future title pages.
\newenvironment{sectionOutcomes}{}{}


\title{Endpoints}

\begin{document}

\begin{abstract}
intersection
\end{abstract}
\maketitle







We have one slight hiccup in our open interval version of closeness and that is endpoints of intervals.








Graph of $y = K(r)$.

\begin{image}
\begin{tikzpicture}
  \begin{axis}[
            domain=-10:10, ymax=10, xmax=10, ymin=-10, xmin=-10,
            axis lines =center, xlabel={$r$}, ylabel={$y$}, grid = major,
            ytick={-10,-8,-6,-4,-2,2,4,6,8,10},
          	xtick={-10,-8,-6,-4,-2,2,4,6,8,10},
          	ticklabel style={font=\scriptsize},
            every axis y label/.style={at=(current axis.above origin),anchor=south},
            every axis x label/.style={at=(current axis.right of origin),anchor=west},
            axis on top
          ]
          
          	\addplot [line width=2, penColor, smooth,samples=100,domain=(-4:2),<-] {x+3};

      		\addplot[color=penColor,fill=white,only marks,mark=*] coordinates{(2,5)};
      		\addplot[color=penColor,fill=penColor,only marks,mark=*] coordinates{(2,-3)};


  \end{axis}
\end{tikzpicture}
\end{image}




If we want to examine the domain number $2$ and its weird function value $-3$, then our open interval idea is in trouble.  There is no filled space to the right of $2$ in the domain where we could carve out an open interval.\\


Actually, we don't need it. \\


We just need the domain numbers that are close to $2$.


\begin{quote}
We need all of the \textbf{\textcolor{purple!85!blue}{domain numbers}} that are close to our domain number under observation.
\end{quote}



We don't need all of the \textbf{\textcolor{purple!85!blue}{real numbers}} that are close to our domain number under observation. We just need the real numbers that are actually members of the domain.




\subsection{Slight Modification}


\textbf{Remember}, we are after some language that works ALL of the time, even for endpoints.   \\



``Space'' will refer to \textbf{\textcolor{red!90!darkgray}{domain space}} that exists ``around'' our domain number under observation. \\



$\blacktriangleright$ \textbf{\textcolor{purple!85!blue}{Language:}}   \\

\begin{center}
\textbf{All of the domain numbers inside a $\delta$-interval around $a$} 
\end{center}






$\blacktriangleright$ \textbf{\textcolor{purple!85!blue}{Notation:}}   \\

\begin{center}
$(a - \delta, a + \delta) \cap$ Domain
\end{center}



The intersection picks out the numbers that are both in the $\delta$-interval and in the domain. \\





$\blacktriangleright$ This actually fixes a hidden problem.  If our expectations are disrupted by a missing domain number, then we shouldn't be including that number in our $\delta$-interval.  However, if we stipulate that we are only looking at domain numbers inside the $\delta$-interval, then everything works. \\




\subsection{Tools}


We have our tools ready.

\begin{itemize}
\item Open intervals for space.
\item $\delta$-intervals for closeness
\end{itemize}


We are ready to begin our algebraic description of interrupted expectations in function values. 























\begin{center}
\textbf{\textcolor{green!50!black}{ooooo=-=-=-=-=-=-=-=-=-=-=-=-=ooOoo=-=-=-=-=-=-=-=-=-=-=-=-=ooooo}} \\

more examples can be found by following this link\\ \link[More Examples of Space]{https://ximera.osu.edu/csccmathematics/precalculus1/precalculus1/space/examples/exampleList}

\end{center}



\end{document}
