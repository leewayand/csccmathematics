\documentclass{ximera}


\graphicspath{
  {./}
  {ximeraTutorial/}
  {basicPhilosophy/}
}

\newcommand{\mooculus}{\textsf{\textbf{MOOC}\textnormal{\textsf{ULUS}}}}


\usepackage{tkz-euclide}\usepackage{tikz}
\usepackage{tikz-cd}
\usetikzlibrary{arrows}
\tikzset{>=stealth,commutative diagrams/.cd,
  arrow style=tikz,diagrams={>=stealth}} %% cool arrow head
\tikzset{shorten <>/.style={ shorten >=#1, shorten <=#1 } } %% allows shorter vectors

\usetikzlibrary{backgrounds} %% for boxes around graphs
\usetikzlibrary{shapes,positioning}  %% Clouds and stars
\usetikzlibrary{matrix} %% for matrix
\usepgfplotslibrary{polar} %% for polar plots
\usepgfplotslibrary{fillbetween} %% to shade area between curves in TikZ
\usetkzobj{all}
\usepackage[makeroom]{cancel} %% for strike outs
%\usepackage{mathtools} %% for pretty underbrace % Breaks Ximera
%\usepackage{multicol}
\usepackage{pgffor} %% required for integral for loops



%% http://tex.stackexchange.com/questions/66490/drawing-a-tikz-arc-specifying-the-center
%% Draws beach ball
\tikzset{pics/carc/.style args={#1:#2:#3}{code={\draw[pic actions] (#1:#3) arc(#1:#2:#3);}}}



\usepackage{array}
\setlength{\extrarowheight}{+.1cm}
\newdimen\digitwidth
\settowidth\digitwidth{9}
\def\divrule#1#2{
\noalign{\moveright#1\digitwidth
\vbox{\hrule width#2\digitwidth}}}
























%%This is to help with formatting on future title pages.
\newenvironment{sectionOutcomes}{}{}


\title{Continuity}

\begin{document}

\begin{abstract}
close as you want
\end{abstract}
\maketitle






A function being \textbf{continuous} at a domain number is the opposite of being discontinuous. \\


Function values are close when the domain numbers are close.










\begin{definition} \textbf{\textcolor{green!50!black}{Continuous}}

The function $f$ is continuous at $a$ if


\begin{itemize}
\item $a \in Dom_f$ - $a$ is in the domain of $f$, which means $f(a) \in \mathbb{R}$, i.e. $f(a)$ exists.
\item For every $\epsilon > 0$ defining an open interval around $f(a)$, namely $I = (f(a) - \epsilon, f(a) + \epsilon)$, there exists a corresponding $\delta > 0$, such that the image of open interval $(a - \delta, a + \delta)$ is a subset of $(f(a)-\epsilon, f(a)+\epsilon)$.
\end{itemize}


For \textbf{\textcolor{red!70!black}{EVERY}} open interval $I$ around $f(a)$, there exists a corresponding open interval around $a$, such that \textbf{\textcolor{red!70!black}{ALL}} of the function values are inside $I$.


\end{definition}





$\blacktriangleright$ \textbf{Discontinuity:}  You can find a single interval around $f(a)$, such that EVERY interval around $a$ contains a domain number whose function value is outside that interval.


$\blacktriangleright$ \textbf{Continuity:} For any and every selected interval around $f(a)$, you can find a corresponding interval around $a$, such that all of their function values are inside the selected interval.










\begin{example}


\[
\text{Let } \, f(x) = 
\begin{cases}
  2x-4 & \text{ on } (-2, 5) \\
  -x + 9  & \text{ on } [5, 9)
\end{cases}
\]





Graph of $y=f(x)$


\begin{image}
\begin{tikzpicture}
  \begin{axis}[
            domain=-10:10, ymax=10, xmax=10, ymin=-10, xmin=-10,
            axis lines =center, xlabel=$x$, ylabel=$y$, grid = major,
            ytick={-10,-8,-6,-4,-2,2,4,6,8,10},
            xtick={-10,-8,-6,-4,-2,2,4,6,8,10},
            ticklabel style={font=\scriptsize},
            every axis y label/.style={at=(current axis.above origin),anchor=south},
            every axis x label/.style={at=(current axis.right of origin),anchor=west},
            axis on top
          ]
          

          \addplot [line width=2, penColor, smooth,samples=200,domain=(-2:5),<-] {2*x-4};
          \addplot [line width=2, penColor, smooth,samples=200,domain=(5:9),->] {-x+9};

            %\addplot [line width=1, gray, dashed,samples=200,domain=(-10:10),<->] ({-4},{x});
            %\addplot [line width=1, gray, dashed,samples=200,domain=(-10:10),<->] ({x},{x});


          \addplot[color=penColor,fill=penColor,only marks,mark=*] coordinates{(5,4)};
          \addplot[color=penColor,fill=white,only marks,mark=*] coordinates{(5,6)};






           

  \end{axis}
\end{tikzpicture}
\end{image}


$f$ is continuous at $3$. \\



\begin{explanation}

First, $f(3) = 2$. \\


Pick ANY open interval around $2$:  $(2 - \epsilon, 2 + \epsilon)$, where $\epsilon > 0 $ is small. 

Then there is a corresponding interval for the domain. Namely choose $\delta = \frac{\epsilon}{4}$.

Let $I = (3 - \delta, 3 + \delta) = \left(3 - \frac{\epsilon}{4}, 3 + \frac{\epsilon}{4}\right)$


The image of $I$, $f(I) = f\left(\,\left(3 - \frac{\epsilon}{4}, 3 + \frac{\epsilon}{4}\right)\,\right) = \left(2 - \frac{\epsilon}{2}, 2 + \frac{\epsilon}{2}\right)$

And, $\left(2 - \frac{\epsilon}{2}, 2 + \frac{\epsilon}{2}\right) \subset  (2 - \epsilon, 2 + \epsilon) $


\textbf{ALL} of the function values land inside the original $(2 - \epsilon, 2 + \epsilon)$ interval.


This can be done \textbf{\textcolor{red!70!black}{FOR ANY}} $\epsilon > 0$. \\


For ANY open interval around $f(a)$, no matter how small, a corresponding open interval around $a$ can be found, such that the function value from that interval around $a$ never jump out of the interval around $f(a)$.










\end{explanation}
\end{example}





\textbf{Continuity:} No matter what $\epsilon$ is choosen for $f(a)$, you can ALWAYS find a corresponding $\delta$ for $a$. \\



\textbf{Discontinuity:} There is an $\epsilon$ for $f(a)$, such that there is no $\delta$ for $a$. \\

























\begin{center}
\textbf{\textcolor{green!50!black}{ooooo-=-=-=-ooOoo-=-=-=-ooooo}} \\

more examples can be found by following this link\\ \link[More Examples of Continuity]{https://ximera.osu.edu/csccmathematics/precalculus1/precalculus1/continuity/examples/exampleList}

\end{center}





\end{document}
