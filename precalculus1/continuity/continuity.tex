\documentclass{ximera}


\graphicspath{
  {./}
  {ximeraTutorial/}
  {basicPhilosophy/}
}

\newcommand{\mooculus}{\textsf{\textbf{MOOC}\textnormal{\textsf{ULUS}}}}


\usepackage{tkz-euclide}\usepackage{tikz}
\usepackage{tikz-cd}
\usetikzlibrary{arrows}
\tikzset{>=stealth,commutative diagrams/.cd,
  arrow style=tikz,diagrams={>=stealth}} %% cool arrow head
\tikzset{shorten <>/.style={ shorten >=#1, shorten <=#1 } } %% allows shorter vectors

\usetikzlibrary{backgrounds} %% for boxes around graphs
\usetikzlibrary{shapes,positioning}  %% Clouds and stars
\usetikzlibrary{matrix} %% for matrix
\usepgfplotslibrary{polar} %% for polar plots
\usepgfplotslibrary{fillbetween} %% to shade area between curves in TikZ
\usetkzobj{all}
\usepackage[makeroom]{cancel} %% for strike outs
%\usepackage{mathtools} %% for pretty underbrace % Breaks Ximera
%\usepackage{multicol}
\usepackage{pgffor} %% required for integral for loops



%% http://tex.stackexchange.com/questions/66490/drawing-a-tikz-arc-specifying-the-center
%% Draws beach ball
\tikzset{pics/carc/.style args={#1:#2:#3}{code={\draw[pic actions] (#1:#3) arc(#1:#2:#3);}}}



\usepackage{array}
\setlength{\extrarowheight}{+.1cm}
\newdimen\digitwidth
\settowidth\digitwidth{9}
\def\divrule#1#2{
\noalign{\moveright#1\digitwidth
\vbox{\hrule width#2\digitwidth}}}
























%%This is to help with formatting on future title pages.
\newenvironment{sectionOutcomes}{}{}


\title{Continuity}

\begin{document}

\begin{abstract}
close as you want
\end{abstract}
\maketitle











A \textbf{discontinuity} in a function is first of all a domain number.  The function has a value at this number. Secondly, around this domain number, the function values are not close to the value at the discontinuity. There is space between the function value at the discontinuity and the surrounding function values, no matter how close you get to the discontinuity. \\












Let $a$ be a discontinuity of $f$, so $a$ is in the domain of $f$.   Then $f(a)$ is a value in the range of $f$. Since $a$ is a discontinuity, there must be a distance (perhaps really small), which we'll call $\epsilon$, which defines an open interval around $f(a)$ - namely $(f(a)-\epsilon, f(a)+\epsilon)$.  \\


And, no matter how close you get to $a$, there are domain numbers closer to $a$ where the function value is outside $(f(a)-\epsilon, f(a)+\epsilon)$. \\



No matter how small $\delta>0$ is, there is always a number, $b \in (a-\delta, a+\delta)$ such that $ f(b) \not\in(f(a)-\epsilon, f(a)+\epsilon)$ \\







\begin{definition} \textbf{\textcolor{green!50!black}{Discontinuity}}

The real number $a$ is a \textbf{discontinuity} of the function $f$ if


\begin{itemize}
\item $a \in Dom_f$ - $a$ is in the domain of $f$, which means $f(a) \in \mathbb{R}$, i.e. $f(a)$ exists.
\item There is an $\epsilon > 0$ defining an open interval around $f(a)$, namely $I = (f(a) - \epsilon, f(a) + \epsilon)$, such that for any $\delta > 0$, the open interval $(a - \delta, a + \delta)$ always contains a domain number $b$, such that $f(b) \not\in(f(a)-\epsilon, f(a)+\epsilon)$.
\end{itemize}


There is an open interval $I$ around $f(a)$, such that EVERY open interval around $a$ contains a domain number where the function value is outside $I$.


\end{definition}

Otherwise, $a$ would be a place of \textbf{continuity} of $f$. \\





\begin{example}


\[
\text{Let } \, f(x) = 
\begin{cases}
  2x-4 & \text{ on } (-2, 5) \\
  -x + 9  & \text{ on } [5, 9)
\end{cases}
\]





Graph of $y=f(x)$


\begin{image}
\begin{tikzpicture}
  \begin{axis}[
            domain=-10:10, ymax=10, xmax=10, ymin=-10, xmin=-10,
            axis lines =center, xlabel=$x$, ylabel=$y$, grid = major,
            ytick={-10,-8,-6,-4,-2,2,4,6,8,10},
            xtick={-10,-8,-6,-4,-2,2,4,6,8,10},
            ticklabel style={font=\scriptsize},
            every axis y label/.style={at=(current axis.above origin),anchor=south},
            every axis x label/.style={at=(current axis.right of origin),anchor=west},
            axis on top
          ]
          

          \addplot [line width=2, penColor, smooth,samples=200,domain=(-2:5),<-] {2*x-4};
          \addplot [line width=2, penColor, smooth,samples=200,domain=(5:9),->] {-x+9};

            %\addplot [line width=1, gray, dashed,samples=200,domain=(-10:10),<->] ({-4},{x});
            %\addplot [line width=1, gray, dashed,samples=200,domain=(-10:10),<->] ({x},{x});


          \addplot[color=penColor,fill=penColor,only marks,mark=*] coordinates{(5,4)};
          \addplot[color=penColor,fill=white,only marks,mark=*] coordinates{(5,6)};






           

  \end{axis}
\end{tikzpicture}
\end{image}


$5$ is a disconinuity for $f$. \\



\begin{explanation}


First, $5$ is in the domain and $f(5)=4$ is in the range.  Our plan is to pick an open interval around $4$ in the range and show that no matter how close you get to $5$ in the domain, there are always domain numbers whose function value is not inside our interval.



For our selection, pick a radius of $\epsilon = 1$ and consider the interval $(4-\epsilon , 4+\epsilon) = (4-1 , 4+1) = (3,5)$. \\










\begin{image}
\begin{tikzpicture}
  \begin{axis}[
            domain=-10:10, ymax=10, xmax=10, ymin=-10, xmin=-10,
            axis lines =center, xlabel=$x$, ylabel=$y$, grid = major,
            ytick={-10,-8,-6,-4,-2,2,4,6,8,10},
            xtick={-10,-8,-6,-4,-2,2,4,6,8,10},
            ticklabel style={font=\scriptsize},
            every axis y label/.style={at=(current axis.above origin),anchor=south},
            every axis x label/.style={at=(current axis.right of origin),anchor=west},
            axis on top
          ]
          

          \addplot [line width=2, penColor, smooth,samples=200,domain=(-2:5),<-] {2*x-4};
          \addplot [line width=2, penColor, smooth,samples=200,domain=(5:9),->] {-x+9};

            %\addplot [line width=1, gray, dashed,samples=200,domain=(-10:10),<->] ({-4},{x});
            %\addplot [line width=1, gray, dashed,samples=200,domain=(-10:10),<->] ({x},{x});


          \addplot[color=red,fill=red,only marks,mark=*] coordinates{(5,4)};
          \addplot[color=penColor,fill=white,only marks,mark=*] coordinates{(5,6)};

          \addplot [line width=3, red, smooth,samples=200,domain=(3:5)] ({0},{x});






           

  \end{axis}
\end{tikzpicture}
\end{image}















$\vartriangleright$ Can we select a domain interval CONTAINING $5$, small enough, so that all domain numbers inside this interval have their function values inside $(3,5)$?  The answer is yes.

Select $[5, 6)$. \\



$\vartriangleright$ Can we select an \textbf{OPEN} interval containing $5$, small enough, so that all domain numbers inside this interval have their function values inside $(3,5)$?  The answer is no.




Let $0.1 > \delta > 0$ be a number as small as you wish.

Consider the open interval $(5-\delta, 5+\delta)$ around $5$. It doesn't matter how small $\delta$ is.  The number $5 - \frac{\delta}{2}$ is always inside this open interval and the function value there is 

\[  f\left(5 - \frac{\delta}{2}\right)     = 2\left(5 - \frac{\delta}{2}\right)-4 = 6 - \delta     \]












\begin{image}
\begin{tikzpicture}
  \begin{axis}[
            domain=-10:10, ymax=10, xmax=10, ymin=-10, xmin=-10,
            axis lines =center, xlabel=$x$, ylabel=$y$, grid = major,
            ytick={-10,-8,-6,-4,-2,2,4,6,8,10},
            xtick={-10,-8,-6,-4,-2,2,4,6,8,10},
            ticklabel style={font=\scriptsize},
            every axis y label/.style={at=(current axis.above origin),anchor=south},
            every axis x label/.style={at=(current axis.right of origin),anchor=west},
            axis on top
          ]
          

          \addplot [line width=2, penColor, smooth,samples=200,domain=(-2:5),<-] {2*x-4};
          \addplot [line width=2, penColor, smooth,samples=200,domain=(5:9),->] {-x+9};

            %\addplot [line width=1, gray, dashed,samples=200,domain=(-10:10),<->] ({-4},{x});
            %\addplot [line width=1, gray, dashed,samples=200,domain=(-10:10),<->] ({x},{x});


          \addplot[color=red,fill=red,only marks,mark=*] coordinates{(5,4)};
          \addplot[color=penColor,fill=white,only marks,mark=*] coordinates{(5,6)};

          \addplot [line width=3, red, smooth,samples=200,domain=(3:5)] ({0},{x});
          \addplot [line width=3, green, smooth,samples=200,domain=(4.5:5.5)] ({x},{0});

          \addplot [line width=1, violet, smooth,samples=200,domain=(0:5.5)] ({4.75},{x});
          \addplot [line width=1, violet, smooth,samples=200,domain=(0:4.75)] ({x},{5.5});






           

  \end{axis}
\end{tikzpicture}
\end{image}












Since $0.1 > \delta$, we know for sure that $6 - \delta > 5$.  $6 - \delta$ is NEVER inside $(3, 5)$, for ANY $\delta$, which means for ANY open interval.

There is no open interval around $5$, such that all function values of all domain numbers inside this interval are always inside $(3,5)$. \\




We have produced an $\epsilon$, such that there is no open interval around $5$ where the function values of domain numbers inside this interval are always inside $(4-\epsilon , 4+\epsilon)$.


That is, algebraically, what is means to be a discontinuity.



\end{explanation}
\end{example}




On the other hand, \textbf{continuity} means that you can ALWAYS find an open domain interval, no matter what $\epsilon$ is choosen for the range. \\






\[
\text{Let } \, f(x) = 
\begin{cases}
  2x-4 & \text{ on } (-2, 5) \\
  -x + 9  & \text{ on } [5, 9)
\end{cases}
\]

$f$ is continuous at $3$. \\
$f(3)=2$ \\

Pick ANY open range interval around $2$:  $(2 - \epsilon, 2 + \epsilon)$, where $\epsilon > 0 $ is small. 

Then there is a corresponding interval for the domain. Namely choose $\delta = \frac{\epsilon}{4}$.

Let $I = (3 - \delta, 3 + \delta) = \left(3 - \frac{\epsilon}{4}, 3 + \frac{\epsilon}{4}\right)$


The image of $I$, $f(I) = f\left(\,\left(3 - \frac{\epsilon}{4}, 3 + \frac{\epsilon}{4}\right)\,\right) = \left(2 - \frac{\epsilon}{2}, 2 + \frac{\epsilon}{2}\right)$

And, $\left(2 - \frac{\epsilon}{2}, 2 + \frac{\epsilon}{2}\right) \subset  (2 - \epsilon, 2 + \epsilon) $


\textbf{ALL} of the function values land inside the original $(2 - \epsilon, 2 + \epsilon)$ interval.


This can be done FOR ANY $\epsilon > 0$. \\


The function value NEVER jumps out of this interval.

















\begin{example}  Discontinuity


\[
\text{Let } \, T(k) = 
\begin{cases}
  0 & \text{ on } \mathbb{R} - \left\{ \frac{1}{n} \, | \, n \in \mathbb{N}     \right\} \\
  1  & \text{ on } \left\{ \frac{1}{n} \, | \, n \in \mathbb{N}     \right\}
\end{cases}
\]



Claim: $0$ is a discontinuity of $f$. \\



\begin{explanation}

$T(k)$ is $0$ everywhere except the reciprocals of the natural numbers.

The natural numbers are $\{ 1, 2, 3, 4, \cdots \}$.

The reciprocals of the natural numbers are $\left\{ 1, \frac{1}{2}, \frac{1}{3}, \frac{1}{4}, \cdots \right\}$.

On the reciprocals, $f\left(\frac{1}{n}\right) = 1$ \\



First, $0$ is in the domain and $f(0) = 0$.

Second, for any small $0.1 > \epsilon > 0$ choosen, the interval $(-\epsilon, \epsilon)$ is an open interval around $f(0) = 0$. \\



Now, consider any $\delta > 0$ and the interval $(-\delta, \delta)$. No mater how small $\delta$ is, there is an $n_0 \in \mathbb{N}$, such that $0 < \frac{1}{n_0} < \delta$ and $f\left(\frac{1}{n_0}\right) = 1$. outside $(-\epsilon, \epsilon)$. \\




For \textbf{\textcolor{red!50!blue!90!black}{ANY}} choosen $\epsilon$-interval around $f(0)= 0$, there is no corresponding $\delta$ such that \textbf{\textcolor{red!50!blue!90!black}{ALL}} of the function values from  $(-\delta, \delta)$ land inside $(-\epsilon, \epsilon)$.


$0$ is a discontinuity of $f$.

\end{explanation}

\end{example}






In the example above, $T(k)$ didn't have a big space surrounding the function value.  Instead it just had an infinite number of single individual values jump away from the target value.


The example demonstrates that functions do just about any and every weird thing you can think of.

But Elementary Functions are not weird.  They are nice.  





\begin{definition} \textbf{\textcolor{green!50!black}{Singularity}}

For Precalculus, a \textbf{singularity} has the same basic definition as a discontinuity, except the singularity is not in the domain. \\


The function is not defined at a singularity.

\end{definition}

\textbf{Note:} In Calculus, we will modify this definition to include places where the derivatives don't exist. 










\section{Elementary Functions}




The Elementary Functions don't do weird things.  They are almost always continuous everywhere (in their domain).  When they do have discontinuities or singularities, they are very nice with obvious jumps.






\begin{itemize} 

\item \textbf{Polynomial Function}  

Polynomials are continuous everywhere. They have no discontinuities or singularities. Constant functions and linear functions are the nicest of the polynomial functions.






\item \textbf{Roots and Radicals}

Roots and radical functions are continuous everywhere.  Not everywhere on the real numbers, because their domains are often not $\mathbb{R}$. Everywhere for functions means everywhere on their domain.






\item \textbf{Exponential Functions}

Exponential functions are continuous everywhere (on their domain).





\item \textbf{Logarithmic Functions}

Logarithmic functions are continuous everywhere (on their domain).






\item \textbf{Rational Functions}

Rational functions are continuous everywhere (on their domain).  They have singularities and these are represented with vertical asymptotes and holes on their graphs.  But singularities are not in the domain.  Rational functions are not defined at a singularity. Rational functions do not have discontinuities.  They are continuous everywhere (on their domain). 





\item \textbf{Absolute Value}

The absolute value function is continuous everywhere.







\item \textbf{Trigonometric Functions}
Trigonometric functions are continuous everywhere (on their domain).  They have singularities, like tangent, and these singularities are represented with vertical asymptotes on the graphs.  But singularities are not in the domain.  The trigonometric functions are not defined at singularities. Trigonometric functions do not have discontinuities.  They are continuous everywhere (on their domain). 








\end{itemize}




The Elementary Functions are very nice.  They have no discontinuites.  They are continuous everywhere on their domains - or just continuous everywhere.





The first example of a simple function with a discontinuity is the Heaviside step (unit step) function.






\[
Heaviside(x) = 
\begin{cases}
  0 & \text{ on } (-\infty, 0) \\
  \tfrac{1}{2} & \text{ at } 0 \\
  1 & \text{ on } (0, \infty) 
\end{cases}
\]






We also have the \textbf{Greatest Integer} function - also called the \textbf{floor} function.


\[
floor(x) = \lfloor x \rfloor = \, \text{ the greatest integer less than or equal to } \, x
\]





Graph of $y = \lfloor x\rfloor$ is below, except the steps keep going up to the right and down to the left.  The extra 3 dots are a graphical symbol communicating that the pattern continues.
\begin{image}
\begin{tikzpicture}
  \begin{axis}[
            domain=-3:5,
            width=6in,
            height=4in,
            axis lines =middle, xlabel=$x$, ylabel=$y$,
            every axis y label/.style={at=(current axis.above origin),anchor=south},
            every axis x label/.style={at=(current axis.right of origin),anchor=west},
            clip=false,
            %axis on top,
          ]
          \addplot [textColor, very thin, domain=(0:2.3)] {0}; % puts the axis back, axis on top clobbers our open holes
          \addplot [textColor, very thin] plot coordinates {(0,0) (0,2)}; % puts the axis back, axis on top clobbers our open holes
          \addplot [very thick, penColor, domain=(-2:-1)] {-2};
          \addplot [very thick, penColor, domain=(-1:0)] {-1};
          \addplot [very thick, penColor, domain=(0:1)] {0};
          \addplot [very thick, penColor, domain=(1:2)] {1};
          \addplot [very thick, penColor, domain=(2:3)] {2};
          \addplot [very thick, penColor, domain=(3:4)] {3};
          \addplot[color=penColor,fill=penColor,only marks,mark=*] coordinates{(-2,-2)};  %% closed hole          
          \addplot[color=penColor,fill=penColor,only marks,mark=*] coordinates{(-1,-1)};  %% closed hole          
          \addplot[color=penColor,fill=penColor,only marks,mark=*] coordinates{(0,0)};  %% closed hole          
          \addplot[color=penColor,fill=penColor,only marks,mark=*] coordinates{(1,1)};  %% closed hole          
          \addplot[color=penColor,fill=penColor,only marks,mark=*] coordinates{(2,2)};  %% closed hole  
          \addplot[color=penColor,fill=penColor,only marks,mark=*] coordinates{(3,3)};  %% closed hole                  
          \addplot[color=penColor,fill=background,only marks,mark=*] coordinates{(-1,-2)};  %% open hole
          \addplot[color=penColor,fill=background,only marks,mark=*] coordinates{(0,-1)};  %% open hole
          \addplot[color=penColor,fill=background,only marks,mark=*] coordinates{(1,0)};  %% open hole
          \addplot[color=penColor,fill=background,only marks,mark=*] coordinates{(2,1)};  %% open hole
          \addplot[color=penColor,fill=background,only marks,mark=*] coordinates{(3,2)};  %% open hole
          \addplot[color=penColor,fill=background,only marks,mark=*] coordinates{(4,3)};  %% open hole

          \addplot[color=penColor,fill=penColor,only marks,mark=*] coordinates{(3.7,3.5) (3.8,3.6) (3.9,3.7)};  %% 3 dots     
          \addplot[color=penColor,fill=penColor,only marks,mark=*] coordinates{(-1.7,-2.5) (-1.8,-2.6) (-1.9,-2.7)};  %% 3 dots     


        \end{axis}
\end{tikzpicture}
\end{image}












The floor function illustrates a general feeling about elementary functions. The only way discontiniities are created is through the use of piecewise defined functions.  To create a discontinuity, we have to break a nice function and just move a piece of it to somewhere else.











\section{Forcing Discontinuities}



Let's make some discontinuities.

We'll take pieces of different elementary functions and glue them together via piecewise defined functions. We have three main types of discontinuities and singularities.





\begin{example} Jumps



A jump discontinuity occurs when the function is continuous to the left and right side of the discontinuity, but the two sides do not match up.



The function must be defined for a discontinuity.  The corresponding point could be an endpoint for either side or neither.



\begin{image}
\begin{tikzpicture}
  \begin{axis}[
            domain=-10:10, ymax=10, xmax=10, ymin=-10, xmin=-10,
            axis lines =center, xlabel=$x$, ylabel={$y=g(x)$}, grid = major,
            ytick={-10,-8,-6,-4,-2,2,4,6,8,10},
          	xtick={-10,-8,-6,-4,-2,2,4,6,8,10},
          	yticklabels={$-10$,$-8$,$-6$,$-4$,$-2$,$2$,$4$,$6$,$8$,$10$}, 
          	xticklabels={$-10$,$-8$,$-6$,$-4$,$-2$,$2$,$4$,$6$,$8$,$10$},
            ticklabel style={font=\scriptsize},
            every axis y label/.style={at=(current axis.above origin),anchor=south},
            every axis x label/.style={at=(current axis.right of origin),anchor=west},
            axis on top
          ]
          
      		\addplot [line width=2, penColor, smooth,samples=100,domain=(-6:2),<-] {-2*x-3};
          	\addplot [line width=2, penColor, smooth,samples=100,domain=(2:8)] {1.75*x-8};

      		%\addplot[color=penColor,fill=penColor2,only marks,mark=*] coordinates{(-6,9)};
      		\addplot[color=penColor,fill=penColor,only marks,mark=*] coordinates{(2,-7)};

      		\addplot[color=penColor,fill=white,only marks,mark=*] coordinates{(2,-4.5)};
      		\addplot[color=penColor,fill=white,only marks,mark=*] coordinates{(8,6)};


           

  \end{axis}
\end{tikzpicture}
\end{image}




















\begin{image}
\begin{tikzpicture}
  \begin{axis}[
            domain=-10:10, ymax=10, xmax=10, ymin=-10, xmin=-10,
            axis lines =center, xlabel=$x$, ylabel={$y=g(x)$}, grid = major,
            ytick={-10,-8,-6,-4,-2,2,4,6,8,10},
          	xtick={-10,-8,-6,-4,-2,2,4,6,8,10},
          	yticklabels={$-10$,$-8$,$-6$,$-4$,$-2$,$2$,$4$,$6$,$8$,$10$}, 
          	xticklabels={$-10$,$-8$,$-6$,$-4$,$-2$,$2$,$4$,$6$,$8$,$10$},
            ticklabel style={font=\scriptsize},
            every axis y label/.style={at=(current axis.above origin),anchor=south},
            every axis x label/.style={at=(current axis.right of origin),anchor=west},
            axis on top
          ]
          
      		\addplot [line width=2, penColor, smooth,samples=100,domain=(-6:2),<-] {-2*x-3};
          	\addplot [line width=2, penColor, smooth,samples=100,domain=(2:8)] {1.75*x-8};

      		%\addplot[color=penColor,fill=penColor2,only marks,mark=*] coordinates{(-6,9)};
      		\addplot[color=penColor,fill=white,only marks,mark=*] coordinates{(2,-7)};

      		\addplot[color=penColor,fill=penColor,only marks,mark=*] coordinates{(2,-4.5)};
      		\addplot[color=penColor,fill=white,only marks,mark=*] coordinates{(8,6)};


           

  \end{axis}
\end{tikzpicture}
\end{image}










Or, the point could be off on its own, isolated.







\begin{image}
\begin{tikzpicture}
  \begin{axis}[
            domain=-10:10, ymax=10, xmax=10, ymin=-10, xmin=-10,
            axis lines =center, xlabel=$x$, ylabel={$y=g(x)$}, grid = major,
            ytick={-10,-8,-6,-4,-2,2,4,6,8,10},
          	xtick={-10,-8,-6,-4,-2,2,4,6,8,10},
          	yticklabels={$-10$,$-8$,$-6$,$-4$,$-2$,$2$,$4$,$6$,$8$,$10$}, 
          	xticklabels={$-10$,$-8$,$-6$,$-4$,$-2$,$2$,$4$,$6$,$8$,$10$},
            ticklabel style={font=\scriptsize},
            every axis y label/.style={at=(current axis.above origin),anchor=south},
            every axis x label/.style={at=(current axis.right of origin),anchor=west},
            axis on top
          ]
          
      		\addplot [line width=2, penColor, smooth,samples=100,domain=(-6:2),<-] {-2*x-3};
          	\addplot [line width=2, penColor, smooth,samples=100,domain=(2:8)] {1.75*x-8};

      		%\addplot[color=penColor,fill=penColor2,only marks,mark=*] coordinates{(-6,9)};
      		\addplot[color=penColor,fill=penColor,only marks,mark=*] coordinates{(2,4)};

      		\addplot[color=penColor,fill=white,only marks,mark=*] coordinates{(2,-4.5)};
      		\addplot[color=penColor,fill=white,only marks,mark=*] coordinates{(8,6)};
      		\addplot[color=penColor,fill=white,only marks,mark=*] coordinates{(2,-7)};


           

  \end{axis}
\end{tikzpicture}
\end{image}






A jump singularity doesn't have a point at all. The number is not included in the domain.











\begin{image}
\begin{tikzpicture}
  \begin{axis}[
            domain=-10:10, ymax=10, xmax=10, ymin=-10, xmin=-10,
            axis lines =center, xlabel=$x$, ylabel={$y=g(x)$}, grid = major,
            ytick={-10,-8,-6,-4,-2,2,4,6,8,10},
          	xtick={-10,-8,-6,-4,-2,2,4,6,8,10},
          	yticklabels={$-10$,$-8$,$-6$,$-4$,$-2$,$2$,$4$,$6$,$8$,$10$}, 
          	xticklabels={$-10$,$-8$,$-6$,$-4$,$-2$,$2$,$4$,$6$,$8$,$10$},
            ticklabel style={font=\scriptsize},
            every axis y label/.style={at=(current axis.above origin),anchor=south},
            every axis x label/.style={at=(current axis.right of origin),anchor=west},
            axis on top
          ]
          
      		\addplot [line width=2, penColor, smooth,samples=100,domain=(-6:2),<-] {-2*x-3};
          	\addplot [line width=2, penColor, smooth,samples=100,domain=(2:8)] {1.75*x-8};

      		%\addplot[color=penColor,fill=penColor2,only marks,mark=*] coordinates{(-6,9)};
      		%\addplot[color=penColor,fill=penColor,only marks,mark=*] coordinates{(2,4)};

      		\addplot[color=penColor,fill=white,only marks,mark=*] coordinates{(2,-4.5)};
      		\addplot[color=penColor,fill=white,only marks,mark=*] coordinates{(8,6)};
      		\addplot[color=penColor,fill=white,only marks,mark=*] coordinates{(2,-7)};


           

  \end{axis}
\end{tikzpicture}
\end{image}






\end{example} 





























\begin{example} Removeable Discontinuity



A removeable discontinuity occurs when the function is continuous to the left and right side of the discontinuity, and the two sides do  match up.  However, the point at the discontinuity is moved, leaving a hole.






\begin{image}
\begin{tikzpicture}
  \begin{axis}[
            domain=-10:10, ymax=10, xmax=10, ymin=-10, xmin=-10,
            axis lines =center, xlabel=$x$, ylabel={$y=g(x)$}, grid = major,
            ytick={-10,-8,-6,-4,-2,2,4,6,8,10},
          	xtick={-10,-8,-6,-4,-2,2,4,6,8,10},
          	yticklabels={$-10$,$-8$,$-6$,$-4$,$-2$,$2$,$4$,$6$,$8$,$10$}, 
          	xticklabels={$-10$,$-8$,$-6$,$-4$,$-2$,$2$,$4$,$6$,$8$,$10$},
            ticklabel style={font=\scriptsize},
            every axis y label/.style={at=(current axis.above origin),anchor=south},
            every axis x label/.style={at=(current axis.right of origin),anchor=west},
            axis on top
          ]
          
      		\addplot [line width=2, penColor, smooth,samples=100,domain=(-6:2),<-] {-2*x-3};
          	%\addplot [line width=2, penColor, smooth,samples=100,domain=(2:8)] {1.75*x-8};

      		%\addplot[color=penColor,fill=penColor2,only marks,mark=*] coordinates{(-6,9)};
      		\addplot[color=penColor,fill=penColor,only marks,mark=*] coordinates{(2,-7)};

      		%\addplot[color=penColor,fill=white,only marks,mark=*] coordinates{(2,-4.5)};
      		%\addplot[color=penColor,fill=white,only marks,mark=*] coordinates{(8,6)};

      		\addplot[color=penColor,fill=white,only marks,mark=*] coordinates{(-2,1)};
      		\addplot[color=penColor,fill=penColor,only marks,mark=*] coordinates{(-2,5)};


           

  \end{axis}
\end{tikzpicture}
\end{image}




\end{example}

This type of discontinuity can be "removed" simply by redefining the function value at that one number and moving the point back onto the graph to plug the hole.



Just as with a removeable discontinuity, a removeable singularity is created by removing a single point from the graph and the number from the domain.











\begin{image}
\begin{tikzpicture}
  \begin{axis}[
            domain=-10:10, ymax=10, xmax=10, ymin=-10, xmin=-10,
            axis lines =center, xlabel=$x$, ylabel={$y=g(x)$}, grid = major,
            ytick={-10,-8,-6,-4,-2,2,4,6,8,10},
          	xtick={-10,-8,-6,-4,-2,2,4,6,8,10},
          	yticklabels={$-10$,$-8$,$-6$,$-4$,$-2$,$2$,$4$,$6$,$8$,$10$}, 
          	xticklabels={$-10$,$-8$,$-6$,$-4$,$-2$,$2$,$4$,$6$,$8$,$10$},
            ticklabel style={font=\scriptsize},
            every axis y label/.style={at=(current axis.above origin),anchor=south},
            every axis x label/.style={at=(current axis.right of origin),anchor=west},
            axis on top
          ]
          
      		\addplot [line width=2, penColor, smooth,samples=100,domain=(-6:2),<-] {-2*x-3};
          	%\addplot [line width=2, penColor, smooth,samples=100,domain=(2:8)] {1.75*x-8};

      		%\addplot[color=penColor,fill=penColor2,only marks,mark=*] coordinates{(-6,9)};
      		\addplot[color=penColor,fill=penColor,only marks,mark=*] coordinates{(2,-7)};

      		%\addplot[color=penColor,fill=white,only marks,mark=*] coordinates{(2,-4.5)};
      		%\addplot[color=penColor,fill=white,only marks,mark=*] coordinates{(8,6)};

      		\addplot[color=penColor,fill=white,only marks,mark=*] coordinates{(-2,1)};
      		%\addplot[color=penColor,fill=penColor,only marks,mark=*] coordinates{(-2,5)};


           

  \end{axis}
\end{tikzpicture}
\end{image}



This type of singularity can be "removed" simply by placing the number back into the domain and defining the function value to plug up the hole in the graph.































\begin{example} Asymptotic


The third type of discontinuity and singularity is when the values of function become unbounded near a domain number.  Graphically, these are represented with asymptotes.  There are several possible configurations.










Discontinuities and singularities where both sides become unbounded.



\begin{image}
\begin{tikzpicture}
  \begin{axis}[
            domain=-10:10, ymax=10, xmax=10, ymin=-10, xmin=-10,
            axis lines =center, xlabel=$x$, ylabel={$y=g(x)$}, grid = major,
            ytick={-10,-8,-6,-4,-2,2,4,6,8,10},
          	xtick={-10,-8,-6,-4,-2,2,4,6,8,10},
          	yticklabels={$-10$,$-8$,$-6$,$-4$,$-2$,$2$,$4$,$6$,$8$,$10$}, 
          	xticklabels={$-10$,$-8$,$-6$,$-4$,$-2$,$2$,$4$,$6$,$8$,$10$},
            ticklabel style={font=\scriptsize},
            every axis y label/.style={at=(current axis.above origin),anchor=south},
            every axis x label/.style={at=(current axis.right of origin),anchor=west},
            axis on top
          ]
          
			\addplot [line width=2, penColor, smooth, domain=(-9:-3.1),<->] {(x-1)/((x+3)*(x-4))};
			\addplot [line width=2, penColor, smooth, domain=(-2.9:3.9),<->] {(x-1)/((x+3)*(x-4))};
			\addplot [line width=2, penColor, smooth, domain=(4.1:9),<->] {(x-1)/((x+3)*(x-4))};

			\addplot [line width=1, gray, dashed, domain=(-9.5:9.5),<->] ({-3},{x});
			\addplot [line width=1, gray, dashed, domain=(-9.5:9.5),<->] ({4},{x});

      		\addplot[color=penColor,fill=penColor,only marks,mark=*] coordinates{(-3,1)};
      		\addplot[color=penColor,fill=penColor,only marks,mark=*] coordinates{(4,-6)};


           

  \end{axis}
\end{tikzpicture}
\end{image}







\begin{image}
\begin{tikzpicture}
  \begin{axis}[
            domain=-10:10, ymax=10, xmax=10, ymin=-10, xmin=-10,
            axis lines =center, xlabel=$x$, ylabel={$y=g(x)$}, grid = major,
            ytick={-10,-8,-6,-4,-2,2,4,6,8,10},
          	xtick={-10,-8,-6,-4,-2,2,4,6,8,10},
          	yticklabels={$-10$,$-8$,$-6$,$-4$,$-2$,$2$,$4$,$6$,$8$,$10$}, 
          	xticklabels={$-10$,$-8$,$-6$,$-4$,$-2$,$2$,$4$,$6$,$8$,$10$},
            ticklabel style={font=\scriptsize},
            every axis y label/.style={at=(current axis.above origin),anchor=south},
            every axis x label/.style={at=(current axis.right of origin),anchor=west},
            axis on top
          ]
          
			\addplot [line width=2, penColor, smooth, domain=(-9:-3.1),<->] {(x-1)/((x+3)*(x-4))};
			\addplot [line width=2, penColor, smooth, domain=(-2.9:3.9),<->] {(x-1)/((x+3)*(x-4))};
			\addplot [line width=2, penColor, smooth, domain=(4.1:9),<->] {(x-1)/((x+3)*(x-4))};

			\addplot [line width=1, gray, dashed, domain=(-9.5:9.5),<->] ({-3},{x});
			\addplot [line width=1, gray, dashed, domain=(-9.5:9.5),<->] ({4},{x});

      		%\addplot[color=penColor,fill=penColor,only marks,mark=*] coordinates{(-3,1)};
      		%\addplot[color=penColor,fill=penColor,only marks,mark=*] coordinates{(4,-6)};


           

  \end{axis}
\end{tikzpicture}
\end{image}



















Discontinuities and singularities where only one side becomes unbounded.



\begin{image}
\begin{tikzpicture}
  \begin{axis}[
            domain=-10:10, ymax=10, xmax=10, ymin=-10, xmin=-10,
            axis lines =center, xlabel=$x$, ylabel={$y=g(x)$}, grid = major,
            ytick={-10,-8,-6,-4,-2,2,4,6,8,10},
          	xtick={-10,-8,-6,-4,-2,2,4,6,8,10},
          	yticklabels={$-10$,$-8$,$-6$,$-4$,$-2$,$2$,$4$,$6$,$8$,$10$}, 
          	xticklabels={$-10$,$-8$,$-6$,$-4$,$-2$,$2$,$4$,$6$,$8$,$10$},
            ticklabel style={font=\scriptsize},
            every axis y label/.style={at=(current axis.above origin),anchor=south},
            every axis x label/.style={at=(current axis.right of origin),anchor=west},
            axis on top
          ]
          
			%\addplot [line width=2, penColor, smooth, domain=(-9:-3.1),<->] {(x-1)/((x+3)*(x-4))};
			\addplot [line width=2, penColor, smooth, domain=(-8:-3),<-] {x-1};
			\addplot [line width=2, penColor, smooth, domain=(-2.9:3.9),<->] {(x-1)/((x+3)*(x-4))};
			\addplot [line width=2, penColor, smooth, domain=(4.1:9),->] {-x+8};

			\addplot [line width=1, gray, dashed, domain=(-9.5:9.5),<->] ({-3},{x});
			\addplot [line width=1, gray, dashed, domain=(-9.5:9.5),<->] ({4},{x});

      		\addplot[color=penColor,fill=penColor,only marks,mark=*] coordinates{(-3,-4)};
      		\addplot[color=penColor,fill=penColor,only marks,mark=*] coordinates{(4,4)};


           

  \end{axis}
\end{tikzpicture}
\end{image}






\begin{image}
\begin{tikzpicture}
  \begin{axis}[
            domain=-10:10, ymax=10, xmax=10, ymin=-10, xmin=-10,
            axis lines =center, xlabel=$x$, ylabel={$y=g(x)$}, grid = major,
            ytick={-10,-8,-6,-4,-2,2,4,6,8,10},
          	xtick={-10,-8,-6,-4,-2,2,4,6,8,10},
          	yticklabels={$-10$,$-8$,$-6$,$-4$,$-2$,$2$,$4$,$6$,$8$,$10$}, 
          	xticklabels={$-10$,$-8$,$-6$,$-4$,$-2$,$2$,$4$,$6$,$8$,$10$},
            ticklabel style={font=\scriptsize},
            every axis y label/.style={at=(current axis.above origin),anchor=south},
            every axis x label/.style={at=(current axis.right of origin),anchor=west},
            axis on top
          ]
          
			%\addplot [line width=2, penColor, smooth, domain=(-9:-3.1),<->] {(x-1)/((x+3)*(x-4))};
			\addplot [line width=2, penColor, smooth, domain=(-8:-3),<-] {x-1};
			\addplot [line width=2, penColor, smooth, domain=(-2.9:3.9),<->] {(x-1)/((x+3)*(x-4))};
			\addplot [line width=2, penColor, smooth, domain=(4.1:9),->] {-x+8};

			\addplot [line width=1, gray, dashed, domain=(-9.5:9.5),<->] ({-3},{x});
			\addplot [line width=1, gray, dashed, domain=(-9.5:9.5),<->] ({4},{x});

      		\addplot[color=penColor,fill=white,only marks,mark=*] coordinates{(-3,-4)};
      		\addplot[color=penColor,fill=white,only marks,mark=*] coordinates{(4,4)};


           

  \end{axis}
\end{tikzpicture}
\end{image}





\end{example}







Those are the three main types of discontinuities and singularities we will encounter.  Rational functions naturally have asymptotic and removeable singularities.  Other than that, we use piecewise defined functions to create discontiuities and singularities.

Elementary functions are continuous everywhere (in their domain).  That means around a domain number, the function behaves like the value of the function at the domain number.


\[      \lim_{x \to a} f(x) = f(a)                   \]



No matter how small of an open interval you choose around $f(a)$, you can find a corresponding interval around $a$, such that ALL of the function values from domain numbers in this interval land inside the chosen range interval.





















\end{document}
