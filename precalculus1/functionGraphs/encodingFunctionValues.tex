\documentclass{ximera}


\graphicspath{
  {./}
  {ximeraTutorial/}
  {basicPhilosophy/}
}

\newcommand{\mooculus}{\textsf{\textbf{MOOC}\textnormal{\textsf{ULUS}}}}


\usepackage{tkz-euclide}\usepackage{tikz}
\usepackage{tikz-cd}
\usetikzlibrary{arrows}
\tikzset{>=stealth,commutative diagrams/.cd,
  arrow style=tikz,diagrams={>=stealth}} %% cool arrow head
\tikzset{shorten <>/.style={ shorten >=#1, shorten <=#1 } } %% allows shorter vectors

\usetikzlibrary{backgrounds} %% for boxes around graphs
\usetikzlibrary{shapes,positioning}  %% Clouds and stars
\usetikzlibrary{matrix} %% for matrix
\usepgfplotslibrary{polar} %% for polar plots
\usepgfplotslibrary{fillbetween} %% to shade area between curves in TikZ
\usetkzobj{all}
\usepackage[makeroom]{cancel} %% for strike outs
%\usepackage{mathtools} %% for pretty underbrace % Breaks Ximera
%\usepackage{multicol}
\usepackage{pgffor} %% required for integral for loops



%% http://tex.stackexchange.com/questions/66490/drawing-a-tikz-arc-specifying-the-center
%% Draws beach ball
\tikzset{pics/carc/.style args={#1:#2:#3}{code={\draw[pic actions] (#1:#3) arc(#1:#2:#3);}}}



\usepackage{array}
\setlength{\extrarowheight}{+.1cm}
\newdimen\digitwidth
\settowidth\digitwidth{9}
\def\divrule#1#2{
\noalign{\moveright#1\digitwidth
\vbox{\hrule width#2\digitwidth}}}
























%%This is to help with formatting on future title pages.
\newenvironment{sectionOutcomes}{}{}


\title{Encoding Visually}


\begin{document}

\begin{abstract}
pairs to dots
\end{abstract}
\maketitle


Function notation allows us to talk about individual pairs inside a function.


\[
\large{ (d, F(d))}
\]

$d$ is sitting in the left position of our ordered pair, therefore it represents a domain number. $F(d)$ represents the value of the function at $d$ and is written on the right in the ordered pair.


\begin{example}
SUppose 4 is a member of the domain of the function $H$. Then $H(4)$ represents the value of $H$ at $4$. $H(4)$ is a member of the range of $H$. $(4, H(4))$ is a pair in the function $H$.

If we happen to know that $(4, 9)$ is a pair in the funciton $H$, then we know $H(4) = 9$.

\end{example}


Aside from talking about individual pairs, we might also like to talk about the whole collection at once.  Our first attempt at this is via pictures. We need a way to visually represent a single pair and then convert all pairs to a picture.  With this picture, we can analyze the function as a whole, identify important places in the domain, detect trands in the data, quick estimate information about the whole function.




























\end{document}
