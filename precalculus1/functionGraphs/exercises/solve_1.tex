\documentclass{ximera}


\graphicspath{
  {./}
  {ximeraTutorial/}
  {basicPhilosophy/}
}

\newcommand{\mooculus}{\textsf{\textbf{MOOC}\textnormal{\textsf{ULUS}}}}


\usepackage{tkz-euclide}\usepackage{tikz}
\usepackage{tikz-cd}
\usetikzlibrary{arrows}
\tikzset{>=stealth,commutative diagrams/.cd,
  arrow style=tikz,diagrams={>=stealth}} %% cool arrow head
\tikzset{shorten <>/.style={ shorten >=#1, shorten <=#1 } } %% allows shorter vectors

\usetikzlibrary{backgrounds} %% for boxes around graphs
\usetikzlibrary{shapes,positioning}  %% Clouds and stars
\usetikzlibrary{matrix} %% for matrix
\usepgfplotslibrary{polar} %% for polar plots
\usepgfplotslibrary{fillbetween} %% to shade area between curves in TikZ
\usetkzobj{all}
\usepackage[makeroom]{cancel} %% for strike outs
%\usepackage{mathtools} %% for pretty underbrace % Breaks Ximera
%\usepackage{multicol}
\usepackage{pgffor} %% required for integral for loops



%% http://tex.stackexchange.com/questions/66490/drawing-a-tikz-arc-specifying-the-center
%% Draws beach ball
\tikzset{pics/carc/.style args={#1:#2:#3}{code={\draw[pic actions] (#1:#3) arc(#1:#2:#3);}}}



\usepackage{array}
\setlength{\extrarowheight}{+.1cm}
\newdimen\digitwidth
\settowidth\digitwidth{9}
\def\divrule#1#2{
\noalign{\moveright#1\digitwidth
\vbox{\hrule width#2\digitwidth}}}
























%%This is to help with formatting on future title pages.
\newenvironment{sectionOutcomes}{}{}


\author{Lee Wayand}

\begin{document}
\begin{exercise}  Behavior  












Below is the graph of $y=M(t)$.  

\begin{image}
\begin{tikzpicture} 
  \begin{axis}[
            domain=-10:10, ymax=10, xmax=10, ymin=-10, xmin=-10,
            axis lines =center, xlabel=$t$, ylabel=$y$, grid = major,
            ytick={-10,-8,-6,-4,-2,2,4,6,8,10},
            xtick={-10,-8,-6,-4,-2,2,4,6,8,10},
            ticklabel style={font=\scriptsize},
            every axis y label/.style={at=(current axis.above origin),anchor=south},
            every axis x label/.style={at=(current axis.right of origin),anchor=west},
            axis on top
          ]
          
          \addplot [line width=2, penColor, smooth,samples=100,domain=(-8:-3)] ({x},{-3*x-16});
          \addplot [line width=2, penColor, smooth,samples=100,domain=(0:9)] ({x},{0.5*x-8});
          \addplot [line width=2, penColor, smooth,samples=100,domain=(-3:0)] ({x},{-x+1});

          \addplot [color=penColor,fill=white,only marks,mark=*] coordinates{(-8,8) (0,-8) (9,-3.5)};
          \addplot [color=penColor,only marks,mark=*] coordinates{(-3,-7) (9,4) (-3,4) (0,1)};


           

  \end{axis}
\end{tikzpicture}
\end{image}




\begin{question}
How many solutions does $P(t) = -4$ have?

\begin{multipleChoice}
\choice {0} 
\choice {1} 
\choice [correct]{2} 
\choice {3} 
\choice {4} 
\end{multipleChoice}
\end{question}







\begin{question}
How many solutions does $P(t) = 0$ have?

\begin{multipleChoice}
\choice {0} 
\choice [correct]{1} 
\choice {2} 
\choice {3} 
\choice {4} 
\end{multipleChoice}
\end{question}







\begin{question}
How many solutions does $P(t) = 4$ have?

\begin{multipleChoice}
\choice {0} 
\choice {1} 
\choice {2} 
\choice [correct]{3} 
\choice {4} 
\end{multipleChoice}
\end{question}







\begin{question}
How many solutions does $P(t) = -8$ have?

\begin{multipleChoice}
\choice [correct]{0} 
\choice {1} 
\choice {2} 
\choice {3} 
\choice {4} 
\end{multipleChoice}
\end{question}











\end{exercise}
\end{document}