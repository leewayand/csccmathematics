\documentclass{ximera}


\graphicspath{
  {./}
  {ximeraTutorial/}
  {basicPhilosophy/}
}

\newcommand{\mooculus}{\textsf{\textbf{MOOC}\textnormal{\textsf{ULUS}}}}


\usepackage{tkz-euclide}\usepackage{tikz}
\usepackage{tikz-cd}
\usetikzlibrary{arrows}
\tikzset{>=stealth,commutative diagrams/.cd,
  arrow style=tikz,diagrams={>=stealth}} %% cool arrow head
\tikzset{shorten <>/.style={ shorten >=#1, shorten <=#1 } } %% allows shorter vectors

\usetikzlibrary{backgrounds} %% for boxes around graphs
\usetikzlibrary{shapes,positioning}  %% Clouds and stars
\usetikzlibrary{matrix} %% for matrix
\usepgfplotslibrary{polar} %% for polar plots
\usepgfplotslibrary{fillbetween} %% to shade area between curves in TikZ
\usetkzobj{all}
\usepackage[makeroom]{cancel} %% for strike outs
%\usepackage{mathtools} %% for pretty underbrace % Breaks Ximera
%\usepackage{multicol}
\usepackage{pgffor} %% required for integral for loops



%% http://tex.stackexchange.com/questions/66490/drawing-a-tikz-arc-specifying-the-center
%% Draws beach ball
\tikzset{pics/carc/.style args={#1:#2:#3}{code={\draw[pic actions] (#1:#3) arc(#1:#2:#3);}}}



\usepackage{array}
\setlength{\extrarowheight}{+.1cm}
\newdimen\digitwidth
\settowidth\digitwidth{9}
\def\divrule#1#2{
\noalign{\moveright#1\digitwidth
\vbox{\hrule width#2\digitwidth}}}
























%%This is to help with formatting on future title pages.
\newenvironment{sectionOutcomes}{}{}


\title{Function Graphs}


\begin{document}

\begin{abstract}

\end{abstract}
\maketitle


Functions are relations, which makes them packages.  They contain three sets.  They contain a set of real numbers called the \textbf{domain}.  They contain a seond set of real numbers called the \textbf{range}. Finally, they contain a third set of pairs of numbers.  The pairs are all constructed with a number from each of the domain and range.

The pairs the point of the function.  The pairs are the connection between the domain and range.  They identify which numbers are associated together from both sets.

We can talk about these pairs one by one with \textbf{function notation}.

\[
\large{F(d)}
\]


$F$ is the name of the function and $d$ is a domain number.  $F(d)$ represents the range partner of $d$.  $F(d)$ and $d$ are partnered together in $F$.

When we write down such a pairing, we usually write  the domain number on the left and the range partner (or funciton value) on the right, separated by a comma, and wrapped with parentheses.

\[ 
\large{(d, F(d))} 
\]

Independent of this, we might actually have an expression for this range partner, say $r$.  We can communicate this with an equation


\[
\large{F(d)} = r
\]


\section{See the Whole Picture}

Aside from talking about individual pairs, we might also like to talk about the whole collection at once.  Our first attempt at this is via pictures. We need a way to visually represent a single pair and then convert all pairs to a picture.  With this picture, we can analyze the function as a whole, identify important places in the domain, detect trands in the data, quick estimate information about the whole function.


















\begin{sectionOutcomes}
After completing this section, students should 

\begin{itemize}
\item encode funciton pairs into dots.
\item decipher dots into function pairs.
\item use function notation to communicate about its graph.
\item evaluate functions via its graph.
\item solve equations involving funciton notation via its graph.
\end{itemize}
\end{sectionOutcomes}

\end{document}
