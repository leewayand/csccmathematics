\documentclass{ximera}


\graphicspath{
  {./}
  {ximeraTutorial/}
  {basicPhilosophy/}
}

\newcommand{\mooculus}{\textsf{\textbf{MOOC}\textnormal{\textsf{ULUS}}}}


\usepackage{tkz-euclide}\usepackage{tikz}
\usepackage{tikz-cd}
\usetikzlibrary{arrows}
\tikzset{>=stealth,commutative diagrams/.cd,
  arrow style=tikz,diagrams={>=stealth}} %% cool arrow head
\tikzset{shorten <>/.style={ shorten >=#1, shorten <=#1 } } %% allows shorter vectors

\usetikzlibrary{backgrounds} %% for boxes around graphs
\usetikzlibrary{shapes,positioning}  %% Clouds and stars
\usetikzlibrary{matrix} %% for matrix
\usepgfplotslibrary{polar} %% for polar plots
\usepgfplotslibrary{fillbetween} %% to shade area between curves in TikZ
\usetkzobj{all}
\usepackage[makeroom]{cancel} %% for strike outs
%\usepackage{mathtools} %% for pretty underbrace % Breaks Ximera
%\usepackage{multicol}
\usepackage{pgffor} %% required for integral for loops



%% http://tex.stackexchange.com/questions/66490/drawing-a-tikz-arc-specifying-the-center
%% Draws beach ball
\tikzset{pics/carc/.style args={#1:#2:#3}{code={\draw[pic actions] (#1:#3) arc(#1:#2:#3);}}}



\usepackage{array}
\setlength{\extrarowheight}{+.1cm}
\newdimen\digitwidth
\settowidth\digitwidth{9}
\def\divrule#1#2{
\noalign{\moveright#1\digitwidth
\vbox{\hrule width#2\digitwidth}}}
























%%This is to help with formatting on future title pages.
\newenvironment{sectionOutcomes}{}{}


\title{Function Graphs}


\begin{document}

\begin{abstract}

\end{abstract}
\maketitle


Functions are relations, which makes them packages.  Our functions are intended to relate measurements, which means they are real-valued functions. Measurements come with units, which we usually have standing ready on the sideline while we investigate the numeric pairs of the function. However, as we begin to use functions to analyze situations, we must consider the units to make sure the puzzle fits together properly.



As relations, functions contain three sets.  They contain a set of real numbers called the \textbf{domain}.  They contain a second set of real numbers called the \textbf{range}. Finally, functions contain a third set of pairs of numbers.  The pairs are all constructed with a number from each of the domain and range.

The pairs are the essence of the function.  The pairs are the connections between the domain and range numbers or measurements.  They identify which numbers are associated together from both sets. As a function, this set adheres to one rule:


\begin{center}
Each domain number is in \underline{exactly} one pair.
\end{center}



We can communicate about these pairs, one by one, with \textbf{function notation}.

\[
\large{F(d)}
\]


$F$ is the name of the function and $d$ is a domain number.  $F(d)$ represents the range partner of $d$.  $d$ and $F(d)$ are partnered together in $F$.

When we write down such a pairing, we usually write the domain number on the left and the range partner (or function value) on the right, separated by a comma, and wrapped with parentheses.

\[ 
\large{(d, F(d))} 
\]

Independent of this, we might separately have an expression for the range partner, say $r$.  We can communicate this with an equation like


\[
\large{F(d) = r}
\]


\section{See the Whole Picture}

We can certainly analyze a function by talking about individual pairs, however, this is an extremely close-up and zoomed-in perspective. We might also like to talk about the whole collection at once.  Our first attempt at this is via pictures. We need a way to visually represent a single pair and then convert all pairs to a picture.  With this picture, we can analyze the function as a whole, identify important places in the domain, detect trends in the data, and quickly estimate information about the whole function. \\

These pictures or visual tools are called \textbf{graphs}. \\






\subsection{Expectations}

\begin{sectionOutcomes}
In this section, students will 

\begin{itemize}
\item encode function pairs into dots.
\item decipher dots into function pairs.
\item use function notation to communicate about its graph.
\item evaluate functions via its graph.
\item solve equations involving function notation via its graph.
\end{itemize}
\end{sectionOutcomes}

\end{document}
