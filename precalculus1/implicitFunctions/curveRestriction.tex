\documentclass{ximera}


\graphicspath{
  {./}
  {ximeraTutorial/}
  {basicPhilosophy/}
}

\newcommand{\mooculus}{\textsf{\textbf{MOOC}\textnormal{\textsf{ULUS}}}}


\usepackage{tkz-euclide}\usepackage{tikz}
\usepackage{tikz-cd}
\usetikzlibrary{arrows}
\tikzset{>=stealth,commutative diagrams/.cd,
  arrow style=tikz,diagrams={>=stealth}} %% cool arrow head
\tikzset{shorten <>/.style={ shorten >=#1, shorten <=#1 } } %% allows shorter vectors

\usetikzlibrary{backgrounds} %% for boxes around graphs
\usetikzlibrary{shapes,positioning}  %% Clouds and stars
\usetikzlibrary{matrix} %% for matrix
\usepgfplotslibrary{polar} %% for polar plots
\usepgfplotslibrary{fillbetween} %% to shade area between curves in TikZ
\usetkzobj{all}
\usepackage[makeroom]{cancel} %% for strike outs
%\usepackage{mathtools} %% for pretty underbrace % Breaks Ximera
%\usepackage{multicol}
\usepackage{pgffor} %% required for integral for loops



%% http://tex.stackexchange.com/questions/66490/drawing-a-tikz-arc-specifying-the-center
%% Draws beach ball
\tikzset{pics/carc/.style args={#1:#2:#3}{code={\draw[pic actions] (#1:#3) arc(#1:#2:#3);}}}



\usepackage{array}
\setlength{\extrarowheight}{+.1cm}
\newdimen\digitwidth
\settowidth\digitwidth{9}
\def\divrule#1#2{
\noalign{\moveright#1\digitwidth
\vbox{\hrule width#2\digitwidth}}}
























%%This is to help with formatting on future title pages.
\newenvironment{sectionOutcomes}{}{}


\title{Restrictions}

\begin{document}

\begin{abstract}
domain and range
\end{abstract}
\maketitle







Suppose we have two quanitities, $A$ and $B$ in our situation, and it is helpful to think of $A$ as dependent on $B$.  So, we would like to think in terms of $A(B)$ and use our knowledge of functions.



However, these two quatities are related via  $B+5 = 3 (A-4)^2$


\begin{image}
\begin{tikzpicture}
  \begin{axis}[
            domain=-10:10, ymax=10, xmax=10, ymin=-10, xmin=-10, unit vector ratio*=1 1 1,
            axis lines =center, xlabel=$B$, ylabel=$A$, grid = major, grid style={dashed},
            ytick={-10,-8,-6,-4,-2,2,4,6,8,10},
            xtick={-10,-8,-6,-4,-2,2,4,6,8,10},
            yticklabels={$-10$,$-8$,$-6$,$-4$,$-2$,$2$,$4$,$6$,$8$,$10$}, 
            xticklabels={$-10$,$-8$,$-6$,$-4$,$-2$,$2$,$4$,$6$,$8$,$10$},
            ticklabel style={font=\scriptsize},
            every axis y label/.style={at=(current axis.above origin),anchor=south},
            every axis x label/.style={at=(current axis.right of origin),anchor=west},
            axis on top
          ]
          
            
      %\addplot [line width=2, penColor, smooth,samples=200,domain=(-1:9)] {5};

		\addplot [line width=2, penColor, smooth, samples=200, domain=(1.75:6.2),<->] ({3*(x-4)^2 - 5},{x});






  \end{axis}
\end{tikzpicture}
\end{image}



This graph is not passing the vertical line test.  We have many domain numbers paired with multiple range numbers.  $A$ is not a function of $B$.





Pretend, we learn in our situation, that 

\begin{itemize}
\item when $B$ is negative, then $A \leq 4$, and 
\item when $B \geq 0$, then $A > 4$ 
\end{itemize}





\begin{image}
\begin{tikzpicture}
  \begin{axis}[
            domain=-10:10, ymax=10, xmax=10, ymin=-10, xmin=-10, unit vector ratio*=1 1 1,
            axis lines =center, xlabel=$B$, ylabel=$A$, grid = major, grid style={dashed},
            ytick={-10,-8,-6,-4,-2,2,4,6,8,10},
            xtick={-10,-8,-6,-4,-2,2,4,6,8,10},
            yticklabels={$-10$,$-8$,$-6$,$-4$,$-2$,$2$,$4$,$6$,$8$,$10$}, 
            xticklabels={$-10$,$-8$,$-6$,$-4$,$-2$,$2$,$4$,$6$,$8$,$10$},
            ticklabel style={font=\scriptsize},
            every axis y label/.style={at=(current axis.above origin),anchor=south},
            every axis x label/.style={at=(current axis.right of origin),anchor=west},
            axis on top
          ]
          
            
      %\addplot [line width=2, penColor, smooth,samples=200,domain=(-1:9)] {5};

		\addplot [line width=2, penColor, smooth, samples=200, domain=(4:2.7)] ({3*(x-4)^2 - 5},{x});

		\addplot [line width=2, penColor, smooth, samples=200, domain=(5.3:6.2),->] ({3*(x-4)^2 - 5},{x});

		\addplot[color=penColor,fill=penColor,only marks,mark=*] coordinates{(0,5.3)};
		\addplot[color=penColor,fill=penColor,only marks,mark=*] coordinates{(-5,4)};
		\addplot[color=penColor,fill=white,only marks,mark=*] coordinates{(0,2.7)};






  \end{axis}
\end{tikzpicture}
\end{image}



We now have a function, $A(B)$.  The relationship becomes a function upon discarding some of the pairs.  Selecting some of a collection and removing the rest is called \textbf{restricting}.  Restricting is a method of stating the situaiton in terms of functions.






\section{Implicitly Defined Functions}



Given a function, $f$, we can greate a graph by plotting the point $(d, f(d))$ for each $d$ in the domain of $f$. We would like to go the other way.  Given a curve (lots of points), we would like to define a function by interpreting the plotted points as pairs in the function.

This works well when the graph passes the vertical line test. However, for other curves, we will need to restrict our attention on only pieces of the curve.








The graph of $(x+4)^2 + (y-2)^2 = 5^2$ does not define $y$ as a function of $x$.


\begin{image}
\begin{tikzpicture}
  \begin{axis}[
            domain=-10:10, ymax=10, xmax=10, ymin=-10, xmin=-10, unit vector ratio*=1 1 1,
            axis lines =center, xlabel=$x$, ylabel=$y$, grid = major, grid style={dashed},
            ytick={-10,-8,-6,-4,-2,2,4,6,8,10},
            xtick={-10,-8,-6,-4,-2,2,4,6,8,10},
            yticklabels={$-10$,$-8$,$-6$,$-4$,$-2$,$2$,$4$,$6$,$8$,$10$}, 
            xticklabels={$-10$,$-8$,$-6$,$-4$,$-2$,$2$,$4$,$6$,$8$,$10$},
            ticklabel style={font=\scriptsize},
            every axis y label/.style={at=(current axis.above origin),anchor=south},
            every axis x label/.style={at=(current axis.right of origin),anchor=west},
            axis on top
          ]
          
            
      %\addplot [line width=2, penColor, smooth,samples=200,domain=(-1:9)] {5};

		\addplot [line width=2, penColor, smooth, samples=200, domain=(0:6.3)] ({5*cos(deg(x)) - 4},{5*sin(deg(x)) + 2});






  \end{axis}
\end{tikzpicture}
\end{image}




In order to get a function, we'll need to restrict the domain and range.


We could restrict the domain, $x$, to $[-4, 1]$, but that doesn't fix the problem.









\begin{image}
\begin{tikzpicture}
  \begin{axis}[
            domain=-10:10, ymax=10, xmax=10, ymin=-10, xmin=-10, unit vector ratio*=1 1 1,
            axis lines =center, xlabel=$x$, ylabel=$y$, grid = major, grid style={dashed},
            ytick={-10,-8,-6,-4,-2,2,4,6,8,10},
            xtick={-10,-8,-6,-4,-2,2,4,6,8,10},
            yticklabels={$-10$,$-8$,$-6$,$-4$,$-2$,$2$,$4$,$6$,$8$,$10$}, 
            xticklabels={$-10$,$-8$,$-6$,$-4$,$-2$,$2$,$4$,$6$,$8$,$10$},
            ticklabel style={font=\scriptsize},
            every axis y label/.style={at=(current axis.above origin),anchor=south},
            every axis x label/.style={at=(current axis.right of origin),anchor=west},
            axis on top
          ]
          
            
      %\addplot [line width=2, penColor, smooth,samples=200,domain=(-1:9)] {5};

		\addplot [line width=2, penColor, smooth, samples=200, domain=(-1.57:1.57)] ({5*cos(deg(x)) - 4},{5*sin(deg(x)) + 2});






  \end{axis}
\end{tikzpicture}
\end{image}



We need to restrict the range as well.



Define $y(x)$ via $(x+4)^2 + (y-2)^2 = 5^2$ with $x \in [-4, 1]$  and $y \in [-3, 2]$.













\begin{image}
\begin{tikzpicture}
  \begin{axis}[
            domain=-10:10, ymax=10, xmax=10, ymin=-10, xmin=-10, unit vector ratio*=1 1 1,
            axis lines =center, xlabel=$x$, ylabel=$y$, grid = major, grid style={dashed},
            ytick={-10,-8,-6,-4,-2,2,4,6,8,10},
            xtick={-10,-8,-6,-4,-2,2,4,6,8,10},
            yticklabels={$-10$,$-8$,$-6$,$-4$,$-2$,$2$,$4$,$6$,$8$,$10$}, 
            xticklabels={$-10$,$-8$,$-6$,$-4$,$-2$,$2$,$4$,$6$,$8$,$10$},
            ticklabel style={font=\scriptsize},
            every axis y label/.style={at=(current axis.above origin),anchor=south},
            every axis x label/.style={at=(current axis.right of origin),anchor=west},
            axis on top
          ]
          
            
      %\addplot [line width=2, penColor, smooth,samples=200,domain=(-1:9)] {5};

		\addplot [line width=2, penColor, smooth, samples=200, domain=(-1.57:0)] ({5*cos(deg(x)) - 4},{5*sin(deg(x)) + 2});
		\addplot[color=penColor,fill=penColor,only marks,mark=*] coordinates{(-4,-3)};
		\addplot[color=penColor,fill=penColor,only marks,mark=*] coordinates{(1,2)};






  \end{axis}
\end{tikzpicture}
\end{image}


Now, we havea function.


Since we didn't have a formula for $y$, we are just stating the relationship via an equation with appropriate restrictions.  Functions defined this way, without an explicit formula, are said to be defined \textbf{implicitly}.




We could have restricted the domain and range in many ways to form functions.  They would all use the same equation, but be different functions.












\begin{example} Restrictions



The following functions all use $(x+4)^2 + (y-2)^2 = 5^2$ to identify pairs.  




\begin{image}
\begin{tikzpicture}
  \begin{axis}[
            domain=-10:10, ymax=10, xmax=10, ymin=-10, xmin=-10, unit vector ratio*=1 1 1,
            axis lines =center, xlabel=$x$, ylabel=$y$, grid = major, grid style={dashed},
            ytick={-10,-8,-6,-4,-2,2,4,6,8,10},
            xtick={-10,-8,-6,-4,-2,2,4,6,8,10},
            yticklabels={$-10$,$-8$,$-6$,$-4$,$-2$,$2$,$4$,$6$,$8$,$10$}, 
            xticklabels={$-10$,$-8$,$-6$,$-4$,$-2$,$2$,$4$,$6$,$8$,$10$},
            ticklabel style={font=\scriptsize},
            every axis y label/.style={at=(current axis.above origin),anchor=south},
            every axis x label/.style={at=(current axis.right of origin),anchor=west},
            axis on top
          ]
          
            
      %\addplot [line width=2, penColor, smooth,samples=200,domain=(-1:9)] {5};

    \addplot [line width=2, penColor, smooth, samples=200, domain=(0:3.14)] ({5*cos(deg(x)) - 4},{5*sin(deg(x)) + 2});
    \addplot[color=penColor,fill=penColor,only marks,mark=*] coordinates{(-9,2)};
    \addplot[color=penColor,fill=penColor,only marks,mark=*] coordinates{(1,2)};



  \end{axis}
\end{tikzpicture}
\end{image}












\begin{image}
\begin{tikzpicture}
  \begin{axis}[
            domain=-10:10, ymax=10, xmax=10, ymin=-10, xmin=-10, unit vector ratio*=1 1 1,
            axis lines =center, xlabel=$x$, ylabel=$y$, grid = major, grid style={dashed},
            ytick={-10,-8,-6,-4,-2,2,4,6,8,10},
            xtick={-10,-8,-6,-4,-2,2,4,6,8,10},
            yticklabels={$-10$,$-8$,$-6$,$-4$,$-2$,$2$,$4$,$6$,$8$,$10$}, 
            xticklabels={$-10$,$-8$,$-6$,$-4$,$-2$,$2$,$4$,$6$,$8$,$10$},
            ticklabel style={font=\scriptsize},
            every axis y label/.style={at=(current axis.above origin),anchor=south},
            every axis x label/.style={at=(current axis.right of origin),anchor=west},
            axis on top
          ]
          
            
      %\addplot [line width=2, penColor, smooth,samples=200,domain=(-1:9)] {5};

    \addplot [line width=2, penColor, smooth, samples=200, domain=(3.14:6.28)] ({5*cos(deg(x)) - 4},{5*sin(deg(x)) + 2});
    \addplot[color=penColor,fill=penColor,only marks,mark=*] coordinates{(-9,2)};
    \addplot[color=penColor,fill=penColor,only marks,mark=*] coordinates{(1,2)};


  \end{axis}
\end{tikzpicture}
\end{image}

















\begin{image}
\begin{tikzpicture}
  \begin{axis}[
            domain=-10:10, ymax=10, xmax=10, ymin=-10, xmin=-10, unit vector ratio*=1 1 1,
            axis lines =center, xlabel=$x$, ylabel=$y$, grid = major, grid style={dashed},
            ytick={-10,-8,-6,-4,-2,2,4,6,8,10},
            xtick={-10,-8,-6,-4,-2,2,4,6,8,10},
            yticklabels={$-10$,$-8$,$-6$,$-4$,$-2$,$2$,$4$,$6$,$8$,$10$}, 
            xticklabels={$-10$,$-8$,$-6$,$-4$,$-2$,$2$,$4$,$6$,$8$,$10$},
            ticklabel style={font=\scriptsize},
            every axis y label/.style={at=(current axis.above origin),anchor=south},
            every axis x label/.style={at=(current axis.right of origin),anchor=west},
            axis on top
          ]
          
            
      %\addplot [line width=2, penColor, smooth,samples=200,domain=(-1:9)] {5};

    \addplot [line width=2, penColor, smooth, samples=200, domain=(0:1.57)] ({5*cos(deg(x)) - 4},{5*sin(deg(x)) + 2});
     \addplot [line width=2, penColor, smooth, samples=200, domain=(3.14:4.71)] ({5*cos(deg(x)) - 4},{5*sin(deg(x)) + 2});

    \addplot[color=penColor,fill=penColor,only marks,mark=*] coordinates{(1,2)};
    \addplot[color=penColor,fill=penColor,only marks,mark=*] coordinates{(-4,7)};
     \addplot[color=penColor,fill=penColor,only marks,mark=*] coordinates{(-9,2)};
    \addplot[color=penColor,fill=white,only marks,mark=*] coordinates{(-4,-3)};


  \end{axis}
\end{tikzpicture}
\end{image}



\end{example}











\end{document}
