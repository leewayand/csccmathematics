\documentclass{ximera}


\graphicspath{
  {./}
  {ximeraTutorial/}
  {basicPhilosophy/}
}

\newcommand{\mooculus}{\textsf{\textbf{MOOC}\textnormal{\textsf{ULUS}}}}


\usepackage{tkz-euclide}\usepackage{tikz}
\usepackage{tikz-cd}
\usetikzlibrary{arrows}
\tikzset{>=stealth,commutative diagrams/.cd,
  arrow style=tikz,diagrams={>=stealth}} %% cool arrow head
\tikzset{shorten <>/.style={ shorten >=#1, shorten <=#1 } } %% allows shorter vectors

\usetikzlibrary{backgrounds} %% for boxes around graphs
\usetikzlibrary{shapes,positioning}  %% Clouds and stars
\usetikzlibrary{matrix} %% for matrix
\usepgfplotslibrary{polar} %% for polar plots
\usepgfplotslibrary{fillbetween} %% to shade area between curves in TikZ
\usetkzobj{all}
\usepackage[makeroom]{cancel} %% for strike outs
%\usepackage{mathtools} %% for pretty underbrace % Breaks Ximera
%\usepackage{multicol}
\usepackage{pgffor} %% required for integral for loops



%% http://tex.stackexchange.com/questions/66490/drawing-a-tikz-arc-specifying-the-center
%% Draws beach ball
\tikzset{pics/carc/.style args={#1:#2:#3}{code={\draw[pic actions] (#1:#3) arc(#1:#2:#3);}}}



\usepackage{array}
\setlength{\extrarowheight}{+.1cm}
\newdimen\digitwidth
\settowidth\digitwidth{9}
\def\divrule#1#2{
\noalign{\moveright#1\digitwidth
\vbox{\hrule width#2\digitwidth}}}
























%%This is to help with formatting on future title pages.
\newenvironment{sectionOutcomes}{}{}


\title{Curves}

\begin{document}

\begin{abstract}
curves from equations
\end{abstract}
\maketitle





We are investigating functions that pair a real number from the domain with a real number from the range: $(d, f(d)$.  These pairs are then interpreted visually and represented as points plotted in the Cartesian plane.

When $f$ is a continuous function, then these graphs are called \textbf{curves}.



However, curves include many more graphical features and characteristics than the graphs of functions.  We will slowly extend our ideas of functions and curves even through Calculus and Vector Calculus.

Curves are again a collection of points in the Cartesian plane but they can come from any equation, rather than a formula - a special type of equation, where the dependent variable has been isolated on one side of the equal sign.


Most equations cannot be put into an equivalent formula form.



For example. all of these are equivalent equations.  The first one has $y$ isolated.  The last one has $x$ isolated.


\begin{itemize}
\item $y = 4 x + 7$
\item $y - 4x = 7$
\item $0 = 4x - y + 7$
\item $\frac{1}{4}y = x + \frac{7}{4}$
\item $\frac{1}{4}y - \frac{7}{4}= x$
\end{itemize}


These are equivalent, because they all of the same solution set.  That is the same pairs of numbers make the equations true.  If we graph these solution pairs as points, each equation would be describing the same curve - here, a line. \\



$\blacktriangleright$ What about equations where you cannot isolate the dependent variable? \\


\begin{example} A Circle


$x^2 + y^2 = 1$ defines a curve we have seen before - the unit circle.






\begin{image}
\begin{tikzpicture}
  \begin{axis}[
            domain=-1.5:1.5, ymax=1.5, xmax=1.5, ymin=-1.5, xmin=-1.5, unit vector ratio*=1 1 1,
            axis lines =center, xlabel=$x$, ylabel=$y$, grid = major, grid style={dashed},
            ytick={-1, -0.5, 0.5, 1},
            xtick={-1, -0.5, 0.5, 1},
            yticklabels={$-1$,$-0.5$,$0.5$,$1$}, 
            xticklabels={$-1$,$-0.5$,$0.5$,$1$},
            ticklabel style={font=\scriptsize},
            every axis y label/.style={at=(current axis.above origin),anchor=south},
            every axis x label/.style={at=(current axis.right of origin),anchor=west},
            axis on top
          ]
          
          \addplot [line width=2, penColor, smooth,samples=200,domain=(0:6.3)] ({cos(deg(x))},{sin(deg(x))});
           %\addplot [line width=2, penColor2, smooth,samples=100,domain=(2:8)] {1.75*x-8};

          %\addplot[color=penColor,fill=penColor2,only marks,mark=*] coordinates{(-6,9)};
          %\addplot[color=penColor,fill=penColor2,only marks,mark=*] coordinates{(2,-7)};

          %\addplot[color=penColor2,fill=white,only marks,mark=*] coordinates{(2,-4.5)};
          %\addplot[color=penColor2,fill=white,only marks,mark=*] coordinates{(8,6)};


           

  \end{axis}
\end{tikzpicture}
\end{image}



This curve is a collection of a lot of points - a lot!  Each point has two coordinates - one for $x$ and one for $y$.  These pairs of numbers staisfy the equation - they make the equation true.



For example, $\left( \frac{\sqrt{3}}{2}, \frac{1}{2} \right)$ is a point on the curve, because 

\[    \left( \frac{\sqrt{3}}{2} \right)^2 + \left( \frac{1}{2} \right)^2 = 1    \]





\end{example}




The unit circle is an example of an \textbf{algebraic curve}, meaning it is described from a polynomial.





\section{Algebraic Curves}



\begin{definition}  Algebraic Curve


Let $p(x,y)$ be a polynomial in two variables.  An \textbf{algebraic curve} is the set of solutions to $p(x,y)=0$ 


\end{definition}



For the unit circle, the polynomial is $x^2 + y^2 - 1$.  The unit circle is the set of points representing zeros for this polynomial.






















\begin{example} A Quadratic


$y = x^2$ defines a curve we have seen before - a parabola.











\begin{image}
\begin{tikzpicture}
  \begin{axis}[
            domain=-10:10, ymax=10, xmax=10, ymin=-10, xmin=-10,
            axis lines =center, xlabel=$x$, ylabel=$y$, grid = major, grid style={dashed},
            ytick={-10,-8,-6,-4,-2,2,4,6,8,10},
            xtick={-10,-8,-6,-4,-2,2,4,6,8,10},
            yticklabels={$-10$,$-8$,$-6$,$-4$,$-2$,$2$,$4$,$6$,$8$,$10$}, 
            xticklabels={$-10$,$-8$,$-6$,$-4$,$-2$,$2$,$4$,$6$,$8$,$10$},
            ticklabel style={font=\scriptsize},
            every axis y label/.style={at=(current axis.above origin),anchor=south},
            every axis x label/.style={at=(current axis.right of origin),anchor=west},
            axis on top
          ]
          
          \addplot [line width=2, penColor, smooth,samples=200,domain=(-3:3),<->] {x^2};
           %\addplot [line width=2, penColor2, smooth,samples=100,domain=(2:8)] {1.75*x-8};

          %\addplot[color=penColor,fill=penColor2,only marks,mark=*] coordinates{(-6,9)};
          %\addplot[color=penColor,fill=penColor2,only marks,mark=*] coordinates{(2,-7)};

          %\addplot[color=penColor2,fill=white,only marks,mark=*] coordinates{(2,-4.5)};
          %\addplot[color=penColor2,fill=white,only marks,mark=*] coordinates{(8,6)};


           

  \end{axis}
\end{tikzpicture}
\end{image}




\end{example}



With algebraic curves, we can get other types of parabolas.














\begin{example} A Parabola


$x = y^2$ defines a curve we have seen before - a parabola.











\begin{image}
\begin{tikzpicture}
  \begin{axis}[
            domain=-10:10, ymax=10, xmax=10, ymin=-10, xmin=-10,
            axis lines =center, xlabel=$x$, ylabel=$y$, grid = major, grid style={dashed},
            ytick={-10,-8,-6,-4,-2,2,4,6,8,10},
            xtick={-10,-8,-6,-4,-2,2,4,6,8,10},
            yticklabels={$-10$,$-8$,$-6$,$-4$,$-2$,$2$,$4$,$6$,$8$,$10$}, 
            xticklabels={$-10$,$-8$,$-6$,$-4$,$-2$,$2$,$4$,$6$,$8$,$10$},
            ticklabel style={font=\scriptsize},
            every axis y label/.style={at=(current axis.above origin),anchor=south},
            every axis x label/.style={at=(current axis.right of origin),anchor=west},
            axis on top
          ]
          
          \addplot [line width=2, penColor, smooth,samples=200,domain=(-3:3),<->] ({x^2},{x});
           %\addplot [line width=2, penColor2, smooth,samples=100,domain=(2:8)] {1.75*x-8};

          %\addplot[color=penColor,fill=penColor2,only marks,mark=*] coordinates{(-6,9)};
          %\addplot[color=penColor,fill=penColor2,only marks,mark=*] coordinates{(2,-7)};

          %\addplot[color=penColor2,fill=white,only marks,mark=*] coordinates{(2,-4.5)};
          %\addplot[color=penColor2,fill=white,only marks,mark=*] coordinates{(8,6)};


           

  \end{axis}
\end{tikzpicture}
\end{image}




This is not the graph of $y$ as a function of $x$.  It is the graph of $x$ as a function of $y$.



\end{example}








\begin{example} A Rotated Parabola


$x^2 + 2 x y + y^2 + \sqrt{2} x - \sqrt{2} y = 0$







\begin{center}
\desmos{zthkvempnv}{400}{300}
\end{center}











This graph describes neither $x$ nor $y$ as a function of the other.



\end{example}













Many of these curves actually belong to the same family of curves.  They are described with an equaiton template that contains a \textbf{parameter}. Each value of the parameter gives a new curve of the family.


\begin{example} A Cardioid Family


$(x^2 + y^2 - 2 a x)^2 = 4 a^2 (x^2 +y^2)$

$a$ is the parameter





\begin{center}
\desmos{9nmltqsgco}{400}{300}
\end{center}











This graph describes neither $x$ nor $y$ as a function of the other.



\end{example}












\end{document}
