\documentclass{ximera}


\graphicspath{
  {./}
  {ximeraTutorial/}
  {basicPhilosophy/}
}

\newcommand{\mooculus}{\textsf{\textbf{MOOC}\textnormal{\textsf{ULUS}}}}


\usepackage{tkz-euclide}\usepackage{tikz}
\usepackage{tikz-cd}
\usetikzlibrary{arrows}
\tikzset{>=stealth,commutative diagrams/.cd,
  arrow style=tikz,diagrams={>=stealth}} %% cool arrow head
\tikzset{shorten <>/.style={ shorten >=#1, shorten <=#1 } } %% allows shorter vectors

\usetikzlibrary{backgrounds} %% for boxes around graphs
\usetikzlibrary{shapes,positioning}  %% Clouds and stars
\usetikzlibrary{matrix} %% for matrix
\usepgfplotslibrary{polar} %% for polar plots
\usepgfplotslibrary{fillbetween} %% to shade area between curves in TikZ
\usetkzobj{all}
\usepackage[makeroom]{cancel} %% for strike outs
%\usepackage{mathtools} %% for pretty underbrace % Breaks Ximera
%\usepackage{multicol}
\usepackage{pgffor} %% required for integral for loops



%% http://tex.stackexchange.com/questions/66490/drawing-a-tikz-arc-specifying-the-center
%% Draws beach ball
\tikzset{pics/carc/.style args={#1:#2:#3}{code={\draw[pic actions] (#1:#3) arc(#1:#2:#3);}}}



\usepackage{array}
\setlength{\extrarowheight}{+.1cm}
\newdimen\digitwidth
\settowidth\digitwidth{9}
\def\divrule#1#2{
\noalign{\moveright#1\digitwidth
\vbox{\hrule width#2\digitwidth}}}
























%%This is to help with formatting on future title pages.
\newenvironment{sectionOutcomes}{}{}


\title{Parabolas}

\begin{document}

\begin{abstract}
quadratic graphs
\end{abstract}
\maketitle





We have three forms of quadratic functions:



$\blacktriangleright$ $S(x) = A \, x^2 + B \, x + C$  with $A \ne 0$ : Standard Form. \\


$\blacktriangleright$ $F(t) = A \, (t - r_1)(t - r_2)$  with $A \ne 0$ : Factored Form. \\


$\blacktriangleright$ $V(d) = A \, (d - h)^2 + k$  with $A \ne 0$ : Vertex Form. \\
Vertex form comes from completing the square.  It gets its name from the graph of quadratic functions.









\subsection{Graphs of Quadratic Functions}



Quadratic functions all have parabolas for graphs.




\begin{image}
\begin{tikzpicture}
     \begin{axis}[
                domain=-10:10, ymax=10, xmax=10, ymin=-10, xmin=-10,
                axis lines =center, xlabel=$x$, ylabel=$y$,
                ytick={-10,-8,-6,-4,-2,2,4,6,8,10},
            	xtick={-10,-8,-6,-4,-2,2,4,6,8,10},
            	ticklabel style={font=\scriptsize},
                every axis y label/.style={at=(current axis.above origin),anchor=south},
                every axis x label/.style={at=(current axis.right of origin),anchor=west},
                axis on top,
                ]



        \addplot [draw=penColor, very thick, smooth, domain=(-7:9),<->] {-0.25*(x+4)*(x-6)};
        %\addplot [line width=1, gray, dashed,samples=100,domain=(-9.5:9.5)] ({3},{x});
        


        \addplot [color=penColor,only marks,mark=*] coordinates{(1,6.25)};
        \node[penColor] at (axis cs:2,8) {vertex};
        %\node[penColor] at (axis cs:4,1.5) {$(h, k)$};
        %\node[penColor] at (axis cs:5,-9) {$-0.5 x^2 - 5 x + 15.5$};



    \end{axis}
\end{tikzpicture}
\end{image}











\begin{image}
\begin{tikzpicture}
     \begin{axis}[
                domain=-10:10, ymax=10, xmax=10, ymin=-10, xmin=-10,
                axis lines =center, xlabel=$x$, ylabel=$y$,
                ytick={-10,-8,-6,-4,-2,2,4,6,8,10},
            	xtick={-10,-8,-6,-4,-2,2,4,6,8,10},
            	ticklabel style={font=\scriptsize},
                every axis y label/.style={at=(current axis.above origin),anchor=south},
                every axis x label/.style={at=(current axis.right of origin),anchor=west},
                axis on top,
                ]



        \addplot [draw=penColor, very thick, smooth, domain=(-9:7),<->] {0.25*(x+6)*(x-4)};
        %\addplot [line width=1, gray, dashed,samples=100,domain=(-9.5:9.5)] ({3},{x});
        


        \addplot [color=penColor,only marks,mark=*] coordinates{(-1,-6.25)};
        \node[penColor] at (axis cs:-3,-7) {vertex};
        %\node[penColor] at (axis cs:4,1.5) {$(h, k)$};
        %\node[penColor] at (axis cs:5,-9) {$-0.5 x^2 - 5 x + 15.5$};



    \end{axis}
\end{tikzpicture}
\end{image}










The extreme point on a parabola is called the \textbf{vertex}.  It is the highest or lowest point on the parabola, depending on whether the parabola opens up or down. 

This can be seen from the vertex form of the formula.




\[
V(d) = A \, (d - h)^2 + k
\]

The squared term, $A \, (d - h)^2$ has the same sign as $A$, except when it equals $0$.  That happens at $h$.  When $d = h$, then $V(h) = k$, which is either the minimum or maximum value of $V$.  The vertex is the graphical representation of the the extrem value of the quadratic funciton and where this maximum occurs in the domain.


In addition, the intercepts represent the zeros of the quadratic function and we have seen there can be $0$, $1$, or $2$ real zeros for a quadratice function.  Therefore, there can be $0$, $1$, or $2$ intercepts for a parabola.
















\end{document}
