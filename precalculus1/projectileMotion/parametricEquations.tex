\documentclass{ximera}


\graphicspath{
  {./}
  {ximeraTutorial/}
  {basicPhilosophy/}
}

\newcommand{\mooculus}{\textsf{\textbf{MOOC}\textnormal{\textsf{ULUS}}}}


\usepackage{tkz-euclide}\usepackage{tikz}
\usepackage{tikz-cd}
\usetikzlibrary{arrows}
\tikzset{>=stealth,commutative diagrams/.cd,
  arrow style=tikz,diagrams={>=stealth}} %% cool arrow head
\tikzset{shorten <>/.style={ shorten >=#1, shorten <=#1 } } %% allows shorter vectors

\usetikzlibrary{backgrounds} %% for boxes around graphs
\usetikzlibrary{shapes,positioning}  %% Clouds and stars
\usetikzlibrary{matrix} %% for matrix
\usepgfplotslibrary{polar} %% for polar plots
\usepgfplotslibrary{fillbetween} %% to shade area between curves in TikZ
\usetkzobj{all}
\usepackage[makeroom]{cancel} %% for strike outs
%\usepackage{mathtools} %% for pretty underbrace % Breaks Ximera
%\usepackage{multicol}
\usepackage{pgffor} %% required for integral for loops



%% http://tex.stackexchange.com/questions/66490/drawing-a-tikz-arc-specifying-the-center
%% Draws beach ball
\tikzset{pics/carc/.style args={#1:#2:#3}{code={\draw[pic actions] (#1:#3) arc(#1:#2:#3);}}}



\usepackage{array}
\setlength{\extrarowheight}{+.1cm}
\newdimen\digitwidth
\settowidth\digitwidth{9}
\def\divrule#1#2{
\noalign{\moveright#1\digitwidth
\vbox{\hrule width#2\digitwidth}}}
























%%This is to help with formatting on future title pages.
\newenvironment{sectionOutcomes}{}{}


\title{Parametric Equations}

\begin{document}

\begin{abstract}
separating dimensions
\end{abstract}
\maketitle



During our investigation of projectile motion we encountered two different decriptions of a projectile's path.

We found that the projectile followed a parabolic trajectory described as 

\[  y = \frac{v_{y0}}{v_{x0}} \, x  - \frac{g}{(v_{x0})^2} \, x^2 \]

This is a quadratic equation showing the relationship between vertical height, $y$, and horizontal distance, $x$.  

However, leading up to this equation we viewed height and distance separately. They were described individually as functions of time, $t$.


\begin{itemize}
\item $y(t) = v_{y0} t - g \, t^2$


\item $x(t) = v_{x0} t$
\end{itemize}



This idea of separating the vertical and horizontal dimensions of a curve and relating them to a third parameter is known as \textbf{parameterization}.  The equaitons are called \textbf{parametric equations}.  The new third variable is called the \textbf{parameter}.



Rather than beginning an investigation of parameterization with parabolas, let's begin thinking in terms of linear functions.





\section{Parameterizing Lines}



Lines are curves (collections of points) whose coordinates satisfy equaitons like $y = m \, x + b$.  $x$ acts like a sliding lever.  You slide the $x$-knob back and forth and a point moves along the line.  The $y$-coordinate automatically adjusts according to the equation.


Almost.

It would be nice to get our hands on the $x$-knob.  Currently, the knob is also in automatic mode.  It moves at a constant speed and the corresponding point also moves at a constant speed.

We would like to grab ahold of the knob and move it differently.  Faster and slower.  Forwards and Backwards. Around in circles.  A parameter allows us to move a point on the line.

The way we accomplish this is to introduce a third variable called a parameter and have both $x$ and $y$ become functions depending on this third parameter. If we call this third variable $t$, then we get a set of parametric equations:



\begin{itemize}
\item $x = x(t)$
\item $y = y(t)$
\end{itemize}


Consider the line described by the equation $y = 2x - 5$, graphed below.






\begin{image}
\begin{tikzpicture}
     \begin{axis}[
            	domain=-10:10, ymax=10, xmax=10, ymin=-10, xmin=-10,
            	axis lines =center, xlabel=$x$, ylabel=$y$,
            	every axis y label/.style={at=(current axis.above origin),anchor=south},
            	every axis x label/.style={at=(current axis.right of origin),anchor=west},
            	axis on top,
          		]


        
        \addplot [draw=penColor, very thick, smooth, domain=(-2.45:7.45),<->] {2*x-5};
   


    \end{axis}
\end{tikzpicture}
\end{image}





look at the axes.






\begin{image}
\begin{tikzpicture}
     \begin{axis}[
            	domain=-10:10, ymax=10, xmax=10, ymin=-10, xmin=-10,
            	axis lines =center, xlabel=$t$, ylabel=$yx$,
            	every axis y label/.style={at=(current axis.above origin),anchor=south},
            	every axis x label/.style={at=(current axis.right of origin),anchor=west},
            	axis on top,
          		]


        
        \addplot [draw=penColor, very thick, smooth, domain=(-3.5:2.5),<->] {x*(x+3)*(x-2)};
        







    \end{axis}
\end{tikzpicture}
\end{image}


We will have $x = t(t+3)(t-2)$ and $y = 2t(t+3)(t-2) - 5$.   This way $x$ and $y$ are still related as $y = 2x - 5$. The curve is the same, a line.  However, the way it is traced has changed.

As $t$ moves steadily from $-\infty$ to $\infty$,  $x(t)$ follows the path above.

\begin{itemize}
\item $x(t)$ begins down near $-\infty$
\item increases to a value of $8.2$, 
\item then decreases to a value of $-4$, 
\item then increases again to $\infty$

\end{itemize}




\begin{center}
\desmos{jxcwaasqna}{400}{300}
\end{center}










(cos,sin)





\end{document}
