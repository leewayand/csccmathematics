\documentclass{ximera}


\graphicspath{
  {./}
  {ximeraTutorial/}
  {basicPhilosophy/}
}

\newcommand{\mooculus}{\textsf{\textbf{MOOC}\textnormal{\textsf{ULUS}}}}


\usepackage{tkz-euclide}\usepackage{tikz}
\usepackage{tikz-cd}
\usetikzlibrary{arrows}
\tikzset{>=stealth,commutative diagrams/.cd,
  arrow style=tikz,diagrams={>=stealth}} %% cool arrow head
\tikzset{shorten <>/.style={ shorten >=#1, shorten <=#1 } } %% allows shorter vectors

\usetikzlibrary{backgrounds} %% for boxes around graphs
\usetikzlibrary{shapes,positioning}  %% Clouds and stars
\usetikzlibrary{matrix} %% for matrix
\usepgfplotslibrary{polar} %% for polar plots
\usepgfplotslibrary{fillbetween} %% to shade area between curves in TikZ
\usetkzobj{all}
\usepackage[makeroom]{cancel} %% for strike outs
%\usepackage{mathtools} %% for pretty underbrace % Breaks Ximera
%\usepackage{multicol}
\usepackage{pgffor} %% required for integral for loops



%% http://tex.stackexchange.com/questions/66490/drawing-a-tikz-arc-specifying-the-center
%% Draws beach ball
\tikzset{pics/carc/.style args={#1:#2:#3}{code={\draw[pic actions] (#1:#3) arc(#1:#2:#3);}}}



\usepackage{array}
\setlength{\extrarowheight}{+.1cm}
\newdimen\digitwidth
\settowidth\digitwidth{9}
\def\divrule#1#2{
\noalign{\moveright#1\digitwidth
\vbox{\hrule width#2\digitwidth}}}
























%%This is to help with formatting on future title pages.
\newenvironment{sectionOutcomes}{}{}



\author{Lee Wayand}

\begin{document}


\begin{exercise} 



A rock is launched straight up into the air from the edge of a $100 foot$ cliff with an initial vertical velocity of $15 \frac{ft}{sec}$.  The rock rises to its peak and then falls past the cliff to the river at the bottom of the cliff.



\begin{question} How far above the cliff did the rock climb? \\




The initial positive is $100 ft$. \\
The initial vertical velocity is $15 \frac{ft}{sec}$ \\


That is gives the following function for vertical position above the river.

\[
h(t)= -\frac{9.81}{2} t^2 + 15 t + 100
\]


This is a quadratic function.  The formula has a negative leading coefficient, so we know there is a maximum value occuring at $t = \frac{-15}{2 \cdot (-\frac{9.81}{2})} \approx 1.529 sec$.



\[
h(1.529) \approx -\frac{9.81}{2} \cdot (1.529)^2 + 15 \cdot 1.529 + 100 \approx 111.468 ft
\]

\end{question}







\begin{question} When does the rock hit the water? \\



We want the time when the height is $0 ft$.



\[
h(t)= -\frac{9.81}{2} t^2 + 15 t + 100 = 0
\]


\[
-\frac{9.81}{2} t^2 + 15 t + 100 = 0
\]



This is a quadratic equation with a negative leading coefficient and positive constant term, which means there are two real solutions.



\[
t = \frac{-15 \pm \sqrt{15^2 - 4 \cdot (-4.905) \cdot 100}}{-9.81} \approx \frac{-15 \pm \sqrt{2187}}{-9.81} 
\]



\[
t \approx 6.296164302  \,  \text{ or } \, t \approx -3.238060327
\]



Since out stopwatch is moving forward from $0$, the rock hits the watter at approximately $6.296 sec$
\end{question}









\end{exercise}


\end{document}