\documentclass{ximera}


\graphicspath{
  {./}
  {ximeraTutorial/}
  {basicPhilosophy/}
}

\newcommand{\mooculus}{\textsf{\textbf{MOOC}\textnormal{\textsf{ULUS}}}}


\usepackage{tkz-euclide}\usepackage{tikz}
\usepackage{tikz-cd}
\usetikzlibrary{arrows}
\tikzset{>=stealth,commutative diagrams/.cd,
  arrow style=tikz,diagrams={>=stealth}} %% cool arrow head
\tikzset{shorten <>/.style={ shorten >=#1, shorten <=#1 } } %% allows shorter vectors

\usetikzlibrary{backgrounds} %% for boxes around graphs
\usetikzlibrary{shapes,positioning}  %% Clouds and stars
\usetikzlibrary{matrix} %% for matrix
\usepgfplotslibrary{polar} %% for polar plots
\usepgfplotslibrary{fillbetween} %% to shade area between curves in TikZ
\usetkzobj{all}
\usepackage[makeroom]{cancel} %% for strike outs
%\usepackage{mathtools} %% for pretty underbrace % Breaks Ximera
%\usepackage{multicol}
\usepackage{pgffor} %% required for integral for loops



%% http://tex.stackexchange.com/questions/66490/drawing-a-tikz-arc-specifying-the-center
%% Draws beach ball
\tikzset{pics/carc/.style args={#1:#2:#3}{code={\draw[pic actions] (#1:#3) arc(#1:#2:#3);}}}



\usepackage{array}
\setlength{\extrarowheight}{+.1cm}
\newdimen\digitwidth
\settowidth\digitwidth{9}
\def\divrule#1#2{
\noalign{\moveright#1\digitwidth
\vbox{\hrule width#2\digitwidth}}}
























%%This is to help with formatting on future title pages.
\newenvironment{sectionOutcomes}{}{}


\title{Gravity}

\begin{document}

\begin{abstract}
inverse square law
\end{abstract}
\maketitle




This section is just to set the scene.  Below is some information to make the rest of the section believable. 




\section{Newton}

Newton's law of gravitation is an inverse square law

\[ F_g = \frac{G m_1 m_2}{r^2}    \]




where $G = 6.67 \times 10^{-11} \frac{m^3}{kg \cdot s^2}$ is the gravitational constant. \\


$F_g$ is the gravitational force between point-masses $m_1$ and $m_2$, which are a distance $r$ apart. \\



This tells us that the gravity we feel from the Earth is given by


\[ g = \frac{F_g}{m_1}  = \frac{G m_e}{(r_e)^2}    \]


\[ g = \frac{(6.67 \times 10^{-11} \frac{m^3}{kg \cdot s^2}) \cdot (5.972 \times 10^{24} kg)}{(6356 km)^2} = 9.81 \frac{m}{s^2}   \]


This is the acceleration we feel on Earth.  Compared to the Earth's radius we don't add very much height (or weight), so we can consider $r$ to be constant for most of our investigations.  That means, for most of our investigations, the acceleration due to the Earth can be considered a constant.




\section{Calculus}

This requires a bit of trust.  \\

In this course we have worked with the average rate-of-change over an interval.  Calculus expands upon this and develops the idea of the instantaneous rate-of-change of a function.This better known as the derivative.  

The derivative of distance with respect to time is velocity or speed. \\

The derivative of velocity with respect to time mis acceleration and above we found out that the Earth's gravitation acceleration is a constant (for the most part).


Working Calculus backwards we get


\[ a(t) = 9.81 \frac{m}{s^2}  \]


\[ v(t) = v_0 - 9.81 t  \]


\[ h(t) = s_0 + v_0 t - \frac{9.81}{2} t^2  \]



$v_0$ is the initial velocity for a projectile thrown into the air. \\

$h_0$ is the initial height for a projectile thrown into the air. \\

$h(t)$ is height as a function of time.  Seconds is the domain measurement and meters is the range measurement.




\section{Projectile Motion}


Actually, all of those quantities are vectors.  

\begin{example}
\item There is a vertical position, a vertical velocity, and a vertical acceleration
\item There is a horizontal position, a horizontal velocity, and a horizontal acceleration
\end{example}



Both the vertical, $y$, and horizontal, $x$, components of position are functions of time.

We have already seen this equation for the height of a projectile.
\[ h_y(t) = s_{y0} + v_{y0} t - \frac{9.81}{2} t^2  \]

Horizontal motion isn't effected by the Earth's gravity, since gravity is a downward acceleration. Therefore, the horizontal distance is just affected by the initial horizontal velocity.


\[ d_x(t) = s_{x0} + v_{x0} t  \]








\section{Parabolas}

Parabolas are geometric curves.  They are defined as the collection of points which are equidistant from a point, called the textbf{focus}, and a line, called teh \textbf{directrix}.  In the diagram below,  we have rotated out viewpoint so that the directrix is hte horizontal line $y = c$ and the focus is $(0,c)$.









\begin{image}
\begin{tikzpicture}
     \begin{axis}[
            	domain=-10:10, ymax=10, xmax=10, ymin=-10, xmin=-10,
            	axis lines =center, xlabel=$x$, ylabel=$y$,
            	every axis y label/.style={at=(current axis.above origin),anchor=south},
            	every axis x label/.style={at=(current axis.right of origin),anchor=west},
            	axis on top,
          		]


	
        \addplot[color=penColor2,fill=penColor2,only marks,mark=*] coordinates{(0,3)};
        
        \addplot [draw=penColor, very thick, smooth, domain=(-8:8),<->] {0.125*x^2};
        \addplot [draw=gray, very thick, dashed, domain=(-8:8),<->] {-3};

        \addplot[color=penColor2,fill=penColor2,only marks,mark=*] coordinates{(7,6.125)};
        \addplot[color=penColor2,fill=penColor2,only marks,mark=*] coordinates{(7,-3)};


        




		\node[penColor] at (axis cs:8.5,6.125) {$(x,y)$};
		\node[penColor] at (axis cs:7,-4) {$(x,-c)$};
		\node[penColor] at (axis cs:-2,3) {$(0,c)$};


        \addplot [ultra thick,penColor2] plot coordinates {(7,6.125) (7,-3)};
        \addplot [ultra thick,penColor2] plot coordinates {(7,6.125) (0,3)};
    





    \end{axis}
\end{tikzpicture}
\end{image}



If we select a random point, $(x,y)$, on the parabola then the distance from this point to the focus has to equal the distance to the directrix.


\[  \sqrt{x^2 + (y-c)^2} = y+c   \]


Solving this for $y$ gives


\[  x^2 + (y-c)^2 = (y+c)^2   \]

\[  x^2 + y^2 - 2 y c + c^2 = y^2 + 2 y c + c^2   \]

\[  x^2  - 2 y c  =  2 y c    \]

\[  x^2   =  4 y c    \]


The coordinates of the point on the parabola satisfy a quadratic equation.  Parabolas are the graphs of quadratic equations.












\section{Projectile Trajectory}


We have descriptions of the vertical height and the horizontal distance for a projectile under the influence of gravity.  Let's put those together and get height as a function of distance.  That will coorespond to our normal experience standing on the ground watching a projectile fly.


\begin{itemize}
\item $h_y(t) = s_{y0} + v_{y0} t - \frac{9.81}{2} t^2$


\item $d_x(t) = s_{x0} + v_{x0} t$
\end{itemize}




First, let's go back to our separate equations and let's just assume we are firing the projectile off the ground.  Then our initial position will be $0$.  

And, let's use the more common symbol $g$, for the acceleration due to gravity, $g = \frac{9.81}{2}$.




\begin{itemize}
\item $y = v_{y0} t - g \, t^2$


\item $x = v_{x0} t$
\end{itemize}


We'll solve for $t$ in the horizonal equation.


\[ t = \frac{x}{v_{x0}} \]

We'll substitute this into the vertical equation.


\[  y = v_{y0} \, \frac{x}{v_{x0}} - g \left(\frac{x}{v_{x0}}\right)^2  \]



\[  y = \frac{v_{y0}}{v_{x0}} \, x  - \frac{g}{(v_{x0})^2} \, x^2 \]



A quadratic! 

The projectile follows a parabola tractory.











\end{document}
