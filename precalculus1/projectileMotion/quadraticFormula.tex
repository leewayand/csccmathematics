\documentclass{ximera}

%\usepackage{todonotes}

\newcommand{\todo}{}

\usepackage{esint} % for \oiint
\ifxake%%https://math.meta.stackexchange.com/questions/9973/how-do-you-render-a-closed-surface-double-integral
\renewcommand{\oiint}{{\large\bigcirc}\kern-1.56em\iint}
\fi


\graphicspath{
  {./}
  {ximeraTutorial/}
  {basicPhilosophy/}
  {functionsOfSeveralVariables/}
  {normalVectors/}
  {lagrangeMultipliers/}
  {vectorFields/}
  {greensTheorem/}
  {shapeOfThingsToCome/}
  {dotProducts/}
  {partialDerivativesAndTheGradientVector/}
  {../productAndQuotientRules/exercises/}
  {../normalVectors/exercisesParametricPlots/}
  {../continuityOfFunctionsOfSeveralVariables/exercises/}
  {../partialDerivativesAndTheGradientVector/exercises/}
  {../directionalDerivativeAndChainRule/exercises/}
  {../commonCoordinates/exercisesCylindricalCoordinates/}
  {../commonCoordinates/exercisesSphericalCoordinates/}
  {../greensTheorem/exercisesCurlAndLineIntegrals/}
  {../greensTheorem/exercisesDivergenceAndLineIntegrals/}
  {../shapeOfThingsToCome/exercisesDivergenceTheorem/}
  {../greensTheorem/}
  {../shapeOfThingsToCome/}
  {../separableDifferentialEquations/exercises/}
  {vectorFields/}
}

\newcommand{\mooculus}{\textsf{\textbf{MOOC}\textnormal{\textsf{ULUS}}}}

\usepackage{tkz-euclide}
\usepackage{tikz}
\usepackage{tikz-cd}
\usetikzlibrary{arrows}
\tikzset{>=stealth,commutative diagrams/.cd,
  arrow style=tikz,diagrams={>=stealth}} %% cool arrow head
\tikzset{shorten <>/.style={ shorten >=#1, shorten <=#1 } } %% allows shorter vectors

\usetikzlibrary{backgrounds} %% for boxes around graphs
\usetikzlibrary{shapes,positioning}  %% Clouds and stars
\usetikzlibrary{matrix} %% for matrix
\usepgfplotslibrary{polar} %% for polar plots
\usepgfplotslibrary{fillbetween} %% to shade area between curves in TikZ
%\usetkzobj{all}
\usepackage[makeroom]{cancel} %% for strike outs
%\usepackage{mathtools} %% for pretty underbrace % Breaks Ximera
%\usepackage{multicol}
\usepackage{pgffor} %% required for integral for loops



%% http://tex.stackexchange.com/questions/66490/drawing-a-tikz-arc-specifying-the-center
%% Draws beach ball
\tikzset{pics/carc/.style args={#1:#2:#3}{code={\draw[pic actions] (#1:#3) arc(#1:#2:#3);}}}



\usepackage{array}
\setlength{\extrarowheight}{+.1cm}
\newdimen\digitwidth
\settowidth\digitwidth{9}
\def\divrule#1#2{
\noalign{\moveright#1\digitwidth
\vbox{\hrule width#2\digitwidth}}}




% \newcommand{\RR}{\mathbb R}
% \newcommand{\R}{\mathbb R}
% \newcommand{\N}{\mathbb N}
% \newcommand{\Z}{\mathbb Z}

\newcommand{\sagemath}{\textsf{SageMath}}


%\renewcommand{\d}{\,d\!}
%\renewcommand{\d}{\mathop{}\!d}
%\newcommand{\dd}[2][]{\frac{\d #1}{\d #2}}
%\newcommand{\pp}[2][]{\frac{\partial #1}{\partial #2}}
% \renewcommand{\l}{\ell}
%\newcommand{\ddx}{\frac{d}{\d x}}

% \newcommand{\zeroOverZero}{\ensuremath{\boldsymbol{\tfrac{0}{0}}}}
%\newcommand{\inftyOverInfty}{\ensuremath{\boldsymbol{\tfrac{\infty}{\infty}}}}
%\newcommand{\zeroOverInfty}{\ensuremath{\boldsymbol{\tfrac{0}{\infty}}}}
%\newcommand{\zeroTimesInfty}{\ensuremath{\small\boldsymbol{0\cdot \infty}}}
%\newcommand{\inftyMinusInfty}{\ensuremath{\small\boldsymbol{\infty - \infty}}}
%\newcommand{\oneToInfty}{\ensuremath{\boldsymbol{1^\infty}}}
%\newcommand{\zeroToZero}{\ensuremath{\boldsymbol{0^0}}}
%\newcommand{\inftyToZero}{\ensuremath{\boldsymbol{\infty^0}}}



% \newcommand{\numOverZero}{\ensuremath{\boldsymbol{\tfrac{\#}{0}}}}
% \newcommand{\dfn}{\textbf}
% \newcommand{\unit}{\,\mathrm}
% \newcommand{\unit}{\mathop{}\!\mathrm}
% \newcommand{\eval}[1]{\bigg[ #1 \bigg]}
% \newcommand{\seq}[1]{\left( #1 \right)}
% \renewcommand{\epsilon}{\varepsilon}
% \renewcommand{\phi}{\varphi}


% \renewcommand{\iff}{\Leftrightarrow}

% \DeclareMathOperator{\arccot}{arccot}
% \DeclareMathOperator{\arcsec}{arcsec}
% \DeclareMathOperator{\arccsc}{arccsc}
% \DeclareMathOperator{\si}{Si}
% \DeclareMathOperator{\scal}{scal}
% \DeclareMathOperator{\sign}{sign}


%% \newcommand{\tightoverset}[2]{% for arrow vec
%%   \mathop{#2}\limits^{\vbox to -.5ex{\kern-0.75ex\hbox{$#1$}\vss}}}
% \newcommand{\arrowvec}[1]{{\overset{\rightharpoonup}{#1}}}
% \renewcommand{\vec}[1]{\arrowvec{\mathbf{#1}}}
% \renewcommand{\vec}[1]{{\overset{\boldsymbol{\rightharpoonup}}{\mathbf{#1}}}}

% \newcommand{\point}[1]{\left(#1\right)} %this allows \vector{ to be changed to \vector{ with a quick find and replace
% \newcommand{\pt}[1]{\mathbf{#1}} %this allows \vec{ to be changed to \vec{ with a quick find and replace
% \newcommand{\Lim}[2]{\lim_{\point{#1} \to \point{#2}}} %Bart, I changed this to point since I want to use it.  It runs through both of the exercise and exerciseE files in limits section, which is why it was in each document to start with.

% \DeclareMathOperator{\proj}{\mathbf{proj}}
% \newcommand{\veci}{{\boldsymbol{\hat{\imath}}}}
% \newcommand{\vecj}{{\boldsymbol{\hat{\jmath}}}}
% \newcommand{\veck}{{\boldsymbol{\hat{k}}}}
% \newcommand{\vecl}{\vec{\boldsymbol{\l}}}
% \newcommand{\uvec}[1]{\mathbf{\hat{#1}}}
% \newcommand{\utan}{\mathbf{\hat{t}}}
% \newcommand{\unormal}{\mathbf{\hat{n}}}
% \newcommand{\ubinormal}{\mathbf{\hat{b}}}

% \newcommand{\dotp}{\bullet}
% \newcommand{\cross}{\boldsymbol\times}
% \newcommand{\grad}{\boldsymbol\nabla}
% \newcommand{\divergence}{\grad\dotp}
% \newcommand{\curl}{\grad\cross}
%\DeclareMathOperator{\divergence}{divergence}
%\DeclareMathOperator{\curl}[1]{\grad\cross #1}
% \newcommand{\lto}{\mathop{\longrightarrow\,}\limits}

% \renewcommand{\bar}{\overline}

\colorlet{textColor}{black}
\colorlet{background}{white}
\colorlet{penColor}{blue!50!black} % Color of a curve in a plot
\colorlet{penColor2}{red!50!black}% Color of a curve in a plot
\colorlet{penColor3}{red!50!blue} % Color of a curve in a plot
\colorlet{penColor4}{green!50!black} % Color of a curve in a plot
\colorlet{penColor5}{orange!80!black} % Color of a curve in a plot
\colorlet{penColor6}{yellow!70!black} % Color of a curve in a plot
\colorlet{fill1}{penColor!20} % Color of fill in a plot
\colorlet{fill2}{penColor2!20} % Color of fill in a plot
\colorlet{fillp}{fill1} % Color of positive area
\colorlet{filln}{penColor2!20} % Color of negative area
\colorlet{fill3}{penColor3!20} % Fill
\colorlet{fill4}{penColor4!20} % Fill
\colorlet{fill5}{penColor5!20} % Fill
\colorlet{gridColor}{gray!50} % Color of grid in a plot

\newcommand{\surfaceColor}{violet}
\newcommand{\surfaceColorTwo}{redyellow}
\newcommand{\sliceColor}{greenyellow}




\pgfmathdeclarefunction{gauss}{2}{% gives gaussian
  \pgfmathparse{1/(#2*sqrt(2*pi))*exp(-((x-#1)^2)/(2*#2^2))}%
}


%%%%%%%%%%%%%
%% Vectors
%%%%%%%%%%%%%

%% Simple horiz vectors
\renewcommand{\vector}[1]{\left\langle #1\right\rangle}


%% %% Complex Horiz Vectors with angle brackets
%% \makeatletter
%% \renewcommand{\vector}[2][ , ]{\left\langle%
%%   \def\nextitem{\def\nextitem{#1}}%
%%   \@for \el:=#2\do{\nextitem\el}\right\rangle%
%% }
%% \makeatother

%% %% Vertical Vectors
%% \def\vector#1{\begin{bmatrix}\vecListA#1,,\end{bmatrix}}
%% \def\vecListA#1,{\if,#1,\else #1\cr \expandafter \vecListA \fi}

%%%%%%%%%%%%%
%% End of vectors
%%%%%%%%%%%%%

%\newcommand{\fullwidth}{}
%\newcommand{\normalwidth}{}



%% makes a snazzy t-chart for evaluating functions
%\newenvironment{tchart}{\rowcolors{2}{}{background!90!textColor}\array}{\endarray}

%%This is to help with formatting on future title pages.
\newenvironment{sectionOutcomes}{}{}



%% Flowchart stuff
%\tikzstyle{startstop} = [rectangle, rounded corners, minimum width=3cm, minimum height=1cm,text centered, draw=black]
%\tikzstyle{question} = [rectangle, minimum width=3cm, minimum height=1cm, text centered, draw=black]
%\tikzstyle{decision} = [trapezium, trapezium left angle=70, trapezium right angle=110, minimum width=3cm, minimum height=1cm, text centered, draw=black]
%\tikzstyle{question} = [rectangle, rounded corners, minimum width=3cm, minimum height=1cm,text centered, draw=black]
%\tikzstyle{process} = [rectangle, minimum width=3cm, minimum height=1cm, text centered, draw=black]
%\tikzstyle{decision} = [trapezium, trapezium left angle=70, trapezium right angle=110, minimum width=3cm, minimum height=1cm, text centered, draw=black]


\title{Quadratic Zeros}

\begin{document}

\begin{abstract}
quadratic formula
\end{abstract}
\maketitle



\subsection{Elementary Functions}

We have two types of elementary functions, so far:

\begin{itemize}
\item Linear functions
\item Quadratic functions
\end{itemize}

One of our goals for analysis is to identify zeros of functions. \\

Unless the linear function is actually a constant function (which is a linear function), the linear function has exactly one zero. We can identify this unique zero by setting the formula equal to $0$ and solving.  This is accomplished by combining like terms and isolating the variable on one side of the equation.

We have seen one approach to identifying zeros of quadratic functions.
\begin{itemize}
\item Completing the Square
\end{itemize}

We have a second approach. \\




\section{The Quadratic Formula}


After stepping through the procedure of completing the square, we could have extended the procedure and solved for $t$.


Completing the square gave us 
\[ a t^2 + b t + c = a \left(t + \frac{b}{2 a} \right)^2 + c - \frac{b^2}{4 a} \]


Now, pretend this was equal to $0$ in a quadratic equation.


\[ a \, t^2 + b \, t + c = 0 \]

\[ a\left(t + \frac{b}{2 a}\right)^2 + c - \frac{b^2}{4 a}  = 0\]

Solve for $t$.



\begin{explanation}
\[ a \left(t + \frac{b}{2 a} \right)^2  = \frac{b^2}{4 a} - c\]

\[ \left(t + \frac{b}{2 a} \right)^2  = \frac{b^2}{4 a^2} - \frac{c}{a}\]

\[ \left(t + \frac{b}{2 a} \right)^2  = \frac{\answer{b^2 - 4 a c}}{4 a^2} \]
\end{explanation}


Either 


\[ t + \frac{b}{2 a}  = \sqrt{\frac{b^2 - 4 a c}{4 a^2}}  = \frac{\sqrt{b^2 - 4 a c}}{| 2a |}   \]

or


\[ t + \frac{b}{2 a}  = -\sqrt{\frac{b^2 - 4 a c}{4 a^2}} = -\frac{\sqrt{b^2 - 4 a c}}{| 2a |}    \]



\begin{itemize}
\item If $a > 0$, then $| 2a | = 2a$.
\item If $a < 0$, then $| 2a | = -2a$.
\end{itemize}

Either way, we still get one negative and one positive fraction.  Therefore, we can drop the absolute value signs.  





Either 


\[ t + \frac{b}{2 a}  = \sqrt{\frac{b^2 - 4 a c}{4 a^2}}  = \frac{\sqrt{b^2 - 4 a c}}{2a}   \]

or


\[ t + \frac{b}{2 a}  = -\sqrt{\frac{b^2 - 4 a c}{4 a^2}} = -\frac{\sqrt{b^2 - 4 a c}}{2a}    \]










And, finally \\


\begin{conclusion}



Either 


\[ t   = - \frac{b}{2 a} + \frac{\sqrt{b^2 - 4 a c}}{2a}  = \frac{-b + \sqrt{b^2 - 4 a c}}{2a}      \]

or


\[ t  = - \frac{b}{2 a}  -\frac{\sqrt{b^2 - 4 a c}}{2a} =    \frac{-b - \sqrt{b^2 - 4 a c}}{2a}      \]

\end{conclusion}






People generally shorthand these two separate solutions as



\[ t  =   \frac{-b \pm \sqrt{b^2 - 4 a c}}{2a}      \]


This is known as \textbf{The Quadratic Formula}.




\begin{definition} \textbf{\textcolor{green!50!black}{The Quadratic Formula}} \\



If $a \, t^2 + b \, t + c = 0$ with $a \ne 0$, then


\[ t   = - \frac{b}{2 a} + \frac{\sqrt{b^2 - 4 a c}}{2a}  = \frac{-b + \sqrt{b^2 - 4 a c}}{2a}      \]

or

\[ t  = - \frac{b}{2 a}  -\frac{\sqrt{b^2 - 4 a c}}{2a} =    \frac{-b - \sqrt{b^2 - 4 a c}}{2a}      \]



Shorthand: 
\[ t  =   \frac{-b \pm \sqrt{b^2 - 4 a c}}{2a}      \]


\end{definition}









Let's apply this to the previous three examples







\begin{example} \textit{Two Real Solutions} 

Solve $4 t^2 - 4 t - 8 = 0$ \\



\begin{explanation}

To use the Quadratic Formula the quadratic equation has to have everything on one side and $0$ on the other.  Our equation is already in that form.

Apply the Quadratic Formula



\begin{procedure}
To use the Quadratic Formula, we match our quadratic to the general quadratic template:

\[
4 t^2 - 4 t - 8 = 0 = a \, t^2 + b \, t + c
\]


\begin{itemize}
\item Matching the leading terms tells us that $a = 4$.
\item Matching the linear terms tells us that $b = -4$.
\item Matching the constant terms tells us that $c = -8$.
\end{itemize}

These are the substitutions we need to perform in the Quadratic Formula.

\end{procedure}
















\[   t = \frac{-(-4) \pm \sqrt{(-4)^2 - 4 (4) (-8)}}{2 (4)}            \]

\[   t = \frac{4 \pm \sqrt{16 + 128}}{8}            \]

\[   t = \frac{4 \pm \sqrt{\answer{144}}}{8}            \]

\[   t = \frac{4 \pm \answer{12}}{8}            \]




Either $t = \frac{4 + 12}{8} $ or $t = \frac{4 - 12}{8} $

Either $t = \frac{16}{8} $ or $t = \frac{-8}{8} $

Either $t = \answer{2}$ or $t = \answer{-1}$

\end{explanation}

\end{example}









\begin{example} \textit{One Real Solution}

Solve $2 x^2 - 12x + 21 = 3$ \\

\begin{explanation}

First, get everything to one side and $0$ on the other side.



\[  2 x^2 - 12x + 18 = 0  \]

Apply the Quadratic Formula


\[   t = \frac{-(-12) \pm \sqrt{(-12)^2 - 4 (2) (18)}}{2 (2)}            \]



\[   t = \frac{12 \pm \sqrt{144 - 144}}{4}            \]

\[   t = \frac{12 \pm \answer{0}}{4}            \]

If you add or subtract $0$, you get the same result.

\[   t = \frac{12}{4}   = 3         \]




\end{explanation}


\end{example}








\begin{example} \textit{No Real Solutions}

Solve $2 m^2 - 12m + 21 = 1$ \\

\begin{explanation}

First, get everything to one side and $0$ on the other side.



\[  2 m^2 - 12m + 20 = 0  \]

Apply the Quadratic Formula


\[   m = \frac{-(-12) \pm \sqrt{(-12)^2 - 4 (2) (20)}}{2 (2)}            \]

\[   m = \frac{12 \pm \sqrt{144 - 160}}{4}            \]

\[   m = \frac{12 \pm \sqrt{\answer{-16}}}{4}            \]



The real numbers don't hold a number equal to $\sqrt{-16}$.  Therefore, there are no real solutions.



\end{explanation}

\end{example}








In each of the examples above the number of solutions was determined by $\sqrt{b^2 - 4 a c}$.  The inside of the square root, $b^2 - 4 a c$, is called the \textbf{discriminant} and its sign tells us how many real solutions the equation has.


\begin{itemize}
\item If $b^2 - 4 a c$ \, \wordChoice{\choice[correct]{$>$}\choice{$=$}\choice{$<$}} \, $0$, then there are two distinct real solutions.
\item If $b^2 - 4 a c$ \, \wordChoice{\choice{$>$}\choice[correct]{$=$}\choice{$<$}} \, $0$, then there is one real solutions.
\item If $b^2 - 4 a c$ \, \wordChoice{\choice{$>$}\choice{$=$}\choice[correct]{$<$}} \, $0$, then there are no distinct real solutions.

\end{itemize}













\begin{center}
\textbf{\textcolor{green!50!black}{ooooo=-=-=-=-=-=-=-=-=-=-=-=-=ooOoo=-=-=-=-=-=-=-=-=-=-=-=-=ooooo}} \\

more examples can be found by following this link\\ \link[More Examples of Quadratics]{https://ximera.osu.edu/csccmathematics/precalculus1/precalculus1/projectileMotion/examples/exampleList}

\end{center}




\end{document}
