\documentclass{ximera}


\graphicspath{
  {./}
  {ximeraTutorial/}
  {basicPhilosophy/}
}

\newcommand{\mooculus}{\textsf{\textbf{MOOC}\textnormal{\textsf{ULUS}}}}


\usepackage{tkz-euclide}\usepackage{tikz}
\usepackage{tikz-cd}
\usetikzlibrary{arrows}
\tikzset{>=stealth,commutative diagrams/.cd,
  arrow style=tikz,diagrams={>=stealth}} %% cool arrow head
\tikzset{shorten <>/.style={ shorten >=#1, shorten <=#1 } } %% allows shorter vectors

\usetikzlibrary{backgrounds} %% for boxes around graphs
\usetikzlibrary{shapes,positioning}  %% Clouds and stars
\usetikzlibrary{matrix} %% for matrix
\usepgfplotslibrary{polar} %% for polar plots
\usepgfplotslibrary{fillbetween} %% to shade area between curves in TikZ
\usetkzobj{all}
\usepackage[makeroom]{cancel} %% for strike outs
%\usepackage{mathtools} %% for pretty underbrace % Breaks Ximera
%\usepackage{multicol}
\usepackage{pgffor} %% required for integral for loops



%% http://tex.stackexchange.com/questions/66490/drawing-a-tikz-arc-specifying-the-center
%% Draws beach ball
\tikzset{pics/carc/.style args={#1:#2:#3}{code={\draw[pic actions] (#1:#3) arc(#1:#2:#3);}}}



\usepackage{array}
\setlength{\extrarowheight}{+.1cm}
\newdimen\digitwidth
\settowidth\digitwidth{9}
\def\divrule#1#2{
\noalign{\moveright#1\digitwidth
\vbox{\hrule width#2\digitwidth}}}
























%%This is to help with formatting on future title pages.
\newenvironment{sectionOutcomes}{}{}


\title{Quadratic Zeros}

\begin{document}

\begin{abstract}
quadratic formula
\end{abstract}
\maketitle







\section{The Quadratic Formula}


After stepping through the procedure of completing the square, we could have extended the procedure and solved for $t$.


Completing the square gave us 
\[ a t^2 + b t + c = a \left(t + \frac{b}{2 a} \right)^2 + c - \frac{b^2}{4 a^2} \]


Now, pretend this was equal to $0$ in a quadratic equation.


\[ a t^2 + b t + c = 0 \]

\[ a\left(t + \frac{b}{2 a}\right)^2 + c - \frac{b^2}{4 a^2}  = 0\]

Solve for $t$.


\[ a \left(t + \frac{b}{2 a} \right)^2  = \frac{b^2}{4 a^2} - c\]

\[ \left(t + \frac{b}{2 a} \right)^2  = \frac{b^2}{4 a^2} - \frac{c}{a}\]

\[ \left(t + \frac{b}{2 a} \right)^2  = \frac{\answer{b^2 - 4 a c}}{4 a^2} \]



Either 


\[ t + \frac{b}{2 a}  = \sqrt{\frac{b^2 - 4 a c}{4 a^2}}  = \frac{\sqrt{b^2 - 4 a c}}{| 2a |}   \]

or


\[ t + \frac{b}{2 a}  = -\sqrt{\frac{b^2 - 4 a c}{4 a^2}} = -\frac{\sqrt{b^2 - 4 a c}}{| 2a |}    \]



\begin{itemize}
\item If $a > 0$, then $| 2a | = 2a$.
\item If $a < 0$, then $| 2a | = -2a$.
\end{itemize}

Either way, we still get one negative and one positive fraction.  Therefore, we can drop the absolute value signs.  




Either 


\[ t + \frac{b}{2 a}  = \sqrt{\frac{b^2 - 4 a c}{4 a^2}}  = \frac{\sqrt{b^2 - 4 a c}}{2a}   \]

or


\[ t + \frac{b}{2 a}  = -\sqrt{\frac{b^2 - 4 a c}{4 a^2}} = -\frac{\sqrt{b^2 - 4 a c}}{2a}    \]










And, finally \\



Either 


\[ t   = - \frac{b}{2 a} + \frac{\sqrt{b^2 - 4 a c}}{2a}  = \frac{\answer{-b} + \sqrt{b^2 - 4 a c}}{2a}      \]

or


\[ t  = - \frac{b}{2 a}  -\frac{\sqrt{b^2 - 4 a c}}{2a} =    \frac{-b - \sqrt{b^2 - 4 a c}}{2a}      \]




People generally shorthand these two separate solutions as



\[ t  =   \frac{-b \pm \sqrt{b^2 - 4 a c}}{2a}      \]


This is known as \textbf{The Quadratic Formula}.


Let's apply this to the previous three examples







\begin{example} \textit{Two Real Solutions} 

Solve $4 t^2 - 4 t - 8 = 0$ \\

To use the Quadratic Formula the quadratic equation has to have everything on one side and $0$ on the other.  Our equation is in that form.

Apply the Quadratic Formula


\[   t = \frac{-(-4) \pm \sqrt{(-4)^2 - 4 (4) (-8)}}{2 (4)}            \]

\[   t = \frac{4 \pm \sqrt{16 + 128}}{8}            \]

\[   t = \frac{4 \pm \sqrt{\answer{144}}}{8}            \]

\[   t = \frac{4 \pm \answer{12}}{8}            \]




Either $t = \frac{4 + 12}{8} $ or $t = \frac{4 - 12}{8} $

Either $t = \frac{16}{8} $ or $t = \frac{-8}{8} $

Either $t = \answer{2}$ or $t = \answer{-1}$



\end{example}









\begin{example} \textit{One Real Solution}

Solve $2 x^2 - 12x + 21 = 3$ \\


First, get everything to one side and $0$ on the other side.



\[  2 x^2 - 12x + 18 = 0  \]

Apply the Quadratic Formula


\[   t = \frac{-(-12) \pm \sqrt{(-12)^2 - 4 (2) (18)}}{2 (2)}            \]



\[   t = \frac{12 \pm \sqrt{144 - 144}}{4}            \]

\[   t = \frac{12 \pm \answer{0}}{4}            \]

If you add or subtract $0$, you get the same result.

\[   t = \frac{12}{4}   = 3         \]







\end{example}








\begin{example} \textit{No Real Solutions}

Solve $2 m^2 - 12m + 21 = 1$ \\


First, get everything to one side and $0$ on the other side.



\[  2 m^2 - 12m + 20 = 0  \]

Apply the Quadratic Formula


\[   m = \frac{-(-12) \pm \sqrt{(-12)^2 - 4 (2) (20)}}{2 (2)}            \]

\[   m = \frac{12 \pm \sqrt{144 - 160}}{4}            \]

\[   m = \frac{12 \pm \sqrt{\answer{-16}}}{4}            \]



The real numbers don't hold a number equal to $\sqrt{-16}$.  Therefore, there are no real solutions.





\end{example}








In each of the examples above the number of solutions was determined by $\sqrt{b^2 - 4 a c}$.  The inside of the square root, $b^2 - 4 a c$, is called the \textbf{discriminant} and its sign tells us how many real solutions the equation has.


\begin{itemize}
\item If $b^2 - 4 a c$ \wordChoice{\choice[correct]{>}\choice{=}\choice{<}} $0$, then there are two distinct real solutions.
\item If $b^2 - 4 a c$ \wordChoice{\choice{>}\choice[correct]{=}\choice{<}} $0$, then there is real solutions.
\item If $b^2 - 4 a c$ \wordChoice{\choice{>}\choice{=}\choice[correct]{<}} $0$, then there are no distinct real solutions.

\end{itemize}







\end{document}
