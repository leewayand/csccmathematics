\documentclass{ximera}


\graphicspath{
  {./}
  {ximeraTutorial/}
  {basicPhilosophy/}
}

\newcommand{\mooculus}{\textsf{\textbf{MOOC}\textnormal{\textsf{ULUS}}}}


\usepackage{tkz-euclide}\usepackage{tikz}
\usepackage{tikz-cd}
\usetikzlibrary{arrows}
\tikzset{>=stealth,commutative diagrams/.cd,
  arrow style=tikz,diagrams={>=stealth}} %% cool arrow head
\tikzset{shorten <>/.style={ shorten >=#1, shorten <=#1 } } %% allows shorter vectors

\usetikzlibrary{backgrounds} %% for boxes around graphs
\usetikzlibrary{shapes,positioning}  %% Clouds and stars
\usetikzlibrary{matrix} %% for matrix
\usepgfplotslibrary{polar} %% for polar plots
\usepgfplotslibrary{fillbetween} %% to shade area between curves in TikZ
\usetkzobj{all}
\usepackage[makeroom]{cancel} %% for strike outs
%\usepackage{mathtools} %% for pretty underbrace % Breaks Ximera
%\usepackage{multicol}
\usepackage{pgffor} %% required for integral for loops



%% http://tex.stackexchange.com/questions/66490/drawing-a-tikz-arc-specifying-the-center
%% Draws beach ball
\tikzset{pics/carc/.style args={#1:#2:#3}{code={\draw[pic actions] (#1:#3) arc(#1:#2:#3);}}}



\usepackage{array}
\setlength{\extrarowheight}{+.1cm}
\newdimen\digitwidth
\settowidth\digitwidth{9}
\def\divrule#1#2{
\noalign{\moveright#1\digitwidth
\vbox{\hrule width#2\digitwidth}}}
























%%This is to help with formatting on future title pages.
\newenvironment{sectionOutcomes}{}{}


\title{Exponential Functions}

\begin{document}

\begin{abstract}
characteristics
\end{abstract}
\maketitle




The formula template for the basic exponential function looks like




\[  a \, r^x   \, \text{ with } \,  a, r \in \mathbb{R} \, | \,  r > 0   \]


As we have seen before, the coefficient $a$ controls vertical stretching or compression.




We have four combinations

\begin{itemize}
\item $a>0$ and $r>1$
\item $a>0$ and $r<1$
\item $a<0$ and $r>1$
\item $a<0$ and $r<1$
\end{itemize}


When $r>1$, then the horizontal axis is a horizontal asymptote in one direction or there other.



\[  \lim_{x \to -\infty} a \, r^x = 0 \, \text{ and } \, \lim_{x \to \infty} a \, r^x = \pm\infty \]


The sign of $a$ dictates if the unbounded growth is positive or negative.




\begin{image}
\begin{tikzpicture}
   \begin{axis}[name = leftgraph, 
            domain=-10:10, ymax=10, xmax=10, ymin=-10, xmin=-10,
            axis lines =center, xlabel=$x$, ylabel={$a>0$},
            every axis y label/.style={at=(current axis.above origin),anchor=south},
            every axis x label/.style={at=(current axis.right of origin),anchor=west},
            axis on top
          ]
          
          \addplot [line width=2, penColor, smooth,samples=100,domain=(-9:8), <->] {1.3^x};
          \addplot [color=penColor,only marks,mark=*] coordinates{(0,1)};
          %\addplot [line width=2, penColor, smooth,samples=100,domain=(1:7)] ({x},{0.5x-7});
          %\addplot [color=penColor,only marks,mark=*] coordinates{(-4,-1) (0,1) (1,-6.5) (7,-3.5)};

          %\addplot [line width=1, penColor2, smooth,samples=100,domain=(-4:0)] ({x},{0});
          %\addplot [line width=1, penColor2, smooth,samples=100,domain=(1:7)] ({x},{0});
           

  \end{axis}
  \begin{axis}[at={(leftgraph.outer east)},anchor=outer west, 
            domain=-10:10, ymax=10, xmax=10, ymin=-10, xmin=-10,
            axis lines =center, xlabel=$x$, ylabel={$a<0$},
            every axis y label/.style={at=(current axis.above origin),anchor=south},
            every axis x label/.style={at=(current axis.right of origin),anchor=west},
            axis on top
          ]
          
        \addplot [line width=2, penColor, smooth,samples=100,domain=(-9:8),<->] {-(1.3^x)};
        \addplot [color=penColor,only marks,mark=*] coordinates{(0,-1)};
          %\addplot [line width=2, penColor, smooth,samples=100,domain=(1:7)] ({x},{0.5x-7});
          %\addplot [color=penColor,only marks,mark=*] coordinates{(-4,-1) (0,1) (1,-6.5) (7,-3.5)};

          %\addplot [line width=1, penColor2, smooth,samples=100,domain=(-4:0)] ({x},{0});
          %\addplot [line width=1, penColor2, smooth,samples=100,domain=(1:7)] ({x},{0});
           

  \end{axis}
\end{tikzpicture}
\end{image}







When $r<1$, then the horizontal axis is a horizontal asymptote in one direction or there other - just the reverse from when $r > 1$.


\[ \lim_{x \to -\infty} a \, r^x = \pm\infty \, \text{ and } \, \lim_{x \to \infty} a \, r^x = 0 \]



The sign of $a$ dictates if the unbounded growth is positive or negative.



\begin{image}
\begin{tikzpicture}
   \begin{axis}[name = leftgraph, 
            domain=-10:10, ymax=10, xmax=10, ymin=-10, xmin=-10,
            axis lines =center, xlabel=$x$, ylabel={$a>0$},
            every axis y label/.style={at=(current axis.above origin),anchor=south},
            every axis x label/.style={at=(current axis.right of origin),anchor=west},
            axis on top
          ]
          
          \addplot [line width=2, penColor, smooth,samples=200,domain=(-8:9), <->] {1.3^(-x)};
          %\addplot [line width=2, penColor, smooth,samples=100,domain=(1:7)] ({x},{0.5x-7});
          \addplot [color=penColor,only marks,mark=*] coordinates{(0,1)};

          %\addplot [line width=1, penColor2, smooth,samples=100,domain=(-4:0)] ({x},{0});
          %\addplot [line width=1, penColor2, smooth,samples=100,domain=(1:7)] ({x},{0});
           

  \end{axis}
  \begin{axis}[at={(leftgraph.outer east)},anchor=outer west, 
            domain=-10:10, ymax=10, xmax=10, ymin=-10, xmin=-10,
            axis lines =center, xlabel=$x$, ylabel={$a<0$},
            every axis y label/.style={at=(current axis.above origin),anchor=south},
            every axis x label/.style={at=(current axis.right of origin),anchor=west},
            axis on top
          ]
          
        \addplot [line width=2, penColor, smooth,samples=200,domain=(-8:9),<->] {-(1.3^(-x)};
        \addplot [color=penColor,only marks,mark=*] coordinates{(0,-1)};
          %\addplot [line width=2, penColor, smooth,samples=100,domain=(1:7)] ({x},{0.5x-7});
          %\addplot [color=penColor,only marks,mark=*] coordinates{(-4,-1) (0,1) (1,-6.5) (7,-3.5)};

          %\addplot [line width=1, penColor2, smooth,samples=100,domain=(-4:0)] ({x},{0});
          %\addplot [line width=1, penColor2, smooth,samples=100,domain=(1:7)] ({x},{0});
           

  \end{axis}
\end{tikzpicture}
\end{image}


All four graphs share a common structure.


\begin{itemize}
\item All have the horizontal axis as an asymptote.
\item All are a distance of $1$ from the asymptote, when the exponent equals $0$.
\end{itemize}

These are the important aspects or characteristics that we will use when shifting and stretching.






\begin{example}  Exponential Function



Analyze   $f(x) = \frac{1}{3} \, 2^{x+5} - 7$ \\


For the basic exponential function graph, the horizontal axis is the horizontal asymptote.  Here, this has been moved down $7$.



The "inside", representing the domain, is $x+5$.  This equals $0$, when $x=-5$.  The exponent is positive for $x>-5$, since the base is $2 > 1$, this is the direction of unbounded growth.  Therefore, the other direction (left) is where the horizontal asymptote is in effect.  Since the coefficiewnt, $\frac{1}{3} > 0$, the unbounded growth is positive.

At $x=-5$, we have our one anchor point for the graph.  The point is $\left(-5, -7 + \frac{1}{3}\right)$, which is $\frac{1}{3}$ above the horizontal asymptote, $y = -7$.


Graph of $y = f(x)$.

\begin{image}
\begin{tikzpicture}
  \begin{axis}[
            domain=-10:10, ymax=10, xmax=10, ymin=-10, xmin=-10,
            axis lines =center, xlabel=$x$, ylabel=$y$, grid = major,
            ytick={-10,-8,-6,-4,-2,2,4,6,8,10},
          	xtick={-10,-8,-6,-4,-2,2,4,6,8,10},
          	ticklabel style={font=\scriptsize},
            every axis y label/.style={at=(current axis.above origin),anchor=south},
            every axis x label/.style={at=(current axis.right of origin),anchor=west},
            axis on top
          ]
          
      		\addplot [line width=2, penColor, smooth,samples=200,domain=(-10:0.6),<->] {0.33 * 2^(x+5)-7};

          	\addplot [line width=1, gray, dashed,samples=200,domain=(-10:10),<->] {-7};


      		\addplot[color=penColor,fill=penColor,only marks,mark=*] coordinates{(-5,-6.66)};





           

  \end{axis}
\end{tikzpicture}
\end{image}




Our graphical analysis tells us that

\begin{itemize}
\item The implied domain of $f$ is $\mathbb{R}$.
\item $f$ is always increasing.
\item $f$ has no maximums or minimums.
\item $\lim_{x \to -\infty} f(x) = -7$
\item $\lim_{x \to \infty} f(x) = \infty$
\end{itemize}





\end{example}
























\begin{example}  Exponential Function



Analyze   $B(t) = -2 \, \frac{2}{3}^{3-t} + 4$ \\



This follows the basic exponential function template $\left( \frac{2}{3}^t \right)$.  The graph of this basic exponential function would have been big and positive towards the left and approached the horizontal axis to the right, because $\frac{2}{3} < 1$




By changing the exponent: $\left( \frac{2}{3}^{3-t} \right)$, the exponent gets big and positive to the left (for negative values of $t$).  Now, the graph approaches the horizontal axis to the left and gets big towards the right.



Changing the the sign of the leading coefficient flips the graph vertically: $-2  \, \frac{2}{3}^{3-t}$.  Now, the graph approaches the horizontal axis to the \wordChoice{\choice[correct]{left} \choice{right}}  and gets big and \wordChoice{\choice[correct]{negative} \choice{positive}} towards the right.


Adding $4$ to the outside shifts the graph vertically up $4$.


$3-t=0$ when $t=3$. Our one anchor point is shifted over to $3$.  Multipying by $-2$, means the dot is $2$ away from the horizontal asymptote, which is now $y=4$.





Graph of $y = B(t)$.

\begin{image}
\begin{tikzpicture}
  \begin{axis}[
            domain=-10:10, ymax=10, xmax=10, ymin=-10, xmin=-10,
            axis lines =center, xlabel=$t$, ylabel=$y$, grid = major,
            ytick={-10,-8,-6,-4,-2,2,4,6,8,10},
          	xtick={-10,-8,-6,-4,-2,2,4,6,8,10},
          	ticklabel style={font=\scriptsize},
            every axis y label/.style={at=(current axis.above origin),anchor=south},
            every axis x label/.style={at=(current axis.right of origin),anchor=west},
            axis on top
          ]
          
      		\addplot [line width=2, penColor, smooth,samples=200,domain=(-10:7.5),<->] {-2 * (0.666^(3-x)) + 4};

          	\addplot [line width=1, gray, dashed,samples=200,domain=(-10:10),<->] {4};


      		\addplot[color=penColor,fill=penColor,only marks,mark=*] coordinates{(3,2)};





           

  \end{axis}
\end{tikzpicture}
\end{image}




Our graphical analysis tells us that 

\begin{itemize}
\item The implied domain of $B$ is $\mathbb{R}$.
\item $B$ is always decreasing.
\item $B$ has no maximums or minimums.
\item $\lim_{t \to -\infty} B(t) = 4$
\item $\lim_{t \to \infty} B(t) = -\infty$
\end{itemize}



\end{example}





















\begin{example}  Exponential Function



Analyze   $K(f) = 3^{5-f} - 5$ \\


\begin{question}. 

The exponent gets big and positive when $f$ gets big and \wordChoice{\choice{positive}\choice[correct]{negative}}.
\end{question}
\begin{question}. 

The graph will widen to the \wordChoice{\choice{right}\choice[correct]{left}}.\\
\end{question}
\begin{question}. 

The graph will approach the asymptote to the \wordChoice{\choice[correct]{right}\choice{left}}.\\
\end{question}
\begin{question}. 

Our one anchor point moves to $\left(\answer{5}, \answer{-4}\right)$.
\end{question}
\begin{question}. 

The graph will become unbounded \wordChoice{\choice[correct]{up}\choice{down}}.\\
\end{question}




Graph of $y = K(f)$.

\begin{image}
\begin{tikzpicture}
  \begin{axis}[
            domain=-10:10, ymax=10, xmax=10, ymin=-10, xmin=-10,
            axis lines =center, xlabel=$f$, ylabel=$y$, grid = major,
            ytick={-10,-8,-6,-4,-2,2,4,6,8,10},
          	xtick={-10,-8,-6,-4,-2,2,4,6,8,10},
          	ticklabel style={font=\scriptsize},
            every axis y label/.style={at=(current axis.above origin),anchor=south},
            every axis x label/.style={at=(current axis.right of origin),anchor=west},
            axis on top
          ]
          
      		\addplot [line width=2, penColor, smooth,samples=200,domain=(2.75:10),<->] {3^(5-x) - 5};

          	\addplot [line width=1, gray, dashed,samples=200,domain=(-10:10),<->] {-5};


      		\addplot[color=penColor,fill=penColor,only marks,mark=*] coordinates{(5,-4)};





           

  \end{axis}
\end{tikzpicture}
\end{image}





Our graphical analysis tells us that

\begin{itemize}
\item The implied domain of $K$ is $\mathbb{R}$.
\item $K$ is always decreasing.
\item $K$ has no maximums or minimums.
\item $\lim_{f \to -\infty} K(f) = \infty$
\item $\lim_{f \to \infty} K(f) = -5$
\end{itemize}


\end{example}











\end{document}
