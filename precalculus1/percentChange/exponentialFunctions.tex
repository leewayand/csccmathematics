\documentclass{ximera}


\graphicspath{
  {./}
  {ximeraTutorial/}
  {basicPhilosophy/}
}

\newcommand{\mooculus}{\textsf{\textbf{MOOC}\textnormal{\textsf{ULUS}}}}


\usepackage{tkz-euclide}\usepackage{tikz}
\usepackage{tikz-cd}
\usetikzlibrary{arrows}
\tikzset{>=stealth,commutative diagrams/.cd,
  arrow style=tikz,diagrams={>=stealth}} %% cool arrow head
\tikzset{shorten <>/.style={ shorten >=#1, shorten <=#1 } } %% allows shorter vectors

\usetikzlibrary{backgrounds} %% for boxes around graphs
\usetikzlibrary{shapes,positioning}  %% Clouds and stars
\usetikzlibrary{matrix} %% for matrix
\usepgfplotslibrary{polar} %% for polar plots
\usepgfplotslibrary{fillbetween} %% to shade area between curves in TikZ
\usetkzobj{all}
\usepackage[makeroom]{cancel} %% for strike outs
%\usepackage{mathtools} %% for pretty underbrace % Breaks Ximera
%\usepackage{multicol}
\usepackage{pgffor} %% required for integral for loops



%% http://tex.stackexchange.com/questions/66490/drawing-a-tikz-arc-specifying-the-center
%% Draws beach ball
\tikzset{pics/carc/.style args={#1:#2:#3}{code={\draw[pic actions] (#1:#3) arc(#1:#2:#3);}}}



\usepackage{array}
\setlength{\extrarowheight}{+.1cm}
\newdimen\digitwidth
\settowidth\digitwidth{9}
\def\divrule#1#2{
\noalign{\moveright#1\digitwidth
\vbox{\hrule width#2\digitwidth}}}
























%%This is to help with formatting on future title pages.
\newenvironment{sectionOutcomes}{}{}


\title{Exponential Functions}

\begin{document}

\begin{abstract}
characteristics
\end{abstract}
\maketitle




The formula template for the basic exponential function looks like




\[  a \, r^x   \, \text{ with } \,  a, r \in \mathbb{R} \, | \,  r > 0   \]


As we have seen before, the coefficient $a$ controls vertical stretching or compression. The sign of $a$ dictates the sign of our function values. $r$ dictates a growing or decaying function.




We have four combinations

\begin{itemize}
\item \textbf{\textcolor{blue!55!black}{$a>0$ and $r>1$}}
\item \textbf{\textcolor{blue!55!black}{$a>0$ and $r<1$}}
\item \textbf{\textcolor{blue!55!black}{$a<0$ and $r>1$}}
\item \textbf{\textcolor{blue!55!black}{$a<0$ and $r<1$}}
\end{itemize}


For graphs of basic exponential functions, the horizontal axis is a horizontal asymptote in one direction or the other.


When $r > 1$ we have a growing function.

\[  \lim_{x \to -\infty} a \, r^x = 0 \, \text{ and } \, \lim_{x \to \infty} a \, r^x = \pm\infty \]


The sign of $a$ dictates if the unbounded growth is positive or negative.




\begin{image}
\begin{tikzpicture}
   \begin{axis}[name = leftgraph, 
            domain=-10:10, ymax=10, xmax=10, ymin=-10, xmin=-10,
            axis lines =center, xlabel={$x$}, ylabel={ },
            every axis y label/.style={at=(current axis.above origin),anchor=south},
            every axis x label/.style={at=(current axis.right of origin),anchor=west},
            axis on top
          ]
          
          \addplot [line width=2, penColor, smooth,samples=100,domain=(-9:8), <->] {1.3^x};
          \addplot [color=penColor,only marks,mark=*] coordinates{(0,1)};


          \node at (axis cs:-5,5) {$a > 0$};


          %\addplot [line width=2, penColor, smooth,samples=100,domain=(1:7)] ({x},{0.5x-7});
          %\addplot [color=penColor,only marks,mark=*] coordinates{(-4,-1) (0,1) (1,-6.5) (7,-3.5)};

          %\addplot [line width=1, penColor2, smooth,samples=100,domain=(-4:0)] ({x},{0});
          %\addplot [line width=1, penColor2, smooth,samples=100,domain=(1:7)] ({x},{0});
           

  \end{axis}
  \begin{axis}[at={(leftgraph.outer east)},anchor=outer west, 
            domain=-10:10, ymax=10, xmax=10, ymin=-10, xmin=-10,
            axis lines =center, xlabel={$x$}, ylabel={ },
            every axis y label/.style={at=(current axis.above origin),anchor=south},
            every axis x label/.style={at=(current axis.right of origin),anchor=west},
            axis on top
          ]
          
        \addplot [line width=2, penColor, smooth,samples=100,domain=(-9:8),<->] {-(1.3^x)};
        \addplot [color=penColor,only marks,mark=*] coordinates{(0,-1)};

         \node at (axis cs:-5,5) {$a < 0$};

          %\addplot [line width=2, penColor, smooth,samples=100,domain=(1:7)] ({x},{0.5x-7});
          %\addplot [color=penColor,only marks,mark=*] coordinates{(-4,-1) (0,1) (1,-6.5) (7,-3.5)};

          %\addplot [line width=1, penColor2, smooth,samples=100,domain=(-4:0)] ({x},{0});
          %\addplot [line width=1, penColor2, smooth,samples=100,domain=(1:7)] ({x},{0});
           

  \end{axis}
\end{tikzpicture}
\end{image}







When $r<1$, the horizontal axis is still a horizontal asymptote, just in the other direction. The function now decays.


\[ \lim_{x \to -\infty} a \, r^x = \pm\infty \, \text{ and } \, \lim_{x \to \infty} a \, r^x = 0 \]



The sign of $a$ dictates if the function decays through positive or negative values.



\begin{image}
\begin{tikzpicture}
   \begin{axis}[name = leftgraph, 
            domain=-10:10, ymax=10, xmax=10, ymin=-10, xmin=-10,
            axis lines =center, xlabel={$x$}, ylabel={ },
            every axis y label/.style={at=(current axis.above origin),anchor=south},
            every axis x label/.style={at=(current axis.right of origin),anchor=west},
            axis on top
          ]
          
          \addplot [line width=2, penColor, smooth,samples=200,domain=(-8:9), <->] {1.3^(-x)};
          %\addplot [line width=2, penColor, smooth,samples=100,domain=(1:7)] ({x},{0.5x-7});
          \addplot [color=penColor,only marks,mark=*] coordinates{(0,1)};

          \node at (axis cs:5,5) {$a > 0$};

          %\addplot [line width=1, penColor2, smooth,samples=100,domain=(-4:0)] ({x},{0});
          %\addplot [line width=1, penColor2, smooth,samples=100,domain=(1:7)] ({x},{0});
           

  \end{axis}
  \begin{axis}[at={(leftgraph.outer east)},anchor=outer west, 
            domain=-10:10, ymax=10, xmax=10, ymin=-10, xmin=-10,
            axis lines =center, xlabel={$x$}, ylabel={ },
            every axis y label/.style={at=(current axis.above origin),anchor=south},
            every axis x label/.style={at=(current axis.right of origin),anchor=west},
            axis on top
          ]
          
        \addplot [line width=2, penColor, smooth,samples=200,domain=(-8:9),<->] {-(1.3^(-x)};
        \addplot [color=penColor,only marks,mark=*] coordinates{(0,-1)};

         \node at (axis cs:5,5) {$a < 0$};


          %\addplot [line width=2, penColor, smooth,samples=100,domain=(1:7)] ({x},{0.5x-7});
          %\addplot [color=penColor,only marks,mark=*] coordinates{(-4,-1) (0,1) (1,-6.5) (7,-3.5)};

          %\addplot [line width=1, penColor2, smooth,samples=100,domain=(-4:0)] ({x},{0});
          %\addplot [line width=1, penColor2, smooth,samples=100,domain=(1:7)] ({x},{0});
           

  \end{axis}
\end{tikzpicture}
\end{image}


All four graphs share a common structure.


\begin{itemize}
\item All have the horizontal axis as an asymptote.
\item All are a distance of $1$ from the asymptote, when the exponent equals $0$.
\end{itemize}

These are the important aspects or characteristics that we use when shifting and stretching graphs of exponential functions.












\begin{observation} \textbf{\textcolor{blue!75!black}{Exponential Behavior}}


Basic exponential functions, $a \cdot r^x$, are either increasing functions or decreasing functions.


$\blacktriangleright$  Base Greater than $1$: $r > 1$


\begin{itemize}
\item greater positive exponents mean multiplying by the base more, which results in larger values.  
\item greater negative exponents mean multiplying by the reciprocal of the base more, which results in smaller values.  
\end{itemize}


The coefficient in front, $a$, tells us if this larger/smaller value is larger positively or negatively.


\begin{itemize}
\item $a > 0$ and $r > 1$ : increasing function
\item $a < 0$ and $r > 1$ : decreasing function  
\end{itemize}






$\blacktriangleright$  Base Less than $1$: $r < 1$


\begin{itemize} 
\item greater positive exponents mean multiplying by the base more, which results in smaller values.  
\item greater negative exponents mean multiplying by the reciprocal of the base more, which results in larger values.  
\end{itemize}


The coefficient in front, $a$, tells us if this larger/smaller value is larger positively or negatively.


\begin{itemize}
\item $a > 0$ and $r < 1$ : decreasing function
\item $a < 0$ and $r < 1$ : increasing function  
\end{itemize}



\end{observation}

















\begin{example}  Exponential Function



Analyze   $f(x) = \frac{1}{3} \, 2^{x+5}$ \\





Categorizing: $f(x) = \frac{1}{3} \, 2^{x+5}$ is an exponential function since it matches our official template, $A r^{B \, x + C}$. \\


\textbf{Domain}

$f$ is an exponential function, which tells us that its domain is $(-\infty, \infty)$. \\


\textbf{Zeros}

$f$ is an exponential function, therefore it has no zeros. \\


\textbf{Continuity}

$f$ is an exponential function, therefore it is continuous. \\



\textbf{Behavior (Increasing and Decreasing)}


\begin{itemize}
\item The base is $2$, which is greater than $1.
\item The leading coeffcient is $\frac{1}{3} > 0$.
\item The leading coeffcient of the linear exponent is $1 > 0$.
\end{itemize}

That tells us that $f$ is increasing and positive.




\textbf{End-Behavior}

$f$ is an exponential function, therefore the end-behavior of one side is $0$ and the other is unbounded. Since $f$ is positive and increasing, we have \\


\[ \lim\limits_{x \to -\infty} f(x) = 0 \]

\[ \lim\limits_{x \to \infty} f(x) = \infty\]










\textbf{Local Maximum and Minimum}

$f$ is an exponential function, therefore it has no local extrema. \\





\textbf{Global Maximum and Minimum}

$f$ is an exponential function, therefore it has no global extrema. \\





\textbf{Range}

 
\begin{itemize}
\item $f$ is continuous
\item $f$ is increasing and positive
\item $\lim\limits_{x \to -\infty} f(x) = 0$
\item $\lim\limits_{x \to \infty} f(x) = \infty$
\end{itemize}


The range is $(0, \infty)$. \\





\textbf{\textcolor{purple!85!blue}{Graphing}} \\

For graphs of exponential functions, the horizontal axis is the horizontal asymptote.  \\



The ``inside'', representing the domain, is $x+5$.  This equals $0$, when $x=-5$.  

At $x=-5$, we have our one anchor point for the graph.  The point is $\left(-5, \frac{1}{3} \right)$, which is $\frac{1}{3}$ above the horizontal asymptote, $y = 0$.


Graph of $y = f(x)$.

\begin{image}
\begin{tikzpicture}
  \begin{axis}[
            domain=-10:10, ymax=10, xmax=10, ymin=-10, xmin=-10,
            axis lines =center, xlabel=$x$, ylabel=$y$, 
            ytick={-10,-8,-6,-4,-2,2,4,6,8,10},
          	xtick={-10,-8,-6,-4,-2,2,4,6,8,10},
          	ticklabel style={font=\scriptsize},
            every axis y label/.style={at=(current axis.above origin),anchor=south},
            every axis x label/.style={at=(current axis.right of origin),anchor=west},
            axis on top
          ]

          \addplot [line width=2, gray, dashed,samples=200,domain=(-10:10),<->] {0};
          
      		\addplot [line width=2, penColor, smooth,samples=200,domain=(-10:-0.3),<->] {0.33 * 2^(x+5)};

      		\addplot[color=penColor,fill=penColor,only marks,mark=*] coordinates{(-5,0.333)};

         


 

  \end{axis}
\end{tikzpicture}
\end{image}




Our graph agrees with our analysis.



\end{example}
























\begin{example}  Exponential Function



Analyze   $B(t) = -2 \, \left( \frac{2}{3} \right)^{3-t}$ \\





Categorizing: $B(t) = -2 \, \left( \frac{2}{3} \right)^{3-t}$ is an exponential function since it matches our official template, $A r^{B \, x + C}$. \\


\begin{idea}


First, observe that the base $\frac{2}{3} < \answer{1}$.


Our base is less than $1$.  Therefore, as its exponent gets large and positive, we multiply by more $\frac{2}{3}$'s and the overall values get smaller.


Except, the variable, $t$, in the exponent is multiplied by $-1$.  Therefore, we need $t$ to get large and negative in order for the exponent to get large and positve.


\begin{itemize}
\item $\left( \frac{2}{3} \right)^{3-t}$ decays when $t$ becomes more negative.
\item $\left( \frac{2}{3} \right)^{3-t}$ grows when $t$ becomes more positive.
\end{itemize}






\begin{model}

The exponential stem of $B(t)$ is $\left( \frac{2}{3} \right)^{-t}$, which is a transformed version of the basic exponential function model $M(t) = \left( \frac{2}{3} \right)^{t}$.  



When $t < 0$, then $-t > 0$ and we get  $\left( \frac{2}{3} \right)^{-t} = \left( \frac{2}{3} \right)^{positive}$ and the stem is becoming smaller, approaching $0$.  





\[ \lim\limits_{t \to -\infty} \left( \frac{2}{3} \right)^{-t} = 0 \]



When $t > 0$, then $-t < 0$ and we get  $\left( \frac{2}{3} \right)^{-t} = \left( \frac{2}{3} \right)^{negative}$ and the stem is becoming larger.  



\[ \lim\limits_{t \to \infty} \left( \frac{2}{3} \right)^{-t} = \infty \]








\end{model}




Finally, all of that is multiplied by $-2$, which switches all of the behavior. \\






Graph of $y = B(t)$.

\begin{image}
\begin{tikzpicture}
  \begin{axis}[
            domain=-10:10, ymax=10, xmax=10, ymin=-10, xmin=-10,
            axis lines =center, xlabel=$t$, ylabel=$y$, 
            ytick={-10,-8,-6,-4,-2,2,4,6,8,10},
            xtick={-10,-8,-6,-4,-2,2,4,6,8,10},
            ticklabel style={font=\scriptsize},
            every axis y label/.style={at=(current axis.above origin),anchor=south},
            every axis x label/.style={at=(current axis.right of origin),anchor=west},
            axis on top
          ]
          
          \addplot [line width=2, gray, dashed,samples=200,domain=(-10:10),<->] {0};

          \addplot [line width=2, penColor, smooth,samples=200,domain=(-10:6.5),<->] {-2 * (0.666^(3-x))};

          \addplot[color=penColor,fill=penColor,only marks,mark=*] coordinates{(3,-2)};

          


  \end{axis}
\end{tikzpicture}
\end{image}



With this thinking, we can create an algebraic analysis.




\end{idea}








\textbf{Domain}

$B$ is an exponential function, therefore its domain is $(-\infty, \infty)$. \\




\textbf{Zeros}

$B$ is an exponential function, therefore it has no zeros. \\





\textbf{Continuity}

$B$ is an exponential function, therefore it is continuous. \\






\textbf{End-Behavior}

$B$ is an exponential function, therefore on one side the end-behavior is $0$ and unbounded on the other side. \\


Since the leading coefficient is $-2 < 0$, we know $B$ will be unbounded negatively.  We just need to figure out which side. \\


\begin{itemize}
  \item the base of $B$ is $\frac{2}{3} < 1$
  \item the leading coefficient of $B$ is $-2$, which is negative.
  \item the leading coefficient of the linear exponent is $-1$, which is negative.
\end{itemize}


This makes $B$ a decreasing function with neagtive values. \\


\[ \lim\limits_{t \to -\infty} B(t) = 0 \]

\[ \lim\limits_{t \to \infty} B(t) = -\infty \]




\textbf{Behavior (Increasing and  Decreasing)} \\


\begin{itemize}
  \item the base of $B$ is $\frac{2}{3} < 1$
  \item the leading coefficient of $B$ is $-2$, which is negative.
  \item the leading coefficient of the linear exponent is $-1$, which is negative.
\end{itemize}


This makes $B$ a decreasing function. \\ 






\textbf{Local Maximum and Minimum}

$B$ is an exponential function, therefore it has no local extrema. \\





\textbf{Global Maximum and Minimum}

$B$ is an exponential function, therefore it has no global extrema. \\





\textbf{Range}

 
\begin{itemize}
\item $B$ is continuous
\item $B$ is increasing and positive.
\item $\lim\limits_{t \to -\infty} B(t) = 0$
\item $\lim\limits_{t \to \infty} B(t) = \infty$
\end{itemize}


The range is $(-\infty, 0)$. \\










Our analysis agrees with the graph. \\




\end{example}





















\begin{example}  Exponential Function



Analyze   $K(f) = 3^{5-f}$ \\



\begin{idea}



\begin{question}

The formula for $K$ matches which template?

\begin{multipleChoice}
\choice{$A \, x + B$}
\choice{$A \, x^2 + B \, x + C$}
\choice{$A \, |B \, x + C | + D$}
\choice[correct]{$A \, r^{B \, x + C}$}
\choice{$A \, r^{B \, x + C} + D$}
\choice{$A \, \ln(B \, x + C) + D$}
\end{multipleChoice}


\end{question}




\begin{question} 


The exponent gets big and positive when $f$ gets big and \wordChoice{\choice{positive}\choice[correct]{negative}}.
\end{question}
\begin{question} 


The graph will grow to the \wordChoice{\choice{right}\choice[correct]{left}}.\\
\end{question}
\begin{question}


The graph will approach the asymptote to the \wordChoice{\choice[correct]{right}\choice{left}}.\\
\end{question}
\begin{question}


Our one anchor point moves to $\left(\answer{5}, \answer{1}\right)$.
\end{question}
\begin{question} 


The graph will become unbounded \wordChoice{\choice[correct]{up}\choice{down}}.\\
\end{question}




Graph of $y = K(f)$.

\begin{image}
\begin{tikzpicture}
  \begin{axis}[
            domain=-10:10, ymax=10, xmax=10, ymin=-10, xmin=-10,
            axis lines =center, xlabel=$f$, ylabel=$y$, 
            ytick={-10,-8,-6,-4,-2,2,4,6,8,10},
          	xtick={-10,-8,-6,-4,-2,2,4,6,8,10},
          	ticklabel style={font=\scriptsize},
            every axis y label/.style={at=(current axis.above origin),anchor=south},
            every axis x label/.style={at=(current axis.right of origin),anchor=west},
            axis on top
          ]
          
      		\addplot [line width=2, penColor, smooth,samples=200,domain=(3:10),<->] {3^(5-x)};

      		\addplot[color=penColor,fill=penColor,only marks,mark=*] coordinates{(5,1)};

          \addplot [line width=2, gray, dashed,samples=200,domain=(-10:10),<->] {0};



           

  \end{axis}
\end{tikzpicture}
\end{image}



With this thinking, we can create an algebraic analysis. \\

\end{idea}



\textbf{Domain}

The natural or implied domain of $K$ is $\mathbb{R}$, because $K$ is an exponential function. \\


\textbf{Zeros}

There are no zeros, because $K$ is an exponential function. \\




\textbf{Continuity}

$K$ is continuous, because $K$ is an exponential function. \\





\textbf{End-Behavior}


\begin{itemize}
  \item The base is $3 > 1$
  \item The leading coefficient is $1$, which is positive
  \item The leading coefficient of the linear exponent is $-1$, which is negative
\end{itemize}


The base is greater than $1$ and the leading coeffcients are of opposite sign.  That tells us that $K$ is decreasing.

The positive leading coefficient tells us that $K$ is a positive function.


The end-behavior of a positive decreasing exponential function is


\[ \lim\limits_{f \to -\infty} K(f) = \infty \]
\[ \lim\limits_{f \to \infty} K(f) = 0 \]



\textbf{Behavior (Increasing and Decreasing)}


The base is greater than $1$ and the leading coeffcients are of opposite sign.  That tells us that $K$ is decreasing.



\textbf{Global Maximum and Minimum}

Exponential functions do not have global maximum or minimum values.





\textbf{Local Maximum and Minimum}

Exponential functions do not have local maximum or minimum values.



\textbf{Range}

\begin{itemize}
\item $K$ is continuous
\item $K$ is decreasing and positive.
\item $\lim\limits_{f \to -\infty} K(f) = \infty $
\item $\lim\limits_{f \to \infty} K(f) = 0$
\end{itemize}


The range is $(0, \infty)$. \\







This all agrees with the graph. \\


\end{example}












\begin{center}
\textbf{\textcolor{green!50!black}{ooooo-=-=-=-ooOoo-=-=-=-ooooo}} \\

more examples can be found by following this link\\ \link[More Examples of Percent Change]{https://ximera.osu.edu/csccmathematics/precalculus1/precalculus1/percentChange/examples/exampleList}

\end{center}





\end{document}
