\documentclass{ximera}


\graphicspath{
  {./}
  {ximeraTutorial/}
  {basicPhilosophy/}
}

\newcommand{\mooculus}{\textsf{\textbf{MOOC}\textnormal{\textsf{ULUS}}}}


\usepackage{tkz-euclide}\usepackage{tikz}
\usepackage{tikz-cd}
\usetikzlibrary{arrows}
\tikzset{>=stealth,commutative diagrams/.cd,
  arrow style=tikz,diagrams={>=stealth}} %% cool arrow head
\tikzset{shorten <>/.style={ shorten >=#1, shorten <=#1 } } %% allows shorter vectors

\usetikzlibrary{backgrounds} %% for boxes around graphs
\usetikzlibrary{shapes,positioning}  %% Clouds and stars
\usetikzlibrary{matrix} %% for matrix
\usepgfplotslibrary{polar} %% for polar plots
\usepgfplotslibrary{fillbetween} %% to shade area between curves in TikZ
\usetkzobj{all}
\usepackage[makeroom]{cancel} %% for strike outs
%\usepackage{mathtools} %% for pretty underbrace % Breaks Ximera
%\usepackage{multicol}
\usepackage{pgffor} %% required for integral for loops



%% http://tex.stackexchange.com/questions/66490/drawing-a-tikz-arc-specifying-the-center
%% Draws beach ball
\tikzset{pics/carc/.style args={#1:#2:#3}{code={\draw[pic actions] (#1:#3) arc(#1:#2:#3);}}}



\usepackage{array}
\setlength{\extrarowheight}{+.1cm}
\newdimen\digitwidth
\settowidth\digitwidth{9}
\def\divrule#1#2{
\noalign{\moveright#1\digitwidth
\vbox{\hrule width#2\digitwidth}}}
























%%This is to help with formatting on future title pages.
\newenvironment{sectionOutcomes}{}{}


\title{Percent Change}

\begin{document}

\begin{abstract}
%Stuff can go here later if we want!
\end{abstract}
\maketitle




\section{A Function's Defining Characteristic}




$\blacktriangleright$ \textbf{\textcolor{blue!55!black}{Linear Functions}} 

The defining characteristic of a linear function is that it has a constant growth rate.


\[   \frac{L(b)-L(a)}{b-a} = m       \]

This led to the equation or formula for linear functions:  $L(x) = m(x-a) + L(a)$


This tells us that no matter where you are in the domain, of $L$, 



\begin{itemize}
\item if you move a distance of $1$ in the domain, then the value of $L$ increases by $m$. \\
\item if you move a distance of $D$ in the domain, then the value of $L$ increases by $m \cdot D$. \\
\end{itemize}

Exponential functions do something similar. \\





$\blacktriangleright$ \textbf{\textcolor{blue!55!black}{Exponential Functions}} 



The general template for an exponential function looks like 

\[   exp(t) = a \cdot b^t   \, \text{ where } \,  a \ne 0  \, \text{ and } \,    b > 0   \]



Their defining characteristic is a constant percentage growth rate: no matter where you are in the domain of an exponential function, if you move the same amount then the function increases by the same percent.


\begin{procedure}

In the domain, begin at $d$, then move a distance $D$. \\

\begin{itemize}
\item $exp(d) = a \cdot b^d$
\item $exp(d+D) = a \cdot b^{d+D} = a \cdot b^d \cdot b^D$
\end{itemize}


If you move a distance $D$ in the domain then the function value is multipled by the same factor of $d^D$.






\end{procedure}


\subsection{Learning Outcomes}



\begin{sectionOutcomes}
In this section, students will 

\begin{itemize}
\item analyze exponential functions.
\item analyze logarithmic functions.
\end{itemize}
\end{sectionOutcomes}















\begin{center}
\textbf{\textcolor{green!50!black}{ooooo=-=-=-=-=-=-=-=-=-=-=-=-=ooOoo=-=-=-=-=-=-=-=-=-=-=-=-=ooooo}} \\

more examples can be found by following this link\\ \link[More Examples of Percent Change]{https://ximera.osu.edu/csccmathematics/precalculus1/precalculus1/percentChange/examples/exampleList}

\end{center}








\end{document}
