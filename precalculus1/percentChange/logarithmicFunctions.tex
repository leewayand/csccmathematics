\documentclass{ximera}


\graphicspath{
  {./}
  {ximeraTutorial/}
  {basicPhilosophy/}
}

\newcommand{\mooculus}{\textsf{\textbf{MOOC}\textnormal{\textsf{ULUS}}}}


\usepackage{tkz-euclide}\usepackage{tikz}
\usepackage{tikz-cd}
\usetikzlibrary{arrows}
\tikzset{>=stealth,commutative diagrams/.cd,
  arrow style=tikz,diagrams={>=stealth}} %% cool arrow head
\tikzset{shorten <>/.style={ shorten >=#1, shorten <=#1 } } %% allows shorter vectors

\usetikzlibrary{backgrounds} %% for boxes around graphs
\usetikzlibrary{shapes,positioning}  %% Clouds and stars
\usetikzlibrary{matrix} %% for matrix
\usepgfplotslibrary{polar} %% for polar plots
\usepgfplotslibrary{fillbetween} %% to shade area between curves in TikZ
\usetkzobj{all}
\usepackage[makeroom]{cancel} %% for strike outs
%\usepackage{mathtools} %% for pretty underbrace % Breaks Ximera
%\usepackage{multicol}
\usepackage{pgffor} %% required for integral for loops



%% http://tex.stackexchange.com/questions/66490/drawing-a-tikz-arc-specifying-the-center
%% Draws beach ball
\tikzset{pics/carc/.style args={#1:#2:#3}{code={\draw[pic actions] (#1:#3) arc(#1:#2:#3);}}}



\usepackage{array}
\setlength{\extrarowheight}{+.1cm}
\newdimen\digitwidth
\settowidth\digitwidth{9}
\def\divrule#1#2{
\noalign{\moveright#1\digitwidth
\vbox{\hrule width#2\digitwidth}}}
























%%This is to help with formatting on future title pages.
\newenvironment{sectionOutcomes}{}{}


\title{Logarithmic Functions}

\begin{document}

\begin{abstract}
exponents
\end{abstract}
\maketitle






An example of a basic exponential funciton is $E(t) = 2^t$.  

Its graph looks like







\begin{image}
\begin{tikzpicture}
  \begin{axis}[
            domain=-10:10, ymax=10, xmax=10, ymin=-10, xmin=-10,
            axis lines =center, xlabel=$x$, ylabel=$y$, grid = major,
            ytick={-10,-8,-6,-4,-2,2,4,6,8,10},
          	xtick={-10,-8,-6,-4,-2,2,4,6,8,10},
          	ticklabel style={font=\scriptsize},
            every axis y label/.style={at=(current axis.above origin),anchor=south},
            every axis x label/.style={at=(current axis.right of origin),anchor=west},
            axis on top
          ]
          
      		\addplot [line width=2, penColor, smooth,samples=200,domain=(-10:3.2),<->] {2^x};

          	\addplot [line width=1, gray, dashed,samples=200,domain=(-10:10),<->] {0};


      		\addplot[color=penColor,fill=penColor,only marks,mark=*] coordinates{(1,0)};





           

  \end{axis}
\end{tikzpicture}
\end{image}






We can evalute this function:
\begin{itemize}
\item $E(0) = 2^0 = 2$
\item $E\left(\frac{1}{2}\right) = 2^{\tfrac{1}{2}} = \sqrt{2}$
\item $E(-1) = 2^{-1} = \frac{1}{2}$
\end{itemize}


When we evaluate a function, we know the domain number and we seek its range partner. In this case, we know $d$ and we seek the pair $(d, 2^d)$, so that we can pull out $2^d$ as the value of the function.





\section{Reverse}

We also think in the reverse.
\begin{itemize}
\item $E(t) = 4$
\item $E(t) = {\tfrac{1}{4}} $
\item $E(t) = 16 $
\end{itemize}


Here, we know the value of the function.  We seek the domain numbers paired with it. We know $2^d$, we seek the pair $(d, 2^d)$, so that we can pull out $d$ as the solution to the equation.




$\bigstar$ As long as we are giving a positive function value, we can find the associated domain number. \\ 






































\end{document}
