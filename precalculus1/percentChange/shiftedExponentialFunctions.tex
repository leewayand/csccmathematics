\documentclass{ximera}


\graphicspath{
  {./}
  {ximeraTutorial/}
  {basicPhilosophy/}
}

\newcommand{\mooculus}{\textsf{\textbf{MOOC}\textnormal{\textsf{ULUS}}}}


\usepackage{tkz-euclide}\usepackage{tikz}
\usepackage{tikz-cd}
\usetikzlibrary{arrows}
\tikzset{>=stealth,commutative diagrams/.cd,
  arrow style=tikz,diagrams={>=stealth}} %% cool arrow head
\tikzset{shorten <>/.style={ shorten >=#1, shorten <=#1 } } %% allows shorter vectors

\usetikzlibrary{backgrounds} %% for boxes around graphs
\usetikzlibrary{shapes,positioning}  %% Clouds and stars
\usetikzlibrary{matrix} %% for matrix
\usepgfplotslibrary{polar} %% for polar plots
\usepgfplotslibrary{fillbetween} %% to shade area between curves in TikZ
\usetkzobj{all}
\usepackage[makeroom]{cancel} %% for strike outs
%\usepackage{mathtools} %% for pretty underbrace % Breaks Ximera
%\usepackage{multicol}
\usepackage{pgffor} %% required for integral for loops



%% http://tex.stackexchange.com/questions/66490/drawing-a-tikz-arc-specifying-the-center
%% Draws beach ball
\tikzset{pics/carc/.style args={#1:#2:#3}{code={\draw[pic actions] (#1:#3) arc(#1:#2:#3);}}}



\usepackage{array}
\setlength{\extrarowheight}{+.1cm}
\newdimen\digitwidth
\settowidth\digitwidth{9}
\def\divrule#1#2{
\noalign{\moveright#1\digitwidth
\vbox{\hrule width#2\digitwidth}}}
























%%This is to help with formatting on future title pages.
\newenvironment{sectionOutcomes}{}{}


\title{Shifted Exponential}

\begin{document}

\begin{abstract}
shifted range
\end{abstract}
\maketitle




The formula template for the basic exponential function looks like




\[  a \, r^x   \, \text{ with } \,  a, r \in \mathbb{R} \, | \,  r > 0   \]


As we have seen before, the coefficient $a$ controls vertical stretching or compression. The sign of $a$ dictates the sign of our function values. $r$ dictates a growing or decaying function.


Shifted exponential functions shift the range by adding a constant. \\


\[  a \, r^x  + b \, \text{ with } \,  a, b, r \in \mathbb{R} \, | \,  r > 0   \]




These no longer have a constant percent growth rate.  However, their analysis is exactly the same as for exponential functions with one big difference in our conclusions. Shifted exponential functions may have zeros. \\











\begin{example}  Shifted Exponential Function



Analyze   $f(x) = \frac{1}{3} \, 2^{x+5} - 7$ \\


\begin{explanation}

For the basic exponential function graph, the horizontal axis is the horizontal asymptote.  Here, this has been moved down $7$.



The ``inside'', representing the domain, is $x+5$.  This equals $0$, when $x=-5$.  The exponent is positive for $x>-5$, since the base is $2 > 1$, this is the direction of unbounded growth.  Therefore, the other direction (left) is where the horizontal asymptote is in effect.  Since the coefficient, $\frac{1}{3} > 0$, the unbounded growth is positive.

At $x=-5$, we have our one anchor point for the graph.  The point is $\left(-5, \frac{1}{3} - 7 \right)$, which is $\frac{1}{3}$ above the horizontal asymptote, $y = -7$.


Graph of $y = f(x)$.

\begin{image}
\begin{tikzpicture}
  \begin{axis}[
            domain=-10:10, ymax=10, xmax=10, ymin=-10, xmin=-10,
            axis lines =center, xlabel=$x$, ylabel=$y$, 
            ytick={-10,-8,-6,-4,-2,2,4,6,8,10},
            xtick={-10,-8,-6,-4,-2,2,4,6,8,10},
            ticklabel style={font=\scriptsize},
            every axis y label/.style={at=(current axis.above origin),anchor=south},
            every axis x label/.style={at=(current axis.right of origin),anchor=west},
            axis on top
          ]

          \addplot [line width=1, gray, dashed,samples=200,domain=(-10:10),<->] {0};
          
          \addplot [line width=2, penColor, smooth,samples=200,domain=(-10:-0.2),<->] {0.33 * 2^(x+5)};

          \addplot[color=penColor,fill=penColor,only marks,mark=*] coordinates{(-5,0.333)};

         


 

  \end{axis}
\end{tikzpicture}
\end{image}




Our graph agrees with our analysis.

\begin{itemize}
\item The natural or implied domain of $f$ is $\mathbb{R}$.
\item $f$ is always increasing.
\item $f$ has no maximums or minimums.
\item $\lim\limits_{x \to -\infty} f(x) = -7$
\item $\lim\limits_{x \to \infty} f(x) = \infty$
\end{itemize}




\end{explanation}

\end{example}























\begin{center}
\textbf{\textcolor{green!50!black}{ooooo=-=-=-=-=-=-=-=-=-=-=-=-=ooOoo=-=-=-=-=-=-=-=-=-=-=-=-=ooooo}} \\

more examples can be found by following this link\\ \link[More Examples of Percent Change]{https://ximera.osu.edu/csccmathematics/precalculus1/precalculus1/percentChange/examples/exampleList}

\end{center}





\end{document}
