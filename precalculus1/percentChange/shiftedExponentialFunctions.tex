\documentclass{ximera}


\graphicspath{
  {./}
  {ximeraTutorial/}
  {basicPhilosophy/}
}

\newcommand{\mooculus}{\textsf{\textbf{MOOC}\textnormal{\textsf{ULUS}}}}


\usepackage{tkz-euclide}\usepackage{tikz}
\usepackage{tikz-cd}
\usetikzlibrary{arrows}
\tikzset{>=stealth,commutative diagrams/.cd,
  arrow style=tikz,diagrams={>=stealth}} %% cool arrow head
\tikzset{shorten <>/.style={ shorten >=#1, shorten <=#1 } } %% allows shorter vectors

\usetikzlibrary{backgrounds} %% for boxes around graphs
\usetikzlibrary{shapes,positioning}  %% Clouds and stars
\usetikzlibrary{matrix} %% for matrix
\usepgfplotslibrary{polar} %% for polar plots
\usepgfplotslibrary{fillbetween} %% to shade area between curves in TikZ
\usetkzobj{all}
\usepackage[makeroom]{cancel} %% for strike outs
%\usepackage{mathtools} %% for pretty underbrace % Breaks Ximera
%\usepackage{multicol}
\usepackage{pgffor} %% required for integral for loops



%% http://tex.stackexchange.com/questions/66490/drawing-a-tikz-arc-specifying-the-center
%% Draws beach ball
\tikzset{pics/carc/.style args={#1:#2:#3}{code={\draw[pic actions] (#1:#3) arc(#1:#2:#3);}}}



\usepackage{array}
\setlength{\extrarowheight}{+.1cm}
\newdimen\digitwidth
\settowidth\digitwidth{9}
\def\divrule#1#2{
\noalign{\moveright#1\digitwidth
\vbox{\hrule width#2\digitwidth}}}
























%%This is to help with formatting on future title pages.
\newenvironment{sectionOutcomes}{}{}


\title{Shifted Exponential}

\begin{document}

\begin{abstract}
shifted range
\end{abstract}
\maketitle




The formula template for the basic exponential function looks like




\[  a \, r^x   \, \text{ with } \,  a, r \in \mathbb{R} \, | \,  r > 0   \]


As we have seen before, the coefficient $a$ controls vertical stretching or compression. The sign of $a$ dictates the sign of our function values. $r$ dictates a growing or decaying function.


Shifted exponential functions shift the range by adding a constant. \\


\[  a \, r^x  + b \, \text{ with } \,  a, b, r \in \mathbb{R} \, | \,  r > 0   \]




These no longer have a constant percent growth rate.  However, their analysis is exactly the same as for exponential functions with one big difference in our conclusions. Shifted exponential functions may have zeros. \\











\begin{example}  Shifted Exponential Function



Analyze   $f(x) = \frac{1}{3} \, 2^{x+5} - 7$ \\

Categorize:  $f(x) = \frac{1}{3} \, 2^{x+5} - 7$ is a shifted exponential function, since it matches our official template,  $A \, r^{B \, x + C} + D$\\







\begin{idea}


For the basic exponential function graph, the horizontal axis is the horizontal asymptote.  Here, this has been moved down $7$.



The ``inside'', representing the domain, is $x+5$.  This equals $0$, when $x=-5$.  The exponent is positive for $x>-5$, since the base is $2 > 1$, this is the direction of unbounded growth.  Therefore, the other direction (left) is where the horizontal asymptote is in effect.  Since the coefficient, $\frac{1}{3} > 0$, the unbounded growth is positive. \\


$f$ is the sum of an exponential function, $\frac{1}{3} \, 2^{x+5}$, and a constant function, $-7$.  The exponential part will approach $0$ on one side of the domain.  Therefore, $f$ will approach $-7$ on that side of the domain. \\





At $x=-5$, we have our one anchor point for the graph.  The point is $\left(-5, \frac{1}{3} - 7 \right)$, which is $\frac{1}{3}$ above the horizontal asymptote, $y = -7$.


Graph of $y = f(x)$.

\begin{image}
\begin{tikzpicture}
  \begin{axis}[
            domain=-10:10, ymax=10, xmax=10, ymin=-10, xmin=-10,
            axis lines =center, xlabel=$x$, ylabel=$y$, 
            ytick={-10,-8,-6,-4,-2,2,4,6,8,10},
            xtick={-10,-8,-6,-4,-2,2,4,6,8,10},
            ticklabel style={font=\scriptsize},
            every axis y label/.style={at=(current axis.above origin),anchor=south},
            every axis x label/.style={at=(current axis.right of origin),anchor=west},
            axis on top
          ]

          \addplot [line width=1, gray, dashed,samples=200,domain=(-10:10),<->] {-7};
          
          \addplot [line width=2, penColor, smooth,samples=200,domain=(-10:0.62),<->] {0.33 * 2^(x+5)-7};

          \addplot[color=penColor,fill=penColor,only marks,mark=*] coordinates{(-5,-6.666)};

         


 

  \end{axis}
\end{tikzpicture}
\end{image}


With these ideas, we can create an algebraic analysis.

\end{idea}




\textbf{Domain}

$f$ is a shifted exponential function, which tells us that its domain is $(-\infty, \infty)$. \\


\textbf{Zeros}

$f$ is a shifted exponential function, therefore it might have a zero. \\




\[ f(x) = \frac{1}{3} \, 2^{x+5} - 7 = 0 \]


\[ \frac{1}{3} \, 2^{x+5}  = 7 \]

\[ 2^{x+5}  = 3 \cdot 7 = 21 \]

\[ x+5 = \log_2(21) \]

\[ x = \log_2(21) - 5 \]


\textbf{Note:} $\log_2(21) - 5 \approx -0.6076825772$, which agrees with the graph. \\





\textbf{Continuity}

$f$ is a shifted exponential function, therefore it is continuous. \\




\textbf{Behavior (Increasing and Decreasing)}


\begin{itemize}
\item The base is $2$, which is greater than $1$.
\item The leading coeffcient is $\frac{1}{3} > 0$.
\item The leading coeffcient of the linear exponent is $1 > 0$.
\end{itemize}

That tells us that $f$ is increasing.





\textbf{End-Behavior}

$f$ is a shifted exponential function, therefore the end-behavior of one side is the constant term, $-7$ and the other is unbounded. The leading coefficient is $\frac{1}{3}$, which tells is that when $f$ becomes unbounded, it will become unbounded positively.  \\


$f$ is increasing and is unbounded positively.  That gives us


\[ \lim\limits_{x \to -\infty} f(x) = -7 \]

\[ \lim\limits_{x \to \infty} f(x) = \infty\]









\textbf{Local Maximum and Minimum}

$f$ is a shifted exponential function, therefore it has no local extrema. \\





\textbf{Global Maximum and Minimum}

$f$ is a shifted exponential function, therefore it has no global extrema. \\





\textbf{Range}

 
\begin{itemize}
\item $f$ is continuous
\item $f$ is increasing
\item $\lim\limits_{x \to -\infty} f(x) = -7$
\item $\lim\limits_{x \to \infty} f(x) = \infty$
\end{itemize}



The range of $(-7, \infty)$. \\









\end{example}





















\begin{example}  Shifted Exponential Function



Analyze   $B(t) = -2 \, \left( \frac{2}{3} \right)^{3-t} + 4$ \\


\begin{idea}


Categorize: $B(t) = -2 \, \left( \frac{2}{3} \right)^{3-t} + 4$  is a shifted exponential function, since it matches our template, $A \, r^{B \, x + C} +D$ \\




First, observe that the base is $\frac{2}{3} < \answer{1}$.


Our base is less than $1$.  Therefore, as its exponent gets large and positive, we multiply by more $\frac{2}{3}$'s and the overall values get smaller.


Except, the variable, $t$, in the exponent is multiplied by $-1$.  Therefore, we need $t$ to get large and negative in order for the exponent to get large and positve.


\begin{itemize}
\item $\left( \frac{2}{3} \right)^{3-t}$ decays when $t$ becomes more negative.
\item $\left( \frac{2}{3} \right)^{3-t}$ grows when $t$ becomes more positive.
\end{itemize}







\begin{model}

The exponential stem of $B(t)$ is $\left( \frac{2}{3} \right)^{-t}$, which is a transformed version of the basic exponential function model $M(t) = \left( \frac{2}{3} \right)^{t}$.  



When $t < 0$, then $-t > 0$ and we get  $\left( \frac{2}{3} \right)^{-t} = \left( \frac{2}{3} \right)^{positive}$ and the stem is becoming smaller, approaching $0$.  





\[ \lim\limits_{t \to -\infty} \left( \frac{2}{3} \right)^{-t} = 0 \]



When $t > 0$, then $-t < 0$ and we get  $\left( \frac{2}{3} \right)^{-t} = \left( \frac{2}{3} \right)^{negative}$ and the stem is becoming larger.  



\[ \lim\limits_{t \to \infty} \left( \frac{2}{3} \right)^{-t} = \infty \]








\end{model}




$\blacktriangleright$ \textbf{\textcolor{blue!55!black}{Graphing}} 






Adding $4$ to the outside shifts the graph vertically up $4$.  The asymptote is $y = 4$ and 

\[ \lim\limits_{t \to -\infty} B(t) = 4 \]





Our exponent is $3 - t = -t + 3$.  Our anchor point for graphing is associated with the exponent equalling $0$.



$3-t=0$ when $t=3$. Our one anchor point is shifted over to $3$.  Multipying by $-2$, means the dot is $2$ away from the horizontal asymptote, which is now $y=4$.





Graph of $y = B(t)$.

\begin{image}
\begin{tikzpicture}
  \begin{axis}[
            domain=-10:10, ymax=10, xmax=10, ymin=-10, xmin=-10,
            axis lines =center, xlabel=$t$, ylabel=$y$, 
            ytick={-10,-8,-6,-4,-2,2,4,6,8,10},
            xtick={-10,-8,-6,-4,-2,2,4,6,8,10},
            ticklabel style={font=\scriptsize},
            every axis y label/.style={at=(current axis.above origin),anchor=south},
            every axis x label/.style={at=(current axis.right of origin),anchor=west},
            axis on top
          ]
          
          \addplot [line width=1, gray, dashed,samples=200,domain=(-10:10),<->] {4};

          \addplot [line width=2, penColor, smooth,samples=200,domain=(-10:7.5),<->] {-2 * (0.666^(3-x)) + 4};

          \addplot[color=penColor,fill=penColor,only marks,mark=*] coordinates{(3,2)};

          


  \end{axis}
\end{tikzpicture}
\end{image}






With these ideas, we can create an algebraic analysis. \\



\end{idea}










\textbf{Domain}

$B$ is a shifted exponential function, therefore its domain is $(-\infty, \infty)$. \\




\textbf{Zeros}

$B$ is a shifted exponential function, therefore it might have a zero. \\




\[ B(t) = -2 \, \left( \frac{2}{3} \right)^{3-t} + 4 = 0 \]


\[ -2 \, \left( \frac{2}{3} \right)^{3-t} = -4 \]

\[ \left( \frac{2}{3} \right)^{3-t} = 2 \]

\[ 3 - t = \log_{\tfrac{2}{3}}(2) \]

\[ 3 - \log_{\tfrac{2}{3}}(2) = t \]


\textbf{Note:} $3 - \log_{\tfrac{2}{3}}(2) \approx -3.584962501$, which agrees with the graph. \\







\textbf{Continuity}

$B$ is a shifted exponential function, therefore it is continuous. \\






\textbf{End-Behavior}

$B$ is a shifted exponential function, therefore on one side the end-behavior is the constant term $4$ and unbounded on the other side. \\


Since the leading coefficient is $-2 < 0$, we know $B$ will be unbounded negatively.  We just need to figure out which side. \\


\begin{itemize}
  \item the base of $B$ is $\frac{2}{3} < 1$
  \item the leading coefficient of $B$ is $-2$, which is negative and tells us that $B$ will become unbounded negatively.
  \item the leading coefficient of the linear exponent is $-1$, which is negative.
\end{itemize}


This makes $B$ a decreasing function, which becomes unbounded negatively.  That gives us \\


\[ \lim\limits_{t \to -\infty} B(t) = 4 \]

\[ \lim\limits_{t \to \infty} B(t) = -\infty \]




\textbf{Behavior (Increasing and  Decreasing)} \\


\begin{itemize}
  \item the base of $B$ is $\frac{2}{3} < 1$
  \item the leading coefficient of $B$ is $-2$, which is negative.
  \item the leading coefficient of the linear exponent is $-1$, which is negative.
\end{itemize}


This makes $B$ a decreasing function. \\ 






\textbf{Local Maximum and Minimum}

$B$ is a shifted exponential function, therefore it has no local extrema. \\





\textbf{Global Maximum and Minimum}

$B$ is a shifted exponential function, therefore it has no global extrema. \\





\textbf{Range}

 
\begin{itemize}
\item $B$ is continuous
\item $\lim\limits_{t \to -\infty} B(t) = 4$
\item $\lim\limits_{t \to \infty} B(t) = -\infty$
\end{itemize}


The range is $(-\infty, 4)$. \\





The graph agrees with our analysis. \\



\end{example}






















\begin{example}  Shifted Exponential Function



Analyze   $K(f) = 3^{5-f} - 5$ \\

\begin{idea}

\begin{question}

The formula for $K$ matches which template?

\begin{multipleChoice}
\choice{$A \, x + B$}
\choice{$A \, x^2 + B \, x + C$}
\choice{$A \, |B \, x + C | + D$}
\choice{$A \, r^{B \, x + C}$}
\choice[correct]{$A \, r^{B \, x + C} + D$}
\choice{$A \, \ln(B \, x + C) + D$}
\end{multipleChoice}


\end{question}


\begin{question}


The exponent gets big and positive when $f$ gets big and \wordChoice{\choice{positive}\choice[correct]{negative}}.
\end{question}
\begin{question}


The graph will grow to the \wordChoice{\choice{right}\choice[correct]{left}}.\\
\end{question}
\begin{question}.


The graph will approach the asymptote to the \wordChoice{\choice[correct]{right}\choice{left}}.\\
\end{question}
\begin{question}


Our one anchor point moves to $\left(\answer{5}, \answer{-4}\right)$.
\end{question}
\begin{question}


The graph will become unbounded \wordChoice{\choice[correct]{up}\choice{down}}.\\
\end{question}




Graph of $y = K(f)$.

\begin{image}
\begin{tikzpicture}
  \begin{axis}[
            domain=-10:10, ymax=10, xmax=10, ymin=-10, xmin=-10,
            axis lines =center, xlabel=$f$, ylabel=$y$, 
            ytick={-10,-8,-6,-4,-2,2,4,6,8,10},
            xtick={-10,-8,-6,-4,-2,2,4,6,8,10},
            ticklabel style={font=\scriptsize},
            every axis y label/.style={at=(current axis.above origin),anchor=south},
            every axis x label/.style={at=(current axis.right of origin),anchor=west},
            axis on top
          ]
          
          \addplot [line width=2, penColor, smooth,samples=200,domain=(2.75:10),<->] {3^(5-x) - 5};

          \addplot[color=penColor,fill=penColor,only marks,mark=*] coordinates{(5,-4)};

          \addplot [line width=1, gray, dashed,samples=200,domain=(-10:10),<->] {-5};



           

  \end{axis}
\end{tikzpicture}
\end{image}


With these ideas, we can create an algebraic analysis. \\


\end{idea}





\textbf{Domain}

The natural or implied domain of $K$ is $\mathbb{R}$, because $K$ is a shifted exponential function. \\


\textbf{Zeros}

$B$ is a shifted exponential function, so it might have a zero. \\





\[ B(t) = 3^{5 - f} - 5 = 0 \]


\[ 3^{5 - f} = 5 \]

\[ 5 - f = \log_3(5) \]

\[ 5 - \log_3(5) = f \]


\textbf{Note:}  $5 - \log_3(5) \approx 3.535026479$, which agrees with the graph. \\




\textbf{Continuity}

$K$ is continuous, because $K$ is a shifted exponential function. \\





\textbf{End-Behavior}


\begin{itemize}
  \item The base is $3 > 1$
  \item The leading coefficient is $1$, which is positive
  \item The leading coefficient of the linear exponent is $-1$, which is negative
\end{itemize}


The base is greater than $1$ and the leading coeffcients are of opposite sign.  That tells us that $K$ is decreasing.

The positive leading coefficient tells us that $K$ becomes unbounded positively. We just need to figure out which side.  The end-behavior on the other side is the constant term, $-5$.


$K$ is a decreasing shifted exponential function that becomes unbounded positively. \\


\[ \lim\limits_{f \to -\infty} K(f) = \infty \]
\[ \lim\limits_{f \to \infty} K(f) = -5 \]



\textbf{Behavior (Increasing and Decreasing)}


The base is greater than $1$ and the leading coeffcients are of opposite sign.  That tells us that $K$ is decreasing.



\textbf{Global Maximum and Minimum}

Shifted exponential functions do not have global maximum or minimum values.





\textbf{Local Maximum and Minimum}

Shifted exponential functions do not have local maximum or minimum values.



\textbf{Range}

\begin{itemize}
\item $K$ is continuous
\item $K$ is decreasing.
\item $\lim\limits_{f \to -\infty} K(f) = \infty $
\item $\lim\limits_{f \to \infty} K(f) = 0$
\end{itemize}


The range is $(0, \infty)$. \\







This all agrees with the graph. \\




\end{example}
















\begin{center}
\textbf{\textcolor{green!50!black}{ooooo-=-=-=-ooOoo-=-=-=-ooooo}} \\

more examples can be found by following this link\\ \link[More Examples of Percent Change]{https://ximera.osu.edu/csccmathematics/precalculus1/precalculus1/percentChange/examples/exampleList}

\end{center}





\end{document}
