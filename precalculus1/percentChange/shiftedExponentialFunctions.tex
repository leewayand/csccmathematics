\documentclass{ximera}


\graphicspath{
  {./}
  {ximeraTutorial/}
  {basicPhilosophy/}
}

\newcommand{\mooculus}{\textsf{\textbf{MOOC}\textnormal{\textsf{ULUS}}}}


\usepackage{tkz-euclide}\usepackage{tikz}
\usepackage{tikz-cd}
\usetikzlibrary{arrows}
\tikzset{>=stealth,commutative diagrams/.cd,
  arrow style=tikz,diagrams={>=stealth}} %% cool arrow head
\tikzset{shorten <>/.style={ shorten >=#1, shorten <=#1 } } %% allows shorter vectors

\usetikzlibrary{backgrounds} %% for boxes around graphs
\usetikzlibrary{shapes,positioning}  %% Clouds and stars
\usetikzlibrary{matrix} %% for matrix
\usepgfplotslibrary{polar} %% for polar plots
\usepgfplotslibrary{fillbetween} %% to shade area between curves in TikZ
\usetkzobj{all}
\usepackage[makeroom]{cancel} %% for strike outs
%\usepackage{mathtools} %% for pretty underbrace % Breaks Ximera
%\usepackage{multicol}
\usepackage{pgffor} %% required for integral for loops



%% http://tex.stackexchange.com/questions/66490/drawing-a-tikz-arc-specifying-the-center
%% Draws beach ball
\tikzset{pics/carc/.style args={#1:#2:#3}{code={\draw[pic actions] (#1:#3) arc(#1:#2:#3);}}}



\usepackage{array}
\setlength{\extrarowheight}{+.1cm}
\newdimen\digitwidth
\settowidth\digitwidth{9}
\def\divrule#1#2{
\noalign{\moveright#1\digitwidth
\vbox{\hrule width#2\digitwidth}}}
























%%This is to help with formatting on future title pages.
\newenvironment{sectionOutcomes}{}{}


\title{Shifted Exponential}

\begin{document}

\begin{abstract}
shifted range
\end{abstract}
\maketitle




The formula template for the basic exponential function looks like




\[  a \, r^x   \, \text{ with } \,  a, r \in \mathbb{R} \, | \,  r > 0   \]


As we have seen before, the coefficient $a$ controls vertical stretching or compression. The sign of $a$ dictates the sign of our function values. $r$ dictates a growing or decaying function.


Shifted exponential functions shift the range by adding a constant. \\


\[  a \, r^x  + b \, \text{ with } \,  a, b, r \in \mathbb{R} \, | \,  r > 0   \]




These no longer have a constant percent growth rate.  However, their analysis is exactly the same as for exponential functions with one big difference in our conclusions. Shifted exponential functions may have zeros. \\









\begin{example}  Shifted Exponential Function



Analyze   $f(x) = \frac{1}{3} \, 2^{x+5} - 7$ \\


\begin{explanation}

For the basic exponential function graph, the horizontal axis is the horizontal asymptote.  Here, this has been moved down $7$.



The ``inside'', representing the domain, is $x+5$.  This equals $0$, when $x=-5$.  The exponent is positive for $x>-5$, since the base is $2 > 1$, this is the direction of unbounded growth.  Therefore, the other direction (left) is where the horizontal asymptote is in effect.  Since the coefficient, $\frac{1}{3} > 0$, the unbounded growth is positive.

At $x=-5$, we have our one anchor point for the graph.  The point is $\left(-5, \frac{1}{3} - 7 \right)$, which is $\frac{1}{3}$ above the horizontal asymptote, $y = -7$.


Graph of $y = f(x)$.

\begin{image}
\begin{tikzpicture}
  \begin{axis}[
            domain=-10:10, ymax=10, xmax=10, ymin=-10, xmin=-10,
            axis lines =center, xlabel=$x$, ylabel=$y$, 
            ytick={-10,-8,-6,-4,-2,2,4,6,8,10},
          	xtick={-10,-8,-6,-4,-2,2,4,6,8,10},
          	ticklabel style={font=\scriptsize},
            every axis y label/.style={at=(current axis.above origin),anchor=south},
            every axis x label/.style={at=(current axis.right of origin),anchor=west},
            axis on top
          ]

          \addplot [line width=1, gray, dashed,samples=200,domain=(-10:10),<->] {-7};
          
      		\addplot [line width=2, penColor, smooth,samples=200,domain=(-10:0.6),<->] {0.33 * 2^(x+5)-7};

      		\addplot[color=penColor,fill=penColor,only marks,mark=*] coordinates{(-5,-6.66)};

         


 

  \end{axis}
\end{tikzpicture}
\end{image}




Our graph agrees with our analysis.

\begin{itemize}
\item The natural or implied domain of $f$ is $\mathbb{R}$.
\item $f$ is always increasing.
\item $f$ has no maximums or minimums.
\item $\lim\limits_{x \to -\infty} f(x) = -7$
\item $\lim\limits_{x \to \infty} f(x) = \infty$
\end{itemize}




\end{explanation}

\end{example}
























\begin{example}  Shifted Exponential Function



Analyze   $B(t) = -2 \, \left( \frac{2}{3} \right)^{3-t} + 4$ \\


\begin{observation}


First, observe that $\frac{2}{3} < \answer{1}$.


Our base is less than $1$.  Therefore, as its exponent gets large and positive, we multiply by more $\frac{2}{3}$'s and the overall values get smaller.


Except, the variable, $t$, in the exponent is multiplied by $-1$.  Therefore, we need $t$ to get large and negative in order for the exponent to get large and positve.


\begin{itemize}
\item $\left( \frac{2}{3} \right)^{3-t}$ decays when $t$ becomes more negative.
\item $\left( \frac{2}{3} \right)^{3-t}$ grows when $t$ becomes more positive.
\end{itemize}





\end{observation}



\begin{explanation}




\begin{model}

The exponential stem of $B(t)$ is $\left( \frac{2}{3} \right)^{-t}$, which is a transformed version of the basic exponential function model $M(t) = \left( \frac{2}{3} \right)^{t}$.  



When $t < 0$, then $-t > 0$ and we get  $\left( \frac{2}{3} \right)^{-t} = \left( \frac{2}{3} \right)^{positive}$ and the stem is becoming smaller, approaching $0$.  





\[ \lim\limits_{t \to -\infty} \left( \frac{2}{3} \right)^{-t} = 0 \]



When $t > 0$, then $-t < 0$ and we get  $\left( \frac{2}{3} \right)^{-t} = \left( \frac{2}{3} \right)^{negative}$ and the stem is becoming larger.  



\[ \lim\limits_{t \to \infty} \left( \frac{2}{3} \right)^{-t} = \infty \]








\end{model}

Next, we have a negative leading coefficient. \\

Since $-2 < 0$, the values of $-2 \, \left( \frac{2}{3} \right)^{power}$ are always negative.



Finally, we also have two shifts:



$\blacktriangleright$ \textbf{\textcolor{blue!55!black}{Vertical Shift}} 

Adding $4$ to the outside shifts the graph vertically up $4$.  The asymptote is $y = 4$ and 

\[ \lim\limits_{t \to -\infty} B(t) = 4 \]




$\blacktriangleright$ \textbf{\textcolor{blue!55!black}{Horizontal Shift}} 

Our exponent is $3 - t = -t + 3$.  Our anchor point for graphing is associated with the exponent equalling $0$.



$3-t=0$ when $t=3$. Our one anchor point is shifted over to $3$.  Multipying by $-2$, means the dot is $2$ away from the horizontal asymptote, which is now $y=4$.










Graph of $y = B(t)$.

\begin{image}
\begin{tikzpicture}
  \begin{axis}[
            domain=-10:10, ymax=10, xmax=10, ymin=-10, xmin=-10,
            axis lines =center, xlabel=$t$, ylabel=$y$, 
            ytick={-10,-8,-6,-4,-2,2,4,6,8,10},
          	xtick={-10,-8,-6,-4,-2,2,4,6,8,10},
          	ticklabel style={font=\scriptsize},
            every axis y label/.style={at=(current axis.above origin),anchor=south},
            every axis x label/.style={at=(current axis.right of origin),anchor=west},
            axis on top
          ]
          
      		\addplot [line width=1, gray, dashed,samples=200,domain=(-10:10),<->] {4};

          \addplot [line width=2, penColor, smooth,samples=200,domain=(-10:7.5),<->] {-2 * (0.666^(3-x)) + 4};

      		\addplot[color=penColor,fill=penColor,only marks,mark=*] coordinates{(3,2)};

          


  \end{axis}
\end{tikzpicture}
\end{image}




Our graphical analysis tells us that 

\begin{itemize}
\item The natural or implied domain of $B$ is $\mathbb{R}$.
\item $B$ is always decreasing.
\item $B$ has no maximums or minimums.
\item $\lim\limits_{t \to -\infty} B(t) = 4$
\item $\lim\limits_{t \to \infty} B(t) = -\infty$
\end{itemize}


\end{explanation}

\end{example}





















\begin{example}  Shifted Exponential Function



Analyze   $K(f) = 3^{5-f} - 5$ \\


\begin{question}. 

The exponent gets big and positive when $f$ gets big and \wordChoice{\choice{positive}\choice[correct]{negative}}.
\end{question}
\begin{question}. 

The graph will grow to the \wordChoice{\choice{right}\choice[correct]{left}}.\\
\end{question}
\begin{question}. 

The graph will approach the asymptote to the \wordChoice{\choice[correct]{right}\choice{left}}.\\
\end{question}
\begin{question}. 

Our one anchor point moves to $\left(\answer{5}, \answer{-4}\right)$.
\end{question}
\begin{question}. 

The graph will become unbounded \wordChoice{\choice[correct]{up}\choice{down}}.\\
\end{question}




Graph of $y = K(f)$.

\begin{image}
\begin{tikzpicture}
  \begin{axis}[
            domain=-10:10, ymax=10, xmax=10, ymin=-10, xmin=-10,
            axis lines =center, xlabel=$f$, ylabel=$y$, 
            ytick={-10,-8,-6,-4,-2,2,4,6,8,10},
          	xtick={-10,-8,-6,-4,-2,2,4,6,8,10},
          	ticklabel style={font=\scriptsize},
            every axis y label/.style={at=(current axis.above origin),anchor=south},
            every axis x label/.style={at=(current axis.right of origin),anchor=west},
            axis on top
          ]
          
      		\addplot [line width=2, penColor, smooth,samples=200,domain=(2.75:10),<->] {3^(5-x) - 5};

      		\addplot[color=penColor,fill=penColor,only marks,mark=*] coordinates{(5,-4)};

          \addplot [line width=1, gray, dashed,samples=200,domain=(-10:10),<->] {-5};



           

  \end{axis}
\end{tikzpicture}
\end{image}





Our graphical analysis tells us that

\begin{itemize}
\item The natural or implied domain of $K$ is $\mathbb{R}$.
\item $K$ is always decreasing.
\item $K$ has no maximums or minimums.
\item $\lim\limits_{f \to -\infty} K(f) = \infty$
\item $\lim\limits_{f \to \infty} K(f) = -5$
\end{itemize}


\end{example}












\begin{center}
\textbf{\textcolor{green!50!black}{ooooo=-=-=-=-=-=-=-=-=-=-=-=-=ooOoo=-=-=-=-=-=-=-=-=-=-=-=-=ooooo}} \\

more examples can be found by following this link\\ \link[More Examples of Percent Change]{https://ximera.osu.edu/csccmathematics/precalculus1/precalculus1/percentChange/examples/exampleList}

\end{center}





\end{document}
