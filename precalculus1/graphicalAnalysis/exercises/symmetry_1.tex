\documentclass{ximera}


\graphicspath{
  {./}
  {ximeraTutorial/}
  {basicPhilosophy/}
}

\newcommand{\mooculus}{\textsf{\textbf{MOOC}\textnormal{\textsf{ULUS}}}}


\usepackage{tkz-euclide}\usepackage{tikz}
\usepackage{tikz-cd}
\usetikzlibrary{arrows}
\tikzset{>=stealth,commutative diagrams/.cd,
  arrow style=tikz,diagrams={>=stealth}} %% cool arrow head
\tikzset{shorten <>/.style={ shorten >=#1, shorten <=#1 } } %% allows shorter vectors

\usetikzlibrary{backgrounds} %% for boxes around graphs
\usetikzlibrary{shapes,positioning}  %% Clouds and stars
\usetikzlibrary{matrix} %% for matrix
\usepgfplotslibrary{polar} %% for polar plots
\usepgfplotslibrary{fillbetween} %% to shade area between curves in TikZ
\usetkzobj{all}
\usepackage[makeroom]{cancel} %% for strike outs
%\usepackage{mathtools} %% for pretty underbrace % Breaks Ximera
%\usepackage{multicol}
\usepackage{pgffor} %% required for integral for loops



%% http://tex.stackexchange.com/questions/66490/drawing-a-tikz-arc-specifying-the-center
%% Draws beach ball
\tikzset{pics/carc/.style args={#1:#2:#3}{code={\draw[pic actions] (#1:#3) arc(#1:#2:#3);}}}



\usepackage{array}
\setlength{\extrarowheight}{+.1cm}
\newdimen\digitwidth
\settowidth\digitwidth{9}
\def\divrule#1#2{
\noalign{\moveright#1\digitwidth
\vbox{\hrule width#2\digitwidth}}}
























%%This is to help with formatting on future title pages.
\newenvironment{sectionOutcomes}{}{}


\outcome{outcome.}
\outcome{outcome.}
\outcome{outcome.}

\author{Lee Wayand}

\begin{document}












Cosine is an even function.  The graph of $y= cos(\theta)$ is displayed below. 

\begin{image}
\begin{tikzpicture}
     \begin{axis}[
                domain=-10:10, ymax=3, xmax=10, ymin=-3, xmin=-10,
                axis lines =center, xlabel={$\theta$}, ylabel=$y$,
                ytick={-2,-1,1,2},
                xtick={-10,-8,-6,-4,-2,2,4,6,8,10},
                ticklabel style={font=\scriptsize},
                every axis y label/.style={at=(current axis.above origin),anchor=south},
                every axis x label/.style={at=(current axis.right of origin),anchor=west},
                axis on top,
                ]

        
        %\addplot [draw=penColor, very thick, smooth, domain=(-3.7, 3.7), <->] {x};
        \addplot [draw=penColor, very thick, smooth, samples=300, domain=(-8.3:8.3), <->] {cos(deg(x)};
        %\addplot [draw=penColor, very thick, smooth, domain=(4:8)] {-2*x+10};


        %\addplot[color=penColor,fill=penColor,only marks,mark=*] coordinates{(-8,-3.3)};
        %\addplot[color=penColor,fill=white,only marks,mark=*] coordinates{(-3,3)};


    \end{axis}
\end{tikzpicture}
\end{image}



We can shift the graph of $y = cos(x)$ by adding a constant to the "inside" of the formula.



\begin{center}
\desmos{6pgzbsv2px}{400}{300}
\end{center}







\begin{exercise}  

Which shifting values will return the graph to the graph of $y=cos(x)$?

\begin{selectAll}
\choice {$\frac{\pi}{3}$}
\choice {$\frac{\pi}{2}$}
\choice {$\frac{5\pi}{6}$}
\choice {$\pi$}
\choice {$\frac{3\pi}{2}$}
\choice {$\frac{7\pi}{4}$}
\choice [correct]{$2\pi$}
\end{selectAll}


\end{exercise}








\begin{exercise}  

For which shifting values is $cos(x + p)$ an odd function?

\begin{selectAll}
\choice {$\frac{\pi}{3}$}
\choice [correct]{$\frac{\pi}{2}$}
\choice {$\frac{5\pi}{6}$}
\choice {$\pi$}
\choice [correct]{$\frac{3\pi}{2}$}
\choice {$\frac{7\pi}{4}$}
\choice {$2\pi$}
\end{selectAll}


\end{exercise}










\end{document}