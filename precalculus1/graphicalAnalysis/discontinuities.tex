\documentclass{ximera}


\graphicspath{
  {./}
  {ximeraTutorial/}
  {basicPhilosophy/}
}

\newcommand{\mooculus}{\textsf{\textbf{MOOC}\textnormal{\textsf{ULUS}}}}


\usepackage{tkz-euclide}\usepackage{tikz}
\usepackage{tikz-cd}
\usetikzlibrary{arrows}
\tikzset{>=stealth,commutative diagrams/.cd,
  arrow style=tikz,diagrams={>=stealth}} %% cool arrow head
\tikzset{shorten <>/.style={ shorten >=#1, shorten <=#1 } } %% allows shorter vectors

\usetikzlibrary{backgrounds} %% for boxes around graphs
\usetikzlibrary{shapes,positioning}  %% Clouds and stars
\usetikzlibrary{matrix} %% for matrix
\usepgfplotslibrary{polar} %% for polar plots
\usepgfplotslibrary{fillbetween} %% to shade area between curves in TikZ
\usetkzobj{all}
\usepackage[makeroom]{cancel} %% for strike outs
%\usepackage{mathtools} %% for pretty underbrace % Breaks Ximera
%\usepackage{multicol}
\usepackage{pgffor} %% required for integral for loops



%% http://tex.stackexchange.com/questions/66490/drawing-a-tikz-arc-specifying-the-center
%% Draws beach ball
\tikzset{pics/carc/.style args={#1:#2:#3}{code={\draw[pic actions] (#1:#3) arc(#1:#2:#3);}}}



\usepackage{array}
\setlength{\extrarowheight}{+.1cm}
\newdimen\digitwidth
\settowidth\digitwidth{9}
\def\divrule#1#2{
\noalign{\moveright#1\digitwidth
\vbox{\hrule width#2\digitwidth}}}
























%%This is to help with formatting on future title pages.
\newenvironment{sectionOutcomes}{}{}


\title{Discontinuities}

\begin{document}

\begin{abstract}
breaks
\end{abstract}
\maketitle



The graphs we have seen so far clearly show all kinds of breaks.  The function represented by the graph is exhibiting some drastic behavior around some individual domain numbers.  Let's examine some of these breaks and classify them into a few types.







\section{Basic Types of Breaks}

A quick review of graphs seen so far in this course shows there are three basic types of breaks.




\begin{image}
\begin{tikzpicture}
     \begin{axis}[
            	domain=-10:10, ymax=10, xmax=10, ymin=-10, xmin=-10,
            	axis lines =center, xlabel=$x$, ylabel=$y$,
            	every axis y label/.style={at=(current axis.above origin),anchor=south},
            	every axis x label/.style={at=(current axis.right of origin),anchor=west},
            	axis on top,
          		]

        
        \addplot [draw=penColor, very thick, smooth, domain=(-7.5:3), <->] {1/(x-3) + 2};
        \addplot [draw=penColor, very thick, smooth, domain=(3:6), <-] {1/(x-3) + 2};
        \addplot [draw=penColor, very thick, smooth, domain=(6:8), ->] {2*x-14};

        \addplot [line width=1, gray, dashed,samples=100,domain=(-9:9)] ({3},{x});
        \addplot [line width=1, gray, dashed,samples=100,domain=(-9:4)] ({x},{2});

        \addplot[color=penColor,fill=penColor,only marks,mark=*] coordinates{(-6,5)};
        \addplot[color=penColor,fill=white,only marks,mark=*] coordinates{(-6,1.88)};
        \addplot[color=penColor,fill=penColor,only marks,mark=*] coordinates{(6,2.33)};
        \addplot[color=penColor,fill=white,only marks,mark=*] coordinates{(6,-2)};

    \end{axis}



\end{tikzpicture}
\end{image}



\begin{itemize}
\item \textbf{Removeable:}  The graph has a very subtle break at $x=-6$. The graph is almost continuous, except one point has been shifted up. This type of break gets the name removeable, because you could remove the break by just sliding that point back in place.

\item \textbf{Asymptotic:}  The break at $x=3$ is not subtle at all.  One side is heading up to infinty and the other side is heading down to negative infinity.  You can get a bigger break than that. Since this type of behavior is due to the asymptote, the break is called an asymptotic break.

\item \textbf{Jump:} The break at $x=6$ is called a jump, because the graph makes a sudden jump from one location to another location.
\end{itemize}









\section{Discontinuities}
One of the strongest purposes of mathematics is simply communication.  We like mathematics to be explicit and this is supported by lots of language. As we investigate breaks in the graphs of functions, we want language that is explicit, so that we can communicate about the underlying function behavior.

With this in mind, let's compare the following two graphs.


\begin{image}
\begin{tikzpicture}
    \begin{axis}[name = without,
            	domain=-10:10, ymax=10, xmax=10, ymin=-10, xmin=-10,
            	axis lines =center, xlabel=$x$, ylabel=$y$,
            	every axis y label/.style={at=(current axis.above origin),anchor=south},
            	every axis x label/.style={at=(current axis.right of origin),anchor=west},
            	axis on top,
          		]

         \addplot [draw=penColor, very thick, smooth, domain=(-7.5:3), <->] {1/(x-3) + 2};
        \addplot [draw=penColor, very thick, smooth, domain=(3:6), <-] {1/(x-3) + 2};
        \addplot [draw=penColor, very thick, smooth, domain=(6:8), ->] {2*x-14};

        \addplot [line width=1, gray, dashed,samples=100,domain=(-9:9)] ({3},{x});
        \addplot [line width=1, gray, dashed,samples=100,domain=(-9:4)] ({x},{2});

        \addplot[color=penColor,fill=penColor,only marks,mark=*] coordinates{(-6,5)};
        \addplot[color=penColor,fill=white,only marks,mark=*] coordinates{(-6,1.88)};
        \addplot[color=penColor,fill=penColor,only marks,mark=*] coordinates{(6,2.33)};
        \addplot[color=penColor,fill=white,only marks,mark=*] coordinates{(6,-2)};
    \end{axis}





     \begin{axis}[
            	at={(without.outer east)}, anchor=outer west, domain=-10:10, ymax=10, xmax=10, ymin=-10, xmin=-10,
            	axis lines =center, xlabel=$x$, ylabel=$y$,
            	every axis y label/.style={at=(current axis.above origin),anchor=south},
            	every axis x label/.style={at=(current axis.right of origin),anchor=west},
            	axis on top,
          		]

        \addplot [draw=penColor, very thick, smooth, domain=(-7.5:3), <->] {1/(x-3) + 2};
        \addplot [draw=penColor, very thick, smooth, domain=(3:6), <-] {1/(x-3) + 2};
        \addplot [draw=penColor, very thick, smooth, domain=(6:8), ->] {2*x-14};

        \addplot [line width=1, gray, dashed,samples=100,domain=(-9:9)] ({3},{x});
        \addplot [line width=1, gray, dashed,samples=100,domain=(-9:4)] ({x},{2});

        \addplot[color=penColor,fill=penColor,only marks,mark=*] coordinates{(-6,5)};
        \addplot[color=penColor,fill=white,only marks,mark=*] coordinates{(-6,1.88)};
        \addplot[color=penColor,fill=penColor,only marks,mark=*] coordinates{(3,-5)};
        \addplot[color=penColor,fill=white,only marks,mark=*] coordinates{(6,2.33)};
        \addplot[color=penColor,fill=white,only marks,mark=*] coordinates{(6,-2)};
    \end{axis}



\end{tikzpicture}
\end{image}


These two graphs (and the underlying functions) are almost identical. They have the same breaks, almost. The difference is the graph on the left has a point for $x=6$ and no point on the asymptote.  The right graph has reversed this.

The difference is the underlying function for the graph on the left includes $6$ in its domain and not $3$. The underlying function for the graph on the right reverses this. In the world of functions, this is significant.  We want our language to note the differences. Therefore, we are going to adopt some language to separate these ideas.


\begin{idea} Language \\
While graphs have breaks, functions will have \textbf{dicontinuities} and \textbf{singularities}.

\begin{itemize}
\item If the domain of a function includes the number where the graph is experiencing a break, then we will say the function has a \textbf{discontinuity} at this domain number.

\item If the domain of a function does not include the number where the graph is experiencing a break, then we will say the function has a \textbf{singularity} at this domain number.
\end{itemize}

\end{idea}




\begin{example} Discontinuities \\

Let $h(t)$ be a function.  The graph of $y= h(t)$ is displayed below. 

\begin{image}
\begin{tikzpicture}
     \begin{axis}[
            	domain=-10:10, ymax=10, xmax=10, ymin=-10, xmin=-10,
            	axis lines =center, xlabel=$t$, ylabel=$y$,
            	every axis y label/.style={at=(current axis.above origin),anchor=south},
            	every axis x label/.style={at=(current axis.right of origin),anchor=west},
            	axis on top,
          		]

        
        \addplot [draw=penColor, very thick, smooth, domain=(-8:-3)] {-x};
        \addplot [draw=penColor, very thick, smooth, domain=(-1:3)] {0.5*x-1};
        \addplot [draw=penColor, very thick, smooth, domain=(4:8)] {-2*x+10};


        \addplot[color=penColor,fill=white,only marks,mark=*] coordinates{(-8,8)};
        \addplot[color=penColor,fill=white,only marks,mark=*] coordinates{(-3,3)};

        \addplot[color=penColor,fill=penColor,only marks,mark=*] coordinates{(-1,-1.5)};
        \addplot[color=penColor,fill=white,only marks,mark=*] coordinates{(3,0.5)};

        \addplot[color=penColor,fill=penColor,only marks,mark=*] coordinates{(4,2)};
        \addplot[color=penColor,fill=white,only marks,mark=*] coordinates{(8,-6)};

    \end{axis}
\end{tikzpicture}
\end{image}

$h(t)$ has no discontinuities.  The idea of a discontinuity is that the value of the function changesabruptly at a number.  The graph moves vertically.  

In the graph above there is horizontal space between the graphical pieces.  The domain number where a discontinuity occurs must be in the middle of a domain interval.  The domain number must be in some  interval of the domain.

\end{example}






\begin{definition} A Discontinuity \\

A real number, $c$, is said to be a \textbf{discontinuity} of the function $f$, if

\begin{itemize}
\item $c$ is a member of the domain of $f$,

\item $c$ is not an isolated number of the domain of $f$.

\begin{itemize}
\item $c \in (a, b)$ for some open interval $(a, b)$ in the domain of $f$, or
\item $c$ is an included endpoint of a maximal interval of the domain of $f$. 
\end{itemize}

\item there is a distance, $d > 0$, such that EVERY nonsingleton interval in the domain containing $c$ also contains a different number $e$, with $ |f(c) - f(e)| > d$.

\end{itemize}

\end{definition}






\begin{example} Discontinuity \\

Let $g(k)$ be a function.  The graph of $y= g(k)$ is displayed below. 

\begin{image}
\begin{tikzpicture}
     \begin{axis}[
            	domain=-10:10, ymax=10, xmax=10, ymin=-10, xmin=-10,
            	axis lines =center, xlabel=$k$, ylabel=$y$,
            	every axis y label/.style={at=(current axis.above origin),anchor=south},
            	every axis x label/.style={at=(current axis.right of origin),anchor=west},
            	axis on top,
          		]

        
        \addplot [draw=penColor, very thick, smooth, domain=(-8:-3)] {-x};
        \addplot [draw=penColor, very thick, smooth, domain=(-1:3)] {0.5*x-1};
        \addplot [draw=penColor, very thick, smooth, domain=(4:8)] {-2*x+10};


        \addplot[color=penColor,fill=white,only marks,mark=*] coordinates{(-8,8)};
        \addplot[color=penColor,fill=white,only marks,mark=*] coordinates{(-3,3)};

        \addplot[color=penColor,fill=penColor,only marks,mark=*] coordinates{(-1,-1.5)};
        \addplot[color=penColor,fill=white,only marks,mark=*] coordinates{(3,0.5)};

        \addplot[color=penColor,fill=penColor,only marks,mark=*] coordinates{(4,2)};
        \addplot[color=penColor,fill=white,only marks,mark=*] coordinates{(8,-6)};

    \end{axis}
\end{tikzpicture}
\end{image}

$h(t)$ has no discontinuities.  The idea of a discontinuity is that the value of the function changes abruptly at a number.  The graph moves vertically.  

In the graph above there is horizontal space between the graphical pieces. The only candidates for discontinuities would be endpoints and they are all attached to their intervals.

\end{example}







\begin{example} Jump Discontinuity \\

Let $g(k)$ be a function.  The graph of $y= g(k)$ is displayed below. 

\begin{image}
\begin{tikzpicture}
     \begin{axis}[
                domain=-10:10, ymax=10, xmax=10, ymin=-10, xmin=-10,
                axis lines =center, xlabel=$k$, ylabel=$y$,
                every axis y label/.style={at=(current axis.above origin),anchor=south},
                every axis x label/.style={at=(current axis.right of origin),anchor=west},
                axis on top,
                ]

        
        \addplot [draw=penColor, very thick, smooth, domain=(-8:-3)] {x};
        \addplot [draw=penColor, very thick, smooth, domain=(-1:4)] {0.5*x+4};
        \addplot [draw=penColor, very thick, smooth, domain=(4:8)] {-2*x+8};


        \addplot[color=penColor,fill=white,only marks,mark=*] coordinates{(-8,-8)};
        \addplot[color=penColor,fill=white,only marks,mark=*] coordinates{(-3,-3)};

        \addplot[color=penColor,fill=penColor,only marks,mark=*] coordinates{(-1,3.5)};
        \addplot[color=penColor,fill=white,only marks,mark=*] coordinates{(4,6)};

        \addplot[color=penColor,fill=penColor,only marks,mark=*] coordinates{(4,0)};
        \addplot[color=penColor,fill=white,only marks,mark=*] coordinates{(8,-8)};
        \addplot[color=penColor,fill=penColor,only marks,mark=*] coordinates{(8,-2)};

        \addplot[color=penColor,fill=penColor,only marks,mark=*] coordinates{(-2, 2)};

    \end{axis}
\end{tikzpicture}
\end{image}


\begin{itemize}
\item $4$ is a discontinuity of $g(k)$.  $4 \in (3.5, 4.5)$, which is an open interval in the domain of $g$. Let $d=0.5$. Any nonsingleton interval in the domain of $g$ containing $4$ must also contain an interval of the form $(4-\epsilon, 4+\epsilon)$, where $0 < \epsilon < 0.1$ is small enough.  Such an interval would ALWAYS contain the number $e = 4-\frac{\epsilon}{2}$.

\[ \left| g(4) - g\left(4-\frac{\epsilon}{2}\right) \right| > 4 > d \]

Therefore, $4$ is a discontinuity of $g$.


\item $-2$ is not a discontinuity of $g$, because $2$ is not a member of an open set of the domain of $g$ and $-2$ is not an included endpoint of a maximal interval of the domain on $g$.




\item $8$ is a discontinuity of $g(k)$.  $8$ is the endpoint of a maximal interval. Any nonsingleton interval in the domain of $g$ containing $8$ must also contain an interval of the form $(8-\epsilon, 8]$, where $0 < \epsilon < 0.1$ is small enough.  Such an interval would ALWAYS contain the number $e = 8-\frac{\epsilon}{2}$.

\[ \left| g(8) - g\left(8-\frac{\epsilon}{2}\right) \right| > 1 > d \]

Therefore, $4$ is a discontinuity of $g$.

\end{itemize}



\end{example}














\begin{example} Removeable Discontinuity \\

Let $T(y)$ be a function.  The graph of $z = T(y)$ is displayed below. 

\begin{image}
\begin{tikzpicture}
     \begin{axis}[
                domain=-10:10, ymax=10, xmax=10, ymin=-10, xmin=-10,
                axis lines =center, xlabel=$y$, ylabel=$z$,
                every axis y label/.style={at=(current axis.above origin),anchor=south},
                every axis x label/.style={at=(current axis.right of origin),anchor=west},
                axis on top,
                ]

        
        \addplot [draw=penColor, very thick, smooth, domain=(-8:-3)] {x};
        \addplot [draw=penColor, very thick, smooth, domain=(-1:4)] {0.5*x+4};
        \addplot [draw=penColor, very thick, smooth, domain=(4:8)] {-2*x+8};


        \addplot[color=penColor,fill=white,only marks,mark=*] coordinates{(-8,-8)};
        \addplot[color=penColor,fill=white,only marks,mark=*] coordinates{(-3,-3)};

        \addplot[color=penColor,fill=penColor,only marks,mark=*] coordinates{(-1,3.5)};
        \addplot[color=penColor,fill=white,only marks,mark=*] coordinates{(4,6)};

        \addplot[color=penColor,fill=penColor,only marks,mark=*] coordinates{(4,0)};
        \addplot[color=penColor,fill=white,only marks,mark=*] coordinates{(8,-8)};

        \addplot[color=penColor,fill=white,only marks,mark=*] coordinates{(-6,-6)};
        \addplot[color=penColor,fill=penColor,only marks,mark=*] coordinates{(-6,-3)};
        \addplot[color=penColor,fill=white,only marks,mark=*] coordinates{(7,-6)};

    \end{axis}
\end{tikzpicture}
\end{image}

\begin{itemize}
\item $7$ is not a discontinuity of $T$, because $7$ is not a member of the domain of $T$.

\item $-6$ is a discontinuity of $T(y)$.  $-6 \in (-6.4, -5.4)$, which is an open interval in the domain of $T$. Let $d=0.5$. Any nonsingleton interval in the domain of $T$ containing $-6$ must also contain an open interval of the form $(-6-\epsilon, -6+\epsilon)$, where $0 < \epsilon < 0.1$ is small enough.  Such an interval would ALWAYS contain the number $e = -6+\frac{\epsilon}{2}$.

\[ \left| T(-6) - T\left(-6+\frac{\epsilon}{2}\right) \right| > 1 > d \]

Therefore, $-6$ is a discontinuity of $T$.

\end{itemize}



\end{example}












\begin{example} Asymptotic Discontinuity \\

Let $f(x)$ be a function.  The graph of $y = f(x)$ is displayed below. 



\begin{image}
\begin{tikzpicture}
     \begin{axis}[
                domain=-10:10, ymax=10, xmax=10, ymin=-10, xmin=-10,
                axis lines =center, xlabel=$x$, ylabel=$y$,
                every axis y label/.style={at=(current axis.above origin),anchor=south},
                every axis x label/.style={at=(current axis.right of origin),anchor=west},
                axis on top,
                ]

        
        \addplot [draw=penColor, very thick, smooth, samples=200, domain=(-8:-5.03), <->] {1/((x-3)*(x+5)) + 2};
        \addplot [draw=penColor, very thick, smooth, samples=200, domain=(-4.97:2.97), <->] {1/((x-3)*(x+5)) + 2};
        \addplot [draw=penColor, very thick, smooth, samples=200, domain=(3.03:9), <->] {1/((x-3)*(x+5)) + 2};

        \addplot [line width=1, gray, dashed,samples=100,domain=(-9:9)] ({3},{x});
        \addplot [line width=1, gray, dashed,samples=100,domain=(-9:9)] ({-5},{x});

        \addplot[color=penColor,fill=penColor,only marks,mark=*] coordinates{(-5,-6)};

    \end{axis}



\end{tikzpicture}
\end{image}

\begin{itemize}
\item $3$ is not a discontinuity of $f$, because $3$ is not a member of the domain of $f$.

\item $-5$ is a discontinuity of $f(x)$.  $-5 \in (-5.3, -4.7)$, which is an open interval in the domain of $f$. Let $d=0.5$. Any nonsingleton interval in the domain of $f$ containing $-5$ must also contain an open interval of the form $(-5-\epsilon, -5+\epsilon)$, where $0 < \epsilon < 0.1$ is small enough.  Such an interval would ALWAYS contain the number $e = -5-\frac{\epsilon}{2}$.

\[ \left| f(-5) - f\left(-5-\frac{\epsilon}{2}\right) \right| > 1 > d \]

Therefore, $-5$ is a discontinuity of $f$.

\end{itemize}

\end{example}





\begin{observation} But, But, But, ...

\textit{But, the examples skipped over numbers not in the domain that are obviously discontinuities.}  \\

The fact that these numbers are not in the domain is significiant when studying the function.  Therefore, we want to distinguish between inclusion and exclusion from the domain.



\end{observation}










\begin{definition} A Singularity 

A real number, $c$, is said to be a \textbf{singularity} of the function $f$, if $c$ is not a member of the domain of $f$, however, the function is behaving like a discontinnuity around $c$.

\begin{itemize}
\item If $c$ was included in the domain of $f$, and $c \in (a, b)$, $c$ would be in some open interval in the domain of $f$, and $f(c)$ could be defined large enough so that $f$ had a discontinuity at $c$, then $c$ is a singularity of $f$.

\item If including $c$ in the domain of $f$ makes $c$ an endpoint of a maximal interval, and $f$ has an asymptoic discontinuity at $c$, then $c$ is a singularity of $f$.

\end{itemize}

\end{definition}


\textbf{Note:} We are not allowing removeable or jump singularities at endpoints. The can be a removeable discontinuity at an endpoint.













\end{document}
