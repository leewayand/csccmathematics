\documentclass{ximera}

%\usepackage{todonotes}

\newcommand{\todo}{}

\usepackage{esint} % for \oiint
\ifxake%%https://math.meta.stackexchange.com/questions/9973/how-do-you-render-a-closed-surface-double-integral
\renewcommand{\oiint}{{\large\bigcirc}\kern-1.56em\iint}
\fi


\graphicspath{
  {./}
  {ximeraTutorial/}
  {basicPhilosophy/}
  {functionsOfSeveralVariables/}
  {normalVectors/}
  {lagrangeMultipliers/}
  {vectorFields/}
  {greensTheorem/}
  {shapeOfThingsToCome/}
  {dotProducts/}
  {partialDerivativesAndTheGradientVector/}
  {../productAndQuotientRules/exercises/}
  {../normalVectors/exercisesParametricPlots/}
  {../continuityOfFunctionsOfSeveralVariables/exercises/}
  {../partialDerivativesAndTheGradientVector/exercises/}
  {../directionalDerivativeAndChainRule/exercises/}
  {../commonCoordinates/exercisesCylindricalCoordinates/}
  {../commonCoordinates/exercisesSphericalCoordinates/}
  {../greensTheorem/exercisesCurlAndLineIntegrals/}
  {../greensTheorem/exercisesDivergenceAndLineIntegrals/}
  {../shapeOfThingsToCome/exercisesDivergenceTheorem/}
  {../greensTheorem/}
  {../shapeOfThingsToCome/}
  {../separableDifferentialEquations/exercises/}
  {vectorFields/}
}

\newcommand{\mooculus}{\textsf{\textbf{MOOC}\textnormal{\textsf{ULUS}}}}

\usepackage{tkz-euclide}
\usepackage{tikz}
\usepackage{tikz-cd}
\usetikzlibrary{arrows}
\tikzset{>=stealth,commutative diagrams/.cd,
  arrow style=tikz,diagrams={>=stealth}} %% cool arrow head
\tikzset{shorten <>/.style={ shorten >=#1, shorten <=#1 } } %% allows shorter vectors

\usetikzlibrary{backgrounds} %% for boxes around graphs
\usetikzlibrary{shapes,positioning}  %% Clouds and stars
\usetikzlibrary{matrix} %% for matrix
\usepgfplotslibrary{polar} %% for polar plots
\usepgfplotslibrary{fillbetween} %% to shade area between curves in TikZ
%\usetkzobj{all}
\usepackage[makeroom]{cancel} %% for strike outs
%\usepackage{mathtools} %% for pretty underbrace % Breaks Ximera
%\usepackage{multicol}
\usepackage{pgffor} %% required for integral for loops



%% http://tex.stackexchange.com/questions/66490/drawing-a-tikz-arc-specifying-the-center
%% Draws beach ball
\tikzset{pics/carc/.style args={#1:#2:#3}{code={\draw[pic actions] (#1:#3) arc(#1:#2:#3);}}}



\usepackage{array}
\setlength{\extrarowheight}{+.1cm}
\newdimen\digitwidth
\settowidth\digitwidth{9}
\def\divrule#1#2{
\noalign{\moveright#1\digitwidth
\vbox{\hrule width#2\digitwidth}}}




% \newcommand{\RR}{\mathbb R}
% \newcommand{\R}{\mathbb R}
% \newcommand{\N}{\mathbb N}
% \newcommand{\Z}{\mathbb Z}

\newcommand{\sagemath}{\textsf{SageMath}}


%\renewcommand{\d}{\,d\!}
%\renewcommand{\d}{\mathop{}\!d}
%\newcommand{\dd}[2][]{\frac{\d #1}{\d #2}}
%\newcommand{\pp}[2][]{\frac{\partial #1}{\partial #2}}
% \renewcommand{\l}{\ell}
%\newcommand{\ddx}{\frac{d}{\d x}}

% \newcommand{\zeroOverZero}{\ensuremath{\boldsymbol{\tfrac{0}{0}}}}
%\newcommand{\inftyOverInfty}{\ensuremath{\boldsymbol{\tfrac{\infty}{\infty}}}}
%\newcommand{\zeroOverInfty}{\ensuremath{\boldsymbol{\tfrac{0}{\infty}}}}
%\newcommand{\zeroTimesInfty}{\ensuremath{\small\boldsymbol{0\cdot \infty}}}
%\newcommand{\inftyMinusInfty}{\ensuremath{\small\boldsymbol{\infty - \infty}}}
%\newcommand{\oneToInfty}{\ensuremath{\boldsymbol{1^\infty}}}
%\newcommand{\zeroToZero}{\ensuremath{\boldsymbol{0^0}}}
%\newcommand{\inftyToZero}{\ensuremath{\boldsymbol{\infty^0}}}



% \newcommand{\numOverZero}{\ensuremath{\boldsymbol{\tfrac{\#}{0}}}}
% \newcommand{\dfn}{\textbf}
% \newcommand{\unit}{\,\mathrm}
% \newcommand{\unit}{\mathop{}\!\mathrm}
% \newcommand{\eval}[1]{\bigg[ #1 \bigg]}
% \newcommand{\seq}[1]{\left( #1 \right)}
% \renewcommand{\epsilon}{\varepsilon}
% \renewcommand{\phi}{\varphi}


% \renewcommand{\iff}{\Leftrightarrow}

% \DeclareMathOperator{\arccot}{arccot}
% \DeclareMathOperator{\arcsec}{arcsec}
% \DeclareMathOperator{\arccsc}{arccsc}
% \DeclareMathOperator{\si}{Si}
% \DeclareMathOperator{\scal}{scal}
% \DeclareMathOperator{\sign}{sign}


%% \newcommand{\tightoverset}[2]{% for arrow vec
%%   \mathop{#2}\limits^{\vbox to -.5ex{\kern-0.75ex\hbox{$#1$}\vss}}}
% \newcommand{\arrowvec}[1]{{\overset{\rightharpoonup}{#1}}}
% \renewcommand{\vec}[1]{\arrowvec{\mathbf{#1}}}
% \renewcommand{\vec}[1]{{\overset{\boldsymbol{\rightharpoonup}}{\mathbf{#1}}}}

% \newcommand{\point}[1]{\left(#1\right)} %this allows \vector{ to be changed to \vector{ with a quick find and replace
% \newcommand{\pt}[1]{\mathbf{#1}} %this allows \vec{ to be changed to \vec{ with a quick find and replace
% \newcommand{\Lim}[2]{\lim_{\point{#1} \to \point{#2}}} %Bart, I changed this to point since I want to use it.  It runs through both of the exercise and exerciseE files in limits section, which is why it was in each document to start with.

% \DeclareMathOperator{\proj}{\mathbf{proj}}
% \newcommand{\veci}{{\boldsymbol{\hat{\imath}}}}
% \newcommand{\vecj}{{\boldsymbol{\hat{\jmath}}}}
% \newcommand{\veck}{{\boldsymbol{\hat{k}}}}
% \newcommand{\vecl}{\vec{\boldsymbol{\l}}}
% \newcommand{\uvec}[1]{\mathbf{\hat{#1}}}
% \newcommand{\utan}{\mathbf{\hat{t}}}
% \newcommand{\unormal}{\mathbf{\hat{n}}}
% \newcommand{\ubinormal}{\mathbf{\hat{b}}}

% \newcommand{\dotp}{\bullet}
% \newcommand{\cross}{\boldsymbol\times}
% \newcommand{\grad}{\boldsymbol\nabla}
% \newcommand{\divergence}{\grad\dotp}
% \newcommand{\curl}{\grad\cross}
%\DeclareMathOperator{\divergence}{divergence}
%\DeclareMathOperator{\curl}[1]{\grad\cross #1}
% \newcommand{\lto}{\mathop{\longrightarrow\,}\limits}

% \renewcommand{\bar}{\overline}

\colorlet{textColor}{black}
\colorlet{background}{white}
\colorlet{penColor}{blue!50!black} % Color of a curve in a plot
\colorlet{penColor2}{red!50!black}% Color of a curve in a plot
\colorlet{penColor3}{red!50!blue} % Color of a curve in a plot
\colorlet{penColor4}{green!50!black} % Color of a curve in a plot
\colorlet{penColor5}{orange!80!black} % Color of a curve in a plot
\colorlet{penColor6}{yellow!70!black} % Color of a curve in a plot
\colorlet{fill1}{penColor!20} % Color of fill in a plot
\colorlet{fill2}{penColor2!20} % Color of fill in a plot
\colorlet{fillp}{fill1} % Color of positive area
\colorlet{filln}{penColor2!20} % Color of negative area
\colorlet{fill3}{penColor3!20} % Fill
\colorlet{fill4}{penColor4!20} % Fill
\colorlet{fill5}{penColor5!20} % Fill
\colorlet{gridColor}{gray!50} % Color of grid in a plot

\newcommand{\surfaceColor}{violet}
\newcommand{\surfaceColorTwo}{redyellow}
\newcommand{\sliceColor}{greenyellow}




\pgfmathdeclarefunction{gauss}{2}{% gives gaussian
  \pgfmathparse{1/(#2*sqrt(2*pi))*exp(-((x-#1)^2)/(2*#2^2))}%
}


%%%%%%%%%%%%%
%% Vectors
%%%%%%%%%%%%%

%% Simple horiz vectors
\renewcommand{\vector}[1]{\left\langle #1\right\rangle}


%% %% Complex Horiz Vectors with angle brackets
%% \makeatletter
%% \renewcommand{\vector}[2][ , ]{\left\langle%
%%   \def\nextitem{\def\nextitem{#1}}%
%%   \@for \el:=#2\do{\nextitem\el}\right\rangle%
%% }
%% \makeatother

%% %% Vertical Vectors
%% \def\vector#1{\begin{bmatrix}\vecListA#1,,\end{bmatrix}}
%% \def\vecListA#1,{\if,#1,\else #1\cr \expandafter \vecListA \fi}

%%%%%%%%%%%%%
%% End of vectors
%%%%%%%%%%%%%

%\newcommand{\fullwidth}{}
%\newcommand{\normalwidth}{}



%% makes a snazzy t-chart for evaluating functions
%\newenvironment{tchart}{\rowcolors{2}{}{background!90!textColor}\array}{\endarray}

%%This is to help with formatting on future title pages.
\newenvironment{sectionOutcomes}{}{}



%% Flowchart stuff
%\tikzstyle{startstop} = [rectangle, rounded corners, minimum width=3cm, minimum height=1cm,text centered, draw=black]
%\tikzstyle{question} = [rectangle, minimum width=3cm, minimum height=1cm, text centered, draw=black]
%\tikzstyle{decision} = [trapezium, trapezium left angle=70, trapezium right angle=110, minimum width=3cm, minimum height=1cm, text centered, draw=black]
%\tikzstyle{question} = [rectangle, rounded corners, minimum width=3cm, minimum height=1cm,text centered, draw=black]
%\tikzstyle{process} = [rectangle, minimum width=3cm, minimum height=1cm, text centered, draw=black]
%\tikzstyle{decision} = [trapezium, trapezium left angle=70, trapezium right angle=110, minimum width=3cm, minimum height=1cm, text centered, draw=black]


\title{Formulas}

\begin{document}

\begin{abstract}

\end{abstract}
\maketitle


Functions are relations, which makes them packages.  They contain three sets.  They contain a set of real numbers called the \textbf{domain}.  They contain a second set of real numbers called the \textbf{range}. Finally, they contain a third set of pairs of numbers.  The pairs are all constructed with a number from each of the domain and range. The defining characteristic of a function is that each domain number is in exactly one pair.

The pairs are the reason we study functions.  The pairs are the connection between the domain and range.  They identify which numbers are associated together from both sets. They itemize the association between the domain and range.

Therefore, it comes as no surprise, that we have a toolbox full of tools to help us describe these pairs.


\section{Function Notation}

We have some written notation to help us communicate about the pairs.  $F(d)$ is called \textbf{function notation}. The name of the function is followed by a domain number inside parentheses. $F(d)$ represents the function value at $d$ or the range partner of $d$.  That brings us to at least four uses of parentheses in mathematics.

\begin{notation}  \textbf{\textcolor{blue!55!black}{Parentheses}}  \\

Mathematics overuses symbols and allows the context in which the symbol is used to affect their meaning.
	\begin{itemize}
		\item \textbf{Grouping:} $(3+5)(6+7)$ is a product.  Grouping expressions might signal multiplication with the multiplication sign omitted.
		\item \textbf{Ordered Pairs:} $(4, 5)$ is an ordered pair and might represent a function pair.
		\item \textbf{Coordinates:} $(4, 5)$ might represent the coordinates of a point in the Cartesian Plane.
		\item \textbf{Intervals:} $(4, 5)$ might repesent the open interval of real numbers from $4$ to $5$.
		\item \textbf{Function Notation:} $Double(5)$ symbolizes a range element that is paired with $5$ from the domain in the \textit{Double} function. It is not multiplication.
	\end{itemize}


The context in which parentheses are used helps the reader interpret the parentheses.  And, these contexts can be intertwined.

\[  ((1+4)(7-6), Double(5))    \]
\end{notation}




\begin{warning} \textbf{$F$, $d$, and $x$}  

The actual letter ``$F$'' itself is not part of function notation.  The first part of function notation is the name of the function, which in most instances is not ``$F$.'' The parentheses are always there.  The letter $d$ is in the domain position and can be replaced by any expression representing domain elements.  We will strive not to overuse $F$ and $d$, or $x$.
\end{warning}









\section{Graphs}

We have visual tools where $F(d)$ is encoded by plotting a dot at $(d, F(d))$ on the Cartesian Plane.  $F(d)$ is the signed distance between the point and the horizontal axis. The collection of all dots formed from the function pairs is called the graph of the function. Graphs are drawings, which makes them inherently inaccurate.  They are not exact tools.  They are global tools.  They give a big picture of the function pairs.  From this bird's eye view, we can see trends in the function values, locations of extreme function values, and important function behavior. The global perspective quickly communicates basic characteristics and points us towards function features.












\section{Formulas}
\textbf{\textcolor{red!80!black}{We also want exactness!}} \\


The price we pay for exactness is that we can only investigate the function, one pair at a time.  Our tools for this type of investigation are called \textbf{formulas} and \textbf{equations}.








\begin{explanation} \textbf{Video: Formulas are Procedures}

[ Click on the arrow to the right to expand for video. ]
\begin{expandable} 

\begin{center}
\youtube{8sX8x27NHdk}
\end{center}

\end{expandable}
\end{explanation}






\section{Learning Outcomes}

\begin{sectionOutcomes}
In this section, students will 

\begin{itemize}
\item evaluate a function via a formula.
\item solve equations involving formulas.
\end{itemize}
\end{sectionOutcomes}
















\begin{center}
\textbf{\textcolor{green!50!black}{ooooo=-=-=-=-=-=-=-=-=-=-=-=-=ooOoo=-=-=-=-=-=-=-=-=-=-=-=-=ooooo}} \\

more examples can be found by following this link\\ \link[More Examples of Formulas]{https://ximera.osu.edu/csccmathematics/precalculus1/precalculus1/formulas/examples/exampleList}

\end{center}






\end{document}
