\documentclass{ximera}


\graphicspath{
  {./}
  {ximeraTutorial/}
  {basicPhilosophy/}
}

\newcommand{\mooculus}{\textsf{\textbf{MOOC}\textnormal{\textsf{ULUS}}}}


\usepackage{tkz-euclide}\usepackage{tikz}
\usepackage{tikz-cd}
\usetikzlibrary{arrows}
\tikzset{>=stealth,commutative diagrams/.cd,
  arrow style=tikz,diagrams={>=stealth}} %% cool arrow head
\tikzset{shorten <>/.style={ shorten >=#1, shorten <=#1 } } %% allows shorter vectors

\usetikzlibrary{backgrounds} %% for boxes around graphs
\usetikzlibrary{shapes,positioning}  %% Clouds and stars
\usetikzlibrary{matrix} %% for matrix
\usepgfplotslibrary{polar} %% for polar plots
\usepgfplotslibrary{fillbetween} %% to shade area between curves in TikZ
\usetkzobj{all}
\usepackage[makeroom]{cancel} %% for strike outs
%\usepackage{mathtools} %% for pretty underbrace % Breaks Ximera
%\usepackage{multicol}
\usepackage{pgffor} %% required for integral for loops



%% http://tex.stackexchange.com/questions/66490/drawing-a-tikz-arc-specifying-the-center
%% Draws beach ball
\tikzset{pics/carc/.style args={#1:#2:#3}{code={\draw[pic actions] (#1:#3) arc(#1:#2:#3);}}}



\usepackage{array}
\setlength{\extrarowheight}{+.1cm}
\newdimen\digitwidth
\settowidth\digitwidth{9}
\def\divrule#1#2{
\noalign{\moveright#1\digitwidth
\vbox{\hrule width#2\digitwidth}}}
























%%This is to help with formatting on future title pages.
\newenvironment{sectionOutcomes}{}{}


\title{Decoding Visually}


\begin{document}

\begin{abstract}
evaluating
\end{abstract}
\maketitle



We can use a graph to estimate function values, but we cannot escape the approximation.  Some functions have an algebraic description of the pairings.  We call this algebraic tool a \textbf{formula}.



\begin{definition}  Formula


A \textbf{formula} for a function is an algebraic expression involving the domain number, that produces the function value at the domain number.
\end{defintion}




\begin{center}
expression involving domain number = function value at the domain number 
\end{center}


\begin{center}
expression involving domain number, $d$ = function, $f$, value at the domain number, $d$
\end{center}


\begin{center}
expression involving domain number, $d$ = $f(d)$
\end{center}




\begin{definition}  Variable


The symbol in the function's formulas, which represents the domain number, is called the \textbf{variable}.
\end{defintion}







\begin{example}

$P(k) = 3k - 2$
domain = $[-2, 4)$
range = $[-8, 10)$


$P(k)$ is our function representing the function's value at $k$.  $k$ must be representing domain values. $3k - 2$ is the expression involing the domain number.

\end{example}











\section{Operating Instructions}

A formula is a tool.  We use it to connect domain numbers to their range partners.  But it is a tool, so there is a way to operate it.



The above formula was $P(k) = 3k - 2$.  How would we use this formula to calculate the value of $P$ at $5$?   In other words, how would we use it to calculate $P(5)$?

We just gave ourselves our first clue. We went from $P(k)$ to $P(5)$ by replacing $k$ with $5$.  We should do the samething with the formula. However, this doesn't work.


Replacing $k$ with $5$ in $3k - 2$ gives us $35-2$, which equals $33$.  $33$ is not $P(5)$. The problem is that our formula is using shorthand notation. Simply replacing the variable with the domain number fails to maintain the meaning of the expression.  In this case, a number next to a variable is shorthand for multiplication and this was lost when we replaced with the $5$.

We want to replace all occurrences of the variable with the domain number, while maintain the meaning of the expression.  As you gain experience with formauls, you will be able to do this on-the-fly.  But a quick rule that cures this problem is to replace all occurrences of the variable in the formula with the domain number wrapped in parentheses.





\begin{example}

$P(k) = 3k - 2$
domain = $[-2, 4)$
range = $[-8, 10)$


$P(5) = 3(5) - 2 = 13$

\end{example}





















\end{document}
