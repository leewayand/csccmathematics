\documentclass{ximera}


\graphicspath{
  {./}
  {ximeraTutorial/}
  {basicPhilosophy/}
}

\newcommand{\mooculus}{\textsf{\textbf{MOOC}\textnormal{\textsf{ULUS}}}}


\usepackage{tkz-euclide}\usepackage{tikz}
\usepackage{tikz-cd}
\usetikzlibrary{arrows}
\tikzset{>=stealth,commutative diagrams/.cd,
  arrow style=tikz,diagrams={>=stealth}} %% cool arrow head
\tikzset{shorten <>/.style={ shorten >=#1, shorten <=#1 } } %% allows shorter vectors

\usetikzlibrary{backgrounds} %% for boxes around graphs
\usetikzlibrary{shapes,positioning}  %% Clouds and stars
\usetikzlibrary{matrix} %% for matrix
\usepgfplotslibrary{polar} %% for polar plots
\usepgfplotslibrary{fillbetween} %% to shade area between curves in TikZ
\usetkzobj{all}
\usepackage[makeroom]{cancel} %% for strike outs
%\usepackage{mathtools} %% for pretty underbrace % Breaks Ximera
%\usepackage{multicol}
\usepackage{pgffor} %% required for integral for loops



%% http://tex.stackexchange.com/questions/66490/drawing-a-tikz-arc-specifying-the-center
%% Draws beach ball
\tikzset{pics/carc/.style args={#1:#2:#3}{code={\draw[pic actions] (#1:#3) arc(#1:#2:#3);}}}



\usepackage{array}
\setlength{\extrarowheight}{+.1cm}
\newdimen\digitwidth
\settowidth\digitwidth{9}
\def\divrule#1#2{
\noalign{\moveright#1\digitwidth
\vbox{\hrule width#2\digitwidth}}}
























%%This is to help with formatting on future title pages.
\newenvironment{sectionOutcomes}{}{}


\title{The Formula Tool}


\begin{document}

\begin{abstract}
part 2
\end{abstract}
\maketitle



A function can be defined via a graph. Each dot on a graph is highlighting a point and the coordinates of this point define a pair in the function. We can use a graph to estimate function values, but we cannot escape the approximation inherent to drawing.  To communicate about exactness, some functions have an algebraic description of the pairings.  We call this algebraic tool a \textbf{formula} or an \textbf{equation}. Not all functions have formulas with which we can calculate.  But when they do, that's what we want to use!  Our goal is to be exact. \\


When functions have formulas, then there is an operation manual to follow. \\





\section*{Step 2}




\begin{center}

\textbf{\textcolor{red!80!black}{It is not that simple!!!!}} \\

\textbf{\textcolor{purple!85!blue}{It never is.}}
 
\end{center}





We have a beginning...\\

\begin{idea}  \textbf{\textcolor{green!50!black}{Formula}} \\ 

A \textbf{formula} for a function is an algebraic expression involving the domain number, that produces the function value at the domain number.
\end{idea}






\begin{definition}  \textbf{\textcolor{green!50!black}{Variable}} (for function notation) \\ 


In function notation, $f(d)$, the symbol inside the parentheses is called the \textbf{variable}. It represents all of the domain values.

\end{definition}



...but, that is just the beginning. \\









We will beginb thinking about function notation in the form 

\[
function_name(variable)
\]


However, we will quickly move to forms like


\[
function_name(expressions)
\]




Function notation is a sophisticated way to communicate.  It will evolve as we wish to communicate more and more information about a function. \\

In funciton notation, the inside of the parentheses may hold the variable.  In that case, the variable for the formula is representing domain values. \\

Or, the inside of the parenthses may hold an expression, which involves the variable.  In this case, the variable iis not representing the domain values. \\


 
\begin{center}

\textbf{\textcolor{blue!55!black}{The variable does not always represent domain numbers.}} \\

\textbf{\textcolor{purple!85!blue}{The inside of the parentheses ALWAYS represents the domain numbers.}}


\end{center}

Sometimes the inside of the parentheses is just the variable and so the variable is representing the domain numbers. \\

Sometimes not. \\


\begin{example}

The function $P$ is defined as follows.

$P(k) = 3k - 2$ \\
domain = $[-2, 6)$ \\
range = $[-8, 16)$ \\


Here, the inside of the parentheses in the function notation is $k$.  That tells us that $k$ represents the domain numbers.  The variable $k$ represents the numbers $[-2, 6)$.


\end{example}






\begin{example}

The function $R$ is defined as follows.

$R(k + 5) = 3k - 2$ \\
domain = $[-2, 6)$ \\


Here, the inside of the parentheses in the function notation is $k + 5$.  That is not the variable. \\


$k$ is the variable.  That is not what is inside the parentheses.\\


The variable $k$ \textbf{DOES NOT} represent the numbers $[-2, 6)$, which is the domain. \\


The expression $k + 2$ represents the domain numbers.   The expression $k + 5$ represents the domain numbers $[-2, 6)$. \\


$k$ represents the numbers that if you added $5$ to them, you would get the numbers $[-2, 6)$.\\


$k$ represents the number $[-7, 1)$.



\begin{question}


$-1$ is in the domain of $R$. How do you evaluate $R(-1)$? \\

\[
R(-1) \ne 3 (-1) - 2 = 5
\]



We want $R(-1)$. \\

Therefore, we are looking for $R(-1) = R(k + 5)$, which means that $k + 5 = \answer{-1}$., which means that $k = \answer{-6}$.


\[
R(-1) = R(-6 + 5) = 3(-6) - 2 = -20
\]



\end{question}

\end{example}






















































\begin{center}
\textbf{\textcolor{green!50!black}{ooooo-=-=-=-ooOoo-=-=-=-ooooo}} \\

more examples can be found by following this link\\ \link[More Examples of Formulas]{https://ximera.osu.edu/csccmathematics/precalculus1/precalculus1/formulas/examples/exampleList}

\end{center}






\end{document}
