\documentclass{ximera}


\graphicspath{
  {./}
  {ximeraTutorial/}
  {basicPhilosophy/}
}

\newcommand{\mooculus}{\textsf{\textbf{MOOC}\textnormal{\textsf{ULUS}}}}


\usepackage{tkz-euclide}\usepackage{tikz}
\usepackage{tikz-cd}
\usetikzlibrary{arrows}
\tikzset{>=stealth,commutative diagrams/.cd,
  arrow style=tikz,diagrams={>=stealth}} %% cool arrow head
\tikzset{shorten <>/.style={ shorten >=#1, shorten <=#1 } } %% allows shorter vectors

\usetikzlibrary{backgrounds} %% for boxes around graphs
\usetikzlibrary{shapes,positioning}  %% Clouds and stars
\usetikzlibrary{matrix} %% for matrix
\usepgfplotslibrary{polar} %% for polar plots
\usepgfplotslibrary{fillbetween} %% to shade area between curves in TikZ
\usetkzobj{all}
\usepackage[makeroom]{cancel} %% for strike outs
%\usepackage{mathtools} %% for pretty underbrace % Breaks Ximera
%\usepackage{multicol}
\usepackage{pgffor} %% required for integral for loops



%% http://tex.stackexchange.com/questions/66490/drawing-a-tikz-arc-specifying-the-center
%% Draws beach ball
\tikzset{pics/carc/.style args={#1:#2:#3}{code={\draw[pic actions] (#1:#3) arc(#1:#2:#3);}}}



\usepackage{array}
\setlength{\extrarowheight}{+.1cm}
\newdimen\digitwidth
\settowidth\digitwidth{9}
\def\divrule#1#2{
\noalign{\moveright#1\digitwidth
\vbox{\hrule width#2\digitwidth}}}
























%%This is to help with formatting on future title pages.
\newenvironment{sectionOutcomes}{}{}


\title{Formula and Graphs}


\begin{document}

\begin{abstract}
bridge
\end{abstract}
\maketitle



Our two main tools for investigating funcitons are formulas and graphs. Not every function has a formula, but when it does there should be a connection between the formula and graph.

\section{domain}
When a function is described with a formula, then the domain is described in writing - usually some sort of set notation.  Our favorite way is through interval notation.  Set builder notation is also used.  Sometimes the domain is just described with words.

All of these numbers are pictured as lying on the horizontal axis in a graph.  They are not plotted as points on the horizontal axis, unless the funciton value just happens to be $0$.

However, during many discussions the horizontal axis is shaded in to highlight some aspect of the domain.







\section{range}
The same idea goes for the range of a function, except these values are imagined along the vertical axis.








\section{pairs}
The pairs are the most important part of a function.  They give the connection between the domain and range.  Formulas do not explicitly give pairs. You can assemble an indiviual pair, one-at-a-time, by evaluating the formula at a particular domain number.

Graphs display pairs.  The dots included in the graph are visually encoding the function pair.  They are deciphered into the domain number and function value. The domain number is the first coordinate and the function value is the second number.

\[
formula \leftrightarrow second coordinate
\]











































\section{Domain Types}

We encounter functions in several ways, each affecting the domain of a function.

\begin{enumeration}
\item \textbf{stated domain}
A function may come already equipped with a stated domain.  Graphs communicate the domain - just collect all of the first coordinates from the points. Many times we use inteval notation to state the domain.


\item \textbf{implied domain}
Mathematicians like shorthand. The best shorthand is just nothing.  Nothing is used all over the place. If a function is described with a formula and there is no stated domain, then there is an implied domain.  The implied domain si all real numbers that don't cause a problem with the formula.

We know of two problems: square (even) roots of negative numbers and fractions with $0$ denominators.  Any real numbers that cause this to happen are removed from the domain.


\item \textbf{applied domain}
We use functions to model many measuring situations. In this case, we want our model to describe the situation.  Therefore, the domain should not contain numbers that don't fit the situation. An applied domain is a subset of the implied domain.  The applied domain includes all of the real numbers that make sense in the situation.

\end{enumeration}






\end{document}
