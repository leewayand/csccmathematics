\documentclass{ximera}

%\usepackage{todonotes}

\newcommand{\todo}{}

\usepackage{esint} % for \oiint
\ifxake%%https://math.meta.stackexchange.com/questions/9973/how-do-you-render-a-closed-surface-double-integral
\renewcommand{\oiint}{{\large\bigcirc}\kern-1.56em\iint}
\fi


\graphicspath{
  {./}
  {ximeraTutorial/}
  {basicPhilosophy/}
  {functionsOfSeveralVariables/}
  {normalVectors/}
  {lagrangeMultipliers/}
  {vectorFields/}
  {greensTheorem/}
  {shapeOfThingsToCome/}
  {dotProducts/}
  {partialDerivativesAndTheGradientVector/}
  {../productAndQuotientRules/exercises/}
  {../normalVectors/exercisesParametricPlots/}
  {../continuityOfFunctionsOfSeveralVariables/exercises/}
  {../partialDerivativesAndTheGradientVector/exercises/}
  {../directionalDerivativeAndChainRule/exercises/}
  {../commonCoordinates/exercisesCylindricalCoordinates/}
  {../commonCoordinates/exercisesSphericalCoordinates/}
  {../greensTheorem/exercisesCurlAndLineIntegrals/}
  {../greensTheorem/exercisesDivergenceAndLineIntegrals/}
  {../shapeOfThingsToCome/exercisesDivergenceTheorem/}
  {../greensTheorem/}
  {../shapeOfThingsToCome/}
  {../separableDifferentialEquations/exercises/}
  {vectorFields/}
}

\newcommand{\mooculus}{\textsf{\textbf{MOOC}\textnormal{\textsf{ULUS}}}}

\usepackage{tkz-euclide}
\usepackage{tikz}
\usepackage{tikz-cd}
\usetikzlibrary{arrows}
\tikzset{>=stealth,commutative diagrams/.cd,
  arrow style=tikz,diagrams={>=stealth}} %% cool arrow head
\tikzset{shorten <>/.style={ shorten >=#1, shorten <=#1 } } %% allows shorter vectors

\usetikzlibrary{backgrounds} %% for boxes around graphs
\usetikzlibrary{shapes,positioning}  %% Clouds and stars
\usetikzlibrary{matrix} %% for matrix
\usepgfplotslibrary{polar} %% for polar plots
\usepgfplotslibrary{fillbetween} %% to shade area between curves in TikZ
%\usetkzobj{all}
\usepackage[makeroom]{cancel} %% for strike outs
%\usepackage{mathtools} %% for pretty underbrace % Breaks Ximera
%\usepackage{multicol}
\usepackage{pgffor} %% required for integral for loops



%% http://tex.stackexchange.com/questions/66490/drawing-a-tikz-arc-specifying-the-center
%% Draws beach ball
\tikzset{pics/carc/.style args={#1:#2:#3}{code={\draw[pic actions] (#1:#3) arc(#1:#2:#3);}}}



\usepackage{array}
\setlength{\extrarowheight}{+.1cm}
\newdimen\digitwidth
\settowidth\digitwidth{9}
\def\divrule#1#2{
\noalign{\moveright#1\digitwidth
\vbox{\hrule width#2\digitwidth}}}




% \newcommand{\RR}{\mathbb R}
% \newcommand{\R}{\mathbb R}
% \newcommand{\N}{\mathbb N}
% \newcommand{\Z}{\mathbb Z}

\newcommand{\sagemath}{\textsf{SageMath}}


%\renewcommand{\d}{\,d\!}
%\renewcommand{\d}{\mathop{}\!d}
%\newcommand{\dd}[2][]{\frac{\d #1}{\d #2}}
%\newcommand{\pp}[2][]{\frac{\partial #1}{\partial #2}}
% \renewcommand{\l}{\ell}
%\newcommand{\ddx}{\frac{d}{\d x}}

% \newcommand{\zeroOverZero}{\ensuremath{\boldsymbol{\tfrac{0}{0}}}}
%\newcommand{\inftyOverInfty}{\ensuremath{\boldsymbol{\tfrac{\infty}{\infty}}}}
%\newcommand{\zeroOverInfty}{\ensuremath{\boldsymbol{\tfrac{0}{\infty}}}}
%\newcommand{\zeroTimesInfty}{\ensuremath{\small\boldsymbol{0\cdot \infty}}}
%\newcommand{\inftyMinusInfty}{\ensuremath{\small\boldsymbol{\infty - \infty}}}
%\newcommand{\oneToInfty}{\ensuremath{\boldsymbol{1^\infty}}}
%\newcommand{\zeroToZero}{\ensuremath{\boldsymbol{0^0}}}
%\newcommand{\inftyToZero}{\ensuremath{\boldsymbol{\infty^0}}}



% \newcommand{\numOverZero}{\ensuremath{\boldsymbol{\tfrac{\#}{0}}}}
% \newcommand{\dfn}{\textbf}
% \newcommand{\unit}{\,\mathrm}
% \newcommand{\unit}{\mathop{}\!\mathrm}
% \newcommand{\eval}[1]{\bigg[ #1 \bigg]}
% \newcommand{\seq}[1]{\left( #1 \right)}
% \renewcommand{\epsilon}{\varepsilon}
% \renewcommand{\phi}{\varphi}


% \renewcommand{\iff}{\Leftrightarrow}

% \DeclareMathOperator{\arccot}{arccot}
% \DeclareMathOperator{\arcsec}{arcsec}
% \DeclareMathOperator{\arccsc}{arccsc}
% \DeclareMathOperator{\si}{Si}
% \DeclareMathOperator{\scal}{scal}
% \DeclareMathOperator{\sign}{sign}


%% \newcommand{\tightoverset}[2]{% for arrow vec
%%   \mathop{#2}\limits^{\vbox to -.5ex{\kern-0.75ex\hbox{$#1$}\vss}}}
% \newcommand{\arrowvec}[1]{{\overset{\rightharpoonup}{#1}}}
% \renewcommand{\vec}[1]{\arrowvec{\mathbf{#1}}}
% \renewcommand{\vec}[1]{{\overset{\boldsymbol{\rightharpoonup}}{\mathbf{#1}}}}

% \newcommand{\point}[1]{\left(#1\right)} %this allows \vector{ to be changed to \vector{ with a quick find and replace
% \newcommand{\pt}[1]{\mathbf{#1}} %this allows \vec{ to be changed to \vec{ with a quick find and replace
% \newcommand{\Lim}[2]{\lim_{\point{#1} \to \point{#2}}} %Bart, I changed this to point since I want to use it.  It runs through both of the exercise and exerciseE files in limits section, which is why it was in each document to start with.

% \DeclareMathOperator{\proj}{\mathbf{proj}}
% \newcommand{\veci}{{\boldsymbol{\hat{\imath}}}}
% \newcommand{\vecj}{{\boldsymbol{\hat{\jmath}}}}
% \newcommand{\veck}{{\boldsymbol{\hat{k}}}}
% \newcommand{\vecl}{\vec{\boldsymbol{\l}}}
% \newcommand{\uvec}[1]{\mathbf{\hat{#1}}}
% \newcommand{\utan}{\mathbf{\hat{t}}}
% \newcommand{\unormal}{\mathbf{\hat{n}}}
% \newcommand{\ubinormal}{\mathbf{\hat{b}}}

% \newcommand{\dotp}{\bullet}
% \newcommand{\cross}{\boldsymbol\times}
% \newcommand{\grad}{\boldsymbol\nabla}
% \newcommand{\divergence}{\grad\dotp}
% \newcommand{\curl}{\grad\cross}
%\DeclareMathOperator{\divergence}{divergence}
%\DeclareMathOperator{\curl}[1]{\grad\cross #1}
% \newcommand{\lto}{\mathop{\longrightarrow\,}\limits}

% \renewcommand{\bar}{\overline}

\colorlet{textColor}{black}
\colorlet{background}{white}
\colorlet{penColor}{blue!50!black} % Color of a curve in a plot
\colorlet{penColor2}{red!50!black}% Color of a curve in a plot
\colorlet{penColor3}{red!50!blue} % Color of a curve in a plot
\colorlet{penColor4}{green!50!black} % Color of a curve in a plot
\colorlet{penColor5}{orange!80!black} % Color of a curve in a plot
\colorlet{penColor6}{yellow!70!black} % Color of a curve in a plot
\colorlet{fill1}{penColor!20} % Color of fill in a plot
\colorlet{fill2}{penColor2!20} % Color of fill in a plot
\colorlet{fillp}{fill1} % Color of positive area
\colorlet{filln}{penColor2!20} % Color of negative area
\colorlet{fill3}{penColor3!20} % Fill
\colorlet{fill4}{penColor4!20} % Fill
\colorlet{fill5}{penColor5!20} % Fill
\colorlet{gridColor}{gray!50} % Color of grid in a plot

\newcommand{\surfaceColor}{violet}
\newcommand{\surfaceColorTwo}{redyellow}
\newcommand{\sliceColor}{greenyellow}




\pgfmathdeclarefunction{gauss}{2}{% gives gaussian
  \pgfmathparse{1/(#2*sqrt(2*pi))*exp(-((x-#1)^2)/(2*#2^2))}%
}


%%%%%%%%%%%%%
%% Vectors
%%%%%%%%%%%%%

%% Simple horiz vectors
\renewcommand{\vector}[1]{\left\langle #1\right\rangle}


%% %% Complex Horiz Vectors with angle brackets
%% \makeatletter
%% \renewcommand{\vector}[2][ , ]{\left\langle%
%%   \def\nextitem{\def\nextitem{#1}}%
%%   \@for \el:=#2\do{\nextitem\el}\right\rangle%
%% }
%% \makeatother

%% %% Vertical Vectors
%% \def\vector#1{\begin{bmatrix}\vecListA#1,,\end{bmatrix}}
%% \def\vecListA#1,{\if,#1,\else #1\cr \expandafter \vecListA \fi}

%%%%%%%%%%%%%
%% End of vectors
%%%%%%%%%%%%%

%\newcommand{\fullwidth}{}
%\newcommand{\normalwidth}{}



%% makes a snazzy t-chart for evaluating functions
%\newenvironment{tchart}{\rowcolors{2}{}{background!90!textColor}\array}{\endarray}

%%This is to help with formatting on future title pages.
\newenvironment{sectionOutcomes}{}{}



%% Flowchart stuff
%\tikzstyle{startstop} = [rectangle, rounded corners, minimum width=3cm, minimum height=1cm,text centered, draw=black]
%\tikzstyle{question} = [rectangle, minimum width=3cm, minimum height=1cm, text centered, draw=black]
%\tikzstyle{decision} = [trapezium, trapezium left angle=70, trapezium right angle=110, minimum width=3cm, minimum height=1cm, text centered, draw=black]
%\tikzstyle{question} = [rectangle, rounded corners, minimum width=3cm, minimum height=1cm,text centered, draw=black]
%\tikzstyle{process} = [rectangle, minimum width=3cm, minimum height=1cm, text centered, draw=black]
%\tikzstyle{decision} = [trapezium, trapezium left angle=70, trapezium right angle=110, minimum width=3cm, minimum height=1cm, text centered, draw=black]


\title{Counter Examples}

\begin{document}

\begin{abstract}
dealing with all
\end{abstract}
\maketitle




Our favorite word in mathematics is \textbf{\textcolor{blue!55!black}{all}}. \\


We love to make statements that \textbf{all} of the things that match some description, have some characteristic. These can be very difficult statements to prove true, which is why we value them so much. \\


Of course, some of our statements will be true and some of our statements will be false.

\begin{itemize}
\item When the statement is true, we would like to explain how we know it is true.
\item When the statement is false, we would like to explain how we know it is false.
\end{itemize}



When dealing with the word \textbf{all}, it is easier to show that a false statement is false than it is to show a true statement is true. \\


\textbf{\textcolor{blue!55!black}{All, Every, Each}} \\

When a statement makes a claim about \textbf{\textcolor{red!80!black}{all}} of the items in a set and we think the statement is false, then a \textbf{counterexample} is our explanation that the statement is false.



\begin{definition}  \textbf{\textcolor{green!50!black}{Counterexample}} \\


A counterexample is one item in the set where the statement is false.


\end{definition}




\subsection{Counterexamples}

Here is the complete graph of the function $G(x)$. \\


\begin{image}
\begin{tikzpicture}
     \begin{axis}[
               domain=-10:10, ymax=10, xmax=10, ymin=-10, xmin=-10,
               axis lines =center, xlabel=$x$, ylabel=$y$,
                ytick={-10,-8,-6,-4,-2,2,4,6,8,10},
                xtick={-10,-8,-6,-4,-2,2,4,6,8,10},
                yticklabels={$-10$,$-8$,$-6$,$-4$,$-2$,$2$,$4$,$6$,$8$,$10$}, 
                xticklabels={$-10$,$-8$,$-6$,$-4$,$-2$,$2$,$4$,$6$,$8$,$10$},
                ticklabel style={font=\scriptsize},
               every axis y label/.style={at=(current axis.above origin),anchor=south},
               every axis x label/.style={at=(current axis.right of origin),anchor=west},
               axis on top,
                    ]

        
        \addplot [draw=penColor, very thick, smooth, domain=(-6:3), <->] {1/(x-3) + 2};
        \addplot [draw=penColor, very thick, smooth, domain=(3:8)] {1/(x-3) + 2};

        \addplot [line width=0.5, gray, dashed,samples=100,domain=(-9:9)] ({3},{x});
        \addplot [line width=0.5, gray, dashed,samples=100,domain=(-9:9)] ({x},{2});

        \addplot[color=penColor,only marks,mark=*] coordinates{(3.2,7)}; 
        \addplot[color=penColor,only marks,mark=*] coordinates{(8,2.2)}; 


    \end{axis}



\end{tikzpicture}
\end{image}





\textbf{\textcolor{blue!55!black}{Statements about $G(x)$}} \\




\begin{itemize}
\item $G(x) < 0$  for all values of $x$ in the domain.
\item $G(x) < 10$  for all values of $x$ in the domain.
\item $G(x) > 4$  for each value of $x$ in the domain such that $x > 0$.
\item $G(x) < 2$  for every value of $x$ in the domain such that $x < 0$.
\end{itemize}



These are all statements presented through the function values of $G$.  They are either true of false.  If we believe a statement is false, then we are going to provide a counterexample. \\

\textbf{\textcolor{red!80!black}{Counterexamples}} for statements about functions are \textbf{\textcolor{red!80!black}{DOMAIN}} numbers where the statement is false. \\


This follows the general flow of discussions around function analysis.  Questions are posed about the function values (what) and their answers are domain numbers (where). \\


Let's consider the statements above one at a time. \\












\textbf{\textcolor{red!90!darkgray}{$\blacktriangleright$}} $G(x) < 0$  for all values of $x$ in the domain. \\


This statement is false. Our counterexample is $2$.  \\

$G(2) \approx 1 > 0$. The number $2$ is just one number in the domain.   The original statement used the word \textbf{all} and that means that every number in the domain would make the inequality true.  To show \textbf{all} is false, we need only supply one domain number where the inequality is false and then \textbf{all} is false.

The inequality is certainly true for \textbf{some} domain numbers, like, $G(2.9) < 0$, since the point corresponding to $2.9$ sits below the horizontal axis.  However, the original statement said that \textbf{all} of the domain numbers would make the inequality true. They don't \textbf{all} make the inequality true.  We have provided a domain number where the inequality is false. We have provided a counterexample. The stament about \textbf{all} domain values is false.\\









\textbf{\textcolor{red!90!darkgray}{$\blacktriangleright$}} $G(x) < 10$  for all values of $x$ in the domain. \\

The graphs makes this statement seem true.  All of the points are positioned vertically below the height of $10$.  This makes us think that there is no counterexample.  \\

Since we think the statement might be true, we could attempt to explain why we think it is true.  But, we are interested in counterexamples here, so we will just move on.















\textbf{\textcolor{red!90!darkgray}{$\blacktriangleright$}} $G(x) > 4$ for each value of $x$ in the domain such that $x > 0$. \\



``for each'' is just another way of saying ``all''.\\

This statement is a claim about the function values, but not all of the function values.  It refers to the function values on a subset of the domain.  It makes a claim about all of the positive domain numbers. This is still an ``all'' statement.  This ``all'' statement does not cover the whole domain.  It covers a subset of the domain.\\



This statement is false. Our counterexample is $1$.  \\


First, $1$ is a domain number included in the described domain subset, because $1 > 0$. \\


Secondly, $G(1) \approx 1.5 < 4$. \\



$1$ is just one number in the domain subset.   The original statement used the word \textbf{all} and that means that every positive number in the domain would make the inequality true.  To show \textbf{all} is false, we need only supply one number in the domain subset where the inequality is false and then \textbf{all} is false. \\

The inequality is certainly true for \textbf{some} domain numbers, like, $G(3.1) > 4$, since the point corresponding to $3.1$ sits at a height greater than $4$.  However, the original statement said that all of the positive domain numbers would make the inequality true. They don't all make the inequality true.  We have provided a positive domain number where the inequality is false. We have provided a counterexample. \\

Notice that $3$ is not a counterexample.  $3$ is a positive number, but the statement was not about positive numbers.  It referred to positive domain numbers and $3$ is not in the domain.














\textbf{\textcolor{red!90!darkgray}{$\blacktriangleright$}} $G(x) < 2$  for every value of $x$ in the domain such that $x < 0$. \\

The graphs makes this statement seem true.  Not all of the points on the whole graph are positioned vertical below the height of $2$, but that is not what the statement claims. The statement was about function values at negative domain numbers. For negative domain numbers, the corresponding points are below a height of $2$.  This makes us think that there is no counterexample.  \\

Since we think the statement might be true, we could attempt to explain why we think it is true.  But, we are interested in counterexamples here, so we will just move on.








\textbf{$\blacktriangleright$ Note:}  Just because \textit{YOU} might not be able to find a counterexample does not mean the statement is true.  Identifying a counterexample shows that the statement is false. Not finding a counterexample means you cannot draw any conclusions.  Showing a statement is true requires more explanation than just you could not find a counterexample. \\







\begin{example}    \textbf{\textcolor{blue!55!black}{Counterexample}} \\



Let the function $H$ be defined by the formula $H(t) = 4 - (t-5)^2$ with its natural domain. \\

\textbf{\textcolor{blue!55!black}{Claim:}} $H(t) \leq 2$.  \\



This statement claims that $H(t)$ is less than $2$.  And, since it is not stating any conditions on $t$, that means it is true for all values of $t$ in the domain.  It is an ``all'' statement. \\

This statement is false. \\

To show it is false, we provide a counterexample. \\ 


$1$ is a counterexample. \\


$1$ is in the domain and  $H(1) = 4 - (1-5)^2 = 4 - 1^2 = 3 > 2$ \\



Some of the values of $H$ are less than or equal to $2$, but this statement claimed all of the values $H$ are less than or equal to $2$.  We have shown the ``all'' statement to be false, because we have provided a counterexample.



\end{example}










\begin{center}
\textbf{\textcolor{green!50!black}{ooooo-=-=-=-ooOoo-=-=-=-ooooo}} \\

more examples can be found by following this link\\ \link[More Examples of Graphical Language]{https://ximera.osu.edu/csccmathematics/precalculus1/precalculus1/graphicalCommunication/examples/exampleList}

\end{center}





\end{document}
