\documentclass{ximera}


\graphicspath{
  {./}
  {ximeraTutorial/}
  {basicPhilosophy/}
}

\newcommand{\mooculus}{\textsf{\textbf{MOOC}\textnormal{\textsf{ULUS}}}}


\usepackage{tkz-euclide}\usepackage{tikz}
\usepackage{tikz-cd}
\usetikzlibrary{arrows}
\tikzset{>=stealth,commutative diagrams/.cd,
  arrow style=tikz,diagrams={>=stealth}} %% cool arrow head
\tikzset{shorten <>/.style={ shorten >=#1, shorten <=#1 } } %% allows shorter vectors

\usetikzlibrary{backgrounds} %% for boxes around graphs
\usetikzlibrary{shapes,positioning}  %% Clouds and stars
\usetikzlibrary{matrix} %% for matrix
\usepgfplotslibrary{polar} %% for polar plots
\usepgfplotslibrary{fillbetween} %% to shade area between curves in TikZ
\usetkzobj{all}
\usepackage[makeroom]{cancel} %% for strike outs
%\usepackage{mathtools} %% for pretty underbrace % Breaks Ximera
%\usepackage{multicol}
\usepackage{pgffor} %% required for integral for loops



%% http://tex.stackexchange.com/questions/66490/drawing-a-tikz-arc-specifying-the-center
%% Draws beach ball
\tikzset{pics/carc/.style args={#1:#2:#3}{code={\draw[pic actions] (#1:#3) arc(#1:#2:#3);}}}



\usepackage{array}
\setlength{\extrarowheight}{+.1cm}
\newdimen\digitwidth
\settowidth\digitwidth{9}
\def\divrule#1#2{
\noalign{\moveright#1\digitwidth
\vbox{\hrule width#2\digitwidth}}}
























%%This is to help with formatting on future title pages.
\newenvironment{sectionOutcomes}{}{}


\title{Travelling Along Lines}

\begin{document}

\begin{abstract}
movement
\end{abstract}
\maketitle



A Cartesian equation describes a curve as a whole object.  The equation provides a way to validate a point's membership in the curve. You substitute the point's coordinates into the variables in the equaiton.  If the resulting equation is true, then the point is included in the curve. \\

You have to provide the point.  The equation will tell you if it is on the curve or not. \\






$\blacktriangleright$  A parameterization manufactures the points on the curve.  You don't bring a point to the parameterization.  You select a value for the parameter and the parameterization hands the point to you.

In this way, the parameterization constructs the curve.  And, it constructs it with a given direction - the direction of the parameterization.


A parameterization is like highway milemarkers along the curve.




Let's see this in action with our favorite curve: a line.




The Cartesian equation for a line looks like: $y = m \cdot x + b$.  The $x$ value is multiplied by a number and then another number is added to that product.




\begin{example} Parameterizing a Line



Let $L(x) = 2 x - 3$




Graph of $y = L(x)$.

\begin{image}
\begin{tikzpicture}
  \begin{axis}[
            domain=-10:10, ymax=10, xmax=10, ymin=-10, xmin=-10,
            axis lines =center, xlabel=$x$, ylabel=$y$, grid = major, grid style={dashed},
            ytick={-10,-8,-6,-4,-2,2,4,6,8,10},
            xtick={-10,-8,-6,-4,-2,2,4,6,8,10},
            yticklabels={$-10$,$-8$,$-6$,$-4$,$-2$,$2$,$4$,$6$,$8$,$10$}, 
            xticklabels={$-10$,$-8$,$-6$,$-4$,$-2$,$2$,$4$,$6$,$8$,$10$},
            ticklabel style={font=\scriptsize},
            every axis y label/.style={at=(current axis.above origin),anchor=south},
            every axis x label/.style={at=(current axis.right of origin),anchor=west},
            axis on top
          ]
          

            \addplot [line width=2, penColor, smooth,samples=100,domain=(-5:5),<->] {2*x-3};

          %\addplot[color=penColor,fill=penColor2,only marks,mark=*] coordinates{(-6,9)};
          %\addplot[color=penColor,fill=penColor2,only marks,mark=*] coordinates{(2,-7)};




           

  \end{axis}
\end{tikzpicture}
\end{image}





Now for a parameterization.

\begin{itemize}
\item $x(t) = \frac{-3}{10 t}$
\item $y(t) = 2 x(t) - 3 = 2 \left( \frac{-3}{10 t} \right) + 3$ 
\end{itemize}


$x(t)$ and $y(t)$ are still related as $y(t) = 2 x(t) - 3$.  $(x(t), y(t))$ will satisfy the equation $y = 2 x - 3$.  Therefore, these points are on the line.  

But where?


As $t$ moves from $-\infty$ to $\infty$, where does the point go?




\begin{center}
\desmos{fuvzr8153y}{400}{300}
\end{center}






\end{example}




The parameterization produces points on the curve.  But it may not produce every point on the curve as the next parameterization.


















\begin{example} Parameterizing a Line



Let $L(x) = x - 3$




Graph of $y = L(x)$.

\begin{image}
\begin{tikzpicture}
  \begin{axis}[
            domain=-10:10, ymax=10, xmax=10, ymin=-10, xmin=-10,
            axis lines =center, xlabel=$x$, ylabel=$y$, grid = major, grid style={dashed},
            ytick={-10,-8,-6,-4,-2,2,4,6,8,10},
            xtick={-10,-8,-6,-4,-2,2,4,6,8,10},
            yticklabels={$-10$,$-8$,$-6$,$-4$,$-2$,$2$,$4$,$6$,$8$,$10$}, 
            xticklabels={$-10$,$-8$,$-6$,$-4$,$-2$,$2$,$4$,$6$,$8$,$10$},
            ticklabel style={font=\scriptsize},
            every axis y label/.style={at=(current axis.above origin),anchor=south},
            every axis x label/.style={at=(current axis.right of origin),anchor=west},
            axis on top
          ]
          

            \addplot [line width=2, penColor, smooth,samples=100,domain=(-6:10),<->] {x-3};

          %\addplot[color=penColor,fill=penColor2,only marks,mark=*] coordinates{(-6,9)};
          %\addplot[color=penColor,fill=penColor2,only marks,mark=*] coordinates{(2,-7)};




           

  \end{axis}
\end{tikzpicture}
\end{image}





Now for a parameterization.

\begin{itemize}
\item $x(t) = 4 cos(t)$
\item $y(t) = 4 cos(t) - 3$ 
\end{itemize}


$x(t)$ and $y(t)$ are related as $y(t) = x(t) - 3$.  $(x(t), y(t))$ will satisfy the equation $y = x - 3$.  Therefore, these points are on the line.  

But where?


As $t$ moves from $-\infty$ to $\infty$, $cos(t)$ oscillates between $1$ and $-1$.  Our $x$-coordinate will be oscillating between $4$ and $-4$.






\begin{image}
\begin{tikzpicture}
  \begin{axis}[
            domain=-10:10, ymax=10, xmax=10, ymin=-10, xmin=-10,
            axis lines =center, xlabel=$x$, ylabel=$y$, grid = major, grid style={dashed},
            ytick={-10,-8,-6,-4,-2,2,4,6,8,10},
            xtick={-10,-8,-6,-4,-2,2,4,6,8,10},
            yticklabels={$-10$,$-8$,$-6$,$-4$,$-2$,$2$,$4$,$6$,$8$,$10$}, 
            xticklabels={$-10$,$-8$,$-6$,$-4$,$-2$,$2$,$4$,$6$,$8$,$10$},
            ticklabel style={font=\scriptsize},
            every axis y label/.style={at=(current axis.above origin),anchor=south},
            every axis x label/.style={at=(current axis.right of origin),anchor=west},
            axis on top
          ]
          

            \addplot [line width=2, penColor, smooth,samples=100,domain=(-4:4)] {x-3};

          \addplot[color=penColor,fill=penColor,only marks,mark=*] coordinates{(-4,-7)};
          \addplot[color=penColor,fill=penColor,only marks,mark=*] coordinates{(4, 1)};




           

  \end{axis}
\end{tikzpicture}
\end{image}



The point keeps oscillating back and forth on this line segment.










\begin{center}
\desmos{dxxtmpuvej}{400}{300}
\end{center}







\end{example}





































\end{document}
