\documentclass{ximera}


\graphicspath{
  {./}
  {ximeraTutorial/}
  {basicPhilosophy/}
}

\newcommand{\mooculus}{\textsf{\textbf{MOOC}\textnormal{\textsf{ULUS}}}}


\usepackage{tkz-euclide}\usepackage{tikz}
\usepackage{tikz-cd}
\usetikzlibrary{arrows}
\tikzset{>=stealth,commutative diagrams/.cd,
  arrow style=tikz,diagrams={>=stealth}} %% cool arrow head
\tikzset{shorten <>/.style={ shorten >=#1, shorten <=#1 } } %% allows shorter vectors

\usetikzlibrary{backgrounds} %% for boxes around graphs
\usetikzlibrary{shapes,positioning}  %% Clouds and stars
\usetikzlibrary{matrix} %% for matrix
\usepgfplotslibrary{polar} %% for polar plots
\usepgfplotslibrary{fillbetween} %% to shade area between curves in TikZ
\usetkzobj{all}
\usepackage[makeroom]{cancel} %% for strike outs
%\usepackage{mathtools} %% for pretty underbrace % Breaks Ximera
%\usepackage{multicol}
\usepackage{pgffor} %% required for integral for loops



%% http://tex.stackexchange.com/questions/66490/drawing-a-tikz-arc-specifying-the-center
%% Draws beach ball
\tikzset{pics/carc/.style args={#1:#2:#3}{code={\draw[pic actions] (#1:#3) arc(#1:#2:#3);}}}



\usepackage{array}
\setlength{\extrarowheight}{+.1cm}
\newdimen\digitwidth
\settowidth\digitwidth{9}
\def\divrule#1#2{
\noalign{\moveright#1\digitwidth
\vbox{\hrule width#2\digitwidth}}}
























%%This is to help with formatting on future title pages.
\newenvironment{sectionOutcomes}{}{}


\title{Parametric Equations}

\begin{document}

\begin{abstract}
separating dimensions
\end{abstract}
\maketitle












An essential aspect of graphs of functions is that the horizontal axis acts like a slider.  The domain numbers are lined up in order and we can picture a slider moving left to right along the horizontal axis, through the domain numbers, highlighting the points on the graph, which point to corresponding range numbers.





Graph of $y = 0.1 (x-4) (x+2) (x+6)$.

\begin{image}
\begin{tikzpicture}
  \begin{axis}[
            domain=-10:10, ymax=10, xmax=10, ymin=-10, xmin=-10, unit vector ratio*=1 1 1,
            axis lines =center, xlabel=$x$, ylabel=$y$, grid = major, grid style={dashed},
            ytick={-10,-8,-6,-4,-2,2,4,6,8,10},
            xtick={-10,-8,-6,-4,-2,2,4,6,8,10},
            yticklabels={$-10$,$-8$,$-6$,$-4$,$-2$,$2$,$4$,$6$,$8$,$10$}, 
            xticklabels={$-10$,$-8$,$-6$,$-4$,$-2$,$2$,$4$,$6$,$8$,$10$},
            ticklabel style={font=\scriptsize},
            every axis y label/.style={at=(current axis.above origin),anchor=south},
            every axis x label/.style={at=(current axis.right of origin),anchor=west},
            axis on top
          ]
          

		\addplot [line width=2, penColor, smooth,samples=200,domain=(-7.5:5.2),<->] {0.1*(x-4)*(x+2)* (x+6)};




  \end{axis}
\end{tikzpicture}
\end{image}


Picture a slider on the $x$-axis.  As the slider moves from left to right, we can imagine a point following on the graph.  The point comes up from the bottom. As the slider passes $-4$, the point reaches the top of a hill and then begins moving down. As the slider passses $1$, the point reaches the bottom of a valley and begins moving up, which it continues doing.



\begin{center}
\desmos{qjfcripp4k}{400}{300}
\end{center}




This same slider idea does not work with all curves.





The Folium of Descartes





\begin{center}
\desmos{bz5gq9mpkn}{400}{300}
\end{center}


If you picture a slider on the $x$-axis sliding from left to right, then what point does it correspond to when $x = 0.3$? \\


Instead, picture the slider as sliding on the curve itself. What if the curve was another axis with its own variable that counted its own milemarker along the curve? At each milemarker along the curve, the two coordinates are calculuate.





\begin{center}
\desmos{a0tnokkl0o}{400}{300}
\end{center}





On the curve above, where is $-\infty$, $0$, and $\infty$?







\section{Parameterization}




We have been thinking of curves as points of the form $(x, f(x))$. $x$ moves and its value or position dictates the value of $f(x)$ - the $y$-coordinate, which then dictates the placement of the point $(x, f(x))$.


Our new idea is that there is a third number line.  It is out of view as far as the $x$- and $y$-axes are concerned.  Let's say that it represents values of $t$.  As $t$ moves, its value dictates both $x$ and $y$. Then the points $(x(t), y(t))$ are plotted.

In this context, $t$ is called a \textbf{parameter}.  $x$ and $y$ have been \textbf{parameterized}.  $(x(t), y(t))$ is a \textbf{parameterization} of the curve.



\begin{example} The Unit Circle


We have already seen a parameterization of the unit circle: $x^2 + y^2 = 1$.  The parameterization is $(cos(t), sin(t))$.

The \textbf{parametric equations} are

\begin{itemize}
\item $x(t) = \cos(t)$
\item $y(t) = \sin(t)$
\end{itemize}




\begin{center}
\desmos{57shzwa7xi}{400}{300}
\end{center}


In this context, $t$ could be thought of as the angle measured counterclockwise from the positive $x$-axis.  Perhaps $\theta$  would be a better parameter.

\begin{itemize}
\item $x(\theta) = cos(\theta)$
\item $y(\theta) = sin(\theta)$
\end{itemize}





The unit circle is fully traced for $\theta \in [0, 2\pi])$.  The unit circle is traced out many times for all of the values of $\theta$.




\end{example}

















\begin{example} FCayley's Sextic

$4(x^2 + y^2 - x) = 27 (x^2 + y^2)^2$

The parametric equations are

\begin{itemize}
\item $x(t) = 4 \cos \left( \frac{t}{3} \right)^3 \cos(t)$
\item $y(t) = 4 \cos \left( \frac{t}{3} \right)^3 \sin(t)$
\end{itemize}






\begin{center}
\desmos{crpjrrbjyo}{400}{300}
\end{center}




\begin{itemize}
\item Which point corresonds to $t=0$? 
\item Where on the curve is $t$ large and negative?
\item Where on the curve is $t$ large and positive?
\end{itemize}






\end{example}









































\end{document}
