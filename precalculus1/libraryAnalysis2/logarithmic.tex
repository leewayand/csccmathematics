\documentclass{ximera}


\graphicspath{
  {./}
  {ximeraTutorial/}
  {basicPhilosophy/}
}

\newcommand{\mooculus}{\textsf{\textbf{MOOC}\textnormal{\textsf{ULUS}}}}


\usepackage{tkz-euclide}\usepackage{tikz}
\usepackage{tikz-cd}
\usetikzlibrary{arrows}
\tikzset{>=stealth,commutative diagrams/.cd,
  arrow style=tikz,diagrams={>=stealth}} %% cool arrow head
\tikzset{shorten <>/.style={ shorten >=#1, shorten <=#1 } } %% allows shorter vectors

\usetikzlibrary{backgrounds} %% for boxes around graphs
\usetikzlibrary{shapes,positioning}  %% Clouds and stars
\usetikzlibrary{matrix} %% for matrix
\usepgfplotslibrary{polar} %% for polar plots
\usepgfplotslibrary{fillbetween} %% to shade area between curves in TikZ
\usetkzobj{all}
\usepackage[makeroom]{cancel} %% for strike outs
%\usepackage{mathtools} %% for pretty underbrace % Breaks Ximera
%\usepackage{multicol}
\usepackage{pgffor} %% required for integral for loops



%% http://tex.stackexchange.com/questions/66490/drawing-a-tikz-arc-specifying-the-center
%% Draws beach ball
\tikzset{pics/carc/.style args={#1:#2:#3}{code={\draw[pic actions] (#1:#3) arc(#1:#2:#3);}}}



\usepackage{array}
\setlength{\extrarowheight}{+.1cm}
\newdimen\digitwidth
\settowidth\digitwidth{9}
\def\divrule#1#2{
\noalign{\moveright#1\digitwidth
\vbox{\hrule width#2\digitwidth}}}
























%%This is to help with formatting on future title pages.
\newenvironment{sectionOutcomes}{}{}


\title{Logarithmic}

\begin{document}

\begin{abstract}
inverse
\end{abstract}
\maketitle


\section*{Logarithmic Functions}




The expression $\log_a(b)$ was defined to be the number that you raise $a$ to, to get $b$.

\[   a^{\log_r(b)} = b  \]




We made a function from this by fixing the base:  $L(x) = \log_a(x)$. \\


This function is the ``reverse'' of $r^x$.  Meaning that if $(A, B)$ is a pair in the $L(x) = \log_r(x)$ function, then $(B, A)$ is a pair in the $r^x$ function. Their domains and ranges are swapped. \\





\begin{itemize}

\item The domain of an exponential function is all real numbers. The range of a logarithmic function is all real numbers.  


\item The range of an exponential function is all positive real number. The domain of a logarithmic function is all positive real numbers.

\item The graphs of exponential functions have a horizontal asymptote. The graphs of logarithmic functions have a vertical asymptote. 

\end{itemize}






Here is the graph of $y = L(x) = \log_2(x)$.

\begin{image}
\begin{tikzpicture} 
  \begin{axis}[
            domain=-10:10, ymax=10, xmax=10, ymin=-10, xmin=-10,
            axis lines =center, xlabel=$x$, ylabel=$y$,
            every axis y label/.style={at=(current axis.above origin),anchor=south},
            every axis x label/.style={at=(current axis.right of origin),anchor=west},
            axis on top
          ]
          
          \addplot [line width=2, penColor, smooth,samples=200,domain=(0:9),<->] {ln(x)/ln(2)};
          \addplot [line width=1, gray, dashed,domain=(-9:9),<->] ({0},{x});

           

  \end{axis}
\end{tikzpicture}
\end{image}





The graph of a basic logarithmic function has a vertical asymptote where the inside of the logarithm equals $0$.  The domain includes only numbers that make the inside of the formula positive. The function increases over its domain and is unbounded.













\begin{example} Shifted Logarithmic Function



Here is the graph of $y = L(x) = \log_2(5-x)$.

\begin{image}
\begin{tikzpicture} 
  \begin{axis}[
            domain=-10:10, ymax=10, xmax=10, ymin=-10, xmin=-10,
            axis lines =center, xlabel=$x$, ylabel=$y$,
            every axis y label/.style={at=(current axis.above origin),anchor=south},
            every axis x label/.style={at=(current axis.right of origin),anchor=west},
            axis on top
          ]
          
          \addplot [line width=1, gray, dashed,domain=(-9:9),<->] ({5},{x});
          \addplot [line width=2, penColor, smooth,samples=200,domain=(-9:5),<->] {ln((5-x)/ln(2)};
          

           

  \end{axis}
\end{tikzpicture}
\end{image}




The inside of the logarithm here is $5-x$ and this equals $0$ when $x=5$.  Therefore, the vertical asymptote in the graph is $x=5$.  The domain is $(-\infty, 5)$, since these are the numbers that make the inside of the logarithm formula positive.





\end{example}











\subsection*{Behavior}







Our general template for logarithmic functions looks like

\[
log(x) = A \cdot log_r(B \, x + C) + D
\]

Of we choose $e$ as the base, then they look like


\[
log(x) = A \cdot ln(B \, x + C) + D
\]


\begin{itemize}
  \item $A$ is the leading coefficent for the function.
  \item $B$ is the leading coefficent of the inside (the argument).
\end{itemize}


Comparing these back to our basic logarithm functions, we get



\begin{itemize}
  \item $A > 0$ and $B > 0$ gives an increasing logarithmic function.
  \item $A < 0$ and $B > 0$ gives a decreasing logarithmic function.
  \item $A > 0$ and $B < 0$ gives a decreasing logarithmic function.
  \item $A < 0$ and $B < 0$ gives an increasing logarithmic function.
\end{itemize}


$\blacktriangleright$ When the leading coefficients are the same sign, then the logarithmic function is increasing. \\

$\blacktriangleright$ When the leading coefficients are different signs, then the logarithmic function is decreasing. \\






\subsection*{Basic Models}



Just like with exponential functions, it is nice to have a mental model in your head of a basic logarithmic function an its characteristics. \\



And, like with exponential functions, we like $e$ as the base. \\


This gives us a basic basic logarithmic function: $\ln(x)$ \\



And, then three other alternative choices, if you prefer. \\


\[
\ln(x) \, \text{ or } \, \ln(-x) \, \text{ or } \, -\ln(x) \, \text{ or } \, -\ln(-x) 
\]



\begin{image}
\begin{tikzpicture}
    \begin{axis}[name = basictop, 
            domain=-10:10, ymax=10, xmax=10, ymin=-10, xmin=-10,
            axis lines =center, xlabel=$x$, ylabel=$y$, 
            ytick={-10,-8,-6,-4,-2,2,4,6,8,10},
            xtick={-10,-8,-6,-4,-2,2,4,6,8,10},
            ticklabel style={font=\scriptsize},
            every axis y label/.style={at=(current axis.above origin),anchor=south},
            every axis x label/.style={at=(current axis.right of origin),anchor=west},
            axis on top
          ]
          
          \addplot [line width=2, penColor, smooth,samples=300,domain=(0.02:9),<->] {ln(x)};
          \addplot [line width=1, gray, dashed,domain=(-9:9),<->] ({0},{x});
    \end{axis}
    \begin{axis}[at={(basictop.outer east)},anchor=outer west, 
            domain=-10:10, ymax=10, xmax=10, ymin=-10, xmin=-10,
            axis lines =center, xlabel=$x$, ylabel=$y$, 
            ytick={-10,-8,-6,-4,-2,2,4,6,8,10},
            xtick={-10,-8,-6,-4,-2,2,4,6,8,10},
            ticklabel style={font=\scriptsize},
            every axis y label/.style={at=(current axis.above origin),anchor=south},
            every axis x label/.style={at=(current axis.right of origin),anchor=west},
            axis on top
          ]
          
          \addplot [line width=2, penColor, smooth,samples=200,domain=(-9:-0.02),<->] {ln(-x)};
          \addplot [line width=1, gray, dashed,domain=(-9:9),<->] ({0},{x});
    \end{axis}




\end{tikzpicture}
\end{image}






\begin{image}
\begin{tikzpicture}
    \begin{axis}[name = basictop, 
            domain=-10:10, ymax=10, xmax=10, ymin=-10, xmin=-10,
            axis lines =center, xlabel=$x$, ylabel=$y$, 
            ytick={-10,-8,-6,-4,-2,2,4,6,8,10},
            xtick={-10,-8,-6,-4,-2,2,4,6,8,10},
            ticklabel style={font=\scriptsize},
            every axis y label/.style={at=(current axis.above origin),anchor=south},
            every axis x label/.style={at=(current axis.right of origin),anchor=west},
            axis on top
          ]
          
          \addplot [line width=2, penColor, smooth,samples=300,domain=(0.02:9),<->] {-ln(x)};
          \addplot [line width=1, gray, dashed,domain=(-9:9),<->] ({0},{x});
    \end{axis}
    \begin{axis}[at={(basictop.outer east)},anchor=outer west, 
            domain=-10:10, ymax=10, xmax=10, ymin=-10, xmin=-10,
            axis lines =center, xlabel=$x$, ylabel=$y$, 
            ytick={-10,-8,-6,-4,-2,2,4,6,8,10},
            xtick={-10,-8,-6,-4,-2,2,4,6,8,10},
            ticklabel style={font=\scriptsize},
            every axis y label/.style={at=(current axis.above origin),anchor=south},
            every axis x label/.style={at=(current axis.right of origin),anchor=west},
            axis on top
          ]
          
          \addplot [line width=2, penColor, smooth,samples=200,domain=(-9:-0.02),<->] {-ln(-x)};
          \addplot [line width=1, gray, dashed,domain=(-9:9),<->] ({0},{x});
    \end{axis}




\end{tikzpicture}
\end{image}









\begin{center}
\textbf{\textcolor{green!50!black}{ooooo=-=-=-=-=-=-=-=-=-=-=-=-=ooOoo=-=-=-=-=-=-=-=-=-=-=-=-=ooooo}} \\

more examples can be found by following this link\\ \link[More Examples of Analysis]{https://ximera.osu.edu/csccmathematics/precalculus1/precalculus1/libraryAnalysis2/examples/exampleList}

\end{center}




\end{document}
