\documentclass{ximera}


\graphicspath{
  {./}
  {ximeraTutorial/}
  {basicPhilosophy/}
}

\newcommand{\mooculus}{\textsf{\textbf{MOOC}\textnormal{\textsf{ULUS}}}}


\usepackage{tkz-euclide}\usepackage{tikz}
\usepackage{tikz-cd}
\usetikzlibrary{arrows}
\tikzset{>=stealth,commutative diagrams/.cd,
  arrow style=tikz,diagrams={>=stealth}} %% cool arrow head
\tikzset{shorten <>/.style={ shorten >=#1, shorten <=#1 } } %% allows shorter vectors

\usetikzlibrary{backgrounds} %% for boxes around graphs
\usetikzlibrary{shapes,positioning}  %% Clouds and stars
\usetikzlibrary{matrix} %% for matrix
\usepgfplotslibrary{polar} %% for polar plots
\usepgfplotslibrary{fillbetween} %% to shade area between curves in TikZ
\usetkzobj{all}
\usepackage[makeroom]{cancel} %% for strike outs
%\usepackage{mathtools} %% for pretty underbrace % Breaks Ximera
%\usepackage{multicol}
\usepackage{pgffor} %% required for integral for loops



%% http://tex.stackexchange.com/questions/66490/drawing-a-tikz-arc-specifying-the-center
%% Draws beach ball
\tikzset{pics/carc/.style args={#1:#2:#3}{code={\draw[pic actions] (#1:#3) arc(#1:#2:#3);}}}



\usepackage{array}
\setlength{\extrarowheight}{+.1cm}
\newdimen\digitwidth
\settowidth\digitwidth{9}
\def\divrule#1#2{
\noalign{\moveright#1\digitwidth
\vbox{\hrule width#2\digitwidth}}}
























%%This is to help with formatting on future title pages.
\newenvironment{sectionOutcomes}{}{}


\title{Shifted Exponential}

\begin{document}

\begin{abstract}
charcateristics
\end{abstract}
\maketitle












\subsection*{Shifted Exponential Functions}


Shifted Exponential functions are sums of exponential functions constant functions. \\

They are not exponential functions, because constant percentage growth doesn't work with an  added constant. \\

However, as far as function properties go, they are pretty much the same. \\





Our general template for shifted exponential functions looks like

\[
exp(x) = A \cdot r^{B \, x + C} + D
\]

Of we choose $e$ as the base, then they look like


\[
exp(x) = A \cdot e^{B \, x + C} + D
\]


\begin{example} Shifted Exponential Function



Here is the graph of $g(x) = \left(\frac{1}{3}\right)^x + 2$.

\begin{image}
\begin{tikzpicture} 
  \begin{axis}[
            domain=-10:10, ymax=10, xmax=10, ymin=-10, xmin=-10,
            axis lines =center, xlabel=$x$, ylabel=$y$, 
            ytick={-10,-8,-6,-4,-2,2,4,6,8,10},
            xtick={-10,-8,-6,-4,-2,2,4,6,8,10},
            ticklabel style={font=\scriptsize},
            every axis y label/.style={at=(current axis.above origin),anchor=south},
            every axis x label/.style={at=(current axis.right of origin),anchor=west},
            axis on top
          ]
          
          \addplot [line width=1, gray, dashed,domain=(-9:9),<->] ({x},{2});
          \addplot [line width=2, penColor, smooth,samples=200,domain=(-1.8:9),<->] {(0.333)^x + 2};
          

           

  \end{axis}
\end{tikzpicture}
\end{image}


$g$ is the sum of an exponential function, $\left(\frac{1}{3}\right)^x$ and a constant $2$. \\



The exponential function values have all been increased by $2$.   \\


Graphically, the horizontal asymptote in the graph has shifted vertically by 2 units to $y=2$.





\end{example}




\begin{warning}

Exponential functions are functions that exhibit a constant percentage growth rate.  There is some constant $p$, such that

\[
f(x+1) = p \cdot f(x)
\]


All exponential functions CAN be written in the form $f(x) = p^{a \, x + b}$.  All exponenetial functions are a number raise to a linear function.  \\


\textbf{Note:} Since $p = e^{ln(p)}$, every exponential function CAN be written in the form $e^{a \, x + b}$.  So, we really only need study base $e$ exponential functions.


$\blacktriangleright$ Shifted exponential functions


$y(x) = \left(\frac{1}{3}\right)^x + 2$ is not of the form $f(x) = p^{a \, x + b}$. \\  


The added constant term prevents $y(x)$ from being written in our exponential form.  It is not an exponential function. \\

It is a shifted exponential form. \\


However, for our purposes this still fits nicely into our overall exponential story.  So, shifted exponenetial functions are a part of the exponential story, just another chapter. \\

\[
shexpf(x) = A \cdot e^{B \, x + C} + D
\]



\end{warning}



In Calculus, we will see that when you write exponential formulas with base $e$, then nice things happen with the calculations.  So, we like base $e$. \\




\subsection*{Behavior}


Shifted exponential functions behave similarly to exponential functions. \\

\[
shexp(x) = A \cdot e^{B \, x + C} + D
\]





\begin{itemize}
  \item $A$ is the leading coefficent for the function.
  \item $B$ is the leading coefficent of the exponent.
\end{itemize}


Comparing these back to our basic exponential functions, we get



\begin{itemize}
  \item $A > 0$ and $B > 0$ gives an increasing shifted exponential function.
  \item $A < 0$ and $B > 0$ gives a decreasing shifted exponential function.
  \item $A > 0$ and $B < 0$ gives a decreasing eshifted xponential function.
  \item $A < 0$ and $B < 0$ gives an increasing shifted exponential function.
\end{itemize}


$\blacktriangleright$ When the leading coefficients are the same sign, then the shifted exponential function is increasing. \\

$\blacktriangleright$ When the leading coefficients are different signs, then the shifted exponential function is decreasing. \\





\subsection*{End-Behavior}



The big difference between exponential functions and shifted exponential functions is the end-behavior. \



While exponential functions tend to $0$ in one tail of the domain, shifted exponential functions tend to the added constant value. \\



$shexp(x) = A \cdot e^{B \, x + C} + D$ will tend to $D$ in the tail where the exponent is negative. \\


$shexp(x) = A \cdot e^{B \, x + C} + D$ will become unbounded in the tail where the exponent is positive.  The sign will be given by the leading coefficient, $A$. \




\begin{center}
\textbf{\textcolor{green!50!black}{ooooo-=-=-=-ooOoo-=-=-=-ooooo}} \\

more examples can be found by following this link\\ \link[More Examples of Analysis]{https://ximera.osu.edu/csccmathematics/precalculus1/precalculus1/libraryAnalysis2/examples/exampleList}

\end{center}




\end{document}
