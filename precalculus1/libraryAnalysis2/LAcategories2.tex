\documentclass{ximera}

%\usepackage{todonotes}

\newcommand{\todo}{}

\usepackage{esint} % for \oiint
\ifxake%%https://math.meta.stackexchange.com/questions/9973/how-do-you-render-a-closed-surface-double-integral
\renewcommand{\oiint}{{\large\bigcirc}\kern-1.56em\iint}
\fi


\graphicspath{
  {./}
  {ximeraTutorial/}
  {basicPhilosophy/}
  {functionsOfSeveralVariables/}
  {normalVectors/}
  {lagrangeMultipliers/}
  {vectorFields/}
  {greensTheorem/}
  {shapeOfThingsToCome/}
  {dotProducts/}
  {partialDerivativesAndTheGradientVector/}
  {../productAndQuotientRules/exercises/}
  {../normalVectors/exercisesParametricPlots/}
  {../continuityOfFunctionsOfSeveralVariables/exercises/}
  {../partialDerivativesAndTheGradientVector/exercises/}
  {../directionalDerivativeAndChainRule/exercises/}
  {../commonCoordinates/exercisesCylindricalCoordinates/}
  {../commonCoordinates/exercisesSphericalCoordinates/}
  {../greensTheorem/exercisesCurlAndLineIntegrals/}
  {../greensTheorem/exercisesDivergenceAndLineIntegrals/}
  {../shapeOfThingsToCome/exercisesDivergenceTheorem/}
  {../greensTheorem/}
  {../shapeOfThingsToCome/}
  {../separableDifferentialEquations/exercises/}
  {vectorFields/}
}

\newcommand{\mooculus}{\textsf{\textbf{MOOC}\textnormal{\textsf{ULUS}}}}

\usepackage{tkz-euclide}
\usepackage{tikz}
\usepackage{tikz-cd}
\usetikzlibrary{arrows}
\tikzset{>=stealth,commutative diagrams/.cd,
  arrow style=tikz,diagrams={>=stealth}} %% cool arrow head
\tikzset{shorten <>/.style={ shorten >=#1, shorten <=#1 } } %% allows shorter vectors

\usetikzlibrary{backgrounds} %% for boxes around graphs
\usetikzlibrary{shapes,positioning}  %% Clouds and stars
\usetikzlibrary{matrix} %% for matrix
\usepgfplotslibrary{polar} %% for polar plots
\usepgfplotslibrary{fillbetween} %% to shade area between curves in TikZ
%\usetkzobj{all}
\usepackage[makeroom]{cancel} %% for strike outs
%\usepackage{mathtools} %% for pretty underbrace % Breaks Ximera
%\usepackage{multicol}
\usepackage{pgffor} %% required for integral for loops



%% http://tex.stackexchange.com/questions/66490/drawing-a-tikz-arc-specifying-the-center
%% Draws beach ball
\tikzset{pics/carc/.style args={#1:#2:#3}{code={\draw[pic actions] (#1:#3) arc(#1:#2:#3);}}}



\usepackage{array}
\setlength{\extrarowheight}{+.1cm}
\newdimen\digitwidth
\settowidth\digitwidth{9}
\def\divrule#1#2{
\noalign{\moveright#1\digitwidth
\vbox{\hrule width#2\digitwidth}}}




% \newcommand{\RR}{\mathbb R}
% \newcommand{\R}{\mathbb R}
% \newcommand{\N}{\mathbb N}
% \newcommand{\Z}{\mathbb Z}

\newcommand{\sagemath}{\textsf{SageMath}}


%\renewcommand{\d}{\,d\!}
%\renewcommand{\d}{\mathop{}\!d}
%\newcommand{\dd}[2][]{\frac{\d #1}{\d #2}}
%\newcommand{\pp}[2][]{\frac{\partial #1}{\partial #2}}
% \renewcommand{\l}{\ell}
%\newcommand{\ddx}{\frac{d}{\d x}}

% \newcommand{\zeroOverZero}{\ensuremath{\boldsymbol{\tfrac{0}{0}}}}
%\newcommand{\inftyOverInfty}{\ensuremath{\boldsymbol{\tfrac{\infty}{\infty}}}}
%\newcommand{\zeroOverInfty}{\ensuremath{\boldsymbol{\tfrac{0}{\infty}}}}
%\newcommand{\zeroTimesInfty}{\ensuremath{\small\boldsymbol{0\cdot \infty}}}
%\newcommand{\inftyMinusInfty}{\ensuremath{\small\boldsymbol{\infty - \infty}}}
%\newcommand{\oneToInfty}{\ensuremath{\boldsymbol{1^\infty}}}
%\newcommand{\zeroToZero}{\ensuremath{\boldsymbol{0^0}}}
%\newcommand{\inftyToZero}{\ensuremath{\boldsymbol{\infty^0}}}



% \newcommand{\numOverZero}{\ensuremath{\boldsymbol{\tfrac{\#}{0}}}}
% \newcommand{\dfn}{\textbf}
% \newcommand{\unit}{\,\mathrm}
% \newcommand{\unit}{\mathop{}\!\mathrm}
% \newcommand{\eval}[1]{\bigg[ #1 \bigg]}
% \newcommand{\seq}[1]{\left( #1 \right)}
% \renewcommand{\epsilon}{\varepsilon}
% \renewcommand{\phi}{\varphi}


% \renewcommand{\iff}{\Leftrightarrow}

% \DeclareMathOperator{\arccot}{arccot}
% \DeclareMathOperator{\arcsec}{arcsec}
% \DeclareMathOperator{\arccsc}{arccsc}
% \DeclareMathOperator{\si}{Si}
% \DeclareMathOperator{\scal}{scal}
% \DeclareMathOperator{\sign}{sign}


%% \newcommand{\tightoverset}[2]{% for arrow vec
%%   \mathop{#2}\limits^{\vbox to -.5ex{\kern-0.75ex\hbox{$#1$}\vss}}}
% \newcommand{\arrowvec}[1]{{\overset{\rightharpoonup}{#1}}}
% \renewcommand{\vec}[1]{\arrowvec{\mathbf{#1}}}
% \renewcommand{\vec}[1]{{\overset{\boldsymbol{\rightharpoonup}}{\mathbf{#1}}}}

% \newcommand{\point}[1]{\left(#1\right)} %this allows \vector{ to be changed to \vector{ with a quick find and replace
% \newcommand{\pt}[1]{\mathbf{#1}} %this allows \vec{ to be changed to \vec{ with a quick find and replace
% \newcommand{\Lim}[2]{\lim_{\point{#1} \to \point{#2}}} %Bart, I changed this to point since I want to use it.  It runs through both of the exercise and exerciseE files in limits section, which is why it was in each document to start with.

% \DeclareMathOperator{\proj}{\mathbf{proj}}
% \newcommand{\veci}{{\boldsymbol{\hat{\imath}}}}
% \newcommand{\vecj}{{\boldsymbol{\hat{\jmath}}}}
% \newcommand{\veck}{{\boldsymbol{\hat{k}}}}
% \newcommand{\vecl}{\vec{\boldsymbol{\l}}}
% \newcommand{\uvec}[1]{\mathbf{\hat{#1}}}
% \newcommand{\utan}{\mathbf{\hat{t}}}
% \newcommand{\unormal}{\mathbf{\hat{n}}}
% \newcommand{\ubinormal}{\mathbf{\hat{b}}}

% \newcommand{\dotp}{\bullet}
% \newcommand{\cross}{\boldsymbol\times}
% \newcommand{\grad}{\boldsymbol\nabla}
% \newcommand{\divergence}{\grad\dotp}
% \newcommand{\curl}{\grad\cross}
%\DeclareMathOperator{\divergence}{divergence}
%\DeclareMathOperator{\curl}[1]{\grad\cross #1}
% \newcommand{\lto}{\mathop{\longrightarrow\,}\limits}

% \renewcommand{\bar}{\overline}

\colorlet{textColor}{black}
\colorlet{background}{white}
\colorlet{penColor}{blue!50!black} % Color of a curve in a plot
\colorlet{penColor2}{red!50!black}% Color of a curve in a plot
\colorlet{penColor3}{red!50!blue} % Color of a curve in a plot
\colorlet{penColor4}{green!50!black} % Color of a curve in a plot
\colorlet{penColor5}{orange!80!black} % Color of a curve in a plot
\colorlet{penColor6}{yellow!70!black} % Color of a curve in a plot
\colorlet{fill1}{penColor!20} % Color of fill in a plot
\colorlet{fill2}{penColor2!20} % Color of fill in a plot
\colorlet{fillp}{fill1} % Color of positive area
\colorlet{filln}{penColor2!20} % Color of negative area
\colorlet{fill3}{penColor3!20} % Fill
\colorlet{fill4}{penColor4!20} % Fill
\colorlet{fill5}{penColor5!20} % Fill
\colorlet{gridColor}{gray!50} % Color of grid in a plot

\newcommand{\surfaceColor}{violet}
\newcommand{\surfaceColorTwo}{redyellow}
\newcommand{\sliceColor}{greenyellow}




\pgfmathdeclarefunction{gauss}{2}{% gives gaussian
  \pgfmathparse{1/(#2*sqrt(2*pi))*exp(-((x-#1)^2)/(2*#2^2))}%
}


%%%%%%%%%%%%%
%% Vectors
%%%%%%%%%%%%%

%% Simple horiz vectors
\renewcommand{\vector}[1]{\left\langle #1\right\rangle}


%% %% Complex Horiz Vectors with angle brackets
%% \makeatletter
%% \renewcommand{\vector}[2][ , ]{\left\langle%
%%   \def\nextitem{\def\nextitem{#1}}%
%%   \@for \el:=#2\do{\nextitem\el}\right\rangle%
%% }
%% \makeatother

%% %% Vertical Vectors
%% \def\vector#1{\begin{bmatrix}\vecListA#1,,\end{bmatrix}}
%% \def\vecListA#1,{\if,#1,\else #1\cr \expandafter \vecListA \fi}

%%%%%%%%%%%%%
%% End of vectors
%%%%%%%%%%%%%

%\newcommand{\fullwidth}{}
%\newcommand{\normalwidth}{}



%% makes a snazzy t-chart for evaluating functions
%\newenvironment{tchart}{\rowcolors{2}{}{background!90!textColor}\array}{\endarray}

%%This is to help with formatting on future title pages.
\newenvironment{sectionOutcomes}{}{}



%% Flowchart stuff
%\tikzstyle{startstop} = [rectangle, rounded corners, minimum width=3cm, minimum height=1cm,text centered, draw=black]
%\tikzstyle{question} = [rectangle, minimum width=3cm, minimum height=1cm, text centered, draw=black]
%\tikzstyle{decision} = [trapezium, trapezium left angle=70, trapezium right angle=110, minimum width=3cm, minimum height=1cm, text centered, draw=black]
%\tikzstyle{question} = [rectangle, rounded corners, minimum width=3cm, minimum height=1cm,text centered, draw=black]
%\tikzstyle{process} = [rectangle, minimum width=3cm, minimum height=1cm, text centered, draw=black]
%\tikzstyle{decision} = [trapezium, trapezium left angle=70, trapezium right angle=110, minimum width=3cm, minimum height=1cm, text centered, draw=black]


\title{Categories}

\begin{document}

\begin{abstract}
forms
\end{abstract}
\maketitle





We are building a library of the elemntary functions.  The idea is to use the library to list characteristics, features, and aspects of all functions within each category.  \\

That way, if we can identify the type of function we have, then we get free information when analyzing functions. \\

The category becomes our reasoning. \\



\begin{center}

\textbf{\textcolor{red!70!black}{These are ``CAN'' questions.}} \\

\end{center}




\textbf{\textcolor{purple!85!blue}{CAN}} the formula we are given be rewritten as one of the official standard forms for each category? \\







\section*{Official Templates}


These elementary function categories are our \textbf{first choice}.  If a function can be represented by one of these standard forms, then we want to describe the function as one of these elementary functions.  That gives us the most information. \\


\begin{formula} \textbf{\textcolor{blue!55!black}{Power Functions}} 

A \textbf{power function} is any function that \textbf{\textcolor{purple!85!blue}{CAN}} be represented with a formula of the form

\[   pow(x) = k \, x^r      \]

where $k$ and $r$ are real numbers.




\end{formula}









\begin{formula} \textbf{\textcolor{blue!55!black}{Polynomial Functions}} 

A \textbf{polynomial function} is any function that \textbf{\textcolor{purple!85!blue}{CAN}} be represented with a formula of the form

\[    poly(x) = a_n x^n + a_{n-1} x^{n-1} + \cdots + a_3 x^3 + a_2 x^2 + a_1 x^1 + a_0 x^0      \]

where the $a_k$ are real numbers and $a_n \ne 0$. \\


\textbf{Special Polynomials}
\begin{itemize}
\item \textbf{\textcolor{blue!55!black}{Constant:}} $C(x) = C$ 
\item \textbf{\textcolor{blue!55!black}{Linear:}} $L(x) = A \, x + B$ 
\item \textbf{\textcolor{blue!55!black}{Quadratic:}} $Q(x) = A \, x^2 + B \, x + C$ where $A \ne 0$
\end{itemize}



Our order of preference inside the poynomial category is first constant function, then linear function, then quadratic function, then polynomial function.





\end{formula}











\begin{formula} \textbf{\textcolor{blue!55!black}{Rational Functions}} 

A \textbf{rational function} is any function that \textbf{\textcolor{purple!85!blue}{CAN}} be represented with a formula of the form

\[   rat(x) = \frac{ a_n x^n + a_{n-1} x^{n-1} + \cdots + a_3 x^3 + a_2 x^2 + a_1 x + a_0  } { b_m x^m + b_{m-1} x^{m-1} + \cdots + b_3 x^3 + b_2 x^2 + b_1 x + b_0 }   \]



where the $a_k$ and $b_k$ are real numbers and $a_n \ne 0$ and $b_m \ne 0$.





\end{formula}

















\begin{formula} \textbf{\textcolor{blue!55!black}{Radical/Root Functions}} 

A \textbf{radical} or \textbf{root function} is any function that \textbf{\textcolor{purple!85!blue}{CAN}} be represented with a formula of the form  

\[   rad(x) = A \sqrt[n]{B \, x + C} + D =  A (B \, x + C)^{\tfrac{1}{n}} + D    \]

where the $A$, $B$, $C$, and $D$ are real numbers and $A \ne 0$ and $B \ne 0$.

\end{formula}














\begin{formula} \textbf{\textcolor{blue!55!black}{Exponential Functions}}

An \textbf{exponential function} is any function that \textbf{\textcolor{purple!85!blue}{CAN}} be represented with a formula of the form


\[      exp(x) = A \cdot r^{B \, x + C}   \]

where $A$, $B$, and $C$ are real numbers, $A$ is a nonzero real number, and $r$ is a positive real number.

We prefer $e$ as the base. \\


\end{formula}








\begin{formula} \textbf{\textcolor{blue!55!black}{Shifted Exponential Functions}}

A \textbf{shifted exponential function} is any function that \textbf{\textcolor{purple!85!blue}{CAN}} be represented with a formula of the form


\[      shexp(x) = A \cdot r^{B \, x + C} + D   \]

where $A$, $B$, $C$, and $D$ are real numbers, $A \ne 0$ and $B \ne 0$, and $r$ is a positive real number.

We prefer $e$ as the base. \\

\end{formula}











\begin{formula} \textbf{\textcolor{blue!55!black}{Logarithmic Functions}}

A \textbf{logarithmic function} is any function that \textbf{\textcolor{purple!85!blue}{CAN}} be represented with a formula of the form

\[     log(x) =    A \log_r(B \, x + C) +D            \]

where $A$, $B$, $C$, and $D$ are real numbers and $r > 0$.

The domain is all positive real numbers that make the inside positive. \\


We prefer $e$ as the base: $\ln(x)$. \\

\begin{itemize}
	\item If the base $r = 10$, then we use the shorthand $A \log(B \, x + C) + D$.
	\item If the base $r = e$, then we use the shorthand $A \ln(B \, x + C) + D$.
\end{itemize}

\end{formula}













\begin{formula} \textbf{\textcolor{blue!55!black}{Sine Functions}}

A \textbf{sine function} is any function that \textbf{\textcolor{purple!85!blue}{CAN}} be represented with a formula of the form

\[     sin(x) =    A \sin(B \, x + C) + D           \]

where $A$, $B$, $C$, and $D$ are real numbers.


\end{formula}














\begin{formula} \textbf{\textcolor{blue!55!black}{Cosine Functions}}

A \textbf{cosine function} is any function that \textbf{\textcolor{purple!85!blue}{CAN}} be represented with a formula of the form

\[     cos(x) =    A \cos(B \, x + C) + D           \]

where $A$, $B$, $C$, and $D$ are real numbers.


\end{formula}

















\begin{formula} \textbf{\textcolor{blue!55!black}{Absolute Value Functions}}

An \textbf{absolute value function} is any function that \textbf{\textcolor{purple!85!blue}{CAN}} be represented with a formula of the form

\[     abs(x) =    A  | B \, x + C | + D           \]

where $A$, $B$, $C$, and $D$ are real numbers.


\end{formula}


















\section*{Operations}


These elementary function categories are our \textbf{first choice}.  If a function can be represented by one of these standard forms, then we want to describe the function as one of these elementary functions.  That gives us the most information. \\


If a function cannot be identified as one of these elementary functions, then our \textbf{second choice} is an operation. \\




\begin{center}

\textbf{\textcolor{red!70!black}{These are ``IS'' questions.}} \\

\end{center}




\textbf{\textcolor{purple!85!blue}{IS}} the formula we are given written as one of the official standard forms for an operation? \\








Our interpretation of every expression is through the \textbf{\textcolor{purple!85!blue}{Order of Operations}}. \\

Every mathematical expression involving mathematical operations is either

\begin{itemize}
	\item a \textbf{Constant Multiple};
	\item a \textbf{Sum};
	\item a \textbf{Difference};
	\item a \textbf{Product}; or 
	\item a \textbf{Quotient};
\end{itemize}



Each mathematical expression is one of these and only one of these.  The Order of Operations tells us which.













\section*{Composition}


These elementary function categories are our \textbf{first choice}.  If a function can be represented by one of these standard forms, then we want to describe the function as one of these elementary functions.  That gives us the most information. \\


If a function cannot be identified as one of these elementary functions, then our \textbf{second choice} is an operation. \\


If a function is not an elementary function and it is not one of our fouor operations, then our \textbf{third choice} is a \textbf{\textcolor{purple!85!blue}{composition}}.  We study composition is great depth later. \\


If a function is not any of these, then it is weird. \\

Incidentally, we like weird functions.



















\begin{center}
\textbf{\textcolor{green!50!black}{ooooo-=-=-=-ooOoo-=-=-=-ooooo}} \\

more examples can be found by following this link\\ \link[More Examples of Analysis]{https://ximera.osu.edu/csccmathematics/precalculus1/precalculus1/libraryAnalysis1/examples/exampleList}

\end{center}





\end{document}
