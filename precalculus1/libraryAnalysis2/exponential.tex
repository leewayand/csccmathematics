\documentclass{ximera}


\graphicspath{
  {./}
  {ximeraTutorial/}
  {basicPhilosophy/}
}

\newcommand{\mooculus}{\textsf{\textbf{MOOC}\textnormal{\textsf{ULUS}}}}


\usepackage{tkz-euclide}\usepackage{tikz}
\usepackage{tikz-cd}
\usetikzlibrary{arrows}
\tikzset{>=stealth,commutative diagrams/.cd,
  arrow style=tikz,diagrams={>=stealth}} %% cool arrow head
\tikzset{shorten <>/.style={ shorten >=#1, shorten <=#1 } } %% allows shorter vectors

\usetikzlibrary{backgrounds} %% for boxes around graphs
\usetikzlibrary{shapes,positioning}  %% Clouds and stars
\usetikzlibrary{matrix} %% for matrix
\usepgfplotslibrary{polar} %% for polar plots
\usepgfplotslibrary{fillbetween} %% to shade area between curves in TikZ
\usetkzobj{all}
\usepackage[makeroom]{cancel} %% for strike outs
%\usepackage{mathtools} %% for pretty underbrace % Breaks Ximera
%\usepackage{multicol}
\usepackage{pgffor} %% required for integral for loops



%% http://tex.stackexchange.com/questions/66490/drawing-a-tikz-arc-specifying-the-center
%% Draws beach ball
\tikzset{pics/carc/.style args={#1:#2:#3}{code={\draw[pic actions] (#1:#3) arc(#1:#2:#3);}}}



\usepackage{array}
\setlength{\extrarowheight}{+.1cm}
\newdimen\digitwidth
\settowidth\digitwidth{9}
\def\divrule#1#2{
\noalign{\moveright#1\digitwidth
\vbox{\hrule width#2\digitwidth}}}
























%%This is to help with formatting on future title pages.
\newenvironment{sectionOutcomes}{}{}


\title{Exponential}

\begin{document}

\begin{abstract}
attributes
\end{abstract}
\maketitle


\section*{Basic Exponential Functions}

Thinking about formulas, basic exponential functions are functions whose formulas look like

\[
exp(x) = A \cdot r^x \, \text{ with } \, A \ne 0 \, \text{ and } \, r > 1
\]

Just a leading coefficent and a base greater than $1$. \\



\begin{example}

Here is the graph of $y = 2^x$.

\begin{image}
\begin{tikzpicture} 
  \begin{axis}[
            domain=-10:10, ymax=10, xmax=10, ymin=-10, xmin=-10,
            axis lines =center, xlabel=$x$, ylabel=$y$, 
            ytick={-10,-8,-6,-4,-2,2,4,6,8,10},
            xtick={-10,-8,-6,-4,-2,2,4,6,8,10},
            ticklabel style={font=\scriptsize},
            every axis y label/.style={at=(current axis.above origin),anchor=south},
            every axis x label/.style={at=(current axis.right of origin),anchor=west},
            axis on top
          ]
          
          \addplot [line width=2, penColor, smooth,samples=200,domain=(-9:3.2),<->] {2^x};
          \addplot [line width=1, gray, dashed,domain=(-9:9),<->] ({x},{0});

           

  \end{axis}
\end{tikzpicture}
\end{image}


The domain of a basic exponential function is all real numbers, $(-\infty, \infty)$ and it increases on this domain.  \\

Graphically, the horizontal axis is a horizontal asymptote.\\



\[  \lim_{x \to -\infty} 2^x = \answer{0}     \, \text{ and } \,  \lim_{x \to \infty} 2^x = \answer{\infty}   \]


\textbf{Note:}. Using the number $e$ as the base of our basic exponential function is very popular.

\end{example}





The roles are reversed when the base is less than $1$ \\











\begin{example}



Here is the graph of $y = \left(\frac{1}{3}\right)^x$.

\begin{image}
\begin{tikzpicture} 
  \begin{axis}[
            domain=-10:10, ymax=10, xmax=10, ymin=-10, xmin=-10,
            axis lines =center, xlabel=$x$, ylabel=$y$, 
            ytick={-10,-8,-6,-4,-2,2,4,6,8,10},
            xtick={-10,-8,-6,-4,-2,2,4,6,8,10},
            ticklabel style={font=\scriptsize},
            every axis y label/.style={at=(current axis.above origin),anchor=south},
            every axis x label/.style={at=(current axis.right of origin),anchor=west},
            axis on top
          ]
          
          \addplot [line width=2, penColor, smooth,samples=200,domain=(-2:9),<->] {(0.333)^x};
          \addplot [line width=1, gray, dashed,domain=(-9:9),<->] ({x},{0});

           

  \end{axis}
\end{tikzpicture}
\end{image}


When the base is less than $1$, then the whole function decreases.  The function is still unbounded to one side and approaches $0$ on the other.

\[  \lim_{x \to -\infty} \left(\frac{1}{3}\right)^x = \answer{\infty}     \, \text{ and } \,  \lim_{x \to \infty} \left(\frac{1}{3}\right)^x = \answer{0}   \]

\end{example}




Note that

\[
\frac{1}{3} = 3^{-1}
\]




The function  $f(x) = \left(\frac{1}{3}\right)^x$ can be written as 

\[
f(x) = \left(\frac{1}{3}\right)^x = \left( 3^{-1} \right)^x = f(x) = \left( 3 \right)^{-x}
\]




We can view any exponential function with a base less than $1$ as an exponential function with a base greater than $1$ and just change the sign of the exponent. \\








\begin{example}



Here is the graph of $y = \left( 3 \right)^{-x}$.

\begin{image}
\begin{tikzpicture} 
  \begin{axis}[
            domain=-10:10, ymax=10, xmax=10, ymin=-10, xmin=-10,
            axis lines =center, xlabel=$x$, ylabel=$y$, 
            ytick={-10,-8,-6,-4,-2,2,4,6,8,10},
            xtick={-10,-8,-6,-4,-2,2,4,6,8,10},
            ticklabel style={font=\scriptsize},
            every axis y label/.style={at=(current axis.above origin),anchor=south},
            every axis x label/.style={at=(current axis.right of origin),anchor=west},
            axis on top
          ]
          
          \addplot [line width=2, penColor, smooth,samples=200,domain=(-2:9),<->] {(0.333)^x};
          \addplot [line width=1, gray, dashed,domain=(-9:9),<->] ({x},{0});

           

  \end{axis}
\end{tikzpicture}
\end{image}


\end{example}



Instead of using numbers greater or less than $1$, we could always use a base graeter than $1$ and just use negative exponents.  In that case, we might as well use $e$ as our base. \\


This gives us a basic basic exponential function: $e^x$ \\



And, then three other alternative choices, if you prefer. \\


\[
e^x \, \text{ or } \, e^{-x} \, \text{ or } \, -e^{x} \, \text{ or } \, -e^{-x} 
\]








\begin{image}
\begin{tikzpicture}
    \begin{axis}[name = basictop, 
            domain=-10:10, ymax=10, xmax=10, ymin=-10, xmin=-10,
            axis lines =center, xlabel=$x$, ylabel=$y$, 
            ytick={-10,-8,-6,-4,-2,2,4,6,8,10},
            xtick={-10,-8,-6,-4,-2,2,4,6,8,10},
            ticklabel style={font=\scriptsize},
            every axis y label/.style={at=(current axis.above origin),anchor=south},
            every axis x label/.style={at=(current axis.right of origin),anchor=west},
            axis on top
          ]
          
          \addplot [line width=2, penColor, smooth,samples=200,domain=(-9:3.2),<->] {2^x};
          \addplot [line width=1, gray, dashed,domain=(-9:9),<->] ({x},{0});
    \end{axis}
    \begin{axis}[at={(basictop.outer east)},anchor=outer west, 
            domain=-10:10, ymax=10, xmax=10, ymin=-10, xmin=-10,
            axis lines =center, xlabel=$x$, ylabel=$y$, 
            ytick={-10,-8,-6,-4,-2,2,4,6,8,10},
            xtick={-10,-8,-6,-4,-2,2,4,6,8,10},
            ticklabel style={font=\scriptsize},
            every axis y label/.style={at=(current axis.above origin),anchor=south},
            every axis x label/.style={at=(current axis.right of origin),anchor=west},
            axis on top
          ]
          
          \addplot [line width=2, penColor, smooth,samples=200,domain=(-3.2:9),<->] {2^(-x)};
          \addplot [line width=1, gray, dashed,domain=(-9:9),<->] ({x},{0});
    \end{axis}




\end{tikzpicture}
\end{image}









\begin{image}
\begin{tikzpicture}
    \begin{axis}[name = basictop, 
            domain=-10:10, ymax=10, xmax=10, ymin=-10, xmin=-10,
            axis lines =center, xlabel=$x$, ylabel=$y$, 
            ytick={-10,-8,-6,-4,-2,2,4,6,8,10},
            xtick={-10,-8,-6,-4,-2,2,4,6,8,10},
            ticklabel style={font=\scriptsize},
            every axis y label/.style={at=(current axis.above origin),anchor=south},
            every axis x label/.style={at=(current axis.right of origin),anchor=west},
            axis on top
          ]
          
          \addplot [line width=2, penColor, smooth,samples=200,domain=(-9:3.2),<->] {-(2^x)};
          \addplot [line width=1, gray, dashed,domain=(-9:9),<->] ({x},{0});
    \end{axis}
    \begin{axis}[at={(basictop.outer east)},anchor=outer west, 
            domain=-10:10, ymax=10, xmax=10, ymin=-10, xmin=-10,
            axis lines =center, xlabel=$x$, ylabel=$y$, 
            ytick={-10,-8,-6,-4,-2,2,4,6,8,10},
            xtick={-10,-8,-6,-4,-2,2,4,6,8,10},
            ticklabel style={font=\scriptsize},
            every axis y label/.style={at=(current axis.above origin),anchor=south},
            every axis x label/.style={at=(current axis.right of origin),anchor=west},
            axis on top
          ]
          
          \addplot [line width=2, penColor, smooth,samples=200,domain=(-3.2:9),<->] {-(2^(-x))};
          \addplot [line width=1, gray, dashed,domain=(-9:9),<->] ({x},{0});
    \end{axis}




\end{tikzpicture}
\end{image}







Basic exponential functions become unbounded in one direction, while approaching $0$ in the other direction. \\


\begin{itemize}
  \item When the base is greater than $1$, exponential functions tend to $0$ in the direction that makes their exponent negative. \\
  \item When the base is less than $1$, exponential functions tend to $0$ in the direction that makes their exponent positive. \\
\end{itemize}

Expoential functions are unbounded in the other direction. \\


Exponential funcitons can be unbounded positively or negatively. This is determined by the sign of the leading coefficient. \\




Those are our basic exponential functions.  Pick one as your own basic exponential function. \\



$exp(x) = e^x$ is very popular. \\


Then compare all other exponential functions to it.





\subsection*{General Exponential Functions}



Our general template for exponential functions looks like

\[
exp(x) = A \cdot r^{B \, x + C}
\]

Of we choose $e$ as the base, then they look like


\[
exp(x) = A \cdot e^{B \, x + C}
\]


\begin{itemize}
  \item $A$ is the leading coefficent for the function.
  \item $B$ is the leading coefficent of the exponent.
\end{itemize}


Comparing these back to our basic exponential functions, we get



\begin{itemize}
  \item $A > 0$ and $B > 0$ gives an increasing exponential function.
  \item $A < 0$ and $B > 0$ gives a decreasing exponential function.
  \item $A > 0$ and $B < 0$ gives a decreasing exponential function.
  \item $A < 0$ and $B < 0$ gives an increasing exponential function.
\end{itemize}


$\blacktriangleright$ When the leading coefficients are the same sign, then the exponential function is increasing. \\

$\blacktriangleright$ When the leading coefficients are different signs, then the exponential function is decreasing. \\




\subsection*{Graphically}

Even though the graphs of exponential functions appear to increase quite quickly, there are no vertical asymptotes.  The domains include all real numbers.  The function just increases or decreases very quickly and continues to do so.


















\begin{example}



Here is the graph of $f(x) = -\left( 3 \right)^{-x + 1}$.

\begin{image}
\begin{tikzpicture} 
  \begin{axis}[
            domain=-10:10, ymax=10, xmax=10, ymin=-10, xmin=-10,
            axis lines =center, xlabel=$x$, ylabel=$y$, 
            ytick={-10,-8,-6,-4,-2,2,4,6,8,10},
            xtick={-10,-8,-6,-4,-2,2,4,6,8,10},
            ticklabel style={font=\scriptsize},
            every axis y label/.style={at=(current axis.above origin),anchor=south},
            every axis x label/.style={at=(current axis.right of origin),anchor=west},
            axis on top
          ]
          
          \addplot [line width=2, penColor, smooth,samples=200,domain=(-6.75:9),<->] {-(3)^(-x-5)};
          \addplot [line width=1, gray, dashed,domain=(-9:9),<->] ({x},{0});

           

  \end{axis}
\end{tikzpicture}
\end{image}



This matches our template for an exponential function. \\


\begin{itemize}
  \item The leading coefficient is $-1$, which is negative. 
  \item The leading coefficient of the exponent is $-1$, which is negative. 
\end{itemize}

Both leading coefficients have the same sign, so this exponential function is increasing. \\



The leading coefficient of the function is $-1$, which is negative. Therefore, the function only has negative values. \\


The exponent is $-x-5$.  This is negative in the positive tail of the domain. The function tends to $0$ in this direction. The function becomes unbounded in the other direction. Unbounded negatively. \\




\[
\lim\limits_{x \to \infty} f(x) = 0 
\]


\[
\lim\limits_{x \to -\infty} f(x) = -\infty 
\]

\end{example}


















\begin{center}
\textbf{\textcolor{green!50!black}{ooooo-=-=-=-ooOoo-=-=-=-ooooo}} \\

more examples can be found by following this link\\ \link[More Examples of Analysis]{https://ximera.osu.edu/csccmathematics/precalculus1/precalculus1/libraryAnalysis2/examples/exampleList}

\end{center}




\end{document}
