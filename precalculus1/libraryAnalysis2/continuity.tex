\documentclass{ximera}


\graphicspath{
  {./}
  {ximeraTutorial/}
  {basicPhilosophy/}
}

\newcommand{\mooculus}{\textsf{\textbf{MOOC}\textnormal{\textsf{ULUS}}}}


\usepackage{tkz-euclide}\usepackage{tikz}
\usepackage{tikz-cd}
\usetikzlibrary{arrows}
\tikzset{>=stealth,commutative diagrams/.cd,
  arrow style=tikz,diagrams={>=stealth}} %% cool arrow head
\tikzset{shorten <>/.style={ shorten >=#1, shorten <=#1 } } %% allows shorter vectors

\usetikzlibrary{backgrounds} %% for boxes around graphs
\usetikzlibrary{shapes,positioning}  %% Clouds and stars
\usetikzlibrary{matrix} %% for matrix
\usepgfplotslibrary{polar} %% for polar plots
\usepgfplotslibrary{fillbetween} %% to shade area between curves in TikZ
\usetkzobj{all}
\usepackage[makeroom]{cancel} %% for strike outs
%\usepackage{mathtools} %% for pretty underbrace % Breaks Ximera
%\usepackage{multicol}
\usepackage{pgffor} %% required for integral for loops



%% http://tex.stackexchange.com/questions/66490/drawing-a-tikz-arc-specifying-the-center
%% Draws beach ball
\tikzset{pics/carc/.style args={#1:#2:#3}{code={\draw[pic actions] (#1:#3) arc(#1:#2:#3);}}}



\usepackage{array}
\setlength{\extrarowheight}{+.1cm}
\newdimen\digitwidth
\settowidth\digitwidth{9}
\def\divrule#1#2{
\noalign{\moveright#1\digitwidth
\vbox{\hrule width#2\digitwidth}}}
























%%This is to help with formatting on future title pages.
\newenvironment{sectionOutcomes}{}{}


\title{Continuity}

\begin{document}

\begin{abstract}
no breaks
\end{abstract}
\maketitle





One of the items in our official analysis list is \textbf{continuity}. \\

Graphically, continuity looks like no breaks in the graph. \\

Algebraically, it is not so easy to describe. \\


The algebraic description will need to dsecribe somehting our eyes see very easily, but not so easy to put into words, or algebra. \\



\begin{center}
\textbf{\textcolor{red!70!black}{Luckily, all of our elementary functions are continuous functions.}}   \\
\end{center}





\begin{itemize}
	\item \textbf{\textcolor{blue!55!black}{Power Functions}} 
	\item \textbf{\textcolor{blue!55!black}{Polynomials}}
	\begin{itemize}
	\end{itemize}
	\item \textbf{\textcolor{blue!55!black}{Rational Functions}}
	\item \textbf{\textcolor{blue!55!black}{Radical Functions}}
	\item \textbf{\textcolor{blue!55!black}{Exponential Functions}}
	\item \textbf{\textcolor{blue!55!black}{Shifted Exponential Functions}}
	\item \textbf{\textcolor{blue!55!black}{Logarithmic Functions}}
	\item \textbf{\textcolor{blue!55!black}{Sine Functions}}
	\item \textbf{\textcolor{blue!55!black}{Cosine Functions}}
	\item \textbf{\textcolor{blue!55!black}{Tangent Functions}}
	\item \textbf{\textcolor{blue!55!black}{Absolute Value Functions}}
\end{itemize}





\begin{center}
\textbf{\textcolor{blue!55!black}{They are all continuous functions.}}   \\
\end{center}

















\subsection*{Operations}


Creating functions through our normal operations, maintains continuity. \\

\begin{itemize}
	\item a \textbf{Sum};
	\item a \textbf{Difference};
	\item a \textbf{Product}; or 
	\item a \textbf{Quotient};
\end{itemize}


\begin{center}
\textbf{\textcolor{blue!55!black}{They are all continuous functions.}}   \\
\end{center}












\subsection*{Composition}


Creating functions through composition, maintains continuity. \\





\begin{center}
\textbf{\textcolor{blue!55!black}{Compositions of continuous functions are continuous functions.}}   \\
\end{center}











\section*{Discontinuities}


If all of our elementary functions are continuous, and all of their combinations and compositions are continuous, then how are we going to study discontinuities? \\


The only way to study discontinuities is to create them with piecewise defined functions. \\

So, that's what we'll do. \\


And, that is why piecewise defined functions are important to us.  \\

They are our only tool for studyig discontinuities.






























\begin{center}
\textbf{\textcolor{green!50!black}{ooooo-=-=-=-ooOoo-=-=-=-ooooo}} \\

more examples can be found by following this link\\ \link[More Examples of Analysis]{https://ximera.osu.edu/csccmathematics/precalculus1/precalculus1/libraryAnalysis1/examples/exampleList}

\end{center}





\end{document}
