\documentclass{ximera}


\graphicspath{
  {./}
  {ximeraTutorial/}
  {basicPhilosophy/}
}

\newcommand{\mooculus}{\textsf{\textbf{MOOC}\textnormal{\textsf{ULUS}}}}


\usepackage{tkz-euclide}\usepackage{tikz}
\usepackage{tikz-cd}
\usetikzlibrary{arrows}
\tikzset{>=stealth,commutative diagrams/.cd,
  arrow style=tikz,diagrams={>=stealth}} %% cool arrow head
\tikzset{shorten <>/.style={ shorten >=#1, shorten <=#1 } } %% allows shorter vectors

\usetikzlibrary{backgrounds} %% for boxes around graphs
\usetikzlibrary{shapes,positioning}  %% Clouds and stars
\usetikzlibrary{matrix} %% for matrix
\usepgfplotslibrary{polar} %% for polar plots
\usepgfplotslibrary{fillbetween} %% to shade area between curves in TikZ
\usetkzobj{all}
\usepackage[makeroom]{cancel} %% for strike outs
%\usepackage{mathtools} %% for pretty underbrace % Breaks Ximera
%\usepackage{multicol}
\usepackage{pgffor} %% required for integral for loops



%% http://tex.stackexchange.com/questions/66490/drawing-a-tikz-arc-specifying-the-center
%% Draws beach ball
\tikzset{pics/carc/.style args={#1:#2:#3}{code={\draw[pic actions] (#1:#3) arc(#1:#2:#3);}}}



\usepackage{array}
\setlength{\extrarowheight}{+.1cm}
\newdimen\digitwidth
\settowidth\digitwidth{9}
\def\divrule#1#2{
\noalign{\moveright#1\digitwidth
\vbox{\hrule width#2\digitwidth}}}
























%%This is to help with formatting on future title pages.
\newenvironment{sectionOutcomes}{}{}


\title{Composition}

\begin{document}

\begin{abstract}
%Stuff can go here later if we want!
\end{abstract}
\maketitle













Composition is an operation on functions.  It takes two functions and produces a third function.


\[  (Outside \circ Inside)(d) = Outside(Inside(d))    \]



We have explored this idea with linear functions.  However, we can compose any two functions.






Normally, we think of running along the real line from left to right and these domain numbers are connected with their $Outside$ function values.  It is a very orderly process.

Now the $Inside$ will be providing the domain numbers to the $Outside$ function and it may not be orderly. The values of the $Inside$ function may increase and decrease, which, in turn, means the domain numbers for $Outside$ may get repeated or run backwards.   The $Iside$ function takes us on a whacky ride through the domain of $Outside$, possibly repeating domain numbers and function values.


In this section, we will map the whole process.
















\begin{sectionOutcomes}
In this section, students will 

\begin{itemize}
\item compose functions.
\end{itemize}
\end{sectionOutcomes}

\end{document}
