\documentclass{ximera}


\graphicspath{
  {./}
  {ximeraTutorial/}
  {basicPhilosophy/}
}

\newcommand{\mooculus}{\textsf{\textbf{MOOC}\textnormal{\textsf{ULUS}}}}


\usepackage{tkz-euclide}\usepackage{tikz}
\usepackage{tikz-cd}
\usetikzlibrary{arrows}
\tikzset{>=stealth,commutative diagrams/.cd,
  arrow style=tikz,diagrams={>=stealth}} %% cool arrow head
\tikzset{shorten <>/.style={ shorten >=#1, shorten <=#1 } } %% allows shorter vectors

\usetikzlibrary{backgrounds} %% for boxes around graphs
\usetikzlibrary{shapes,positioning}  %% Clouds and stars
\usetikzlibrary{matrix} %% for matrix
\usepgfplotslibrary{polar} %% for polar plots
\usepgfplotslibrary{fillbetween} %% to shade area between curves in TikZ
\usetkzobj{all}
\usepackage[makeroom]{cancel} %% for strike outs
%\usepackage{mathtools} %% for pretty underbrace % Breaks Ximera
%\usepackage{multicol}
\usepackage{pgffor} %% required for integral for loops



%% http://tex.stackexchange.com/questions/66490/drawing-a-tikz-arc-specifying-the-center
%% Draws beach ball
\tikzset{pics/carc/.style args={#1:#2:#3}{code={\draw[pic actions] (#1:#3) arc(#1:#2:#3);}}}



\usepackage{array}
\setlength{\extrarowheight}{+.1cm}
\newdimen\digitwidth
\settowidth\digitwidth{9}
\def\divrule#1#2{
\noalign{\moveright#1\digitwidth
\vbox{\hrule width#2\digitwidth}}}
























%%This is to help with formatting on future title pages.
\newenvironment{sectionOutcomes}{}{}


\title{Assemble}

\begin{document}

\begin{abstract}
putting together
\end{abstract}
\maketitle






Let $Out(x) = 2|x-3|+1$ with its natural or implied domain. \\
Let $In(t) = -(t-1)^2 + 5$ with its natural or implied domain. \\









Graph of $y = Out(x) =2|x-3|+1$

\begin{image}
\begin{tikzpicture}
  \begin{axis}[
            domain=-10:10, ymax=10, xmax=10, ymin=-10, xmin=-10,
            axis lines =center, xlabel=$x$, ylabel=$y$, grid = major,
            ytick={-10,-8,-6,-4,-2,2,4,6,8,10},
            xtick={-10,-8,-6,-4,-2,2,4,6,8,10},
            yticklabels={$-10$,$-8$,$-6$,$-4$,$-2$,$2$,$4$,$6$,$8$,$10$}, 
            xticklabels={$-10$,$-8$,$-6$,$-4$,$-2$,$2$,$4$,$6$,$8$,$10$},
            ticklabel style={font=\scriptsize},
            every axis y label/.style={at=(current axis.above origin),anchor=south},
            every axis x label/.style={at=(current axis.right of origin),anchor=west},
            axis on top
          ]
          
          %\addplot [line width=2, penColor2, smooth,samples=100,domain=(-6:2)] {-2*x-3};
			\addplot [line width=2, penColor, smooth,samples=100,domain=(-1:7),<->] {2*abs(x-3)+1};

          %\addplot[color=penColor,fill=penColor2,only marks,mark=*] coordinates{(-6,9)};
          %\addplot[color=penColor,fill=penColor2,only marks,mark=*] coordinates{(2,-7)};



           

  \end{axis}
\end{tikzpicture}
\end{image}


Graph of $z = In(t) = -(t-1)^2 + 5$





\begin{image}
\begin{tikzpicture}
  \begin{axis}[
            domain=-10:10, ymax=10, xmax=10, ymin=-10, xmin=-10,
            axis lines =center, xlabel=$t$, ylabel=$z$, grid = major,
            ytick={-10,-8,-6,-4,-2,2,4,6,8,10},
            xtick={-10,-8,-6,-4,-2,2,4,6,8,10},
            yticklabels={$-10$,$-8$,$-6$,$-4$,$-2$,$2$,$4$,$6$,$8$,$10$}, 
            xticklabels={$-10$,$-8$,$-6$,$-4$,$-2$,$2$,$4$,$6$,$8$,$10$},
            ticklabel style={font=\scriptsize},
            every axis y label/.style={at=(current axis.above origin),anchor=south},
            every axis x label/.style={at=(current axis.right of origin),anchor=west},
            axis on top
          ]
          
          %\addplot [line width=2, penColor2, smooth,samples=100,domain=(-6:2)] {-2*x-3};
			\addplot [line width=2, penColor, smooth,samples=100,domain=(-2.5:4.5),<->] {-(x-1)^2 + 5)};

          %\addplot[color=penColor,fill=penColor2,only marks,mark=*] coordinates{(-6,9)};
          %\addplot[color=penColor,fill=penColor2,only marks,mark=*] coordinates{(2,-7)};



           

  \end{axis}
\end{tikzpicture}
\end{image}





Now to examine the composition. \\



$\blacktriangleright$  First, the inside function: $In(t) = -(t-1)^2 + 5$ \\

The implied domain of $In(t)$ is the whole real line. As we move from left to right along the real line, the domain numbers increase from $-\infty$ to $\infty$.

The corresponding movement in the range has the function values increasing from $-\infty$ to $5$.  The maximum function value $5$ occurs at $1$ in the domain.  As we keep moving beyond $1$ in the domain, the corresponding movement in the range is for the function values to decrease from $1$ to $-\infty$.


This movement in the range of $In(t)$ becomes the movement in the domain of $Out(x)$. \\


$\blacktriangleright$ Movement inside the domain of $Out(x)$.\\



Inside the implied domain of $Out(x) =2|x-3|+1$. the domain numbers will increase from $-\infty$ to $5$.  Then they will decrease from $5$ back down to $-\infty$. \\

On our graph of $Out(x)$, we will move from the far left toward the right until we reach $5$.  Then we will turn around and move toward the left.\\



$\blacktriangleright$ Put those together. \\

The whole composition, $(Out \circ In)$, never sees the inside speeding up, slowing down, turning around, running in circles, or any other swivelling and swirling.  It just sees the normal domain movement left to right from $-\infty$ to $\infty$.  And, it sees whatever function values from $Out$ that the come out of the process.  We know there is some switching going on in the middle.

We have the input to $Out$ moving from $-\infty$ up to $5$ and then back down to $-\infty$.  What are the corresponding function values for $Out$?





\begin{itemize}

\item As the domain numbers move through $(-\infty, 5]$, The values of $Out$ are decreasing from $\infty$ to $1$.  This is the corner that occurs at $3$.  So, we really should look at $(-\infty, 5]$  as  $(-\infty, 3] \cup [3,5]$

	\begin{itemize}[label=$\star$]
		\item On $(-\infty, 3]$, $Out$ decreases from $-\infty$ to $1$.

		\item On $[3,5]$, $Out$ increases from $1$ to $5$

	\end{itemize}


\item The reverse happens as the domain numbers move backwards through $(\infty. 5])$, The values of $Out$ are decreasing from $5$ to $1$.  This is the corner that occurs at $3$.  So, we really should look at $(-\infty, 5]$  as  $(-\infty, 3] \cup [3,5]$

	\begin{itemize}[label=$\star$]
		\item As the domain moves from $5$ to $3$, $Out$ decreases from $5$ to $1$.

		\item As the domain moves from $3$ to $-\infty$, $Out$ increases from $1$ to $\infty$.

	\end{itemize}

\end{itemize}

Since we reverse our travelling direction inside the domain, we hit the corner twice.  Plus, right at our reversal, we will create a hill in the graph made by our own retracing of the domain steps.




















Graph of $w = Out(In(m)) =2|(-(m-1)^2 + 5)-3|+1 = 2|-(m-1)^2 + 2| + 1$







\begin{image}
\begin{tikzpicture}
  \begin{axis}[
            domain=-10:10, ymax=10, xmax=10, ymin=-10, xmin=-10,
            axis lines =center, xlabel=$x$, ylabel=$y$, grid = major,
            ytick={-10,-8,-6,-4,-2,2,4,6,8,10},
            xtick={-10,-8,-6,-4,-2,2,4,6,8,10},
            yticklabels={$-10$,$-8$,$-6$,$-4$,$-2$,$2$,$4$,$6$,$8$,$10$}, 
            xticklabels={$-10$,$-8$,$-6$,$-4$,$-2$,$2$,$4$,$6$,$8$,$10$},
            ticklabel style={font=\scriptsize},
            every axis y label/.style={at=(current axis.above origin),anchor=south},
            every axis x label/.style={at=(current axis.right of origin),anchor=west},
            axis on top
          ]
          	\addplot [line width=2, penColor, smooth,samples=100,domain=(-1.5:3.5),<->] {2*abs(-(x-1)^2 + 2)+1};

   

  \end{axis}
\end{tikzpicture}
\end{image}



















\begin{example}  Composition



Let $Out(x) = -2|x-3|+2$ with its natural or implied domain. \\
Let $In(t) = 2 \sin(t)+2$ with its natural or implied domain. \\






Graph of $y = Out(x) = -2|x-3|+2$

\begin{image}
\begin{tikzpicture}
  \begin{axis}[
            domain=-10:10, ymax=10, xmax=10, ymin=-10, xmin=-10,
            axis lines =center, xlabel=$x$, ylabel=$y$, grid = major,
            ytick={-10,-8,-6,-4,-2,2,4,6,8,10},
            xtick={-10,-8,-6,-4,-2,2,4,6,8,10},
            yticklabels={$-10$,$-8$,$-6$,$-4$,$-2$,$2$,$4$,$6$,$8$,$10$}, 
            xticklabels={$-10$,$-8$,$-6$,$-4$,$-2$,$2$,$4$,$6$,$8$,$10$},
            ticklabel style={font=\scriptsize},
            every axis y label/.style={at=(current axis.above origin),anchor=south},
            every axis x label/.style={at=(current axis.right of origin),anchor=west},
            axis on top
          ]
          
          %\addplot [line width=2, penColor2, smooth,samples=100,domain=(-6:2)] {-2*x-3};
      \addplot [line width=2, penColor, smooth,samples=100,domain=(-3:9),<->] {-2*abs(x-3)+2};

          %\addplot[color=penColor,fill=penColor2,only marks,mark=*] coordinates{(-6,9)};
          %\addplot[color=penColor,fill=penColor2,only marks,mark=*] coordinates{(2,-7)};



           

  \end{axis}
\end{tikzpicture}
\end{image}


Graph of $z = In(t) = 2 \sin(t)+2$





\begin{image}
\begin{tikzpicture}
  \begin{axis}[
            domain=-10:10, ymax=10, xmax=10, ymin=-10, xmin=-10,
            axis lines =center, xlabel=$t$, ylabel=$z$, grid = major,
            ytick={-10,-8,-6,-4,-2,2,4,6,8,10},
            xtick={-10,-8,-6,-4,-2,2,4,6,8,10},
            yticklabels={$-10$,$-8$,$-6$,$-4$,$-2$,$2$,$4$,$6$,$8$,$10$}, 
            xticklabels={$-10$,$-8$,$-6$,$-4$,$-2$,$2$,$4$,$6$,$8$,$10$},
            ticklabel style={font=\scriptsize},
            every axis y label/.style={at=(current axis.above origin),anchor=south},
            every axis x label/.style={at=(current axis.right of origin),anchor=west},
            axis on top
          ]
          
          %\addplot [line width=2, penColor2, smooth,samples=100,domain=(-6:2)] {-2*x-3};
      \addplot [line width=2, penColor, smooth,samples=100,domain=(-9:9),<->] {2*sin(deg(x))+2};

          %\addplot[color=penColor,fill=penColor2,only marks,mark=*] coordinates{(-6,9)};
          %\addplot[color=penColor,fill=penColor2,only marks,mark=*] coordinates{(2,-7)};



           

  \end{axis}
\end{tikzpicture}
\end{image}




The input into $In(t)$ is the whole real line.  We naturally think of moving left to right, from $-\infty$ to $\infty$.  As we do this, the outputs from $In(t)$ oscillate between $0$ and $\answer{4}$.  Therefore, the inputs into $Out(x)$ osciallte back and forth along the interval $\left( \answer{0}, \answer{4} \right)$.





The input into $Out(x)$ keep going back and forth acrtoss the interval $(0,4)$. 



\begin{image}
\begin{tikzpicture}
  \begin{axis}[
            domain=-10:10, ymax=10, xmax=10, ymin=-10, xmin=-10,
            axis lines =center, xlabel=$x$, ylabel=$y$, grid = major,
            ytick={-10,-8,-6,-4,-2,2,4,6,8,10},
            xtick={-10,-8,-6,-4,-2,2,4,6,8,10},
            yticklabels={$-10$,$-8$,$-6$,$-4$,$-2$,$2$,$4$,$6$,$8$,$10$}, 
            xticklabels={$-10$,$-8$,$-6$,$-4$,$-2$,$2$,$4$,$6$,$8$,$10$},
            ticklabel style={font=\scriptsize},
            every axis y label/.style={at=(current axis.above origin),anchor=south},
            every axis x label/.style={at=(current axis.right of origin),anchor=west},
            axis on top
          ]
          
          %\addplot [line width=2, penColor2, smooth,samples=100,domain=(-6:2)] {-2*x-3};
      \addplot [line width=2, penColor, smooth,samples=100,domain=(-3:9),<->] {-2*abs(x-3)+2};

      \addplot [line width=1, penColor2, smooth,samples=300,domain=(-9:9),<->] ({2*sin(deg(x))+2},{0.2*x});

          %\addplot[color=penColor,fill=penColor2,only marks,mark=*] coordinates{(-6,9)};
          %\addplot[color=penColor,fill=penColor2,only marks,mark=*] coordinates{(2,-7)};



           

  \end{axis}
\end{tikzpicture}
\end{image}







Therefore, that part of the graph of $Out$ just keeps repeating. The smooth rounding of the sine curve will round out the corners of the absolute value graph.







\begin{image}
\begin{tikzpicture}
  \begin{axis}[
            domain=-10:10, ymax=10, xmax=10, ymin=-10, xmin=-10,
            axis lines =center, xlabel=$x$, ylabel=$y$, grid = major,
            ytick={-10,-8,-6,-4,-2,2,4,6,8,10},
            xtick={-10,-8,-6,-4,-2,2,4,6,8,10},
            yticklabels={$-10$,$-8$,$-6$,$-4$,$-2$,$2$,$4$,$6$,$8$,$10$}, 
            xticklabels={$-10$,$-8$,$-6$,$-4$,$-2$,$2$,$4$,$6$,$8$,$10$},
            ticklabel style={font=\scriptsize},
            every axis y label/.style={at=(current axis.above origin),anchor=south},
            every axis x label/.style={at=(current axis.right of origin),anchor=west},
            axis on top
          ]
          
          %\addplot [line width=2, penColor2, smooth,samples=100,domain=(-6:2)] {-2*x-3};
      \addplot [line width=2, penColor, smooth,samples=100,domain=(-9:9),<->] {-2*abs((2*sin(deg(x))+2)-3)+2};

      %\addplot [line width=1, penColor2, smooth,samples=300,domain=(-9:9),<->] ({2*sin(deg(x))+2},{0.2*x});

          %\addplot[color=penColor,fill=penColor2,only marks,mark=*] coordinates{(-6,9)};
          %\addplot[color=penColor,fill=penColor2,only marks,mark=*] coordinates{(2,-7)};



           

  \end{axis}
\end{tikzpicture}
\end{image}






\end{example}


Graphically, we have to keep our eyes on several things at once.  We watch the original input into $In$, then we watch the output of $In$ and picture that as the new input into $Out$, then we watch the output coming from $Out$.

Algebraically, we just replace.



\section{Algebraically}







Let $Out(x) = 3x^3 + \sin(4x) - \frac{3}{2-x}$ with its natural or implied domain. \\
Let $In(t) = \frac{5t-7}{t^2-8}$ with its natural or implied domain. \\



Algebraically, composition is accomplished by replacing all occrrences of the variable with the entire formula for the other function.




\[
(Out \circ In)(k) = Out(In(k)) = 3 {\frac{5t-7}{t^2-8}}^3 + \sin(4 \frac{5t-7}{t^2-8}) - \frac{3}{2-\frac{5t-7}{t^2-8}}
\]





Obviously, parentheses are vital here.





\[
(Out \circ In)(k) = Out(In(k)) = 3 \left( \frac{5t-7}{t^2-8} \right)^3 + \sin\left( 4 \left( \frac{5t-7}{t^2-8} \right) \right) - \frac{3}{2 - \left( \frac{5t-7}{t^2-8} \right)}
\]






We can always make two compositions from two functions.




\[
(In \circ Out)(w) = In(Out(w)) = \frac{5 \left( 3x^3 + \sin(4x) - \frac{3}{2-x} \right)-7}{\left( 3x^3 + \sin(4x) - \frac{3}{2-x} \right)^2-8}
\]





\begin{example} Composition 



Let $f(x) = 5x^2 - 4x + 2$ with its natural or implied domain. \\
Let $g(t) = \frac{3}{6-t}$ with its natural or implied domain. \\



\[
(f \circ g)(y) = f(g(y)) = 5 \left( \answer{\frac{3}{6-y}} \right)^2 - 4 \left( \answer{\frac{3}{6-y}} \right) + 2
\]





\[
(g \circ f)(w) = g(f(w)) = \frac{3}{6 - \left( \answer{5w^2 - 4w + 2} \right)}
\]



\end{example}










\begin{example} Composition 



Let $f(x) = 3x^2 - x + 5$ with its natural or implied domain. \\
Let $g(t) = \frac{4}{7-t}$ with its natural or implied domain. \\





There are two numbers where $f(x) = 7$.  One of them is $1$, the other is $\answer{\frac{-2}{3}}$.






\[
(g \circ f)(w) = g(f(w)) = \frac{4}{7 - (5w^2 - 4w + 2)}
\]



In the $g \circ f$ composition, $f$ is no longer allowed to equal $\answer{7}$.

Select all real numbers which cannot be in the domain of $g \circ f$.

\begin{selectAll}
\choice{$7$}
\choice{$0$}
\choice[correct]{$1$}
\choice{$\frac{2}{3}$}
\choice{$-1$}
\choice[correct]{$\frac{-2}{3}$}
\end{selectAll}


\end{example}





















\end{document}
