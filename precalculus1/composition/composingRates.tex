\documentclass{ximera}

%\usepackage{todonotes}

\newcommand{\todo}{}

\usepackage{esint} % for \oiint
\ifxake%%https://math.meta.stackexchange.com/questions/9973/how-do-you-render-a-closed-surface-double-integral
\renewcommand{\oiint}{{\large\bigcirc}\kern-1.56em\iint}
\fi


\graphicspath{
  {./}
  {ximeraTutorial/}
  {basicPhilosophy/}
  {functionsOfSeveralVariables/}
  {normalVectors/}
  {lagrangeMultipliers/}
  {vectorFields/}
  {greensTheorem/}
  {shapeOfThingsToCome/}
  {dotProducts/}
  {partialDerivativesAndTheGradientVector/}
  {../productAndQuotientRules/exercises/}
  {../normalVectors/exercisesParametricPlots/}
  {../continuityOfFunctionsOfSeveralVariables/exercises/}
  {../partialDerivativesAndTheGradientVector/exercises/}
  {../directionalDerivativeAndChainRule/exercises/}
  {../commonCoordinates/exercisesCylindricalCoordinates/}
  {../commonCoordinates/exercisesSphericalCoordinates/}
  {../greensTheorem/exercisesCurlAndLineIntegrals/}
  {../greensTheorem/exercisesDivergenceAndLineIntegrals/}
  {../shapeOfThingsToCome/exercisesDivergenceTheorem/}
  {../greensTheorem/}
  {../shapeOfThingsToCome/}
  {../separableDifferentialEquations/exercises/}
  {vectorFields/}
}

\newcommand{\mooculus}{\textsf{\textbf{MOOC}\textnormal{\textsf{ULUS}}}}

\usepackage{tkz-euclide}
\usepackage{tikz}
\usepackage{tikz-cd}
\usetikzlibrary{arrows}
\tikzset{>=stealth,commutative diagrams/.cd,
  arrow style=tikz,diagrams={>=stealth}} %% cool arrow head
\tikzset{shorten <>/.style={ shorten >=#1, shorten <=#1 } } %% allows shorter vectors

\usetikzlibrary{backgrounds} %% for boxes around graphs
\usetikzlibrary{shapes,positioning}  %% Clouds and stars
\usetikzlibrary{matrix} %% for matrix
\usepgfplotslibrary{polar} %% for polar plots
\usepgfplotslibrary{fillbetween} %% to shade area between curves in TikZ
%\usetkzobj{all}
\usepackage[makeroom]{cancel} %% for strike outs
%\usepackage{mathtools} %% for pretty underbrace % Breaks Ximera
%\usepackage{multicol}
\usepackage{pgffor} %% required for integral for loops



%% http://tex.stackexchange.com/questions/66490/drawing-a-tikz-arc-specifying-the-center
%% Draws beach ball
\tikzset{pics/carc/.style args={#1:#2:#3}{code={\draw[pic actions] (#1:#3) arc(#1:#2:#3);}}}



\usepackage{array}
\setlength{\extrarowheight}{+.1cm}
\newdimen\digitwidth
\settowidth\digitwidth{9}
\def\divrule#1#2{
\noalign{\moveright#1\digitwidth
\vbox{\hrule width#2\digitwidth}}}




% \newcommand{\RR}{\mathbb R}
% \newcommand{\R}{\mathbb R}
% \newcommand{\N}{\mathbb N}
% \newcommand{\Z}{\mathbb Z}

\newcommand{\sagemath}{\textsf{SageMath}}


%\renewcommand{\d}{\,d\!}
%\renewcommand{\d}{\mathop{}\!d}
%\newcommand{\dd}[2][]{\frac{\d #1}{\d #2}}
%\newcommand{\pp}[2][]{\frac{\partial #1}{\partial #2}}
% \renewcommand{\l}{\ell}
%\newcommand{\ddx}{\frac{d}{\d x}}

% \newcommand{\zeroOverZero}{\ensuremath{\boldsymbol{\tfrac{0}{0}}}}
%\newcommand{\inftyOverInfty}{\ensuremath{\boldsymbol{\tfrac{\infty}{\infty}}}}
%\newcommand{\zeroOverInfty}{\ensuremath{\boldsymbol{\tfrac{0}{\infty}}}}
%\newcommand{\zeroTimesInfty}{\ensuremath{\small\boldsymbol{0\cdot \infty}}}
%\newcommand{\inftyMinusInfty}{\ensuremath{\small\boldsymbol{\infty - \infty}}}
%\newcommand{\oneToInfty}{\ensuremath{\boldsymbol{1^\infty}}}
%\newcommand{\zeroToZero}{\ensuremath{\boldsymbol{0^0}}}
%\newcommand{\inftyToZero}{\ensuremath{\boldsymbol{\infty^0}}}



% \newcommand{\numOverZero}{\ensuremath{\boldsymbol{\tfrac{\#}{0}}}}
% \newcommand{\dfn}{\textbf}
% \newcommand{\unit}{\,\mathrm}
% \newcommand{\unit}{\mathop{}\!\mathrm}
% \newcommand{\eval}[1]{\bigg[ #1 \bigg]}
% \newcommand{\seq}[1]{\left( #1 \right)}
% \renewcommand{\epsilon}{\varepsilon}
% \renewcommand{\phi}{\varphi}


% \renewcommand{\iff}{\Leftrightarrow}

% \DeclareMathOperator{\arccot}{arccot}
% \DeclareMathOperator{\arcsec}{arcsec}
% \DeclareMathOperator{\arccsc}{arccsc}
% \DeclareMathOperator{\si}{Si}
% \DeclareMathOperator{\scal}{scal}
% \DeclareMathOperator{\sign}{sign}


%% \newcommand{\tightoverset}[2]{% for arrow vec
%%   \mathop{#2}\limits^{\vbox to -.5ex{\kern-0.75ex\hbox{$#1$}\vss}}}
% \newcommand{\arrowvec}[1]{{\overset{\rightharpoonup}{#1}}}
% \renewcommand{\vec}[1]{\arrowvec{\mathbf{#1}}}
% \renewcommand{\vec}[1]{{\overset{\boldsymbol{\rightharpoonup}}{\mathbf{#1}}}}

% \newcommand{\point}[1]{\left(#1\right)} %this allows \vector{ to be changed to \vector{ with a quick find and replace
% \newcommand{\pt}[1]{\mathbf{#1}} %this allows \vec{ to be changed to \vec{ with a quick find and replace
% \newcommand{\Lim}[2]{\lim_{\point{#1} \to \point{#2}}} %Bart, I changed this to point since I want to use it.  It runs through both of the exercise and exerciseE files in limits section, which is why it was in each document to start with.

% \DeclareMathOperator{\proj}{\mathbf{proj}}
% \newcommand{\veci}{{\boldsymbol{\hat{\imath}}}}
% \newcommand{\vecj}{{\boldsymbol{\hat{\jmath}}}}
% \newcommand{\veck}{{\boldsymbol{\hat{k}}}}
% \newcommand{\vecl}{\vec{\boldsymbol{\l}}}
% \newcommand{\uvec}[1]{\mathbf{\hat{#1}}}
% \newcommand{\utan}{\mathbf{\hat{t}}}
% \newcommand{\unormal}{\mathbf{\hat{n}}}
% \newcommand{\ubinormal}{\mathbf{\hat{b}}}

% \newcommand{\dotp}{\bullet}
% \newcommand{\cross}{\boldsymbol\times}
% \newcommand{\grad}{\boldsymbol\nabla}
% \newcommand{\divergence}{\grad\dotp}
% \newcommand{\curl}{\grad\cross}
%\DeclareMathOperator{\divergence}{divergence}
%\DeclareMathOperator{\curl}[1]{\grad\cross #1}
% \newcommand{\lto}{\mathop{\longrightarrow\,}\limits}

% \renewcommand{\bar}{\overline}

\colorlet{textColor}{black}
\colorlet{background}{white}
\colorlet{penColor}{blue!50!black} % Color of a curve in a plot
\colorlet{penColor2}{red!50!black}% Color of a curve in a plot
\colorlet{penColor3}{red!50!blue} % Color of a curve in a plot
\colorlet{penColor4}{green!50!black} % Color of a curve in a plot
\colorlet{penColor5}{orange!80!black} % Color of a curve in a plot
\colorlet{penColor6}{yellow!70!black} % Color of a curve in a plot
\colorlet{fill1}{penColor!20} % Color of fill in a plot
\colorlet{fill2}{penColor2!20} % Color of fill in a plot
\colorlet{fillp}{fill1} % Color of positive area
\colorlet{filln}{penColor2!20} % Color of negative area
\colorlet{fill3}{penColor3!20} % Fill
\colorlet{fill4}{penColor4!20} % Fill
\colorlet{fill5}{penColor5!20} % Fill
\colorlet{gridColor}{gray!50} % Color of grid in a plot

\newcommand{\surfaceColor}{violet}
\newcommand{\surfaceColorTwo}{redyellow}
\newcommand{\sliceColor}{greenyellow}




\pgfmathdeclarefunction{gauss}{2}{% gives gaussian
  \pgfmathparse{1/(#2*sqrt(2*pi))*exp(-((x-#1)^2)/(2*#2^2))}%
}


%%%%%%%%%%%%%
%% Vectors
%%%%%%%%%%%%%

%% Simple horiz vectors
\renewcommand{\vector}[1]{\left\langle #1\right\rangle}


%% %% Complex Horiz Vectors with angle brackets
%% \makeatletter
%% \renewcommand{\vector}[2][ , ]{\left\langle%
%%   \def\nextitem{\def\nextitem{#1}}%
%%   \@for \el:=#2\do{\nextitem\el}\right\rangle%
%% }
%% \makeatother

%% %% Vertical Vectors
%% \def\vector#1{\begin{bmatrix}\vecListA#1,,\end{bmatrix}}
%% \def\vecListA#1,{\if,#1,\else #1\cr \expandafter \vecListA \fi}

%%%%%%%%%%%%%
%% End of vectors
%%%%%%%%%%%%%

%\newcommand{\fullwidth}{}
%\newcommand{\normalwidth}{}



%% makes a snazzy t-chart for evaluating functions
%\newenvironment{tchart}{\rowcolors{2}{}{background!90!textColor}\array}{\endarray}

%%This is to help with formatting on future title pages.
\newenvironment{sectionOutcomes}{}{}



%% Flowchart stuff
%\tikzstyle{startstop} = [rectangle, rounded corners, minimum width=3cm, minimum height=1cm,text centered, draw=black]
%\tikzstyle{question} = [rectangle, minimum width=3cm, minimum height=1cm, text centered, draw=black]
%\tikzstyle{decision} = [trapezium, trapezium left angle=70, trapezium right angle=110, minimum width=3cm, minimum height=1cm, text centered, draw=black]
%\tikzstyle{question} = [rectangle, rounded corners, minimum width=3cm, minimum height=1cm,text centered, draw=black]
%\tikzstyle{process} = [rectangle, minimum width=3cm, minimum height=1cm, text centered, draw=black]
%\tikzstyle{decision} = [trapezium, trapezium left angle=70, trapezium right angle=110, minimum width=3cm, minimum height=1cm, text centered, draw=black]


\title{Rate for Composition}

\begin{document}

\begin{abstract}
rates of rates
\end{abstract}
\maketitle




Rates of Change is our measurement of \textbf{function behavior}.


$\vartriangleright$  \textbf{\textcolor{blue!55!black}{Functions increase.}}  However, they could increase slowly with a small positive rate of change.  They could increase quickly with a large positive rate of change. 



$\vartriangleright$  \textbf{\textcolor{blue!55!black}{Functions decrease.}}  However, they could decrease slowly with a small negative rate of change.  They could decrease quickly with a large negative rate of change. 





\textbf{\textcolor{purple!85!blue}{What about compositions?}}





What happens when you


\begin{itemize}
\item compose an increasing function with an increasing function?
\item compose an increasing function with an decreasing function?
\item compose an decreasing function with an increasing function?
\item compose an decreasing function with an decreasing function?
\end{itemize}



To investigate these we need to remember the definitions of increasing and decreasing.






\begin{summary} \textbf{\textcolor{green!50!black}{Increasing}} 


Let $f$ be a function defined on the domain $D$. \\
Let $S \subset D$ be any subset of $D$.

$f$ is \textbf{increasing} on $S$ provided $f$ possesses this property:  


\begin{center}
For every pair $a, b \in S$, when $a \leq b$ then $f(a) \leq f(b)$.
\end{center}

\end{summary}






\begin{summary} \textbf{\textcolor{green!50!black}{Decreasing}} 


Let $f$ be a function defined on the domain $D$. \\
Let $S \subset D$ be any subset of $D$.

$f$ is \textbf{decreasing} on $S$ provided $f$ possesses this property:  


\begin{center}
For every pair $a, b \in S$, when $a \leq b$ then $f(a) \geq f(b)$.
\end{center}

\end{summary}











\subsection*{Increasing $\circ$ Increasing}


Suppose that both $F$ and $G$ are increasing functions. \\

$\vartriangleright$ That means that when the numbers going into $F$ are increasing, then the numbers coming out of $F$ are increasing. \\

$\vartriangleright$ That means that when the numbers going into $G$ are increasing, then the numbers coming out of $G$ are increasing. \\



Now, consider the composition, $F \circ G$. \\


Suppose the numbers going in $F \circ G$ are increasing.  What are the output numbers doing? \\




\[ (F \circ G)(x) = F(G(x)) \]


Pretend that the values of $x$ are increasing. Then the values of $G(x)$ are increasing, since $G$ is an increasing function.

These values of $G(x)$, which are increasing, are going into $F$.  Therefore, the output of $F$ is increasing.  But, these are the values of the composition.



\begin{center}
\textbf{\textcolor{red!70!black}{When $x$ increases, then $(F \circ G)(x) = F(G(x))$ increases.}}
\end{center}






\begin{fact}
$increasing \circ increasing = increasing$


Let $F$ be an increasing function. \\
Let $G$ be an increasing function. \\


Consider, $F \circ G$.

Suppose $a$ and $b$ are in the domain of $F \circ G$, with $a < b$. \\

Then $G(a) < G(b)$, because $G$ is an increasing function. \\

Then $F(G(a)) < F(G(b))$, because $F$ is an increasing function.


\end{fact}





















\subsection*{Increasing $\circ$ Decreasing}


Suppose that $F$ is an increasing function. \\
Suppose that $G$ is a decreasing function. \\


$\vartriangleright$ That means that when the numbers going into $F$ are increasing, then the numbers coming out of $F$ are increasing. \\

$\vartriangleright$ It also means that when the numbers going into $F$ are decreasing, then the numbers coming out of $F$ are decreasing. \\



$\vartriangleright$ That means that when the numbers going into $G$ are increasing, then the numbers coming out of $G$ are decreasing. \\



Now, consider the composition, $F \circ G$. \\


Suppose the numbers going in $F \circ G$ are increasing.  What are the output numbers doing? \\




\[ (F \circ G)(x) = F(G(x)) \]


Pretend that the values of $x$ are increasing. Then the values of $G(x)$ are decreasing, since $G$ is a decreasing function.

These values of $G(x)$, which are decreasing, are going into $F$.  Therefore, the output of $F$ is decreasing, since $F$ is an increasing function.  But, these are the values of the composition.


\begin{center}
\textbf{\textcolor{red!70!black}{When $x$ increases, then $(F \circ G)(x) = F(G(x))$ decreases.}}
\end{center}



\begin{fact}
$increasing \circ decreasing = decreasing$


Let $F$ be an increasing function. \\
Let $G$ be a decreasing function. \\


Consider, $F \circ G$.

Suppose $a$ and $b$ are in the domain of $F \circ G$, with $a < b$. \\

Then $G(b) < G(a)$, because $G$ is a decreasing function. \\

Then $F(G(b)) < F(G(a))$, because $F$ is an increasing function.


\end{fact}





































\subsection*{Decreasing $\circ$ Increasing}


Suppose that $F$ is a decreasing function. \\
Suppose that $G$ is an increasing function. \\


$\vartriangleright$ That means that when the numbers going into $F$ are increasing, then the numbers coming out of $F$ are decreasing. \\

$\vartriangleright$ It also means that when the numbers going into $F$ are decreasing, then the numbers coming out of $F$ are increasing. \\



$\vartriangleright$ That means that when the numbers going into $G$ are increasing, then the numbers coming out of $G$ are increasing. \\


$\vartriangleright$ It also means that when the numbers going into $G$ are decreasing, then the numbers coming out of $G$ are decreasing. \\



Now, consider the composition, $F \circ G$. \\


Suppose the numbers going in $F \circ G$ are increasing.  What are the output numbers doing? \\




\[ (F \circ G)(x) = F(G(x)) \]


Pretend that the values of $x$ are increasing. Then the values of $G(x)$ are increasing, since $G$ is a increasing function.

These values of $G(x)$, which are increasing, are going into $F$.  Therefore, the output of $F$ is decreasing, since $F$ is an decreasing function.  But, these are the values of the composition.


\begin{center}
\textbf{\textcolor{red!70!black}{When $x$ increases, then $(F \circ G)(x) = F(G(x))$ decreases.}}
\end{center}



\begin{fact}
$decreasing \circ increasing = decreasing$


Let $F$ be a decreasing function. \\
Let $G$ be an increasing function. \\


Consider, $F \circ G$.

Suppose $a$ and $b$ are in the domain of $F \circ G$, with $a < b$. \\

Then $G(a) < G(b)$, because $G$ is an increasing function. \\

Then $F(G(b)) < F(G(a))$, because $F$ is an decreasing function.


\end{fact}






















\subsection*{Decreasing $\circ$ Decreasing}


Suppose that both $F$ and $G$ are decreasing functions. \\

$\vartriangleright$ That means that when the numbers going into $F$ are increasing, then the numbers coming out of $F$ are decreasing. \\


$\vartriangleright$ It also means that when the numbers going into $F$ are decreasing, then the numbers coming out of $F$ are increasing. \\


$\vartriangleright$ That means that when the numbers going into $G$ are increasing, then the numbers coming out of $G$ are decreasing. \\



$\vartriangleright$ It also means that when the numbers going into $G$ are decreasing, then the numbers coming out of $G$ are increasing. \\


Now, consider the composition, $F \circ G$. \\


Suppose the numbers going in $F \circ G$ are increasing.  What are the output numbers doing? \\




\[ (F \circ G)(x) = F(G(x)) \]


Pretend that the values of $x$ are increasing. Then the values of $G(x)$ are decreasing, since $G$ is an decreasing function.

These values of $G(x)$, which are decreasing, are going into $F$.  Therefore, the output of $F$ is increasing, because $F$ is a decreasing function.  But, these are the values of the composition.



\begin{center}
\textbf{\textcolor{red!70!black}{When $x$ increases, then $(F \circ G)(x) = F(G(x))$ increases.}}
\end{center}






\begin{fact}
$decreasing \circ decreasing = increasing$


Let $F$ be a decreasing function. \\
Let $G$ be a decreasing function. \\


Consider, $F \circ G$.

Suppose $a$ and $b$ are in the domain of $F \circ G$, with $a < b$. \\

Then $G(a) > G(b)$, because $G$ is a decreasing function. \\

Then $F(G(a)) < F(G(b))$, because $F$ is a decreasing function.


\end{fact}







\begin{example}   $e^{-(x+3)^2}$




\end{example}













\begin{example}   $ln(x^2 + 3x + 5)$




\end{example}



















\begin{example}   Quartic



Let $F(x) = (x+7)(x-1)$ \\


Let $G(k) = (k+4)(k-3)$ \\



Where is the composition, $F \circ G$, increasing and decreasing?


$\vartriangleright$   $(F \circ G)(t) = t^4 +2 t^3 -17 t^2 - 18 t + 65$





We can approximate the critical numbers from the graph.



\begin{center}
\desmos{4uzzkkj7co}{400}{300}
\end{center}


The critical numbers are approximately $-3.541$, $-0.5$, and $2.541$.


\textbf{\textcolor{red!70!black}{Algebraically}}


Now, let's obtain the critical numbers algebraically. \\


We have a composition of two quadratic functions.  We need to know where they individually increase and decrease.

Both $F$ and $G$ have a positive leading coefficient.  Therefore, they both decrease and the increase. The derivative will reveal the specifics.



$\blacktriangleright$  $F'(x) = 2x + 6$

This tells us that the critical number is $-3$.  $F$ decreases on $(-\infty, -3)$ and increases on $(-3, \infty)$.

$\blacktriangleright$  $G'(k) = 2k + 1$

This tells us that the critical number is $-\frac{1}{2}$.  $F$ decreases on $(-\infty, -\frac{1}{2})$ and increases on $(-\frac{1}{2}, \infty)$.







\textbf{\textcolor{red!70!darkgray}{$\blacktriangleright$}} \textbf{\textcolor{purple!85!blue}{Overlaping Intervals}}  \\



We need to know when the output from $G$ crosses $-3$, which is where $F$ changes its behavior.





\begin{align*}
(k+4)(k-3) & = -3 \\
 k^2 + k - 12           & = -3  \\
 k^2 + k - 9         & = 0
\end{align*}


\[
k = \frac{-1 \pm \sqrt{1^2 - 4 (1) (-9)}}{2} = \frac{-1 \pm \sqrt{37}}{2} 
\]


We are feeling good, because $\frac{-1 - \sqrt{37}}{2} \approx -3.541$ and $\frac{-1 + \sqrt{37}}{2} \approx 2.541$, which are the approximations from the graph.






\textbf{\textcolor{blue!55!black}{$\blacktriangleright (-\infty, \frac{-1 - \sqrt{37}}{2})$}}


\begin{itemize}
\item On this interval $G$ is a decreasing function.
\item The range of $G$ on this interval is $(-3, \infty)$.
\item On $(-3, \infty)$, $F$ is an increasing function.
\item Therefore, the composition $F \circ G$ is a decreasing function.
\end{itemize}









\textbf{\textcolor{blue!55!black}{$\blacktriangleright (\frac{-1 - \sqrt{37}}{2}, \frac{1}{2})$}}


\begin{itemize}
\item On this interval $G$ is a decreasing function.
\item On this interval, $G < -3$.
\item On $(-\infty, -3)$, $F$ is a decreasing function.
\item Therefore, the composition $F \circ G$ is an increasing function.
\end{itemize}








\textbf{\textcolor{blue!55!black}{$\blacktriangleright (\frac{1}{2}, \frac{-1 + \sqrt{37}}{2})$}}


\begin{itemize}
\item On this interval $G$ is an increasing function.
\item On this interval, $G < -3$.
\item On $(-\infty, -3)$, $F$ is a decreasing function.
\item Therefore, the composition $F \circ G$ is a decreasing function.
\end{itemize}










\textbf{\textcolor{blue!55!black}{$\blacktriangleright (\frac{-1 + \sqrt{37}}{2}, \infty)$}}


\begin{itemize}
\item On this interval $G$ is an increasing function.
\item On this interval, $G > -3$.
\item On $(-3, \infty)$, $F$ is an increasing function.
\item Therefore, the composition $F \circ G$ is an increasing function.
\end{itemize}





We now know that 

\begin{itemize}
\item $F \circ G$ decreases on $(-\infty, \frac{-1 - \sqrt{37}}{2})$ \\
\item $F \circ G$ increases on $(\frac{-1 - \sqrt{37}}{2}, \frac{1}{2})$ \\
\item $F \circ G$ decreases on $(\frac{1}{2}, \frac{-1 + \sqrt{37}}{2})$ \\
\item $F \circ G$ increases on $(\frac{-1 + \sqrt{37}}{2}, \infty)$
\end{itemize}




At $\frac{-1 - \sqrt{37}}{2}$ there is a local minimum of $(F \circ G)\left( \frac{-1 - \sqrt{37}}{2} \right)$ \\


At $-\frac{1}{2}$ there is a local maximum of $(F \circ G)\left( -\frac{1}{2} \right)$ \\


At $\frac{-1 + \sqrt{37}}{2}$ there is a local minimum of $(F \circ G)\left( \frac{-1 + \sqrt{37}}{2} \right)$ \\




\end{example}



How do these compare with our graph information? \




\[
(F \circ G)\left( \frac{-1 - \sqrt{37}}{2} \right) = -16
\]

That is what the graph says! \\






\[
(F \circ G)\left( -\frac{1}{2} \right) = \frac{1113}{16} \approx 69.5625
\]

That is what the graph says! \\


\begin{center}
\textbf{\textcolor{red!80!black}{Wonderful !!!}}
\end{center}

















\begin{center}
\textbf{\textcolor{green!50!black}{ooooo-=-=-=-ooOoo-=-=-=-ooooo}} \\

more examples can be found by following this link\\ \link[More Examples of Composition]{https://ximera.osu.edu/csccmathematics/precalculus1/precalculus1/composition/examples/exampleList}

\end{center}




\end{document}
