\documentclass{ximera}


\graphicspath{
  {./}
  {ximeraTutorial/}
  {basicPhilosophy/}
}

\newcommand{\mooculus}{\textsf{\textbf{MOOC}\textnormal{\textsf{ULUS}}}}


\usepackage{tkz-euclide}\usepackage{tikz}
\usepackage{tikz-cd}
\usetikzlibrary{arrows}
\tikzset{>=stealth,commutative diagrams/.cd,
  arrow style=tikz,diagrams={>=stealth}} %% cool arrow head
\tikzset{shorten <>/.style={ shorten >=#1, shorten <=#1 } } %% allows shorter vectors

\usetikzlibrary{backgrounds} %% for boxes around graphs
\usetikzlibrary{shapes,positioning}  %% Clouds and stars
\usetikzlibrary{matrix} %% for matrix
\usepgfplotslibrary{polar} %% for polar plots
\usepgfplotslibrary{fillbetween} %% to shade area between curves in TikZ
\usetkzobj{all}
\usepackage[makeroom]{cancel} %% for strike outs
%\usepackage{mathtools} %% for pretty underbrace % Breaks Ximera
%\usepackage{multicol}
\usepackage{pgffor} %% required for integral for loops



%% http://tex.stackexchange.com/questions/66490/drawing-a-tikz-arc-specifying-the-center
%% Draws beach ball
\tikzset{pics/carc/.style args={#1:#2:#3}{code={\draw[pic actions] (#1:#3) arc(#1:#2:#3);}}}



\usepackage{array}
\setlength{\extrarowheight}{+.1cm}
\newdimen\digitwidth
\settowidth\digitwidth{9}
\def\divrule#1#2{
\noalign{\moveright#1\digitwidth
\vbox{\hrule width#2\digitwidth}}}
























%%This is to help with formatting on future title pages.
\newenvironment{sectionOutcomes}{}{}


\title{End Behavior}

\begin{document}

\begin{abstract}
approaching infinity
\end{abstract}
\maketitle







Eventually, outside some interval, the function doesn't do anything.  It settles down into a simple pattern.  We call this eventual pattern the \textbf{end-behavior}.



What are the simple patterns on $(-\infty, a) \cup (a, \infty)$?  Where $a$ is "big enough".  What does "big enough" mean?




\begin{example} big enough





Compare the graphs of $y = f(x) = \frac{1}{2}(3x-2)(x+5)(x+3)$ and $y=g(x) = \frac{3}{2}x^3$.



\begin{center}
\desmos{hldmst701o}{400}{300}
\end{center}







\begin{center}
\desmos{kfhf8jbk3s}{400}{300}
\end{center}







These two functions are either completely different or almost the same.  It depends on your viewpoint.

Inside $[-10, 10]$, they behave much differently. Their zeros are different.  $f(x)$ has local maximums and minimums.  Its graph has hills and valleys.  $g(x)$ is always increasing.

However, if you change your view point to $(-\infty, -50) \cup (50, \infty)$, then they are almost identical.



We would say $g(x) = \frac{3}{2}x^3$ is the end-behavior of $f(x) = \frac{1}{2}(3x-2)(x+5)(x+3)$.






\end{example}































\end{document}
