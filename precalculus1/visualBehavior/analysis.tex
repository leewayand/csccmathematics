\documentclass{ximera}

%\usepackage{todonotes}

\newcommand{\todo}{}

\usepackage{esint} % for \oiint
\ifxake%%https://math.meta.stackexchange.com/questions/9973/how-do-you-render-a-closed-surface-double-integral
\renewcommand{\oiint}{{\large\bigcirc}\kern-1.56em\iint}
\fi


\graphicspath{
  {./}
  {ximeraTutorial/}
  {basicPhilosophy/}
  {functionsOfSeveralVariables/}
  {normalVectors/}
  {lagrangeMultipliers/}
  {vectorFields/}
  {greensTheorem/}
  {shapeOfThingsToCome/}
  {dotProducts/}
  {partialDerivativesAndTheGradientVector/}
  {../productAndQuotientRules/exercises/}
  {../normalVectors/exercisesParametricPlots/}
  {../continuityOfFunctionsOfSeveralVariables/exercises/}
  {../partialDerivativesAndTheGradientVector/exercises/}
  {../directionalDerivativeAndChainRule/exercises/}
  {../commonCoordinates/exercisesCylindricalCoordinates/}
  {../commonCoordinates/exercisesSphericalCoordinates/}
  {../greensTheorem/exercisesCurlAndLineIntegrals/}
  {../greensTheorem/exercisesDivergenceAndLineIntegrals/}
  {../shapeOfThingsToCome/exercisesDivergenceTheorem/}
  {../greensTheorem/}
  {../shapeOfThingsToCome/}
  {../separableDifferentialEquations/exercises/}
  {vectorFields/}
}

\newcommand{\mooculus}{\textsf{\textbf{MOOC}\textnormal{\textsf{ULUS}}}}

\usepackage{tkz-euclide}
\usepackage{tikz}
\usepackage{tikz-cd}
\usetikzlibrary{arrows}
\tikzset{>=stealth,commutative diagrams/.cd,
  arrow style=tikz,diagrams={>=stealth}} %% cool arrow head
\tikzset{shorten <>/.style={ shorten >=#1, shorten <=#1 } } %% allows shorter vectors

\usetikzlibrary{backgrounds} %% for boxes around graphs
\usetikzlibrary{shapes,positioning}  %% Clouds and stars
\usetikzlibrary{matrix} %% for matrix
\usepgfplotslibrary{polar} %% for polar plots
\usepgfplotslibrary{fillbetween} %% to shade area between curves in TikZ
%\usetkzobj{all}
\usepackage[makeroom]{cancel} %% for strike outs
%\usepackage{mathtools} %% for pretty underbrace % Breaks Ximera
%\usepackage{multicol}
\usepackage{pgffor} %% required for integral for loops



%% http://tex.stackexchange.com/questions/66490/drawing-a-tikz-arc-specifying-the-center
%% Draws beach ball
\tikzset{pics/carc/.style args={#1:#2:#3}{code={\draw[pic actions] (#1:#3) arc(#1:#2:#3);}}}



\usepackage{array}
\setlength{\extrarowheight}{+.1cm}
\newdimen\digitwidth
\settowidth\digitwidth{9}
\def\divrule#1#2{
\noalign{\moveright#1\digitwidth
\vbox{\hrule width#2\digitwidth}}}




% \newcommand{\RR}{\mathbb R}
% \newcommand{\R}{\mathbb R}
% \newcommand{\N}{\mathbb N}
% \newcommand{\Z}{\mathbb Z}

\newcommand{\sagemath}{\textsf{SageMath}}


%\renewcommand{\d}{\,d\!}
%\renewcommand{\d}{\mathop{}\!d}
%\newcommand{\dd}[2][]{\frac{\d #1}{\d #2}}
%\newcommand{\pp}[2][]{\frac{\partial #1}{\partial #2}}
% \renewcommand{\l}{\ell}
%\newcommand{\ddx}{\frac{d}{\d x}}

% \newcommand{\zeroOverZero}{\ensuremath{\boldsymbol{\tfrac{0}{0}}}}
%\newcommand{\inftyOverInfty}{\ensuremath{\boldsymbol{\tfrac{\infty}{\infty}}}}
%\newcommand{\zeroOverInfty}{\ensuremath{\boldsymbol{\tfrac{0}{\infty}}}}
%\newcommand{\zeroTimesInfty}{\ensuremath{\small\boldsymbol{0\cdot \infty}}}
%\newcommand{\inftyMinusInfty}{\ensuremath{\small\boldsymbol{\infty - \infty}}}
%\newcommand{\oneToInfty}{\ensuremath{\boldsymbol{1^\infty}}}
%\newcommand{\zeroToZero}{\ensuremath{\boldsymbol{0^0}}}
%\newcommand{\inftyToZero}{\ensuremath{\boldsymbol{\infty^0}}}



% \newcommand{\numOverZero}{\ensuremath{\boldsymbol{\tfrac{\#}{0}}}}
% \newcommand{\dfn}{\textbf}
% \newcommand{\unit}{\,\mathrm}
% \newcommand{\unit}{\mathop{}\!\mathrm}
% \newcommand{\eval}[1]{\bigg[ #1 \bigg]}
% \newcommand{\seq}[1]{\left( #1 \right)}
% \renewcommand{\epsilon}{\varepsilon}
% \renewcommand{\phi}{\varphi}


% \renewcommand{\iff}{\Leftrightarrow}

% \DeclareMathOperator{\arccot}{arccot}
% \DeclareMathOperator{\arcsec}{arcsec}
% \DeclareMathOperator{\arccsc}{arccsc}
% \DeclareMathOperator{\si}{Si}
% \DeclareMathOperator{\scal}{scal}
% \DeclareMathOperator{\sign}{sign}


%% \newcommand{\tightoverset}[2]{% for arrow vec
%%   \mathop{#2}\limits^{\vbox to -.5ex{\kern-0.75ex\hbox{$#1$}\vss}}}
% \newcommand{\arrowvec}[1]{{\overset{\rightharpoonup}{#1}}}
% \renewcommand{\vec}[1]{\arrowvec{\mathbf{#1}}}
% \renewcommand{\vec}[1]{{\overset{\boldsymbol{\rightharpoonup}}{\mathbf{#1}}}}

% \newcommand{\point}[1]{\left(#1\right)} %this allows \vector{ to be changed to \vector{ with a quick find and replace
% \newcommand{\pt}[1]{\mathbf{#1}} %this allows \vec{ to be changed to \vec{ with a quick find and replace
% \newcommand{\Lim}[2]{\lim_{\point{#1} \to \point{#2}}} %Bart, I changed this to point since I want to use it.  It runs through both of the exercise and exerciseE files in limits section, which is why it was in each document to start with.

% \DeclareMathOperator{\proj}{\mathbf{proj}}
% \newcommand{\veci}{{\boldsymbol{\hat{\imath}}}}
% \newcommand{\vecj}{{\boldsymbol{\hat{\jmath}}}}
% \newcommand{\veck}{{\boldsymbol{\hat{k}}}}
% \newcommand{\vecl}{\vec{\boldsymbol{\l}}}
% \newcommand{\uvec}[1]{\mathbf{\hat{#1}}}
% \newcommand{\utan}{\mathbf{\hat{t}}}
% \newcommand{\unormal}{\mathbf{\hat{n}}}
% \newcommand{\ubinormal}{\mathbf{\hat{b}}}

% \newcommand{\dotp}{\bullet}
% \newcommand{\cross}{\boldsymbol\times}
% \newcommand{\grad}{\boldsymbol\nabla}
% \newcommand{\divergence}{\grad\dotp}
% \newcommand{\curl}{\grad\cross}
%\DeclareMathOperator{\divergence}{divergence}
%\DeclareMathOperator{\curl}[1]{\grad\cross #1}
% \newcommand{\lto}{\mathop{\longrightarrow\,}\limits}

% \renewcommand{\bar}{\overline}

\colorlet{textColor}{black}
\colorlet{background}{white}
\colorlet{penColor}{blue!50!black} % Color of a curve in a plot
\colorlet{penColor2}{red!50!black}% Color of a curve in a plot
\colorlet{penColor3}{red!50!blue} % Color of a curve in a plot
\colorlet{penColor4}{green!50!black} % Color of a curve in a plot
\colorlet{penColor5}{orange!80!black} % Color of a curve in a plot
\colorlet{penColor6}{yellow!70!black} % Color of a curve in a plot
\colorlet{fill1}{penColor!20} % Color of fill in a plot
\colorlet{fill2}{penColor2!20} % Color of fill in a plot
\colorlet{fillp}{fill1} % Color of positive area
\colorlet{filln}{penColor2!20} % Color of negative area
\colorlet{fill3}{penColor3!20} % Fill
\colorlet{fill4}{penColor4!20} % Fill
\colorlet{fill5}{penColor5!20} % Fill
\colorlet{gridColor}{gray!50} % Color of grid in a plot

\newcommand{\surfaceColor}{violet}
\newcommand{\surfaceColorTwo}{redyellow}
\newcommand{\sliceColor}{greenyellow}




\pgfmathdeclarefunction{gauss}{2}{% gives gaussian
  \pgfmathparse{1/(#2*sqrt(2*pi))*exp(-((x-#1)^2)/(2*#2^2))}%
}


%%%%%%%%%%%%%
%% Vectors
%%%%%%%%%%%%%

%% Simple horiz vectors
\renewcommand{\vector}[1]{\left\langle #1\right\rangle}


%% %% Complex Horiz Vectors with angle brackets
%% \makeatletter
%% \renewcommand{\vector}[2][ , ]{\left\langle%
%%   \def\nextitem{\def\nextitem{#1}}%
%%   \@for \el:=#2\do{\nextitem\el}\right\rangle%
%% }
%% \makeatother

%% %% Vertical Vectors
%% \def\vector#1{\begin{bmatrix}\vecListA#1,,\end{bmatrix}}
%% \def\vecListA#1,{\if,#1,\else #1\cr \expandafter \vecListA \fi}

%%%%%%%%%%%%%
%% End of vectors
%%%%%%%%%%%%%

%\newcommand{\fullwidth}{}
%\newcommand{\normalwidth}{}



%% makes a snazzy t-chart for evaluating functions
%\newenvironment{tchart}{\rowcolors{2}{}{background!90!textColor}\array}{\endarray}

%%This is to help with formatting on future title pages.
\newenvironment{sectionOutcomes}{}{}



%% Flowchart stuff
%\tikzstyle{startstop} = [rectangle, rounded corners, minimum width=3cm, minimum height=1cm,text centered, draw=black]
%\tikzstyle{question} = [rectangle, minimum width=3cm, minimum height=1cm, text centered, draw=black]
%\tikzstyle{decision} = [trapezium, trapezium left angle=70, trapezium right angle=110, minimum width=3cm, minimum height=1cm, text centered, draw=black]
%\tikzstyle{question} = [rectangle, rounded corners, minimum width=3cm, minimum height=1cm,text centered, draw=black]
%\tikzstyle{process} = [rectangle, minimum width=3cm, minimum height=1cm, text centered, draw=black]
%\tikzstyle{decision} = [trapezium, trapezium left angle=70, trapezium right angle=110, minimum width=3cm, minimum height=1cm, text centered, draw=black]


\title{Analysis}

\begin{document}

\begin{abstract}
rigor
\end{abstract}
\maketitle





The goal of Precalculus is learning how to analyze functions.  We are beginning with the Elementary Functions and then functions built from them. \\

What does it mean to \textbf{\textcolor{blue!55!black}{analyze}} a function? \\ 



Analyzing a function means listing its characteristics, features, and aspects along with an explanation of how you decided on these characteristics, features, and aspects.


We want \textbf{BOTH}: \\
\textbf{\textcolor{red!90!darkgray}{$\blacktriangleright$}} A description of the characteristic \\
\textbf{\textcolor{red!90!darkgray}{$\blacktriangleright$}} Your reasoning on how you decided \\



\begin{definition} \textbf{\textcolor{green!50!black}{Rigor}} \\


\textbf{\textcolor{red!90!darkgray}{$\blacktriangleright$}} \textbf{\textcolor{purple!85!blue}{Rigor}} lives inside your reasoning. \\

\textbf{\textcolor{red!90!darkgray}{$\blacktriangleright$}} \textbf{\textcolor{purple!85!blue}{Rigor}} lives inside your explanations. \\



Rigor means that you are using proper and precise mathematical language to explain how you know you are correct. This includes explaining how you know you have accounted for everything.


\end{definition}



\subsection*{The List}

The list of function characteristics, features, and aspects doesn't change for our analysis. \\





\begin{explanation}  \textbf{\textcolor{blue!75!black}{Domain}} \\

For us, just starting out, the domain is a list of all real numbers that have been paired with a function value (another real number). \\


The domain might be stated as part of the function definition. The domain might be the natural domain implied by a formula.  The domain might be the collection of first coordinates from points on a graph.  The domain might be restricted by situational constraints. Your reasoning explains how you decided. \\


Domains are usually described with interval notation.

\end{explanation}











\begin{explanation}  \textbf{\textcolor{blue!75!black}{Zeros}} \\

Function zeros are domain numbers where the function value is $0$.

\end{explanation}












\begin{explanation}  \textbf{\textcolor{blue!75!black}{Discontinuities and Singularities}} \\

Discontinuities are domain numbers around which the function is behaving ``wierd''. For us, just starting out, ``weird'' means that close domain numbers do not have close function values.   \\




Singularities are non-domain numbers around which the function is behaving ``wierd''.   \\


\end{explanation}



\begin{explanation}  \textbf{\textcolor{blue!75!black}{Continuity}} \\

In Algebra, functions are continuous over domain intervals.  For us, just starting out, we generally identify discontinuities and singularities, remove those, and are left with domain intervals where the function is contniuous.

\end{explanation}







\begin{explanation}  \textbf{\textcolor{blue!75!black}{End-Behavior}} \\

The end-behavior of a function is a simple description of how the function values behave as the domain values tend to $-\infty$ or $\infty$, the tails of the domain.

\end{explanation}









\begin{explanation}  \textbf{\textcolor{blue!75!black}{Behavior}} \\

The behavior of a function is a simple description of how the function values change.  This includes where in the domain the function \textbf{\textcolor{purple!85!blue}{increases}} and \textbf{\textcolor{purple!85!blue}{decreases}}. \\


We will evetually extend this to more detailed descriptions around discontinuities and singularities.


We will use \textbf{rates of change} to \textbf{measure} function behavior.

\end{explanation}









\begin{explanation}  \textbf{\textcolor{blue!75!black}{Extrema}} \\

The extreme values of a function include the \textbf{\textcolor{purple!85!blue}{global maximum}} and \textbf{\textcolor{purple!85!blue}{global minimum}} values. These are also called the \textbf{\textcolor{purple!85!blue}{absolute maximum}} and \textbf{\textcolor{purple!85!blue}{absolute minimum}} values.  \\

Extrema also includes \textbf{\textcolor{purple!85!blue}{local maximum}} and \textbf{\textcolor{purple!85!blue}{local minimum}} values. These are also called \textbf{\textcolor{purple!85!blue}{relative maximum}} and \textbf{\textcolor{purple!85!blue}{relative minimum}} values.  \\


Along with the maximum and minmum function values, we want to know where in the domain these occur.


\end{explanation}







\begin{explanation}  \textbf{\textcolor{blue!75!black}{Range}} \\

The range is the collection of function values. \\

The range is usually described with interval notation.

\end{explanation}















\begin{example} Analysis \\

Let $N(z)$ be a function.  The graph of $y = N(z)$ is displayed below. 

\begin{image}
\begin{tikzpicture}
     \begin{axis}[
            	domain=-10:10, ymax=10, xmax=10, ymin=-10, xmin=-10,
            	axis lines =center, xlabel=$z$, ylabel=$y$,
                xtick={-10,-8,-6,-4,-2,2,4,6,8,10},
                xticklabels={$-10$,$-8$,$-6$,$-4$,$-2$,$2$,$4$,$6$,$8$,$10$},
                ticklabel style={font=\scriptsize},
            	every axis y label/.style={at=(current axis.above origin),anchor=south},
            	every axis x label/.style={at=(current axis.right of origin),anchor=west},
            	axis on top,
          		]

        
        \addplot [draw=penColor, very thick, smooth, domain=(-8:-3), <-] {(x+7)*(x+2)};
        \addplot [draw=penColor, very thick, smooth, domain=(-3:2)] {-x};
        \addplot [draw=penColor, very thick, smooth, domain=(2:8)] {-0.25*x-4.5};


        \addplot[color=penColor,fill=white,only marks,mark=*] coordinates{(-3,-4)};
        \addplot[color=penColor,fill=white,only marks,mark=*] coordinates{(-3,3)};
        \addplot[color=penColor,fill=penColor,only marks,mark=*] coordinates{(-3,5)};
       

        \addplot[color=penColor,fill=penColor,only marks,mark=*] coordinates{(2,-5)};
        \addplot[color=penColor,fill=white,only marks,mark=*] coordinates{(2,-2)};
        \addplot[color=penColor,fill=penColor,only marks,mark=*] coordinates{(8,-8)};
        \addplot[color=penColor,fill=white,only marks,mark=*] coordinates{(8,-6.5)};

    \end{axis}
\end{tikzpicture}
\end{image}





\textbf{Domain:}  The domain of $N$ is $(-\infty, 8]$. \\

\textbf{Zeros:}   $-7$ and $0$ are the zeros of $N$, \\

\textbf{Continuity:}  $-3$, $2$, and $8$ are discontinuities of $N$. There are no singularities. \\
$N$ is continuous on the intervals

\[
(-\infty, -3), (-3, 2), \, \text{and} \, (2, 8)
\]


\textbf{End-Behavior:}  $N$ becomes unbounded positively as the domain becomes unbounded negatively.\\

\textbf{Behavior:}  

\begin{itemize}
	\item $N$ is decreasing on $(-\infty, -4.5)$.
	\item $N$ is increasing on $(-4.5, -3)$.
	\item $N$ is decreasing on $(-3, 2)$.
	\item $N$ is decreasing on $(2, 8)$.
\end{itemize}



\textbf{Global Maximum and Minimum:} The end-behavior tells us that $N$ has no global maximum.  The global minimum is $-8)$, which occurs at $8$.  \\


\textbf{Local Maximum and Minimum:} 

\begin{itemize}
	\item $N$ has a local minimum of $-8$, which occurs at $8$, because the global minimum is automatically a local minimum.
	\item $N$ has a local minimum of $-6$, which occurs at $-4.5$.
	\item $N$ has a local maximum of $5$, which occurs at $-3$.

\end{itemize}

\textbf{Range:} The range of $N$ is $\{ -8 \} \cup [-6, \infty)$. \\ 


\end{example}


\textbf{Note:} $2$ is not a local maximum or local minimum of $N$.  That is because FOR EVERY domain interval around $2$, there are ALWAYS domain number to the left and right where the funciotn value is greater than or less than $N(2)$.  We can tells this is true, because the graph has points above and below the point fo $2$ in EVERY POSSIBLE interval around $2$. \\













\begin{warning}  \textbf{\textcolor{blue!55!black}{Algebraic vs. Graphical Reasoning}}


We want precise analysis.  We want to be exact. \\


The only way to do this is by using algebra.  Graphs are inherently inaccurate and there is nothing we can do about this. \\


Our goal is rigorous algebraic reasong. \\



However, we are just starting out.  We don't have all of our algebraic tools yet.  That is what this course is supplying.  So, sometimes we have to provide graphical reasoning.  It will be approximate rather than precise.  We'll just have to understand this.  When we can be precise with algebra, then that is what we want. \\


We want precise analysis, which comes through logical reasoning and algebraic calculations, when we can get it. \\


We want to explain how we know we are exactly correct, but will settle for graphical feelings of correct, if that is all we can get. \\


\end{warning}



Learning when the algebra will not produce the reasoning we want is part of learning mathematics.  Algebra and function reasoning first. Then graphical reasoning.  But, always some reasoning.  Always an explanation. Not just a declaration of facts.








\begin{center}
\textbf{\textcolor{green!50!black}{ooooo-=-=-=-ooOoo-=-=-=-ooooo}} \\

more examples can be found by following this link\\ \link[More Examples of Visual Behavior]{https://ximera.osu.edu/csccmathematics/precalculus1/precalculus1/visualBehavior/examples/exampleList}

\end{center}





\end{document}
