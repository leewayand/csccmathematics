\documentclass{ximera}


\graphicspath{
  {./}
  {ximeraTutorial/}
  {basicPhilosophy/}
}

\newcommand{\mooculus}{\textsf{\textbf{MOOC}\textnormal{\textsf{ULUS}}}}


\usepackage{tkz-euclide}\usepackage{tikz}
\usepackage{tikz-cd}
\usetikzlibrary{arrows}
\tikzset{>=stealth,commutative diagrams/.cd,
  arrow style=tikz,diagrams={>=stealth}} %% cool arrow head
\tikzset{shorten <>/.style={ shorten >=#1, shorten <=#1 } } %% allows shorter vectors

\usetikzlibrary{backgrounds} %% for boxes around graphs
\usetikzlibrary{shapes,positioning}  %% Clouds and stars
\usetikzlibrary{matrix} %% for matrix
\usepgfplotslibrary{polar} %% for polar plots
\usepgfplotslibrary{fillbetween} %% to shade area between curves in TikZ
\usetkzobj{all}
\usepackage[makeroom]{cancel} %% for strike outs
%\usepackage{mathtools} %% for pretty underbrace % Breaks Ximera
%\usepackage{multicol}
\usepackage{pgffor} %% required for integral for loops



%% http://tex.stackexchange.com/questions/66490/drawing-a-tikz-arc-specifying-the-center
%% Draws beach ball
\tikzset{pics/carc/.style args={#1:#2:#3}{code={\draw[pic actions] (#1:#3) arc(#1:#2:#3);}}}



\usepackage{array}
\setlength{\extrarowheight}{+.1cm}
\newdimen\digitwidth
\settowidth\digitwidth{9}
\def\divrule#1#2{
\noalign{\moveright#1\digitwidth
\vbox{\hrule width#2\digitwidth}}}
























%%This is to help with formatting on future title pages.
\newenvironment{sectionOutcomes}{}{}


\title{Equivalent Forms}

\begin{document}

\begin{abstract}
%Stuff can go here later if we want!
\end{abstract}
\maketitle




Formulas are not functions.  Graphs are not functions. \\


A function is a collection of three sets: a domain, a range, and a set of pairs, along with one rule, which states that each domain item is included in exactly one pair.

Formulas and Graphs are tools representing those pairs.  \\


In fact, there are many formulas representing the same function.

For example, each formula on $(-\infty, \infty)$ below represents the same function.


\begin{itemize}
\item $x^2$
\item $(x - 1)^2 + 2 x - 1$
\item $\sqrt{x^4}$
\item $ x \cdot x$
\item $ 4 \cdot \left( \frac{x}{2} \right)^2$
\item $\begin{cases}
  \frac{x^3}{x} &\text{if $x \ne 0$,}\\
  0 &\text{if $x = 0$}.
\end{cases}$
\end{itemize}









\subsection{Expectations}


\begin{sectionOutcomes}
In this section, students will 

\begin{itemize}
\item compare equivalent forms.
\item compare nearly equivalent forms.
\end{itemize}
\end{sectionOutcomes}

\end{document}
