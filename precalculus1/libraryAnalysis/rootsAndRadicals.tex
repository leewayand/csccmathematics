\documentclass{ximera}


\graphicspath{
  {./}
  {ximeraTutorial/}
  {basicPhilosophy/}
}

\newcommand{\mooculus}{\textsf{\textbf{MOOC}\textnormal{\textsf{ULUS}}}}


\usepackage{tkz-euclide}\usepackage{tikz}
\usepackage{tikz-cd}
\usetikzlibrary{arrows}
\tikzset{>=stealth,commutative diagrams/.cd,
  arrow style=tikz,diagrams={>=stealth}} %% cool arrow head
\tikzset{shorten <>/.style={ shorten >=#1, shorten <=#1 } } %% allows shorter vectors

\usetikzlibrary{backgrounds} %% for boxes around graphs
\usetikzlibrary{shapes,positioning}  %% Clouds and stars
\usetikzlibrary{matrix} %% for matrix
\usepgfplotslibrary{polar} %% for polar plots
\usepgfplotslibrary{fillbetween} %% to shade area between curves in TikZ
\usetkzobj{all}
\usepackage[makeroom]{cancel} %% for strike outs
%\usepackage{mathtools} %% for pretty underbrace % Breaks Ximera
%\usepackage{multicol}
\usepackage{pgffor} %% required for integral for loops



%% http://tex.stackexchange.com/questions/66490/drawing-a-tikz-arc-specifying-the-center
%% Draws beach ball
\tikzset{pics/carc/.style args={#1:#2:#3}{code={\draw[pic actions] (#1:#3) arc(#1:#2:#3);}}}



\usepackage{array}
\setlength{\extrarowheight}{+.1cm}
\newdimen\digitwidth
\settowidth\digitwidth{9}
\def\divrule#1#2{
\noalign{\moveright#1\digitwidth
\vbox{\hrule width#2\digitwidth}}}
























%%This is to help with formatting on future title pages.
\newenvironment{sectionOutcomes}{}{}


\title{Roots and Radicals}

\begin{document}

\begin{abstract}
aspects
\end{abstract}
\maketitle










\section{Even Roots}

This course is the study of the real numbers. As a result we don't take even roots of negative numbers.

Even roots

\[   Even(x) = \sqrt[2n]{x} = x^{\tfrac{1}{2n}}          \]

look a lot like the square root


\[   Even(x) = \sqrt{x} = x^{\tfrac{1}{2}}          \]



Their domains do not include negative numbers: $[0, \infty)$, the \textbf{nonnegative} numbers.  And, they increase very slowly over this domain, becoming unbounded.







The graph of $y = Square(x) = \sqrt{x} $


\begin{image}
\begin{tikzpicture} 
  \begin{axis}[
            domain=-10:10, ymax=10, xmax=10, ymin=-10, xmin=-10,
            axis lines =center, xlabel=$w$, ylabel=$y$,
            every axis y label/.style={at=(current axis.above origin),anchor=south},
            every axis x label/.style={at=(current axis.right of origin),anchor=west},
            axis on top
          ]
          
          	%\addplot [line width=2, penColor, smooth, domain=(-9:0),<->] {(x-1)/((x+3)*(x-4))};
          	\addplot [line width=2, penColor, smooth, samples=200,domain=(0:9),->] {sqrt(x)};
   
 			\addplot[color=penColor,fill=penColor,only marks,mark=*] coordinates{(0,0)};

           

  \end{axis}
\end{tikzpicture}
\end{image}



The square root function begins where the inside of the formula equals $0$.  It then moves in the direction of positive inside.




\begin{example} Even Root

The graph of $y = f(v) = \sqrt{v+3} $


\begin{image}
\begin{tikzpicture} 
  \begin{axis}[
            domain=-10:10, ymax=10, xmax=10, ymin=-10, xmin=-10,
            axis lines =center, xlabel=$v$, ylabel=$y$,
            every axis y label/.style={at=(current axis.above origin),anchor=south},
            every axis x label/.style={at=(current axis.right of origin),anchor=west},
            axis on top
          ]
          
          	%\addplot [line width=2, penColor, smooth, domain=(-9:0),<->] {(x-1)/((x+3)*(x-4))};
          	\addplot [line width=2, penColor, smooth, samples=200,domain=(-3:9),->] {sqrt(x+3)};
   
 			\addplot[color=penColor,fill=penColor,only marks,mark=*] coordinates{(-3,0)};

           

  \end{axis}
\end{tikzpicture}
\end{image}


Here the inside $v+3=0$ equals $0$ when $v=-3$.  That is the start of the domain.  Then the inside is positive $v+3>0$, when $v>-3$, which means the graph moves up to the right.


\end{example}















\begin{example} Even Root

The graph of $y = T(k) = \sqrt{-k+5} $


\begin{image}
\begin{tikzpicture} 
  \begin{axis}[
            domain=-10:10, ymax=10, xmax=10, ymin=-10, xmin=-10,
            axis lines =center, xlabel=$k$, ylabel=$y$,
            every axis y label/.style={at=(current axis.above origin),anchor=south},
            every axis x label/.style={at=(current axis.right of origin),anchor=west},
            axis on top
          ]
          
          	%\addplot [line width=2, penColor, smooth, domain=(-9:0),<->] {(x-1)/((x+3)*(x-4))};
          	\addplot [line width=2, penColor, smooth, samples=200,domain=(-9:5),<-] {sqrt(-x+5)};
   
 			\addplot[color=penColor,fill=penColor,only marks,mark=*] coordinates{(5,0)};

           

  \end{axis}
\end{tikzpicture}
\end{image}


Here the inside $-k+5=0$ equals $0$ when $k=5$.  That is the start of the domain.  Then the inside is positive $-k+5>0$, when $k<5$, which means the graph moves up to the left.  The domain is $(-\infty,5]$.




\end{example}














\section{Odd Roots}

This course is the study of the real numbers. As a result we don't take even roots of negative numbers, we do have odd roots of negative numbers

Odd roots

\[   Odd(x) = \sqrt[2n+1]{x} = x^{\tfrac{1}{2n+1}}          \]

look a lot like the cube root


\[   Odd(x) = \sqrt[3]{x} = x^{\tfrac{1}{3}}          \]



Their domains dall real numbers: $(-\infty, \infty)$.  They increase very slowly over this domain, becoming unbounded.







The graph of $y = Cube(x) = \sqrt[3]{x} $


\begin{image}
\begin{tikzpicture} 
  \begin{axis}[
            domain=-10:10, ymax=10, xmax=10, ymin=-10, xmin=-10,
            axis lines =center, xlabel=$w$, ylabel=$y$,
            every axis y label/.style={at=(current axis.above origin),anchor=south},
            every axis x label/.style={at=(current axis.right of origin),anchor=west},
            axis on top
          ]
          
          	%\addplot [line width=2, penColor, smooth, domain=(-9:0),<->] {(x-1)/((x+3)*(x-4))};
          	\addplot [line width=2, penColor, smooth, samples=200,domain=(0:9),->] {x^0.33333};
          	\addplot [line width=2, penColor, smooth, samples=200,domain=(-9:0),<-] {-(-x)^0.33333};
   
 			\addplot[color=penColor,fill=penColor,only marks,mark=*] coordinates{(0,0)};

           

  \end{axis}
\end{tikzpicture}
\end{image}



The cube root function has a vertical tangent line where the inside of the formula equals $0$.



\begin{example} Odd Root

The graph of $y = f(v) = \sqrt[3]{v+3} $


\begin{image}
\begin{tikzpicture} 
  \begin{axis}[
            domain=-10:10, ymax=10, xmax=10, ymin=-10, xmin=-10,
            axis lines =center, xlabel=$v$, ylabel=$y$,
            every axis y label/.style={at=(current axis.above origin),anchor=south},
            every axis x label/.style={at=(current axis.right of origin),anchor=west},
            axis on top
          ]
          
          	\addplot [line width=2, penColor, smooth, samples=200,domain=(-3:9),->] {(x+3)^0.33333};
          	\addplot [line width=2, penColor, smooth, samples=200,domain=(-9:-3),<-] {-((-x-3)^0.33333)};

          	\addplot[color=penColor,fill=penColor,only marks,mark=*] coordinates{(-3,0)};

           

  \end{axis}
\end{tikzpicture}
\end{image}


Here the inside $v+3=0$ equals $0$ when $v=-3$.  The graph has a vertical tangent line there.  Otherwise, the cube root function is increasing everywhere, but increasing slower and slower and slower.


\end{example}








\end{document}
