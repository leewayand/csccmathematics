\documentclass{ximera}


\graphicspath{
  {./}
  {ximeraTutorial/}
  {basicPhilosophy/}
}

\newcommand{\mooculus}{\textsf{\textbf{MOOC}\textnormal{\textsf{ULUS}}}}


\usepackage{tkz-euclide}\usepackage{tikz}
\usepackage{tikz-cd}
\usetikzlibrary{arrows}
\tikzset{>=stealth,commutative diagrams/.cd,
  arrow style=tikz,diagrams={>=stealth}} %% cool arrow head
\tikzset{shorten <>/.style={ shorten >=#1, shorten <=#1 } } %% allows shorter vectors

\usetikzlibrary{backgrounds} %% for boxes around graphs
\usetikzlibrary{shapes,positioning}  %% Clouds and stars
\usetikzlibrary{matrix} %% for matrix
\usepgfplotslibrary{polar} %% for polar plots
\usepgfplotslibrary{fillbetween} %% to shade area between curves in TikZ
\usetkzobj{all}
\usepackage[makeroom]{cancel} %% for strike outs
%\usepackage{mathtools} %% for pretty underbrace % Breaks Ximera
%\usepackage{multicol}
\usepackage{pgffor} %% required for integral for loops



%% http://tex.stackexchange.com/questions/66490/drawing-a-tikz-arc-specifying-the-center
%% Draws beach ball
\tikzset{pics/carc/.style args={#1:#2:#3}{code={\draw[pic actions] (#1:#3) arc(#1:#2:#3);}}}



\usepackage{array}
\setlength{\extrarowheight}{+.1cm}
\newdimen\digitwidth
\settowidth\digitwidth{9}
\def\divrule#1#2{
\noalign{\moveright#1\digitwidth
\vbox{\hrule width#2\digitwidth}}}
























%%This is to help with formatting on future title pages.
\newenvironment{sectionOutcomes}{}{}


\title{Number Functions}

\begin{document}

\begin{abstract}
measurements
\end{abstract}
\maketitle





Most of the comparisons that interest us in mathematics involve measurements and we measure eveything...\\


Count, Length, Weight, Volume, Odor, Density, Brightness, Strength, Pressure, Heat, Temperature, Loudness, Change, Speed, Direction, Angle, Moisture, Voltage, Current, Tone, Notes, Satisfaction, Likelihood, Distance, Absorption, Reflection, Position, Heat, Magnetism, Sweetness, Sour, Focus, Flexibility, Pollution, Time, Smoothness, Humor, Stress, Area, Rates, Shape, Location, Orientation, Health... \\



...it just keeps going and going and going. \\


We compare all of these to each other:



\begin{itemize}
\item Odor vs. Direction
\item Pollution vs. Location
\item Flexibility vs. Density
\item Heat vs. Pressure
\item Voltage vs. Shape 
\end{itemize}







\section{Number Functions}

We make functions connecting just about everything.  In particular, we could make functions where both the domain and codomain are sets of measurements. These are the types of functions we study in Calculus.  Therefore, we are making yet another effort to focus our attention.  And, since, measurements are real numbers accompanied by a unit, we will concentrate on functions that connect sets of real numbers with sets of real numbers. \\

Technically, our functions will connect sets of measurements with sets of measurements. However, we usually hold the measurement units off to the side, work with the numbers, and then bring back the units when we interpret our results.  Once we arrive at any conclusions, then we will interpret our findings within the context of the situation under investigation and the measurements involved. \\




$\blacktriangleright$ \textbf{The Harpo Chalk Company} \\

The Harpo Chalk company sells chalk in bulk to schools and school districts. In an effort to increase sales, the company lowers the price per box of chalk as the order size increases.  The price per box is given in the table below. \\




\underline{Table 1. Chalk Prices per Box}
\[
\begin{array}{lll}
\text{from box number} & \text{to box number}  & \text{the price per box is} \\
0 &  100 &  \$0.25 \text{ per box}   \\
101 &  500 &  \$0.23 \text{ per box}   \\
501 &  750 &  \$0.20 \text{ per box}   \\
751 &  1000 &  \$0.17 \text{ per box}   \\
1001 &  2000 &  \$0.15 \text{ per box}   \\
2001 &  \text{unlimited} &  \$0.11 \text{ per box}   
\end{array}
\]




For example, if you purchased $150$ box of chalk, the first $100$ boxes would cost $\$0.25$ each for a total cost of $\$25.00$.  The final $50$ boxes (boxes numbered $101$ to $150$) would cost $\$0.23$ each for a total cost of $\$11.50$.  The total cost of the entire order would be $\$25.00 + \$11.50 = \$36.50$.


\begin{question}
How much would an order of $550$ boxes cost?

\begin{explanation}

\begin{itemize}
\item The first $100$ boxes cost $\$\answer{0.25}$ each for a total of $\$\answer{25.00}$.
\item The next $400$ boxes cost $\$\answer{0.23}$ each for a total of $\$\answer{92.00}$.
\item The next $50$ boxes cost $\$\answer{0.20}$ each for a total of $\$\answer{10.00}$.
\end{itemize}

The total cost for the $550$ boxes is $\$\answer{127}$.
\end{explanation}
\end{question}




\begin{question}
If the bill is $\$173.29$, then how many boxes of chalk were ordered?

\begin{explanation}

We can see from the previous question that more than $550$ boxes were ordered.  The next cut-off is at $750$ boxes.  How much do $750$ boxes cost? \\

\begin{itemize}
\item The first $100$ boxes cost $\$25$.
\item The next $400$ boxes cost $\$92$.
\item The next $250$ boxes cost $\$\answer{0.20}$ each for a total of $\$\answer{50.00}$.
\end{itemize}
The total cost for the $750$ boxes is $\$\answer{167}$. More than $750$ boxes were ordered.  We need to buy $\$173.29 - \$167 = \$6.29$.  These will cost $\$0.17$ each, which gives us $\answer{37}$ boxes.  \\


Total number of boxes ordered is $750 + 37 = 787$ boxes.


\end{explanation}



If $787$ boxes of chalk were purchsed for $\$173.29$, then, on average, each box cost $\$\answer[tolerance=0.01]{0.22}$. This is called the \textbf{effective price}.
\end{question}



The previous story compares boxes to dollars.  Those were the units for measurements. \\



We can also have functions that just relate numbers. \\







\begin{question} \textit{Double} \\
The \textit{Double} function pairs a real number with its double.

domain = all real numbers  \\ 
codomain = all real numbers


\begin{itemize}
\item \textit{Double}($3$) = $6$.
\item \textit{Double}($-4$) = $-8$.
\item \textit{Double}($\pi$) = $2 \pi$.

\item $\textit{Double}(7) = \answer{14}$.
\end{itemize}

\end{question} 



\begin{question} \textit{Half} \\
The \textit{Half} function pairs a real number with its half.

domain = positive real numbers  \\ 
codomain = positive real numbers


Solve \textit{Half}(d) = $8$ \\

The solution is $d = \answer{16}$.

\end{question} 




\begin{question} \textit{Successor} \\
The \textit{Successor} function pairs an integer with one more than the integer.

domain = all integers  \\ 
codomain = all integers


\begin{itemize}
\item \textit{Successor}($3$) = $4$.
\item \textit{Successor}($-4$) = $-3$.
\item \textit{Successor}($0$) = $1$.
\item \textit{Successor}($\pi$) = $undefined$.

\item $\textit{Successor}(-7) = \answer{-8}$.
\end{itemize}


Solve \textit{Successor}(z) = $-1$ \\

The solution is $d = \answer{-2}$.

\end{question} 







\begin{question} \textit{Successor} \\
Is every integer in the range of \textit{Successor}?


\begin{multipleChoice}
	\choice[correct]{Yes}
	\choice{No}
\end{multipleChoice}
\begin{feedback}
Every integer has a previous integer.  
\end{feedback}


Is \textit{Successor} an onto function?
\begin{multipleChoice}
	\choice[correct]{Yes}
	\choice{No}
\end{multipleChoice}


\end{question} 











If our function domains and ranges will be sets of real numbers, then it seems we should know about sets of real numbers.

Therefore, we need a way to communicate about sets of real numbers.
















\end{document}
