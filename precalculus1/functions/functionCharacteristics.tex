\documentclass{ximera}


\graphicspath{
  {./}
  {ximeraTutorial/}
  {basicPhilosophy/}
}

\newcommand{\mooculus}{\textsf{\textbf{MOOC}\textnormal{\textsf{ULUS}}}}


\usepackage{tkz-euclide}\usepackage{tikz}
\usepackage{tikz-cd}
\usetikzlibrary{arrows}
\tikzset{>=stealth,commutative diagrams/.cd,
  arrow style=tikz,diagrams={>=stealth}} %% cool arrow head
\tikzset{shorten <>/.style={ shorten >=#1, shorten <=#1 } } %% allows shorter vectors

\usetikzlibrary{backgrounds} %% for boxes around graphs
\usetikzlibrary{shapes,positioning}  %% Clouds and stars
\usetikzlibrary{matrix} %% for matrix
\usepgfplotslibrary{polar} %% for polar plots
\usepgfplotslibrary{fillbetween} %% to shade area between curves in TikZ
\usetkzobj{all}
\usepackage[makeroom]{cancel} %% for strike outs
%\usepackage{mathtools} %% for pretty underbrace % Breaks Ximera
%\usepackage{multicol}
\usepackage{pgffor} %% required for integral for loops



%% http://tex.stackexchange.com/questions/66490/drawing-a-tikz-arc-specifying-the-center
%% Draws beach ball
\tikzset{pics/carc/.style args={#1:#2:#3}{code={\draw[pic actions] (#1:#3) arc(#1:#2:#3);}}}



\usepackage{array}
\setlength{\extrarowheight}{+.1cm}
\newdimen\digitwidth
\settowidth\digitwidth{9}
\def\divrule#1#2{
\noalign{\moveright#1\digitwidth
\vbox{\hrule width#2\digitwidth}}}
























%%This is to help with formatting on future title pages.
\newenvironment{sectionOutcomes}{}{}


\title{Function Characteristics}

\begin{document}

\begin{abstract}
properties, features
\end{abstract}
\maketitle





Numbers have properties.  These properties define sets of numbers.

\begin{itemize}
\item Evens:  $\{ \cdots, -6, -4, -2, 0, 2, 4, 6, \cdots \} = \{ 2n   \, | \, n \in \mathbb{Z} \}$
\item Squares:  $\{ 1, 4, 9, 16, \cdots \} = \{ n^2   \, | \, n \in \mathbb{N} \}$
\item Primes:  $\{ 2, 3, 5, 7, 11,  \cdots \}$
\end{itemize}

Once we know that a number belongs to one of these sets, i.e. has the property, then we automatically know some information about the number. \\


$\blacktriangleright$ Same with functions. \\



We will develop a list of function properties.  Once we know a function has a property, then we get a lot of free information about the function. \\









\section{Function Characteristics or Properties}


Let's start off with two basic properties: \textbf{Onto} and \textbf{One-to-One}.







\begin{definition} \textbf{\textcolor{green!50!black}{Onto}} \\

If the range does equal the codomain, then the function is said to be \textbf{onto}.

\end{definition}









\begin{example} \textit{SuperBowlWinner}


\textit{SuperBowlWinner} is not an onto function.


\begin{itemize}
\item The Cleveland Browns are not in the range of \textit{SuperBowlWinner}.  
\item The Pittsburgh Steelers are in the range of \textit{SuperBowlWinner}.
\end{itemize}

\end{example}







Some functions are onto. Some are not. \\

Some functions are one-to-one. Some are not. \\




















\begin{question} \textit{Successor} \\
The \textit{Successor} function pairs an integer with one more than itself.

domain = all integers  \\ 
codomain = all integers


\begin{itemize}
\item \textit{Successor}($3$) = $4$.
\item \textit{Successor}($-4$) = $-3$.
\item \textit{Successor}($0$) = $1$.
\item \textit{Successor}($\pi$) = $undefined$.

\item $\textit{Successor}(-7) = \answer{-8}$.
\end{itemize}


Solve \textit{Successor}(z) = $-1$ \\

The solution is $d = \answer{-2}$.

\end{question} 







\begin{question} \textit{Successor} \\
Is every integer in the range of \textit{Successor}?


\begin{multipleChoice}
	\choice[correct]{Yes}
	\choice{No}
\end{multipleChoice}
\begin{feedback}
Every integer has a previous integer.  
\end{feedback}


Is \textit{Successor} an onto function?
\begin{multipleChoice}
	\choice[correct]{Yes}
	\choice{No}
\end{multipleChoice}


\end{question} 








The one and only rule for a function is that each domain item is in exactly one pair.  Range items can be in multiple pairs or no pairs. \\

However, if the range also follows this rule, then the function is called a \textbf{one-to-one} function.






\begin{definition} \textbf{\textcolor{green!50!black}{One-to-One}} \\

A \textbf{one-to-one} function is a function in which each range number is in exactly one pair.

\end{definition}



\begin{example}

The \textit{Successor} function is a one-to-one function. \\

Each integer has exactly one previous integer.

\end{example}














\end{document}
