\documentclass{ximera}


\graphicspath{
  {./}
  {ximeraTutorial/}
  {basicPhilosophy/}
}

\newcommand{\mooculus}{\textsf{\textbf{MOOC}\textnormal{\textsf{ULUS}}}}


\usepackage{tkz-euclide}\usepackage{tikz}
\usepackage{tikz-cd}
\usetikzlibrary{arrows}
\tikzset{>=stealth,commutative diagrams/.cd,
  arrow style=tikz,diagrams={>=stealth}} %% cool arrow head
\tikzset{shorten <>/.style={ shorten >=#1, shorten <=#1 } } %% allows shorter vectors

\usetikzlibrary{backgrounds} %% for boxes around graphs
\usetikzlibrary{shapes,positioning}  %% Clouds and stars
\usetikzlibrary{matrix} %% for matrix
\usepgfplotslibrary{polar} %% for polar plots
\usepgfplotslibrary{fillbetween} %% to shade area between curves in TikZ
\usetkzobj{all}
\usepackage[makeroom]{cancel} %% for strike outs
%\usepackage{mathtools} %% for pretty underbrace % Breaks Ximera
%\usepackage{multicol}
\usepackage{pgffor} %% required for integral for loops



%% http://tex.stackexchange.com/questions/66490/drawing-a-tikz-arc-specifying-the-center
%% Draws beach ball
\tikzset{pics/carc/.style args={#1:#2:#3}{code={\draw[pic actions] (#1:#3) arc(#1:#2:#3);}}}



\usepackage{array}
\setlength{\extrarowheight}{+.1cm}
\newdimen\digitwidth
\settowidth\digitwidth{9}
\def\divrule#1#2{
\noalign{\moveright#1\digitwidth
\vbox{\hrule width#2\digitwidth}}}
























%%This is to help with formatting on future title pages.
\newenvironment{sectionOutcomes}{}{}


\title{Change}

\begin{document}

\begin{abstract}
regular
\end{abstract}
\maketitle




There are many relationships between measurements that exhibit proportional changes.  







\subsection{Proportional Changes}




\begin{example} Speed


Suppose a car is travelling on the highway at a constant speed of $60 \, mph$.


$\blacktriangleright$ Whenever the distance measurement changes by $60 \, miles$, the time measurement changes by $1 \, hour$. \\
$\blacktriangleright$ Whenever the time measurement changes by $1 \, hour$, the distance measurement changes by $60 \, miles$. \\




\begin{itemize}
\item When the time changes by $2$ hours, the distance changes by $\answer{120}$ miles. \\
\item When the time changes by $5$ hours, the distance changes by $\answer{300}$ miles. \\
\item When the time changes by $0.5$ hours, the distance changes by $\answer{30}$ miles. \\
\end{itemize}



We have a constant conversion factor of $\frac{60 \, miles}{1 \, hour}$ for converting time changes into distance changes. \\






\begin{itemize}
\item When the distance changes by $2$ miles, the time changes by $\answer{\frac{1}{30}}$ hours. \\
\item When the distance changes by $5$ miles, the time changes by $\answer{\frac{1}{12}}$ hours. \\
\item When the distance changes by $0.5$ miles, the time changes by $\answer{\frac{1}{120}}$ hours. \\
\end{itemize}



We have a constant conversion factor of $\frac{1 \, hour}{60 \, mile}$ for converting distance changes into time changes. \\




$\blacktriangleright$ One viewpoint is that, in this situation, $60 \, miles = 1 hour$.  When one change occurs, the other must occur as well.





\end{example} 









\begin{notation}  \textbf{\textcolor{red!80!black}{$\Delta$}} \\


Mathematics has a shorthand symbol for ``change''.  It is a capital Greek Delta: $\Delta$

\end{notation}

In the previous example, it might be more appropriate to say \textbf{\textcolor{purple!85!blue}{$\Delta 60 \, miles = \Delta 1 hour$}} . It is the change in the measurements that is proportional.




\begin{example} Temperature


Temperature can be measured in degrees Fahrenheit or degrees Celsius and these measurements change proportionally. 


$\blacktriangleright$ Whenever the Fahrenheit measurement changes by $9^{\circ}$, the Celsius measurement changes by $5^{\circ}$. \\
$\blacktriangleright$ Whenever the Celsius measurement changes by $5^{\circ}$, the Fahrenheit measurement changes by $9^{\circ}$. \\



Water freezes at $0^{\circ}$ C and $32^{\circ}$ F.  From here, if the Celsius measurement changes by $100$, then the Fahrenheit measurement changes by $180^{\circ}$.  $18$ each of $5$'s and $9$'s. We get water boiling at $100^{\circ}$ celsius and $212^{\circ}$ degrees.


In this situation, $\frac{\Delta 5^{\circ}C}{\Delta 9^{\circ}F}$ or just $\frac{5^{\circ}C}{9^{\circ}F}$ is the conversion factor from Fahrenheit change to Celsius change, and $\frac{9^{\circ}F}{5^{\circ}C}$ is the conversion factor from Celsius to Fahrenheit.






\end{example} 
















\subsection{Rates}


Those conversions are examples of \textbf{rates}. 








\begin{definition} \textbf{\textcolor{green!50!black}{Rates}} \\

A \textbf{rate} is a ratio between two quantities with different units.   \\



We usually represent rates with factions, although we also use phrases involving ``per'' or equalities.

\end{definition}


For us, rates measure how fast one quantity changes compared to the change in another quantity.











\begin{example} Speed


Suppose a car is travelling on the highway at a constant speed of $60 \, mph$.


$\blacktriangleright$ Whenever the distance measurement changes by $60 \, miles$, the time measurement changes by $1 \, hour$. \\
$\blacktriangleright$ Whenever the time measurement changes by $1 \, hour$, the distance measurement changes by $60 \, miles$. \\




If we compare these related measurement changes in a rate, then we get

\[
\frac{\Delta Distance}{\Delta Time} = \frac{60 \, miles}{1 \, hour} = 60 \,miles \, per \, hour
\]


\end{example} 



This example gives us the equation for linear motion: $Distance = Rate \times Time$, which we could represent with a function.

\[
D(t) = R \cdot t =  \frac{60 \, miles}{1 \, hour} \cdot t
\]









\begin{example} Temperature


Temperature can be measured in degrees Fahrenheit or degrees Celsius and these measurements have a function relationship. This relationship has a special property.


$\blacktriangleright$ Whenever the Fahrenheit measurement changes by $9^{\circ}$, the Celsius measurement changes by $5^{\circ}$. \\
$\blacktriangleright$ Whenever the Celsius measurement changes by $5^{\circ}$, the Fahrenheit measurement changes by $9^{\circ}$. \\



Water freezes at $0^{\circ}$ C and $32^{\circ}$ F.  From here, if the Celsius measurement changes by $100$, then the Fahrenheit measurement changes by $180^{\circ}$.  $18$ each of $5$'s and $9$'s. We get water boiling at $100^{\circ}$ celsius and $212^{\circ}$ degrees.


If we compare these related measurement changes in a rate, then we get

\[
\frac{\Delta Degrees}{\Delta Celsius} = \frac{9^{\circ}F}{5^{\circ}C} \, \text{ or } \, \frac{\Delta Celsius}{\Delta Degrees} = \frac{5^{\circ}C}{9^{\circ}F}
\]


\end{example} 


This second example gives us the conversion between Fahrenheit change and Celsius change, which we could express with a function.  We just have to remember that $0^{\circ}C = 32^{\circ}F$.

\[
F(C) = 32 + \frac{9}{5} \cdot C
\]


The formula converts each change of $5$ in $C$ into a change of $9$ for $F$.







\begin{definition} \textbf{\textcolor{green!50!black}{Constant Rate of Change}} \\


A rate is a comparison of how measurements CHANGE. A rate is not a comparison of the measurement values, but a comparison of how they change. \\


Two measurements share a \textbf{constant rate of change} of $\tfrac{A}{B}$ if whenever measurement 1 changes by $A$, then measurement 2 changes by $B$ and vice versa, and this rate is not dependent on the amount of each measurement present.  The rate is the same throughout the situation.  It is a constant.




\end{definition}











\begin{definition} \textbf{\textcolor{green!50!black}{Linear Functions}} \\

Linear functions are those functions where the domain and range share a constant rate of change.  

\end{definition}


Each linear function has its own constant rate of change. \\


Suppose $L$ is a linear function.  Let $a$ and $b$ be numbers in the domain of $L$.  Then $L(a)$ and $L(b)$ are the corresponding range values.

Since $L$ is a linear function, we know that $\frac{L(b) - L(a)}{b - a} = constant$.  And, this works for ANY two domain numbers.

Otherwise, it is not a linear function.







\begin{example} \textit{Constant Rate of Change}


Suppose $f$ is a function, which contains the pairs $(3, 7)$, $(5, 17)$, and $(6, 23)$.


The rate-of-change from $3$ to $5$ is $\answer{5}$.

The rate-of-change from $5$ to $6$ is $\answer{6}$.

\begin{question} 
$f$ is a linear function.
\begin{multipleChoice}
\choice {True}
\choice [correct]{False}
\end{multipleChoice}
\end{question}

\end{example}


Somewhere in history, $m$ became a popular choice for the constant rate of change of a linear function.



\[
\frac{L(b) - L(a)}{b - a} = m
\]





No matter which two numbers you select from the domain of $L$, the rate of change always turns out to be $m$.  Each linear function has its own $m$ - its own constant rate of change.






\begin{model} \textit{Mileage vs. Weight}

\textbf{How does the weight of a car affect its gas mileage?} \\



$392$ makes and models of automobiles manufactured between 1970 and 1982 were considered.  

\begin{itemize}
\item How many gallons of gasoline were used to travel $100$ miles?   (gallons)
\item What is the weight of the car?    ($1000$ pounds)
\end{itemize}


These pairings are plotted below.




\begin{center}
\desmos{4ysvvryemv}{400}{300}
\end{center}


We can see from the graph that this is not the graph of a function.  \\


Our goal is to make predictions about other cars from this data.  Therefore, we would like a \textbf{model} for this data.  That is, we would like a function that approximates the data relationship.  The easiest model is a linear model.


A linear model would be a linear function whose graph does the best job a line can do of pretending to be the dots.  The line should pass through the ``middle of the dots''.

It the DESMOS window above, change the values of ``$m$'' and ``$b$'' to find such a line. \\

When you have your line, turn on the model found by DESMOS by clicking on the circle next to $y_1 ~ a_1 x + a_2$.


Source:   \textit{regressit.com}
\end{model}



\begin{question}


DESMOS found the linear model $y_1 = 0.450808 x_1 + 0.820606$ \\


The units of $x_1$ are $1000 lbs$. \\
The units of $y_1$ are $gallons$.



What are the units of $0.450808$?

\begin{multipleChoice}
\choice {$gallons$}
\choice {$pounds$}
\choice[correct] {$\frac{gallon}{pound}$}
\choice {$\frac{pound}{gallon}$}
\end{mulitpleCHoice}


\end{question}








\begin{question}


DESMOS found the linear model $y_1 = 0.450808 x_1 + 0.820606$ \\


The units of $x_1$ are $1000 lbs$. \\
The units of $y_1$ are $gallons$.



Which is the correct statement?

\begin{multipleChoice}
\choice[correct] {For every $1000 pounds$ added to a car, an extra $4.50808 gallons$ of gas is needed to travel $100$ miles.}
\choice {Every added gallon of gas can carry $4508.08 pounds$ an additional $100 miles$.}
\end{mulitpleChoice}


\end{question}




















\begin{center}
\textbf{\textcolor{green!50!black}{ooooo=-=-=-=-=-=-=-=-=-=-=-=-=ooOoo=-=-=-=-=-=-=-=-=-=-=-=-=ooooo}} \\

more examples can be found by following this link\\ \link[More Examples of Linear Functions]{https://ximera.osu.edu/csccmathematics/precalculus1/precalculus1/linearFunctions/examples/exampleList}

\end{center}





\end{document}
