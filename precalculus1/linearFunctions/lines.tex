\documentclass{ximera}


\graphicspath{
  {./}
  {ximeraTutorial/}
  {basicPhilosophy/}
}

\newcommand{\mooculus}{\textsf{\textbf{MOOC}\textnormal{\textsf{ULUS}}}}


\usepackage{tkz-euclide}\usepackage{tikz}
\usepackage{tikz-cd}
\usetikzlibrary{arrows}
\tikzset{>=stealth,commutative diagrams/.cd,
  arrow style=tikz,diagrams={>=stealth}} %% cool arrow head
\tikzset{shorten <>/.style={ shorten >=#1, shorten <=#1 } } %% allows shorter vectors

\usetikzlibrary{backgrounds} %% for boxes around graphs
\usetikzlibrary{shapes,positioning}  %% Clouds and stars
\usetikzlibrary{matrix} %% for matrix
\usepgfplotslibrary{polar} %% for polar plots
\usepgfplotslibrary{fillbetween} %% to shade area between curves in TikZ
\usetkzobj{all}
\usepackage[makeroom]{cancel} %% for strike outs
%\usepackage{mathtools} %% for pretty underbrace % Breaks Ximera
%\usepackage{multicol}
\usepackage{pgffor} %% required for integral for loops



%% http://tex.stackexchange.com/questions/66490/drawing-a-tikz-arc-specifying-the-center
%% Draws beach ball
\tikzset{pics/carc/.style args={#1:#2:#3}{code={\draw[pic actions] (#1:#3) arc(#1:#2:#3);}}}



\usepackage{array}
\setlength{\extrarowheight}{+.1cm}
\newdimen\digitwidth
\settowidth\digitwidth{9}
\def\divrule#1#2{
\noalign{\moveright#1\digitwidth
\vbox{\hrule width#2\digitwidth}}}
























%%This is to help with formatting on future title pages.
\newenvironment{sectionOutcomes}{}{}


\title{Lines}

\begin{document}

\begin{abstract}
constant slope
\end{abstract}
\maketitle



\begin{definition} \textbf{\textcolor{green!50!black}{Linear Functions}} \\

Linear functions are those functions with a constant growth rate or rate of change.  

\end{definition}








If you select any two distinct numbers from the domain of a linear function and calculate the rate of change over the associated interval, then you will get the same number - no matter which two domain numbers you select.

$\blacktriangleright$ Suppose $L$ is a linear function and $a \ne b$ are two numbers in the domain of $L$. Then, 

\[    \frac{L(b) - L(a)}{b - a} = m  \text{ for some fixed constant } m \]




No matter which two numbers you select from the domain of $L$, the rate of change always turns out to be $m$.  Each linear function has its own $m$ - its own constant rate of change.




\begin{example} \textit{Constant Rate of Change}


Suppose $f$ is a function, which contains the pairs $(3, 7)$, $(5, 17)$, and $(6, 23)$.


The rate-of-change from $3$ to $5$ is $\answer{5}$.

The rate-of-change from $5$ to $6$ is $\answer{6}$.

$f$ is a linear function.
\begin{multipleChoice}
\choice {True}
\choice [correct]{False}
\end{multipleChoice}


\end{example}



\section{A Formula}

Let $L$ be a linear function.  That means it has is own constant rate of change.  Let's call it $m$. \\

Let $(a, b)$ be one specific point in $L$.\\


Let $(x, y)$ represent any other pair in the function $L$. Then the rate of change from $(a,b)$ to $(x, y)$ has to equal $m$.


\[  \frac{y - \answer{b}}{x-\answer{a}} = m \]

Since $(a, b)$ is a pair in $L$, we know that $b = L(a)$.  And, since $(x, y)$ is a pair in $L$, we know that $y = L(x)$.  Replacing these in the equation for constant slope gives


\[  \frac{L(x) - L(a)}{x-a} = m \]

Solving this for $L(x)$ gives

\[  L(x) = m (x-a) + L(a)     \]



\textbf{Note:} In advanced mathematics, a similar idea called \textit{linear maps} are required to include the pair $(0,0)$.  This would result in $L(0) = 0$  We will not require this for Calculus.




\begin{example} \textit{A pair and a rate-of-change}


Suppose $W$ is a linear function with constant rate of change equal to $5$ and $(3, -1)$ is one pair in $W$.  \\

CReate a formula for $W$.

\begin{explanation}

The template $L(x) = m (x-a) + L(a)$ tell us that a formula for $W$ looks like 


\[  W(x) = 5 (x-3) + \left(\answer{-1}\right)     \]


\[  W(x) = 5 (x-3) - 1     \]

We could multiply this out and collect like terms and obtain the equivalent equation


\[  W(x) = \answer{5}x - \answer{16}   \]

\end{explanation}

Perhaps, we do not like $x$ as the variable for our formula.  Perhaps $v$ suits our situation better.

\[  W(v) = 5v - 16   \]


Or, $k$.

\[  W(k) = 5k - 16   \]


Or, $A$.

\[  W(A) = 5A - 16   \]


It is always advantageous to select a variable that is a nice reminder of the domain measurement.

\end{example}









\begin{example} \textit{Two Pairs}


Suppose $g$ is a linear function with $(0, 6)$ and $(-2, -5)$ as two pairs in $g$.

Then the template: $L(x) = m (x-a) + L(a)$ will need a rate of change



\[  m = \frac{-5 - 6}{\answer{-2} - \answer{0}} = \frac{-11}{-2} = \frac{11}{2}  \]

A formula for $g$ is


\[  g(t) = \frac{11}{2} (t-0) + 6     \]


\[  g(t) = \frac{11}{2} t + 6    \]



\end{example}










\section{A Graph}




A linear function has a line as its graph.  The line includes a point for each pair in the function.  And, since it is a line, only two points are needed to draw the graph.  Any two distinct points will do.





\begin{example} \textit{A Line}


Let $g(k) = 0.5k - 4$ be a linear function with its natural domain.


\begin{explanation}
Let's select two random domain numbers: $-4$ and $6$.  The function values at these domain numbers are $g(-4) = \answer{-6}$ and $g(6) = \answer{-1}$.  Therefore, the points $(-4, -6)$ and $(6, -1)$ are on the graph, which is a line.  We'll plot the two points and draw the line through them.


Below is the graph of $y=g(k)$.


\begin{image}
\begin{tikzpicture}
     \begin{axis}[
            	domain=-10:10, ymax=10, xmax=10, ymin=-10, xmin=-10,
            	axis lines =center, xlabel=$k$, ylabel=$y$, grid = major,
                ytick={-10,-8,-6,-4,-2,2,4,6,8,10},
                xtick={-10,-8,-6,-4,-2,2,4,6,8,10},
                ticklabel style={font=\scriptsize},
            	every axis y label/.style={at=(current axis.above origin),anchor=south},
            	every axis x label/.style={at=(current axis.right of origin),anchor=west},
            	axis on top,
          		]

        
        \addplot [draw=penColor, very thick, smooth, domain=(-8:8),<->] {0.5*x-4};

        \addplot[color=penColor,fill=penColor,only marks,mark=*] coordinates{(-4,-6)};
        \addplot[color=penColor,fill=penColor,only marks,mark=*] coordinates{(6,-1)};


    \end{axis}
\end{tikzpicture}
\end{image}


\end{explanation}

\end{example}










\begin{example} \textit{A Line}


Let $B(t)$ be a linear function.    Below is the graph of $y = B(t)$. From the graph obtain a formula for $B$.


\begin{image}
\begin{tikzpicture}
     \begin{axis}[
            	domain=-10:10, ymax=10, xmax=10, ymin=-10, xmin=-10,
            	axis lines =center, xlabel=$t$, ylabel=$y$, grid = major,
                ytick={-10,-8,-6,-4,-2,2,4,6,8,10},
                xtick={-10,-8,-6,-4,-2,2,4,6,8,10},
                ticklabel style={font=\scriptsize},
            	every axis y label/.style={at=(current axis.above origin),anchor=south},
            	every axis x label/.style={at=(current axis.right of origin),anchor=west},
            	axis on top,
          		]

        
        \addplot [draw=penColor, very thick, smooth, domain=(-3:8),<->] {(3/2)*x-5};

        %\addplot[color=penColor,fill=penColor,only marks,mark=*] coordinates{(-4,-6)};
        %\addplot[color=penColor,fill=penColor,only marks,mark=*] coordinates{(6,-1)};


    \end{axis}
\end{tikzpicture}
\end{image}

\begin{explanation}

From the graph, we can approximate the points $(0, -5)$ and $(3.3, 0)$.  These give a slope of

\[  slope = \frac{0 - \left(\answer{-5}\right)}{\answer{3.3} - 0} = \frac{5}{33} (\approx 1.5)     \]


This would give the formula $B(t) = \frac{5}{33} (t - 0) - 5 = \frac{5}{33} t - 5$

\end{explanation}

\end{example}


We used the point $(0, -5)$ to create the equation.  We could also have used $(3.3,0)$.



$B(t) = \frac{5}{33} (t - 3.3) - 0 = \frac{5}{33} t - \frac{5}{33}\cdot 3.3 = \frac{5}{33} t -  5$ \\


Same equation.  You can use any point on the line.















\section{Linear Equations}


\begin{definition} \textbf{\textcolor{green!50!black}{Linear Equation}} \\


A \textbf{linear equation in $x_1$ and $x_2$} is an equation that is equivalent to 

\[
a \, x_1 + b \, x_2 + c = 0
\]


$x_1$ and $x_2$ are the \textbf{variables}.

$a$, $b$, $c$ are constants called \textbf{coefficients}.



\end{definition}










\begin{definition} \textbf{\textcolor{green!50!black}{Solution}} \\


A \textbf{solution} to the linear equation  $a \, x_1 + b \, x_2 + c = 0$ is an ordered pair of numbers that \textbf{satisfy} the equation.

One of the solution numbers is designated for $x_1$ and one is designated for $x_2$.  (That is what order means.) Upon substituting these numbers into the equation for $x_1$ and $x_2$, the resulting statement is a true statement, i.e. the equation is satisfied.






\end{definition}



We often write solution pairs as an ordered pair: $(a, b)$.


To do so, we first have to choose which variable values will be written in the first or left slot and which wil be written in the second or right slot.


With this decision made, each solution pair can be interpreted as coordinates for a point on the Cartesian plane and plotted as a dot.




\begin{example}






Consider the linear equation  $3 t + 2 y = 12$.


Let's select $t$ to be the first variable and $y$ the second variable.

With this agreement, $(4, 0)$ is a solution to the equation.   $(4,0)$ can be viewed as a point and plotted as a dot on the Cartesian plane.



\begin{image}
\begin{tikzpicture}
     \begin{axis}[
                domain=-10:10, ymax=10, xmax=10, ymin=-10, xmin=-10,
                axis lines =center, xlabel=$t$, ylabel=$y$, grid = major,
                ytick={-10,-8,-6,-4,-2,2,4,6,8,10},
                xtick={-10,-8,-6,-4,-2,2,4,6,8,10},
                ticklabel style={font=\scriptsize},
                every axis y label/.style={at=(current axis.above origin),anchor=south},
                every axis x label/.style={at=(current axis.right of origin),anchor=west},
                axis on top,
                ]

        
        %\addplot [draw=penColor, very thick, smooth, domain=(-3:8),<->] {(3/2)*x-5};

         \addplot[color=penColor,fill=penColor,only marks,mark=*] coordinates{(4,0)};


    \end{axis}
\end{tikzpicture}
\end{image}







If we plot a dot for each and every solution pair for the equation, then we obtain the graph of the equation.  






\begin{image}
\begin{tikzpicture}
     \begin{axis}[
                domain=-10:10, ymax=10, xmax=10, ymin=-10, xmin=-10,
                axis lines =center, xlabel=$t$, ylabel=$y$, grid = major,
                ytick={-10,-8,-6,-4,-2,2,4,6,8,10},
                xtick={-10,-8,-6,-4,-2,2,4,6,8,10},
                ticklabel style={font=\scriptsize},
                every axis y label/.style={at=(current axis.above origin),anchor=south},
                every axis x label/.style={at=(current axis.right of origin),anchor=west},
                axis on top,
                ]

        
        \addplot [draw=penColor, very thick, smooth, domain=(-2:8),<->] {(-1.5)*(x-4)};

         \addplot[color=penColor,fill=penColor,only marks,mark=*] coordinates{(4,0)};


    \end{axis}
\end{tikzpicture}
\end{image}

The graph of a linear eqaution is a line, which can be drawn using only two points - two solution pairs.














\end{example}
Except for vertical lines, lines are graphs of functions. \\


If we interpret the first or left variable as the domain variable and the second or right variable as the function, then linear equations describe linear functions.  We can solve the equation for the function variable to obtain a formula.



In the example above, $3 t + 2 y = 12$ can be rewritten as $y = \answer{\frac{-3}{2}} t + \answer{6}$.

To emphasize that we are now thinking in terms of functions, we might write $y(t) = \tfrac{-3}{2} t + 6$.




We could just as easily have chosen $y$ as the domain variable and $t$ as the function variable.  In this case, $3 t + 2 y = 12$ can be rewritten as $t = \answer{\frac{-2}{3}} y + \answer{4}$.  We might write this as $t(y) = \tfrac{-2}{3} y + 4$




In either case, the function formula matches the "$y = m \, x + b$" template.  $\answer{m}$ is the slope of the line as well as the rate of change of the function.








\end{document}
