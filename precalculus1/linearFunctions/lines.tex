\documentclass{ximera}


\graphicspath{
  {./}
  {ximeraTutorial/}
  {basicPhilosophy/}
}

\newcommand{\mooculus}{\textsf{\textbf{MOOC}\textnormal{\textsf{ULUS}}}}


\usepackage{tkz-euclide}\usepackage{tikz}
\usepackage{tikz-cd}
\usetikzlibrary{arrows}
\tikzset{>=stealth,commutative diagrams/.cd,
  arrow style=tikz,diagrams={>=stealth}} %% cool arrow head
\tikzset{shorten <>/.style={ shorten >=#1, shorten <=#1 } } %% allows shorter vectors

\usetikzlibrary{backgrounds} %% for boxes around graphs
\usetikzlibrary{shapes,positioning}  %% Clouds and stars
\usetikzlibrary{matrix} %% for matrix
\usepgfplotslibrary{polar} %% for polar plots
\usepgfplotslibrary{fillbetween} %% to shade area between curves in TikZ
\usetkzobj{all}
\usepackage[makeroom]{cancel} %% for strike outs
%\usepackage{mathtools} %% for pretty underbrace % Breaks Ximera
%\usepackage{multicol}
\usepackage{pgffor} %% required for integral for loops



%% http://tex.stackexchange.com/questions/66490/drawing-a-tikz-arc-specifying-the-center
%% Draws beach ball
\tikzset{pics/carc/.style args={#1:#2:#3}{code={\draw[pic actions] (#1:#3) arc(#1:#2:#3);}}}



\usepackage{array}
\setlength{\extrarowheight}{+.1cm}
\newdimen\digitwidth
\settowidth\digitwidth{9}
\def\divrule#1#2{
\noalign{\moveright#1\digitwidth
\vbox{\hrule width#2\digitwidth}}}
























%%This is to help with formatting on future title pages.
\newenvironment{sectionOutcomes}{}{}


\title{Lines}

\begin{document}

\begin{abstract}
constant slope
\end{abstract}
\maketitle


Linear functions are those functions with a constant growth rate or rate-of-change.  If you select any two distinct numbers from the domain of a linear function and calculate the rate-of-change, you will get the same number, no matter which two domain numbers you select.

Suppose $L$ is a linear function and $a \ne b$ are two numbers in the domain of $L$. Then, 

\[    \frac{L(b) - L(a)}{b - a} = m   \]

for some constant $m$.


No matter which two numbers you select from the domain of $L$, the rate-of-change always turns out to be $m$.  Each linear function has its own constant rate-of-change.



\begin{example} \textit{nonLinear}

\end{example}

Suppose $f$ is a function, which contains the pairs $(3, 7)$, $(5, 17)$, and $(6, 23)$.


The rate-of-change from $(3, 7)$ to $(5, 17)$ is $\answer{5}$.

The rate-of-change from $(5, 17)$ to $(6, 23)$ is $\answer{6}$.

$f$ is a linear function.
\begin{multipleChoice}
\choice {Yes}
\choice [correct]{No}
\end{multipleChoice}






































\end{document}
