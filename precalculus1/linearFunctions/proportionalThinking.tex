\documentclass{ximera}


\graphicspath{
  {./}
  {ximeraTutorial/}
  {basicPhilosophy/}
}

\newcommand{\mooculus}{\textsf{\textbf{MOOC}\textnormal{\textsf{ULUS}}}}


\usepackage{tkz-euclide}\usepackage{tikz}
\usepackage{tikz-cd}
\usetikzlibrary{arrows}
\tikzset{>=stealth,commutative diagrams/.cd,
  arrow style=tikz,diagrams={>=stealth}} %% cool arrow head
\tikzset{shorten <>/.style={ shorten >=#1, shorten <=#1 } } %% allows shorter vectors

\usetikzlibrary{backgrounds} %% for boxes around graphs
\usetikzlibrary{shapes,positioning}  %% Clouds and stars
\usetikzlibrary{matrix} %% for matrix
\usepgfplotslibrary{polar} %% for polar plots
\usepgfplotslibrary{fillbetween} %% to shade area between curves in TikZ
\usetkzobj{all}
\usepackage[makeroom]{cancel} %% for strike outs
%\usepackage{mathtools} %% for pretty underbrace % Breaks Ximera
%\usepackage{multicol}
\usepackage{pgffor} %% required for integral for loops



%% http://tex.stackexchange.com/questions/66490/drawing-a-tikz-arc-specifying-the-center
%% Draws beach ball
\tikzset{pics/carc/.style args={#1:#2:#3}{code={\draw[pic actions] (#1:#3) arc(#1:#2:#3);}}}



\usepackage{array}
\setlength{\extrarowheight}{+.1cm}
\newdimen\digitwidth
\settowidth\digitwidth{9}
\def\divrule#1#2{
\noalign{\moveright#1\digitwidth
\vbox{\hrule width#2\digitwidth}}}
























%%This is to help with formatting on future title pages.
\newenvironment{sectionOutcomes}{}{}


\title{Proportional Reasoning}

\begin{document}

\begin{abstract}
similarity
\end{abstract}
\maketitle



When two quantities are \textbf{proportional}, then they are connected by a linear function of the form $f(x) = m x$.  The ratio of corresponding amounts is a constant: $\frac{f(x)}{x} = m$.   

The graph is a line going through the origin with slope $m$.  This means that the ratio of change is also a constant - the same constant $\frac{\Delta f(x)}{\Delta x} = m$. 

The value of one quantity is always the same multiple of the other quantity.




\begin{example}
In a simple circuit with a fixed resistor, the flow of current through the resistor, $I$, and the voltage drop across the resistor, $V$, are proportial.

\[   V = R I  \]

Here the rate-of-change is the resistance, $R$.  It is the multiplier.   When the current changes, that change is multiplied by $R$ giving the corresponding change in voltage $V$.


\end{example}







We can also change multiple proportions together.



\begin{example}

Water rushes through a turbine, which turns a generator to make electricity. The water posses potential energy, but not all of this energy is transmitted through the turbine.  The turbine has an 86\% efficiency. The energy is the processed through the generator at an efficiency 92\%.

Suppose the generator produces $864 \, gigaJoules$ of energy in the form  of electricity.  What amount of energy did the water originally hold?


We have two proportions:

\begin{itemize}
\item $T = 0.86 W$
\item $G = 0.92 T$
\end{itemize} 

Combining these gives 
\[   G = 0.92 T = 0.92 0.86 W   \]

\[   864 \, gigaJoules = 0.92 0.86 W    \]


\[   1092 \, gigaJoules = W    \]





\end{example}


Written in function language, the two proportions would have looked like

\begin{itemize}
\item $T(W) = 0.86 W$
\item $G(T) = 0.92 T$
\end{itemize} 

And, the chaining would have look like, $G(T(W)) = 0.92 (0.86 W))$.  In the function world, this is known as \textbf{composition}.












inverse


change








\end{document}
