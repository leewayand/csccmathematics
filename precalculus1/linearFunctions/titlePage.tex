\documentclass{ximera}


\graphicspath{
  {./}
  {ximeraTutorial/}
  {basicPhilosophy/}
}

\newcommand{\mooculus}{\textsf{\textbf{MOOC}\textnormal{\textsf{ULUS}}}}


\usepackage{tkz-euclide}\usepackage{tikz}
\usepackage{tikz-cd}
\usetikzlibrary{arrows}
\tikzset{>=stealth,commutative diagrams/.cd,
  arrow style=tikz,diagrams={>=stealth}} %% cool arrow head
\tikzset{shorten <>/.style={ shorten >=#1, shorten <=#1 } } %% allows shorter vectors

\usetikzlibrary{backgrounds} %% for boxes around graphs
\usetikzlibrary{shapes,positioning}  %% Clouds and stars
\usetikzlibrary{matrix} %% for matrix
\usepgfplotslibrary{polar} %% for polar plots
\usepgfplotslibrary{fillbetween} %% to shade area between curves in TikZ
\usetkzobj{all}
\usepackage[makeroom]{cancel} %% for strike outs
%\usepackage{mathtools} %% for pretty underbrace % Breaks Ximera
%\usepackage{multicol}
\usepackage{pgffor} %% required for integral for loops



%% http://tex.stackexchange.com/questions/66490/drawing-a-tikz-arc-specifying-the-center
%% Draws beach ball
\tikzset{pics/carc/.style args={#1:#2:#3}{code={\draw[pic actions] (#1:#3) arc(#1:#2:#3);}}}



\usepackage{array}
\setlength{\extrarowheight}{+.1cm}
\newdimen\digitwidth
\settowidth\digitwidth{9}
\def\divrule#1#2{
\noalign{\moveright#1\digitwidth
\vbox{\hrule width#2\digitwidth}}}
























%%This is to help with formatting on future title pages.
\newenvironment{sectionOutcomes}{}{}


\title{Linear Functions}

\begin{document}

\begin{abstract}
%Stuff can go here later if we want!
\end{abstract}
\maketitle









Our first investigation into the Elementary Functions is with linear functions. The distinguishing characteristic about linear functions is that they have a constant growth rate or rate of change.  


In other words, if you calculate the rate of change from any pair in the function to any other pair, you always get the same result.

Let $f$ be the linear function.  Let $(a, f(a))$ and $(b, f(b))$ any two pairs in the function.

The rate of change from $a$ to $b$ is 

\[
\frac{f(b) - f(a)}{b-a}
\]


For a linear function, this calculation gives the same result, no matter which two distinct points you choose.













From this we discover that linear functions are those functions that can be described as

\[  L(d) = A \cdot d + B \]

where $A$ and $B$ are real numbers and $A \ne 0$.


The graphs of linear functions are lines and we would like to be able to work back and forth between the algebra and geometry.




















\begin{sectionOutcomes}
In this section, students will 

\begin{itemize}
\item examine constant growth rates.
\item produce graphs of linear functions.
\item produce formulas from graphs.
\item create tangent lines.
\item solve systems of linear equations.
\end{itemize}
\end{sectionOutcomes}

\end{document}
