\documentclass{ximera}


\graphicspath{
  {./}
  {ximeraTutorial/}
  {basicPhilosophy/}
}

\newcommand{\mooculus}{\textsf{\textbf{MOOC}\textnormal{\textsf{ULUS}}}}


\usepackage{tkz-euclide}\usepackage{tikz}
\usepackage{tikz-cd}
\usetikzlibrary{arrows}
\tikzset{>=stealth,commutative diagrams/.cd,
  arrow style=tikz,diagrams={>=stealth}} %% cool arrow head
\tikzset{shorten <>/.style={ shorten >=#1, shorten <=#1 } } %% allows shorter vectors

\usetikzlibrary{backgrounds} %% for boxes around graphs
\usetikzlibrary{shapes,positioning}  %% Clouds and stars
\usetikzlibrary{matrix} %% for matrix
\usepgfplotslibrary{polar} %% for polar plots
\usepgfplotslibrary{fillbetween} %% to shade area between curves in TikZ
\usetkzobj{all}
\usepackage[makeroom]{cancel} %% for strike outs
%\usepackage{mathtools} %% for pretty underbrace % Breaks Ximera
%\usepackage{multicol}
\usepackage{pgffor} %% required for integral for loops



%% http://tex.stackexchange.com/questions/66490/drawing-a-tikz-arc-specifying-the-center
%% Draws beach ball
\tikzset{pics/carc/.style args={#1:#2:#3}{code={\draw[pic actions] (#1:#3) arc(#1:#2:#3);}}}



\usepackage{array}
\setlength{\extrarowheight}{+.1cm}
\newdimen\digitwidth
\settowidth\digitwidth{9}
\def\divrule#1#2{
\noalign{\moveright#1\digitwidth
\vbox{\hrule width#2\digitwidth}}}
























%%This is to help with formatting on future title pages.
\newenvironment{sectionOutcomes}{}{}


\outcome{outcome.}
\outcome{outcome.}
\outcome{outcome.}

\author{Lee Wayand}

\begin{document}





\begin{exercise} Intersection Point \\


Below are the graphs of 

\begin{itemize}
  \item $2A - 3B = 8$
  \item $3A + B = 1$
\end{itemize}

\begin{image}
\begin{tikzpicture}
  \begin{axis}[
            domain=-10:10, ymax=10, xmax=10, ymin=-10, xmin=-10,
            axis lines =center, xlabel=$t$, ylabel=$v$, 
            ytick={-10,-8,-6,-4,-2,2,4,6,8,10},
            xtick={-10,-8,-6,-4,-2,2,4,6,8,10},
            ticklabel style={font=\scriptsize},
            every axis y label/.style={at=(current axis.above origin),anchor=south},
            every axis x label/.style={at=(current axis.right of origin),anchor=west},
            axis on top
          ]
          

			\addplot [draw=penColor, very thick, smooth, domain=(-8:8),->] {1.5*x+4};
      \addplot [draw=penColor, very thick, smooth, domain=(-8:8),->] {-0.333*x+0.333};
			%\addplot[color=penColor,fill=penColor,only marks,mark=*] coordinates{(0,50)};






  \end{axis}
\end{tikzpicture}
\end{image}



To identify the intersection point, solve the second equation for $B$ to get $B = \answer{1-3A}$. \\

Substitute this expression in for $B$ in the first equation to get $2A - 3 \answer{1-3A} = 8$. \\





\begin{align*}
2 \, A - 3 (1-3 \, A) &= 8 \\
2 \, A - 3 + \answer{9} A &= 8 \\
\answer{11} A  - 3   &= 8 \\
\answer{11} A    &= \answer{11} \\
A &= \answer{1}
\end{align*}

It follows that $B = \answer{-2}$.

\end{exercise}











\begin{exercise} System \\


The following two equations are a system of linear equations.

\begin{itemize}
  \item $3x - 5y = 11$
  \item $-x + 3y = 4$
\end{itemize}


According to these two equations $3x - 5y - x + 3y = \answer{15}$


\end{exercise}














\begin{exercise} Proportionality \\


Let $2T + 7p = 9$. \\


According to this equation $6T + 21p = \answer{27}$.


\end{exercise}

















\end{document}