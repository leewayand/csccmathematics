\documentclass{ximera}


\graphicspath{
  {./}
  {ximeraTutorial/}
  {basicPhilosophy/}
}

\newcommand{\mooculus}{\textsf{\textbf{MOOC}\textnormal{\textsf{ULUS}}}}


\usepackage{tkz-euclide}\usepackage{tikz}
\usepackage{tikz-cd}
\usetikzlibrary{arrows}
\tikzset{>=stealth,commutative diagrams/.cd,
  arrow style=tikz,diagrams={>=stealth}} %% cool arrow head
\tikzset{shorten <>/.style={ shorten >=#1, shorten <=#1 } } %% allows shorter vectors

\usetikzlibrary{backgrounds} %% for boxes around graphs
\usetikzlibrary{shapes,positioning}  %% Clouds and stars
\usetikzlibrary{matrix} %% for matrix
\usepgfplotslibrary{polar} %% for polar plots
\usepgfplotslibrary{fillbetween} %% to shade area between curves in TikZ
\usetkzobj{all}
\usepackage[makeroom]{cancel} %% for strike outs
%\usepackage{mathtools} %% for pretty underbrace % Breaks Ximera
%\usepackage{multicol}
\usepackage{pgffor} %% required for integral for loops



%% http://tex.stackexchange.com/questions/66490/drawing-a-tikz-arc-specifying-the-center
%% Draws beach ball
\tikzset{pics/carc/.style args={#1:#2:#3}{code={\draw[pic actions] (#1:#3) arc(#1:#2:#3);}}}



\usepackage{array}
\setlength{\extrarowheight}{+.1cm}
\newdimen\digitwidth
\settowidth\digitwidth{9}
\def\divrule#1#2{
\noalign{\moveright#1\digitwidth
\vbox{\hrule width#2\digitwidth}}}
























%%This is to help with formatting on future title pages.
\newenvironment{sectionOutcomes}{}{}



\author{Lee Wayand}

\begin{document}
\begin{exercise}





\[
g(t) = 
\begin{cases}
  -(t+4)(t-2)      & \text{ if } [-5, 3)  \\
   6               & \text{ if } t = 3  \\
  -8               & \text{ if } t = 4  \\
  2t-10              & \text{ if } (4,8)
\end{cases}
\]






Graph of $y = g(t)$.



\begin{image}
\begin{tikzpicture} 
  \begin{axis}[
            domain=-10:10, ymax=10, xmax=10, ymin=-10, xmin=-10,
            axis lines =center, xlabel=$t$, ylabel=$y$, grid = major,
            ytick={-10,-8,-6,-4,-2,2,4,6,8,10},
            xtick={-10,-8,-6,-4,-2,2,4,6,8,10},
            ticklabel style={font=\scriptsize},
            every axis y label/.style={at=(current axis.above origin),anchor=south},
            every axis x label/.style={at=(current axis.right of origin),anchor=west},
            axis on top
          ]
          
          \addplot [line width=2, penColor, smooth,samples=100,domain=(-5:3)] {-(x+4)*(x-2)};
       	  \addplot [line width=2, penColor, smooth,samples=100,domain=(4:8)] {2*x-10};
       		%\addplot [line width=2, penColor, smooth,samples=100,domain=(4:8)] {3*x-16};




			\addplot[color=penColor,fill=penColor,only marks,mark=*] coordinates{(-5,-7)};

			
			\addplot[color=penColor,fill=white,only marks,mark=*] coordinates{(3,-7)};
      \addplot[color=penColor,fill=penColor,only marks,mark=*] coordinates{(3,6)};

			\addplot[color=penColor,fill=penColor,only marks,mark=*] coordinates{(4,-8)};
			\addplot[color=penColor,fill=white,only marks,mark=*] coordinates{(4,-2)};

			\addplot[color=penColor,fill=white,only marks,mark=*] coordinates{(8,6)};

  \end{axis}
\end{tikzpicture}
\end{image}






The graph suggests that $g(3) = 6$ is a local maximum.  Show this is correct. \\





\begin{explanation}



First, we need to select a small open interval around $3$.  We'll select $\left( 3 - \frac{1}{2}, 3 + \frac{1}{2} \right) = (2.5, 3.5)$.


On $(2.5, 3)$,  $g(t) = -(t+4)(t-2)$, which is a quadratic function.  $iRoC_g(t) = \answer{-2t - 2}$. \\

On $(2.5, 3)$,$iRoC_g(t) = -2t - 2 < 0$, therefore $g$ is \wordChoice{\choice{increasing} \choice[correct]{decreasing}} . \\

Since $g$ is decreasing on $(2.5, 3)$, we have $g(t) < -(2.5 + 4)(2.5 - 2) = -3.25 < 6$. \\



There are no domain values in the interval $(3, 3.5)$.


Together, these give us $g(x) \leq g(3)$ for all domain numbers in $(2.5, 3.5)$.

\end{explanation}










\end{exercise}
\end{document}