\documentclass{ximera}


\graphicspath{
  {./}
  {ximeraTutorial/}
  {basicPhilosophy/}
}

\newcommand{\mooculus}{\textsf{\textbf{MOOC}\textnormal{\textsf{ULUS}}}}


\usepackage{tkz-euclide}\usepackage{tikz}
\usepackage{tikz-cd}
\usetikzlibrary{arrows}
\tikzset{>=stealth,commutative diagrams/.cd,
  arrow style=tikz,diagrams={>=stealth}} %% cool arrow head
\tikzset{shorten <>/.style={ shorten >=#1, shorten <=#1 } } %% allows shorter vectors

\usetikzlibrary{backgrounds} %% for boxes around graphs
\usetikzlibrary{shapes,positioning}  %% Clouds and stars
\usetikzlibrary{matrix} %% for matrix
\usepgfplotslibrary{polar} %% for polar plots
\usepgfplotslibrary{fillbetween} %% to shade area between curves in TikZ
\usetkzobj{all}
\usepackage[makeroom]{cancel} %% for strike outs
%\usepackage{mathtools} %% for pretty underbrace % Breaks Ximera
%\usepackage{multicol}
\usepackage{pgffor} %% required for integral for loops



%% http://tex.stackexchange.com/questions/66490/drawing-a-tikz-arc-specifying-the-center
%% Draws beach ball
\tikzset{pics/carc/.style args={#1:#2:#3}{code={\draw[pic actions] (#1:#3) arc(#1:#2:#3);}}}



\usepackage{array}
\setlength{\extrarowheight}{+.1cm}
\newdimen\digitwidth
\settowidth\digitwidth{9}
\def\divrule#1#2{
\noalign{\moveright#1\digitwidth
\vbox{\hrule width#2\digitwidth}}}
























%%This is to help with formatting on future title pages.
\newenvironment{sectionOutcomes}{}{}


\title{Fractions}

\begin{document}

\begin{abstract}
packages
\end{abstract}
\maketitle



\begin{definition}  \textbf{\textcolor{green!50!black}{Fractions}} \\


\textbf{Fractions} are packages. \\


Fractions are vertically arranged packages.  There is a horizontal bar separating a top position called the \textbf{numerator} and a bottom position called the \textbf{denominator}.



\[
\frac{numerator}{denominator}
\]


There is only one rule:

\begin{center}
\textbf{\textcolor{red!70!black}{The denominator of a fraction cannot equal $0$.}}
\end{center}


\end{definition}






\begin{warning}

We do not use ``typewriter fractions''.  We don't write things like 3/4.  We write $\frac{3}{4}$.


The reason is that typewriter fractions cause confusion.

\[
(x+1)(x+3)/2 e^x (x-4) = ???
\]


If you are going by the Order of Operations, then this says


\[
\frac{(x+1)(x+3)}{2} e^x (x-4) = ???
\]


But the author could mean any one of several fractions.  We can't tell.


This confusion often shows up in the middle of algebraic calculations and everything goes off the rails.




\[
\frac{numerator}{denominator}
\]


\end{warning}



Fractions are tools we use to represent many mathematical objects.



\subsection*{Numbers} \\



We use fractions to represent numbers.  Fractions give us an unlimited supply of representations for every number.

The number four can be represented with the following fractions:

\[
4 = \frac{12}{3} = \frac{-24}{-6} = \frac{4}{1} = \frac{1}{\tfrac{1}{4}} = \frac{\tfrac{1}{5}}{\tfrac{1}{20}} = \frac{100\pi}{25\pi}
\]








\textbf{\textcolor{purple!85!blue}{$\blacktriangleright$ 1}} 



Having an endless supply of representations is very helpful when thinking about numbers, especially the number $1$. \\


\[
1 = \frac{2}{2} = \frac{-7}{-7} = \frac{\pi}{\pi} = \frac{\sqrt{2}}{\sqrt{2}} 
\]


All of these options for the number $1$, help us find alternatives for another numbers.


\[
8 = \frac{8}{1} = \frac{8}{1} \cdot 1 = \frac{8}{1} \cdot \frac{3}{3} = \frac{24}{3}
\]




\begin{idea}

The number $1$ can be represented by any fraction of the form $\frac{N}{N}$, where $N$ is any number, except $0$.


\end{idea}







Multiplication by the number $1$ is one of our most important way to compare numbers.




\begin{example}

How do we compare $\frac{17}{23}$ and $\frac{27}{37}$ ? \\

We multiply both by $1$.


\[
\frac{17}{23} = \frac{17}{23} \cdot 1 = \frac{17}{23} \cdot \frac{37}{37} = \frac{629}{851}
\]


\[
\frac{27}{37} = \frac{27}{37} \cdot 1 = \frac{27}{37} \cdot \frac{23}{23} = \frac{621}{851}
\]



\[
\frac{17}{23} > \frac{27}{37}
\]


\end{example}


All of the representations for $1$ follow the same pattern.  Both the numerator and the denominator are the same number. \\




\textbf{\textcolor{purple!85!blue}{$\blacktriangleright$ 0}} 


Similar to the number $1$, All of our fractional representations of $0$ follow a pattern.






\begin{idea}

The number $0$ can be represented by any fraction of the form $\frac{0}{N}$, where $N$ is any number, except $0$.


\end{idea}

\textbf{Note:} the reason $1$ and $0$ cannot be represented with a fraction whose denominator equals $0$ is because fractions cannot have denominators equal to $0$.






\begin{warning}


$\frac{N}{0}$ is NOT a fraction, no matter what $N$ is.



\end{warning}




\begin{example}


For what values of $A$ is  $\frac{(A-3)(A-5)}{(A+2)(A-5)} = 0$ ?


\begin{explanation}

$\frac{(A-3)(A-5)}{(A+2)(A-5)} = 0$ when $A = 3$. \\


If $A = 5$, then $\frac{(A-3)(A-5)}{(A+2)(A-5)} = \frac{2\cdot0}{7\cdot0} = \frac{0}{0}$, which isn't a fraction.



\end{explanation}


\end{example}



\subsection*{Ratios and Rates}


We use fractions to represent ratios and rates between measurements. \\




The rate  ``$24$ hours per day'' can be represented with the fraction $\frac{24 hours}{1 day}$. \\



In this context, $\frac{24 \, hours}{1 \, day} = 1$, since $24 \, hours = 1 \, day$. \\


We use fractions to represent rates when thinking about dimensional analysis. \\





\[
4 \, days = 4 \, days \cdot \frac{24 \, hours}{1 \, day} \cdot \frac{60 \, mins}{1 \, hour} \cdot \frac{60 \, secs}{1 \, min} = 345600 \, seconds
\]






\subsection*{Quotient Functions}


We use fractions to represent quotient functions, which we will study in this course. \\




\[
\frac{x+1}{\sqrt{x}}
\]


\[
\frac{e^{2t}}{\cos(t)}
\]


\[
\frac{\ln(2k+1)-5}{7 - |3k+6|}
\]



\subsection*{Arithmetic}



No matter what you are representing with fractions, they all follow the same arithmetic. \\



\begin{formula}

\[
\frac{A}{B} + \frac{C}{B} = \frac{A + C}{B}
\]

\end{formula}







\begin{formula}

\[
\frac{A}{B} \cdot \frac{C}{D} = \frac{A \cdot C}{B \cdot D}
\]

\end{formula}









\begin{example}


Create a single fraction equivalent to the sum $\frac{4x+1}{x-2} + \frac{1}{x}$. \\

\[
\frac{4x+1}{x-2} + \frac{3}{x}
\]


\[
\frac{4x+1}{x-2} \cdot 1 + \frac{3}{x} \cdot 1
\]


\[
\frac{4x+1}{x-2} \cdot \frac{x}{x} + \frac{3}{x} \cdot \frac{x-2}{x-2}
\]



\[
\frac{(4x+1)x}{(x-2)x} + \frac{3(x-2)}{x(x-2)}
\]






\[
\frac{(4x+1)x + 3(x-2)}{(x-2)x} 
\]








\end{example}





































\begin{center}
\textbf{\textcolor{green!50!black}{ooooo-=-=-=-ooOoo-=-=-=-ooooo}} \\

more examples can be found by following this link\\ \link[More Examples of Real-Valued Functions]{https://ximera.osu.edu/csccmathematics/precalculus1/precalculus1/realValued/examples/exampleList}

\end{center}












\end{document}

