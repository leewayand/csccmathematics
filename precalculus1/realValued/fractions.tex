\documentclass{ximera}

%\usepackage{todonotes}

\newcommand{\todo}{}

\usepackage{esint} % for \oiint
\ifxake%%https://math.meta.stackexchange.com/questions/9973/how-do-you-render-a-closed-surface-double-integral
\renewcommand{\oiint}{{\large\bigcirc}\kern-1.56em\iint}
\fi


\graphicspath{
  {./}
  {ximeraTutorial/}
  {basicPhilosophy/}
  {functionsOfSeveralVariables/}
  {normalVectors/}
  {lagrangeMultipliers/}
  {vectorFields/}
  {greensTheorem/}
  {shapeOfThingsToCome/}
  {dotProducts/}
  {partialDerivativesAndTheGradientVector/}
  {../productAndQuotientRules/exercises/}
  {../normalVectors/exercisesParametricPlots/}
  {../continuityOfFunctionsOfSeveralVariables/exercises/}
  {../partialDerivativesAndTheGradientVector/exercises/}
  {../directionalDerivativeAndChainRule/exercises/}
  {../commonCoordinates/exercisesCylindricalCoordinates/}
  {../commonCoordinates/exercisesSphericalCoordinates/}
  {../greensTheorem/exercisesCurlAndLineIntegrals/}
  {../greensTheorem/exercisesDivergenceAndLineIntegrals/}
  {../shapeOfThingsToCome/exercisesDivergenceTheorem/}
  {../greensTheorem/}
  {../shapeOfThingsToCome/}
  {../separableDifferentialEquations/exercises/}
  {vectorFields/}
}

\newcommand{\mooculus}{\textsf{\textbf{MOOC}\textnormal{\textsf{ULUS}}}}

\usepackage{tkz-euclide}
\usepackage{tikz}
\usepackage{tikz-cd}
\usetikzlibrary{arrows}
\tikzset{>=stealth,commutative diagrams/.cd,
  arrow style=tikz,diagrams={>=stealth}} %% cool arrow head
\tikzset{shorten <>/.style={ shorten >=#1, shorten <=#1 } } %% allows shorter vectors

\usetikzlibrary{backgrounds} %% for boxes around graphs
\usetikzlibrary{shapes,positioning}  %% Clouds and stars
\usetikzlibrary{matrix} %% for matrix
\usepgfplotslibrary{polar} %% for polar plots
\usepgfplotslibrary{fillbetween} %% to shade area between curves in TikZ
%\usetkzobj{all}
\usepackage[makeroom]{cancel} %% for strike outs
%\usepackage{mathtools} %% for pretty underbrace % Breaks Ximera
%\usepackage{multicol}
\usepackage{pgffor} %% required for integral for loops



%% http://tex.stackexchange.com/questions/66490/drawing-a-tikz-arc-specifying-the-center
%% Draws beach ball
\tikzset{pics/carc/.style args={#1:#2:#3}{code={\draw[pic actions] (#1:#3) arc(#1:#2:#3);}}}



\usepackage{array}
\setlength{\extrarowheight}{+.1cm}
\newdimen\digitwidth
\settowidth\digitwidth{9}
\def\divrule#1#2{
\noalign{\moveright#1\digitwidth
\vbox{\hrule width#2\digitwidth}}}




% \newcommand{\RR}{\mathbb R}
% \newcommand{\R}{\mathbb R}
% \newcommand{\N}{\mathbb N}
% \newcommand{\Z}{\mathbb Z}

\newcommand{\sagemath}{\textsf{SageMath}}


%\renewcommand{\d}{\,d\!}
%\renewcommand{\d}{\mathop{}\!d}
%\newcommand{\dd}[2][]{\frac{\d #1}{\d #2}}
%\newcommand{\pp}[2][]{\frac{\partial #1}{\partial #2}}
% \renewcommand{\l}{\ell}
%\newcommand{\ddx}{\frac{d}{\d x}}

% \newcommand{\zeroOverZero}{\ensuremath{\boldsymbol{\tfrac{0}{0}}}}
%\newcommand{\inftyOverInfty}{\ensuremath{\boldsymbol{\tfrac{\infty}{\infty}}}}
%\newcommand{\zeroOverInfty}{\ensuremath{\boldsymbol{\tfrac{0}{\infty}}}}
%\newcommand{\zeroTimesInfty}{\ensuremath{\small\boldsymbol{0\cdot \infty}}}
%\newcommand{\inftyMinusInfty}{\ensuremath{\small\boldsymbol{\infty - \infty}}}
%\newcommand{\oneToInfty}{\ensuremath{\boldsymbol{1^\infty}}}
%\newcommand{\zeroToZero}{\ensuremath{\boldsymbol{0^0}}}
%\newcommand{\inftyToZero}{\ensuremath{\boldsymbol{\infty^0}}}



% \newcommand{\numOverZero}{\ensuremath{\boldsymbol{\tfrac{\#}{0}}}}
% \newcommand{\dfn}{\textbf}
% \newcommand{\unit}{\,\mathrm}
% \newcommand{\unit}{\mathop{}\!\mathrm}
% \newcommand{\eval}[1]{\bigg[ #1 \bigg]}
% \newcommand{\seq}[1]{\left( #1 \right)}
% \renewcommand{\epsilon}{\varepsilon}
% \renewcommand{\phi}{\varphi}


% \renewcommand{\iff}{\Leftrightarrow}

% \DeclareMathOperator{\arccot}{arccot}
% \DeclareMathOperator{\arcsec}{arcsec}
% \DeclareMathOperator{\arccsc}{arccsc}
% \DeclareMathOperator{\si}{Si}
% \DeclareMathOperator{\scal}{scal}
% \DeclareMathOperator{\sign}{sign}


%% \newcommand{\tightoverset}[2]{% for arrow vec
%%   \mathop{#2}\limits^{\vbox to -.5ex{\kern-0.75ex\hbox{$#1$}\vss}}}
% \newcommand{\arrowvec}[1]{{\overset{\rightharpoonup}{#1}}}
% \renewcommand{\vec}[1]{\arrowvec{\mathbf{#1}}}
% \renewcommand{\vec}[1]{{\overset{\boldsymbol{\rightharpoonup}}{\mathbf{#1}}}}

% \newcommand{\point}[1]{\left(#1\right)} %this allows \vector{ to be changed to \vector{ with a quick find and replace
% \newcommand{\pt}[1]{\mathbf{#1}} %this allows \vec{ to be changed to \vec{ with a quick find and replace
% \newcommand{\Lim}[2]{\lim_{\point{#1} \to \point{#2}}} %Bart, I changed this to point since I want to use it.  It runs through both of the exercise and exerciseE files in limits section, which is why it was in each document to start with.

% \DeclareMathOperator{\proj}{\mathbf{proj}}
% \newcommand{\veci}{{\boldsymbol{\hat{\imath}}}}
% \newcommand{\vecj}{{\boldsymbol{\hat{\jmath}}}}
% \newcommand{\veck}{{\boldsymbol{\hat{k}}}}
% \newcommand{\vecl}{\vec{\boldsymbol{\l}}}
% \newcommand{\uvec}[1]{\mathbf{\hat{#1}}}
% \newcommand{\utan}{\mathbf{\hat{t}}}
% \newcommand{\unormal}{\mathbf{\hat{n}}}
% \newcommand{\ubinormal}{\mathbf{\hat{b}}}

% \newcommand{\dotp}{\bullet}
% \newcommand{\cross}{\boldsymbol\times}
% \newcommand{\grad}{\boldsymbol\nabla}
% \newcommand{\divergence}{\grad\dotp}
% \newcommand{\curl}{\grad\cross}
%\DeclareMathOperator{\divergence}{divergence}
%\DeclareMathOperator{\curl}[1]{\grad\cross #1}
% \newcommand{\lto}{\mathop{\longrightarrow\,}\limits}

% \renewcommand{\bar}{\overline}

\colorlet{textColor}{black}
\colorlet{background}{white}
\colorlet{penColor}{blue!50!black} % Color of a curve in a plot
\colorlet{penColor2}{red!50!black}% Color of a curve in a plot
\colorlet{penColor3}{red!50!blue} % Color of a curve in a plot
\colorlet{penColor4}{green!50!black} % Color of a curve in a plot
\colorlet{penColor5}{orange!80!black} % Color of a curve in a plot
\colorlet{penColor6}{yellow!70!black} % Color of a curve in a plot
\colorlet{fill1}{penColor!20} % Color of fill in a plot
\colorlet{fill2}{penColor2!20} % Color of fill in a plot
\colorlet{fillp}{fill1} % Color of positive area
\colorlet{filln}{penColor2!20} % Color of negative area
\colorlet{fill3}{penColor3!20} % Fill
\colorlet{fill4}{penColor4!20} % Fill
\colorlet{fill5}{penColor5!20} % Fill
\colorlet{gridColor}{gray!50} % Color of grid in a plot

\newcommand{\surfaceColor}{violet}
\newcommand{\surfaceColorTwo}{redyellow}
\newcommand{\sliceColor}{greenyellow}




\pgfmathdeclarefunction{gauss}{2}{% gives gaussian
  \pgfmathparse{1/(#2*sqrt(2*pi))*exp(-((x-#1)^2)/(2*#2^2))}%
}


%%%%%%%%%%%%%
%% Vectors
%%%%%%%%%%%%%

%% Simple horiz vectors
\renewcommand{\vector}[1]{\left\langle #1\right\rangle}


%% %% Complex Horiz Vectors with angle brackets
%% \makeatletter
%% \renewcommand{\vector}[2][ , ]{\left\langle%
%%   \def\nextitem{\def\nextitem{#1}}%
%%   \@for \el:=#2\do{\nextitem\el}\right\rangle%
%% }
%% \makeatother

%% %% Vertical Vectors
%% \def\vector#1{\begin{bmatrix}\vecListA#1,,\end{bmatrix}}
%% \def\vecListA#1,{\if,#1,\else #1\cr \expandafter \vecListA \fi}

%%%%%%%%%%%%%
%% End of vectors
%%%%%%%%%%%%%

%\newcommand{\fullwidth}{}
%\newcommand{\normalwidth}{}



%% makes a snazzy t-chart for evaluating functions
%\newenvironment{tchart}{\rowcolors{2}{}{background!90!textColor}\array}{\endarray}

%%This is to help with formatting on future title pages.
\newenvironment{sectionOutcomes}{}{}



%% Flowchart stuff
%\tikzstyle{startstop} = [rectangle, rounded corners, minimum width=3cm, minimum height=1cm,text centered, draw=black]
%\tikzstyle{question} = [rectangle, minimum width=3cm, minimum height=1cm, text centered, draw=black]
%\tikzstyle{decision} = [trapezium, trapezium left angle=70, trapezium right angle=110, minimum width=3cm, minimum height=1cm, text centered, draw=black]
%\tikzstyle{question} = [rectangle, rounded corners, minimum width=3cm, minimum height=1cm,text centered, draw=black]
%\tikzstyle{process} = [rectangle, minimum width=3cm, minimum height=1cm, text centered, draw=black]
%\tikzstyle{decision} = [trapezium, trapezium left angle=70, trapezium right angle=110, minimum width=3cm, minimum height=1cm, text centered, draw=black]


\title{Fractions}

\begin{document}

\begin{abstract}
packages
\end{abstract}
\maketitle



\begin{definition}  \textbf{\textcolor{green!50!black}{Fractions}} \\


\textbf{Fractions} are packages. \\


Fractions are vertically arranged packages.  There is a horizontal bar separating a top position called the \textbf{numerator} and a bottom position called the \textbf{denominator}.



\[
\frac{numerator}{denominator}
\]


There is only one rule:

\begin{center}
\textbf{\textcolor{red!70!black}{The denominator of a fraction cannot equal $0$.}}
\end{center}


\end{definition}


Fractions are tools we use to represent many mathematical objects.



\subsection*{Numbers} \\



We use fractions to represent numbers.  Fractions give us an unlimited supply of representations for every number.

The number four can be represented with the following fractions:

\[
4 = \frac{12}{3} = \frac{-24}{-6} = \frac{4}{1} = \frac{1}{\tfrac{1}{4}} = \frac{\tfrac{1}{5}}{\tfrac{1}{20}} = \frac{100\pi}{25\pi}
\]








\textbf{\textcolor{purple!85!blue}{$\blacktriangleright$ 1}} 



Having an endless supply of representations is very helpful when thinking about numbers, especially the number $1$. \\


\[
1 = \frac{2}{2} = \frac{-7}{-7} = \frac{\pi}{\pi} = \frac{\sqrt{2}}{\sqrt{2}} 
\]


All of these options for the number $1$, help us find alternatives for another numbers.


\[
8 = \frac{8}{1} = \frac{8}{1} \cdot 1 = \frac{8}{1} \cdot \frac{3}{3} = \frac{24}{3}
\]




\begin{idea}

The number $1$ can be represented by any fraction of the form $\frac{N}{N}$, where $N$ is any number, except $0$.


\end{idea}







Multiplication by the number $1$ is one of our most important way to compare numbers.




\begin{example}

How do we compare $\frac{17}{23}$ and $\frac{27}{37}$ ? \\

We multiply both by $1$.


\[
\frac{17}{23} = \frac{17}{23} \cdot 1 = \frac{17}{23} \cdot \frac{37}{37} = \frac{629}{851}
\]


\[
\frac{27}{37} = \frac{27}{37} \cdot 1 = \frac{27}{37} \cdot \frac{23}{23} = \frac{621}{851}
\]



\[
\frac{17}{23} > \frac{27}{37}
\]


\end{example}


All of the representations for $1$ follow the same pattern.  Both the numerator and the denominator are the same number. \\




\textbf{\textcolor{purple!85!blue}{$\blacktriangleright$ 0}} 


Similar to the number $1$, All of our fractional representations of $0$ follow a pattern.






\begin{idea}

The number $0$ can be represented by any fraction of the form $\frac{0}{N}$, where $N$ is any number, except $0$.


\end{idea}

\textbf{Note:} the reason $1$ and $0$ cannot be represented with a fraction whose denominator equals $0$ is because fractions cannot have denominators equal to $0$.






\begin{warning}


$\frac{N}{0}$ is NOT a fraction, no matter what $N$ is.



\end{warning}




\begin{example}


For what values of $A$ is  $\frac{(A-3)(A-5)}{(A+2)(A-5)} = 0$ ?


\begin{explanation}

$\frac{(A-3)(A-5)}{(A+2)(A-5)} = 0$ when $A = 3$. \\


If $A = 5$, then $\frac{(A-3)(A-5)}{(A+2)(A-5)} = \frac{2\cdot0}{7\cdot0} = \frac{0}{0}$, which isn't a fraction.



\end{explanation}


\end{example}



\subsection*{Ratios and Rates}


We use fractions to represent ratios and rates between measurements. \\




The rate  ``$24$ hours per day'' can be represented with the fraction $\frac{24 hours}{1 day}$. \\



In this context, $\frac{24 \, hours}{1 \, day} = 1$, since $24 \, hours = 1 \, day$. \\


We use fractions to represent rates when thinking about dimensional analysis. \\





\[
4 \, days = 4 \, days \cdot \frac{24 \, hours}{1 \, day} \cdot \frac{60 \, mins}{1 \, hour} \cdot \frac{60 \, secs}{1 \, min} = 345600 \, seconds
\]






\subsection*{Quotient Functions}


We use fractions to represent quotient functions, which we will study in this course. \\




\[
\frac{x+1}{\sqrt{x}}
\]


\[
\frac{e^{2t}}{\cos(t)}
\]


\[
\frac{\ln(2k+1)-5}{7 - |3k+6|}
\]



\subsection*{Arithmetic}



No matter what you are representing with fractions, they all follow the same arithmetic. \\



\begin{formula}

\[
\frac{A}{B} + \frac{C}{B} = \frac{A + C}{B}
\]

\end{formula}







\begin{formula}

\[
\frac{A}{B} \cdot \frac{C}{D} = \frac{A \cdot C}{B \cdot D}
\]

\end{formula}











































\begin{center}
\textbf{\textcolor{green!50!black}{ooooo-=-=-=-ooOoo-=-=-=-ooooo}} \\

more examples can be found by following this link\\ \link[More Examples of Real-Valued Functions]{https://ximera.osu.edu/csccmathematics/precalculus1/precalculus1/realValued/examples/exampleList}

\end{center}












\end{document}

