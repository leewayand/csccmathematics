\documentclass{ximera}


\graphicspath{
  {./}
  {ximeraTutorial/}
  {basicPhilosophy/}
}

\newcommand{\mooculus}{\textsf{\textbf{MOOC}\textnormal{\textsf{ULUS}}}}


\usepackage{tkz-euclide}\usepackage{tikz}
\usepackage{tikz-cd}
\usetikzlibrary{arrows}
\tikzset{>=stealth,commutative diagrams/.cd,
  arrow style=tikz,diagrams={>=stealth}} %% cool arrow head
\tikzset{shorten <>/.style={ shorten >=#1, shorten <=#1 } } %% allows shorter vectors

\usetikzlibrary{backgrounds} %% for boxes around graphs
\usetikzlibrary{shapes,positioning}  %% Clouds and stars
\usetikzlibrary{matrix} %% for matrix
\usepgfplotslibrary{polar} %% for polar plots
\usepgfplotslibrary{fillbetween} %% to shade area between curves in TikZ
\usetkzobj{all}
\usepackage[makeroom]{cancel} %% for strike outs
%\usepackage{mathtools} %% for pretty underbrace % Breaks Ximera
%\usepackage{multicol}
\usepackage{pgffor} %% required for integral for loops



%% http://tex.stackexchange.com/questions/66490/drawing-a-tikz-arc-specifying-the-center
%% Draws beach ball
\tikzset{pics/carc/.style args={#1:#2:#3}{code={\draw[pic actions] (#1:#3) arc(#1:#2:#3);}}}



\usepackage{array}
\setlength{\extrarowheight}{+.1cm}
\newdimen\digitwidth
\settowidth\digitwidth{9}
\def\divrule#1#2{
\noalign{\moveright#1\digitwidth
\vbox{\hrule width#2\digitwidth}}}
























%%This is to help with formatting on future title pages.
\newenvironment{sectionOutcomes}{}{}


\title{Shifting Domain}

\begin{document}

\begin{abstract}
same characteristics
\end{abstract}
\maketitle








Below is the piecewise defined function, $T(v)$.  $v$ is representing the domain values from $(-4,-1] \cup [1,7)$.




\[
T(v) = 
\begin{cases}
  2v-1 & \text{ if }  -4 < v \leq -1 \\
  -v+3 & \text{ if } 1 \leq v < 7
\end{cases}
\]


On the interval $(-4, -1]$, the graph should be a line for $T(v) = 2v-1$. On the interval $[1, 7)$, the graph should be another line for $T(v) = -v+3$




Graph of $y = T(v)$.
\begin{image}
\begin{tikzpicture}
	\begin{axis}[
            domain=-10:10, ymax=10, xmax=10, ymin=-10, xmin=-10,
            axis lines =center, xlabel=$v$, ylabel=$y$,
            every axis y label/.style={at=(current axis.above origin),anchor=south},
            every axis x label/.style={at=(current axis.right of origin),anchor=west},
            axis on top
          ]
          
	\addplot [draw=penColor,very thick,smooth,domain=(-4:-1)] {2*x-1};
	\addplot [draw=penColor,very thick,smooth,domain=(1:7)] {-x+3};
	\addplot[color=penColor,only marks,mark=*] coordinates{(-1,-3)}; 
	\addplot[color=penColor,fill=white,only marks,mark=*] coordinates{(-4,-9)}; 
	\addplot[color=penColor,only marks,mark=*] coordinates{(1,2)}; 
	\addplot[color=penColor,fill=white,only marks,mark=*] coordinates{(7,-4)}; 


    \end{axis}
\end{tikzpicture}
\end{image}







The domain of $T$ has two maximal intervals:, $(-4,-1]$ and $[1,7)$.  These correspond to two line segments on the graph. The endpoints give us four important points on the graph: 

\begin{itemize}

\item $(-4, -9)$, which is an open point on the graph.
\item $(-1, -3)$, which is a closed point on the graph.
\item $(1, 2)$, which is a closed point on the graph.
\item $(7, -4)$, which is an open point on the graph.

\end{itemize}




\section{A New Function}

$B(k)$ is a new function, but its definition is based on $T(v)$.


$B(k) = T(k+3)$ with the implied domain.


\textbf{First Question:} What is the domain of $B$?

$k$ is representing the values of the domain of $B$ and $k+3$ is now representing the domain values of $T$.  Therefore $k+3$ must take on the values in $(-4,-1] \cup [1,7)$.

\[     k+3 \in      (-4,-1] \cup [1,7)     \]


$k$ must be in a simialr set, but shifted tothe left by $3$.


\[     k \in      (-7,-4] \cup [-2,4)     \]


The domain of $B$ is $(-7,-4] \cup [-2,4)$.   It was implied from the domain of $T$ and the definieiotn of $B$, which is using $T$.  The values in the domain of $B$ are real numbers, $k$, such that $k+3$ is in the domain of $T$.

The domain has shifted. 


But, the structure of the function hasn't changed.  The same formulas are in use.  They are just used by different domain numbers.



\begin{itemize}

\item When $k \in (-7,-4]$, then $k+3 \in (-4,-1]$.  $v=k+3$, so $v \in (-4,-1]$ and we are using the formula $2v-1$ with $v=k+3$.
All together, when $k \in (-7,-4]$, then $B(k) = 2(k+3)-1 = 2k+5$.

\item When $k \in [-2,4)$, then $k+3 \in [1,7)$.  $v=k+3$, so $v \in [1,7)$ and we are using the formula $-v+3$ with $v=k+3$.
All together, when $k \in [-2,4)$, then $B(k) = -(k+3)+3 = -k$.


\end{itemize}









\[
B(k) = 
\begin{cases}
  2k+5 & \text{ if }  -7 < k \leq -4 \\
  -k & \text{ if } -2 \leq k < 4
\end{cases}
\]





Graph of $z = B(k)$.

\begin{image}
\begin{tikzpicture}
	\begin{axis}[
            domain=-10:10, ymax=10, xmax=10, ymin=-10, xmin=-10,
            axis lines =center, xlabel=$v$, ylabel=$z$,
            every axis y label/.style={at=(current axis.above origin),anchor=south},
            every axis x label/.style={at=(current axis.right of origin),anchor=west},
            axis on top
          ]
          
	\addplot [draw=penColor,very thick,smooth,domain=(-7:-4)] {2*x+5};
	\addplot [draw=penColor,very thick,smooth,domain=(-2:4)] {-x};
	\addplot[color=penColor,only marks,mark=*] coordinates{(-4,-3)}; 
	\addplot[color=penColor,fill=white,only marks,mark=*] coordinates{(-7,-9)}; 
	\addplot[color=penColor,only marks,mark=*] coordinates{(-2,2)}; 
	\addplot[color=penColor,fill=white,only marks,mark=*] coordinates{(4,-4)}; 


    \end{axis}
\end{tikzpicture}
\end{image}






When we look at the graphs side-by-side, we can see that the graph has simply shifted left $3$.








\begin{image}
\begin{tikzpicture}
	\begin{axis}[name = leftgraph,
            domain=-10:10, ymax=10, xmax=10, ymin=-10, xmin=-10,
            axis lines =center, xlabel=$v$, ylabel=$y$,
            every axis y label/.style={at=(current axis.above origin),anchor=south},
            every axis x label/.style={at=(current axis.right of origin),anchor=west},
            axis on top
          ]
          
	\addplot [draw=penColor,very thick,smooth,domain=(-4:-1)] {2*x-1};
	\addplot [draw=penColor,very thick,smooth,domain=(1:7)] {-x+3};
	\addplot[color=penColor,only marks,mark=*] coordinates{(-1,-3)}; 
	\addplot[color=penColor,fill=white,only marks,mark=*] coordinates{(-4,-9)}; 
	\addplot[color=penColor,only marks,mark=*] coordinates{(1,2)}; 
	\addplot[color=penColor,fill=white,only marks,mark=*] coordinates{(7,-4)}; 


    \end{axis}
	\begin{axis}[at={(leftgraph.outer east)},anchor=outer west, 
            domain=-10:10, ymax=10, xmax=10, ymin=-10, xmin=-10,
            axis lines =center, xlabel=$v$, ylabel=$z$,
            every axis y label/.style={at=(current axis.above origin),anchor=south},
            every axis x label/.style={at=(current axis.right of origin),anchor=west},
            axis on top
          ]
          
	\addplot [draw=penColor,very thick,smooth,domain=(-7:-4)] {2*x+5};
	\addplot [draw=penColor,very thick,smooth,domain=(-2:4)] {-x};
	\addplot[color=penColor,only marks,mark=*] coordinates{(-4,-3)}; 
	\addplot[color=penColor,fill=white,only marks,mark=*] coordinates{(-7,-9)}; 
	\addplot[color=penColor,only marks,mark=*] coordinates{(-2,2)}; 
	\addplot[color=penColor,fill=white,only marks,mark=*] coordinates{(4,-4)}; 


    \end{axis}


\end{tikzpicture}
\end{image}




From the definition, $B(k) = T(k+3)$, we can see that $v=k+3$, or $k=v-3$.  All of the $k$-values are $3$ less than the $v$-values.  $B$ is $T$ shifted left $3$.





































\end{document}
