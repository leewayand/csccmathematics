\documentclass{ximera}


\graphicspath{
  {./}
  {ximeraTutorial/}
  {basicPhilosophy/}
}

\newcommand{\mooculus}{\textsf{\textbf{MOOC}\textnormal{\textsf{ULUS}}}}


\usepackage{tkz-euclide}\usepackage{tikz}
\usepackage{tikz-cd}
\usetikzlibrary{arrows}
\tikzset{>=stealth,commutative diagrams/.cd,
  arrow style=tikz,diagrams={>=stealth}} %% cool arrow head
\tikzset{shorten <>/.style={ shorten >=#1, shorten <=#1 } } %% allows shorter vectors

\usetikzlibrary{backgrounds} %% for boxes around graphs
\usetikzlibrary{shapes,positioning}  %% Clouds and stars
\usetikzlibrary{matrix} %% for matrix
\usepgfplotslibrary{polar} %% for polar plots
\usepgfplotslibrary{fillbetween} %% to shade area between curves in TikZ
\usetkzobj{all}
\usepackage[makeroom]{cancel} %% for strike outs
%\usepackage{mathtools} %% for pretty underbrace % Breaks Ximera
%\usepackage{multicol}
\usepackage{pgffor} %% required for integral for loops



%% http://tex.stackexchange.com/questions/66490/drawing-a-tikz-arc-specifying-the-center
%% Draws beach ball
\tikzset{pics/carc/.style args={#1:#2:#3}{code={\draw[pic actions] (#1:#3) arc(#1:#2:#3);}}}



\usepackage{array}
\setlength{\extrarowheight}{+.1cm}
\newdimen\digitwidth
\settowidth\digitwidth{9}
\def\divrule#1#2{
\noalign{\moveright#1\digitwidth
\vbox{\hrule width#2\digitwidth}}}
























%%This is to help with formatting on future title pages.
\newenvironment{sectionOutcomes}{}{}


\title{Function Analysis}

\begin{document}

\begin{abstract}
%Stuff can go here later if we want!
\end{abstract}
\maketitle




Analyzing a function means telling its story, describing all of its features and characteristics.  A complete analysis includes exact information from algebra and global information from graphs.  The Algebra and graphical information should match.

A complete anlaysis includes the domain and range, zeros, singularities, discontinuities, intervals of continuity, end-behavior, critical numbers, intervals where the function is increasing and decreasing, global maximum and minimums, and local maximums and minimums.

It also includes a nice graph.  Not necessarily an exact graph, but rather a graph that effectiovely communicates the function's story. It should include intercepts, dashed asymptotes, closed and hollow dots, and arrows to help the reader understand the function.  The graph should take up most of the region you have set aside for drawing.  Your drawing should not be 90\% empty space with the details of the graph squished together.  That does not communicate.
























\begin{sectionOutcomes}
In this section, students will 

\begin{itemize}
\item completely analyze functions.
\item produce nice graphs.
\item .
\item .
\item .
\end{itemize}
\end{sectionOutcomes}

\end{document}
