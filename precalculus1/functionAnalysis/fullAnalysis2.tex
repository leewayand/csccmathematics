\documentclass{ximera}

%\usepackage{todonotes}

\newcommand{\todo}{}

\usepackage{esint} % for \oiint
\ifxake%%https://math.meta.stackexchange.com/questions/9973/how-do-you-render-a-closed-surface-double-integral
\renewcommand{\oiint}{{\large\bigcirc}\kern-1.56em\iint}
\fi


\graphicspath{
  {./}
  {ximeraTutorial/}
  {basicPhilosophy/}
  {functionsOfSeveralVariables/}
  {normalVectors/}
  {lagrangeMultipliers/}
  {vectorFields/}
  {greensTheorem/}
  {shapeOfThingsToCome/}
  {dotProducts/}
  {partialDerivativesAndTheGradientVector/}
  {../productAndQuotientRules/exercises/}
  {../normalVectors/exercisesParametricPlots/}
  {../continuityOfFunctionsOfSeveralVariables/exercises/}
  {../partialDerivativesAndTheGradientVector/exercises/}
  {../directionalDerivativeAndChainRule/exercises/}
  {../commonCoordinates/exercisesCylindricalCoordinates/}
  {../commonCoordinates/exercisesSphericalCoordinates/}
  {../greensTheorem/exercisesCurlAndLineIntegrals/}
  {../greensTheorem/exercisesDivergenceAndLineIntegrals/}
  {../shapeOfThingsToCome/exercisesDivergenceTheorem/}
  {../greensTheorem/}
  {../shapeOfThingsToCome/}
  {../separableDifferentialEquations/exercises/}
  {vectorFields/}
}

\newcommand{\mooculus}{\textsf{\textbf{MOOC}\textnormal{\textsf{ULUS}}}}

\usepackage{tkz-euclide}
\usepackage{tikz}
\usepackage{tikz-cd}
\usetikzlibrary{arrows}
\tikzset{>=stealth,commutative diagrams/.cd,
  arrow style=tikz,diagrams={>=stealth}} %% cool arrow head
\tikzset{shorten <>/.style={ shorten >=#1, shorten <=#1 } } %% allows shorter vectors

\usetikzlibrary{backgrounds} %% for boxes around graphs
\usetikzlibrary{shapes,positioning}  %% Clouds and stars
\usetikzlibrary{matrix} %% for matrix
\usepgfplotslibrary{polar} %% for polar plots
\usepgfplotslibrary{fillbetween} %% to shade area between curves in TikZ
%\usetkzobj{all}
\usepackage[makeroom]{cancel} %% for strike outs
%\usepackage{mathtools} %% for pretty underbrace % Breaks Ximera
%\usepackage{multicol}
\usepackage{pgffor} %% required for integral for loops



%% http://tex.stackexchange.com/questions/66490/drawing-a-tikz-arc-specifying-the-center
%% Draws beach ball
\tikzset{pics/carc/.style args={#1:#2:#3}{code={\draw[pic actions] (#1:#3) arc(#1:#2:#3);}}}



\usepackage{array}
\setlength{\extrarowheight}{+.1cm}
\newdimen\digitwidth
\settowidth\digitwidth{9}
\def\divrule#1#2{
\noalign{\moveright#1\digitwidth
\vbox{\hrule width#2\digitwidth}}}




% \newcommand{\RR}{\mathbb R}
% \newcommand{\R}{\mathbb R}
% \newcommand{\N}{\mathbb N}
% \newcommand{\Z}{\mathbb Z}

\newcommand{\sagemath}{\textsf{SageMath}}


%\renewcommand{\d}{\,d\!}
%\renewcommand{\d}{\mathop{}\!d}
%\newcommand{\dd}[2][]{\frac{\d #1}{\d #2}}
%\newcommand{\pp}[2][]{\frac{\partial #1}{\partial #2}}
% \renewcommand{\l}{\ell}
%\newcommand{\ddx}{\frac{d}{\d x}}

% \newcommand{\zeroOverZero}{\ensuremath{\boldsymbol{\tfrac{0}{0}}}}
%\newcommand{\inftyOverInfty}{\ensuremath{\boldsymbol{\tfrac{\infty}{\infty}}}}
%\newcommand{\zeroOverInfty}{\ensuremath{\boldsymbol{\tfrac{0}{\infty}}}}
%\newcommand{\zeroTimesInfty}{\ensuremath{\small\boldsymbol{0\cdot \infty}}}
%\newcommand{\inftyMinusInfty}{\ensuremath{\small\boldsymbol{\infty - \infty}}}
%\newcommand{\oneToInfty}{\ensuremath{\boldsymbol{1^\infty}}}
%\newcommand{\zeroToZero}{\ensuremath{\boldsymbol{0^0}}}
%\newcommand{\inftyToZero}{\ensuremath{\boldsymbol{\infty^0}}}



% \newcommand{\numOverZero}{\ensuremath{\boldsymbol{\tfrac{\#}{0}}}}
% \newcommand{\dfn}{\textbf}
% \newcommand{\unit}{\,\mathrm}
% \newcommand{\unit}{\mathop{}\!\mathrm}
% \newcommand{\eval}[1]{\bigg[ #1 \bigg]}
% \newcommand{\seq}[1]{\left( #1 \right)}
% \renewcommand{\epsilon}{\varepsilon}
% \renewcommand{\phi}{\varphi}


% \renewcommand{\iff}{\Leftrightarrow}

% \DeclareMathOperator{\arccot}{arccot}
% \DeclareMathOperator{\arcsec}{arcsec}
% \DeclareMathOperator{\arccsc}{arccsc}
% \DeclareMathOperator{\si}{Si}
% \DeclareMathOperator{\scal}{scal}
% \DeclareMathOperator{\sign}{sign}


%% \newcommand{\tightoverset}[2]{% for arrow vec
%%   \mathop{#2}\limits^{\vbox to -.5ex{\kern-0.75ex\hbox{$#1$}\vss}}}
% \newcommand{\arrowvec}[1]{{\overset{\rightharpoonup}{#1}}}
% \renewcommand{\vec}[1]{\arrowvec{\mathbf{#1}}}
% \renewcommand{\vec}[1]{{\overset{\boldsymbol{\rightharpoonup}}{\mathbf{#1}}}}

% \newcommand{\point}[1]{\left(#1\right)} %this allows \vector{ to be changed to \vector{ with a quick find and replace
% \newcommand{\pt}[1]{\mathbf{#1}} %this allows \vec{ to be changed to \vec{ with a quick find and replace
% \newcommand{\Lim}[2]{\lim_{\point{#1} \to \point{#2}}} %Bart, I changed this to point since I want to use it.  It runs through both of the exercise and exerciseE files in limits section, which is why it was in each document to start with.

% \DeclareMathOperator{\proj}{\mathbf{proj}}
% \newcommand{\veci}{{\boldsymbol{\hat{\imath}}}}
% \newcommand{\vecj}{{\boldsymbol{\hat{\jmath}}}}
% \newcommand{\veck}{{\boldsymbol{\hat{k}}}}
% \newcommand{\vecl}{\vec{\boldsymbol{\l}}}
% \newcommand{\uvec}[1]{\mathbf{\hat{#1}}}
% \newcommand{\utan}{\mathbf{\hat{t}}}
% \newcommand{\unormal}{\mathbf{\hat{n}}}
% \newcommand{\ubinormal}{\mathbf{\hat{b}}}

% \newcommand{\dotp}{\bullet}
% \newcommand{\cross}{\boldsymbol\times}
% \newcommand{\grad}{\boldsymbol\nabla}
% \newcommand{\divergence}{\grad\dotp}
% \newcommand{\curl}{\grad\cross}
%\DeclareMathOperator{\divergence}{divergence}
%\DeclareMathOperator{\curl}[1]{\grad\cross #1}
% \newcommand{\lto}{\mathop{\longrightarrow\,}\limits}

% \renewcommand{\bar}{\overline}

\colorlet{textColor}{black}
\colorlet{background}{white}
\colorlet{penColor}{blue!50!black} % Color of a curve in a plot
\colorlet{penColor2}{red!50!black}% Color of a curve in a plot
\colorlet{penColor3}{red!50!blue} % Color of a curve in a plot
\colorlet{penColor4}{green!50!black} % Color of a curve in a plot
\colorlet{penColor5}{orange!80!black} % Color of a curve in a plot
\colorlet{penColor6}{yellow!70!black} % Color of a curve in a plot
\colorlet{fill1}{penColor!20} % Color of fill in a plot
\colorlet{fill2}{penColor2!20} % Color of fill in a plot
\colorlet{fillp}{fill1} % Color of positive area
\colorlet{filln}{penColor2!20} % Color of negative area
\colorlet{fill3}{penColor3!20} % Fill
\colorlet{fill4}{penColor4!20} % Fill
\colorlet{fill5}{penColor5!20} % Fill
\colorlet{gridColor}{gray!50} % Color of grid in a plot

\newcommand{\surfaceColor}{violet}
\newcommand{\surfaceColorTwo}{redyellow}
\newcommand{\sliceColor}{greenyellow}




\pgfmathdeclarefunction{gauss}{2}{% gives gaussian
  \pgfmathparse{1/(#2*sqrt(2*pi))*exp(-((x-#1)^2)/(2*#2^2))}%
}


%%%%%%%%%%%%%
%% Vectors
%%%%%%%%%%%%%

%% Simple horiz vectors
\renewcommand{\vector}[1]{\left\langle #1\right\rangle}


%% %% Complex Horiz Vectors with angle brackets
%% \makeatletter
%% \renewcommand{\vector}[2][ , ]{\left\langle%
%%   \def\nextitem{\def\nextitem{#1}}%
%%   \@for \el:=#2\do{\nextitem\el}\right\rangle%
%% }
%% \makeatother

%% %% Vertical Vectors
%% \def\vector#1{\begin{bmatrix}\vecListA#1,,\end{bmatrix}}
%% \def\vecListA#1,{\if,#1,\else #1\cr \expandafter \vecListA \fi}

%%%%%%%%%%%%%
%% End of vectors
%%%%%%%%%%%%%

%\newcommand{\fullwidth}{}
%\newcommand{\normalwidth}{}



%% makes a snazzy t-chart for evaluating functions
%\newenvironment{tchart}{\rowcolors{2}{}{background!90!textColor}\array}{\endarray}

%%This is to help with formatting on future title pages.
\newenvironment{sectionOutcomes}{}{}



%% Flowchart stuff
%\tikzstyle{startstop} = [rectangle, rounded corners, minimum width=3cm, minimum height=1cm,text centered, draw=black]
%\tikzstyle{question} = [rectangle, minimum width=3cm, minimum height=1cm, text centered, draw=black]
%\tikzstyle{decision} = [trapezium, trapezium left angle=70, trapezium right angle=110, minimum width=3cm, minimum height=1cm, text centered, draw=black]
%\tikzstyle{question} = [rectangle, rounded corners, minimum width=3cm, minimum height=1cm,text centered, draw=black]
%\tikzstyle{process} = [rectangle, minimum width=3cm, minimum height=1cm, text centered, draw=black]
%\tikzstyle{decision} = [trapezium, trapezium left angle=70, trapezium right angle=110, minimum width=3cm, minimum height=1cm, text centered, draw=black]


\title{Describe Everything}

\begin{document}

\begin{abstract}
features
\end{abstract}
\maketitle





\begin{example}  Complete Analysis


Completely analyze   

\[   B(f) = \frac{(f+3)(2f-1)}{(f+5)(f-2)}        \]

with


\[
B'(f) = \frac{f^2 - 34f - 41}{(f+5)^2(f-2)^2}
\]





Categorize:  $B(f)$ is a rational function, because it is of the form $\frac{polynomial}{polynomial}$




\textbf{Domain}

The natural domain of a rational function is all real numbers except the zeros of the denominator, which are $-5$ and $2$. \\

The domain is $(-\infty, -5) \cup (-5, 2) \cup (2, \infty)$.







\textbf{Zeros}

The zeros of a rational function are the zeros of the numerator, which are not also zeros of the denominator, i.e. they are in the domain. \\





$-3$ and $\answer{\frac{1}{2}}$ are the zeros of $B$. \\





\textbf{Continuity}


Rational functions are continuous. \\

Rational functions can have singularities.  Here the zeros of the denominator are $-5$ and $2$.  These are not zeros of the numerator.  That makes them asymptotic singularities. So, we know $B$ is unbounded near them.  We just need to figure out the sign.\\


Let's map out the signs of $B$. \\



$B$ has four zeros and/or singularities. In order, they are $-5$, $-3$, $\frac{1}{2}$, and $2$.\\

They all have multiplicity $1$, which is odd.  That means $B$ switches signs across them. \\


The end-behavior will get us started on the signs.
















\textbf{End-Behavior}


$B$ is a rational function.

The degree of the numerator is $\answer{2}$. \\


The degree of the denominator is $\answer{2}$. \\


$B$ is a rational function and the degree of the denominator is \wordChoice{\choice{greater than} \choice{less than} \choice[correct]{equal to}} the degree of the numerator. \\


Since the degrees of the numerator and denominator are equal, the end-behvior is the quotient of the leading coefficients.


\[ \lim_{f \to -\infty} B(f) = \frac{2}{1} = 2   \, \text{ and }  \,  \lim_{f \to \infty} B(f) = \frac{2}{1} = 2    \]





\begin{explanation} Continuing with singularity behavior....




We now know that $B$ is positive for large negative domain numbers.  This switches at $-5$, which is the first singularity.

$B$ is positive on $(-\infty, -5)$ \\

That tells us that 

\[
\lim\limits_{f \to -5^-} B(f) = \infty 
\]

The sign of $B$ switches at $-5$, which gives us 


\[
\lim\limits_{f \to -5^+} B(f) = -\infty 
\]



$B$ switches signs at $-3$, which is the first zero. \\


Therefore, $B$ is negative on $(-5, -3)$.  Then, $B$ switches to positive on $\left( -3, \frac{1}{2} \right)$ Then, $B$ switches to negative on $\left( \frac{1}{2}, 2 \right)$.


That tells us that 

\[
\lim\limits_{f \to 2^-} B(f) = -\infty 
\]


Then, $B$ switches to positive on $(2, \infty)$.


\[
\lim\limits_{f \to 2^+} B(f) = \infty 
\]


\end{explanation}
















\textbf{Behavior (Increasing and Decreasing)}


We'll use the derivative to help us with behavior. \\


\[
B'(f) = \frac{f^2 - 34f - 41}{(f+5)^2(f-2)^2}
\]


$B'(f)$ is also a rational function.  So, it is continous.  It has the same singularities as $B$.  But it has different zeros. \\


To get the critical numbers, we need to factor the numerator.


\[
f = \frac{34 \pm \sqrt{(-34)^2 - 4 (1)(-41)}}{2(1)} = \frac{34 \pm \sqrt{1320}}{2} = \frac{34 \pm 2\sqrt{330}}{2} = 17 \pm \sqrt{330}
\]




Let's check with a graph.

$17 - \sqrt{330} \approx -1.165902125$

$17 + \sqrt{330} \approx 35.16590212$







\textbf{\textcolor{blue!55!black}{$\blacktriangleright$ desmos graph}} 
\begin{center}
\desmos{yjhhl3llii}{400}{300}
\end{center}





\textbf{\textcolor{blue!55!black}{$\blacktriangleright$ desmos graph}} 
\begin{center}
\desmos{le6wkkt3uz}{400}{300}
\end{center}




It seems our algebra is good.



\[
B'(f) = \frac{(f - (17 - \sqrt{330}))(f - (17 + \sqrt{330}))}{(f+5)^2(f-2)^2}
\]


$B'$ has two zeros and two singularities.

\[
17 - \sqrt{330}, 17 + \sqrt{330}, -5, 2
\]

We need their order. \\




The graph suggests that $-5 < 17 - \sqrt{330}$.  Let's see if we can show that is true. \\



\[
-5 < -1 = 17 - 18 = 17 -\sqrt{324} < 17 - \sqrt{330}
\]


Since $ \sqrt{330} > \sqrt{289} = 17$, we know that $17 - \sqrt{330} < 0$. \\


We also know that $2 < 17 + \sqrt{330} $. \\


That gives us

\[
-5 < 17 - \sqrt{330} < 2 < 17 + \sqrt{330}
\]


$17 - \sqrt{330}$ and $17 + \sqrt{330}$ are zeros of $B'$ with multiplicity $1$, which is odd.  $B'$ will change signs across them.



$-5$ and $2$ are singularities of $B'$ with multiplicity $2$, which is even.  $B'$ will not change signs across them.


We just need a starting sign. \\



For very large negative numbers (anything less than $17 - \sqrt{330}$), 

\[
B'(f) = \frac{negative \cdot negative}{positive \cdot positive} = positive
\]


Therefore, the signs of $B'$ are


\begin{itemize}
  \item $B'$ is positive on $(-\infty, -5)$
  \item $B'$ is positive on $(-5, 17 - \sqrt{330})$
  \item $B'$ is negative on $(17 - \sqrt{330}, 2)$
  \item $B'$ is negative on $(2, 17 + \sqrt{330})$
  \item $B'$ is positive on $(17 + \sqrt{330}, \infty)$
\end{itemize}



This give us the behavior of $B$.





\begin{itemize}
  \item $B$ is increasing on $(-\infty, -5)$
  \item $B$ is increasing on $(-5, 17 - \sqrt{330})$
  \item $B$ is decreasing on $(17 - \sqrt{330}, 2)$
  \item $B$ is decreasing on $(2, 17 + \sqrt{330})$
  \item $B$ is increasing on $(17 + \sqrt{330}, \infty)$
\end{itemize}



















\textbf{Global Maximums and Minimums}


$B$ has no global maximum or minimum, because


\[
\lim\limits_{f \to 2^-} B(f) = -\infty 
\]




\[
\lim\limits_{f \to 2^+} B(f) = \infty 
\]





\textbf{Local Maximums and Minimums}


$B$ has two critical numbers.  They are locations for local maximums or minimums.






\begin{itemize}
  \item $B$ is increasing on $(-5, 17 - \sqrt{330})$
  \item $B$ is decreasing on $(17 - \sqrt{330}, 2)$
\end{itemize}


Therefore, $B(17 - \sqrt{330})$ is a local maximum occurring at $17 - \sqrt{330}$.





\begin{itemize}
  \item $B$ is decreasing on $(2, 17 + \sqrt{330})$
  \item $B$ is increasing on $(17 + \sqrt{330}, \infty)$
\end{itemize}



Therefore, $B(17 + \sqrt{330})$ is a local minimum occurring at $17 + \sqrt{330}$.









\textbf{Range}


To figure out the range, we need to know if there is a gap.  We need to know if $B(17 + \sqrt{330})$ or $B(17 - \sqrt{330})$ is greater. \\

The graph suggests that 

\[
B(17 + \sqrt{330})  > B(17 - \sqrt{330})
\]


That is going to be a bit of algebra. Let's ask Wolfram Alpha to simply each of those.



\[
B(17 - \sqrt{330}) = \frac{1}{49} (61 - 2 \sqrt{330})
\]




\[
B(17 + \sqrt{330})  =  \frac{1}{49} (61 + 2 \sqrt{330})
\]


Confirmed.


\[
B(17 + \sqrt{330})  > B(17 - \sqrt{330})
\]


There is a gap in the range.


The range of $B$ is 


\[
\left( -\infty, \frac{1}{49} (61 - 2 \sqrt{330}) \right] \cup  \left[ \frac{1}{49} (61 + 2 \sqrt{330}), \infty \right) 
\]





\end{example}














\begin{center}
\textbf{\textcolor{green!50!black}{ooooo-=-=-=-ooOoo-=-=-=-ooooo}} \\

more examples can be found by following this link\\ \link[More Examples of Function Analysis]{https://ximera.osu.edu/csccmathematics/precalculus1/precalculus1/functionAnalysis/examples/exampleList}

\end{center}




\end{document}
