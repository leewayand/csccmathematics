\documentclass{ximera}


\graphicspath{
  {./}
  {ximeraTutorial/}
  {basicPhilosophy/}
}

\newcommand{\mooculus}{\textsf{\textbf{MOOC}\textnormal{\textsf{ULUS}}}}


\usepackage{tkz-euclide}\usepackage{tikz}
\usepackage{tikz-cd}
\usetikzlibrary{arrows}
\tikzset{>=stealth,commutative diagrams/.cd,
  arrow style=tikz,diagrams={>=stealth}} %% cool arrow head
\tikzset{shorten <>/.style={ shorten >=#1, shorten <=#1 } } %% allows shorter vectors

\usetikzlibrary{backgrounds} %% for boxes around graphs
\usetikzlibrary{shapes,positioning}  %% Clouds and stars
\usetikzlibrary{matrix} %% for matrix
\usepgfplotslibrary{polar} %% for polar plots
\usepgfplotslibrary{fillbetween} %% to shade area between curves in TikZ
\usetkzobj{all}
\usepackage[makeroom]{cancel} %% for strike outs
%\usepackage{mathtools} %% for pretty underbrace % Breaks Ximera
%\usepackage{multicol}
\usepackage{pgffor} %% required for integral for loops



%% http://tex.stackexchange.com/questions/66490/drawing-a-tikz-arc-specifying-the-center
%% Draws beach ball
\tikzset{pics/carc/.style args={#1:#2:#3}{code={\draw[pic actions] (#1:#3) arc(#1:#2:#3);}}}



\usepackage{array}
\setlength{\extrarowheight}{+.1cm}
\newdimen\digitwidth
\settowidth\digitwidth{9}
\def\divrule#1#2{
\noalign{\moveright#1\digitwidth
\vbox{\hrule width#2\digitwidth}}}
























%%This is to help with formatting on future title pages.
\newenvironment{sectionOutcomes}{}{}


\title{Describe Everything}

\begin{document}

\begin{abstract}
features
\end{abstract}
\maketitle





\begin{example}  Complete Analysis


Completely analyze   

\[   B(f) = \frac{(f+3)(f-1)}{(f+5)(f-2)^2}        \]


\textbf{\textcolor{purple!50!blue!90!black}{explanation}}



The natural domain is all real numbers except $-5$ and $2$, which make the denominator equal to $0$. These are singularities of $B$.  Otherwise, $B$ is continuous.  There are no discontinuities.

Domain = $(-\infty, -5) \cup (-5, 2) \cup (2, \infty)$

The graph will help us with the range, so we'll wait on that.

$B$ has zeros at $-3$ and $1$.  These make the numerator equal to $0$. These are represented on the graph with $(-5,0)$ and $(2,0)$

The graph will have vertical asymptotes $f=-5$ and $f=-2$.  $B$ will change signs over $f=-5$, since the multiplicity is odd.  $B$ will not change signs over $f=2$, since the multiplicity is even.

$B$ is a rational function and the deree of the denominator is greater than the degree of the numerator. Therefore, 


\[ \lim_{f \to -\infty} B(f) = 0   \, \text{ and }  \,  \lim_{f \to \infty} B(f) = 0    \]


The graph has a horizontal asymptote at $y=0$. \\

From this, we can sketch a graph of $y = B(t)$. \\






\begin{image}
\begin{tikzpicture}
  \begin{axis}[
            domain=-10:10, ymax=10, xmax=10, ymin=-10, xmin=-10,
            axis lines =center, xlabel=$f$, ylabel={$y=B(f)$}, grid = major, grid style={dashed},
            ytick={-10,-8,-6,-4,-2,2,4,6,8,10},
            xtick={-10,-8,-6,-4,-2,2,4,6,8,10},
            yticklabels={$-10$,$-8$,$-6$,$-4$,$-2$,$2$,$4$,$6$,$8$,$10$}, 
            xticklabels={$-10$,$-8$,$-6$,$-4$,$-2$,$2$,$4$,$6$,$8$,$10$},
            ticklabel style={font=\scriptsize},
            every axis y label/.style={at=(current axis.above origin),anchor=south},
            every axis x label/.style={at=(current axis.right of origin),anchor=west},
            axis on top
          ]
          

          \addplot [line width=1, gray, dashed, domain=(-9.5:9.5),<->] ({-5},{x});
          \addplot [line width=1, gray, dashed, domain=(-9.5:9.5),<->] ({2},{x});
          \addplot [line width=1, gray, dashed, domain=(-10:10)] {0};



          \addplot [line width=2, penColor, smooth,samples=200,domain=(-9:-5.03),<->] {((x+3)*(x-1))/((x+5)*(x-2)^2)};
          \addplot [line width=2, penColor, smooth,samples=200,domain=(-4.97:1.75),<->] {((x+3)*(x-1))/((x+5)*(x-2)^2)};
          \addplot [line width=2, penColor, smooth,samples=200,domain=(2.3:9),<->] {((x+3)*(x-1))/((x+5)*(x-2)^2)};

          \addplot[color=penColor,fill=penColor,only marks,mark=*] coordinates{(-3,0)};
          \addplot[color=penColor,fill=penColor,only marks,mark=*] coordinates{(1,0)};


           

  \end{axis}
\end{tikzpicture}
\end{image}


Both zeros have odd multiplicity, therefore the graph must go below the $f$-axis between them.  There must be a local minimum. From the graph we can estimate that the critical number is approximately $0.5$. $B(0.5) = \frac{14}{99} \approx -0.1414$ is a local minimum. With some Calculus tools, we may be able to get an exact value.


$B$ has no global maximum or minimum.

$B$ has no local maximum.


\begin{itemize}
\item $B$ decreases on $(-\infty, -5)$.
\item $B$ decreases on $(-5, 0.5)$.
\item $B$ increases on $(0.5 2)$.
\item $B$ decreases on $(2, \infty)$.
\end{itemize}



The graph makes is evident that the range is all real numbers.


\end{example}








With some graphing tools, we can get a better approximation of the critical number.




\begin{center}
\desmos{sumsbfeafr}{400}{300}
\end{center}



The critical number is approximately $0.181$ and the local minimum value of $B$ is approximately $-1.152$.









\begin{example} Analyze


Completely analyze $K(w) = w \sqrt{9-w}$.


\textbf{\textcolor{purple!50!blue!90!black}{explanation}}



The domain is restricted by the square root to be $(-\infty, \answer{9}]$.


The formula is a product, therefore the zeros come from each factor.

$w$ gives a zero of $\answer{0}$.

$\sqrt{9-w}$ gives a zero of $\answer{9}$.


Since $\sqrt{9-w}$ is always positive, the sign of $K$ is the same as the sign of $w$.

\begin{itemize}
\item $K$ is negative on $\left( \answer{-\infty},\answer{0} \right)$
\item $K$ is positive on $\left( \answer{0}, \answer{\infty} \right)$
\end{itemize}




From this, we can sketch a graph of $y = K(w)$. \\






\begin{image}
\begin{tikzpicture}
  \begin{axis}[
            domain=-10:10, ymax=10, xmax=10, ymin=-10, xmin=-10,
            axis lines =center, xlabel=$w$, ylabel={$y=K(w)$}, grid = major, grid style={dashed},
            ytick={-10,-8,-6,-4,-2,2,4,6,8,10},
            xtick={-10,-8,-6,-4,-2,2,4,6,8,10},
            yticklabels={$-10$,$-8$,$-6$,$-4$,$-2$,$2$,$4$,$6$,$8$,$10$}, 
            xticklabels={$-10$,$-8$,$-6$,$-4$,$-2$,$2$,$4$,$6$,$8$,$10$},
            ticklabel style={font=\scriptsize},
            every axis y label/.style={at=(current axis.above origin),anchor=south},
            every axis x label/.style={at=(current axis.right of origin),anchor=west},
            axis on top
          ]


          \addplot [line width=2, penColor, smooth,samples=200,domain=(-2.5:0.5)] {x * sqrt(9-x)};
          \addplot [line width=2, penColor, smooth,samples=200,domain=(8.8:9)] {x * sqrt(9-x)};


          \addplot[color=penColor,fill=penColor,only marks,mark=*] coordinates{(0,0)};
          \addplot[color=penColor,fill=penColor,only marks,mark=*] coordinates{(9,0)};


           

  \end{axis}
\end{tikzpicture}
\end{image}


Our sketch suggest a global maximum between $0$ and $9$.


If we had the derivative, then we could identify this critical number exactly. Currently, we'll need a graph to approximate values.




\begin{center}
\desmos{zbhmb5usvv}{400}{300}
\end{center}





\begin{itemize}
\item $K$ has a local and global maximum of $10.39$ at $6$.
\item $K$ has a local minimum of $0$ at $9$.
\item $K$ has no global minimum.
\end{itemize}




\begin{itemize}
\item $K$ is increasing on $(-\infty, 6]$.
\item $K$ is decreasing on $[6, 9]$.
\end{itemize}


Finally, the range is $(-\infty, 10.39]$. \\


The graph has no vertical asymptotes, $\lim\limits_{w \to -\infty} K(w) = -\infty.$


\end{example}



We do have alternatives to the derivative ofr some types of functions. \\







\begin{procedure} Critical Number


Earlier, we saw an algebraic method of identifying this critical number.


We think there is a hump in the graph of $K$ and at the highest point, the tangent line would be horizontal.



\begin{image}
\begin{tikzpicture}
  \begin{axis}[
            domain=-10:10, ymax=13, xmax=10, ymin=-10, xmin=-10,
            axis lines =center, xlabel=$w$, ylabel={$y=K(w)$}, grid = major, grid style={dashed},
            ytick={-10,-8,-6,-4,-2,2,4,6,8,10,12},
            xtick={-10,-8,-6,-4,-2,2,4,6,8,10},
            yticklabels={$-10$,$-8$,$-6$,$-4$,$-2$,$2$,$4$,$6$,$8$,$10$,$12$}, 
            xticklabels={$-10$,$-8$,$-6$,$-4$,$-2$,$2$,$4$,$6$,$8$,$10$},
            ticklabel style={font=\scriptsize},
            every axis y label/.style={at=(current axis.above origin),anchor=south},
            every axis x label/.style={at=(current axis.right of origin),anchor=west},
            axis on top
          ]


          \addplot [line width=2, penColor, smooth,samples=200,domain=(-2.5:8.1),<-] {x * sqrt(9-x)};
          \addplot [line width=2, penColor, smooth,samples=300,domain=(8:9)] {x * sqrt(9-x)};
          \addplot [line width=2, penColor2, smooth,samples=200,domain=(-2.5:9),<->] {10.39};
         


          \addplot[color=penColor,fill=penColor,only marks,mark=*] coordinates{(0,0)};
          \addplot[color=penColor,fill=penColor,only marks,mark=*] coordinates{(9,0)};
          \addplot[color=penColor2,fill=penColor2,only marks,mark=*] coordinates{(6,10.39)};


          \node at (axis cs:6,12) [penColor] {$(A, A\sqrt{9-A})$};


           

  \end{axis}
\end{tikzpicture}
\end{image}


Now, create a new function called $F$.

\[
F(x) = K(x) - K(A) = K(x) - A\sqrt{9-A}
\]


As we saw earlier, since we have a tangent line, the difference of the original function and the linear function has a double root at $A$.  $x-A$ divides in evenly into $F$.  And, the resulting function has $A$ as a root.  

First, factor out $x-A$ from $F$.


\[
x\sqrt{9-x} - A\sqrt{9-A} = \frac{x\sqrt{9-x} - A\sqrt{9-A}}{1} \cdot \frac{x\sqrt{9-x} + A\sqrt{9-A}}{x\sqrt{9-x} + A\sqrt{9-A}}
\]

\[
= \frac{x^2 (9-x) - A^2 (9-A)}{x\sqrt{9-x} + A\sqrt{9-A}} = \frac{9x^2 - x^3 - 9A^2 + A^3}{x\sqrt{9-x} + A\sqrt{9-A}}
\]


\[
= \frac{9(x^2-A^2) - (x^3 - A^3)}{x\sqrt{9-x} + A\sqrt{9-A}} = \frac{(x-A)(9(x+A)-(x^2 + xA + A^2))}{x\sqrt{9-x} + A\sqrt{9-A}}
\]


$x-A$ divides into this evenly leaving

\[
\frac{9(x+A)-(x^2 + xA + A^2)}{x\sqrt{9-x} + A\sqrt{9-A}}
\]

$A$ is a root of this, which means $A$ is a root of the numerator.


\[
9(A+A)-(A^2 + A^2 + A^2) = 0
\]

\[
18A - 3A^2 = 0
\]

\[
3A(6-A) = 0
\]


The zero we are looking for is $6$.


\end{procedure}


\begin{itemize}
\item $K$ has a local and global maximum of $6\sqrt{3}$ at $6$.
\item $K$ has a local minimum of $0$ at $9$.
\item $K$ has no global minimum.
\end{itemize}




\begin{itemize}
\item $K$ is increasing on $(-\infty, 6]$.
\item $K$ is decreasing on $[6, 9]$.
\end{itemize}


Finally, the range is $(-\infty, 6\sqrt{3}]$. \\





\end{document}
