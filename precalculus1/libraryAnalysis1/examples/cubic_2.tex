\documentclass{ximera}

%\usepackage{todonotes}

\newcommand{\todo}{}

\usepackage{esint} % for \oiint
\ifxake%%https://math.meta.stackexchange.com/questions/9973/how-do-you-render-a-closed-surface-double-integral
\renewcommand{\oiint}{{\large\bigcirc}\kern-1.56em\iint}
\fi


\graphicspath{
  {./}
  {ximeraTutorial/}
  {basicPhilosophy/}
  {functionsOfSeveralVariables/}
  {normalVectors/}
  {lagrangeMultipliers/}
  {vectorFields/}
  {greensTheorem/}
  {shapeOfThingsToCome/}
  {dotProducts/}
  {partialDerivativesAndTheGradientVector/}
  {../productAndQuotientRules/exercises/}
  {../normalVectors/exercisesParametricPlots/}
  {../continuityOfFunctionsOfSeveralVariables/exercises/}
  {../partialDerivativesAndTheGradientVector/exercises/}
  {../directionalDerivativeAndChainRule/exercises/}
  {../commonCoordinates/exercisesCylindricalCoordinates/}
  {../commonCoordinates/exercisesSphericalCoordinates/}
  {../greensTheorem/exercisesCurlAndLineIntegrals/}
  {../greensTheorem/exercisesDivergenceAndLineIntegrals/}
  {../shapeOfThingsToCome/exercisesDivergenceTheorem/}
  {../greensTheorem/}
  {../shapeOfThingsToCome/}
  {../separableDifferentialEquations/exercises/}
  {vectorFields/}
}

\newcommand{\mooculus}{\textsf{\textbf{MOOC}\textnormal{\textsf{ULUS}}}}

\usepackage{tkz-euclide}
\usepackage{tikz}
\usepackage{tikz-cd}
\usetikzlibrary{arrows}
\tikzset{>=stealth,commutative diagrams/.cd,
  arrow style=tikz,diagrams={>=stealth}} %% cool arrow head
\tikzset{shorten <>/.style={ shorten >=#1, shorten <=#1 } } %% allows shorter vectors

\usetikzlibrary{backgrounds} %% for boxes around graphs
\usetikzlibrary{shapes,positioning}  %% Clouds and stars
\usetikzlibrary{matrix} %% for matrix
\usepgfplotslibrary{polar} %% for polar plots
\usepgfplotslibrary{fillbetween} %% to shade area between curves in TikZ
%\usetkzobj{all}
\usepackage[makeroom]{cancel} %% for strike outs
%\usepackage{mathtools} %% for pretty underbrace % Breaks Ximera
%\usepackage{multicol}
\usepackage{pgffor} %% required for integral for loops



%% http://tex.stackexchange.com/questions/66490/drawing-a-tikz-arc-specifying-the-center
%% Draws beach ball
\tikzset{pics/carc/.style args={#1:#2:#3}{code={\draw[pic actions] (#1:#3) arc(#1:#2:#3);}}}



\usepackage{array}
\setlength{\extrarowheight}{+.1cm}
\newdimen\digitwidth
\settowidth\digitwidth{9}
\def\divrule#1#2{
\noalign{\moveright#1\digitwidth
\vbox{\hrule width#2\digitwidth}}}




% \newcommand{\RR}{\mathbb R}
% \newcommand{\R}{\mathbb R}
% \newcommand{\N}{\mathbb N}
% \newcommand{\Z}{\mathbb Z}

\newcommand{\sagemath}{\textsf{SageMath}}


%\renewcommand{\d}{\,d\!}
%\renewcommand{\d}{\mathop{}\!d}
%\newcommand{\dd}[2][]{\frac{\d #1}{\d #2}}
%\newcommand{\pp}[2][]{\frac{\partial #1}{\partial #2}}
% \renewcommand{\l}{\ell}
%\newcommand{\ddx}{\frac{d}{\d x}}

% \newcommand{\zeroOverZero}{\ensuremath{\boldsymbol{\tfrac{0}{0}}}}
%\newcommand{\inftyOverInfty}{\ensuremath{\boldsymbol{\tfrac{\infty}{\infty}}}}
%\newcommand{\zeroOverInfty}{\ensuremath{\boldsymbol{\tfrac{0}{\infty}}}}
%\newcommand{\zeroTimesInfty}{\ensuremath{\small\boldsymbol{0\cdot \infty}}}
%\newcommand{\inftyMinusInfty}{\ensuremath{\small\boldsymbol{\infty - \infty}}}
%\newcommand{\oneToInfty}{\ensuremath{\boldsymbol{1^\infty}}}
%\newcommand{\zeroToZero}{\ensuremath{\boldsymbol{0^0}}}
%\newcommand{\inftyToZero}{\ensuremath{\boldsymbol{\infty^0}}}



% \newcommand{\numOverZero}{\ensuremath{\boldsymbol{\tfrac{\#}{0}}}}
% \newcommand{\dfn}{\textbf}
% \newcommand{\unit}{\,\mathrm}
% \newcommand{\unit}{\mathop{}\!\mathrm}
% \newcommand{\eval}[1]{\bigg[ #1 \bigg]}
% \newcommand{\seq}[1]{\left( #1 \right)}
% \renewcommand{\epsilon}{\varepsilon}
% \renewcommand{\phi}{\varphi}


% \renewcommand{\iff}{\Leftrightarrow}

% \DeclareMathOperator{\arccot}{arccot}
% \DeclareMathOperator{\arcsec}{arcsec}
% \DeclareMathOperator{\arccsc}{arccsc}
% \DeclareMathOperator{\si}{Si}
% \DeclareMathOperator{\scal}{scal}
% \DeclareMathOperator{\sign}{sign}


%% \newcommand{\tightoverset}[2]{% for arrow vec
%%   \mathop{#2}\limits^{\vbox to -.5ex{\kern-0.75ex\hbox{$#1$}\vss}}}
% \newcommand{\arrowvec}[1]{{\overset{\rightharpoonup}{#1}}}
% \renewcommand{\vec}[1]{\arrowvec{\mathbf{#1}}}
% \renewcommand{\vec}[1]{{\overset{\boldsymbol{\rightharpoonup}}{\mathbf{#1}}}}

% \newcommand{\point}[1]{\left(#1\right)} %this allows \vector{ to be changed to \vector{ with a quick find and replace
% \newcommand{\pt}[1]{\mathbf{#1}} %this allows \vec{ to be changed to \vec{ with a quick find and replace
% \newcommand{\Lim}[2]{\lim_{\point{#1} \to \point{#2}}} %Bart, I changed this to point since I want to use it.  It runs through both of the exercise and exerciseE files in limits section, which is why it was in each document to start with.

% \DeclareMathOperator{\proj}{\mathbf{proj}}
% \newcommand{\veci}{{\boldsymbol{\hat{\imath}}}}
% \newcommand{\vecj}{{\boldsymbol{\hat{\jmath}}}}
% \newcommand{\veck}{{\boldsymbol{\hat{k}}}}
% \newcommand{\vecl}{\vec{\boldsymbol{\l}}}
% \newcommand{\uvec}[1]{\mathbf{\hat{#1}}}
% \newcommand{\utan}{\mathbf{\hat{t}}}
% \newcommand{\unormal}{\mathbf{\hat{n}}}
% \newcommand{\ubinormal}{\mathbf{\hat{b}}}

% \newcommand{\dotp}{\bullet}
% \newcommand{\cross}{\boldsymbol\times}
% \newcommand{\grad}{\boldsymbol\nabla}
% \newcommand{\divergence}{\grad\dotp}
% \newcommand{\curl}{\grad\cross}
%\DeclareMathOperator{\divergence}{divergence}
%\DeclareMathOperator{\curl}[1]{\grad\cross #1}
% \newcommand{\lto}{\mathop{\longrightarrow\,}\limits}

% \renewcommand{\bar}{\overline}

\colorlet{textColor}{black}
\colorlet{background}{white}
\colorlet{penColor}{blue!50!black} % Color of a curve in a plot
\colorlet{penColor2}{red!50!black}% Color of a curve in a plot
\colorlet{penColor3}{red!50!blue} % Color of a curve in a plot
\colorlet{penColor4}{green!50!black} % Color of a curve in a plot
\colorlet{penColor5}{orange!80!black} % Color of a curve in a plot
\colorlet{penColor6}{yellow!70!black} % Color of a curve in a plot
\colorlet{fill1}{penColor!20} % Color of fill in a plot
\colorlet{fill2}{penColor2!20} % Color of fill in a plot
\colorlet{fillp}{fill1} % Color of positive area
\colorlet{filln}{penColor2!20} % Color of negative area
\colorlet{fill3}{penColor3!20} % Fill
\colorlet{fill4}{penColor4!20} % Fill
\colorlet{fill5}{penColor5!20} % Fill
\colorlet{gridColor}{gray!50} % Color of grid in a plot

\newcommand{\surfaceColor}{violet}
\newcommand{\surfaceColorTwo}{redyellow}
\newcommand{\sliceColor}{greenyellow}




\pgfmathdeclarefunction{gauss}{2}{% gives gaussian
  \pgfmathparse{1/(#2*sqrt(2*pi))*exp(-((x-#1)^2)/(2*#2^2))}%
}


%%%%%%%%%%%%%
%% Vectors
%%%%%%%%%%%%%

%% Simple horiz vectors
\renewcommand{\vector}[1]{\left\langle #1\right\rangle}


%% %% Complex Horiz Vectors with angle brackets
%% \makeatletter
%% \renewcommand{\vector}[2][ , ]{\left\langle%
%%   \def\nextitem{\def\nextitem{#1}}%
%%   \@for \el:=#2\do{\nextitem\el}\right\rangle%
%% }
%% \makeatother

%% %% Vertical Vectors
%% \def\vector#1{\begin{bmatrix}\vecListA#1,,\end{bmatrix}}
%% \def\vecListA#1,{\if,#1,\else #1\cr \expandafter \vecListA \fi}

%%%%%%%%%%%%%
%% End of vectors
%%%%%%%%%%%%%

%\newcommand{\fullwidth}{}
%\newcommand{\normalwidth}{}



%% makes a snazzy t-chart for evaluating functions
%\newenvironment{tchart}{\rowcolors{2}{}{background!90!textColor}\array}{\endarray}

%%This is to help with formatting on future title pages.
\newenvironment{sectionOutcomes}{}{}



%% Flowchart stuff
%\tikzstyle{startstop} = [rectangle, rounded corners, minimum width=3cm, minimum height=1cm,text centered, draw=black]
%\tikzstyle{question} = [rectangle, minimum width=3cm, minimum height=1cm, text centered, draw=black]
%\tikzstyle{decision} = [trapezium, trapezium left angle=70, trapezium right angle=110, minimum width=3cm, minimum height=1cm, text centered, draw=black]
%\tikzstyle{question} = [rectangle, rounded corners, minimum width=3cm, minimum height=1cm,text centered, draw=black]
%\tikzstyle{process} = [rectangle, minimum width=3cm, minimum height=1cm, text centered, draw=black]
%\tikzstyle{decision} = [trapezium, trapezium left angle=70, trapezium right angle=110, minimum width=3cm, minimum height=1cm, text centered, draw=black]



\author{Lee Wayand}

\begin{document}
\begin{exercise}



The best way to analyze a polynomial is with its factored form. \\


\textbf{Analyze B(t)} \\

\[
B(t) = 2 t^2 - 3 t^2 - 13 t - 28 \, \text { with its natural domain } 
\]






\subsection*{Algebraic Language}



\textbf{\textcolor{blue!55!black}{$\blacktriangleright$ Domain: }} We are given that the domain is the natural domain and the natural domain of all polynomials is $(-\infty, \infty)$.


\textbf{\textcolor{blue!55!black}{$\blacktriangleright$ Continuity: }}  All polynomials are continuous on their domain.  So, $Q$ has no discontinuities.  Since, the domain is all real numbers, there can be no singularities.



\textbf{\textcolor{blue!55!black}{$\blacktriangleright$ Zeros: }}  


There are several approaches to factoring.  Identifying zeros or roots is one.  For this we need a graph.




\begin{center}
\desmos{zvp1r3rcfr}{400}{300}
\end{center}



The DESMOS graph is suggesting that perhaps $4$ is the only root.  Let's check:



\[
B(4) = 2 (4)^2 - 3 (4)^2 - 13 (4) - 28 = 0
\]




Now, we know that $(t-4)$ is a factor.  




\[
B(t) = 2 t^2 - 3 t^2 - 13 t - 28 = (t-4)(A t^2 + B t + C)
\]




\[
B(t) = 2 t^2 - 3 t^2 - 13 t - 28 = A t^3 + (B - 4 A) t^2 + (C - 4 B) t - 4 C 
\]


Comparing the constant terms tells us that $C = 7$ \\



Comparing leading terms tells us that $A = 2$.



\[
B(t) = 2 t^2 - 3 t^2 - 13 t - 28 = 2 t^3 + (B - 8) t^2 + (7 - 4 B) t - 28 
\]


$B = 5$



\[
B(t) = 2 t^2 - 3 t^2 - 13 t - 28 = (t-4)(2 t^2 + 5 t + 7)
\]




The quadratic formula will give us the roots of $2 t^2 + 5 t + 7$, which we thnk should both be nonreal, since there was only one intercept.


\[
\frac{-5 \pm \sqrt{25-4 (2) (7)}}{4} = \frac{-5 \pm \sqrt{-31}}{4}
\]







The only root of $B(t)$ is $4$.  It has a multiplicity of $1$, which is odd.  Therefore, $B$ changes signs across $4$.










\textbf{\textcolor{blue!55!black}{$\blacktriangleright$ End-Behavior: }} Polynomials with odd degree (like cubics) have the different end-behavior on either side.  Since $B$ has a positive leading coefficient, $B$ is unbounded negatively as $t$ tends to $-\infty$. and $B$ is unbounded positively as $t$ tends to $\infty$.

\[
\lim\limits_{t \to -\infty} B(t) = -\infty
\]


\[
\lim\limits_{t \to \infty} B(t) = \infty
\]


There is no global maximum or global minimum. \\



\textbf{\textcolor{blue!55!black}{$\blacktriangleright$ Behavior: }}  $B$ is a cubic, which means it has three possible types. Since, it has only one real zero, we cannot tell algebraically if there are any local extrema (hills and valleys in the graph).




We'll need Calculus to know how to get a derivative for cubic polynomials.  So, we cannot get the criticval numbers for $B$.  That means we cannot get exact information about the local maximum or minimum.  However, we can establish some information.







At this point, we can only turn to the graph for approximations.


\textbf{\textcolor{blue!55!black}{$\blacktriangleright$ Extrema: }}  

$B$ has a local maximum of approximately $-19.973$ at approximately $-1.055$.  $-1.055$ is a critical number. \\





$B$ has a local minimum of approximately $-50.027$ at approximately $2.055$.  $2.055$ is a critical number. \\









\textbf{\textcolor{blue!55!black}{$\blacktriangleright$ Range: }}


$B$ is continuous and  $\lim\limits_{t \to -\infty} B(t) = -\infty$  and $\lim\limits_{t \to \infty} B(t) = \infty$.  This tells us that the range is $(-\infty, \infty)$.








\subsection*{Graphical Language}







\textbf{\textcolor{blue!55!black}{$\blacktriangleright$ }}  The graph of $y = B(t)$ is that of a cubic with one hill and one valley. It has no holes or breaks. \\

\textbf{\textcolor{blue!55!black}{$\blacktriangleright$ }}  The graph has one intercept: $(4, 0)$.









\textbf{\textcolor{blue!55!black}{$\blacktriangleright$ }}  The graph slopes up to the right until it hits the top of a hill at approximately the point, $(-1.055, -19.973)$, then it slopes down to the right until it hits the bottom of a valley at approximately the point, $(2.055, -50.-027)$, then is slopes up tot he right.



The graph has no highest or lowest points.




From this graphical information we can approximate that $B$ increases on $(-\infty, -1.055)$, decreases on $(-1.055, 2.055)$, and increases on $(2.055, \infty)$.



\end{exercise}
\end{document}