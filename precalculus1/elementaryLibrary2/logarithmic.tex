\documentclass{ximera}

%\usepackage{todonotes}

\newcommand{\todo}{}

\usepackage{esint} % for \oiint
\ifxake%%https://math.meta.stackexchange.com/questions/9973/how-do-you-render-a-closed-surface-double-integral
\renewcommand{\oiint}{{\large\bigcirc}\kern-1.56em\iint}
\fi


\graphicspath{
  {./}
  {ximeraTutorial/}
  {basicPhilosophy/}
  {functionsOfSeveralVariables/}
  {normalVectors/}
  {lagrangeMultipliers/}
  {vectorFields/}
  {greensTheorem/}
  {shapeOfThingsToCome/}
  {dotProducts/}
  {partialDerivativesAndTheGradientVector/}
  {../productAndQuotientRules/exercises/}
  {../normalVectors/exercisesParametricPlots/}
  {../continuityOfFunctionsOfSeveralVariables/exercises/}
  {../partialDerivativesAndTheGradientVector/exercises/}
  {../directionalDerivativeAndChainRule/exercises/}
  {../commonCoordinates/exercisesCylindricalCoordinates/}
  {../commonCoordinates/exercisesSphericalCoordinates/}
  {../greensTheorem/exercisesCurlAndLineIntegrals/}
  {../greensTheorem/exercisesDivergenceAndLineIntegrals/}
  {../shapeOfThingsToCome/exercisesDivergenceTheorem/}
  {../greensTheorem/}
  {../shapeOfThingsToCome/}
  {../separableDifferentialEquations/exercises/}
  {vectorFields/}
}

\newcommand{\mooculus}{\textsf{\textbf{MOOC}\textnormal{\textsf{ULUS}}}}

\usepackage{tkz-euclide}
\usepackage{tikz}
\usepackage{tikz-cd}
\usetikzlibrary{arrows}
\tikzset{>=stealth,commutative diagrams/.cd,
  arrow style=tikz,diagrams={>=stealth}} %% cool arrow head
\tikzset{shorten <>/.style={ shorten >=#1, shorten <=#1 } } %% allows shorter vectors

\usetikzlibrary{backgrounds} %% for boxes around graphs
\usetikzlibrary{shapes,positioning}  %% Clouds and stars
\usetikzlibrary{matrix} %% for matrix
\usepgfplotslibrary{polar} %% for polar plots
\usepgfplotslibrary{fillbetween} %% to shade area between curves in TikZ
%\usetkzobj{all}
\usepackage[makeroom]{cancel} %% for strike outs
%\usepackage{mathtools} %% for pretty underbrace % Breaks Ximera
%\usepackage{multicol}
\usepackage{pgffor} %% required for integral for loops



%% http://tex.stackexchange.com/questions/66490/drawing-a-tikz-arc-specifying-the-center
%% Draws beach ball
\tikzset{pics/carc/.style args={#1:#2:#3}{code={\draw[pic actions] (#1:#3) arc(#1:#2:#3);}}}



\usepackage{array}
\setlength{\extrarowheight}{+.1cm}
\newdimen\digitwidth
\settowidth\digitwidth{9}
\def\divrule#1#2{
\noalign{\moveright#1\digitwidth
\vbox{\hrule width#2\digitwidth}}}




% \newcommand{\RR}{\mathbb R}
% \newcommand{\R}{\mathbb R}
% \newcommand{\N}{\mathbb N}
% \newcommand{\Z}{\mathbb Z}

\newcommand{\sagemath}{\textsf{SageMath}}


%\renewcommand{\d}{\,d\!}
%\renewcommand{\d}{\mathop{}\!d}
%\newcommand{\dd}[2][]{\frac{\d #1}{\d #2}}
%\newcommand{\pp}[2][]{\frac{\partial #1}{\partial #2}}
% \renewcommand{\l}{\ell}
%\newcommand{\ddx}{\frac{d}{\d x}}

% \newcommand{\zeroOverZero}{\ensuremath{\boldsymbol{\tfrac{0}{0}}}}
%\newcommand{\inftyOverInfty}{\ensuremath{\boldsymbol{\tfrac{\infty}{\infty}}}}
%\newcommand{\zeroOverInfty}{\ensuremath{\boldsymbol{\tfrac{0}{\infty}}}}
%\newcommand{\zeroTimesInfty}{\ensuremath{\small\boldsymbol{0\cdot \infty}}}
%\newcommand{\inftyMinusInfty}{\ensuremath{\small\boldsymbol{\infty - \infty}}}
%\newcommand{\oneToInfty}{\ensuremath{\boldsymbol{1^\infty}}}
%\newcommand{\zeroToZero}{\ensuremath{\boldsymbol{0^0}}}
%\newcommand{\inftyToZero}{\ensuremath{\boldsymbol{\infty^0}}}



% \newcommand{\numOverZero}{\ensuremath{\boldsymbol{\tfrac{\#}{0}}}}
% \newcommand{\dfn}{\textbf}
% \newcommand{\unit}{\,\mathrm}
% \newcommand{\unit}{\mathop{}\!\mathrm}
% \newcommand{\eval}[1]{\bigg[ #1 \bigg]}
% \newcommand{\seq}[1]{\left( #1 \right)}
% \renewcommand{\epsilon}{\varepsilon}
% \renewcommand{\phi}{\varphi}


% \renewcommand{\iff}{\Leftrightarrow}

% \DeclareMathOperator{\arccot}{arccot}
% \DeclareMathOperator{\arcsec}{arcsec}
% \DeclareMathOperator{\arccsc}{arccsc}
% \DeclareMathOperator{\si}{Si}
% \DeclareMathOperator{\scal}{scal}
% \DeclareMathOperator{\sign}{sign}


%% \newcommand{\tightoverset}[2]{% for arrow vec
%%   \mathop{#2}\limits^{\vbox to -.5ex{\kern-0.75ex\hbox{$#1$}\vss}}}
% \newcommand{\arrowvec}[1]{{\overset{\rightharpoonup}{#1}}}
% \renewcommand{\vec}[1]{\arrowvec{\mathbf{#1}}}
% \renewcommand{\vec}[1]{{\overset{\boldsymbol{\rightharpoonup}}{\mathbf{#1}}}}

% \newcommand{\point}[1]{\left(#1\right)} %this allows \vector{ to be changed to \vector{ with a quick find and replace
% \newcommand{\pt}[1]{\mathbf{#1}} %this allows \vec{ to be changed to \vec{ with a quick find and replace
% \newcommand{\Lim}[2]{\lim_{\point{#1} \to \point{#2}}} %Bart, I changed this to point since I want to use it.  It runs through both of the exercise and exerciseE files in limits section, which is why it was in each document to start with.

% \DeclareMathOperator{\proj}{\mathbf{proj}}
% \newcommand{\veci}{{\boldsymbol{\hat{\imath}}}}
% \newcommand{\vecj}{{\boldsymbol{\hat{\jmath}}}}
% \newcommand{\veck}{{\boldsymbol{\hat{k}}}}
% \newcommand{\vecl}{\vec{\boldsymbol{\l}}}
% \newcommand{\uvec}[1]{\mathbf{\hat{#1}}}
% \newcommand{\utan}{\mathbf{\hat{t}}}
% \newcommand{\unormal}{\mathbf{\hat{n}}}
% \newcommand{\ubinormal}{\mathbf{\hat{b}}}

% \newcommand{\dotp}{\bullet}
% \newcommand{\cross}{\boldsymbol\times}
% \newcommand{\grad}{\boldsymbol\nabla}
% \newcommand{\divergence}{\grad\dotp}
% \newcommand{\curl}{\grad\cross}
%\DeclareMathOperator{\divergence}{divergence}
%\DeclareMathOperator{\curl}[1]{\grad\cross #1}
% \newcommand{\lto}{\mathop{\longrightarrow\,}\limits}

% \renewcommand{\bar}{\overline}

\colorlet{textColor}{black}
\colorlet{background}{white}
\colorlet{penColor}{blue!50!black} % Color of a curve in a plot
\colorlet{penColor2}{red!50!black}% Color of a curve in a plot
\colorlet{penColor3}{red!50!blue} % Color of a curve in a plot
\colorlet{penColor4}{green!50!black} % Color of a curve in a plot
\colorlet{penColor5}{orange!80!black} % Color of a curve in a plot
\colorlet{penColor6}{yellow!70!black} % Color of a curve in a plot
\colorlet{fill1}{penColor!20} % Color of fill in a plot
\colorlet{fill2}{penColor2!20} % Color of fill in a plot
\colorlet{fillp}{fill1} % Color of positive area
\colorlet{filln}{penColor2!20} % Color of negative area
\colorlet{fill3}{penColor3!20} % Fill
\colorlet{fill4}{penColor4!20} % Fill
\colorlet{fill5}{penColor5!20} % Fill
\colorlet{gridColor}{gray!50} % Color of grid in a plot

\newcommand{\surfaceColor}{violet}
\newcommand{\surfaceColorTwo}{redyellow}
\newcommand{\sliceColor}{greenyellow}




\pgfmathdeclarefunction{gauss}{2}{% gives gaussian
  \pgfmathparse{1/(#2*sqrt(2*pi))*exp(-((x-#1)^2)/(2*#2^2))}%
}


%%%%%%%%%%%%%
%% Vectors
%%%%%%%%%%%%%

%% Simple horiz vectors
\renewcommand{\vector}[1]{\left\langle #1\right\rangle}


%% %% Complex Horiz Vectors with angle brackets
%% \makeatletter
%% \renewcommand{\vector}[2][ , ]{\left\langle%
%%   \def\nextitem{\def\nextitem{#1}}%
%%   \@for \el:=#2\do{\nextitem\el}\right\rangle%
%% }
%% \makeatother

%% %% Vertical Vectors
%% \def\vector#1{\begin{bmatrix}\vecListA#1,,\end{bmatrix}}
%% \def\vecListA#1,{\if,#1,\else #1\cr \expandafter \vecListA \fi}

%%%%%%%%%%%%%
%% End of vectors
%%%%%%%%%%%%%

%\newcommand{\fullwidth}{}
%\newcommand{\normalwidth}{}



%% makes a snazzy t-chart for evaluating functions
%\newenvironment{tchart}{\rowcolors{2}{}{background!90!textColor}\array}{\endarray}

%%This is to help with formatting on future title pages.
\newenvironment{sectionOutcomes}{}{}



%% Flowchart stuff
%\tikzstyle{startstop} = [rectangle, rounded corners, minimum width=3cm, minimum height=1cm,text centered, draw=black]
%\tikzstyle{question} = [rectangle, minimum width=3cm, minimum height=1cm, text centered, draw=black]
%\tikzstyle{decision} = [trapezium, trapezium left angle=70, trapezium right angle=110, minimum width=3cm, minimum height=1cm, text centered, draw=black]
%\tikzstyle{question} = [rectangle, rounded corners, minimum width=3cm, minimum height=1cm,text centered, draw=black]
%\tikzstyle{process} = [rectangle, minimum width=3cm, minimum height=1cm, text centered, draw=black]
%\tikzstyle{decision} = [trapezium, trapezium left angle=70, trapezium right angle=110, minimum width=3cm, minimum height=1cm, text centered, draw=black]


\title{Logarithmic}

\begin{document}

\begin{abstract}
reverse
\end{abstract}
\maketitle







\section*{Backwards}

What if we have a function value for an exponential function and we would like to know which domain numbers are associated with it?  In other words, we would like to solve


\[    g(t) = k \cdot a^t  =   g_0    \]


How would we solve for $t$?






\begin{example} If the function value ``works well'', then we could probably guess.

Let $T(f) = 4 \cdot 3^f$.  


Solve $T(f) = 36$

$4 \cdot 3^f = 36$

$3^f = 9$

$f = 2$

\end{example}









Most function values are not going to be so obvious. 


For example, solve $ 3^x = 17$.


We may not be able to quickly think up this number or even an approximation for it.  However, we can still talk about it.

\begin{center}
We are looking for the number that you raise $3$ to, to get $17$.
\end{center}


That is a specific number. We can see from the graph that there is only one such number and we could visually approximate it around $2.5$.


\begin{example}
The following are all descriptions that identify unique real numbers.

\begin{itemize}
\item The number that you raise $5$ to, to get $97$.
\item The number that you raise $\frac{3}{4}$ to, to get $6$.
\item The number that you raise $7$ to, to get $\frac{1}{2}$.
\item The number that you raise $101$ to, to get $34$.
\item The number that you raise $10$ to, to get $1,000$.
\end{itemize}

\end{example}



As with all mathematical phrases, we have shorthand notation for these desciptions.








\begin{definition} \textbf{\textcolor{green!50!black}{Logarithm Base A of B}}

Let $a$ and $b$ be positive real numbers.  The number  you raise $a$ to, to get $b$ is called the \textbf{logarithm base a of b}.

The symbol for the logarithm base a of b is $\log_a(b)$.


$\log_a(b)$ is  the number you raise $a$ to, to get $b$.

\[     a^{\log_a(b)} = b          \]




\end{definition}





\begin{example}
The following are all descriptions that identify unique real numbers.

\begin{itemize}
\item The number that you raise $5$ to, to get $97$ is $\log_5(97)$. \\
\item The number that you raise $\frac{3}{4}$ to, to get $6$ is $\log_{\tfrac{3}{4}}(6)$. \\
\item The number that you raise $7$ to, to get $\frac{1}{2}$ is $\log_7\left(\frac{1}{2}\right)$. \\
\item The number that you raise $101$ to, to get $34$ is $\log_{101}(34)$. \\
\item The number that you raise $10$ to, to get $1,000$ is $\log_{10}(1000)$. \\
\end{itemize}

\end{example}



\begin{example}
Evaluate the following expressions

\begin{itemize}
\item  $3^{\log_3{56}} = \answer{56}$
\item  $13^{\log_{13}{21}} = \answer{21}$
\item  $\pi^{\log_{\pi}{82}} = \answer{82}$
\item  $4^{\log_4{\sqrt{7}}} = \answer{\sqrt{7}}$
\item  $85^{\log_{85}{2}} = \answer{2}$

\end{itemize}

\end{example}












Logarithms are exponents.  They can be positive or negative.



$9^{-2} = \frac{1}{81}$, therefore  $\log_{9}\left(\frac{1}{81}\right) = -2$



We also know that raising a positve number to any exponent cannot produce a negative number or $0$.  Therefore, the number \textit{inside} the logarithm must be positive



It sounds like we have a new category of functions.




\begin{definition} \textbf{\textcolor{green!50!black}{Basic Logarithmic Functions}}

A \textbf{Basic Logarithmic Function} is a function that can be represented by formulas of the form

\[     L(x) =    \log_b(x)            \]

where $b > 0$.

The domain is positive real numbers and the range is all real numbers.

\end{definition}












\begin{example}

Here is the graph of $y = L(x) = \log_2(x)$.

\begin{image}
\begin{tikzpicture} 
  \begin{axis}[
            domain=-10:10, ymax=10, xmax=10, ymin=-10, xmin=-10,
            axis lines =center, xlabel=$x$, ylabel=$y$,
            ytick={-10,-8,-6,-4,-2,2,4,6,8,10},
            xtick={-10,-8,-6,-4,-2,2,4,6,8,10},
            ticklabel style={font=\scriptsize},
            every axis y label/.style={at=(current axis.above origin),anchor=south},
            every axis x label/.style={at=(current axis.right of origin),anchor=west},
            axis on top
          ]
          
          \addplot [line width=2, penColor, smooth,samples=200,domain=(0:9),<->] {ln(x)/ln(2)};
          \addplot [line width=1, gray, dashed,domain=(-9:9),<->] ({0},{x});

           

  \end{axis}
\end{tikzpicture}
\end{image}


The intercept is $(1,0)$, because $\log_2(1) = 0$, because $2^0 = 1$.

On the interval $(0,1)$, we are looking at $\log_2(x)$ for $0<x<1$.  Remember, $\log_2(x)$ is the number you raise $2$ to, to get $x$, but here $0<x<1$.  Therefore, $2$ needs a negative exponent or $\log_2(x) < 0$.  And, the smaller (closer to $0$) you want $x$, the bigger the negative exponent.





\end{example}



If we switch the base from something greater than $1$, to something less than $1$, then all of the exponents flip.  The graph flips.






\begin{example}

Here is the graph of $y = L(x) = \log_{\tfrac{1}{2}}(x)$.

\begin{image}
\begin{tikzpicture} 
  \begin{axis}[
            domain=-10:10, ymax=10, xmax=10, ymin=-10, xmin=-10,
            axis lines =center, xlabel=$x$, ylabel=$y$,
            ytick={-10,-8,-6,-4,-2,2,4,6,8,10},
            xtick={-10,-8,-6,-4,-2,2,4,6,8,10},
            ticklabel style={font=\scriptsize},
            every axis y label/.style={at=(current axis.above origin),anchor=south},
            every axis x label/.style={at=(current axis.right of origin),anchor=west},
            axis on top
          ]
          
          \addplot [line width=2, penColor, smooth,samples=200,domain=(0:9),<->] {ln(x)/ln(0.5)};
          \addplot [line width=1, gray, dashed,domain=(-9:9),<->] ({0},{x});

           

  \end{axis}
\end{tikzpicture}
\end{image}


The intercept is $(1,0)$, because $\log_{\tfrac{1}{2}}(1) = 0$, because $\left(\frac{1}{2}\right)^0 = 1$.

On the interval $(0,1)$, we are looking at $\log_{\tfrac{1}{2}}(x)$ for $0<x<1$.  Now we just need large positive exponents of $\frac{1}{2}$ to get small numbers.  On the other hand, to get large positive numbers we need to raise $\frac{1}{2}$ to negative powers.





\end{example}











\begin{definition} \textbf{\textcolor{green!50!black}{Logarithmic Functions}}

A \textbf{Logarithmic Function} is a function that can be represented by formulas of the form

\[     L(x) =    A \log_b(B x + C) +D            \]

where $A$, $B$, $C$, and $D$ are real numbers and $b > 0$.

The domain is all positive real numbers that make the inside positive.

\end{definition}








\begin{center}
\textbf{\textcolor{green!50!black}{ooooo=-=-=-=-=-=-=-=-=-=-=-=-=ooOoo=-=-=-=-=-=-=-=-=-=-=-=-=ooooo}} \\

more examples can be found by following this link\\ \link[More Examples of Elementary Functions]{https://ximera.osu.edu/csccmathematics/precalculus1/precalculus1/elementaryLibrary2/examples/exampleList}

\end{center}




\end{document}
