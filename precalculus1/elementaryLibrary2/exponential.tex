\documentclass{ximera}


\graphicspath{
  {./}
  {ximeraTutorial/}
  {basicPhilosophy/}
}

\newcommand{\mooculus}{\textsf{\textbf{MOOC}\textnormal{\textsf{ULUS}}}}


\usepackage{tkz-euclide}\usepackage{tikz}
\usepackage{tikz-cd}
\usetikzlibrary{arrows}
\tikzset{>=stealth,commutative diagrams/.cd,
  arrow style=tikz,diagrams={>=stealth}} %% cool arrow head
\tikzset{shorten <>/.style={ shorten >=#1, shorten <=#1 } } %% allows shorter vectors

\usetikzlibrary{backgrounds} %% for boxes around graphs
\usetikzlibrary{shapes,positioning}  %% Clouds and stars
\usetikzlibrary{matrix} %% for matrix
\usepgfplotslibrary{polar} %% for polar plots
\usepgfplotslibrary{fillbetween} %% to shade area between curves in TikZ
\usetkzobj{all}
\usepackage[makeroom]{cancel} %% for strike outs
%\usepackage{mathtools} %% for pretty underbrace % Breaks Ximera
%\usepackage{multicol}
\usepackage{pgffor} %% required for integral for loops



%% http://tex.stackexchange.com/questions/66490/drawing-a-tikz-arc-specifying-the-center
%% Draws beach ball
\tikzset{pics/carc/.style args={#1:#2:#3}{code={\draw[pic actions] (#1:#3) arc(#1:#2:#3);}}}



\usepackage{array}
\setlength{\extrarowheight}{+.1cm}
\newdimen\digitwidth
\settowidth\digitwidth{9}
\def\divrule#1#2{
\noalign{\moveright#1\digitwidth
\vbox{\hrule width#2\digitwidth}}}
























%%This is to help with formatting on future title pages.
\newenvironment{sectionOutcomes}{}{}


\title{Exponential}

\begin{document}

\begin{abstract}
percentage growth
\end{abstract}
\maketitle




The defining characteristic of linear functions is that the pairs experience a constant growth rate. If you calculate the rate of change between any two pairs, $(a, f(a))$ and $(b, f(b))$, in the function, you get the same value, $m$.


\[   \frac{f(b)-f(a)}{b-a} = m       \]

This led to the equation or formula for linear functions:  $f(x) = m(x-a) + f(a)$ \\


Said another way:  If you move a fixed amount anywhere in the domain, $\Delta x$, then the function always grows by a proportionally fixed amount, $f(x + \Delta x) - f(x) = m(\Delta x)$. This fixed growth rate is always the same multiple of the change in the domain. That multiple is the constant growth rate.


$\blacktriangleright$ Exponential functions are similar, but it is their \textbf{percentage} growth rate that is constant.   \\


\begin{example} Cancer Cells 

\link{https://www.ncbi.nlm.nih.gov/pmc/articles/PMC6695196/}

\textbf{Abstract:}. Most models of cancer cell population expansion assume exponential growth kinetics at low cell densities, with deviations to account for observed slowing of growth rate only at higher densities due to limited resources such as space and nutrients. [...] \\


\begin{image}
\includegraphics{pics/cancer_growth.png}
\end{image}


The number of cancer cells is given by $N(t) = N_0 \, e^{g \, t}$.

The graphs above show $N_0 = 3, 8, 16$

The first graph illustrate classical exponential growth.  \\

The second graph captures the ``per capita growth rate''.  This would be the growth in the number of cancer cells divided by the number of cancer cells. This gives the percentage growth.  The graph shows a constant function, because the percentage growth rate is a constant.

The third graph shows the logarithm of the first graph, which turns out to be a linear function.



\end{example}







The growth of an exponential function, $g(t)$ over the interval $[a, b]$ is $g(b)-g(a)$. To get a percentage, we compare this back to the starting value, $g(a)$: 

\[      \frac{g(b)-g(a)}{g(a)}    \]

And, then to get the percentage growth rate we average over the interval



\[      \frac{\frac{g(b)-g(a)}{g(a)}}{b-a}    \]



For an exponential function, this is constant


\[      \frac{\frac{g(b)-g(a)}{g(a)}}{b-a}  = r  \]

or

\[      \frac{g(b)-g(a)}{g(a)(b-a)}  = r  \]



Let's build such a function up from a given value of $g(0)$. 


\begin{procedure}
Moving from $0$ to $1$ gives

\[      \frac{\frac{g(1)-g(0)}{g(0)}}{1-0}  = r  \]


\[      g(1)-g(0) = r \, g(0)  \]


\[      g(1) = r \, g(0) + g(0)  \]

\[      g(1) =  g(0) (r + 1)  \]



Moving from $1$ to $2$ gives

\[      g(2) - g(1) =  g(1) r  \]

\[      g(2) =  g(1) (r + 1)  \]

\[      g(2) =  g(0) (r + 1) (r + 1)  \]

\[      g(2) =  g(0) (r + 1)^2  \]




Moving from $2$ to $3$ gives

\[      g(3) =  g(0) (r + 1)^3  \]

\end{procedure}


In general, formulas for exponential functions look like


\[      g(t) = g(0) \cdot a^t   \]

















\section*{Exponential Functions}


\begin{definition} \textbf{\textcolor{green!50!black}{Basic Exponential Functions}}

Basic exponential functions are those functions that exhibit a constant percentage rate of change.  Their formulas look like


\[      f(x) = A \cdot b^x   \]

where $A$ is a nonzero real number, and $b$ is a positive real number.


\end{definition}



Said another way:  If you move a fixed amount in the domain, $\Delta x$, then the change in an exponential function, $k \cdot a^{x + \Delta x} - k \cdot a^x = k \cdot a^x (a^{\Delta x} - 1)$, is the same fixed multiple of the \textbf{function value}.

\begin{itemize}
\item Linear growth is a fixed multiple of the change in domain value. It is independent of the function value.
\item Exponential growth (for a fixed change in domain value) is a fixed multiple of the function value. 
\end{itemize}





There are two types of exponential functions corresponding to $0<a<1$ or $1<a$.






\begin{itemize}
\item If $0<a<1$, then greater positive exponents make the function value smaller.   \\
In the other direction, a negative exponent essentially gives us the reciprocal of $a$, which would be greater than $1$ here.  Therefore, greater negative exponents result in bigger function values. \\

\item If $1<a$, then greater positive exponents make the function value bigger.   \\ 
In the other direction, a negative exponent essentially gives us the reciprocal of $a$, which would be less than $1$ here.  Therefore, greater negative exponents result in smaller postive function values.
\end{itemize}



The graphs of $y = Y(x) = 3 \cdot \left(\frac{1}{2}\right)^x$ and $z = W(t) = 3 \cdot 2^t$ are shown below.




\begin{image}
\begin{tikzpicture}
  \begin{axis}[name = leftgraph, 
            domain=-10:10, ymax=10, xmax=10, ymin=-10, xmin=-10,
            axis lines =center, xlabel=$x$, ylabel=$y$,
            every axis y label/.style={at=(current axis.above origin),anchor=south},
            every axis x label/.style={at=(current axis.right of origin),anchor=west},
            axis on top
          ]
          
          \addplot [line width=2, penColor, smooth, samples=200, domain=(-1.5:9),<->] {03*(0.5^x)};
   

  \end{axis}
  \begin{axis}[at={(leftgraph.outer east)},anchor=outer west, 
            domain=-10:10, ymax=10, xmax=10, ymin=-10, xmin=-10,
            axis lines =center, xlabel=$t$, ylabel=$z$,
            every axis y label/.style={at=(current axis.above origin),anchor=south},
            every axis x label/.style={at=(current axis.right of origin),anchor=west},
            axis on top
          ]
          
          \addplot [line width=2, penColor, smooth, samples=200, domain=(-9:2.1),<->] {2*(2^x)};


  \end{axis}



\end{tikzpicture}
\end{image}






There is no vertical asymptote.  The domain of exponential functions is all real numbers.  $y=0$ is a horizontal asymptote on both graphs. The sign of an exponential function is given by the coefficient.


Since these formulas are centered around the exponent, they follow the exponent rules:



\begin{itemize}
\item $a^n \cdot a^m = a^{n+m}$

\item $\frac{a^n}{a^m} = a^{n-m}$

\item $(a^n)^m = a^{n \cdot m}$

\item $a^n \cdot b^n = (a \cdot b)^n$

\item $\frac{a^n}{b^n} = \left(\frac{a}{b}\right)^n$


\end{itemize}




\begin{example}

Let $T(f) = 4 \cdot 3^f$.  Evaluate the following.

\begin{itemize}
\item $T(0) = \answer{4}$ 
\item $T(1) = \answer{12}$
\item $T(-1) = \answer{\frac{4}{3}}$
\end{itemize}
\end{example}
















\begin{definition} \textbf{\textcolor{green!50!black}{Exponential Functions}}

Exponential functions are those functions that can be represented by formulas of the form


\[      f(x) = A \cdot b^{B x + C}   \]

where $A$, $B$, and $C$ are real numbers, $A$ is a nonzero real number, and $b$ is a positive real number.


\end{definition}



This is the most general form of an exponential function.  It is equivalent to the basic form, meaning it can be transformed into the basic form using our exponent rules.




\[
f(x) = A \cdot b^{B x + C} 
\]

\[
f(x) = A \cdot b^{B x} \cdot  b^C 
\]

\[
f(x) = A \cdot (b^{B})^x \cdot  b^C 
\]


\[
f(x) = A \cdot b^C \cdot (b^{B})^x = a \cdot r^x
\]





$A \cdot b^C$ is the new leading coefficient. \\

$b^{B}$ is the new base.









\section*{Shifted Exponential Functions}


\begin{definition} \textbf{\textcolor{green!50!black}{Shifted Exponential Functions}}

Shifted exponential functions are exponential functions with a number added on.


\[      f(x) = A \cdot b^{B x + C} + D   \]

where $A$, $B$, $C$, and $D$ are real numbers, $A \ne 0$ and $B \ne 0$, and $b$ is a positive real number.


\end{definition}




The main different between shifted exponential and exponential functions is that while exponential functions do not have zeros, a shifted exponential function may have a zero. \\







\begin{example}

Here is the graph of $g(x) = 2^x - 3$.

\begin{image}
\begin{tikzpicture} 
  \begin{axis}[
            domain=-10:10, ymax=10, xmax=10, ymin=-10, xmin=-10,
            axis lines =center, xlabel=$x$, ylabel=$y$, 
            ytick={-10,-8,-6,-4,-2,2,4,6,8,10},
            xtick={-10,-8,-6,-4,-2,2,4,6,8,10},
            ticklabel style={font=\scriptsize},
            every axis y label/.style={at=(current axis.above origin),anchor=south},
            every axis x label/.style={at=(current axis.right of origin),anchor=west},
            axis on top
          ]
          
          \addplot [line width=2, penColor, smooth,samples=200,domain=(-9:3.2),<->] {2^x-3};
          \addplot [line width=1, gray, dashed,domain=(-9:9),<->] ({x},{-3});

           

  \end{axis}
\end{tikzpicture}
\end{image}


$\log_2(3)$ is the zero of $g(x)$.



\end{example}










\begin{center}
\textbf{\textcolor{green!50!black}{ooooo=-=-=-=-=-=-=-=-=-=-=-=-=ooOoo=-=-=-=-=-=-=-=-=-=-=-=-=ooooo}} \\

more examples can be found by following this link\\ \link[More Examples of Elementary Functions]{https://ximera.osu.edu/csccmathematics/precalculus1/precalculus1/elementaryLibrary2/examples/exampleList}

\end{center}




\end{document}
