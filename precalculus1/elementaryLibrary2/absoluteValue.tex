\documentclass{ximera}


\graphicspath{
  {./}
  {ximeraTutorial/}
  {basicPhilosophy/}
}

\newcommand{\mooculus}{\textsf{\textbf{MOOC}\textnormal{\textsf{ULUS}}}}


\usepackage{tkz-euclide}\usepackage{tikz}
\usepackage{tikz-cd}
\usetikzlibrary{arrows}
\tikzset{>=stealth,commutative diagrams/.cd,
  arrow style=tikz,diagrams={>=stealth}} %% cool arrow head
\tikzset{shorten <>/.style={ shorten >=#1, shorten <=#1 } } %% allows shorter vectors

\usetikzlibrary{backgrounds} %% for boxes around graphs
\usetikzlibrary{shapes,positioning}  %% Clouds and stars
\usetikzlibrary{matrix} %% for matrix
\usepgfplotslibrary{polar} %% for polar plots
\usepgfplotslibrary{fillbetween} %% to shade area between curves in TikZ
\usetkzobj{all}
\usepackage[makeroom]{cancel} %% for strike outs
%\usepackage{mathtools} %% for pretty underbrace % Breaks Ximera
%\usepackage{multicol}
\usepackage{pgffor} %% required for integral for loops



%% http://tex.stackexchange.com/questions/66490/drawing-a-tikz-arc-specifying-the-center
%% Draws beach ball
\tikzset{pics/carc/.style args={#1:#2:#3}{code={\draw[pic actions] (#1:#3) arc(#1:#2:#3);}}}



\usepackage{array}
\setlength{\extrarowheight}{+.1cm}
\newdimen\digitwidth
\settowidth\digitwidth{9}
\def\divrule#1#2{
\noalign{\moveright#1\digitwidth
\vbox{\hrule width#2\digitwidth}}}
























%%This is to help with formatting on future title pages.
\newenvironment{sectionOutcomes}{}{}


\title{Absolute Value}

\begin{document}

\begin{abstract}
distance
\end{abstract}
\maketitle



The absolute value function, $|x|$, gives the distance on the number line between a number, $x$, and $0$.  Since distance cannot be negative, it appears that the function just ``makes numbers positive''.


To do this, the absolute value function just returns nonnegative numbers unharmed, and makes negative numbers turn positive.  Algebraically, this is accomplished through a piecewise defined function.






\[
|x| = 
\begin{cases}
  -x &\text{if $x<0$,}\\
  x & \text{if $x\ge 0$}.
\end{cases}
\]


Technically speaking, $|-3| = -(-3)$ and then $-(-3) = 3$.  The absolute value function negates negative numbers.


\textbf{Note:} Just because you see a negaitve sign doesn't mean you have a negative number.



\begin{example}
\begin{itemize}
\item $|4| = 4$
\item $|0| = 0$
\item $|-\pi| = -(-\pi) = \pi$
\item $|\cos(\pi)| = |-1| = -(-1) = 1$
\item $|\sin(\tfrac{\pi}{4})| = \tfrac{1}{\sqrt{2}}$
\end{itemize}
\end{example}





\begin{example}
\begin{itemize}
\item $|-\sqrt{5}| = \answer{\sqrt{5}}$
\item $|4-4| = \answer{0}$
\item $\left|\frac{-4}{-3}\right| = \answer{\frac{4}{3}}$
\item $|\cos(\tfrac{\pi}{2})| = \answer{0}$
\item $|\sin(\tfrac{3\pi}{2})| = \answer{1}$
\item $|\tan(\tfrac{3\pi}{4})| = \answer{1}$
\end{itemize}
\end{example}
There is no arithmetic operation called ``make positive''.  If a number is negative, then you make it positive by negating it. 


Negating is arithmetic.
``Make positive'' is not arithmetic.




Graph of $y = A(t) = |t|$.

\begin{image}
\begin{tikzpicture}
  \begin{axis}[
            domain=-10:10, ymax=10, xmax=10, ymin=-10, xmin=-10,
            axis lines =center, xlabel=$t$, ylabel=$y$, grid = major,
            ytick={-10,-8,-6,-4,-2,2,4,6,8,10},
            xtick={-10,-8,-6,-4,-2,2,4,6,8,10},
            ticklabel style={font=\scriptsize},
            every axis y label/.style={at=(current axis.above origin),anchor=south},
            every axis x label/.style={at=(current axis.right of origin),anchor=west},
            axis on top
          ]
          

          \addplot [line width=2, penColor, smooth, domain=(-9:9), <->] {abs(x)};
   


           

  \end{axis}
\end{tikzpicture}
\end{image}







\begin{example}

Solve $|m| = 18$.


Either $m = 18$  or $m = -18$

\end{example}







The absolute value function is a continuous function.

\begin{itemize}
\item On $(-\infty, 0)$ we have $| x | = -x$, which is a linear function, which is continuous.
\item On $(0, \infty)$ we have $| x | = x$, which is a linear function, which is continuous.
\end{itemize}

So, the only questions is $0$ itself.  The graph certainly suggests continuity, but we want a rigorous explanation. \\


$| 0 | = 0$, so the only other possibility for $0$ is that it might be a discontinuity.  We need to show it is not a discontinuity.  We need to show that when $x$ is close to $0$, then also $| x |$ is close to $0$. \\

Select any small interval you want surrounding the function value $| 0 | = 0$.  Like, $(0 - \epsilon, 0 + \epsilon) = (-\epsilon, \epsilon)$. \\



Is it possible to find a small inteval around the domain number $0$, like, $(-\delta, \delta)$, so that



\[
| x | \in (-\epsilon, \epsilon) \, \text{ whenever } \, x \in (-\delta, \delta)
\]


Yes.  Just pick $\delta = \epsilon$.

So, $0$ is not a discontinuity and the absolute value function is a continuous function.










\begin{example}  Behavior


Where is $A(x) = | x |$ increasing and decreasing?



\begin{explanation}


First, get rid of the absolute value bars.

\begin{itemize}
\item $x < 0$ when $x < \answer{0}$
\item $x > 0$ when $x > \answer{0}$
\end{itemize}



\[
A(x) = 
\begin{cases}
  -x & \text{ on } (-\infty, 0)   \\
  x  & \text{ on } [0, \infty)
\end{cases}
\]



Each piece is a restricted linear function.  We can get their iRoC.


\[
iRoC_A(x) = 
\begin{cases}
  -1 & \text{ on } (-\infty, 0)   \\
  1  & \text{ on } (0, \infty).
\end{cases}
\]


$A(x) = | x |$ is decreasing on $(-\infty, 0)$ and increasing on $(0, \infty)$.



\end{explanation}


\end{example}




Absolute value is a continuous function, which is decreasing on $(-\infty, 0)$ and increasing on $(0, \infty)$. That makes $| 0 | = 0$ the global minimum. \\


Since the absolute value function is a increasing linear function on $(0, \infty)$, we know that

\[
\lim\limits_{x \to \intfy}| x | = \infty
\]



Therefore, the range is $[0, \infty)$.








\begin{example}  Critical Numbers


WWhat are the cricital numbers for $A(x) = | x |$ ?



\begin{explanation}


First, $0$ is in the domain. \\

The previous examples shows that the $iRoC_{|x|}$ is defined and nonzero everywhere except $0$. \\

The only possible critical number is $0$.  We will show that there is no tangent line at $(0,0)$. \\


Suppose there is a tangent line to the graph of $y = | x |$ at $(0,0)$.  Then, it has to match the slope on the left, which is $-1$ and it has to match the slope on the right, which is $1$.  A line cannot have two slopes. \\ 


There is not tangent line at $(0,0)$, which means there is no slope. \\



\[
iRoC_{|x|}(0) \, \text{ does not exist }
\]


$0$ is the only critical number.



\end{explanation}


\end{example}









\begin{example}  Behavior


Rewrite $A(x) = | x - 3|$ without absolute value bars.



\begin{explanation}


First, get rid of the absolute value bars.

\begin{itemize}
\item $x - 3 < 0$ when $x < \answer{3}$
\item $x - 3 > 0$ when $x > \answer{3}$
\end{itemize}



\[
A(x) = 
\begin{cases}
  -(x - 3) & \text{ on } (-\infty, 3)   \\
  x - 3  & \text{ on } [3, \infty).
\end{cases}
\]






\end{explanation}


\end{example}


















\begin{center}
\textbf{\textcolor{green!50!black}{ooooo=-=-=-=-=-=-=-=-=-=-=-=-=ooOoo=-=-=-=-=-=-=-=-=-=-=-=-=ooooo}} \\

more examples can be found by following this link\\ \link[More Examples of Elementary Functions]{https://ximera.osu.edu/csccmathematics/precalculus1/precalculus1/elementaryLibrary2/examples/exampleList}

\end{center}







\end{document}
