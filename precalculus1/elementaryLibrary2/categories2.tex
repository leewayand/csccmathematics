\documentclass{ximera}


\graphicspath{
  {./}
  {ximeraTutorial/}
  {basicPhilosophy/}
}

\newcommand{\mooculus}{\textsf{\textbf{MOOC}\textnormal{\textsf{ULUS}}}}


\usepackage{tkz-euclide}\usepackage{tikz}
\usepackage{tikz-cd}
\usetikzlibrary{arrows}
\tikzset{>=stealth,commutative diagrams/.cd,
  arrow style=tikz,diagrams={>=stealth}} %% cool arrow head
\tikzset{shorten <>/.style={ shorten >=#1, shorten <=#1 } } %% allows shorter vectors

\usetikzlibrary{backgrounds} %% for boxes around graphs
\usetikzlibrary{shapes,positioning}  %% Clouds and stars
\usetikzlibrary{matrix} %% for matrix
\usepgfplotslibrary{polar} %% for polar plots
\usepgfplotslibrary{fillbetween} %% to shade area between curves in TikZ
\usetkzobj{all}
\usepackage[makeroom]{cancel} %% for strike outs
%\usepackage{mathtools} %% for pretty underbrace % Breaks Ximera
%\usepackage{multicol}
\usepackage{pgffor} %% required for integral for loops



%% http://tex.stackexchange.com/questions/66490/drawing-a-tikz-arc-specifying-the-center
%% Draws beach ball
\tikzset{pics/carc/.style args={#1:#2:#3}{code={\draw[pic actions] (#1:#3) arc(#1:#2:#3);}}}



\usepackage{array}
\setlength{\extrarowheight}{+.1cm}
\newdimen\digitwidth
\settowidth\digitwidth{9}
\def\divrule#1#2{
\noalign{\moveright#1\digitwidth
\vbox{\hrule width#2\digitwidth}}}
























%%This is to help with formatting on future title pages.
\newenvironment{sectionOutcomes}{}{}


\title{Categories}

\begin{document}

\begin{abstract}
Templates
\end{abstract}
\maketitle





We are building a library of the elemntary functions.  The idea is to use the library to list characteristics, features, and aspects of all functions within each category.  \\

That way, if we can identify the type of function we have, then we get free information when analyzing functions. \\

The category becomes our reasoning. \\



\begin{center}

\textbf{\textcolor{red!70!black}{These are ``CAN'' questions.}} \\

\end{center}




\textbf{\textcolor{purple!85!blue}{CAN}} the formula we are given be rewritten as one of the official standard forms for each category? \\







\section*{Official Templates}




\begin{formula} \textbf{\textcolor{blue!55!black}{Power Functions}} 

A power function is any function that \textbf{\textcolor{purple!85!blue}{CAN}} be represented with a formula of the form

\[   f(x) = k \, x^p      \]

where $k$ and $p$ are real numbers.




\end{formula}












Polynomial functions are sums of power functions with powers that are nonnegative integers (whole numbers).


\begin{formula} \textbf{\textcolor{blue!55!black}{Polynomial Functions}} 

A polynomial function is any function that \textbf{\textcolor{purple!85!blue}{CAN}} be represented with a formula of the form

\[    a_n x^n + a_{n-1} x^{n-1} + \cdots + a_3 x^3 + a_2 x^2 + a_1 x^1 + a_0 x^0      \]

where the $a_k$ are real numbers and $a_n \ne 0$.


\end{formula}











\begin{formula} \textbf{\textcolor{blue!55!black}{Rational Functions}} 

A rational function is any function that \textbf{\textcolor{purple!85!blue}{CAN}} be represented with a formula of the form

\[   \frac{ a_n x^n + a_{n-1} x^{n-1} + \cdots + a_3 x^3 + a_2 x^2 + a_1 x + a_0  } { b_m x^m + b_{m-1} x^{m-1} + \cdots + b_3 x^3 + b_2 x^2 + b_1 x + b_0 }   \]



where the $a_k$ and $b_k$ are real numbers and $a_n \ne 0$ and $b_m \ne 0$.





\end{formula}

















\begin{formula} \textbf{\textcolor{blue!55!black}{Radical/Root Functions}} 

A radical or root function is any function that \textbf{\textcolor{purple!85!blue}{CAN}} be represented with a formula of the form  

\[   A \sqrt[n]{B \, x + C} + D =  A (B \, x + C)^{\tfrac{1}{n}} + D    \]

where the $A$, $B$, $C$, and $D$ are real numbers and $A \ne 0$ and $B \ne 0$.

\end{formula}














\begin{formula} \textbf{\textcolor{blue!55!black}{Exponential Functions}}

An exponential function is any function that \textbf{\textcolor{purple!85!blue}{CAN}} be represented with a formula of the form


\[      f(x) = A \cdot r^{B \, x + C}   \]

where $A$, $B$, and $C$ are real numbers, $A$ is a nonzero real number, and $r$ is a positive real number.


\end{formula}








\begin{formula} \textbf{\textcolor{blue!55!black}{Shifted Exponential Functions}}

A shifted exponential function is any function that \textbf{\textcolor{purple!85!blue}{CAN}} be represented with a formula of the form


\[      f(x) = A \cdot r^{B \, x + C} + D   \]

where $A$, $B$, $C$, and $D$ are real numbers, $A \ne 0$ and $B \ne 0$, and $r$ is a positive real number.


\end{formula}











\begin{formula} \textbf{\textcolor{blue!55!black}{Logarithmic Functions}}

A \textbf{Logarithmic Function} is any function that \textbf{\textcolor{purple!85!blue}{CAN}} be represented with a formula of the form

\[     L(x) =    A \log_r(B \, x + C) +D            \]

where $A$, $B$, $C$, and $D$ are real numbers and $r > 0$.

The domain is all positive real numbers that make the inside positive.

\end{formula}















\begin{formula} \textbf{\textcolor{blue!55!black}{Basic Sine Functions}}

A \textbf{Basic Sine Function} is any function that \textbf{\textcolor{purple!85!blue}{CAN}} be represented with a formula of the form

\[     S(x) =    A \sin(x)           \]

where $A$ is a real number.


\end{formula}











\begin{formula} \textbf{\textcolor{blue!55!black}{Sine Functions}}

A \textbf{Sine Function} is any function that \textbf{\textcolor{purple!85!blue}{CAN}} be represented with a formula of the form

\[     S(x) =    A \sin(B \, x + C) + D           \]

where $A$, $B$, $C$, and $D$ are real numbers.


\end{formula}
















\begin{formula} \textbf{\textcolor{blue!55!black}{Basic Cosine Functions}}

A \textbf{Basic Cosine Function} is any function that \textbf{\textcolor{purple!85!blue}{CAN}} be represented with a formula of the form

\[     C(x) =    A \cos(x)           \]

where $A$ is a real number.


\end{formula}











\begin{formula} \textbf{\textcolor{blue!55!black}{Cosine Functions}}

A \textbf{Cosine Function} is any function that \textbf{\textcolor{purple!85!blue}{CAN}} be represented with a formula of the form

\[     C(x) =    A \cos(B \, x + C) + D           \]

where $A$, $B$, $C$, and $D$ are real numbers.


\end{formula}

















\begin{formula} \textbf{\textcolor{blue!55!black}{Absolute Value Functions}}

An \textbf{Absolute Value Function} is any function that \textbf{\textcolor{purple!85!blue}{CAN}} be represented with a formula of the form

\[     g(x) =    A  | B \, x + C | + D           \]

where $A$, $B$, $C$, and $D$ are real numbers, with $A \ne 0$ and $B \ne 0$.


\end{formula}




















\begin{center}
\textbf{\textcolor{green!50!black}{ooooo-=-=-=-ooOoo-=-=-=-ooooo}} \\

more examples can be found by following this link\\ \link[More Examples of Elementary Functions]{https://ximera.osu.edu/csccmathematics/precalculus1/precalculus1/elementaryLibrary1/examples/exampleList}

\end{center}





\end{document}
