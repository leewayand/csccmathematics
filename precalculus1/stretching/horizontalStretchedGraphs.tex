\documentclass{ximera}


\graphicspath{
  {./}
  {ximeraTutorial/}
  {basicPhilosophy/}
}

\newcommand{\mooculus}{\textsf{\textbf{MOOC}\textnormal{\textsf{ULUS}}}}


\usepackage{tkz-euclide}\usepackage{tikz}
\usepackage{tikz-cd}
\usetikzlibrary{arrows}
\tikzset{>=stealth,commutative diagrams/.cd,
  arrow style=tikz,diagrams={>=stealth}} %% cool arrow head
\tikzset{shorten <>/.style={ shorten >=#1, shorten <=#1 } } %% allows shorter vectors

\usetikzlibrary{backgrounds} %% for boxes around graphs
\usetikzlibrary{shapes,positioning}  %% Clouds and stars
\usetikzlibrary{matrix} %% for matrix
\usepgfplotslibrary{polar} %% for polar plots
\usepgfplotslibrary{fillbetween} %% to shade area between curves in TikZ
\usetkzobj{all}
\usepackage[makeroom]{cancel} %% for strike outs
%\usepackage{mathtools} %% for pretty underbrace % Breaks Ximera
%\usepackage{multicol}
\usepackage{pgffor} %% required for integral for loops



%% http://tex.stackexchange.com/questions/66490/drawing-a-tikz-arc-specifying-the-center
%% Draws beach ball
\tikzset{pics/carc/.style args={#1:#2:#3}{code={\draw[pic actions] (#1:#3) arc(#1:#2:#3);}}}



\usepackage{array}
\setlength{\extrarowheight}{+.1cm}
\newdimen\digitwidth
\settowidth\digitwidth{9}
\def\divrule#1#2{
\noalign{\moveright#1\digitwidth
\vbox{\hrule width#2\digitwidth}}}
























%%This is to help with formatting on future title pages.
\newenvironment{sectionOutcomes}{}{}


\title{Left and Right}

\begin{document}

\begin{abstract}
stretch the domain
\end{abstract}
\maketitle





We have seen that addition (or subtraction) shifts the domain or range and hifts the graph as a rigid object. ALl of the movement is the same for every point.

Mulitplication behaves differently.




\section{Stretching Horizontally}







Consider the function $T$ defined as

\[
T(k) = 
\begin{cases}
  -2k-6 & \text{ if }  -8 \leq k < -2 \\
  -k+3 & \text{ if } 0 \leq k < 10
\end{cases}
\]






Graph of $y = T(k)$.





\begin{image}
\begin{tikzpicture}
  \begin{axis}[
            domain=-10:10, ymax=10, xmax=10, ymin=-10, xmin=-10,
            axis lines =center, xlabel=$k$, ylabel=$y$,
            every axis y label/.style={at=(current axis.above origin),anchor=south},
            every axis x label/.style={at=(current axis.right of origin),anchor=west},
            axis on top
          ]
          
  \addplot [draw=penColor,very thick,smooth,domain=(-8:-2)] {-2*x-6};
  \addplot [draw=penColor,very thick,smooth,domain=(0:10)] {-x+3};

  \addplot[color=penColor,only marks,mark=*] coordinates{(-8,10)}; 
  \addplot[color=penColor,fill=white,only marks,mark=*] coordinates{(-2,-2)}; 
  \addplot[color=penColor,only marks,mark=*] coordinates{(0,3)}; 
  \addplot[color=penColor,fill=white,only marks,mark=*] coordinates{(10,-7)}; 


    \end{axis}
\end{tikzpicture}
\end{image}




Now, Let's define a new function based on $T$.



Define $W$ as $W(g) = T(2g)$ with the induced domain.



$g$ represents numbers in the the domain of $W$ and $2g$ represents numbers in the domain of $T$.  

The domain of $T$ is $[-8,-2) \cup [0,10)$.


If $2g \in [-8,-2) \cup [0,10)$, then $g \in [-4,-1) \cup [0,5)$

It is like the domain of $T$ is on a rubberband and it shrunk in half. Wherever you are evaluatinng $W$, you get that value by evaluatinng $T$ at twice the $W$-domain number.



\begin{example}  Evaluating $W$

\begin{itemize}
\item $W(4) = T(8) = -8+3=-5$
\item $W(-2) = T(-4) = 2$
\item $W(1) = T(2) = 1$
\end{itemize}


\end{example}


Except for $0$. $W(0) = T(0)$.  The rubberband is pinned at $0$, because any multiple of $0$ is still $0$.  



It seems backwards.  We multiplied by $2$ and the resulting domain was a compressed version of the original domain.  That is because this is a backwards view of what happened.



We multiplied $g$ by $2$, but $g$ is from the domain $W$.  We want to know what happened to the domain of $T$ and $k$ represents the domain of $T$.

In our definition of $W$, we have $k=2g$, which gives us $g=\frac{k}{2}$.  $k$ is cut in half, which is what we see in the domain and in the graph.




\[
W(g) = 
\begin{cases}
  -2(2g)-6    & \text{ if }  -8 \leq 2g < -2 \\
  -(2g)+3   & \text{ if } 0 \leq 2g < 10
\end{cases}
\]





\[
W(g) = 
\begin{cases}
  -4g-6    & \text{ if }  -4 \leq g < -1 \\
  -2g + 3   & \text{ if } 0 \leq g < 5
\end{cases}
\]








Graph of $z = W(g)$.






\begin{image}
\begin{tikzpicture}
  \begin{axis}[
            domain=-10:10, ymax=10, xmax=10, ymin=-10, xmin=-10,
            axis lines =center, xlabel=$g$, ylabel=$z$,
            every axis y label/.style={at=(current axis.above origin),anchor=south},
            every axis x label/.style={at=(current axis.right of origin),anchor=west},
            axis on top
          ]
          
  \addplot [draw=penColor,very thick,smooth,domain=(-8:-1)] {-4*x-6};
  \addplot [draw=penColor,very thick,smooth,domain=(0:5)] {-2*x+3};

  \addplot[color=penColor,only marks,mark=*] coordinates{(-4,10)}; 
  \addplot[color=penColor,fill=white,only marks,mark=*] coordinates{(-1,-2)}; 
  \addplot[color=penColor,only marks,mark=*] coordinates{(0,3)}; 
  \addplot[color=penColor,fill=white,only marks,mark=*] coordinates{(5,-7)}; 


    \end{axis}
\end{tikzpicture}
\end{image}



$T$ and $W$ have the same maximums and minimums.  The height of the points didn't change.  The points to the left of the vertical axis moved to the right. The points to the right of the vertical axis moved to the left as the graph compressed horizontally.






\section{Negative Coefficients}


What if we mulitply by $-2$ rather than $2$?













Now, Let's define a new function based on $T$.



Define $M$ as $M(r) = T(-2r)$ with the induced domain.



$r$ represents numbers in the the domain of $M$ and now $-2r$ represents numbers in the domain of $T$.  

The domain of $T$ is $[-8,-2) \cup [0,10)$.


If $-2r \in [-8,-2) \cup [0,10)$, then $r \in (-5,0] \cup (1, 4]$

It is like the domain of $T$ is on a rubberband and it shrunk in half and then rotated around the vertical axis.  As $r$ walks through the domain of $M$ from the left, the corresponding walking in the domain of $T$ is from the right.



\begin{example}  Evaluating $W$

\begin{itemize}
\item $M(4) = T(-8) = 10$
\item $M(-2) = T(4) = -1$
\item $M(0) = T(0) = 3$
\end{itemize}


\end{example}


Except for $0$. $M(0) = T(0)$.  The rubberband is pinned at $0$, because any multiple of $0$ is still $0$.  While addition caused shifts and changed the zeros, multiplication does not change the zeros.









Since the mulitplication was on the inside of the formla, it only affects the domain.  Multiplication by $-1$, reflects the graph across the vertical axis - the domain changed sign. Multiplication by $2$ compresses the graph.



Graph of $m = M(r)$.






\begin{image}
\begin{tikzpicture}
  \begin{axis}[
            domain=-10:10, ymax=10, xmax=10, ymin=-10, xmin=-10,
            axis lines =center, xlabel=$r$, ylabel=$m$,
            every axis y label/.style={at=(current axis.above origin),anchor=south},
            every axis x label/.style={at=(current axis.right of origin),anchor=west},
            axis on top
          ]
          
  \addplot [draw=penColor,very thick,smooth,domain=(1:4)] {4*x-6};
  \addplot [draw=penColor,very thick,smooth,domain=(-5:0)] {2*x+3};

  \addplot[color=penColor,only marks,mark=*] coordinates{(4,10)}; 
  \addplot[color=penColor,fill=white,only marks,mark=*] coordinates{(1,-2)}; 
  \addplot[color=penColor,only marks,mark=*] coordinates{(0,3)}; 
  \addplot[color=penColor,fill=white,only marks,mark=*] coordinates{(-5,-7)}; 


    \end{axis}
\end{tikzpicture}
\end{image}


























\begin{example} Sine



Graph of $y = sin(2\theta)$.

\begin{image}
\begin{tikzpicture} 
  \begin{axis}[
            domain=-10:10, ymax=1.5, xmax=10, ymin=-1.5, xmin=-10,
            xtick={-6.28, -3.14, 3.14, 6.28}, 
            xticklabels={$-2\pi$, $-\pi$, $\pi$, $2\pi$},
            axis lines =center,  xlabel={$\theta$}, ylabel=$y$,
            every axis y label/.style={at=(current axis.above origin),anchor=south},
            every axis x label/.style={at=(current axis.right of origin),anchor=west},
            axis on top
          ]
          
          	\addplot [line width=2, penColor, smooth,samples=200,domain=(-9:9), <->] {sin(2*deg(x))};

           

  \end{axis}
\end{tikzpicture}
\end{image}


The graph is compressed horizontally by a factor of $2$.


Graph of $y = cos\left(\frac{1}{2}\theta\right)$.

\begin{image}
\begin{tikzpicture} 
  \begin{axis}[
            domain=-10:10, ymax=1.5, xmax=10, ymin=-1.5, xmin=-10,
            xtick={-6.28, -3.14, 3.14, 6.28}, 
            xticklabels={$-2\pi$, $-\pi$, $\pi$, $2\pi$},
            axis lines =center,  xlabel={$\theta$}, ylabel=$y$,
            every axis y label/.style={at=(current axis.above origin),anchor=south},
            every axis x label/.style={at=(current axis.right of origin),anchor=west},
            axis on top
          ]
          
          	\addplot [line width=2, penColor, smooth,samples=200,domain=(-9:9), <->] {cos(0.5*deg(x))};

           

  \end{axis}
\end{tikzpicture}
\end{image}



\end{example}



The graph has been stretched norizontally by a factor of $2$.

Multiplication of the domain by a constant doesn't change the shape of the graph.  It might squish it horizontally or stretch it horizontally, but the shape remains.  All of the measurements are relatively the same in each graph. The maximums and minimums are still in relatively the same places.  

You may have to view this in reverse, if the multiplication coefficient was negative, but all of the characteristics and features remain.








\begin{example} Absolute Value



Graph of $y = |3x|$.



\begin{image}
\begin{tikzpicture} 
  \begin{axis}[
            domain=-10:10, ymax=10, xmax=10, ymin=-10, xmin=-10,
            axis lines =center, xlabel=$r$, ylabel=$y$,
            every axis y label/.style={at=(current axis.above origin),anchor=south},
            every axis x label/.style={at=(current axis.right of origin),anchor=west},
            axis on top
          ]
          
          \addplot [line width=2, penColor, smooth, samples=200, domain=(-3:3),<->] {3*abs(x)};
        

  \end{axis}
\end{tikzpicture}
\end{image}



Multiplication by $3$ compresses the graph, because all domain numbers here correspond to $3$ times their values in the domain of $|x|$, therefore the graph is steeper.






\end{example}







\begin{example}

Here is the graph of $y = log_2(-t)$.

\begin{itemize}
\item vertical asymptote: $t+5=0$, when $t=-5$
\item horizontal intercept: $t+5=1$, when $t=-4$
\end{itemize}


\begin{image}
\begin{tikzpicture} 
  \begin{axis}[
            domain=-10:10, ymax=10, xmax=10, ymin=-10, xmin=-10,
            axis lines =center, xlabel=$x$, ylabel=$y$,
            every axis y label/.style={at=(current axis.above origin),anchor=south},
            every axis x label/.style={at=(current axis.right of origin),anchor=west},
            axis on top
          ]
          
          \addplot [line width=2, penColor, smooth,samples=200,domain=(-8.1:-0.07),<->] {ln(-x)/ln(2)};
          \addplot [line width=1, gray, dashed,domain=(-9:9),<->] ({0},{x});

          \addplot[color=penColor,only marks,mark=*] coordinates{(-1,0)}; 

           

  \end{axis}
\end{tikzpicture}
\end{image}


The graph looks the same as the basic logarithm graph, just reflected about the vertical axis.




\end{example}









\begin{example} Stretching Domains

Compared to the graph of the function $y=f(x)$, the graph of $z=g(r)=f(4r)$ looks to be 

\begin{multipleChoice}

\choice{shifted left by $4$}
\choice{shifted right by $4$}
\choice{stretched by a factor of $4$}
\choice[correct]{compressed by a factor of $4$}
\end{multipleChoice}


\end{example}
























\end{document}
