\documentclass{ximera}

%\usepackage{todonotes}

\newcommand{\todo}{}

\usepackage{esint} % for \oiint
\ifxake%%https://math.meta.stackexchange.com/questions/9973/how-do-you-render-a-closed-surface-double-integral
\renewcommand{\oiint}{{\large\bigcirc}\kern-1.56em\iint}
\fi


\graphicspath{
  {./}
  {ximeraTutorial/}
  {basicPhilosophy/}
  {functionsOfSeveralVariables/}
  {normalVectors/}
  {lagrangeMultipliers/}
  {vectorFields/}
  {greensTheorem/}
  {shapeOfThingsToCome/}
  {dotProducts/}
  {partialDerivativesAndTheGradientVector/}
  {../productAndQuotientRules/exercises/}
  {../normalVectors/exercisesParametricPlots/}
  {../continuityOfFunctionsOfSeveralVariables/exercises/}
  {../partialDerivativesAndTheGradientVector/exercises/}
  {../directionalDerivativeAndChainRule/exercises/}
  {../commonCoordinates/exercisesCylindricalCoordinates/}
  {../commonCoordinates/exercisesSphericalCoordinates/}
  {../greensTheorem/exercisesCurlAndLineIntegrals/}
  {../greensTheorem/exercisesDivergenceAndLineIntegrals/}
  {../shapeOfThingsToCome/exercisesDivergenceTheorem/}
  {../greensTheorem/}
  {../shapeOfThingsToCome/}
  {../separableDifferentialEquations/exercises/}
  {vectorFields/}
}

\newcommand{\mooculus}{\textsf{\textbf{MOOC}\textnormal{\textsf{ULUS}}}}

\usepackage{tkz-euclide}
\usepackage{tikz}
\usepackage{tikz-cd}
\usetikzlibrary{arrows}
\tikzset{>=stealth,commutative diagrams/.cd,
  arrow style=tikz,diagrams={>=stealth}} %% cool arrow head
\tikzset{shorten <>/.style={ shorten >=#1, shorten <=#1 } } %% allows shorter vectors

\usetikzlibrary{backgrounds} %% for boxes around graphs
\usetikzlibrary{shapes,positioning}  %% Clouds and stars
\usetikzlibrary{matrix} %% for matrix
\usepgfplotslibrary{polar} %% for polar plots
\usepgfplotslibrary{fillbetween} %% to shade area between curves in TikZ
%\usetkzobj{all}
\usepackage[makeroom]{cancel} %% for strike outs
%\usepackage{mathtools} %% for pretty underbrace % Breaks Ximera
%\usepackage{multicol}
\usepackage{pgffor} %% required for integral for loops



%% http://tex.stackexchange.com/questions/66490/drawing-a-tikz-arc-specifying-the-center
%% Draws beach ball
\tikzset{pics/carc/.style args={#1:#2:#3}{code={\draw[pic actions] (#1:#3) arc(#1:#2:#3);}}}



\usepackage{array}
\setlength{\extrarowheight}{+.1cm}
\newdimen\digitwidth
\settowidth\digitwidth{9}
\def\divrule#1#2{
\noalign{\moveright#1\digitwidth
\vbox{\hrule width#2\digitwidth}}}




% \newcommand{\RR}{\mathbb R}
% \newcommand{\R}{\mathbb R}
% \newcommand{\N}{\mathbb N}
% \newcommand{\Z}{\mathbb Z}

\newcommand{\sagemath}{\textsf{SageMath}}


%\renewcommand{\d}{\,d\!}
%\renewcommand{\d}{\mathop{}\!d}
%\newcommand{\dd}[2][]{\frac{\d #1}{\d #2}}
%\newcommand{\pp}[2][]{\frac{\partial #1}{\partial #2}}
% \renewcommand{\l}{\ell}
%\newcommand{\ddx}{\frac{d}{\d x}}

% \newcommand{\zeroOverZero}{\ensuremath{\boldsymbol{\tfrac{0}{0}}}}
%\newcommand{\inftyOverInfty}{\ensuremath{\boldsymbol{\tfrac{\infty}{\infty}}}}
%\newcommand{\zeroOverInfty}{\ensuremath{\boldsymbol{\tfrac{0}{\infty}}}}
%\newcommand{\zeroTimesInfty}{\ensuremath{\small\boldsymbol{0\cdot \infty}}}
%\newcommand{\inftyMinusInfty}{\ensuremath{\small\boldsymbol{\infty - \infty}}}
%\newcommand{\oneToInfty}{\ensuremath{\boldsymbol{1^\infty}}}
%\newcommand{\zeroToZero}{\ensuremath{\boldsymbol{0^0}}}
%\newcommand{\inftyToZero}{\ensuremath{\boldsymbol{\infty^0}}}



% \newcommand{\numOverZero}{\ensuremath{\boldsymbol{\tfrac{\#}{0}}}}
% \newcommand{\dfn}{\textbf}
% \newcommand{\unit}{\,\mathrm}
% \newcommand{\unit}{\mathop{}\!\mathrm}
% \newcommand{\eval}[1]{\bigg[ #1 \bigg]}
% \newcommand{\seq}[1]{\left( #1 \right)}
% \renewcommand{\epsilon}{\varepsilon}
% \renewcommand{\phi}{\varphi}


% \renewcommand{\iff}{\Leftrightarrow}

% \DeclareMathOperator{\arccot}{arccot}
% \DeclareMathOperator{\arcsec}{arcsec}
% \DeclareMathOperator{\arccsc}{arccsc}
% \DeclareMathOperator{\si}{Si}
% \DeclareMathOperator{\scal}{scal}
% \DeclareMathOperator{\sign}{sign}


%% \newcommand{\tightoverset}[2]{% for arrow vec
%%   \mathop{#2}\limits^{\vbox to -.5ex{\kern-0.75ex\hbox{$#1$}\vss}}}
% \newcommand{\arrowvec}[1]{{\overset{\rightharpoonup}{#1}}}
% \renewcommand{\vec}[1]{\arrowvec{\mathbf{#1}}}
% \renewcommand{\vec}[1]{{\overset{\boldsymbol{\rightharpoonup}}{\mathbf{#1}}}}

% \newcommand{\point}[1]{\left(#1\right)} %this allows \vector{ to be changed to \vector{ with a quick find and replace
% \newcommand{\pt}[1]{\mathbf{#1}} %this allows \vec{ to be changed to \vec{ with a quick find and replace
% \newcommand{\Lim}[2]{\lim_{\point{#1} \to \point{#2}}} %Bart, I changed this to point since I want to use it.  It runs through both of the exercise and exerciseE files in limits section, which is why it was in each document to start with.

% \DeclareMathOperator{\proj}{\mathbf{proj}}
% \newcommand{\veci}{{\boldsymbol{\hat{\imath}}}}
% \newcommand{\vecj}{{\boldsymbol{\hat{\jmath}}}}
% \newcommand{\veck}{{\boldsymbol{\hat{k}}}}
% \newcommand{\vecl}{\vec{\boldsymbol{\l}}}
% \newcommand{\uvec}[1]{\mathbf{\hat{#1}}}
% \newcommand{\utan}{\mathbf{\hat{t}}}
% \newcommand{\unormal}{\mathbf{\hat{n}}}
% \newcommand{\ubinormal}{\mathbf{\hat{b}}}

% \newcommand{\dotp}{\bullet}
% \newcommand{\cross}{\boldsymbol\times}
% \newcommand{\grad}{\boldsymbol\nabla}
% \newcommand{\divergence}{\grad\dotp}
% \newcommand{\curl}{\grad\cross}
%\DeclareMathOperator{\divergence}{divergence}
%\DeclareMathOperator{\curl}[1]{\grad\cross #1}
% \newcommand{\lto}{\mathop{\longrightarrow\,}\limits}

% \renewcommand{\bar}{\overline}

\colorlet{textColor}{black}
\colorlet{background}{white}
\colorlet{penColor}{blue!50!black} % Color of a curve in a plot
\colorlet{penColor2}{red!50!black}% Color of a curve in a plot
\colorlet{penColor3}{red!50!blue} % Color of a curve in a plot
\colorlet{penColor4}{green!50!black} % Color of a curve in a plot
\colorlet{penColor5}{orange!80!black} % Color of a curve in a plot
\colorlet{penColor6}{yellow!70!black} % Color of a curve in a plot
\colorlet{fill1}{penColor!20} % Color of fill in a plot
\colorlet{fill2}{penColor2!20} % Color of fill in a plot
\colorlet{fillp}{fill1} % Color of positive area
\colorlet{filln}{penColor2!20} % Color of negative area
\colorlet{fill3}{penColor3!20} % Fill
\colorlet{fill4}{penColor4!20} % Fill
\colorlet{fill5}{penColor5!20} % Fill
\colorlet{gridColor}{gray!50} % Color of grid in a plot

\newcommand{\surfaceColor}{violet}
\newcommand{\surfaceColorTwo}{redyellow}
\newcommand{\sliceColor}{greenyellow}




\pgfmathdeclarefunction{gauss}{2}{% gives gaussian
  \pgfmathparse{1/(#2*sqrt(2*pi))*exp(-((x-#1)^2)/(2*#2^2))}%
}


%%%%%%%%%%%%%
%% Vectors
%%%%%%%%%%%%%

%% Simple horiz vectors
\renewcommand{\vector}[1]{\left\langle #1\right\rangle}


%% %% Complex Horiz Vectors with angle brackets
%% \makeatletter
%% \renewcommand{\vector}[2][ , ]{\left\langle%
%%   \def\nextitem{\def\nextitem{#1}}%
%%   \@for \el:=#2\do{\nextitem\el}\right\rangle%
%% }
%% \makeatother

%% %% Vertical Vectors
%% \def\vector#1{\begin{bmatrix}\vecListA#1,,\end{bmatrix}}
%% \def\vecListA#1,{\if,#1,\else #1\cr \expandafter \vecListA \fi}

%%%%%%%%%%%%%
%% End of vectors
%%%%%%%%%%%%%

%\newcommand{\fullwidth}{}
%\newcommand{\normalwidth}{}



%% makes a snazzy t-chart for evaluating functions
%\newenvironment{tchart}{\rowcolors{2}{}{background!90!textColor}\array}{\endarray}

%%This is to help with formatting on future title pages.
\newenvironment{sectionOutcomes}{}{}



%% Flowchart stuff
%\tikzstyle{startstop} = [rectangle, rounded corners, minimum width=3cm, minimum height=1cm,text centered, draw=black]
%\tikzstyle{question} = [rectangle, minimum width=3cm, minimum height=1cm, text centered, draw=black]
%\tikzstyle{decision} = [trapezium, trapezium left angle=70, trapezium right angle=110, minimum width=3cm, minimum height=1cm, text centered, draw=black]
%\tikzstyle{question} = [rectangle, rounded corners, minimum width=3cm, minimum height=1cm,text centered, draw=black]
%\tikzstyle{process} = [rectangle, minimum width=3cm, minimum height=1cm, text centered, draw=black]
%\tikzstyle{decision} = [trapezium, trapezium left angle=70, trapezium right angle=110, minimum width=3cm, minimum height=1cm, text centered, draw=black]


\title{Just the Algebra}

\begin{document}

\begin{abstract}
no pictures
\end{abstract}
\maketitle




The equation relating two functions tells us the algebraic transformations for the domain and range. \\

We just need to read it correctly. \\



\textbf{\textcolor{blue!55!black}{$\blacktriangleright$}}  Suppose we have a function $H$ with the set $[-5, 1) \cup (7, 11]$ as its domain and the set $(-3, 9]$ as its range. \\

Let's use $d$ to represent domain values of $H$.  That means $d$ represents the numbers $[-5, 1) \cup (7, 11]$. \\

Then, the symbol $H(d)$ represents the numbers $(-3, 9]$, \\


\textbf{\textcolor{blue!55!black}{$\blacktriangleright$}} Now let's introduce a second function called $P$, which is related to $H$. \\

Let's choose $k$ to represent the domain values of $P$. \\


\textbf{\textcolor{blue!55!black}{$\blacktriangleright$}} Let's suppose that $P$ and $H$ are related by the equation 

\[ 
P(k) = 4 H(k-1) - 2
\]



\textbf{\textcolor{blue!55!black}{What can we say about $P$?}}  \\



\subsection*{Domain}


We would like to collect information about the domain of $P$. \\

We know that $P(k) = 4 H(k-1) - 2$.

$k$ is representing the domain values of $P$.  If we can figure out what values $k$ can have, then we would know what values are in the domain of $P$. \\

So, what do we know about $k$? \\


We know that $k-1$ represents domain values of $H$.  We know this because the equation uses the notation $H(k-1)$.  The expression inside the parentheses represents the domain values of $H$.  That's what function notation means. \\


That means $k-1$ represents the values  $[-5, 1) \cup (7, 11]$.


\[
k - 1 \in [-5, 1) \cup (7, 11]
\]


\[
k \in [-4, 2) \cup (8, 12]
\]



We now know the values that $k$ can represent, which means we know the domain of $P$. \\












\subsection*{Range}


We would like to collect information about the range of $P$. \\

In other words, we would like to know the function values of $P$. \\ 


We know that $P(k) = 4 H(k-1) - 2$.\\


Or, we know that $P = 4 H - 2$. \\



We know that $H$ represents the values $(-3, 9]$. \\


Then, $4 H$ represents the values $(4 \cdot -3, 4 \cdot 9] = (-12, 36]$ \\



Then, $4 H - 2$ represents the values $(-12 - 2, 36 - 2] = (-14, 34]$ \\


But $P = 4 H - 2$.  Therefore, $P$ represents the values $(-14, 34]$. \\


The range of $P$ is $(-14, 34]$.






\begin{example}




\textbf{\textcolor{blue!55!black}{$\blacktriangleright$}}  Suppose we have a function $m$ with the set $[-7, 11)$ as its domain and the set $(-12, 8]$ as its range. \\

Let's use $y$ to represent domain values of $m$.  



\begin{question}

That means $y$ represents the numbers 

\[
\left[ \answer{-7}, \answer{11} \right)
\]

\end{question}





\begin{question}

Then, the symbol $m(y)$ represents the numbers

\[
\left( \answer{-12}, \answer{8} \right]
\]

\end{question}




\textbf{\textcolor{blue!55!black}{$\blacktriangleright$}} Now let's introduce a second function called $W$, which is related to $m$. \\

Let's choose $t$ to represent the domain values of $W$. \\


\textbf{\textcolor{blue!55!black}{$\blacktriangleright$}} Let's suppose that $W$ and $m$ are related by the equation 

\[ 
W(t) = -2 m(3t + 2) - 5
\]





\begin{question}

In this function equation, what is representing the domain values of $W$?

\begin{multipleChoice}
\choice [correct]{$t$}
\choice {$k$}
\choice {$3k+2$}
\choice {$3t+2$}
\end{multipleChoice}

\end{question}




\begin{question}

In this function equation, what is representing the domain values of $m$?

\begin{multipleChoice}
\choice {$t$}
\choice {$k$}
\choice {$3k+2$}
\choice [correct]{$3t+2$}
\end{multipleChoice}



The expression $3t + 2$ represents what numbers?


\[
\left[ \answer{-7}, \answer{11} \right)
\]


What numbers does $t$ represent?


\[
\left[ \answer{-3}, \answer{3} \right)
\]


\end{question}



\begin{question}

What is the domain of $W$?



\[
\left[ \answer{-3}, \answer{3} \right)
\]
\end{question}



\end{example}

























\begin{example}




\textbf{\textcolor{blue!55!black}{$\blacktriangleright$}}  Suppose we have a function $m$ with the set $[-7, 11)$ as its domain and the set $(-12, 8]$ as its range. \\

Let's use $y$ to represent domain values of $m$.  



\begin{question}

That means $m$ represents the numbers 

\[
\left( \answer{-12}, \answer{8} \right]
\]

\end{question}






\textbf{\textcolor{blue!55!black}{$\blacktriangleright$}} Now let's introduce a second function called $W$, which is related to $m$. \\

Let's choose $t$ to represent the domain values of $W$. \\


\textbf{\textcolor{blue!55!black}{$\blacktriangleright$}} Let's suppose that $W$ and $m$ are related by the equation 

\[ 
W(t) = -2 m(3t + 2) - 5
\]





\begin{question}

$-2 (-12) - 5 = \answer{19}$ \\

$-2 (8) - 5 = \answer{-21}$ \\

\end{question}






\begin{question}

What are the function values of $W$?


\[
\left[ \answer{-12}, \answer{19} \right)
\]


\end{question}


\end{example}


















\begin{center}
\textbf{\textcolor{green!50!black}{ooooo-=-=-=-ooOoo-=-=-=-ooooo}} \\

more examples can be found by following this link\\ \link[More Examples of Stretching]{https://ximera.osu.edu/csccmathematics/precalculus1/precalculus1/stretching/examples/exampleList}

\end{center}



\end{document}
