\documentclass{ximera}


\graphicspath{
  {./}
  {ximeraTutorial/}
  {basicPhilosophy/}
}

\newcommand{\mooculus}{\textsf{\textbf{MOOC}\textnormal{\textsf{ULUS}}}}


\usepackage{tkz-euclide}\usepackage{tikz}
\usepackage{tikz-cd}
\usetikzlibrary{arrows}
\tikzset{>=stealth,commutative diagrams/.cd,
  arrow style=tikz,diagrams={>=stealth}} %% cool arrow head
\tikzset{shorten <>/.style={ shorten >=#1, shorten <=#1 } } %% allows shorter vectors

\usetikzlibrary{backgrounds} %% for boxes around graphs
\usetikzlibrary{shapes,positioning}  %% Clouds and stars
\usetikzlibrary{matrix} %% for matrix
\usepgfplotslibrary{polar} %% for polar plots
\usepgfplotslibrary{fillbetween} %% to shade area between curves in TikZ
\usetkzobj{all}
\usepackage[makeroom]{cancel} %% for strike outs
%\usepackage{mathtools} %% for pretty underbrace % Breaks Ximera
%\usepackage{multicol}
\usepackage{pgffor} %% required for integral for loops



%% http://tex.stackexchange.com/questions/66490/drawing-a-tikz-arc-specifying-the-center
%% Draws beach ball
\tikzset{pics/carc/.style args={#1:#2:#3}{code={\draw[pic actions] (#1:#3) arc(#1:#2:#3);}}}



\usepackage{array}
\setlength{\extrarowheight}{+.1cm}
\newdimen\digitwidth
\settowidth\digitwidth{9}
\def\divrule#1#2{
\noalign{\moveright#1\digitwidth
\vbox{\hrule width#2\digitwidth}}}
























%%This is to help with formatting on future title pages.
\newenvironment{sectionOutcomes}{}{}


\title{Famous Piecewise}

\begin{document}

\begin{abstract}
examples
\end{abstract}
\maketitle



Absolute value, step, greatest integer function, on/off switch.







\begin{example} Absolute Value Function
The Absolute Value function is made from pieces of two linear functions: $L_1(x) = x$ and $L_2(x) = -x$. 

\begin{itemize}
\item From $L_1$ we'll take pairs with nonnegative domain values.
\item From $L_2$ we'll take pairs with negative domain values.
\end{itemize}

The traditional way of notating the absolute value function is with little vertical bars.


\[
|x| = 
\begin{cases}
  -x & \text{ if }  x < 0 \\
  x & \text{ if } x \geq 0
\end{cases}
\]



Graph of $y = |x|$.
\begin{image}
\begin{tikzpicture}
  \begin{axis}[
            domain=-10:10, ymax=10, xmax=10, ymin=-10, xmin=-10,
            axis lines =center, xlabel=$x$, ylabel=$y$,
            every axis y label/.style={at=(current axis.above origin),anchor=south},
            every axis x label/.style={at=(current axis.right of origin),anchor=west},
            axis on top
          ]
          
  \addplot [draw=penColor,very thick,smooth,domain=(-6:0),<-] {-x};
  \addplot [draw=penColor,very thick,smooth,domain=(0:6),->] {x};


    \end{axis}
\end{tikzpicture}
\end{image}


\end{example}















\begin{example} Greatest Integer Function
The Greatest Integer function (GIF) is made from an infinite number of pieces fo constant functions 

\begin{itemize}
\item From $L_1$ we'll take pairs with nonnegative domain values.
\item From $L_2$ we'll take pairs with negative domain values.
\end{itemize}

The traditional way of notating the absolute value function is with little vertical bars.


\[
|x| = 
\begin{cases}
  -x & \text{ if }  x < 0 \\
  x & \text{ if } x \geq 0
\end{cases}
\]



Graph of $y = |x|$.
\begin{image}
\begin{tikzpicture}
  \begin{axis}[
            domain=-10:10, ymax=10, xmax=10, ymin=-10, xmin=-10,
            axis lines =center, xlabel=$x$, ylabel=$y$,
            every axis y label/.style={at=(current axis.above origin),anchor=south},
            every axis x label/.style={at=(current axis.right of origin),anchor=west},
            axis on top
          ]
          
  \addplot [draw=penColor,very thick,smooth,domain=(-6:0),<-] {-x};
  \addplot [draw=penColor,very thick,smooth,domain=(0:6),->] {x};


    \end{axis}
\end{tikzpicture}
\end{image}


\end{example}








Our next example may be a function that is new to you. It is the
\textit{greatest integer function}.

\begin{example}
Consider the \textbf{greatest integer function}.  This function maps
any real number $x$ to the greatest integer less than or equal to $x$.
%\[
%f(x) = \parbox{3in}{The function that maps any real number $x$
%  to the greatest integer less than or equal to $x$.}
%\]
People sometimes write this as $f(x) = \lfloor x\rfloor$, where those
funny symbols mean exactly the words above describing the
function. For your viewing pleasure, here is a graph of the greatest
integer function:
\begin{image}
\begin{tikzpicture}
  \begin{axis}[
            domain=-2:4,
            width=6in,
            height=3in,
            axis lines =middle, xlabel=$x$, ylabel=$y$,
            every axis y label/.style={at=(current axis.above origin),anchor=south},
            every axis x label/.style={at=(current axis.right of origin),anchor=west},
            clip=false,
            %axis on top,
          ]
          \addplot [textColor, very thin, domain=(0:2.3)] {0}; % puts the axis back, axis on top clobbers our open holes
          \addplot [textColor, very thin] plot coordinates {(0,0) (0,2)}; % puts the axis back, axis on top clobbers our open holes
    \addplot [very thick, penColor, domain=(-2:-1)] {-2};
          \addplot [very thick, penColor, domain=(-1:0)] {-1};
          \addplot [very thick, penColor, domain=(0:1)] {0};
          \addplot [very thick, penColor, domain=(1:2)] {1};
          \addplot [very thick, penColor, domain=(2:3)] {2};
          \addplot [very thick, penColor, domain=(3:4)] {3};
          \addplot[color=penColor,fill=penColor,only marks,mark=*] coordinates{(-2,-2)};  %% closed hole          
          \addplot[color=penColor,fill=penColor,only marks,mark=*] coordinates{(-1,-1)};  %% closed hole          
          \addplot[color=penColor,fill=penColor,only marks,mark=*] coordinates{(0,0)};  %% closed hole          
          \addplot[color=penColor,fill=penColor,only marks,mark=*] coordinates{(1,1)};  %% closed hole          
          \addplot[color=penColor,fill=penColor,only marks,mark=*] coordinates{(2,2)};  %% closed hole  
          \addplot[color=penColor,fill=penColor,only marks,mark=*] coordinates{(3,3)};  %% closed hole                  
          \addplot[color=penColor,fill=background,only marks,mark=*] coordinates{(-1,-2)};  %% open hole
          \addplot[color=penColor,fill=background,only marks,mark=*] coordinates{(0,-1)};  %% open hole
          \addplot[color=penColor,fill=background,only marks,mark=*] coordinates{(1,0)};  %% open hole
          \addplot[color=penColor,fill=background,only marks,mark=*] coordinates{(2,1)};  %% open hole
          \addplot[color=penColor,fill=background,only marks,mark=*] coordinates{(3,2)};  %% open hole
          \addplot[color=penColor,fill=background,only marks,mark=*] coordinates{(4,3)};  %% open hole
        \end{axis}
\end{tikzpicture}
%% \caption{A plot of $f(x)=\lfloor x\rfloor$. Here we can see that for each input (a
%%   value on the $x$-axis), there is exactly one output (a value on the
%%   $y$-axis).}
%% \label{plot:greatest-integer fxn}
\end{image}
Observe that here we have multiple inputs that give
the same output.  This is not a problem! To be a function, we
merely need to check that for each input, there is exactly one output,
and this condition is satisfied.
\end{example}

\begin{question}
  Compute:
  \[
  \lfloor 2.4 \rfloor
  \begin{prompt}
    =\answer{2}
  \end{prompt}
  \]
  \begin{question}
  Compute:
  \[
  \lfloor -2.4 \rfloor
  \begin{prompt}
    =\answer{-3}
  \end{prompt}
  \]
\end{question}
\end{question}


Notice that both the functions described above pass the so-called
\textit{vertical line test}.

\begin{theorem}
The curve $y=f(x)$ represents $y$ as a function of $x$ at $x=a$ if and
only if the vertical line $x=a$ intersects the curve $y=f(x)$ at
exactly one point. This is called the \textbf{vertical line test}.
\end{theorem}



\end{document}
