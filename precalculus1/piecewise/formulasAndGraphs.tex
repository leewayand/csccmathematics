\documentclass{ximera}


\graphicspath{
  {./}
  {ximeraTutorial/}
  {basicPhilosophy/}
}

\newcommand{\mooculus}{\textsf{\textbf{MOOC}\textnormal{\textsf{ULUS}}}}


\usepackage{tkz-euclide}\usepackage{tikz}
\usepackage{tikz-cd}
\usetikzlibrary{arrows}
\tikzset{>=stealth,commutative diagrams/.cd,
  arrow style=tikz,diagrams={>=stealth}} %% cool arrow head
\tikzset{shorten <>/.style={ shorten >=#1, shorten <=#1 } } %% allows shorter vectors

\usetikzlibrary{backgrounds} %% for boxes around graphs
\usetikzlibrary{shapes,positioning}  %% Clouds and stars
\usetikzlibrary{matrix} %% for matrix
\usepgfplotslibrary{polar} %% for polar plots
\usepgfplotslibrary{fillbetween} %% to shade area between curves in TikZ
\usetkzobj{all}
\usepackage[makeroom]{cancel} %% for strike outs
%\usepackage{mathtools} %% for pretty underbrace % Breaks Ximera
%\usepackage{multicol}
\usepackage{pgffor} %% required for integral for loops



%% http://tex.stackexchange.com/questions/66490/drawing-a-tikz-arc-specifying-the-center
%% Draws beach ball
\tikzset{pics/carc/.style args={#1:#2:#3}{code={\draw[pic actions] (#1:#3) arc(#1:#2:#3);}}}



\usepackage{array}
\setlength{\extrarowheight}{+.1cm}
\newdimen\digitwidth
\settowidth\digitwidth{9}
\def\divrule#1#2{
\noalign{\moveright#1\digitwidth
\vbox{\hrule width#2\digitwidth}}}
























%%This is to help with formatting on future title pages.
\newenvironment{sectionOutcomes}{}{}


\title{Piecewise Functions}

\begin{document}

\begin{abstract}
pieces and parts
\end{abstract}
\maketitle



Piecewise functions are defined by using pieces of other functions.




\begin{example} Step Function


The step function uses different formulas depending on the domain number.  If the domain number is positive, then the value of $step$ is $1$. If the value of the domain number is nonpositive, then the value of $step$ is $0$.



\begin{itemize}
\item $step(-5.3) = 0$
\item $step(-0.7) = 0$
\item $step(0) = 0$
\item $step(1.2) = 1$
\item $step(142) = 1$
\end{itemize}




\begin{image}
\begin{tikzpicture}
	\begin{axis}[
            domain=-10:10, ymax=10,xmax=10, ymin=-10, xmin=10,
            axis lines =center, xlabel=$x$, ylabel={$y = step(x)$},
            every axis y label/.style={at=(current axis.above origin),anchor=south},
            every axis x label/.style={at=(current axis.right of origin),anchor=west},
            axis on top,
          ]
          
	\addplot [draw=penColor,very thick,smooth,domain=(-9,0),<-] {0};
	\addplot [draw=penColor2,very thick,smooth,domain=(0,9),->] {1};
	\addplot[color=penColor,only marks,mark=*] coordinates{(0,0)}; 
	\addplot[color=penColor,fill=white,only marks,mark=*] coordinates{(0,1)}; 

    \end{axis}
\end{tikzpicture}
\end{image}





\end{example}

























\end{document}
