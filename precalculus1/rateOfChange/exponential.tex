\documentclass{ximera}


\graphicspath{
  {./}
  {ximeraTutorial/}
  {basicPhilosophy/}
}

\newcommand{\mooculus}{\textsf{\textbf{MOOC}\textnormal{\textsf{ULUS}}}}


\usepackage{tkz-euclide}\usepackage{tikz}
\usepackage{tikz-cd}
\usetikzlibrary{arrows}
\tikzset{>=stealth,commutative diagrams/.cd,
  arrow style=tikz,diagrams={>=stealth}} %% cool arrow head
\tikzset{shorten <>/.style={ shorten >=#1, shorten <=#1 } } %% allows shorter vectors

\usetikzlibrary{backgrounds} %% for boxes around graphs
\usetikzlibrary{shapes,positioning}  %% Clouds and stars
\usetikzlibrary{matrix} %% for matrix
\usepgfplotslibrary{polar} %% for polar plots
\usepgfplotslibrary{fillbetween} %% to shade area between curves in TikZ
\usetkzobj{all}
\usepackage[makeroom]{cancel} %% for strike outs
%\usepackage{mathtools} %% for pretty underbrace % Breaks Ximera
%\usepackage{multicol}
\usepackage{pgffor} %% required for integral for loops



%% http://tex.stackexchange.com/questions/66490/drawing-a-tikz-arc-specifying-the-center
%% Draws beach ball
\tikzset{pics/carc/.style args={#1:#2:#3}{code={\draw[pic actions] (#1:#3) arc(#1:#2:#3);}}}



\usepackage{array}
\setlength{\extrarowheight}{+.1cm}
\newdimen\digitwidth
\settowidth\digitwidth{9}
\def\divrule#1#2{
\noalign{\moveright#1\digitwidth
\vbox{\hrule width#2\digitwidth}}}
























%%This is to help with formatting on future title pages.
\newenvironment{sectionOutcomes}{}{}


\title{Exponential}

\begin{document}

\begin{abstract}
itself
\end{abstract}
\maketitle





\textbf{\textcolor{red!70!darkgray}{$\blacktriangleright$ A Peek ahead to Calculus}}





\textbf{\textcolor{blue!55!black}{In Calculus}}, we will see all of the rules for obtaining the derivative of any funciton for ourselves. \\

Right now we know three rules:

\begin{itemize}
	\item if $f(x) = a \, x^2 + b \, x + c$, then $f'(x) = 2a \, x + b$
	\item if $f(x) = a \, x + b$, then $f'(x) = a$
	\item if $f(x) = a$, then $f'(x) = 0$
\end{itemize}



When we get to Calculus, we'll be able to ``differentiate'' any function we want. \\

Right now, you would be given the derivative of a function, which you could then use to analyze the function. \\



\begin{example}

When analyzing $f(x) = (3x - 2) e^{4x + 5}$, you might be given the derivative.


\[
f'(x) = 3 \, e^{4x + 5} + (3x - 2) e^{4x + 5} \cdot 4
\]


We could then use this derivative to identify critical numbers and decide where the function $f$ is increasing and decreasing.




\textbf{Critical Numbers} \\


To find critical numbers, we need to solve $f'(x) = 0$. \\


\[
f'(x) = 0
\]


\[
3 \, e^{4x + 5} + (3x - 2) e^{4x + 5} \cdot 4 = 0
\]


\[
e^{4x + 5}  (3 + 4 \, (3x - 2)) = 0
\]


\[
e^{4x + 5}  (12x - 5) = 0
\]



By the zero product property, one of the factors must equal $0$. \\



$\blacktriangleright$ $e^{4x + 5}$ is never $0$, since it is an exponential function. \\


$\blacktriangleright$ $(12x - 5) = 0$, when $x = \frac{5}{12}$. \\



$f$ has $\frac{5}{12}$ as its only critical number. \\




\textbf{Behavior}: Increasing and Decreasing \\



The sign of the derivative will tell us where $f$ is increasing and decreasing. \\


\textbf{\textcolor{blue!55!black}{$\blacktriangleright$}} Where is $e^{4x + 5}  (12x - 5)$ positive and negative?


This is a product, so we need the signs of the factors. \\


$\blacktriangleright$  $e^{4x + 5}$ is always positive, since it is an exponential function with a positive leading coefficient. \\


$\blacktriangleright$. $(12x - 5)$ is a linear function with a positive leading coefficient, which means the linear function is increasing, which means it is negative to the left and positive to the right of its zero, which is $\frac{5}{12}$. \\



\[
f'(x) < 0 \, \text{ on } \, \left( -\infty, \frac{5}{12} \right)
\]


\[
f'(x) > 0 \, \text{ on } \, \left( \frac{5}{12}, -\infty \right)
\]





\[
f(x)  \, \text{ decreases on } \, \left( -\infty, \frac{5}{12} \right)
\]


\[
f(x)  \, \text{ increases on } \, \left( \frac{5}{12}, -\infty \right)
\]



Since $f(x)$ is continuous, this tells us that there is a local minimum (possibly global) at $\frac{5}{12}$.


\end{example}






\begin{idea}


We know exponential functions as a description of constant percentage growth. 

\begin{itemize}
	\item If you change the same amount in the domain, then the function changes by the same percentage.
	\item If you change the same amount in the domain, then the function value is multiplied by the same constant.
\end{itemize}



This description compares changes in the function with changes in the domain. That sounds a lot like a derivative. \\

Exponential functions can be described through characteristics of a derivative.


\begin{center}

Exponential functions, $exp(x) = A \, r^x$ are the only functions who are their own derivatives. \\

\[
\frac{d}{dx} A \, e^x = A \, e^x
\]


\end{center}


\end{idea}


We need some more experience with limits to see why this is true. Expereince which Calculus will give us. \\


For now, you would need to be given such derivatives.  You don't know how to get these derivative yet.




\begin{center}
\textbf{\textcolor{green!50!black}{ooooo-=-=-=-ooOoo-=-=-=-ooooo}} \\

more examples can be found by following this link\\ \link[More Examples of Rate of Change]{https://ximera.osu.edu/csccmathematics/precalculus1/precalculus1/rateOfChange/examples/exampleList}

\end{center}






\end{document}
