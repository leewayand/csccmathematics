\documentclass{ximera}


\graphicspath{
  {./}
  {ximeraTutorial/}
  {basicPhilosophy/}
}

\newcommand{\mooculus}{\textsf{\textbf{MOOC}\textnormal{\textsf{ULUS}}}}


\usepackage{tkz-euclide}\usepackage{tikz}
\usepackage{tikz-cd}
\usetikzlibrary{arrows}
\tikzset{>=stealth,commutative diagrams/.cd,
  arrow style=tikz,diagrams={>=stealth}} %% cool arrow head
\tikzset{shorten <>/.style={ shorten >=#1, shorten <=#1 } } %% allows shorter vectors

\usetikzlibrary{backgrounds} %% for boxes around graphs
\usetikzlibrary{shapes,positioning}  %% Clouds and stars
\usetikzlibrary{matrix} %% for matrix
\usepgfplotslibrary{polar} %% for polar plots
\usepgfplotslibrary{fillbetween} %% to shade area between curves in TikZ
\usetkzobj{all}
\usepackage[makeroom]{cancel} %% for strike outs
%\usepackage{mathtools} %% for pretty underbrace % Breaks Ximera
%\usepackage{multicol}
\usepackage{pgffor} %% required for integral for loops



%% http://tex.stackexchange.com/questions/66490/drawing-a-tikz-arc-specifying-the-center
%% Draws beach ball
\tikzset{pics/carc/.style args={#1:#2:#3}{code={\draw[pic actions] (#1:#3) arc(#1:#2:#3);}}}



\usepackage{array}
\setlength{\extrarowheight}{+.1cm}
\newdimen\digitwidth
\settowidth\digitwidth{9}
\def\divrule#1#2{
\noalign{\moveright#1\digitwidth
\vbox{\hrule width#2\digitwidth}}}
























%%This is to help with formatting on future title pages.
\newenvironment{sectionOutcomes}{}{}


\title{Comparing}

\begin{document}

\begin{abstract}
Mean Value Theorem
\end{abstract}
\maketitle






\subsection*{Slope is Velocity}



We know that a steeper line rises faster than a flatter line - a line with a greater slope rises faster than a line with a lesser slope. 


We can say the same thing about functions, by using the slopes of tangent lines.


\begin{quote}

If the graph of $f$ has tangent lines with greater slopes than tangent lines on the graph of $g$, then the graph of $f$ rises faster than the graph of $g$.

\end{quote}


This graphical story can be translated to a function story.  


\begin{quote}

If $f'(x) > g'(x)$, then the values of $f$ rise faster than the values of $g$.

\end{quote}





The derivative tells us how function values are changing. \\

Derivatives measure function behavior. \\






















\subsection*{Horse Races}



The following stories, which weave functions and their derivaties together, are the basis of the \textbf{Mean Value Theorem}, which can be understood through horse races. \\










\begin{idea} \textbf{\textcolor{blue!55!black}{Horse Race \#1}}   


Let two horses start together at the starting line. Suppose the first horse always runs faster than the second horse during the whole race.

Then the first horse wins the race.



\begin{model} Function Translation:

\begin{itemize}
\item Let $f$ and $g$ be functions defined on the interval $[a, b]$. 
\item Let $f(a) = g(a)$
\item Suppose $f'(x) > g'(x)$ for every $x \in (a,b)$. 
\end{itemize}

Under these circumstances, we can conclude that $f(b) > g(b)$.

\end{model} 




\end{idea}












\begin{idea} \textbf{\textcolor{blue!55!black}{Horse Race \#2}}   


Let two horses start together at the starting line. Suppose the first horse always runs as faster or faster than the second horse during the whole race.

Then the first horse is never behind during the whole race.



\begin{model} Function Translation:

\begin{itemize}
\item Let $f$ and $g$ be functions defined on the interval $[a, b]$. 
\item Let $f(a) = g(a)$
\item Suppose $f'(x) \geq g'(x)$ for every $x \in (a,b)$. 
\end{itemize}

Under these circumstances, we can conclude that $f(x) \geq g(x)$ for all $x \in [a,b]$.

\end{model}

\end{idea}








\begin{example} $\sin(x)$ vs. $x$


Let S$(x) = \sin(x)$

It turns out that the derivative of $\sin(x)$ is $\cos(x)$.

$S'(x) = \cos(x)$.

Let $L(x) = x$.  This is a linear function, and as such has a constant rate-of-change of $1$.  Therefore, $L'(x) = 1$. 


\begin{itemize}
\item We have two functions: $\sin(x)$ and $x$. 
\item These two functions are equal at $0$: $S(0) = \sin(0) = 0$ and $L(0) = 0$.
\item $L'(x) \geq S'(x)$ for every $x \in [0,1]$.
\end{itemize}




Therefore, by the Mean Value Theorem (horse race \#2), we can conclude that $x \geq \sin(x)$ on $[0,1]$.

The graph of $y = x$ will always be above the graph of $y = \sin(x)$, which is not so easy to see visually near $0$.





\begin{center}
\desmos{la97djtuzk}{400}{300}
\end{center}







\end{example}




































\begin{center}
\textbf{\textcolor{green!50!black}{ooooo-=-=-=-ooOoo-=-=-=-ooooo}} \\

more examples can be found by following this link\\ \link[More Examples of Rate of Change]{https://ximera.osu.edu/csccmathematics/precalculus1/precalculus1/rateOfChange/examples/exampleList}

\end{center}












\end{document}
