\documentclass{ximera}


\graphicspath{
  {./}
  {ximeraTutorial/}
  {basicPhilosophy/}
}

\newcommand{\mooculus}{\textsf{\textbf{MOOC}\textnormal{\textsf{ULUS}}}}


\usepackage{tkz-euclide}\usepackage{tikz}
\usepackage{tikz-cd}
\usetikzlibrary{arrows}
\tikzset{>=stealth,commutative diagrams/.cd,
  arrow style=tikz,diagrams={>=stealth}} %% cool arrow head
\tikzset{shorten <>/.style={ shorten >=#1, shorten <=#1 } } %% allows shorter vectors

\usetikzlibrary{backgrounds} %% for boxes around graphs
\usetikzlibrary{shapes,positioning}  %% Clouds and stars
\usetikzlibrary{matrix} %% for matrix
\usepgfplotslibrary{polar} %% for polar plots
\usepgfplotslibrary{fillbetween} %% to shade area between curves in TikZ
\usetkzobj{all}
\usepackage[makeroom]{cancel} %% for strike outs
%\usepackage{mathtools} %% for pretty underbrace % Breaks Ximera
%\usepackage{multicol}
\usepackage{pgffor} %% required for integral for loops



%% http://tex.stackexchange.com/questions/66490/drawing-a-tikz-arc-specifying-the-center
%% Draws beach ball
\tikzset{pics/carc/.style args={#1:#2:#3}{code={\draw[pic actions] (#1:#3) arc(#1:#2:#3);}}}



\usepackage{array}
\setlength{\extrarowheight}{+.1cm}
\newdimen\digitwidth
\settowidth\digitwidth{9}
\def\divrule#1#2{
\noalign{\moveright#1\digitwidth
\vbox{\hrule width#2\digitwidth}}}
























%%This is to help with formatting on future title pages.
\newenvironment{sectionOutcomes}{}{}


\title{Logarithmic}

\begin{document}

\begin{abstract}
reciprocal
\end{abstract}
\maketitle












\textbf{\textcolor{red!70!darkgray}{$\blacktriangleright$ A Peek ahead to Calculus}}





\textbf{\textcolor{blue!55!black}{In Calculus}}, we will see all of the rules for obtaining the derivative of any funciton for ourselves. \\

Right now we know three rules:

\begin{itemize}
	\item if $f(x) = a \, x^2 + b \, x + c$, then $f'(x) = 2a \, x + b$
	\item if $f(x) = a \, x + b$, then $f'(x) = a$
	\item if $f(x) = a$, then $f'(x) = 0$
\end{itemize}



When we get to Calculus, we'll be able to ``differentiate'' any function we want. \\

Right now, you would be given the derivative of a function, which you could then use to analyze the function. \\



\begin{example}

When analyzing $g(t) = t - \ln(t)$, you might be given the derivative.


\[
g'(t) = 1 + \frac{1}{t} 
\]


We could then use this derivative to identify critical numbers and decide where the function $g$ is increasing and decreasing.




\textbf{Critical Numbers} \\


\textbf{Note:}. The domain of $g$ is $(0, \infty)$. \\



To find critical numbers, we need to solve $g'(t) = 0$. \\


\[
g'(t) = 0
\]


\[
1 - \frac{1}{t}  = 0
\]


\[
1  = \frac{1}{t}
\]





$\blacktriangleright$ $1  = \frac{1}{t}$, when $t = 1$. \\



$g$ has $1$ as its only critical number. \\




\textbf{Behavior}: Increasing and Decreasing \\



The sign of the derivative will tell us where $g$ is increasing and decreasing. \\


\textbf{\textcolor{blue!55!black}{$\blacktriangleright$}} Where is $1  = \frac{1}{t}$ positive and negative?





\[
g'(t) < 0 \, \text{ on } \, (0, 1)
\]


\[
g'(t) > 0 \, \text{ on } \, (1, \infty)
\]





\[
g(t)  \, \text{ decreases on } \, \left( -\infty, \frac{5}{12} \right)(0, 1)
\]


\[
g(t)  \, \text{ increases on } \, (1, \infty)
\]



Since $g$ is continuous, this tells us that there is a local (possibly global) minimum at $1$.


\end{example}













\end{document}
