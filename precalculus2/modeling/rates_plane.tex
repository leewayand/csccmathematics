\documentclass{ximera}


\graphicspath{
  {./}
  {ximeraTutorial/}
  {basicPhilosophy/}
}

\newcommand{\mooculus}{\textsf{\textbf{MOOC}\textnormal{\textsf{ULUS}}}}


\usepackage{tkz-euclide}\usepackage{tikz}
\usepackage{tikz-cd}
\usetikzlibrary{arrows}
\tikzset{>=stealth,commutative diagrams/.cd,
  arrow style=tikz,diagrams={>=stealth}} %% cool arrow head
\tikzset{shorten <>/.style={ shorten >=#1, shorten <=#1 } } %% allows shorter vectors

\usetikzlibrary{backgrounds} %% for boxes around graphs
\usetikzlibrary{shapes,positioning}  %% Clouds and stars
\usetikzlibrary{matrix} %% for matrix
\usepgfplotslibrary{polar} %% for polar plots
\usepgfplotslibrary{fillbetween} %% to shade area between curves in TikZ
\usetkzobj{all}
\usepackage[makeroom]{cancel} %% for strike outs
%\usepackage{mathtools} %% for pretty underbrace % Breaks Ximera
%\usepackage{multicol}
\usepackage{pgffor} %% required for integral for loops



%% http://tex.stackexchange.com/questions/66490/drawing-a-tikz-arc-specifying-the-center
%% Draws beach ball
\tikzset{pics/carc/.style args={#1:#2:#3}{code={\draw[pic actions] (#1:#3) arc(#1:#2:#3);}}}



\usepackage{array}
\setlength{\extrarowheight}{+.1cm}
\newdimen\digitwidth
\settowidth\digitwidth{9}
\def\divrule#1#2{
\noalign{\moveright#1\digitwidth
\vbox{\hrule width#2\digitwidth}}}
























%%This is to help with formatting on future title pages.
\newenvironment{sectionOutcomes}{}{}


\title{Length}

\begin{document}

\begin{abstract}
rate, length, and time
\end{abstract}
\maketitle




\begin{center}
\textbf{\textcolor{purple!85!blue}{Distance = Rate $\cdot$ Time}} 
\end{center}

This relationship is extremely common in our situations.  We might see it as



\begin{center}
\textbf{\textcolor{purple!85!blue}{Length = Rate $\cdot$ Time}} 
\end{center}



However, these give the different perspective from what we are studying.  These give an accumulation perspective. You  know the elapsed time you can calculate the accumulated distance.  Our focus, right now, is on rates of change. \\


The rate is our focus and it is a comparison measurement between \textbf{\textcolor{red!80!black}{change in length}} and \textbf{\textcolor{red!80!black}{change in time}}.



\[
\text{rate} = \frac{\Delta \text{length}}{\Delta \text{time}} = \frac{\Delta \text{distance}}{\Delta \text{time}}
\]

\textbf{Note:} $\Delta$ is our shorthand notation for ``change in''. \\






\subsection{Airplane}

An airplane is flying overhead on a level flight path, 5 miles above the ground.  The plane is travelling at a constant speed and will travel directly over a tracking station. The tracking station's radar antennea measures the distance from the station to the plane.






\textbf{\textcolor{purple!85!blue}{Step 1: A Picture}}


\begin{image}
\includegraphics{pics/plane_1.png}
\end{image}




\textbf{\textcolor{purple!85!blue}{Step 2: Identify Geometric Objects}}



\begin{image}
\includegraphics{pics/plane_2.png}
\end{image}







\textbf{\textcolor{purple!85!blue}{Step 3: Identify Lengths}}

$\blacktriangleright$ We have three pertinent lengths:

\begin{itemize}
\item $D$ is the distance between the station and the plane.
\item $F$ is the flight distance between the plane the point directly above the station.
\item $H$ is the height of the flight path above the station.
\end{itemize}

All of these are functions of time, $t$: $D(t)$, $F(t)$, and $H(t)$.


\begin{question} 


Which distance measurement is a constant function?

\begin{multipleChoice}
\choice {$D$}
\choice {$F$}
\choice[correct] {$H$}
\end{multipleChoice}

\end{question}



\begin{question} 


Is $D(t)$ an increasing or decreasing function with respect to $t$?

\begin{multipleChoice}
\choice {Increasing}
\choice[correct] {Decreasing}
\end{multipleChoice}

\end{question}



\begin{question} 


Is $F(t)$ an increasing or decreasing function with respoect to $t$?

\begin{multipleChoice}
\choice {Increasing}
\choice[correct] {Decreasing}
\end{multipleChoice}

\end{question}










\textbf{\textcolor{purple!85!blue}{Step 4: Relationships}}


An airplane is flying overhead on a level flight path, 5 miles above the ground.  The plane is travelling at a constant speed and will travel directly over a tracking station. The tracking station's radar antennea measures the distance from the station to the plane. If the distance between the station and the plane is decreasing at a rate of $350$ miles per hour when that distance is $10$ miles, then what is the speed of the plane?


\begin{image}
\includegraphics{pics/plane_2.png}
\end{image}





\begin{question} 


What geometric shape connects these measurements?

\begin{multipleChoice}
\choice {Rectangle}
\choice[correct] {Right Triangle}
\choice {Circle}
\end{multipleChoice}

\end{question}







\begin{question} 


What geometric relationship connects these measurements?

\begin{multipleChoice}
\choice[correct] {Pythagorean Theorem}
\choice {Circumference}
\choice {Similar Triangles}
\end{multipleChoice}

\end{question}






\begin{question} 


Which relationship is suggested by the diagram?

\begin{multipleChoice}
\choice {$D^2 + F^2 = H^2$}
\choice[correct] {$H^2 + F^2 = D^2$}
\choice {$H^2 + D^2 = F^2$}
\end{multipleChoice}

\end{question}






\begin{question} 


$350$ miles per hour is the rate of change of which measurement?

\begin{multipleChoice}
\choice[correct] {$D$}
\choice {$F$}
\choice {$H$}
\end{multipleChoice}

\end{question}















\begin{center}
\textbf{\textcolor{green!50!black}{ooooo=-=-=-=-=-=-=-=-=-=-=-=-=ooOoo=-=-=-=-=-=-=-=-=-=-=-=-=ooooo}} \\

more examples can be found by following this link\\ \link[More Examples of Modeling]{https://ximera.osu.edu/csccmathematics/precalculus2/precalculus2/modeling/examples/exampleList}

\end{center}






\end{document}
