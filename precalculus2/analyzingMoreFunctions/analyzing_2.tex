\documentclass{ximera}


\graphicspath{
  {./}
  {ximeraTutorial/}
  {basicPhilosophy/}
}

\newcommand{\mooculus}{\textsf{\textbf{MOOC}\textnormal{\textsf{ULUS}}}}


\usepackage{tkz-euclide}\usepackage{tikz}
\usepackage{tikz-cd}
\usetikzlibrary{arrows}
\tikzset{>=stealth,commutative diagrams/.cd,
  arrow style=tikz,diagrams={>=stealth}} %% cool arrow head
\tikzset{shorten <>/.style={ shorten >=#1, shorten <=#1 } } %% allows shorter vectors

\usetikzlibrary{backgrounds} %% for boxes around graphs
\usetikzlibrary{shapes,positioning}  %% Clouds and stars
\usetikzlibrary{matrix} %% for matrix
\usepgfplotslibrary{polar} %% for polar plots
\usepgfplotslibrary{fillbetween} %% to shade area between curves in TikZ
\usetkzobj{all}
\usepackage[makeroom]{cancel} %% for strike outs
%\usepackage{mathtools} %% for pretty underbrace % Breaks Ximera
%\usepackage{multicol}
\usepackage{pgffor} %% required for integral for loops



%% http://tex.stackexchange.com/questions/66490/drawing-a-tikz-arc-specifying-the-center
%% Draws beach ball
\tikzset{pics/carc/.style args={#1:#2:#3}{code={\draw[pic actions] (#1:#3) arc(#1:#2:#3);}}}



\usepackage{array}
\setlength{\extrarowheight}{+.1cm}
\newdimen\digitwidth
\settowidth\digitwidth{9}
\def\divrule#1#2{
\noalign{\moveright#1\digitwidth
\vbox{\hrule width#2\digitwidth}}}
























%%This is to help with formatting on future title pages.
\newenvironment{sectionOutcomes}{}{}


\title{Analyzing}

\begin{document}

\begin{abstract}
describe everything
\end{abstract}
\maketitle




$\blacktriangleright$ \textbf{\textcolor{red!80!black}{Reasoning:}} Reasoning is a logical explanation that describes our conclusions, how we arrived at those conclusions, and why we think those conclusions are correct. \\

Analysis is not a list of conclusions. We are not looking for such a list. \\

We are looking for the thought process that arrived at the list of conclusions. \\




\begin{example}


\textbf{\textcolor{purple!85!blue}{Completely analyze}}

\[
p(k) = k \sqrt{4-k^2}
\]


with


\[
p'(k) = \sqrt{4-k^2} + k \cdot \frac{1}{2} \cdot (4-k^2)^{-\tfrac{1}{2}} \cdot (-2k)
\]

\begin{remark}

$p$ is not an elementary function.  It is a product of two functions. One of the factors is a linear funciton, $k$. \\

The other factor is not a linear function.  It is a composition of a root function and a quadratic function. \\

\end{remark}




$\blacktriangleright$  \textbf{\textcolor{blue!55!black}{Domain:}} 

The domain of a product is the intersection of the domains of the two factors. \\

The first factor is a linear function, so its domain is $(-\infty, \infty)$. \\

The second factor is a square root. We need the inside of the square root to be nonnegative (positive or zero).  So, we need $k^2 \leq 4$. That gives us $[-2, 2]$. \\




All together, the intersection leaves us with a natural domain of $[-2, 2]$ for $p$.





$\blacktriangleright$  \textbf{\textcolor{blue!55!black}{Continuity:}} 

Both $k$ and $\sqrt{4 - k^2}$ are continuous. \\


$k$ is a linear function. \\


The other factor is the composition of a square root and a quadratic. Both of those are continuous, which makes the second factor continuous. \\



$p$ is the product of continuous functions, so it is continuous.












$\blacktriangleright$  \textbf{\textcolor{blue!55!black}{Zeros:}} 


\[  p(k) = 0   \]

\[  k \sqrt{4-k^2} = 0  \]


By the Zero Product Property, we have that either $k = 0$ or $\sqrt{4-k^2} = 0$.


Zeros are $-2$, $0$, and $2$. \\








$\blacktriangleright$ \textbf{\textcolor{blue!55!black}{End-Behavior:}}  


End-behavior is only a feature of a function with a domain that is unbounded.  It the domain does not approach $-\infty$ or $\infty$, then there is no end-behavior.\\


Here, the domain is $[-2, 2]$.  So, there is no emdbehavior. \\













$\blacktriangleright$ \textbf{\textcolor{blue!55!black}{Behavior:}}  



We will use the derivative to decipher the behavior of $p$.  But first, let's see how far some algebraic thinking will get us. \\


\begin{idea}

Since $\sqrt{k^2 - 4} \geq 0$, the sign of $p(k)$ is the same as the sign of $\answer{k}$.

\begin{itemize}
\item  $p(k)$ \wordChoice{\choice[correct]{$<$} \choice{$>$}}  $0$ on $(-2, 0)$.
\item  $p(k)$ \wordChoice{\choice{$<$} \choice[correct]{$>$}}  $0$ on $(0, 2)$.
\end{itemize}


Since $p$ is continuous, we must have 


\begin{itemize}
\item  $p(-2) = 0$.
\item  We have $p(-2)=0$ and $p(k)<0$ on $(-2,0)$, therefore, $p(k)$ begins \wordChoice{\choice{increasing} \choice[correct]{decreasing}}  at $-2$.
\item  $p(k)$ eventually \wordChoice{\choice[correct]{increases} \choice{decreases}}  back to $p(0) = 0$.
\item  $p(k)$ continues \wordChoice{\choice[correct]{increasing} \choice{decreasing}}  beyond $p(0) = 0$.
\item  $p(k)$ eventually \wordChoice{\choice{increases} \choice[correct]{decreases}}  back to $p(2) = 0$.
\end{itemize}


There must be at least two critical numbers.  One inside $\left( \answer{-2}, \answer{0} \right)$ and one inside $\left( \answer{0}, \answer{2} \right)$.












Graph of $y = p(k)$.

\begin{image}
\begin{tikzpicture}
  \begin{axis}[
            domain=-2:2, ymax=4, xmax=4, ymin=-4, xmin=-4,
            axis lines =center, xlabel=$k$, ylabel={$y$}, grid = major, grid style={dashed},
            ytick={-4,-2,2,4},
            xtick={-4,-2,2,4},
            yticklabels={$-4$,$-2$,$2$,$4$}, 
            xticklabels={$-4$,$-2$,$2$,$4$},
            ticklabel style={font=\scriptsize},
            every axis y label/.style={at=(current axis.above origin),anchor=south},
            every axis x label/.style={at=(current axis.right of origin),anchor=west},
            axis on top
          ]
          

            %\addplot [line width=2, penColor, smooth,samples=100,domain=(-9:9)] {5/(1 + 3 * e^(-x/2))};
            \addplot [line width=2, penColor, smooth,samples=100,domain=(-2:2)] {x * sqrt(4 - x^2)};


            \addplot[color=penColor,fill=penColor,only marks,mark=*] coordinates{(-2,0) (0,0) (2,0)}; 





           

  \end{axis}
\end{tikzpicture}
\end{image}






\textbf{\textcolor{blue!55!black}{$\blacktriangleright$ desmos graph}} 
\begin{center}
\desmos{lz1d961td4}{400}{300}
\end{center}








With a little help from DESMOS, we can \textbf{approximate} the critical numbers as $-1.414$ and $1.414$.



\begin{itemize}
\item  $p(k)$ decreasing on $[-2, -1.414]$.
\item  $p(k)$ increases on $[-1.414, 1.414]$.
\item  $p(k)$ decreases on $[1.414, 2]$.
\end{itemize}




$p(-1.414)$ is the approximate global (and local) minimum. \\

$p(1.414)$ is the approximate global (and local) maximum. \\


With this thinking, we can make an algebraic approach \\


\end{idea}




We can use the sign of the derivative to help us figure out where $p$ increses and decreases, \\


Step #1 is finding the critical numbers. To do this, we need the derivative in factored form. \\





\[
p'(k) = \sqrt{4-k^2} + k \cdot \frac{1}{2} \cdot (4-k^2)^{-\tfrac{1}{2}}  \cdot (-2k) = \sqrt{4-k^2} + k \frac{-2 k}{2 \sqrt{4-k^2}} 
\]


\[
= \sqrt{4-k^2} - k \frac{k}{\sqrt{4-k^2}} = \frac{4-k^2}{\sqrt{4-k^2}} - \frac{k^2}{\sqrt{4-k^2}}
\]




\[
= - \frac{4-2k^2}{\sqrt{4-k^2}} 
\]


Where is $p'$ equal to $0$ or where does $p'$ not exist? \\



$4-2k^2 = 0$ at $k = -\sqrt{2}$ and $k = -\sqrt{2}$, which are both in the domain.

$p'$ is undefined where the denominator equals $0$, which is at $k=-2$ and $k=2$.  Both of these are in the domain. \\



$p$ has four critical numbers.  Two of these are the endpoints of the domain. So, we have three intervals to investigate. \\



On $\left( -2, -\sqrt{2} \right)$, $2k^2 > 4$.


\[
= - \frac{4-2k^2}{\sqrt{4-k^2}} = negative \cdot \frac{negative}{positive} = positive
\]


$p$ is increasing.\\



On $\left( -\sqrt{2}, \sqrt{2} \right)$, $2k^2 < 4$.


\[
= - \frac{4-2k^2}{\sqrt{4-k^2}} = negative \cdot \frac{positive}{positive} = negative
\]


$p$ is decreasing.\\



On $\left( \sqrt{2}, 2 \right)$


\[
= - \frac{4-2k^2}{\sqrt{4-k^2}} = negative \cdot \frac{negative}{positive} = positive
\]


$p$ is increasing.\\



















$\blacktriangleright$ \textbf{\textcolor{blue!55!black}{Global Extrema:}}  


$p$ is continuous on a closed interval.  We just need to compare the function values at the critical numbers.


\begin{itemize}
  \item $p(-2) = 0$
  \item $p(-\sqrt{2}) = -\sqrt{2}\sqrt{2} = -2$
  \item $p(\sqrt{2}) = \sqrt{2}\sqrt{2} = 2$
  \item $p(2) = 0$
\end{itemize}


The global maximum is $2$, which occurs at $\sqrt{2}$. \\


The global minimum is $-2$, which occurs at $-\sqrt{2}$. \\









$\blacktriangleright$ \textbf{\textcolor{blue!55!black}{Local Extrema:}}  


Since global extreme values are also local extreme values, we have 


$p$ has a local maximum of $2$, which occurs at $\sqrt{2}$. \\


$p$ has a local maximum of $-2$, which occurs at $-\sqrt{2}$. \\


We have two other local extrema.






First, select a small neighborhood interval around $-2$, say $\left( -2 - \frac{1}{10}, -2 + \frac{1}{10} \right)$. For the domain numbers in this interval, $\left[ -2, -2 + \frac{1}{10} \right)$, $p$ is increasing. \\

Since $p$ is increasing, $p(-2)= 0$ a local maximum. \\







Second, select a small neighborhood interval around $2$, say $\left( 2 - \frac{1}{10}, 2 + \frac{1}{10} \right)$. For the domain numbers in this interval, $\left( 2 + \frac{1}{10}, 2 \right]$, $p$ is decreasing. \\

Since $p$ is decreasing, $p(-2)= 0$ a local minimum. \\










$\blacktriangleright$ \textbf{\textcolor{blue!55!black}{Range:}}


$p$ is continuous. $2$ is the global maximum.  $-2$ is the global minimum.  \\


The range of $p$ is $[-2, 2]$. \\






All of this agrees with the graph.

\end{example}













\section*{with Calculus}


Calculus will give you all the rules you need to obtaining formulas for derivative.  In Precalculus, you will need to be given the formula for the derivative.\\


We only know the rules for linear and quadratic functions. \\


In this course, we will also discover the derivative rules for sine and cosine. \\

That will give us four rules with which to practice. \\


The rest will come in Calculus. 











\begin{center}
\textbf{\textcolor{green!50!black}{ooooo-=-=-=-ooOoo-=-=-=-ooooo}} \\

more examples can be found by following this link\\ \link[More Examples of Analyzing More Functions]{https://ximera.osu.edu/csccmathematics/precalculus2/precalculus2/analyzingMoreFunctions/examples/exampleList}

\end{center}





\end{document}
