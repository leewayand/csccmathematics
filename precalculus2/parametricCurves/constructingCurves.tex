\documentclass{ximera}


\graphicspath{
  {./}
  {ximeraTutorial/}
  {basicPhilosophy/}
}

\newcommand{\mooculus}{\textsf{\textbf{MOOC}\textnormal{\textsf{ULUS}}}}


\usepackage{tkz-euclide}\usepackage{tikz}
\usepackage{tikz-cd}
\usetikzlibrary{arrows}
\tikzset{>=stealth,commutative diagrams/.cd,
  arrow style=tikz,diagrams={>=stealth}} %% cool arrow head
\tikzset{shorten <>/.style={ shorten >=#1, shorten <=#1 } } %% allows shorter vectors

\usetikzlibrary{backgrounds} %% for boxes around graphs
\usetikzlibrary{shapes,positioning}  %% Clouds and stars
\usetikzlibrary{matrix} %% for matrix
\usepgfplotslibrary{polar} %% for polar plots
\usepgfplotslibrary{fillbetween} %% to shade area between curves in TikZ
\usetkzobj{all}
\usepackage[makeroom]{cancel} %% for strike outs
%\usepackage{mathtools} %% for pretty underbrace % Breaks Ximera
%\usepackage{multicol}
\usepackage{pgffor} %% required for integral for loops



%% http://tex.stackexchange.com/questions/66490/drawing-a-tikz-arc-specifying-the-center
%% Draws beach ball
\tikzset{pics/carc/.style args={#1:#2:#3}{code={\draw[pic actions] (#1:#3) arc(#1:#2:#3);}}}



\usepackage{array}
\setlength{\extrarowheight}{+.1cm}
\newdimen\digitwidth
\settowidth\digitwidth{9}
\def\divrule#1#2{
\noalign{\moveright#1\digitwidth
\vbox{\hrule width#2\digitwidth}}}
























%%This is to help with formatting on future title pages.
\newenvironment{sectionOutcomes}{}{}


\title{Constructing Curves}

\begin{document}

\begin{abstract}
vs. verifying
\end{abstract}
\maketitle






The polar equation $r(\theta) = 1 + \cos(\theta)$ describes a curve.



    \begin{image}%% 45
       \begin{tikzpicture}
          \begin{polaraxis}[
              xtick={0,45,...,360},
              xticklabels={$0$,$\frac{\pi}{4}$,$\frac{\pi}{2}$,$\frac{3\pi}{4}$,$\pi$,$\frac{5\pi}{4}$,$\frac{3\pi}{2}$,$\frac{7\pi}{4}$,$2\pi$},
              ytick={.5,1,...,2},
            ]
            \addplot+[very thick, mark=none,penColor,domain=0:360,samples=100,smooth] {1+cos(x)};
          \end{polaraxis}
         \end{tikzpicture}
    \end{image}



We can use the conversion equations to convert this into a parametric description.


Using    $x = r \cdot \cos(\theta)$  and  $y = r \cdot \sin(\theta)$ we get


\[
x(\theta) = (1 + \cos(\theta) \cos(\theta) \, \text{ and } \,  y(\theta) = (1 + \cos(\theta) \sin(\theta)
\]



Now, our curve is described as the points 

\[
( (1 + \cos(\theta) \cos(\theta) ,   (1 + \cos(\theta) \sin(\theta) )
\]











\subsection{Conic Sections}




We have already parametrized the unit circle:   $( \cos(\theta), \sin(\theta) )$.


We could also give each coordinate its own radius:  $( 4 \cos(\theta), 7 \sin(\theta) )$.  This parameterization is called an ellipse.










\begin{image}
\begin{tikzpicture}
  \begin{axis}[
            xmin=-10,xmax=10,ymin=-10,ymax=10,
            axis lines=center,
            width=4in,
            xtick={-8,-6,-4,-2,0,2,4,6,8},
            ytick={-8,-6,-4,-2,0,2,4,6,8},
            clip=false,
            unit vector ratio*=1 1 1,
            xlabel=$x$, ylabel=$y$,
            every axis y label/.style={at=(current axis.above origin),anchor=south},
            every axis x label/.style={at=(current axis.right of origin),anchor=west},
          ]        

          

          \addplot [very thick, penColor,smooth, domain=(0:360)] ({4*cos(x)},{7*sin(x)}); %% unit circle



        \end{axis}
\end{tikzpicture}
\end{image}





We can use any two functions as coordinate functions.  The relationship between the coordinate funcitons will determine the shape of the curve.



\begin{example}

A linear relationsship.



$x(t) = t^2$   and  $y(t) = t^2 + 3$ have a linear relationship:  $y = x + 3$.  The curve will be a line, except that both $x(t)$ and $y(t)$ cannot be negative.  Therefore, we will get a ray.




\[
( t^2, t^2 + 3    )
\]










\begin{image}
\begin{tikzpicture}
  \begin{axis}[
            xmin=-10,xmax=10,ymin=-10,ymax=10,
            axis lines=center,
            width=4in,
            xtick={-8,-6,-4,-2,0,2,4,6,8},
            ytick={-8,-6,-4,-2,0,2,4,6,8},
            clip=false,
            unit vector ratio*=1 1 1,
            xlabel=$x$, ylabel=$y$,
            every axis y label/.style={at=(current axis.above origin),anchor=south},
            every axis x label/.style={at=(current axis.right of origin),anchor=west},
          ]        

          

          \addplot [very thick, penColor,smooth, domain=(0:3)] ({x^2},{x^2 + 3}); %% unit circle



        \end{axis}
\end{tikzpicture}
\end{image}



\end{example}

















\end{document}
