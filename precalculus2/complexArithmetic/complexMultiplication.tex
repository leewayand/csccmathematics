\documentclass{ximera}

%\usepackage{todonotes}

\newcommand{\todo}{}

\usepackage{esint} % for \oiint
\ifxake%%https://math.meta.stackexchange.com/questions/9973/how-do-you-render-a-closed-surface-double-integral
\renewcommand{\oiint}{{\large\bigcirc}\kern-1.56em\iint}
\fi


\graphicspath{
  {./}
  {ximeraTutorial/}
  {basicPhilosophy/}
  {functionsOfSeveralVariables/}
  {normalVectors/}
  {lagrangeMultipliers/}
  {vectorFields/}
  {greensTheorem/}
  {shapeOfThingsToCome/}
  {dotProducts/}
  {partialDerivativesAndTheGradientVector/}
  {../productAndQuotientRules/exercises/}
  {../normalVectors/exercisesParametricPlots/}
  {../continuityOfFunctionsOfSeveralVariables/exercises/}
  {../partialDerivativesAndTheGradientVector/exercises/}
  {../directionalDerivativeAndChainRule/exercises/}
  {../commonCoordinates/exercisesCylindricalCoordinates/}
  {../commonCoordinates/exercisesSphericalCoordinates/}
  {../greensTheorem/exercisesCurlAndLineIntegrals/}
  {../greensTheorem/exercisesDivergenceAndLineIntegrals/}
  {../shapeOfThingsToCome/exercisesDivergenceTheorem/}
  {../greensTheorem/}
  {../shapeOfThingsToCome/}
  {../separableDifferentialEquations/exercises/}
  {vectorFields/}
}

\newcommand{\mooculus}{\textsf{\textbf{MOOC}\textnormal{\textsf{ULUS}}}}

\usepackage{tkz-euclide}
\usepackage{tikz}
\usepackage{tikz-cd}
\usetikzlibrary{arrows}
\tikzset{>=stealth,commutative diagrams/.cd,
  arrow style=tikz,diagrams={>=stealth}} %% cool arrow head
\tikzset{shorten <>/.style={ shorten >=#1, shorten <=#1 } } %% allows shorter vectors

\usetikzlibrary{backgrounds} %% for boxes around graphs
\usetikzlibrary{shapes,positioning}  %% Clouds and stars
\usetikzlibrary{matrix} %% for matrix
\usepgfplotslibrary{polar} %% for polar plots
\usepgfplotslibrary{fillbetween} %% to shade area between curves in TikZ
%\usetkzobj{all}
\usepackage[makeroom]{cancel} %% for strike outs
%\usepackage{mathtools} %% for pretty underbrace % Breaks Ximera
%\usepackage{multicol}
\usepackage{pgffor} %% required for integral for loops



%% http://tex.stackexchange.com/questions/66490/drawing-a-tikz-arc-specifying-the-center
%% Draws beach ball
\tikzset{pics/carc/.style args={#1:#2:#3}{code={\draw[pic actions] (#1:#3) arc(#1:#2:#3);}}}



\usepackage{array}
\setlength{\extrarowheight}{+.1cm}
\newdimen\digitwidth
\settowidth\digitwidth{9}
\def\divrule#1#2{
\noalign{\moveright#1\digitwidth
\vbox{\hrule width#2\digitwidth}}}




% \newcommand{\RR}{\mathbb R}
% \newcommand{\R}{\mathbb R}
% \newcommand{\N}{\mathbb N}
% \newcommand{\Z}{\mathbb Z}

\newcommand{\sagemath}{\textsf{SageMath}}


%\renewcommand{\d}{\,d\!}
%\renewcommand{\d}{\mathop{}\!d}
%\newcommand{\dd}[2][]{\frac{\d #1}{\d #2}}
%\newcommand{\pp}[2][]{\frac{\partial #1}{\partial #2}}
% \renewcommand{\l}{\ell}
%\newcommand{\ddx}{\frac{d}{\d x}}

% \newcommand{\zeroOverZero}{\ensuremath{\boldsymbol{\tfrac{0}{0}}}}
%\newcommand{\inftyOverInfty}{\ensuremath{\boldsymbol{\tfrac{\infty}{\infty}}}}
%\newcommand{\zeroOverInfty}{\ensuremath{\boldsymbol{\tfrac{0}{\infty}}}}
%\newcommand{\zeroTimesInfty}{\ensuremath{\small\boldsymbol{0\cdot \infty}}}
%\newcommand{\inftyMinusInfty}{\ensuremath{\small\boldsymbol{\infty - \infty}}}
%\newcommand{\oneToInfty}{\ensuremath{\boldsymbol{1^\infty}}}
%\newcommand{\zeroToZero}{\ensuremath{\boldsymbol{0^0}}}
%\newcommand{\inftyToZero}{\ensuremath{\boldsymbol{\infty^0}}}



% \newcommand{\numOverZero}{\ensuremath{\boldsymbol{\tfrac{\#}{0}}}}
% \newcommand{\dfn}{\textbf}
% \newcommand{\unit}{\,\mathrm}
% \newcommand{\unit}{\mathop{}\!\mathrm}
% \newcommand{\eval}[1]{\bigg[ #1 \bigg]}
% \newcommand{\seq}[1]{\left( #1 \right)}
% \renewcommand{\epsilon}{\varepsilon}
% \renewcommand{\phi}{\varphi}


% \renewcommand{\iff}{\Leftrightarrow}

% \DeclareMathOperator{\arccot}{arccot}
% \DeclareMathOperator{\arcsec}{arcsec}
% \DeclareMathOperator{\arccsc}{arccsc}
% \DeclareMathOperator{\si}{Si}
% \DeclareMathOperator{\scal}{scal}
% \DeclareMathOperator{\sign}{sign}


%% \newcommand{\tightoverset}[2]{% for arrow vec
%%   \mathop{#2}\limits^{\vbox to -.5ex{\kern-0.75ex\hbox{$#1$}\vss}}}
% \newcommand{\arrowvec}[1]{{\overset{\rightharpoonup}{#1}}}
% \renewcommand{\vec}[1]{\arrowvec{\mathbf{#1}}}
% \renewcommand{\vec}[1]{{\overset{\boldsymbol{\rightharpoonup}}{\mathbf{#1}}}}

% \newcommand{\point}[1]{\left(#1\right)} %this allows \vector{ to be changed to \vector{ with a quick find and replace
% \newcommand{\pt}[1]{\mathbf{#1}} %this allows \vec{ to be changed to \vec{ with a quick find and replace
% \newcommand{\Lim}[2]{\lim_{\point{#1} \to \point{#2}}} %Bart, I changed this to point since I want to use it.  It runs through both of the exercise and exerciseE files in limits section, which is why it was in each document to start with.

% \DeclareMathOperator{\proj}{\mathbf{proj}}
% \newcommand{\veci}{{\boldsymbol{\hat{\imath}}}}
% \newcommand{\vecj}{{\boldsymbol{\hat{\jmath}}}}
% \newcommand{\veck}{{\boldsymbol{\hat{k}}}}
% \newcommand{\vecl}{\vec{\boldsymbol{\l}}}
% \newcommand{\uvec}[1]{\mathbf{\hat{#1}}}
% \newcommand{\utan}{\mathbf{\hat{t}}}
% \newcommand{\unormal}{\mathbf{\hat{n}}}
% \newcommand{\ubinormal}{\mathbf{\hat{b}}}

% \newcommand{\dotp}{\bullet}
% \newcommand{\cross}{\boldsymbol\times}
% \newcommand{\grad}{\boldsymbol\nabla}
% \newcommand{\divergence}{\grad\dotp}
% \newcommand{\curl}{\grad\cross}
%\DeclareMathOperator{\divergence}{divergence}
%\DeclareMathOperator{\curl}[1]{\grad\cross #1}
% \newcommand{\lto}{\mathop{\longrightarrow\,}\limits}

% \renewcommand{\bar}{\overline}

\colorlet{textColor}{black}
\colorlet{background}{white}
\colorlet{penColor}{blue!50!black} % Color of a curve in a plot
\colorlet{penColor2}{red!50!black}% Color of a curve in a plot
\colorlet{penColor3}{red!50!blue} % Color of a curve in a plot
\colorlet{penColor4}{green!50!black} % Color of a curve in a plot
\colorlet{penColor5}{orange!80!black} % Color of a curve in a plot
\colorlet{penColor6}{yellow!70!black} % Color of a curve in a plot
\colorlet{fill1}{penColor!20} % Color of fill in a plot
\colorlet{fill2}{penColor2!20} % Color of fill in a plot
\colorlet{fillp}{fill1} % Color of positive area
\colorlet{filln}{penColor2!20} % Color of negative area
\colorlet{fill3}{penColor3!20} % Fill
\colorlet{fill4}{penColor4!20} % Fill
\colorlet{fill5}{penColor5!20} % Fill
\colorlet{gridColor}{gray!50} % Color of grid in a plot

\newcommand{\surfaceColor}{violet}
\newcommand{\surfaceColorTwo}{redyellow}
\newcommand{\sliceColor}{greenyellow}




\pgfmathdeclarefunction{gauss}{2}{% gives gaussian
  \pgfmathparse{1/(#2*sqrt(2*pi))*exp(-((x-#1)^2)/(2*#2^2))}%
}


%%%%%%%%%%%%%
%% Vectors
%%%%%%%%%%%%%

%% Simple horiz vectors
\renewcommand{\vector}[1]{\left\langle #1\right\rangle}


%% %% Complex Horiz Vectors with angle brackets
%% \makeatletter
%% \renewcommand{\vector}[2][ , ]{\left\langle%
%%   \def\nextitem{\def\nextitem{#1}}%
%%   \@for \el:=#2\do{\nextitem\el}\right\rangle%
%% }
%% \makeatother

%% %% Vertical Vectors
%% \def\vector#1{\begin{bmatrix}\vecListA#1,,\end{bmatrix}}
%% \def\vecListA#1,{\if,#1,\else #1\cr \expandafter \vecListA \fi}

%%%%%%%%%%%%%
%% End of vectors
%%%%%%%%%%%%%

%\newcommand{\fullwidth}{}
%\newcommand{\normalwidth}{}



%% makes a snazzy t-chart for evaluating functions
%\newenvironment{tchart}{\rowcolors{2}{}{background!90!textColor}\array}{\endarray}

%%This is to help with formatting on future title pages.
\newenvironment{sectionOutcomes}{}{}



%% Flowchart stuff
%\tikzstyle{startstop} = [rectangle, rounded corners, minimum width=3cm, minimum height=1cm,text centered, draw=black]
%\tikzstyle{question} = [rectangle, minimum width=3cm, minimum height=1cm, text centered, draw=black]
%\tikzstyle{decision} = [trapezium, trapezium left angle=70, trapezium right angle=110, minimum width=3cm, minimum height=1cm, text centered, draw=black]
%\tikzstyle{question} = [rectangle, rounded corners, minimum width=3cm, minimum height=1cm,text centered, draw=black]
%\tikzstyle{process} = [rectangle, minimum width=3cm, minimum height=1cm, text centered, draw=black]
%\tikzstyle{decision} = [trapezium, trapezium left angle=70, trapezium right angle=110, minimum width=3cm, minimum height=1cm, text centered, draw=black]


\title{Complex Multiplication}

\begin{document}

\begin{abstract}
products
\end{abstract}
\maketitle


Multiplication of complex numbers looks weird.





\begin{definition} \textbf{\textcolor{green!50!black}{Complex Multiplication}}


\[    (a + b \, i) \cdot (c + d \, i) = (ac-bd) + (ad+bc) \, i           \]

\end{definition}


It is defined this way to maintain the distributive property, which we want to keep from the real numbers.  The distributive property is vital to our algebra.



$\blacktriangleright$ Division requires a little discussion.



With real numbers, we have stories that give meaning to division.


$10 \div 5$ has two meanings.  This expresion might represent how many groups are needed to split a group of $10$ marbles into groups of size $5$.  That would be $2$ groups.  It could also represent how many marbles would be in each group, if $10$ marbles were split up into $5$ equal groups.  That would be $2$ marbles per group.


But, even real numbers stretch this story.  $10 \div \sqrt{2}$ ?  What does it mean to split a pile of marbles into $\sqrt{2}$ piles?

This marble story is also going to break with complex numbers.  Fortunately, the real numbers have a different view of division, which we can bring with us over to complex numbers.





Just from a strictly algebaic viewpoint, $10 \div 5$ represents the number that you multiply $5$ by, to get $10$.  In this way, we can think about division in terms of multiplication.  This thought leads to reciprocals and fractions.  Luckily, every real number, except $0$ has a reciprocal.  




\textbf{Note:}  $10 \div 0$ has no meaning in real numbers, because there is no real number that you can mulitply by $0$ and get $10$. \\









What about complex numbers?  Do they have reciprocals?  Does each complex number have a partner complex number such that their product equals $1$?



 $\blacktriangleright$ Can we solve $1 = (2 + 3 \, i) \cdot (a + b \, i)  $?



\begin{example} Multiplicative Inverse


Solve $1 = (2 + 3 \, i) \cdot (a + b \, i)  $?


\begin{explanation}

\[   1 = (2 + 3 \, i) \cdot (a + b \, i)  =  (2a - 3b) + (2b+3a) \, i      \]






We need $1 = \answer{2a - 3b}$ and $0 = \answer{2b+3a}$.


The second  equation tells us that $\frac{-3a}{2} = b$.  Substituting this into the first equation tells us that $1 = 2a - 3 \cdot \frac{-3a}{2}$.  Now we can solve for $a$.



\begin{align*}
1     & = 2a - 3 \left( \frac{-3a}{2} \right)  \\
      & =   2a + \answer{\frac{9a}{2}}   \\
      & =    \frac{13a}{2}   \\
  \frac{2}{13}    & =  a
\end{align*}




With that, we can get a value for $b$.   


\[ b = \frac{-3a}{2} = \frac{2}{13}  \cdot \frac{-3}{2} = -\frac{3}{13} \]

Let's check:



\[    (2 + 3 \, i) \cdot \left(\frac{2}{13} - \frac{3}{13} \, i \right)     =   \left( 2 \cdot  \frac{2}{13} - 3 \cdot \left(- \frac{3}{13}\right)\right)  + \left( 3 \cdot \frac{2}{13} - 2 \cdot \frac{3}{13} \right) \, i = \frac{4+9}{13} + \frac{6-6}{13} \, i = 1  \]



\end{explanation}



\end{example}


Does this process work for all nonzero complex numbers? \\


$\blacktriangleright$  We know it works for $r + 0 \, i$ with $r \ne 0$, because these are real numbers.


\[   (r + 0 \, i) \cdot \left(\frac{1}{r} + 0 \, i \right) = 1        \]







$\blacktriangleright$  Similarly, it works for $0 + r \, i$ with $r \ne 0$.


\[   (0 + r \, i) \cdot \left(0 - \frac{1}{r} \, i \right) = 1        \]





$\blacktriangleright$ Now for complex numbers where neither component is $0$


\begin{explanation}


Given real numbers $A \ne 0$ and $B \ne 0$, solve the following for $x$ and $y$.


\[       1 = (A + B \, i) \cdot (x + y \, i)            \]



\begin{align*}
1          & = (A + B \, i) \cdot (x + y \, i)      \\
           & = (Ax-By) + (Bx+Ay) \, i
\end{align*}


For this to work, we need $\answer{A x - B y} = 1$ and $\answer{B x + A y} = 0$.



The second equation tells us that $x = \answer{-\frac{A y}{B}}$, which is ok, since $B \ne 0$ in this case.

Substituting that into the first equation, for $x$, gives us


\[   A \cdot \left(-\frac{A y}{B}\right) - B y = 1     \]


\[   -\frac{A^2 y}{B} - B y = 1     \]

\[   -\frac{A^2 y}{B} - \frac{B^2}{B} y = 1     \]


\[   -\frac{A^2 + B^2}{B} y = 1     \]


\[  y = \frac{-B}{A^2 + B^2}     \]

This is ok, since $A^2 + B^2 \ne 0$, because neither $A$ nor $B$ equals $0$.



This gives us $x = -\frac{A y}{B} = -\frac{A}{B} \cdot \frac{-B}{A^2 + B^2} = \frac{A}{A^2 + B^2} $.







\[       (A + B \, i) \cdot \left( \frac{A}{A^2 + B^2} - \frac{B}{A^2 + B^2} \, i \right) = 1            \]



\end{explanation}












$\blacktriangleright$  Every nonzero complex number has a reciprocal.  We will use this to define division as multiplication by the reciprocal.





\section*{Complex Conjugate}


The \textbf{modulus} of a complex number is its distance from the origin or $0$.  The symbol for the modulus of a complex number, $z$, is $|z|$.  \\

If $z = a + b \, i$, then

\[    |z| = |a + b \, i| = \sqrt{a^2 + b^2}          \]



This is helpful, because a sum of two squares factors over the Complex numbers.



\[   (a + b \, i) \cdot    (a - b \, i)  = a^2 + b^2   \]



$\blacktriangleright$  $a - b \, i$ is called the \textbf{complex conjugate} of $a + b \, i$ and it will help us with division, by converting complex denominators to real numbers.




\[   \frac{1}{a + b \, i} =   \frac{1}{a + b \, i} \cdot 1 = \frac{1}{a + b \, i} \cdot \frac{a - b \, i}{a - b \, i}  =  \frac{a - b \, i}{(a + b \, i)(a - b \, i)} =   \frac{a - b \, i}{a^2 + b^2}  \]


\[  \frac{1}{a + b \, i}   =  \frac{a}{a^2 + b^2} - \frac{b}{a^2 + b^2} \, i      \]


We can use the complex conjugate of the denominator to get us back to standard form.















\begin{center}
\textbf{\textcolor{green!50!black}{ooooo-=-=-=-ooOoo-=-=-=-ooooo}} \\

more examples can be found by following this link\\ \link[More Examples of Complex Arithmetic]{https://ximera.osu.edu/csccmathematics/precalculus2/precalculus2/complexArithmetic/examples/exampleList}

\end{center}


\end{document}
