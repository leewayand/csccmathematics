\documentclass{ximera}

%\usepackage{todonotes}

\newcommand{\todo}{}

\usepackage{esint} % for \oiint
\ifxake%%https://math.meta.stackexchange.com/questions/9973/how-do-you-render-a-closed-surface-double-integral
\renewcommand{\oiint}{{\large\bigcirc}\kern-1.56em\iint}
\fi


\graphicspath{
  {./}
  {ximeraTutorial/}
  {basicPhilosophy/}
  {functionsOfSeveralVariables/}
  {normalVectors/}
  {lagrangeMultipliers/}
  {vectorFields/}
  {greensTheorem/}
  {shapeOfThingsToCome/}
  {dotProducts/}
  {partialDerivativesAndTheGradientVector/}
  {../productAndQuotientRules/exercises/}
  {../normalVectors/exercisesParametricPlots/}
  {../continuityOfFunctionsOfSeveralVariables/exercises/}
  {../partialDerivativesAndTheGradientVector/exercises/}
  {../directionalDerivativeAndChainRule/exercises/}
  {../commonCoordinates/exercisesCylindricalCoordinates/}
  {../commonCoordinates/exercisesSphericalCoordinates/}
  {../greensTheorem/exercisesCurlAndLineIntegrals/}
  {../greensTheorem/exercisesDivergenceAndLineIntegrals/}
  {../shapeOfThingsToCome/exercisesDivergenceTheorem/}
  {../greensTheorem/}
  {../shapeOfThingsToCome/}
  {../separableDifferentialEquations/exercises/}
  {vectorFields/}
}

\newcommand{\mooculus}{\textsf{\textbf{MOOC}\textnormal{\textsf{ULUS}}}}

\usepackage{tkz-euclide}
\usepackage{tikz}
\usepackage{tikz-cd}
\usetikzlibrary{arrows}
\tikzset{>=stealth,commutative diagrams/.cd,
  arrow style=tikz,diagrams={>=stealth}} %% cool arrow head
\tikzset{shorten <>/.style={ shorten >=#1, shorten <=#1 } } %% allows shorter vectors

\usetikzlibrary{backgrounds} %% for boxes around graphs
\usetikzlibrary{shapes,positioning}  %% Clouds and stars
\usetikzlibrary{matrix} %% for matrix
\usepgfplotslibrary{polar} %% for polar plots
\usepgfplotslibrary{fillbetween} %% to shade area between curves in TikZ
%\usetkzobj{all}
\usepackage[makeroom]{cancel} %% for strike outs
%\usepackage{mathtools} %% for pretty underbrace % Breaks Ximera
%\usepackage{multicol}
\usepackage{pgffor} %% required for integral for loops



%% http://tex.stackexchange.com/questions/66490/drawing-a-tikz-arc-specifying-the-center
%% Draws beach ball
\tikzset{pics/carc/.style args={#1:#2:#3}{code={\draw[pic actions] (#1:#3) arc(#1:#2:#3);}}}



\usepackage{array}
\setlength{\extrarowheight}{+.1cm}
\newdimen\digitwidth
\settowidth\digitwidth{9}
\def\divrule#1#2{
\noalign{\moveright#1\digitwidth
\vbox{\hrule width#2\digitwidth}}}




% \newcommand{\RR}{\mathbb R}
% \newcommand{\R}{\mathbb R}
% \newcommand{\N}{\mathbb N}
% \newcommand{\Z}{\mathbb Z}

\newcommand{\sagemath}{\textsf{SageMath}}


%\renewcommand{\d}{\,d\!}
%\renewcommand{\d}{\mathop{}\!d}
%\newcommand{\dd}[2][]{\frac{\d #1}{\d #2}}
%\newcommand{\pp}[2][]{\frac{\partial #1}{\partial #2}}
% \renewcommand{\l}{\ell}
%\newcommand{\ddx}{\frac{d}{\d x}}

% \newcommand{\zeroOverZero}{\ensuremath{\boldsymbol{\tfrac{0}{0}}}}
%\newcommand{\inftyOverInfty}{\ensuremath{\boldsymbol{\tfrac{\infty}{\infty}}}}
%\newcommand{\zeroOverInfty}{\ensuremath{\boldsymbol{\tfrac{0}{\infty}}}}
%\newcommand{\zeroTimesInfty}{\ensuremath{\small\boldsymbol{0\cdot \infty}}}
%\newcommand{\inftyMinusInfty}{\ensuremath{\small\boldsymbol{\infty - \infty}}}
%\newcommand{\oneToInfty}{\ensuremath{\boldsymbol{1^\infty}}}
%\newcommand{\zeroToZero}{\ensuremath{\boldsymbol{0^0}}}
%\newcommand{\inftyToZero}{\ensuremath{\boldsymbol{\infty^0}}}



% \newcommand{\numOverZero}{\ensuremath{\boldsymbol{\tfrac{\#}{0}}}}
% \newcommand{\dfn}{\textbf}
% \newcommand{\unit}{\,\mathrm}
% \newcommand{\unit}{\mathop{}\!\mathrm}
% \newcommand{\eval}[1]{\bigg[ #1 \bigg]}
% \newcommand{\seq}[1]{\left( #1 \right)}
% \renewcommand{\epsilon}{\varepsilon}
% \renewcommand{\phi}{\varphi}


% \renewcommand{\iff}{\Leftrightarrow}

% \DeclareMathOperator{\arccot}{arccot}
% \DeclareMathOperator{\arcsec}{arcsec}
% \DeclareMathOperator{\arccsc}{arccsc}
% \DeclareMathOperator{\si}{Si}
% \DeclareMathOperator{\scal}{scal}
% \DeclareMathOperator{\sign}{sign}


%% \newcommand{\tightoverset}[2]{% for arrow vec
%%   \mathop{#2}\limits^{\vbox to -.5ex{\kern-0.75ex\hbox{$#1$}\vss}}}
% \newcommand{\arrowvec}[1]{{\overset{\rightharpoonup}{#1}}}
% \renewcommand{\vec}[1]{\arrowvec{\mathbf{#1}}}
% \renewcommand{\vec}[1]{{\overset{\boldsymbol{\rightharpoonup}}{\mathbf{#1}}}}

% \newcommand{\point}[1]{\left(#1\right)} %this allows \vector{ to be changed to \vector{ with a quick find and replace
% \newcommand{\pt}[1]{\mathbf{#1}} %this allows \vec{ to be changed to \vec{ with a quick find and replace
% \newcommand{\Lim}[2]{\lim_{\point{#1} \to \point{#2}}} %Bart, I changed this to point since I want to use it.  It runs through both of the exercise and exerciseE files in limits section, which is why it was in each document to start with.

% \DeclareMathOperator{\proj}{\mathbf{proj}}
% \newcommand{\veci}{{\boldsymbol{\hat{\imath}}}}
% \newcommand{\vecj}{{\boldsymbol{\hat{\jmath}}}}
% \newcommand{\veck}{{\boldsymbol{\hat{k}}}}
% \newcommand{\vecl}{\vec{\boldsymbol{\l}}}
% \newcommand{\uvec}[1]{\mathbf{\hat{#1}}}
% \newcommand{\utan}{\mathbf{\hat{t}}}
% \newcommand{\unormal}{\mathbf{\hat{n}}}
% \newcommand{\ubinormal}{\mathbf{\hat{b}}}

% \newcommand{\dotp}{\bullet}
% \newcommand{\cross}{\boldsymbol\times}
% \newcommand{\grad}{\boldsymbol\nabla}
% \newcommand{\divergence}{\grad\dotp}
% \newcommand{\curl}{\grad\cross}
%\DeclareMathOperator{\divergence}{divergence}
%\DeclareMathOperator{\curl}[1]{\grad\cross #1}
% \newcommand{\lto}{\mathop{\longrightarrow\,}\limits}

% \renewcommand{\bar}{\overline}

\colorlet{textColor}{black}
\colorlet{background}{white}
\colorlet{penColor}{blue!50!black} % Color of a curve in a plot
\colorlet{penColor2}{red!50!black}% Color of a curve in a plot
\colorlet{penColor3}{red!50!blue} % Color of a curve in a plot
\colorlet{penColor4}{green!50!black} % Color of a curve in a plot
\colorlet{penColor5}{orange!80!black} % Color of a curve in a plot
\colorlet{penColor6}{yellow!70!black} % Color of a curve in a plot
\colorlet{fill1}{penColor!20} % Color of fill in a plot
\colorlet{fill2}{penColor2!20} % Color of fill in a plot
\colorlet{fillp}{fill1} % Color of positive area
\colorlet{filln}{penColor2!20} % Color of negative area
\colorlet{fill3}{penColor3!20} % Fill
\colorlet{fill4}{penColor4!20} % Fill
\colorlet{fill5}{penColor5!20} % Fill
\colorlet{gridColor}{gray!50} % Color of grid in a plot

\newcommand{\surfaceColor}{violet}
\newcommand{\surfaceColorTwo}{redyellow}
\newcommand{\sliceColor}{greenyellow}




\pgfmathdeclarefunction{gauss}{2}{% gives gaussian
  \pgfmathparse{1/(#2*sqrt(2*pi))*exp(-((x-#1)^2)/(2*#2^2))}%
}


%%%%%%%%%%%%%
%% Vectors
%%%%%%%%%%%%%

%% Simple horiz vectors
\renewcommand{\vector}[1]{\left\langle #1\right\rangle}


%% %% Complex Horiz Vectors with angle brackets
%% \makeatletter
%% \renewcommand{\vector}[2][ , ]{\left\langle%
%%   \def\nextitem{\def\nextitem{#1}}%
%%   \@for \el:=#2\do{\nextitem\el}\right\rangle%
%% }
%% \makeatother

%% %% Vertical Vectors
%% \def\vector#1{\begin{bmatrix}\vecListA#1,,\end{bmatrix}}
%% \def\vecListA#1,{\if,#1,\else #1\cr \expandafter \vecListA \fi}

%%%%%%%%%%%%%
%% End of vectors
%%%%%%%%%%%%%

%\newcommand{\fullwidth}{}
%\newcommand{\normalwidth}{}



%% makes a snazzy t-chart for evaluating functions
%\newenvironment{tchart}{\rowcolors{2}{}{background!90!textColor}\array}{\endarray}

%%This is to help with formatting on future title pages.
\newenvironment{sectionOutcomes}{}{}



%% Flowchart stuff
%\tikzstyle{startstop} = [rectangle, rounded corners, minimum width=3cm, minimum height=1cm,text centered, draw=black]
%\tikzstyle{question} = [rectangle, minimum width=3cm, minimum height=1cm, text centered, draw=black]
%\tikzstyle{decision} = [trapezium, trapezium left angle=70, trapezium right angle=110, minimum width=3cm, minimum height=1cm, text centered, draw=black]
%\tikzstyle{question} = [rectangle, rounded corners, minimum width=3cm, minimum height=1cm,text centered, draw=black]
%\tikzstyle{process} = [rectangle, minimum width=3cm, minimum height=1cm, text centered, draw=black]
%\tikzstyle{decision} = [trapezium, trapezium left angle=70, trapezium right angle=110, minimum width=3cm, minimum height=1cm, text centered, draw=black]


\title{Operations}

\begin{document}

\begin{abstract}
addition and subtraction
\end{abstract}
\maketitle



Complex Numbers are 2-dimensional numbers, which we represent as $a + b \, i$, where $i = \sqrt{-1}$. \\





On one hand, $\sqrt{-1}$ feels a little weird.  We are used to squaring real numbers and getting positive values. On the other hand, we have worked with expressions like $5 + 3 \sqrt{2}$ and $3 - 4 \sqrt{7}$ many times.  In this respect, $5 + 3 \sqrt{-1}$ is no different.  $i = \sqrt{-1}$ is simply the number you square to get $-1$.  That's all.



\begin{notation}



We have plenty of shorthand notations.

\begin{itemize}
\item $3 + 1 \, i = 3 + i$
\item $7 + (-2) \, i = 7 - 2 \, i$
\item $4 + 0 \, i = 4$
\item $0 + 5 \, i = 5 \, i$
\item $0 + (-6) \, i = -6 \, i$
\end{itemize}


\end{notation}





Addition and Subtraction are componentwise operations.


\begin{itemize}
	\item $(a + b \, i) + (c + d \, i) = (a+c) + (b+d) \, i$
	\item $(a + b \, i) - (c + d \, i) = (a-c) + (b-d) \, i$
\end{itemize}

Add the real parts. Add the imaginary parts. \\
Subtract the real parts. Subtract the imaginary parts. \\








Multiplication is a little more involved, because we want the Distributive Property.

\[
(a + b \, i) \cdot (c + d \, i) 
\]

\[
a \cdot (c + d \, i) + b \, i \cdot (c + d \, i) 
\]

\[
= a \cdot c + a \cdot d \, i + b \, i \cdot c  + b \, i \cdot d \, i
\]

\[
a c  + (a d) \, i + (b c) \, i  + (b d) \, i^2
\]


\[
a c  + (a d) \, i + (b c) \, i + (b d) \, (-1)
\]


\[
(a c - b d) + (a d + b c) \, i 
\]



\begin{itemize}
	\item $(a + b \, i) \cdot (c + d \, i) = (ac - bd) + (ad + bc) \, i$
\end{itemize}










\begin{question}


\begin{itemize}
\item   $(4 + 5 \, i) + (3 + 7 \, i) = \answer{7} + \answer{12} i$
\item   $(4 + (-5) \, i) + (3 + 7 \, i) = \answer{7} + \answer{2} i$
\item   $(4 + 5 \, i) + (3 - 7 \, i) = \answer{7} + \answer{-2} i = \answer{7} - \answer{2} i$
\item   $(-4 + 5 \, i) + (3 + 7 \, i) = \answer{-1} + \answer{12} i = -\answer{1} + \answer{12} i$
\item   $(4 - 5 \, i) + (-3 + 7 \, i) = \answer{1} + \answer{2} i$
\end{itemize}



\end{question}





\begin{question}


\begin{itemize}
\item   $(4 + 5 \, i) - (3 + 7 \, i) = \answer{1} + \answer{-2} i = \answer{1} - \answer{2} i$
\item   $(4 + (-5) \, i) - (3 + 7 \, i) = \answer{1} + \answer{-12} i = \answer{1} - \answer{12} i$
\item   $(4 + 5 \, i) - (3 - 7 \, i) = \answer{1} + \answer{12} i$
\item   $(-4 + 5 \, i) - (3 + 7 \, i) = \answer{-7} + \answer{-2} i = \answer{-7} - \answer{2} i$
\item   $(4 - 5 \, i) - (-3 + 7 \, i) = \answer{7} + \answer{-12} i = \answer{7} - \answer{12} i$
\end{itemize}



\end{question}











\begin{question}


\begin{itemize}
\item   $(4 + 5 \, i) \cdot (3 + 7 \, i) = \answer{-23} + \answer{43} i$
\item   $(3 - 2 \, i) \cdot (4 + \, i) = \answer{14} + \answer{-5} i = \answer{14} - \answer{5} i$
\end{itemize}



\end{question}








\begin{question}


\begin{itemize}
\item   $(\frac{2}{3} +  \, i) + (3 + \frac{1}{2} \, i) = \answer{\frac{11}{3}} + \answer{\frac{3}{2}} i$
\item   $(\pi + (-5) \, i) + (3 - i) = \answer{3+\pi} + \answer{-6} i$
\item   $(4 + \sqrt{2} \, i) + (3 - \sqrt{2} \, i) = \answer{7}$
\item   $(-4 + \sqrt{3} \, i) + (4 + 7 \, i) = \answer{7 + \sqrt{3}} i$
\end{itemize}



\end{question}





\begin{question}


\item   $(\frac{1}{\sqrt{2}} +  \frac{1}{\sqrt{2}} \, i) \cdot (\frac{1}{\sqrt{2}} + \frac{1}{\sqrt{2}} \, i) = \answer{0} + \answer{1} i$


\end{question}










\begin{question}


\item   $(a +  b \, i) + (a - b \, i) = \answer{2a}$


\end{question}






\begin{question}


\item   $(b \, i) \cdot  (-b \, i) = \answer{b^2}$


\end{question}






We can see that our usual operations on real numbers still hold.

\begin{itemize}
\item $(r + 0 \, i) + (t + 0 \, i) = (r + t) + 0 \, i = r + t$ \\ 
\item $(r + 0 \, i) - (t + 0 \, i) = (r - t) + 0 \, i = r - t$ \\ 
\item $(r + 0 \, i) \cdot (t + 0 \, i) = (r \cdot t - 0 \cdot 0) + (r \cdot 0 + t \cdot 0) \, i = r \cdot t$ \\ 
\end{itemize}


Real numbers combine to still give real numbers.  It is also possible for complex numbers to combine to give real numbers. \\


One way is to add complex numbers with opposite imaginary parts:

\[
(a + b \, i) + (a - b \, i) = 2a
\]



Another way is to multiply imaginary numbers:

\[
(b \, i) \cdot (-b \, i) = b^2
\]



This is what we have been experiencing with quadratic functions, but just couldn't see it.  We have had numbers outside the real numbers combining and the result landing inside the numbers.




This relation involving imaginary parts of opposite sign is proving to be very important.  It deserves a name.



\begin{definition} \textbf{\textcolor{green!50!black}{Complex Conjugate}}

If $z = a + b \, i$ is a complex number, then $\bar{z} = \overline{a + b \, i} = a - b \, i$ is called the \textbf{complex conjugate} of $z = a + b \, i$.

\end{definition}
















\begin{center}
\textbf{\textcolor{green!50!black}{ooooo-=-=-=-ooOoo-=-=-=-ooooo}} \\

more examples can be found by following this link\\ \link[More Examples of Complex Arithmetic]{https://ximera.osu.edu/csccmathematics/precalculus2/precalculus2/complexArithmetic/examples/exampleList}

\end{center}




\end{document}
