\documentclass{ximera}

%\usepackage{todonotes}

\newcommand{\todo}{}

\usepackage{esint} % for \oiint
\ifxake%%https://math.meta.stackexchange.com/questions/9973/how-do-you-render-a-closed-surface-double-integral
\renewcommand{\oiint}{{\large\bigcirc}\kern-1.56em\iint}
\fi


\graphicspath{
  {./}
  {ximeraTutorial/}
  {basicPhilosophy/}
  {functionsOfSeveralVariables/}
  {normalVectors/}
  {lagrangeMultipliers/}
  {vectorFields/}
  {greensTheorem/}
  {shapeOfThingsToCome/}
  {dotProducts/}
  {partialDerivativesAndTheGradientVector/}
  {../productAndQuotientRules/exercises/}
  {../normalVectors/exercisesParametricPlots/}
  {../continuityOfFunctionsOfSeveralVariables/exercises/}
  {../partialDerivativesAndTheGradientVector/exercises/}
  {../directionalDerivativeAndChainRule/exercises/}
  {../commonCoordinates/exercisesCylindricalCoordinates/}
  {../commonCoordinates/exercisesSphericalCoordinates/}
  {../greensTheorem/exercisesCurlAndLineIntegrals/}
  {../greensTheorem/exercisesDivergenceAndLineIntegrals/}
  {../shapeOfThingsToCome/exercisesDivergenceTheorem/}
  {../greensTheorem/}
  {../shapeOfThingsToCome/}
  {../separableDifferentialEquations/exercises/}
  {vectorFields/}
}

\newcommand{\mooculus}{\textsf{\textbf{MOOC}\textnormal{\textsf{ULUS}}}}

\usepackage{tkz-euclide}
\usepackage{tikz}
\usepackage{tikz-cd}
\usetikzlibrary{arrows}
\tikzset{>=stealth,commutative diagrams/.cd,
  arrow style=tikz,diagrams={>=stealth}} %% cool arrow head
\tikzset{shorten <>/.style={ shorten >=#1, shorten <=#1 } } %% allows shorter vectors

\usetikzlibrary{backgrounds} %% for boxes around graphs
\usetikzlibrary{shapes,positioning}  %% Clouds and stars
\usetikzlibrary{matrix} %% for matrix
\usepgfplotslibrary{polar} %% for polar plots
\usepgfplotslibrary{fillbetween} %% to shade area between curves in TikZ
%\usetkzobj{all}
\usepackage[makeroom]{cancel} %% for strike outs
%\usepackage{mathtools} %% for pretty underbrace % Breaks Ximera
%\usepackage{multicol}
\usepackage{pgffor} %% required for integral for loops



%% http://tex.stackexchange.com/questions/66490/drawing-a-tikz-arc-specifying-the-center
%% Draws beach ball
\tikzset{pics/carc/.style args={#1:#2:#3}{code={\draw[pic actions] (#1:#3) arc(#1:#2:#3);}}}



\usepackage{array}
\setlength{\extrarowheight}{+.1cm}
\newdimen\digitwidth
\settowidth\digitwidth{9}
\def\divrule#1#2{
\noalign{\moveright#1\digitwidth
\vbox{\hrule width#2\digitwidth}}}




% \newcommand{\RR}{\mathbb R}
% \newcommand{\R}{\mathbb R}
% \newcommand{\N}{\mathbb N}
% \newcommand{\Z}{\mathbb Z}

\newcommand{\sagemath}{\textsf{SageMath}}


%\renewcommand{\d}{\,d\!}
%\renewcommand{\d}{\mathop{}\!d}
%\newcommand{\dd}[2][]{\frac{\d #1}{\d #2}}
%\newcommand{\pp}[2][]{\frac{\partial #1}{\partial #2}}
% \renewcommand{\l}{\ell}
%\newcommand{\ddx}{\frac{d}{\d x}}

% \newcommand{\zeroOverZero}{\ensuremath{\boldsymbol{\tfrac{0}{0}}}}
%\newcommand{\inftyOverInfty}{\ensuremath{\boldsymbol{\tfrac{\infty}{\infty}}}}
%\newcommand{\zeroOverInfty}{\ensuremath{\boldsymbol{\tfrac{0}{\infty}}}}
%\newcommand{\zeroTimesInfty}{\ensuremath{\small\boldsymbol{0\cdot \infty}}}
%\newcommand{\inftyMinusInfty}{\ensuremath{\small\boldsymbol{\infty - \infty}}}
%\newcommand{\oneToInfty}{\ensuremath{\boldsymbol{1^\infty}}}
%\newcommand{\zeroToZero}{\ensuremath{\boldsymbol{0^0}}}
%\newcommand{\inftyToZero}{\ensuremath{\boldsymbol{\infty^0}}}



% \newcommand{\numOverZero}{\ensuremath{\boldsymbol{\tfrac{\#}{0}}}}
% \newcommand{\dfn}{\textbf}
% \newcommand{\unit}{\,\mathrm}
% \newcommand{\unit}{\mathop{}\!\mathrm}
% \newcommand{\eval}[1]{\bigg[ #1 \bigg]}
% \newcommand{\seq}[1]{\left( #1 \right)}
% \renewcommand{\epsilon}{\varepsilon}
% \renewcommand{\phi}{\varphi}


% \renewcommand{\iff}{\Leftrightarrow}

% \DeclareMathOperator{\arccot}{arccot}
% \DeclareMathOperator{\arcsec}{arcsec}
% \DeclareMathOperator{\arccsc}{arccsc}
% \DeclareMathOperator{\si}{Si}
% \DeclareMathOperator{\scal}{scal}
% \DeclareMathOperator{\sign}{sign}


%% \newcommand{\tightoverset}[2]{% for arrow vec
%%   \mathop{#2}\limits^{\vbox to -.5ex{\kern-0.75ex\hbox{$#1$}\vss}}}
% \newcommand{\arrowvec}[1]{{\overset{\rightharpoonup}{#1}}}
% \renewcommand{\vec}[1]{\arrowvec{\mathbf{#1}}}
% \renewcommand{\vec}[1]{{\overset{\boldsymbol{\rightharpoonup}}{\mathbf{#1}}}}

% \newcommand{\point}[1]{\left(#1\right)} %this allows \vector{ to be changed to \vector{ with a quick find and replace
% \newcommand{\pt}[1]{\mathbf{#1}} %this allows \vec{ to be changed to \vec{ with a quick find and replace
% \newcommand{\Lim}[2]{\lim_{\point{#1} \to \point{#2}}} %Bart, I changed this to point since I want to use it.  It runs through both of the exercise and exerciseE files in limits section, which is why it was in each document to start with.

% \DeclareMathOperator{\proj}{\mathbf{proj}}
% \newcommand{\veci}{{\boldsymbol{\hat{\imath}}}}
% \newcommand{\vecj}{{\boldsymbol{\hat{\jmath}}}}
% \newcommand{\veck}{{\boldsymbol{\hat{k}}}}
% \newcommand{\vecl}{\vec{\boldsymbol{\l}}}
% \newcommand{\uvec}[1]{\mathbf{\hat{#1}}}
% \newcommand{\utan}{\mathbf{\hat{t}}}
% \newcommand{\unormal}{\mathbf{\hat{n}}}
% \newcommand{\ubinormal}{\mathbf{\hat{b}}}

% \newcommand{\dotp}{\bullet}
% \newcommand{\cross}{\boldsymbol\times}
% \newcommand{\grad}{\boldsymbol\nabla}
% \newcommand{\divergence}{\grad\dotp}
% \newcommand{\curl}{\grad\cross}
%\DeclareMathOperator{\divergence}{divergence}
%\DeclareMathOperator{\curl}[1]{\grad\cross #1}
% \newcommand{\lto}{\mathop{\longrightarrow\,}\limits}

% \renewcommand{\bar}{\overline}

\colorlet{textColor}{black}
\colorlet{background}{white}
\colorlet{penColor}{blue!50!black} % Color of a curve in a plot
\colorlet{penColor2}{red!50!black}% Color of a curve in a plot
\colorlet{penColor3}{red!50!blue} % Color of a curve in a plot
\colorlet{penColor4}{green!50!black} % Color of a curve in a plot
\colorlet{penColor5}{orange!80!black} % Color of a curve in a plot
\colorlet{penColor6}{yellow!70!black} % Color of a curve in a plot
\colorlet{fill1}{penColor!20} % Color of fill in a plot
\colorlet{fill2}{penColor2!20} % Color of fill in a plot
\colorlet{fillp}{fill1} % Color of positive area
\colorlet{filln}{penColor2!20} % Color of negative area
\colorlet{fill3}{penColor3!20} % Fill
\colorlet{fill4}{penColor4!20} % Fill
\colorlet{fill5}{penColor5!20} % Fill
\colorlet{gridColor}{gray!50} % Color of grid in a plot

\newcommand{\surfaceColor}{violet}
\newcommand{\surfaceColorTwo}{redyellow}
\newcommand{\sliceColor}{greenyellow}




\pgfmathdeclarefunction{gauss}{2}{% gives gaussian
  \pgfmathparse{1/(#2*sqrt(2*pi))*exp(-((x-#1)^2)/(2*#2^2))}%
}


%%%%%%%%%%%%%
%% Vectors
%%%%%%%%%%%%%

%% Simple horiz vectors
\renewcommand{\vector}[1]{\left\langle #1\right\rangle}


%% %% Complex Horiz Vectors with angle brackets
%% \makeatletter
%% \renewcommand{\vector}[2][ , ]{\left\langle%
%%   \def\nextitem{\def\nextitem{#1}}%
%%   \@for \el:=#2\do{\nextitem\el}\right\rangle%
%% }
%% \makeatother

%% %% Vertical Vectors
%% \def\vector#1{\begin{bmatrix}\vecListA#1,,\end{bmatrix}}
%% \def\vecListA#1,{\if,#1,\else #1\cr \expandafter \vecListA \fi}

%%%%%%%%%%%%%
%% End of vectors
%%%%%%%%%%%%%

%\newcommand{\fullwidth}{}
%\newcommand{\normalwidth}{}



%% makes a snazzy t-chart for evaluating functions
%\newenvironment{tchart}{\rowcolors{2}{}{background!90!textColor}\array}{\endarray}

%%This is to help with formatting on future title pages.
\newenvironment{sectionOutcomes}{}{}



%% Flowchart stuff
%\tikzstyle{startstop} = [rectangle, rounded corners, minimum width=3cm, minimum height=1cm,text centered, draw=black]
%\tikzstyle{question} = [rectangle, minimum width=3cm, minimum height=1cm, text centered, draw=black]
%\tikzstyle{decision} = [trapezium, trapezium left angle=70, trapezium right angle=110, minimum width=3cm, minimum height=1cm, text centered, draw=black]
%\tikzstyle{question} = [rectangle, rounded corners, minimum width=3cm, minimum height=1cm,text centered, draw=black]
%\tikzstyle{process} = [rectangle, minimum width=3cm, minimum height=1cm, text centered, draw=black]
%\tikzstyle{decision} = [trapezium, trapezium left angle=70, trapezium right angle=110, minimum width=3cm, minimum height=1cm, text centered, draw=black]


\title{Inverse}

\begin{document}

\begin{abstract}
too many directions now
\end{abstract}
\maketitle





We know that every nonzero Complex number can be written as the product of a positive real number (a scalar) and a Complex number on the unit circle.  Euler's Formula tells us how each Complex number on the unit circle can be written as a Complex exponential. \\




$\blacktriangleright$ \textbf{\textcolor{purple!85!blue}{Euler's Formula}}   


Every complex number on the unit circle can be written in the form


\[   e^{i \theta} = \cos(\theta) + i \sin(\theta)         \]


And, we already know how to write any real number in exponential form: $r = e^{\ln(r)}$.  





$\blacktriangleright$ \textbf{\textcolor{purple!85!blue}{Complex Numbers}}   




Combining these together, we get that every nonzero complex number can be written in the form  
\[
r \cdot (\cos(\theta) + i \sin(\theta))  = e^{\ln(r)} \cdot e^{i \, \theta} = e^{\ln(r) + i \, \theta}
\]



If $z = a + b \, i$, then $r = \sqrt{a^2 + b^2}$ and $\theta$ is the counterclockwise angle from the positive real-axis to $z$.




\begin{center}

\textbf{\textcolor{red!80!black}{EVERY complex number can be written in the form $e^{\ln(r) + i \, \theta}$}}

\end{center}




\begin{example} Complex Logarithm


Let $z = -3 - 3 \, i$.


$|z| = \sqrt{(-3)^2 + (-3)^2} =  \sqrt{18} = 3 \sqrt{2}$


$\arctan\left(\frac{-3}{-3}\right) = \frac{\pi}{4}$.  This reference angle gives us $\theta = \frac{5 \pi}{4}$, since $z$ is in the third quadrant.




$z = -3 - 3 \, i =  e^{\ln(3 \sqrt{2}) + \left( \tfrac{5 \pi}{4} \pm 2 k \pi \right) \, i}     \, \text{ where } \, k \in \mathbb{N}  $


There are many complex logarithms for $-3 - 3 \, i$. One for each turn around the unit circle.



\end{example}




There are an infinite number of complex logarithms for each complex number.



\textbf{\textcolor{red!80!black}{Whoops: }}   Just like when we were creating the inverse trigonometric functions, this Complex logarithm is not following our one and only rule for functions.  

We will fix this in the same waythat we fixed $\arcsin$ and $\arccos$.  We'll pick a range. \\





Following the idea behind our range restrictions for $\arcsin$ and $\arccos$, we pick a \textbf{principal value} for the complex logarithm. Since we would like to keep the positive real axis as clean as possible, we chooose the interval $(-\pi, \pi]$ for the principal value of the complex logarithm.

This interval is known as the \textbf{principal branch} of the complex logarithm.  The negative real axis is called a \textbf{branch cut} for the complex logarithm. It is a cut, because the angle changes abruptly from $\pi$ to $-\pi$.





\begin{definition}   \textbf{\textcolor{green!50!black}{Complex Logarithm}}
 
If $z = a + b \, i$, then 

\[   \ln(z) = \ln(\sqrt{a^2 + b^2}) + \theta \, i       \]



where $\theta \in (-\pi, \pi]$ and is the counterclockwise angle from the positive real axis to $z$.



This $\theta$ is often referred to as the \textbf{argument} of $z$, $Arg(z)$.



\end{definition}




Since $Arg(r) = 0$ for any positive real number, our complex logarithm will agree with our real logarithm.








\begin{example} Some Logarithms


\begin{itemize}
\item $\ln(5) = \ln(5) + 0 \, i = \ln(5)$

\item $\ln(2i) = \ln(2) + \frac{\pi}{2} \, i$

\item $\ln(-5 + 5 \, i) = \ln(5 \sqrt{2}) + \frac{3\pi}{4}\, i$

\item $\ln(-1) = \pi \, i$



\end{itemize}




\end{example}







We are bringing over all of our exponent arithmetic.



\begin{example}  $i^i$


\[    i^i =  e^{\ln(i^i)} = e^{i \cdot \ln(i)} = e^{ i \cdot \tfrac{\pi}{2} \, i}   = e^{-\tfrac{\pi}{2}} \approx  0.2078795764\]



\begin{center}

$i^i$ is a real number!
\end{center}


\end{example}




























\begin{center}
\textbf{\textcolor{green!50!black}{ooooo-=-=-=-ooOoo-=-=-=-ooooo}} \\

more examples can be found by following this link\\ \link[More Examples of Complex Logarithms]{https://ximera.osu.edu/csccmathematics/precalculus2/precalculus2/complexLogarithm/examples/exampleList}

\end{center}






\end{document}
