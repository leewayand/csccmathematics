\documentclass{ximera}

%\usepackage{todonotes}

\newcommand{\todo}{}

\usepackage{esint} % for \oiint
\ifxake%%https://math.meta.stackexchange.com/questions/9973/how-do-you-render-a-closed-surface-double-integral
\renewcommand{\oiint}{{\large\bigcirc}\kern-1.56em\iint}
\fi


\graphicspath{
  {./}
  {ximeraTutorial/}
  {basicPhilosophy/}
  {functionsOfSeveralVariables/}
  {normalVectors/}
  {lagrangeMultipliers/}
  {vectorFields/}
  {greensTheorem/}
  {shapeOfThingsToCome/}
  {dotProducts/}
  {partialDerivativesAndTheGradientVector/}
  {../productAndQuotientRules/exercises/}
  {../normalVectors/exercisesParametricPlots/}
  {../continuityOfFunctionsOfSeveralVariables/exercises/}
  {../partialDerivativesAndTheGradientVector/exercises/}
  {../directionalDerivativeAndChainRule/exercises/}
  {../commonCoordinates/exercisesCylindricalCoordinates/}
  {../commonCoordinates/exercisesSphericalCoordinates/}
  {../greensTheorem/exercisesCurlAndLineIntegrals/}
  {../greensTheorem/exercisesDivergenceAndLineIntegrals/}
  {../shapeOfThingsToCome/exercisesDivergenceTheorem/}
  {../greensTheorem/}
  {../shapeOfThingsToCome/}
  {../separableDifferentialEquations/exercises/}
  {vectorFields/}
}

\newcommand{\mooculus}{\textsf{\textbf{MOOC}\textnormal{\textsf{ULUS}}}}

\usepackage{tkz-euclide}
\usepackage{tikz}
\usepackage{tikz-cd}
\usetikzlibrary{arrows}
\tikzset{>=stealth,commutative diagrams/.cd,
  arrow style=tikz,diagrams={>=stealth}} %% cool arrow head
\tikzset{shorten <>/.style={ shorten >=#1, shorten <=#1 } } %% allows shorter vectors

\usetikzlibrary{backgrounds} %% for boxes around graphs
\usetikzlibrary{shapes,positioning}  %% Clouds and stars
\usetikzlibrary{matrix} %% for matrix
\usepgfplotslibrary{polar} %% for polar plots
\usepgfplotslibrary{fillbetween} %% to shade area between curves in TikZ
%\usetkzobj{all}
\usepackage[makeroom]{cancel} %% for strike outs
%\usepackage{mathtools} %% for pretty underbrace % Breaks Ximera
%\usepackage{multicol}
\usepackage{pgffor} %% required for integral for loops



%% http://tex.stackexchange.com/questions/66490/drawing-a-tikz-arc-specifying-the-center
%% Draws beach ball
\tikzset{pics/carc/.style args={#1:#2:#3}{code={\draw[pic actions] (#1:#3) arc(#1:#2:#3);}}}



\usepackage{array}
\setlength{\extrarowheight}{+.1cm}
\newdimen\digitwidth
\settowidth\digitwidth{9}
\def\divrule#1#2{
\noalign{\moveright#1\digitwidth
\vbox{\hrule width#2\digitwidth}}}




% \newcommand{\RR}{\mathbb R}
% \newcommand{\R}{\mathbb R}
% \newcommand{\N}{\mathbb N}
% \newcommand{\Z}{\mathbb Z}

\newcommand{\sagemath}{\textsf{SageMath}}


%\renewcommand{\d}{\,d\!}
%\renewcommand{\d}{\mathop{}\!d}
%\newcommand{\dd}[2][]{\frac{\d #1}{\d #2}}
%\newcommand{\pp}[2][]{\frac{\partial #1}{\partial #2}}
% \renewcommand{\l}{\ell}
%\newcommand{\ddx}{\frac{d}{\d x}}

% \newcommand{\zeroOverZero}{\ensuremath{\boldsymbol{\tfrac{0}{0}}}}
%\newcommand{\inftyOverInfty}{\ensuremath{\boldsymbol{\tfrac{\infty}{\infty}}}}
%\newcommand{\zeroOverInfty}{\ensuremath{\boldsymbol{\tfrac{0}{\infty}}}}
%\newcommand{\zeroTimesInfty}{\ensuremath{\small\boldsymbol{0\cdot \infty}}}
%\newcommand{\inftyMinusInfty}{\ensuremath{\small\boldsymbol{\infty - \infty}}}
%\newcommand{\oneToInfty}{\ensuremath{\boldsymbol{1^\infty}}}
%\newcommand{\zeroToZero}{\ensuremath{\boldsymbol{0^0}}}
%\newcommand{\inftyToZero}{\ensuremath{\boldsymbol{\infty^0}}}



% \newcommand{\numOverZero}{\ensuremath{\boldsymbol{\tfrac{\#}{0}}}}
% \newcommand{\dfn}{\textbf}
% \newcommand{\unit}{\,\mathrm}
% \newcommand{\unit}{\mathop{}\!\mathrm}
% \newcommand{\eval}[1]{\bigg[ #1 \bigg]}
% \newcommand{\seq}[1]{\left( #1 \right)}
% \renewcommand{\epsilon}{\varepsilon}
% \renewcommand{\phi}{\varphi}


% \renewcommand{\iff}{\Leftrightarrow}

% \DeclareMathOperator{\arccot}{arccot}
% \DeclareMathOperator{\arcsec}{arcsec}
% \DeclareMathOperator{\arccsc}{arccsc}
% \DeclareMathOperator{\si}{Si}
% \DeclareMathOperator{\scal}{scal}
% \DeclareMathOperator{\sign}{sign}


%% \newcommand{\tightoverset}[2]{% for arrow vec
%%   \mathop{#2}\limits^{\vbox to -.5ex{\kern-0.75ex\hbox{$#1$}\vss}}}
% \newcommand{\arrowvec}[1]{{\overset{\rightharpoonup}{#1}}}
% \renewcommand{\vec}[1]{\arrowvec{\mathbf{#1}}}
% \renewcommand{\vec}[1]{{\overset{\boldsymbol{\rightharpoonup}}{\mathbf{#1}}}}

% \newcommand{\point}[1]{\left(#1\right)} %this allows \vector{ to be changed to \vector{ with a quick find and replace
% \newcommand{\pt}[1]{\mathbf{#1}} %this allows \vec{ to be changed to \vec{ with a quick find and replace
% \newcommand{\Lim}[2]{\lim_{\point{#1} \to \point{#2}}} %Bart, I changed this to point since I want to use it.  It runs through both of the exercise and exerciseE files in limits section, which is why it was in each document to start with.

% \DeclareMathOperator{\proj}{\mathbf{proj}}
% \newcommand{\veci}{{\boldsymbol{\hat{\imath}}}}
% \newcommand{\vecj}{{\boldsymbol{\hat{\jmath}}}}
% \newcommand{\veck}{{\boldsymbol{\hat{k}}}}
% \newcommand{\vecl}{\vec{\boldsymbol{\l}}}
% \newcommand{\uvec}[1]{\mathbf{\hat{#1}}}
% \newcommand{\utan}{\mathbf{\hat{t}}}
% \newcommand{\unormal}{\mathbf{\hat{n}}}
% \newcommand{\ubinormal}{\mathbf{\hat{b}}}

% \newcommand{\dotp}{\bullet}
% \newcommand{\cross}{\boldsymbol\times}
% \newcommand{\grad}{\boldsymbol\nabla}
% \newcommand{\divergence}{\grad\dotp}
% \newcommand{\curl}{\grad\cross}
%\DeclareMathOperator{\divergence}{divergence}
%\DeclareMathOperator{\curl}[1]{\grad\cross #1}
% \newcommand{\lto}{\mathop{\longrightarrow\,}\limits}

% \renewcommand{\bar}{\overline}

\colorlet{textColor}{black}
\colorlet{background}{white}
\colorlet{penColor}{blue!50!black} % Color of a curve in a plot
\colorlet{penColor2}{red!50!black}% Color of a curve in a plot
\colorlet{penColor3}{red!50!blue} % Color of a curve in a plot
\colorlet{penColor4}{green!50!black} % Color of a curve in a plot
\colorlet{penColor5}{orange!80!black} % Color of a curve in a plot
\colorlet{penColor6}{yellow!70!black} % Color of a curve in a plot
\colorlet{fill1}{penColor!20} % Color of fill in a plot
\colorlet{fill2}{penColor2!20} % Color of fill in a plot
\colorlet{fillp}{fill1} % Color of positive area
\colorlet{filln}{penColor2!20} % Color of negative area
\colorlet{fill3}{penColor3!20} % Fill
\colorlet{fill4}{penColor4!20} % Fill
\colorlet{fill5}{penColor5!20} % Fill
\colorlet{gridColor}{gray!50} % Color of grid in a plot

\newcommand{\surfaceColor}{violet}
\newcommand{\surfaceColorTwo}{redyellow}
\newcommand{\sliceColor}{greenyellow}




\pgfmathdeclarefunction{gauss}{2}{% gives gaussian
  \pgfmathparse{1/(#2*sqrt(2*pi))*exp(-((x-#1)^2)/(2*#2^2))}%
}


%%%%%%%%%%%%%
%% Vectors
%%%%%%%%%%%%%

%% Simple horiz vectors
\renewcommand{\vector}[1]{\left\langle #1\right\rangle}


%% %% Complex Horiz Vectors with angle brackets
%% \makeatletter
%% \renewcommand{\vector}[2][ , ]{\left\langle%
%%   \def\nextitem{\def\nextitem{#1}}%
%%   \@for \el:=#2\do{\nextitem\el}\right\rangle%
%% }
%% \makeatother

%% %% Vertical Vectors
%% \def\vector#1{\begin{bmatrix}\vecListA#1,,\end{bmatrix}}
%% \def\vecListA#1,{\if,#1,\else #1\cr \expandafter \vecListA \fi}

%%%%%%%%%%%%%
%% End of vectors
%%%%%%%%%%%%%

%\newcommand{\fullwidth}{}
%\newcommand{\normalwidth}{}



%% makes a snazzy t-chart for evaluating functions
%\newenvironment{tchart}{\rowcolors{2}{}{background!90!textColor}\array}{\endarray}

%%This is to help with formatting on future title pages.
\newenvironment{sectionOutcomes}{}{}



%% Flowchart stuff
%\tikzstyle{startstop} = [rectangle, rounded corners, minimum width=3cm, minimum height=1cm,text centered, draw=black]
%\tikzstyle{question} = [rectangle, minimum width=3cm, minimum height=1cm, text centered, draw=black]
%\tikzstyle{decision} = [trapezium, trapezium left angle=70, trapezium right angle=110, minimum width=3cm, minimum height=1cm, text centered, draw=black]
%\tikzstyle{question} = [rectangle, rounded corners, minimum width=3cm, minimum height=1cm,text centered, draw=black]
%\tikzstyle{process} = [rectangle, minimum width=3cm, minimum height=1cm, text centered, draw=black]
%\tikzstyle{decision} = [trapezium, trapezium left angle=70, trapezium right angle=110, minimum width=3cm, minimum height=1cm, text centered, draw=black]


\title{Reals}

\begin{document}

\begin{abstract}
linear and quadratic
\end{abstract}
\maketitle




Polynomial functions are functions that can be described with expressions like

\[   f(x) = a_n x^n + a_{n-1} x^{n-1} + \cdots + a_2 x^2 + a_1 x + a_0     \]


\begin{itemize}
\item \textbf{over the complex numbers} means that all of the coefficients are complex numbers:   $a_k \in \mathbb{C}$
\item \textbf{over the real numbers} means that all of the coefficients are real numbers:   $a_k \in \mathbb{R}$
\item \textbf{over the rationals} means that all of the coefficients are rational numbers:   $a_k \in \mathbb{Q}$
\item \textbf{over the integers} means that all of the coefficients are integers:   $a_k \in \mathbb{Z}$
\end{itemize}



Since integers are real numbers and real numbers are complex numbers, the smallest possible set is usually cited for polynomials.



\begin{example}  Over


\begin{itemize}
\item $2i x^2 + 4 x - (2 + 7 \,i)$ is a quadratic polynomial over the complex numbers.

\item $4 t^3 + \pi t^2 + 4 t - \sqrt{2}$ is a cubic polynomial over the real numbers.

\item $ y^4 +  \frac{1}{2} y^2 - 3$ is a quartic polynomial over the rationals.

\item $-3 y^4 +  y^2 + 5$ is a quartic polynomial over the integers.
\end{itemize}




\end{example}






Polynomials with complex coefficients include every possible polynomial.  \\



The point of Algebra, which we will soon see, is the \textbf{Fundamental Theorem of Algebra}. \\



The \textbf{Fundamental Theorem of Algebra} tells us that these polynomials factor completely into a product of linear factors, if we are allowed to use Complex Numbers. There are as many linear factors as the degree of the polynomial.



\[   f(x) = a_n x^n + a_{n-1} x^{n-1} + \cdots + a_2 x^2 + a_1 x + a_0   =   a_n (x - r_1) (x - r_2) \cdots (x - r_n)  \]


However, the subject of polynomials over the Complex numbers is too general for us.  We are narrowing our focus down to polynomials over the reals.




\section*{Polynomials over the Real Numbers}


The polynomials we are considering only have real numbers as coefficients.


\begin{example} Over the Real Numbers



\begin{itemize}
\item $4 x^2 + 2 x -1$
\item $\frac{3}{4} y^5 - y^2 + \pi y$
\item $\sqrt{3} f^8 + 4 f^5 - \frac{1}{\sqrt{5}} f^2 + 0.456 f - \frac{1}{e}$
\item $x^5 + x^4 + x^3 + x^2 + x + 1$
\end{itemize}



\end{example}






Over the reals means that all of the coeffcients are real numbers.



Polynomials over the real numbers can still have complex roots.

\[  x^2 + 1 = (x - i)(x + i)       \]



We have seen that if a complex number, $z$, is a root of a polynomial over the real numbers, then so is its complex conjugate, $\bar{z}$.

We have our first sign of structure.




\begin{example} Roots

Let $w(v)$ be a polynomial over the real numbers.

Suppose $w(v)$ has an odd degree.


According to the Fundamental Theorem of Algebra, $w(v)$ factors into an odd number of linear factors over the complex numbers.

Any factor with a complex root has a partner factor with the complex conjugate root.  They come in pairs.  If we pair up all of the factors with complex roots and their conjugates, then we have accounted for an even number of factors.  

There must be at least one factor of $w(v)$ without a partner.  This factor cannot have a complex root, because then it would have been paired already.  Therefore, this factor has to correspond to a real root.

Therefore, $w(v)$ has a real root.




\end{example}








\begin{theorem} \textbf{\textcolor{blue!55!black}{Odd Degree}} 

A polynomial over the real numbers with an odd degree must have a real root.

Its graph must cross the horizontal axis.

\end{theorem}






\begin{explanation}


A polynomial, $p(x)$, over the real numbers with an odd degree must have a real root. \\




A polynomial function of odd degree is unbounded and its end-behavior depends on the sign of the leading coefficient.



If the leading coefficient is positive, then 

\[   \lim\limits_{x \to -\infty} p(x) = -\infty    \,   \text{ and } \,       \lim\limits_{x \to \infty} p(x) = \infty      \]



If the leading coefficient is negative, then 

\[   \lim\limits_{x \to -\infty} p(x) = \infty    \,   \text{ and } \,       \lim\limits_{x \to \infty} p(x) = -\infty      \]


Either way, one side is unbounded positively and the othere side is unbounded negatively. \\



Polynomial functions are also continuous on the real line. \\


Together, these tell us that range is all real numbers and that $p$ must have a real root.




\end{explanation}








$\blacktriangleright$  \textbf{Irreducible Quadratics}


Let $w(v)$ be a polynomial over the real numbers.

We know that if a complex number, $a + b \, i$, is a root of $w(v)$, then so is its complex conjugate, $a - b \,i$.

If we pair these associated factors of $w(v)$ together and multiply out, then we get a quadratic over the reals.

\[    (v - (a + b \, i)) (v -(a - b \, i))  = v^2 - 2a v + (a^2 + b^2)      \]


This quadratic cannot be factored over the real numbers, because it has complex roots.  It is said to be \textbf{irreducible} over the real numbers, meaning it needs complex numbers to factor.






$\blacktriangleright$  \textbf{Factoring}




Given any polynomial with real coefficients, we can factor it completely over the complex numbers.  Then we can pair up factors with conjugate roots. Multiply those out to get irreducible quadratics over the real numbers.  The other factors all have real roots.





\begin{theorem}  \textbf{\textcolor{blue!55!black}{FTA Over the Real Numbers}} \\

Every polynomial with real coefficients factors into a product of linear factors and irreducible quadratics with real coefficients.



\end{theorem}




































































\begin{center}
\textbf{\textcolor{green!50!black}{ooooo-=-=-=-ooOoo-=-=-=-ooooo}} \\

more examples can be found by following this link\\ \link[More Examples of Zeros]{https://ximera.osu.edu/csccmathematics/precalculus2/precalculus2/solvingEquations/examples/exampleList}

\end{center}







\end{document}
