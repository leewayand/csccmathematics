\documentclass{ximera}


\graphicspath{
  {./}
  {ximeraTutorial/}
  {basicPhilosophy/}
}

\newcommand{\mooculus}{\textsf{\textbf{MOOC}\textnormal{\textsf{ULUS}}}}


\usepackage{tkz-euclide}\usepackage{tikz}
\usepackage{tikz-cd}
\usetikzlibrary{arrows}
\tikzset{>=stealth,commutative diagrams/.cd,
  arrow style=tikz,diagrams={>=stealth}} %% cool arrow head
\tikzset{shorten <>/.style={ shorten >=#1, shorten <=#1 } } %% allows shorter vectors

\usetikzlibrary{backgrounds} %% for boxes around graphs
\usetikzlibrary{shapes,positioning}  %% Clouds and stars
\usetikzlibrary{matrix} %% for matrix
\usepgfplotslibrary{polar} %% for polar plots
\usepgfplotslibrary{fillbetween} %% to shade area between curves in TikZ
\usetkzobj{all}
\usepackage[makeroom]{cancel} %% for strike outs
%\usepackage{mathtools} %% for pretty underbrace % Breaks Ximera
%\usepackage{multicol}
\usepackage{pgffor} %% required for integral for loops



%% http://tex.stackexchange.com/questions/66490/drawing-a-tikz-arc-specifying-the-center
%% Draws beach ball
\tikzset{pics/carc/.style args={#1:#2:#3}{code={\draw[pic actions] (#1:#3) arc(#1:#2:#3);}}}



\usepackage{array}
\setlength{\extrarowheight}{+.1cm}
\newdimen\digitwidth
\settowidth\digitwidth{9}
\def\divrule#1#2{
\noalign{\moveright#1\digitwidth
\vbox{\hrule width#2\digitwidth}}}
























%%This is to help with formatting on future title pages.
\newenvironment{sectionOutcomes}{}{}


\title{More Zeros}

\begin{document}

\begin{abstract}
%
\end{abstract}
\maketitle




We are developing some skill at analyzing functions.  The big picture is that you'll find yourself in some situation involving quantitative information, perhaps data.  You'll create a function that models some measurement of the situation.  You'll examine the function and describe conclusions of the model.  Then, you'll take these conclusions back to the real situation and see if they help you understand the situation better.




A considerable amount of your efforts will not be on how much you have, but how is it growing.  We understand the world in terms of how it is changing.  That means understanding rates of change.  Our tool for this is called the derivative.




Understanding the growth of a situation largely comes down to locating the zeros of the derivative.


Zeros of functions are at the top of our to-do list!  

And, zeros also means factors.  


We have also discovered that the real numbers don't contain all of the zeros for our functions.  


We need to understand complex zeros and how they help us factor.  


\begin{center}
\textbf{\textcolor{red!80!black}{How do we use multiplication to break apart functions?}}
\end{center} 





















\subsection*{Learning Outcomes}


\begin{sectionOutcomes}
In this section, students will 

\begin{itemize}
\item examine the relationships between complex zeros.
\item examine zeros and factoring.
\end{itemize}
\end{sectionOutcomes}










\begin{center}
\textbf{\textcolor{green!50!black}{ooooo-=-=-=-ooOoo-=-=-=-ooooo}} \\

more examples can be found by following this link\\ \link[More Examples of Zeros]{https://ximera.osu.edu/csccmathematics/precalculus2/precalculus2/solvingEquations/examples/exampleList}

\end{center}



\end{document}
