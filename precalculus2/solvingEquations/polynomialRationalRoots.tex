\documentclass{ximera}


\graphicspath{
  {./}
  {ximeraTutorial/}
  {basicPhilosophy/}
}

\newcommand{\mooculus}{\textsf{\textbf{MOOC}\textnormal{\textsf{ULUS}}}}


\usepackage{tkz-euclide}\usepackage{tikz}
\usepackage{tikz-cd}
\usetikzlibrary{arrows}
\tikzset{>=stealth,commutative diagrams/.cd,
  arrow style=tikz,diagrams={>=stealth}} %% cool arrow head
\tikzset{shorten <>/.style={ shorten >=#1, shorten <=#1 } } %% allows shorter vectors

\usetikzlibrary{backgrounds} %% for boxes around graphs
\usetikzlibrary{shapes,positioning}  %% Clouds and stars
\usetikzlibrary{matrix} %% for matrix
\usepgfplotslibrary{polar} %% for polar plots
\usepgfplotslibrary{fillbetween} %% to shade area between curves in TikZ
\usetkzobj{all}
\usepackage[makeroom]{cancel} %% for strike outs
%\usepackage{mathtools} %% for pretty underbrace % Breaks Ximera
%\usepackage{multicol}
\usepackage{pgffor} %% required for integral for loops



%% http://tex.stackexchange.com/questions/66490/drawing-a-tikz-arc-specifying-the-center
%% Draws beach ball
\tikzset{pics/carc/.style args={#1:#2:#3}{code={\draw[pic actions] (#1:#3) arc(#1:#2:#3);}}}



\usepackage{array}
\setlength{\extrarowheight}{+.1cm}
\newdimen\digitwidth
\settowidth\digitwidth{9}
\def\divrule#1#2{
\noalign{\moveright#1\digitwidth
\vbox{\hrule width#2\digitwidth}}}
























%%This is to help with formatting on future title pages.
\newenvironment{sectionOutcomes}{}{}


\title{Polynomials}

\begin{document}

\begin{abstract}
rational roots
\end{abstract}
\maketitle




Let $f(x)$ be a polynomial with rational coefficients.

\[   f(x) = b_n x^n + b_{n-1} x^{n-1} + \cdots + b_2 x^2 + b_1 x + b_0     \]


where $b_k \in \mathbb{Q}$:  $b_k = \frac{n_k}{d_k}$


Then we could get a common denominator of all of the coefficients and factor that out front



\[   f(x) = \frac{1}{d}(c_n x^n + c_{n-1} x^{n-1} + \cdots + c_2 x^2 + c_1 x + c_0)    \]


These are the same polynomial, so that have the same roots and factorization.


So, when we are investigating zeros and roots and factors, we can always assume that a polynomial with rational coefficients just has integer coefficents.




\section{Rational Roots}

Let $f(x)$ be a polynomial with integer coefficients.

\[   f(x) = a_n x^n + a_{n-1} x^{n-1} + \cdots + a_2 x^2 + a_1 x + a_0     \]


Suppose $f(x)$ has a rational root:  $\frac{N}{D}$ in reduced form.  That is $N$ and $D$ do not share any prime factors.



That means


\[    f \left( \frac{N}{D} \right) = a_n \left( \frac{N}{D} \right)^n + a_{n-1} \left( \frac{N}{D} \right)^{n-1} + \cdots + a_2 \left( \frac{N}{D} \right)^2 + a_1 \left( \frac{N}{D} \right) + a_0  = 0       \]




































\end{document}
