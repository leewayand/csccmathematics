\documentclass{ximera}


\graphicspath{
  {./}
  {ximeraTutorial/}
  {basicPhilosophy/}
}

\newcommand{\mooculus}{\textsf{\textbf{MOOC}\textnormal{\textsf{ULUS}}}}


\usepackage{tkz-euclide}\usepackage{tikz}
\usepackage{tikz-cd}
\usetikzlibrary{arrows}
\tikzset{>=stealth,commutative diagrams/.cd,
  arrow style=tikz,diagrams={>=stealth}} %% cool arrow head
\tikzset{shorten <>/.style={ shorten >=#1, shorten <=#1 } } %% allows shorter vectors

\usetikzlibrary{backgrounds} %% for boxes around graphs
\usetikzlibrary{shapes,positioning}  %% Clouds and stars
\usetikzlibrary{matrix} %% for matrix
\usepgfplotslibrary{polar} %% for polar plots
\usepgfplotslibrary{fillbetween} %% to shade area between curves in TikZ
\usetkzobj{all}
\usepackage[makeroom]{cancel} %% for strike outs
%\usepackage{mathtools} %% for pretty underbrace % Breaks Ximera
%\usepackage{multicol}
\usepackage{pgffor} %% required for integral for loops



%% http://tex.stackexchange.com/questions/66490/drawing-a-tikz-arc-specifying-the-center
%% Draws beach ball
\tikzset{pics/carc/.style args={#1:#2:#3}{code={\draw[pic actions] (#1:#3) arc(#1:#2:#3);}}}



\usepackage{array}
\setlength{\extrarowheight}{+.1cm}
\newdimen\digitwidth
\settowidth\digitwidth{9}
\def\divrule#1#2{
\noalign{\moveright#1\digitwidth
\vbox{\hrule width#2\digitwidth}}}
























%%This is to help with formatting on future title pages.
\newenvironment{sectionOutcomes}{}{}


\title{Pairs}

\begin{document}

\begin{abstract}
conjugate roots
\end{abstract}
\maketitle







Before discussing polynomials, we need some knowledge about complex conjugates.




\begin{definition} \textbf{\textcolor{green!50!black}{Complex Conjugate}} \\

Let $z$ be a complex number.  Therefore, $z = a + b \, i$, where $a$ and $b$ are real numbers.



The \textbf{complex conjugate of $z$}, denoted by $\bar{z}$, is given by $\bar{z} = a - b \, i$.



\end{definition}








\begin{definition}   \textbf{\textcolor{green!50!black}{Absolute Value}} \\

The absolute value of $z = a + b \, i$ is given by $|z| = a^2 + b^2$. 


\end{definition}









\begin{definition}   \textbf{\textcolor{green!50!black}{Modulus}} \\

The modulus of $z = a + b \, i$ is given by $\sqrt{|z|} = \sqrt{a^2 + b^2}$. 


\end{definition}

The modulus gives the distance between the complex number and the origin on the Cartesian plane. \\



$\blacktriangleright$ \textbf{Absolute Value}  \\

The absolute value of $z$ is also the product of $z$ and its conjugate.


\begin{explanation}

\begin{align*}
z \cdot \bar{z} &= (a + b \, i) \cdot (a - b \, i) \\
                &= a^2 + ab \, i - ab \, i - b^2 \, i^2 \\
                &= a^2 - b^2 (-1) \\
                &= a^2 + b^2  \\
                &= |z|  
\end{align*}


\end{explanation}






We have seen that each complex number can be represented in different forms.  $a + b \, i$ is the rectangular form, which corresponds to the Cartesian plane and the vector $\langle a, b\rangle$.  We have also seen a polar form $(r, \theta)$.  This lead us to the idea of a unit vector or direction vector and length factor, $r$.

\[ z = a + b \, i = r (\cos(\theta) + i \sin(\theta))  \]


$\cos(\theta) + i \sin(\theta)$ being the rectangular representation of a complex number on the unit circle and $r$ stretching this out to $z$.






$\blacktriangleright$ \textbf{Polar Form}

If $z = r (\cos(\theta) + i \sin(\theta))$, then $\bar{z} = r (\cos(-\theta) + i \sin(-\theta))$, because negating the imaginary part just causes the angle to move in the opposite direction.

$\bar{z} = (r, -\theta)$




$\blacktriangleright$ \textbf{Powers}



We have seen that in polar form, multiplication of complex numbers follow a nice pattern.


\[   (r_1, \theta_1) \cdot (r_2, \theta_2) = (r_1 \cdot r_2, \theta_1 + \theta_2)                \]



This tells us that 


\[   (r, \theta)^2 =  (r, \theta) \cdot (r, \theta) = (r^2, 2\theta)                \]



We can keep multiplying for any whole number power to get 


\[   z^n = (r, \theta)^n =  (r, \theta) \cdot (r, \theta) \cdots (r, \theta)= (r^n, n\theta)                \]




\[   (\bar{z})^n = (r, -\theta)^n =  (r, -\theta) \cdot (r, -\theta) \cdots (r, -\theta)= (r^n, -n\theta)  = \overline{z^n}              \]









$\blacktriangleright$ \textbf{Scalar Multiplication}


Let $z = a + b \, i$  and $r \in \mathbb{R}$. \\

Then $r \cdot \bar{z} = r(a - b \, i) = r \cdot a - r \cdot b \, i = \overline{r \cdot z}$







$\blacktriangleright$ \textbf{Addition}


Let $z = a + b \, i$ and $w = c + d \, i$.  Then we get


\begin{align*}
\overline{z + w} & = \overline{(a + b \, i) + (c + d \, i)}  \\
                & = \overline{(a + c) + (b + d) \, i}   \\
                & = (a + c) - (b + d) \, i  \\
                & = (a - b \, i) + (c - d \, i)   \\
                & = \overline{(a + b \, i)} + \overline{(c + d \, i)}  \\
                & = \overline{z} + \overline{w}
\end{align*}























\section{Complex Conjugate Pairs}




\begin{theorem} \textbf{\textcolor{blue!55!black}{Conjugate Roots}}   \\



Let $p(x)$ be a polynomial with real coefficients.

$p(x) = a_n x^n + a_{n-a} x^{n-1} + \cdots a_2 x^2 + a_1 x + a_0$   with $a_k \in \mathbb{R}$

Suppose $z \in \mathbb{C}$ is a complex root of $p(x)$, then $\bar{z}$ is also a complex root of $p(x)$.


\end{theorem}







$\blacktriangleright$ \textbf{Reasoning}




If $z$ is a root of $p(x)$, then $p(z) = 0$.


\[    p(z) = a_n z^n + a_{n-a} z^{n-1} + \cdots a_2 z^2 + a_1 z + a_0  = 0   \]


Let's start thinking of what $p(\bar{z})$ looks like



\[    p(\bar{z}) = a_n \bar{z}^n + a_{n-a} \bar{z}^{n-1} + \cdots a_2 \bar{z}^2 + a_1 \bar{z} + a_0    \]


By the conjugate powers from above, we get 


\[    p(\bar{z}) = a_n \overline{z^n} + a_{n-a} \overline{z^{n-1}} + \cdots a_2 \overline{z^2} + a_1 \overline{z} + a_0    \]



Since $a_k \in \mathbb{R}$, scalar multiplication gives us




\[    p(\bar{z}) = \overline{a_n z^n} +  \overline{a_{n-a} z^{n-1}} + \cdots  \overline{a_2 z^2} +  \overline{a_1 z} + a_0    \]



$a_0 \in \mathbb{R}$, which means $a_0 = \overline{a_0}$.



\[    p(\bar{z}) = \overline{a_n z^n} +  \overline{a_{n-a} z^{n-1}} + \cdots  \overline{a_2 z^2} +  \overline{a_1 z} + \overline{a_0}    \]





We have also seen that the conjugate of a sum is the sum of the conjugates


\[    p(\bar{z}) = \overline{a_n z^n + a_{n-a} z^{n-1} + \cdots  a_2 z^2 +  a_1 z + a_0}   \]



But this is $\overline{p(z)}$, and we already know  $p(z) = 0$.



\[    p(\bar{z}) = \overline{p(z)} = \bar{0} = 0  \]


$\bar{z}$ is also a root of $p(x)$.




\end{document}
