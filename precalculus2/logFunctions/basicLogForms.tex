\documentclass{ximera}


\graphicspath{
  {./}
  {ximeraTutorial/}
  {basicPhilosophy/}
}

\newcommand{\mooculus}{\textsf{\textbf{MOOC}\textnormal{\textsf{ULUS}}}}


\usepackage{tkz-euclide}\usepackage{tikz}
\usepackage{tikz-cd}
\usetikzlibrary{arrows}
\tikzset{>=stealth,commutative diagrams/.cd,
  arrow style=tikz,diagrams={>=stealth}} %% cool arrow head
\tikzset{shorten <>/.style={ shorten >=#1, shorten <=#1 } } %% allows shorter vectors

\usetikzlibrary{backgrounds} %% for boxes around graphs
\usetikzlibrary{shapes,positioning}  %% Clouds and stars
\usetikzlibrary{matrix} %% for matrix
\usepgfplotslibrary{polar} %% for polar plots
\usepgfplotslibrary{fillbetween} %% to shade area between curves in TikZ
\usetkzobj{all}
\usepackage[makeroom]{cancel} %% for strike outs
%\usepackage{mathtools} %% for pretty underbrace % Breaks Ximera
%\usepackage{multicol}
\usepackage{pgffor} %% required for integral for loops



%% http://tex.stackexchange.com/questions/66490/drawing-a-tikz-arc-specifying-the-center
%% Draws beach ball
\tikzset{pics/carc/.style args={#1:#2:#3}{code={\draw[pic actions] (#1:#3) arc(#1:#2:#3);}}}



\usepackage{array}
\setlength{\extrarowheight}{+.1cm}
\newdimen\digitwidth
\settowidth\digitwidth{9}
\def\divrule#1#2{
\noalign{\moveright#1\digitwidth
\vbox{\hrule width#2\digitwidth}}}
























%%This is to help with formatting on future title pages.
\newenvironment{sectionOutcomes}{}{}


\title{Basic Forms}

\begin{document}

\begin{abstract}
basic
\end{abstract}
\maketitle






\textbf{\textcolor{blue!55!black}{A Different Perspective}} 


Basic logarithmic functions, are those functions which \textbf{\textcolor{red!80!black}{CAN}} be represented by formulas of the form $A \, \log_r(B \, x + C) + D$.  \\


We can decide whether the function is increasing or decreasing by the value of $r$, the sign of $A$, and the sign of $B$. \\





\begin{itemize}
\item $r > 1$ and $A > 0$ and $B > 0$ : increasing positive function
\item $r > 1$ and $A > 0$ and $B < 0$ : decreasing negative function  
\item $r > 1$ and $A < 0$ and $B > 0$ : decreasing positive function
\item $r > 1$ and $A < 0$ and $B < 0$ : increasing positive function
\item $r < 1$ and $A > 0$ and $B > 0$ : decreasing positive function
\item $r < 1$ and $A > 0$ and $B < 0$ : increasing negative function  
\item $r < 1$ and $A < 0$ and $B > 0$ : increasing positive function
\item $r < 1$ and $A < 0$ and $B < 0$ : decreasing positive function
\end{itemize}





\textbf{On the other hand,} we have the algebra rule $\frac{1}{b} = b^{-1}$.  We could think of the base of the logarithmic formula as always being greater than $1$, and just use positive and negative exponents to switch between increasing and decreasing functions. \\

We can rewrite and logarithmic expression with an equivalent expression that has a base greater than $1$. \\



Suppose $0 < r < 1$.\\

Then $\frac{1}{r} > 1$.  We would like to use this new base, which we can accomplish through the change of base formula.\\

\[
\log_r(inside)
\]


\[
\frac{\log_{\tfrac{1}{r}}(inside)}{\log_{\tfrac{1}{r}}(r)}
\]


$\log_{\tfrac{1}{r}}(r)$ is the thing you raise $\frac{1}{r}$ to, to get $r$, which is $-1$. \\


That gives us


\[
\frac{\log_{\tfrac{1}{r}}(inside)}{-1}
\]



\[
-\log_{\tfrac{1}{r}}(inside)
\]



\textbf{\textcolor{red!90!darkgray}{$\blacktriangleright$}} We can switch and logarithm from a base less than $1$, to a base greater than $1$ just by negating the logarithm.




\textbf{New Idea:} Basic logarithmic functions, are those functions which \textbf{\textcolor{red!80!black}{CAN}} be represented by formulas of the form $A \, \log_r(B \, x + C) + D$, where $r > 1$.  \\


In this case, we would have the following behaviors: \\


\begin{itemize}
\item $A > 0$ and $B > 0$ : increasing positive function
\item $A > 0$ and $B < 0$ : decreasing negative function  
\item $A < 0$ and $B > 0$ : decreasing positive function
\item $A < 0$ and $B < 0$ : increasing positive function 
\end{itemize}

We would decide function behavior (increasing or decreasing) by the signs of \textbf{both} leading coefficients, $A$ and $A$.

\textbf{\textcolor{red!90!darkgray}{$\blacktriangleright$}} If $A$ and $B$ are the same sign, then we have an increasing function. \\

\textbf{\textcolor{red!90!darkgray}{$\blacktriangleright$}} If $A$ and $B$ are different signs, then we have a decreasing function. \\








\section*{(e)}



In this model, we are using bases that are greater than $1$.  \\


If this is the case, then we might as well use $e$ as our base. \\


\begin{explanation}


If our formula looks like. $A \, \log_r(B \, x + C) + D$ and $r>1$, then we can use the change of base formula to rewrite our formula.



\[
\log_r(inside) = \frac{\log_e(inside)}{\log_e(r)} = \frac{\ln(inside)}{\ln(r)}
\]


\end{explanation}




\textbf{\textcolor{red!90!darkgray}{$\blacktriangleright$}}  Basic logarithmic functions, are those functions which \textbf{\textcolor{red!80!black}{CAN}} be represented by formulas of the form $A \, \ln(B \, x + C) + D$.  \\






In this model, our basic forms to memorize would be \\






\begin{image}
\begin{tikzpicture}
   \begin{axis}[name = leftgraph, 
            domain=-10:10, ymax=10, xmax=10, ymin=-10, xmin=-10,
            axis lines =center, xlabel=$x$, ylabel={$\ln(x)$},
            every axis y label/.style={at=(current axis.above origin),anchor=south},
            every axis x label/.style={at=(current axis.right of origin),anchor=west},
            axis on top
          ]
          
         
          \addplot [line width=1, gray, dashed, domain=(-9.5:9.5),<->] ({0},{x});

          \addplot [line width=2, penColor, smooth,samples=300,domain=(0.001:9), <->] {ln(x)};
          \addplot [color=penColor,only marks,mark=*] coordinates{(1,0)};

           

  \end{axis}
  \begin{axis}[at={(leftgraph.outer east)},anchor=outer west, 
            domain=-10:10, ymax=10, xmax=10, ymin=-10, xmin=-10,
            axis lines =center, xlabel=$x$, ylabel={-\ln(x)$},
            every axis y label/.style={at=(current axis.above origin),anchor=south},
            every axis x label/.style={at=(current axis.right of origin),anchor=west},
            axis on top
          ]
          
         \addplot [line width=1, gray, dashed, domain=(-9.5:9.5),<->] ({0},{x});

          \addplot [line width=2, penColor, smooth,samples=300,domain=(0.001:9),<->] {-ln(x))};
          \addplot [color=penColor,only marks,mark=*] coordinates{(1,0)};

           

  \end{axis}
\end{tikzpicture}
\end{image}









\begin{image}
\begin{tikzpicture}
   \begin{axis}[name = leftgraph, 
            domain=-10:10, ymax=10, xmax=10, ymin=-10, xmin=-10,
            axis lines =center, xlabel=$x$, ylabel={$\ln(-x)$},
            every axis y label/.style={at=(current axis.above origin),anchor=south},
            every axis x label/.style={at=(current axis.right of origin),anchor=west},
            axis on top
          ]
          
          \addplot [line width=1, gray, dashed, domain=(-9.5:9.5),<->] ({0},{x});

          \addplot [line width=2, penColor, smooth,samples=300,domain=(-9:-0.001), <->] {ln(-x)};
          \addplot [color=penColor,only marks,mark=*] coordinates{(-1,0)};
           

  \end{axis}
  \begin{axis}[at={(leftgraph.outer east)},anchor=outer west, 
            domain=-10:10, ymax=10, xmax=10, ymin=-10, xmin=-10,
            axis lines =center, xlabel=$x$, ylabel={$-\ln(-x)$},
            every axis y label/.style={at=(current axis.above origin),anchor=south},
            every axis x label/.style={at=(current axis.right of origin),anchor=west},
            axis on top
          ]
          
          \addplot [line width=1, gray, dashed, domain=(-9.5:9.5),<->] ({0},{x});

          \addplot [line width=2, penColor, smooth,samples=300,domain=(-9:-0.001),<->] {-ln(-x)};
          \addplot [color=penColor,only marks,mark=*] coordinates{(-1,0)};
           

  \end{axis}
\end{tikzpicture}
\end{image}








\subsection*{Pick Your Form}

You should pick your own logarithmic exponential form that you like and understand.  \\

Then, you can change anything given to you into that form.


You might pick a particular number greater than $1$ as the base you like.  $e$ is a very popular choice, because it shows up quite frequently in mathematics, like Calculus. \\











If you pick a base you like and change formulas to use that base, then you have a better chance of quickly deciding on function characteristics. \\



\textbf{\textcolor{blue!55!black}{Basic Basic Form:}}  $\ln(x)$ \\



\textbf{\textcolor{blue!55!black}{Basic Basic Forms:}}  $\ln(x)$, $\ln(-x)$, $-\ln(x)$, $-\ln(-x)$ \\


Now increasing and decreasing become questions about the signs of the two leading coefficents - the leading coefficient of the entire formula and the leading coefficient of the exponent.











\begin{center}
\textbf{\textcolor{green!50!black}{ooooo-=-=-=-ooOoo-=-=-=-ooooo}} \\

more examples can be found by following this link\\ \link[More Examples of Logarithmic Functions]{https://ximera.osu.edu/csccmathematics/precalculus2/precalculus2/logFunctions/examples/exampleList}

\end{center}








\end{document}
