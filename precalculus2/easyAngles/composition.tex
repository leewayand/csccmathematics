\documentclass{ximera}


\graphicspath{
  {./}
  {ximeraTutorial/}
  {basicPhilosophy/}
}

\newcommand{\mooculus}{\textsf{\textbf{MOOC}\textnormal{\textsf{ULUS}}}}


\usepackage{tkz-euclide}\usepackage{tikz}
\usepackage{tikz-cd}
\usetikzlibrary{arrows}
\tikzset{>=stealth,commutative diagrams/.cd,
  arrow style=tikz,diagrams={>=stealth}} %% cool arrow head
\tikzset{shorten <>/.style={ shorten >=#1, shorten <=#1 } } %% allows shorter vectors

\usetikzlibrary{backgrounds} %% for boxes around graphs
\usetikzlibrary{shapes,positioning}  %% Clouds and stars
\usetikzlibrary{matrix} %% for matrix
\usepgfplotslibrary{polar} %% for polar plots
\usepgfplotslibrary{fillbetween} %% to shade area between curves in TikZ
\usetkzobj{all}
\usepackage[makeroom]{cancel} %% for strike outs
%\usepackage{mathtools} %% for pretty underbrace % Breaks Ximera
%\usepackage{multicol}
\usepackage{pgffor} %% required for integral for loops



%% http://tex.stackexchange.com/questions/66490/drawing-a-tikz-arc-specifying-the-center
%% Draws beach ball
\tikzset{pics/carc/.style args={#1:#2:#3}{code={\draw[pic actions] (#1:#3) arc(#1:#2:#3);}}}



\usepackage{array}
\setlength{\extrarowheight}{+.1cm}
\newdimen\digitwidth
\settowidth\digitwidth{9}
\def\divrule#1#2{
\noalign{\moveright#1\digitwidth
\vbox{\hrule width#2\digitwidth}}}
























%%This is to help with formatting on future title pages.
\newenvironment{sectionOutcomes}{}{}


\title{As Input}

\begin{document}

\begin{abstract}
composition
\end{abstract}
\maketitle




Now that we can move around the circle and obtain value of sine and cosine, we can use these funciton values as input to additional functions, via composition. \\




Let's examine 
\[ g(x) = \frac{1}{2} (\sin(x) - 3)^2 - 4 \]


We can view this as a composition: $\sin(x)$ has been composed with $\frac{1}{2} (t - 3)^2 - 4$.



$P(t) = \frac{1}{2} (t - 3)^2 - 4$ is a quadratic funciton whose graph is a parabola.






\begin{image}
\begin{tikzpicture}
  \begin{axis}[
            domain=-10:10, ymax=10, xmax=10, ymin=-10, xmin=-10,
            axis lines =center, xlabel=$t$, ylabel={$y=P(t)$}, grid = major, grid style={dashed},
            ytick={-10,-8,-6,-4,-2,2,4,6,8,10},
            xtick={-10,-8,-6,-4,-2,2,4,6,8,10},
            yticklabels={$-10$,$-8$,$-6$,$-4$,$-2$,$2$,$4$,$6$,$8$,$10$}, 
            xticklabels={$-10$,$-8$,$-6$,$-4$,$-2$,$2$,$4$,$6$,$8$,$10$},
            ticklabel style={font=\scriptsize},
            every axis y label/.style={at=(current axis.above origin),anchor=south},
            every axis x label/.style={at=(current axis.right of origin),anchor=west},
            axis on top
          ]
          

			\addplot [line width=2, penColor, smooth,samples=200,domain=(-2.2:8.2),<->] {0.5*(x-3)^2 - 4};

          	%\addplot[color=penColor,fill=penColor2,only marks,mark=*] coordinates{(-6,9)};
          	%\addplot[color=penColor,fill=penColor2,only marks,mark=*] coordinates{(2,-7)};


  \end{axis}
\end{tikzpicture}
\end{image}



We are replacing $t$ with $\sin(x)$ to get $g(x) = \frac{1}{2} (\sin(x) - 3)^2 - 4$.\\

$\sin(x)$ has values from $-1$ to $1$ and those values will be going in for $t$ into $P(t) = \frac{1}{2} (t - 3)^2 - 4$. \\

Therefore, we want to look at $P(t)$ on the interval $[-1,1]$. \\





\begin{image}
\begin{tikzpicture}
  \begin{axis}[
            domain=-10:10, ymax=10, xmax=10, ymin=-10, xmin=-10,
            axis lines =center, xlabel=$t$, ylabel={$y=P(t)$}, grid = major, grid style={dashed},
            ytick={-10,-8,-6,-4,-2,2,4,6,8,10},
            xtick={-10,-8,-6,-4,-2,2,4,6,8,10},
            yticklabels={$-10$,$-8$,$-6$,$-4$,$-2$,$2$,$4$,$6$,$8$,$10$}, 
            xticklabels={$-10$,$-8$,$-6$,$-4$,$-2$,$2$,$4$,$6$,$8$,$10$},
            ticklabel style={font=\scriptsize},
            every axis y label/.style={at=(current axis.above origin),anchor=south},
            every axis x label/.style={at=(current axis.right of origin),anchor=west},
            axis on top
          ]
          

			\addplot [line width=2, penColor, smooth,samples=200,domain=(-1:1)] {0.5*(x-3)^2 - 4};

          	%\addplot[color=penColor,fill=penColor2,only marks,mark=*] coordinates{(-6,9)};
          	%\addplot[color=penColor,fill=penColor2,only marks,mark=*] coordinates{(2,-7)};


  \end{axis}
\end{tikzpicture}
\end{image}


On this interval $P$ is decreasing, which means the maximum occurs on the left and the minimum on the right.\\

As $\sin(x)$ moves from $-1$ to $1$, $P(t)$ moves from $P(-1)=4$ to $P(1)=-2$. \\


To get $\sin(x)$ to move from $-1$ to $1$, $x$ needs to start at $-\frac{\pi}{2}$ to $\frac{\pi}{2}$.








\begin{image}
\begin{tikzpicture}
  \begin{axis}[
            domain=-10:10, ymax=10, xmax=10, ymin=-10, xmin=-10,
            axis lines =center, xlabel=$x$, ylabel={$y=g(x)$}, grid = major, grid style={dashed},
            ytick={-10,-8,-6,-4,-2,2,4,6,8,10},
            xtick={-10,-8,-6,-4,-2,2,4,6,8,10},
            yticklabels={$-10$,$-8$,$-6$,$-4$,$-2$,$2$,$4$,$6$,$8$,$10$}, 
            xticklabels={$-10$,$-8$,$-6$,$-4$,$-2$,$2$,$4$,$6$,$8$,$10$},
            ticklabel style={font=\scriptsize},
            every axis y label/.style={at=(current axis.above origin),anchor=south},
            every axis x label/.style={at=(current axis.right of origin),anchor=west},
            axis on top
          ]
          

			\addplot [line width=2, penColor, smooth,samples=200,domain=(-1.570:1.570)] {0.5*(sin(deg(x))-3)^2 - 4};

          	%\addplot[color=penColor,fill=penColor2,only marks,mark=*] coordinates{(-6,9)};
          	%\addplot[color=penColor,fill=penColor2,only marks,mark=*] coordinates{(2,-7)};


  \end{axis}
\end{tikzpicture}
\end{image}





As $x$ moves from $\frac{\pi}{2}$ to $\frac{3\pi}{2}$, $\sin(x)$ to moves from $1$ to $-1$. This piece of the parabola is traced backwards as $x$ moves forward.








\begin{image}
\begin{tikzpicture}
  \begin{axis}[
            domain=-10:10, ymax=10, xmax=10, ymin=-10, xmin=-10,
            axis lines =center, xlabel=$x$, ylabel={$y=g(x)$}, grid = major, grid style={dashed},
            ytick={-10,-8,-6,-4,-2,2,4,6,8,10},
            xtick={-10,-8,-6,-4,-2,2,4,6,8,10},
            yticklabels={$-10$,$-8$,$-6$,$-4$,$-2$,$2$,$4$,$6$,$8$,$10$}, 
            xticklabels={$-10$,$-8$,$-6$,$-4$,$-2$,$2$,$4$,$6$,$8$,$10$},
            ticklabel style={font=\scriptsize},
            every axis y label/.style={at=(current axis.above origin),anchor=south},
            every axis x label/.style={at=(current axis.right of origin),anchor=west},
            axis on top
          ]
          

			\addplot [line width=2, penColor, smooth,samples=200,domain=(-1.570:4.71)] {0.5*(sin(deg(x))-3)^2 - 4};

          	%\addplot[color=penColor,fill=penColor2,only marks,mark=*] coordinates{(-6,9)};
          	%\addplot[color=penColor,fill=penColor2,only marks,mark=*] coordinates{(2,-7)};


  \end{axis}
\end{tikzpicture}
\end{image}



And, this repeats.












\begin{image}
\begin{tikzpicture}
  \begin{axis}[
            domain=-10:10, ymax=10, xmax=10, ymin=-10, xmin=-10,
            axis lines =center, xlabel=$x$, ylabel={$y=g(x)$}, grid = major, grid style={dashed},
            ytick={-10,-8,-6,-4,-2,2,4,6,8,10},
            xtick={-10,-8,-6,-4,-2,2,4,6,8,10},
            yticklabels={$-10$,$-8$,$-6$,$-4$,$-2$,$2$,$4$,$6$,$8$,$10$}, 
            xticklabels={$-10$,$-8$,$-6$,$-4$,$-2$,$2$,$4$,$6$,$8$,$10$},
            ticklabel style={font=\scriptsize},
            every axis y label/.style={at=(current axis.above origin),anchor=south},
            every axis x label/.style={at=(current axis.right of origin),anchor=west},
            axis on top
          ]
          

			\addplot [line width=2, penColor, smooth,samples=200,domain=(-9.5:9.5),<->] {0.5*(sin(deg(x))-3)^2 - 4};

          	%\addplot[color=penColor,fill=penColor2,only marks,mark=*] coordinates{(-6,9)};
          	%\addplot[color=penColor,fill=penColor2,only marks,mark=*] coordinates{(2,-7)};


  \end{axis}
\end{tikzpicture}
\end{image}


$g(x)$ has a maximum value whenever $t=-1$ and a minimum value whenever $t=1$. \\

This corresponds to when $x=\frac{-\pi}{2} \pm 2\pi$ and $=\frac{\pi}{2} \pm 2\pi$. \\


$g(x)$ has a maximum value of 


\begin{align*}
g\left( -\frac{\pi}{2} \right) &= \frac{1}{2} \left(\left( -\frac{\pi}{2} \right) - 3 \right)^2 - 4 \\
                               &= \frac{1}{2} (-1 - 3)^2 - 4 \\
                               &= \frac{1}{2} \cdot 16 - 4 \\
                               &=  4
\end{align*}





$g(x)$ has a minimum value of
\begin{align*}
g\left( -\frac{\pi}{2} \right) &= \frac{1}{2} \left(\left( \frac{\pi}{2} \right) - 3 \right)^2 - 4 \\
                               &= \frac{1}{2} (1 - 3)^2 - 4 \\
                               &= \frac{1}{2} \cdot 4 - 4 \\
                               &=  -2
\end{align*}



$g$ is decreasing on 
\[
\left[ \frac{-\pi}{2} + 2k\pi, \frac{\pi}{2} + 2k\pi \right], \text{ where } \,  k \in \mathbb{Z}
\]




$g$ is decreasing on 
\[
\left[ \frac{\pi}{2} + 2k\pi, \frac{3\pi}{2} + 2k\pi \right], \text{ where } \,  k \in \mathbb{Z}
\]





\end{document}

