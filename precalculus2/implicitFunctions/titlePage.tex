\documentclass{ximera}


\graphicspath{
  {./}
  {ximeraTutorial/}
  {basicPhilosophy/}
}

\newcommand{\mooculus}{\textsf{\textbf{MOOC}\textnormal{\textsf{ULUS}}}}


\usepackage{tkz-euclide}\usepackage{tikz}
\usepackage{tikz-cd}
\usetikzlibrary{arrows}
\tikzset{>=stealth,commutative diagrams/.cd,
  arrow style=tikz,diagrams={>=stealth}} %% cool arrow head
\tikzset{shorten <>/.style={ shorten >=#1, shorten <=#1 } } %% allows shorter vectors

\usetikzlibrary{backgrounds} %% for boxes around graphs
\usetikzlibrary{shapes,positioning}  %% Clouds and stars
\usetikzlibrary{matrix} %% for matrix
\usepgfplotslibrary{polar} %% for polar plots
\usepgfplotslibrary{fillbetween} %% to shade area between curves in TikZ
\usetkzobj{all}
\usepackage[makeroom]{cancel} %% for strike outs
%\usepackage{mathtools} %% for pretty underbrace % Breaks Ximera
%\usepackage{multicol}
\usepackage{pgffor} %% required for integral for loops



%% http://tex.stackexchange.com/questions/66490/drawing-a-tikz-arc-specifying-the-center
%% Draws beach ball
\tikzset{pics/carc/.style args={#1:#2:#3}{code={\draw[pic actions] (#1:#3) arc(#1:#2:#3);}}}



\usepackage{array}
\setlength{\extrarowheight}{+.1cm}
\newdimen\digitwidth
\settowidth\digitwidth{9}
\def\divrule#1#2{
\noalign{\moveright#1\digitwidth
\vbox{\hrule width#2\digitwidth}}}
























%%This is to help with formatting on future title pages.
\newenvironment{sectionOutcomes}{}{}


\title{Implicit Functions}

\begin{document}

\begin{abstract}
%
\end{abstract}
\maketitle



A formula is a special type of equation.  A formula is an equation in which the function name, or dependent variable, has been isolated on one side - ``solved for''.


Given an equation for a function, we prefer to work with an equivalent equation where the function has been isolated.  However, that is not always possible.

In these situations, it is quite common for the equation to allow multiple values for the dependent variable, which violates our one and only rule for functions.  Solving for the function requires choices from these multiple values.

But, this might be ok.  We might just be examining the prospective function around a particular domain number. Restricting the algebraic description provided by the equation to that area might clear up the problem.

Functions defined via these types of restrictions are called \textbf{implicitly defined functions} or \textbf{implicit functions}.

If you can isolate the dependent variable and obtain a formula for the function, then we use the word \textbf{explicit}.











\subsection*{Learning Outcomes}


\begin{sectionOutcomes}
In this section, students will 

\begin{itemize}
\item analyze functions via equations.
\end{itemize}
\end{sectionOutcomes}












\begin{center}
\textbf{\textcolor{green!50!black}{ooooo-=-=-=-ooOoo-=-=-=-ooooo}} \\

more examples can be found by following this link\\ \link[More Examples of Implicit Functions]{https://ximera.osu.edu/csccmathematics/precalculus2/precalculus2/implicitFunctions/examples/exampleList}

\end{center}





\end{document}
