\documentclass{ximera}


\graphicspath{
  {./}
  {ximeraTutorial/}
  {basicPhilosophy/}
}

\newcommand{\mooculus}{\textsf{\textbf{MOOC}\textnormal{\textsf{ULUS}}}}


\usepackage{tkz-euclide}\usepackage{tikz}
\usepackage{tikz-cd}
\usetikzlibrary{arrows}
\tikzset{>=stealth,commutative diagrams/.cd,
  arrow style=tikz,diagrams={>=stealth}} %% cool arrow head
\tikzset{shorten <>/.style={ shorten >=#1, shorten <=#1 } } %% allows shorter vectors

\usetikzlibrary{backgrounds} %% for boxes around graphs
\usetikzlibrary{shapes,positioning}  %% Clouds and stars
\usetikzlibrary{matrix} %% for matrix
\usepgfplotslibrary{polar} %% for polar plots
\usepgfplotslibrary{fillbetween} %% to shade area between curves in TikZ
\usetkzobj{all}
\usepackage[makeroom]{cancel} %% for strike outs
%\usepackage{mathtools} %% for pretty underbrace % Breaks Ximera
%\usepackage{multicol}
\usepackage{pgffor} %% required for integral for loops



%% http://tex.stackexchange.com/questions/66490/drawing-a-tikz-arc-specifying-the-center
%% Draws beach ball
\tikzset{pics/carc/.style args={#1:#2:#3}{code={\draw[pic actions] (#1:#3) arc(#1:#2:#3);}}}



\usepackage{array}
\setlength{\extrarowheight}{+.1cm}
\newdimen\digitwidth
\settowidth\digitwidth{9}
\def\divrule#1#2{
\noalign{\moveright#1\digitwidth
\vbox{\hrule width#2\digitwidth}}}
























%%This is to help with formatting on future title pages.
\newenvironment{sectionOutcomes}{}{}


\title{Function Forms}

\begin{document}

\begin{abstract}
%
\end{abstract}
\maketitle



Our goal in Calculus is anlayzing functions.

The very first step in analyzing a funciton is identifying the type of function you have. We have a preference to our categories. \\

We first decide if we have an elementary function. If not, then we decide if we are looking at one our four traditional operations.  If not, then it must be a composition.\\



\textbf{\textcolor{blue!55!black}{Choice 1:}}
\begin{itemize}
\item Constant
\item Linear
\item Quadratic
\item Polynomial
\item Rational
\item Exponential
\item Logarithmic
\item Root or Radical
\item Absolute Value
\item Trigonometric
\end{itemize}



\textbf{\textcolor{blue!55!black}{Choice 2:}}
\begin{itemize}
\item Sum
\item Difference
\item Product
\item Quotient
\end{itemize}



\textbf{\textcolor{blue!55!black}{Choice 3:}}
\begin{itemize}
\item Composition
\end{itemize}




Of course, it could also be a piecewise defined function.  Those are usually easy to spot.
\begin{itemize}
\item Piecewise Defined
\end{itemize}







\subsection*{Learning Outcomes}


\begin{sectionOutcomes}
In this section, students will 

\begin{itemize}
\item identify function forms.
\end{itemize}
\end{sectionOutcomes}












\begin{center}
\textbf{\textcolor{green!50!black}{ooooo-=-=-=-ooOoo-=-=-=-ooooo}} \\

more examples can be found by following this link\\ \link[More Examples of Function Forms]{https://ximera.osu.edu/csccmathematics/precalculus2/precalculus2/functionForm/examples/exampleList}

\end{center}




\end{document}
