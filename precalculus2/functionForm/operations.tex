\documentclass{ximera}

%\usepackage{todonotes}

\newcommand{\todo}{}

\usepackage{esint} % for \oiint
\ifxake%%https://math.meta.stackexchange.com/questions/9973/how-do-you-render-a-closed-surface-double-integral
\renewcommand{\oiint}{{\large\bigcirc}\kern-1.56em\iint}
\fi


\graphicspath{
  {./}
  {ximeraTutorial/}
  {basicPhilosophy/}
  {functionsOfSeveralVariables/}
  {normalVectors/}
  {lagrangeMultipliers/}
  {vectorFields/}
  {greensTheorem/}
  {shapeOfThingsToCome/}
  {dotProducts/}
  {partialDerivativesAndTheGradientVector/}
  {../productAndQuotientRules/exercises/}
  {../normalVectors/exercisesParametricPlots/}
  {../continuityOfFunctionsOfSeveralVariables/exercises/}
  {../partialDerivativesAndTheGradientVector/exercises/}
  {../directionalDerivativeAndChainRule/exercises/}
  {../commonCoordinates/exercisesCylindricalCoordinates/}
  {../commonCoordinates/exercisesSphericalCoordinates/}
  {../greensTheorem/exercisesCurlAndLineIntegrals/}
  {../greensTheorem/exercisesDivergenceAndLineIntegrals/}
  {../shapeOfThingsToCome/exercisesDivergenceTheorem/}
  {../greensTheorem/}
  {../shapeOfThingsToCome/}
  {../separableDifferentialEquations/exercises/}
  {vectorFields/}
}

\newcommand{\mooculus}{\textsf{\textbf{MOOC}\textnormal{\textsf{ULUS}}}}

\usepackage{tkz-euclide}
\usepackage{tikz}
\usepackage{tikz-cd}
\usetikzlibrary{arrows}
\tikzset{>=stealth,commutative diagrams/.cd,
  arrow style=tikz,diagrams={>=stealth}} %% cool arrow head
\tikzset{shorten <>/.style={ shorten >=#1, shorten <=#1 } } %% allows shorter vectors

\usetikzlibrary{backgrounds} %% for boxes around graphs
\usetikzlibrary{shapes,positioning}  %% Clouds and stars
\usetikzlibrary{matrix} %% for matrix
\usepgfplotslibrary{polar} %% for polar plots
\usepgfplotslibrary{fillbetween} %% to shade area between curves in TikZ
%\usetkzobj{all}
\usepackage[makeroom]{cancel} %% for strike outs
%\usepackage{mathtools} %% for pretty underbrace % Breaks Ximera
%\usepackage{multicol}
\usepackage{pgffor} %% required for integral for loops



%% http://tex.stackexchange.com/questions/66490/drawing-a-tikz-arc-specifying-the-center
%% Draws beach ball
\tikzset{pics/carc/.style args={#1:#2:#3}{code={\draw[pic actions] (#1:#3) arc(#1:#2:#3);}}}



\usepackage{array}
\setlength{\extrarowheight}{+.1cm}
\newdimen\digitwidth
\settowidth\digitwidth{9}
\def\divrule#1#2{
\noalign{\moveright#1\digitwidth
\vbox{\hrule width#2\digitwidth}}}




% \newcommand{\RR}{\mathbb R}
% \newcommand{\R}{\mathbb R}
% \newcommand{\N}{\mathbb N}
% \newcommand{\Z}{\mathbb Z}

\newcommand{\sagemath}{\textsf{SageMath}}


%\renewcommand{\d}{\,d\!}
%\renewcommand{\d}{\mathop{}\!d}
%\newcommand{\dd}[2][]{\frac{\d #1}{\d #2}}
%\newcommand{\pp}[2][]{\frac{\partial #1}{\partial #2}}
% \renewcommand{\l}{\ell}
%\newcommand{\ddx}{\frac{d}{\d x}}

% \newcommand{\zeroOverZero}{\ensuremath{\boldsymbol{\tfrac{0}{0}}}}
%\newcommand{\inftyOverInfty}{\ensuremath{\boldsymbol{\tfrac{\infty}{\infty}}}}
%\newcommand{\zeroOverInfty}{\ensuremath{\boldsymbol{\tfrac{0}{\infty}}}}
%\newcommand{\zeroTimesInfty}{\ensuremath{\small\boldsymbol{0\cdot \infty}}}
%\newcommand{\inftyMinusInfty}{\ensuremath{\small\boldsymbol{\infty - \infty}}}
%\newcommand{\oneToInfty}{\ensuremath{\boldsymbol{1^\infty}}}
%\newcommand{\zeroToZero}{\ensuremath{\boldsymbol{0^0}}}
%\newcommand{\inftyToZero}{\ensuremath{\boldsymbol{\infty^0}}}



% \newcommand{\numOverZero}{\ensuremath{\boldsymbol{\tfrac{\#}{0}}}}
% \newcommand{\dfn}{\textbf}
% \newcommand{\unit}{\,\mathrm}
% \newcommand{\unit}{\mathop{}\!\mathrm}
% \newcommand{\eval}[1]{\bigg[ #1 \bigg]}
% \newcommand{\seq}[1]{\left( #1 \right)}
% \renewcommand{\epsilon}{\varepsilon}
% \renewcommand{\phi}{\varphi}


% \renewcommand{\iff}{\Leftrightarrow}

% \DeclareMathOperator{\arccot}{arccot}
% \DeclareMathOperator{\arcsec}{arcsec}
% \DeclareMathOperator{\arccsc}{arccsc}
% \DeclareMathOperator{\si}{Si}
% \DeclareMathOperator{\scal}{scal}
% \DeclareMathOperator{\sign}{sign}


%% \newcommand{\tightoverset}[2]{% for arrow vec
%%   \mathop{#2}\limits^{\vbox to -.5ex{\kern-0.75ex\hbox{$#1$}\vss}}}
% \newcommand{\arrowvec}[1]{{\overset{\rightharpoonup}{#1}}}
% \renewcommand{\vec}[1]{\arrowvec{\mathbf{#1}}}
% \renewcommand{\vec}[1]{{\overset{\boldsymbol{\rightharpoonup}}{\mathbf{#1}}}}

% \newcommand{\point}[1]{\left(#1\right)} %this allows \vector{ to be changed to \vector{ with a quick find and replace
% \newcommand{\pt}[1]{\mathbf{#1}} %this allows \vec{ to be changed to \vec{ with a quick find and replace
% \newcommand{\Lim}[2]{\lim_{\point{#1} \to \point{#2}}} %Bart, I changed this to point since I want to use it.  It runs through both of the exercise and exerciseE files in limits section, which is why it was in each document to start with.

% \DeclareMathOperator{\proj}{\mathbf{proj}}
% \newcommand{\veci}{{\boldsymbol{\hat{\imath}}}}
% \newcommand{\vecj}{{\boldsymbol{\hat{\jmath}}}}
% \newcommand{\veck}{{\boldsymbol{\hat{k}}}}
% \newcommand{\vecl}{\vec{\boldsymbol{\l}}}
% \newcommand{\uvec}[1]{\mathbf{\hat{#1}}}
% \newcommand{\utan}{\mathbf{\hat{t}}}
% \newcommand{\unormal}{\mathbf{\hat{n}}}
% \newcommand{\ubinormal}{\mathbf{\hat{b}}}

% \newcommand{\dotp}{\bullet}
% \newcommand{\cross}{\boldsymbol\times}
% \newcommand{\grad}{\boldsymbol\nabla}
% \newcommand{\divergence}{\grad\dotp}
% \newcommand{\curl}{\grad\cross}
%\DeclareMathOperator{\divergence}{divergence}
%\DeclareMathOperator{\curl}[1]{\grad\cross #1}
% \newcommand{\lto}{\mathop{\longrightarrow\,}\limits}

% \renewcommand{\bar}{\overline}

\colorlet{textColor}{black}
\colorlet{background}{white}
\colorlet{penColor}{blue!50!black} % Color of a curve in a plot
\colorlet{penColor2}{red!50!black}% Color of a curve in a plot
\colorlet{penColor3}{red!50!blue} % Color of a curve in a plot
\colorlet{penColor4}{green!50!black} % Color of a curve in a plot
\colorlet{penColor5}{orange!80!black} % Color of a curve in a plot
\colorlet{penColor6}{yellow!70!black} % Color of a curve in a plot
\colorlet{fill1}{penColor!20} % Color of fill in a plot
\colorlet{fill2}{penColor2!20} % Color of fill in a plot
\colorlet{fillp}{fill1} % Color of positive area
\colorlet{filln}{penColor2!20} % Color of negative area
\colorlet{fill3}{penColor3!20} % Fill
\colorlet{fill4}{penColor4!20} % Fill
\colorlet{fill5}{penColor5!20} % Fill
\colorlet{gridColor}{gray!50} % Color of grid in a plot

\newcommand{\surfaceColor}{violet}
\newcommand{\surfaceColorTwo}{redyellow}
\newcommand{\sliceColor}{greenyellow}




\pgfmathdeclarefunction{gauss}{2}{% gives gaussian
  \pgfmathparse{1/(#2*sqrt(2*pi))*exp(-((x-#1)^2)/(2*#2^2))}%
}


%%%%%%%%%%%%%
%% Vectors
%%%%%%%%%%%%%

%% Simple horiz vectors
\renewcommand{\vector}[1]{\left\langle #1\right\rangle}


%% %% Complex Horiz Vectors with angle brackets
%% \makeatletter
%% \renewcommand{\vector}[2][ , ]{\left\langle%
%%   \def\nextitem{\def\nextitem{#1}}%
%%   \@for \el:=#2\do{\nextitem\el}\right\rangle%
%% }
%% \makeatother

%% %% Vertical Vectors
%% \def\vector#1{\begin{bmatrix}\vecListA#1,,\end{bmatrix}}
%% \def\vecListA#1,{\if,#1,\else #1\cr \expandafter \vecListA \fi}

%%%%%%%%%%%%%
%% End of vectors
%%%%%%%%%%%%%

%\newcommand{\fullwidth}{}
%\newcommand{\normalwidth}{}



%% makes a snazzy t-chart for evaluating functions
%\newenvironment{tchart}{\rowcolors{2}{}{background!90!textColor}\array}{\endarray}

%%This is to help with formatting on future title pages.
\newenvironment{sectionOutcomes}{}{}



%% Flowchart stuff
%\tikzstyle{startstop} = [rectangle, rounded corners, minimum width=3cm, minimum height=1cm,text centered, draw=black]
%\tikzstyle{question} = [rectangle, minimum width=3cm, minimum height=1cm, text centered, draw=black]
%\tikzstyle{decision} = [trapezium, trapezium left angle=70, trapezium right angle=110, minimum width=3cm, minimum height=1cm, text centered, draw=black]
%\tikzstyle{question} = [rectangle, rounded corners, minimum width=3cm, minimum height=1cm,text centered, draw=black]
%\tikzstyle{process} = [rectangle, minimum width=3cm, minimum height=1cm, text centered, draw=black]
%\tikzstyle{decision} = [trapezium, trapezium left angle=70, trapezium right angle=110, minimum width=3cm, minimum height=1cm, text centered, draw=black]


\title{Operations}

\begin{document}

\begin{abstract}
new functions from old
\end{abstract}
\maketitle



Values of functions are numbers.  So, it is no surprise that we can make new functions by combining old functions with addition, subtraction, multiplication, and division. Thinking ahead to Calculus, it is also very handy to see multiplication by a constant.\\




\begin{warning}  \textbf{\textcolor{red!70!black}{Is}}


When categorizing a formula, our first thought is that the function is an elementary function and we would categorize it from our library. \\

If it is not, then we turn to the structure of the expression where the \textbf{Order of Operations} dictates how we view the algebra. \\

When thinking of the Order of Operations, we are thinking of what the expression actually looks like as it is written.


\begin{center}
These are ``\textbf{\textcolor{red!70!black}{Is}}'' questions.
\end{center}

We are not thinking of how the expression might be changed algebraically.  We are identifying how it is currently written. \\

The Order of Operations instructs us on how to make this decision.

\end{warning}



\subsection*{Constant Multiple}


\begin{template}  \textbf{\textcolor{blue!55!black}{Constant Multiple}} \\


If  $f$ is a function and $A$ is a constant, then $A \, f$ is called a \textbf{\textcolor{green!50!black}{constant multiple}} of $f$. \\

The domain of a constant multiple of a function is equal to the domain of the function. \\


\[ A \cdot f(x)  \]



\end{template}



\begin{observation}

If $f$ belongs to a nice category of functions, then the constant multiple also belongs to the same nice category.

\end{observation}




\begin{example}

$f(x) = 4 \, e^x$ is a constant multiple of $e^x$. \\
$Y(t) = -5 \, \sin(4t + 8)$ is a constant multiple of $\sin(4t + 8)$. \\
$m(k) = 4 (k+1)(k-5)$ is a constant multiple of $(k+1)(k-5)$. \\


$T(n) = 4 (n+1) + 6$ is not a constant multiple of $(n+1) + 6$. \\

\end{example}


















\subsection*{Sums}


\begin{template}  \textbf{\textcolor{blue!55!black}{Sum}} \\


If  $f$ and $g$ are functions, then $f + g$ is called the \textbf{\textcolor{green!50!black}{sum}} of $f$ and $g$. \\

The sum of two functions is defined on the intersection of their two domains. \\


\[ (f + g)(a) = f(a) + g(a)  \]



\end{template}



\begin{warning}

If $f$ or $g$ belongs to a nice category of functions, their sum usually does not.

\end{warning}




\begin{example}

$f(x) = e^x$ is an exponential function. \\
$g(t) = 3 t + 4$ is a linear function. \\

However, $(f + g)(k) = f(k) + g(k) = e^k + 3 k + 4$ is neither exponential nor linear.

\end{example}








\subsection*{Differences}



\begin{template}  \textbf{\textcolor{blue!55!black}{Difference}} \\


If  $f$ and $g$ are functions, then $f - g$ is called the \textbf{\textcolor{green!50!black}{difference}} of $f$ and $g$. \\

The difference of two functions is defined on the intersection of their two domains. \\


\[ (f - g)(a) = f(a) - g(a)  \]



\end{template}



\begin{warning}

If $f$ or $g$ belongs to a nice category of functions, their difference usually does not.

\end{warning}





\begin{example}

$f(x) = e^x$ is an exponential function. \\
$g(t) = 3 t + 4$ is a linear function. \\

However, $(f - g)(k) = f(k) - g(k) = e^k - 3 k - 4$ is neither exponential nor linear.

\end{example}
















\subsection*{Products}



\begin{template}  \textbf{\textcolor{blue!55!black}{Product}} \\


If  $f$ and $g$ are functions, then $f \cdot g$, or just $f g$, is called the \textbf{\textcolor{green!50!black}{product}} of $f$ and $g$. \\

The product of two functions is defined on the intersection of their two domains. \\


\[ (f \cdot g)(a) = f(a) \cdot g(a)  \]



\end{template}



\begin{warning}

If $f$ or $g$ belongs to a nice category of functions, their product usually does not.

\end{warning}





\begin{example}

$f(x) = e^x$ is an exponential function. \\
$g(t) = 3 t + 4$ is a linear function. \\

However, $(f \cdot g)(k) = f(k) \cdot g(k) = e^k (3 k + 4)$ is neither exponential nor linear.

\end{example}























\subsection*{Quotients}



It has been very common to see numbers involved in division: $15 \div 7$. However, functions are never written as division. They are always written as quotients (fractions).




\begin{template}  \textbf{\textcolor{blue!55!black}{Quotients}} \\


If  $f$ and $g$ are functions, then $\frac{f}{g}$ is called the \textbf{\textcolor{green!50!black}{quotient}} of $f$ and $g$. \\

The quotient of two functions is defined on the intersection of their two domains except at the zeros of the function in the denominator. \\


\[ \left(\frac{f}{g}\right)(a) = \frac{f(a)}{g(a)}  \]



\end{template}



\begin{warning}

If $f$ or $g$ belongs to a nice category of functions, their quotient usually does not.

\end{warning}





\begin{example}

$f(x) = e^x$ is an exponential function. \\
$g(t) = 3 t + 4$ is a linear function. \\

However, $\left(\frac{f}{g}\right)(k) = \frac{f(k)}{g(k)} = \frac{e^k}{(3 k + 4)}$ is neither exponential nor linear.

\end{example}











\begin{warning}


\begin{itemize}
\item Just because you see an exponential formula does not mean you have an exponential function. \\ 
\item Just because you see a logarithmic formula does not mean you have a logarithmic function. \\ 
\item Just because you see a quadratic formula does not mean you have a quadratic function. \\ 
\item Just because you see a linear formula does not mean you have a linear function. \\ 
\end{itemize}

\end{warning}












\begin{center}
\textbf{\textcolor{green!50!black}{ooooo-=-=-=-ooOoo-=-=-=-ooooo}} \\

more examples can be found by following this link\\ \link[More Examples of the Function Forms]{https://ximera.osu.edu/csccmathematics/precalculus2/precalculus2/functionForm/examples/exampleList}

\end{center}







\end{document}
