\documentclass{ximera}


\graphicspath{
  {./}
  {ximeraTutorial/}
  {basicPhilosophy/}
}

\newcommand{\mooculus}{\textsf{\textbf{MOOC}\textnormal{\textsf{ULUS}}}}


\usepackage{tkz-euclide}\usepackage{tikz}
\usepackage{tikz-cd}
\usetikzlibrary{arrows}
\tikzset{>=stealth,commutative diagrams/.cd,
  arrow style=tikz,diagrams={>=stealth}} %% cool arrow head
\tikzset{shorten <>/.style={ shorten >=#1, shorten <=#1 } } %% allows shorter vectors

\usetikzlibrary{backgrounds} %% for boxes around graphs
\usetikzlibrary{shapes,positioning}  %% Clouds and stars
\usetikzlibrary{matrix} %% for matrix
\usepgfplotslibrary{polar} %% for polar plots
\usepgfplotslibrary{fillbetween} %% to shade area between curves in TikZ
\usetkzobj{all}
\usepackage[makeroom]{cancel} %% for strike outs
%\usepackage{mathtools} %% for pretty underbrace % Breaks Ximera
%\usepackage{multicol}
\usepackage{pgffor} %% required for integral for loops



%% http://tex.stackexchange.com/questions/66490/drawing-a-tikz-arc-specifying-the-center
%% Draws beach ball
\tikzset{pics/carc/.style args={#1:#2:#3}{code={\draw[pic actions] (#1:#3) arc(#1:#2:#3);}}}



\usepackage{array}
\setlength{\extrarowheight}{+.1cm}
\newdimen\digitwidth
\settowidth\digitwidth{9}
\def\divrule#1#2{
\noalign{\moveright#1\digitwidth
\vbox{\hrule width#2\digitwidth}}}
























%%This is to help with formatting on future title pages.
\newenvironment{sectionOutcomes}{}{}


\title{Analysis}

\begin{document}

\begin{abstract}
8 characteristics
\end{abstract}
\maketitle



To analyze polynomials, we almost always need a derivative.  If the polynomial is special or we just happen to know unusual information about the polynomial, then we might be able to analyze it fully. \\

Polynomials are difficult. Usually we need the derivative. \\

To have any chance at analyzing a general polynomial, our first thought is factoring. What are the zeros? \\


Let's analyze a cubic polynomial given in factored form and given its derivative. \\


\subsection*{Analysis}


We are given the polynomial and its derivative. \\


\[
P(x) = x^3 + 3 \, x^2 - 10 \, x - 24
\]


\[
P'(x) = 3 \, x^2 + 6 \, x - 10
\]



First, let's factor $P$. \\

Using the Ratiopnal Roots Theorem, we have some guesses on roots.  The smaller factors of $24$ are $\pm8$, $\pm6$, $\pm4$, $\pm3$, $\pm2$, $\pm1$.  



\begin{itemize}
	\item $P(1) = (1)^3 + 3 \, (1)^2 - 10 \, (1) - 24 = -30$
	\item $P(-1) = (1)^3 + 3 \, (1)^2 - 10 \, (1) - 24 = -12$
	\item $P(2) = (1)^3 + 3 \, (1)^2 - 10 \, (1) - 24 = -24$
	\item $P(-2) = (1)^3 + 3 \, (1)^2 - 10 \, (1) - 24 = 0$
	\item $P(3) = (1)^3 + 3 \, (1)^2 - 10 \, (1) - 24 = 0$
	\item $P(-3) = (1)^3 + 3 \, (1)^2 - 10 \, (1) - 24 = 6$
\end{itemize}

$-2$ is a zero (root), which means $x+2$ is a factor. \\

$3$ is a zero (root), which means $x-3$ is a factor. \\


Since $P$ is a cubic polynomial with $1$ as the leading coefficient, we have

\[
x^3 + 3 \, x^2 - 10 \, x - 24 = (x+2) (x-3) (x-A)
\]

If these are equal as functions, then they would be equal at $0$.


\[
(0)^3 + 3 \, (0)^2 - 10 \, (0) - 24 = (0+2) (0-3) (0-A)
\]


\[
-24 = -6 (-A)
\]


\[
A = -4
\]



\[
x^3 + 3 \, x^2 - 10 \, x - 24 = (x+2) (x-3) (x+4)
\]



With this factorizztion, we are ready to begin our analysis. \\



\textbf{Domain}

$P$ is a ploynomial, which makes is natural domain, $(-\infty, \infty)$. \\





\textbf{Zeros}

$P(x) = (x+2) (x-3) (x+4)$, which makes the zeros $-4$, $-2$, and $3$. \\





\textbf{Contiuity}

$P$ is a ploynomial, which makes it continuous. \\







\textbf{End-Behavior}


$P$ is not a constant polynomial, which menas is is unbounded. \\

$P$ is a ploynomial of degree $3$, which is odd.  That tells us that the two end-behaviors are opposite. \\

Since the leading coefficient is $1$, which is positive, we have


\[
\lim\limits_{x \to -\infty} P(x) = -\infty 
\]




\[
\lim\limits_{x \to \infty} P(x) = \infty 
\]


















\textbf{Behavior}

We will use the derivative for behavior. When $P'$ is positive, then $P$ is increasing.  When $P'$ is negative, then $P$ is decreasing.  


First critical numbers.  Since $P$ is a polynomial, the only critical numbers are zeros of $P'$.  


We need to factor. $P'(x) = 3 \, x^2 + 6 \, x - 10$.  This is a quadratic, so we can use the quadratic formula to get  the zeros and use those to factor. \\


\[
\frac{-6 \pb \sqrt{6^2 - 4 (3)(-10)}}{2(3)} = \frac{-6 \pm \sqrt{156}}{6} = \frac{-6 \pm 2\sqrt{39}}{6} = \frac{-3 \pm \sqrt{39}}{3}
\]




\[
P'(x) = 3 \, x^2 + 6 \, x - 10 = 3 \left( x - \frac{-3 - \sqrt{39}}{3} \right) \left( x - \frac{-3 + \sqrt{39}}{3}  \right)
\]



The critical numbers give us three intervals to examine.

\[
\left( -\infty, \frac{-3 - \sqrt{39}}{3} \right), \left( \frac{-3 - \sqrt{39}}{3}, \frac{-3 + \sqrt{39}}{3} \right), \text{ and } \left( \frac{-3 + \sqrt{39}}{3}, \infty \right)
\]


Since, $P'$ is a quadratic with a positive leading coefficient, we know that is negative between the zeros and positive outside. \\



On $\left( -\infty, \frac{-3 - \sqrt{39}}{3} \right)$, $P' > 0$ and $P$ is increasing. \\


On $\left( \frac{-3 - \sqrt{39}}{3}, \frac{-3 + \sqrt{39}}{3} \right)$, $P' < 0$ and $P$ is decreasing. \\


On $\left( \frac{-3 + \sqrt{39}}{3}, \infty \right)$, $P' > 0$ and $P$ is increasing. \\





\textbf{Global Maximum and Minimum}


From the end-behaqvior, we know

\[
\lim\limits_{x \to -\infty} P(x) = -\infty 
\]




\[
\lim\limits_{x \to \infty} P(x) = \infty 
\]



Therefore, $P$ has no global maximum or minimum. \\







\textbf{Local Maximum and Minimum}


$P$ is continuous on $(-\infty, \infty)$ with two critical numbers. We have local extrema at these domain numbers.\\







$P$ is increasing on $\left( -\infty, \frac{-3 - \sqrt{39}}{3} \right)$ and decreasing on $\left( \frac{-3 - \sqrt{39}}{3}, \frac{-3 + \sqrt{39}}{3} \right)$.

$P$ has a local maximum at $\frac{-3 - \sqrt{39}}{3}$.


Evaluating $P\left( \frac{-3 - \sqrt{39}}{3} \right)$ is too much algebra.  But, we can get the internet to tell us.

\[
P\left( \frac{-3 - \sqrt{39}}{3} \right) = \frac{2}{9} (13 \sqrt{39} - 54)
\]


$P$ is decreasing on $\left( \frac{-3 - \sqrt{39}}{3}, \frac{-3 + \sqrt{39}}{3} \right)$ and increasing on $\left( \frac{-3 + \sqrt{39}}{3}, \infty \right)$


$P$ has a local minimum at $\frac{-3 + \sqrt{39}}{3}$.



Evaluating $P\left( \frac{-3 + \sqrt{39}}{3} \right)$ is too much algebra.  But, we can get the internet to tell us.

\[
P\left( \frac{-3 + \sqrt{39}}{3} \right) = -\frac{2}{9} (13 \sqrt{39} + 54)
\]






\textbf{Range}

Since $P$ is contunuous with no singularities, the end-behavior tells us that the range is $(-\infty, \infty)$.\\








$\frac{2}{9} (13 \sqrt{39} - 54) \approx 6.041105329$


$\frac{2}{9} (13 \sqrt{39} + 54) \approx -30.04110533$


These agree with the graph.






\textbf{\textcolor{blue!55!black}{$\blacktriangleright$ desmos graph}} 
\begin{center}
\desmos{dddjcmtve0}{400}{300}
\end{center}















\begin{center}
\textbf{\textcolor{green!50!black}{ooooo-=-=-=-ooOoo-=-=-=-ooooo}} \\

more examples can be found by following this link\\ \link[More Examples of the Derivative]{https://ximera.osu.edu/csccmathematics/precalculus2/precalculus2/theDerivative/examples/exampleList}

\end{center}










\end{document}
