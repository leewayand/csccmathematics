\documentclass{ximera}


\graphicspath{
  {./}
  {ximeraTutorial/}
  {basicPhilosophy/}
}

\newcommand{\mooculus}{\textsf{\textbf{MOOC}\textnormal{\textsf{ULUS}}}}


\usepackage{tkz-euclide}\usepackage{tikz}
\usepackage{tikz-cd}
\usetikzlibrary{arrows}
\tikzset{>=stealth,commutative diagrams/.cd,
  arrow style=tikz,diagrams={>=stealth}} %% cool arrow head
\tikzset{shorten <>/.style={ shorten >=#1, shorten <=#1 } } %% allows shorter vectors

\usetikzlibrary{backgrounds} %% for boxes around graphs
\usetikzlibrary{shapes,positioning}  %% Clouds and stars
\usetikzlibrary{matrix} %% for matrix
\usepgfplotslibrary{polar} %% for polar plots
\usepgfplotslibrary{fillbetween} %% to shade area between curves in TikZ
\usetkzobj{all}
\usepackage[makeroom]{cancel} %% for strike outs
%\usepackage{mathtools} %% for pretty underbrace % Breaks Ximera
%\usepackage{multicol}
\usepackage{pgffor} %% required for integral for loops



%% http://tex.stackexchange.com/questions/66490/drawing-a-tikz-arc-specifying-the-center
%% Draws beach ball
\tikzset{pics/carc/.style args={#1:#2:#3}{code={\draw[pic actions] (#1:#3) arc(#1:#2:#3);}}}



\usepackage{array}
\setlength{\extrarowheight}{+.1cm}
\newdimen\digitwidth
\settowidth\digitwidth{9}
\def\divrule#1#2{
\noalign{\moveright#1\digitwidth
\vbox{\hrule width#2\digitwidth}}}
























%%This is to help with formatting on future title pages.
\newenvironment{sectionOutcomes}{}{}


\title{One More}

\begin{document}

\begin{abstract}
analysis
\end{abstract}
\maketitle












Analyze $F(x) = -3 \log_4(\sqrt{3 - 5x} - 1)$ with its natural domain. \\




\subsection*{Analysis}





Our plan is to view this as a composition. \\



But, first, let's get a look at this thing. \\



\begin{center}
\desmos{tinu7usdun}{400}{300}
\end{center}









We are looking for functions $T$ and $B$ such that $F(x) = (T \circ B)(x) = T(B(x)) = -3 \log_4(\sqrt{3 - 5x} - 1) + 2$ \\


First, identify ``insides'' of functions.  In this case, $\sqrt{3 - 5x} - 1$ is the inside of $-3 \log_4(x) + 2$. \\



\begin{itemize}
\item Let $T(w) = -3 \log_4(w) + 2$ \\
\item Let $B(y) = \sqrt{3 - 5y} - 1$ \\
\end{itemize}

Algebraically, this produces the formula want,  $(T \circ B)(x) = T(B(x)) = -3 \log_4(\sqrt{3 - 5x} - 1) + 2$.


To get us started, we are thinking that each component function comes with its natural domain. \\


$T(w) = -3 \log_4(w) + 2$ is a logarithmic function, so $w \in (0, \infty)$ \\


$B(y) = \sqrt{3 - 5y} - 1$ is a square root function, so we need $3 - 5y >= 0$.

$3 - 5y$ is a linear function with a negative leading coefficient and positive left of its zero.  It's zero is $\frac{3}{5}$.

That makes the natural domain of $B$ $\left(-\infty, \frac{3}{5} \right]$. \\



\begin{itemize}
\item Let $T(w) = -3 \log_4(w) + 2$ with domain $(0, \infty)$\\
\item Let $B(y) = \sqrt{3 - 5y} - 1$ with domain $\left(-\infty, \frac{3}{5} \right]$\\
\end{itemize}

We are also ready to revise and modify and restrict these domains to fit the composition together. \\




We are ready to analyze. \\




\textbf{\textcolor{blue!55!black}{Domain:}} \\



Since $F(x) = (T \circ B)(x) = T(B(x))$, the domain of $F$ is inside the domain of $B$, which is $\left(-\infty, \frac{3}{5} \right]$. \\

We need to match up the range of $B$ with the domain of $T$. \\


$T(w) = -3 \log_4(w) + 2$ has a domain of $(0, \infty)$.  We need the output of $B$ to be inside $(0, \infty)$. \\



$B(y) = \sqrt{3 - 5y} - 1$ \\



$B(y) = \sqrt{3 - 5y} - 1$ is a square root function with a positive leading coefficient.  The range is $[-1, \infty)$.   This does not fit inside $(0, \infty)$, which is the domain of $T$.  So, we cannot use the whole domain of $B$ to get the whole range of $B$, which fits inside of the domain of $T$. \\


We need to remove $[-1, 0]$ from the range of $B$, which is $[-1, \infty)$. \\


Where does $B(y) = 0$? \\


\[
B(y) = \sqrt{3 - 5y} - 1 = 0
\]


\[
\sqrt{3 - 5y} = 1
\]


\[
3 - 5y = 1
\]

\[
y = \frac{2}{5}
\]

We need to cut the domain of $B$ off at $\frac{2}{5}$. \\


The usable part of the domain of $B$ is  $\left(-\infty, \frac{2}{5} \right)$. \\


The domain of $F$ is $\left(-\infty, \frac{2}{5} \right)$. \\


This agrees with the DESMOS graph. \\













\textbf{\textcolor{blue!55!black}{Zeros:}} \\



$H$ is $0$ wherever $T$ is $0$.






\[
T(w) = -3 \log_4(w) + 2 = 0
\]


\[
 \log_4(w) = \frac{2}{3}
\]


\[
w = 4^{\tfrac{2}{3}}
\]



We need $4^{\tfrac{2}{3}}$ coming into $T$, which means we need $4^{\tfrac{2}{3}}$ coming out of $B$.





\[
B(y) = \sqrt{3 - 5y} - 1 = 4^{\tfrac{2}{3}}
\]



\[
\sqrt{3 - 5y} = 4^{\tfrac{2}{3}} + 1
\]



\[
3 - 5y = \left( 4^{\tfrac{2}{3}} + 1 \right)^2
\]



\[
y = -\frac{1}{5} \left( \left( 4^{\tfrac{2}{3}} + 1 \right)^2 - 3 \right)
\]


To check with the graph, we need a decimal approximation.






\[
-\frac{1}{5} \left( \left( 4^{\tfrac{2}{3}} + 1 \right)^2 - 3 \right) \approx -1.877857681
\]


That agrees with the graph.\\



































\textbf{\textcolor{blue!55!black}{Continuity:}} \\


$F$ is a composition of a logarithmic function and a root function.  Both of these are continuous, which means their composition is continuous. \\

Logrithmic functions do not have discontinuites, but they do have singularities.  


$T(w) = -3 \log_4(w) + 2$ has a singularity at $0$, which means $F$ has a singularty when the value of $B$ is $0$.



\[
\sqrt{3 - 5y} - 1 = 0
\]



\[
3 - 5y = 1
\]



\[
y = \frac{2}{5}
\]

$\frac{2}{5}$ is a singularity, which agree with the graph. \\












\textbf{\textcolor{blue!55!black}{Singularity-Behavior:}} \\

How does $F$ behavior around this singularity? \\



Since, $T(w) = -3 \log_4(w) + 2$, we know that 


\[
\lim\limits_{w \to 0^+} T(w) = \infty
\]



We need to find out where the values of $B$ approach $0^+$, meaning from the positive side. \\


We already fount the zeros of $B$, $\frac{2}{5}$.  Now we have to think about which side of $\frac{2}{5}$ makes the value of $B$ positive. \\


Since $B(y) = \sqrt{3 - 5y} + 1$ is a root with a positive leading coefficent, it will be positive to the left of the zero. 

\begin{itemize}
\item approaching $\frac{2}{5}$ from the negative/left side
\end{itemize}

This will result in $B$ approaching $0$ from the positive direction. \\



\[
\lim\limits_{x \to -\frac{2}{5}^-} K(x)  = \lim\limits_{w \to 0^+} T(w) = \infty
\]
















\textbf{\textcolor{blue!55!black}{End-Behavior:}} \\

The domain of $F$ does not approach $\infty$, but it does approach $-\infty$. \\


We'll look at this limit through the two step of the composition. \\





\[
\lim\limits_{y \to -\infty} B(y)  = \lim\limits_{y \to -\infty} (\sqrt{3 - 5y} - 1)  = \infty
\]


This is because $B(y) = \sqrt{3 - 5y} - 1$ is a square root function with a negative leading coefficient. \\


This is going into $T$. \\



\[
\lim\limits_{x \to -\infty} H(x)  = \lim\limits_{w \to \infty} (-3 \log_4(w) + 2)  = -\infty
\]


This is because $-3 \log_4(w) + 2$ is a logarithmic function with a negative leading coefficient. \\




This agree with the graph.  \\
























\textbf{\textcolor{blue!55!black}{Behavior:}} \\


We want to figure out where $F$ is increasing and decreasing. \\

$F$ is a composition, so we need to know how the component functions behave individually. \\





$T(w) = -3 \log_4(w) + 2$ is a decreasing function, because $T$ is a logarithmic function with a negative leading coefficient while the leading coefficient of the inside linear function is  positive. \\

$B(y) = \sqrt{3 - 5y} - 1$ is a root function with a positive leading coefficient and a negative leading coefficient for the inner linear function.  Therefore, $B$ is a decreasing function.\\





\[
F(x) = (T \circ B)(x) = decreasing \circ decreasing = increasing
\]



$F$ is always increasing. \\































\textbf{\textcolor{blue!55!black}{Global/Local Maximum and Minimum:}} \\
\textbf{Extrema:} \\

$F$ is continuous with \\




\[
\lim\limits_{x \to -\infty} H(x)  = \lim\limits_{w \to \infty} (-3 \log_4(w) + 2)  = -\infty
\]



\[
\lim\limits_{x \to -\frac{2}{5}^-} K(x)  = \lim\limits_{w \to 0^+} T(w) = \infty
\]


Therefore, there is not global maximum or minimum. \\


Since, $T$ is always increasing, there are no critical numbers.  That means there are no local mximums or minimums. \\












\textbf{\textcolor{blue!55!black}{Range:}} \\


$F$ is continuous with \\




\[
\lim\limits_{x \to -\infty} H(x)  = \lim\limits_{w \to \infty} (-3 \log_4(w) + 2)  = -\infty
\]



\[
\lim\limits_{x \to -\frac{2}{5}^-} K(x)  = \lim\limits_{w \to 0^+} T(w) = \infty
\]


Therefore, the domain of $F$ is $(\infty, \infty)$.\\





This all agrees with the graph at the top of the page. 






















\begin{center}
\textbf{\textcolor{green!50!black}{ooooo-=-=-=-ooOoo-=-=-=-ooooo}} \\

more examples can be found by following this link\\ \link[More Examples of Composition]{https://ximera.osu.edu/csccmathematics/precalculus2/precalculus2/composition/examples/exampleList}

\end{center}





\end{document}
