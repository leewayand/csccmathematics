\documentclass{ximera}


\graphicspath{
  {./}
  {ximeraTutorial/}
  {basicPhilosophy/}
}

\newcommand{\mooculus}{\textsf{\textbf{MOOC}\textnormal{\textsf{ULUS}}}}


\usepackage{tkz-euclide}\usepackage{tikz}
\usepackage{tikz-cd}
\usetikzlibrary{arrows}
\tikzset{>=stealth,commutative diagrams/.cd,
  arrow style=tikz,diagrams={>=stealth}} %% cool arrow head
\tikzset{shorten <>/.style={ shorten >=#1, shorten <=#1 } } %% allows shorter vectors

\usetikzlibrary{backgrounds} %% for boxes around graphs
\usetikzlibrary{shapes,positioning}  %% Clouds and stars
\usetikzlibrary{matrix} %% for matrix
\usepgfplotslibrary{polar} %% for polar plots
\usepgfplotslibrary{fillbetween} %% to shade area between curves in TikZ
\usetkzobj{all}
\usepackage[makeroom]{cancel} %% for strike outs
%\usepackage{mathtools} %% for pretty underbrace % Breaks Ximera
%\usepackage{multicol}
\usepackage{pgffor} %% required for integral for loops



%% http://tex.stackexchange.com/questions/66490/drawing-a-tikz-arc-specifying-the-center
%% Draws beach ball
\tikzset{pics/carc/.style args={#1:#2:#3}{code={\draw[pic actions] (#1:#3) arc(#1:#2:#3);}}}



\usepackage{array}
\setlength{\extrarowheight}{+.1cm}
\newdimen\digitwidth
\settowidth\digitwidth{9}
\def\divrule#1#2{
\noalign{\moveright#1\digitwidth
\vbox{\hrule width#2\digitwidth}}}
























%%This is to help with formatting on future title pages.
\newenvironment{sectionOutcomes}{}{}


\title{Function Type}

\begin{document}

\begin{abstract}
categorize
\end{abstract}
\maketitle








The best way to begin analyzing a function is to identify what type of function it is, its category.

This is especially important in Calculus.   \\


In Calculus, you will be applying the rules of differentiation to functions. That means selecting a derivative rule.  Each of the derivative rules corresponds to a type of expression. \\

If you cannot identify the type of function, then there is nothing to do in Calculus. \\

Everything rests on identifying the function type, its category. \\



In Precalculus, we want to supply reasoning for our analysis choices.  Identifying the category for the function often is the reaosning, because we have a list of characteristics for each function type. \\



We have three levels of function categories.

\begin{itemize}
    \item Elementary Function
    \item Operation
    \item Composition
\end{itemize} 



\section{Elementary Function}

If our function belongs to one of the elementary function categories, then that is important and very helpful. \\


\textbf{\textcolor{purple!85!blue}{CAN:}}   This is a ``can'' question. \\

\begin{center}
\textbf{\textcolor{red!80!black}{CAN the function be written in the standard form for an elementary function?}}

\end{center}

















\section{Operation}




\textbf{\textcolor{purple!85!blue}{IS:}}   This is an ``is'' question. \\

\begin{center}
\textbf{\textcolor{red!80!black}{IS the function written as a sum, difference, product, or quotient}}

\end{center}






















\section{Composition}




This is our default category.  It signals to use the Chain Rule in Calculus. \\
















\section{Piecewise Defined}




If a funbction is a piecewise defined funciton, then we usually see that right away, because the formula uses a big curly brace and a list of formulas and domain pieces.













\begin{center}
\textbf{\textcolor{green!50!black}{ooooo=-=-=-=-=-=-=-=-=-=-=-=-=ooOoo=-=-=-=-=-=-=-=-=-=-=-=-=ooooo}} \\

more examples can be found by following this link\\ \link[More Examples of the Elementary Library]{https://ximera.osu.edu/csccmathematics/precalculus2/precalculus2/composition/examples/exampleList}

\end{center}







\end{document}
