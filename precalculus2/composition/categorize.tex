\documentclass{ximera}


\graphicspath{
  {./}
  {ximeraTutorial/}
  {basicPhilosophy/}
}

\newcommand{\mooculus}{\textsf{\textbf{MOOC}\textnormal{\textsf{ULUS}}}}


\usepackage{tkz-euclide}\usepackage{tikz}
\usepackage{tikz-cd}
\usetikzlibrary{arrows}
\tikzset{>=stealth,commutative diagrams/.cd,
  arrow style=tikz,diagrams={>=stealth}} %% cool arrow head
\tikzset{shorten <>/.style={ shorten >=#1, shorten <=#1 } } %% allows shorter vectors

\usetikzlibrary{backgrounds} %% for boxes around graphs
\usetikzlibrary{shapes,positioning}  %% Clouds and stars
\usetikzlibrary{matrix} %% for matrix
\usepgfplotslibrary{polar} %% for polar plots
\usepgfplotslibrary{fillbetween} %% to shade area between curves in TikZ
\usetkzobj{all}
\usepackage[makeroom]{cancel} %% for strike outs
%\usepackage{mathtools} %% for pretty underbrace % Breaks Ximera
%\usepackage{multicol}
\usepackage{pgffor} %% required for integral for loops



%% http://tex.stackexchange.com/questions/66490/drawing-a-tikz-arc-specifying-the-center
%% Draws beach ball
\tikzset{pics/carc/.style args={#1:#2:#3}{code={\draw[pic actions] (#1:#3) arc(#1:#2:#3);}}}



\usepackage{array}
\setlength{\extrarowheight}{+.1cm}
\newdimen\digitwidth
\settowidth\digitwidth{9}
\def\divrule#1#2{
\noalign{\moveright#1\digitwidth
\vbox{\hrule width#2\digitwidth}}}
























%%This is to help with formatting on future title pages.
\newenvironment{sectionOutcomes}{}{}


\title{Function Type}

\begin{document}

\begin{abstract}
categorize
\end{abstract}
\maketitle








The best way to begin analyzing a function is to identify what type of function it is, its category.

This is especially important in Calculus.   \\


In Calculus, you will be applying the rules of differentiation to functions. That means selecting a derivative rule.  Each of the derivative rules corresponds to a type of expression. \\

If you cannot identify the type of function, then there is nothing to do in Calculus. \\

Everything rests on identifying the function type, its category. \\



In Precalculus, we want to supply reasoning for our analysis choices.  Identifying the category for the function often is the reaosning, because we have a list of characteristics for each function type. \\



We have three levels of function categories.

\begin{enumerate}
    \item Elementary Function
    \item Operation
    \item Composition
\end{enumerate} 



\section*{Elementary Function}

\textbf{\textcolor{purple!85!blue}{First Choice}} 


If our function belongs to one of the elementary function categories, then that is important and very helpful. \\


\textbf{\textcolor{purple!85!blue}{CAN:}}   This is a ``can'' question. \\

\begin{center}
\textbf{\textcolor{red!70!black}{CAN the function be written in the standard form for an elementary function?}}

\end{center}



Our categories for Elementary Functions are

\begin{itemize}
    \item Constant
    \item Linear
    \item Quadratic
    \item Polynomial
    \item Rational
    \item Radical or Root
    \item Exponential
    \item Shifted Exponential 
    \item Logarithmic
    \item Absolute Value
    \item Power
    \item Sine
    \item Cosine
    \item Tangent
\end{itemize} 





The standard forms for formulas look like



\[
\begin{array}{ll}
\text{Constant}  & C(x) = A  \\
\text{Linear}  & L(x) = A \, x + B  \\
\text{Quadratic}  & Q(x) = A \, x^2 + B \, x + C  \\
\text{Polynomial}  & P(x) = a_n x^n + \cdots + a_1 x + a_0  \\
\text{Rational}  & R(x) = \frac{a_n x^n + \cdots + a_1 x + a_0}{b_m x^m + \cdots + b_1 x + b_0}  \\
\text{Root}  & R(x) = \sqrt[n]{A \, x + B}  \\
\text{Exponential}  & E(x) = A \, r^{B \, x + C}  \\
\text{Shifted Exponential}  & SE(x) = A \, r^{B \, x + C} + D  \\
\text{Logarithmic}  & L(x) = A \, \log_r(B \, x + C) + D  \\
\text{Absolute Value}  & AV(x) = A \, |B \, x + C| + D \\
\text{Power}  & P(x) = A \, x^r  \\
\text{Sine}  & S(x) = A \, \sin(B \, x + C) + D  \\
\text{Cosine}  & C(x) = A \, \cos(B \, x + C) + D  \\
\text{Tangent}  & T(x) = A \, \tan(B \, x + C) + D  
\end{array}
\]


\textbf{Note:} All of the \textit{insides} are linear functions. \\


When trying to categorize a function, the formula you have may not be in one of these forms.  That doesn't matter.  The question is whether or not, with a little algebra, you can get an equivalent formula that is in one of these forms. \\


This is our first choice.  If we can identify our function as an elementary function, then that gives us the most information. \\


For us, in Precalculus, if the inside of the function is not linear, then it isn't an elementary function. \\







































\section*{Operation}

\textbf{\textcolor{purple!85!blue}{Second Choice}} 


\textbf{\textcolor{purple!70!blue}{IS:}}   This is an ``is'' question. \\

\begin{center}
\textbf{\textcolor{red!80!black}{IS the function written as a sum, difference, product, or quotient}}

\end{center}




\[
A \, f(x) 
\]


\[
f(x) + g(x)
\]


\[
f(x) - g(x)
\]


\[
f(x) \cdot g(x)
\]


\[
\frac{f(x)}{g(x)}
\]





This is our second choice.  If we can identify our function as an elementary function, then that gives us the most information. If not, then we would like some helpful information about its structure. \\



If we can view the function as an operation, then that is helpful information.\\







































\section*{Composition}


\textbf{\textcolor{purple!85!blue}{Third Choice}} 

This is our default category.  It signals to use the Chain Rule in Calculus. \\


When we view our function as a composition, then we would also like to know its component functions. These are usually elementary functions. \\



\[
(f \circ g)(x) = f(g(x))
\]



Through the component functions, we have a way to analyze compositions.  We have a Precalculus version of the chain rule.


\begin{itemize}
\item  $inc \circ inc = inc$
\item  $inc \circ dec = dec$
\item  $dec \circ inc = dec$
\item  $dec \circ dec = inc$
\end{itemize}













\section*{Piecewise Defined}


\textbf{\textcolor{purple!85!blue}{Obvious Choice}} 

If a function is a piecewise defined function, then we usually see that right away, because the formula uses a big curly brace and a list of formulas and domain pieces.




\[
f(x) = 
\begin{cases}
  formula 1 & domain 1  \\
  formula 2 & domain 2  \\
  formula 3 & domain 3  \\
  formula 4 & domain 4  
\end{cases}
\]










\begin{center}
\textbf{\textcolor{green!50!black}{ooooo-=-=-=-ooOoo-=-=-=-ooooo}} \\

more examples can be found by following this link\\ \link[More Examples of Composition]{https://ximera.osu.edu/csccmathematics/precalculus2/precalculus2/composition/examples/exampleList}

\end{center}








\end{document}
