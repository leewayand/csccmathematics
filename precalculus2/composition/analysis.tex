\documentclass{ximera}


\graphicspath{
  {./}
  {ximeraTutorial/}
  {basicPhilosophy/}
}

\newcommand{\mooculus}{\textsf{\textbf{MOOC}\textnormal{\textsf{ULUS}}}}


\usepackage{tkz-euclide}\usepackage{tikz}
\usepackage{tikz-cd}
\usetikzlibrary{arrows}
\tikzset{>=stealth,commutative diagrams/.cd,
  arrow style=tikz,diagrams={>=stealth}} %% cool arrow head
\tikzset{shorten <>/.style={ shorten >=#1, shorten <=#1 } } %% allows shorter vectors

\usetikzlibrary{backgrounds} %% for boxes around graphs
\usetikzlibrary{shapes,positioning}  %% Clouds and stars
\usetikzlibrary{matrix} %% for matrix
\usepgfplotslibrary{polar} %% for polar plots
\usepgfplotslibrary{fillbetween} %% to shade area between curves in TikZ
\usetkzobj{all}
\usepackage[makeroom]{cancel} %% for strike outs
%\usepackage{mathtools} %% for pretty underbrace % Breaks Ximera
%\usepackage{multicol}
\usepackage{pgffor} %% required for integral for loops



%% http://tex.stackexchange.com/questions/66490/drawing-a-tikz-arc-specifying-the-center
%% Draws beach ball
\tikzset{pics/carc/.style args={#1:#2:#3}{code={\draw[pic actions] (#1:#3) arc(#1:#2:#3);}}}



\usepackage{array}
\setlength{\extrarowheight}{+.1cm}
\newdimen\digitwidth
\settowidth\digitwidth{9}
\def\divrule#1#2{
\noalign{\moveright#1\digitwidth
\vbox{\hrule width#2\digitwidth}}}
























%%This is to help with formatting on future title pages.
\newenvironment{sectionOutcomes}{}{}


\title{Analysis}

\begin{document}

\begin{abstract}
8 characteristics
\end{abstract}
\maketitle












Analyze $H(x) = 5 \sqrt{x^2-3}$ with its natural domain. \\

Our plan is to view this as a composition. \\






We are looking for functions $f$ and $g$ such that $(f \circ g)(x) = f(g(x)) = 5 \sqrt{x^2-3}$ \\


First, identify ``insides'' of functions.  In this case, $x^2 - 3$ is the inside of $5 \sqrt{t}$. \\




Let $f(t) = 5 \sqrt{t}$. \\

Let $g(y) = y^2 - 3$. \\


Algebraically, this produces the formula want, $(f \circ g)(x) = f(g(x)) = 5 \sqrt{x^2-3}$.



We are ready to analyze. \\




\textbf{\textcolor{blue!55!black}{Domain:}} \\


$H$ is not a square root function, since the inside is not a linear function.  However, the formula includes a square root. We need $x^2-3 \geq 0$.  Therefore, the domain of $H$ is $(-\infty, -\sqrt{3}] \cup [\sqrt{3}, \infty)$. \\



\textbf{\textcolor{blue!55!black}{Zeros:}} \\




\[
5 \sqrt{x^2-3} = 0
\]


\[
\sqrt{x^2-3} = 0
\]


\[
x^2-3 = 0
\]

\[
x^2 = 3
\]


We have two zeros, $-\sqrt{3}$ and $\sqrt{3}$. \\




\textbf{\textcolor{blue!55!black}{Continuity:}} \\


$H$ is a composition of a root function and a quadratic funciton.  Both of these are continuous, which means their composition is continuous. \\

Neither root functions nor quadratics have discontinuites or singularities. \\





\textbf{\textcolor{blue!55!black}{End-Behavior:}} \\


As $x \to -\infty$, $x^2 -3 \to \infty$, which means $5 \sqrt{x^2 -3} \to \infty$ \\

\[
\lim\limits_{x \to -\infty} H(x) = \infty
\]

As $x \to \infty$, $x^2 -3 \to \infty$, which means $5 \sqrt{x^2 -3} \to \infty$ \\


\[
\lim\limits_{x \to \infty} H(x) = \infty
\]











\textbf{\textcolor{blue!55!black}{Behavior:}} \\


We want to figure out where $H$ is increasing and decreasing. \\

$H$ is a composition, so we need to know how the component funcitons behave. \\





$f(t) = 5 \sqrt{t}$ is an increasing function. \\

$g(y) = y^2 - 3$ is a quadratic with a positive leading coefficient.  Therefore, it decreases and then increases.   It switches at the criticial number.\\



\[
\frac{-b}{2 a} = \frac{0}{2} = 0
\]


$g$ decreases on $(-\infty, 0)$ and increases on $(0, \infty)$.  However, $0$ is not in the domain of $H$.





On $(-\infty, -\sqrt{3}]$, $H$ is $increasing \circ decreasing$, which is decreasing. \\


On $[\sqrt{3}, \infty)$, $H$ is $increasing \circ increasing$, which is increasing. \\






\textbf{\textcolor{blue!55!black}{Extrema:}} \\



On $(-\infty, -\sqrt{3}]$, $H$ is $decreasing$, which makes $H(-\sqrt{3}) = 0$ a local minimum.  \\


On $[\sqrt{3}, \infty)$, $H$ is $increasing$, which makes $H(\sqrt{3}) = 0$ a local minimum.  \\





Since $5 \sqrt{x^2-3} \geq 0$, t$0$ is a global minimum occuring at $-\sqrt{3}$ and $\sqrt{3}$. \\



There are no other critical numbers or endpoints.


Since, $\lim\limits_{x \to -\infty} H(x) = \infty$ and $H$ is continuous, there is no global maximum. 



The range is $[0, \infty)$.




This all agrees with the graph. \\





\begin{image}
\begin{tikzpicture}
    \begin{axis}[name = sinax, domain=-10:10, ymax=10, xmax=10, ymin=-10, xmin=-10,
            axis lines =center, xlabel=$x$, ylabel={$y=f(x)$}, grid = major, grid style={dashed},
            ytick={-10,-8,-6,-4,-2,2,4,6,8,10},
            xtick={-10,-8,-6,-4,-2,2,4,6,8,10},
            yticklabels={$-10$,$-8$,$-6$,$-4$,$-2$,$2$,$4$,$6$,$8$,$10$}, 
            xticklabels={$-10$,$-8$,$-6$,$-4$,$-2$,$2$,$4$,$6$,$8$,$10$},
            ticklabel style={font=\scriptsize},
            every axis y label/.style={at=(current axis.above origin),anchor=south},
            every axis x label/.style={at=(current axis.right of origin),anchor=west},
            axis on top
          ]
          
          \addplot [line width=2, penColor, smooth,samples=100,domain=(-2.5:-1.7),<-] ({x},{sqrt(x^2 - 3)});
          \addplot [color=penColor,only marks,mark=*] coordinates{(-1.7,0)};


          \addplot [line width=2, penColor, smooth,samples=100,domain=(1.7:2.5),->] ({x},{sqrt(x^2 - 3)});
          \addplot [color=penColor,only marks,mark=*] coordinates{(1.7,0)};

           

    \end{axis}

\end{tikzpicture}
\end{image}






























\begin{center}
\textbf{\textcolor{green!50!black}{ooooo=-=-=-=-=-=-=-=-=-=-=-=-=ooOoo=-=-=-=-=-=-=-=-=-=-=-=-=ooooo}} \\

more examples can be found by following this link\\ \link[More Examples of the Elementary Library]{https://ximera.osu.edu/csccmathematics/precalculus2/precalculus2/composition/examples/exampleList}

\end{center}







\end{document}
