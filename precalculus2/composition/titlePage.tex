\documentclass{ximera}


\graphicspath{
  {./}
  {ximeraTutorial/}
  {basicPhilosophy/}
}

\newcommand{\mooculus}{\textsf{\textbf{MOOC}\textnormal{\textsf{ULUS}}}}


\usepackage{tkz-euclide}\usepackage{tikz}
\usepackage{tikz-cd}
\usetikzlibrary{arrows}
\tikzset{>=stealth,commutative diagrams/.cd,
  arrow style=tikz,diagrams={>=stealth}} %% cool arrow head
\tikzset{shorten <>/.style={ shorten >=#1, shorten <=#1 } } %% allows shorter vectors

\usetikzlibrary{backgrounds} %% for boxes around graphs
\usetikzlibrary{shapes,positioning}  %% Clouds and stars
\usetikzlibrary{matrix} %% for matrix
\usepgfplotslibrary{polar} %% for polar plots
\usepgfplotslibrary{fillbetween} %% to shade area between curves in TikZ
\usetkzobj{all}
\usepackage[makeroom]{cancel} %% for strike outs
%\usepackage{mathtools} %% for pretty underbrace % Breaks Ximera
%\usepackage{multicol}
\usepackage{pgffor} %% required for integral for loops



%% http://tex.stackexchange.com/questions/66490/drawing-a-tikz-arc-specifying-the-center
%% Draws beach ball
\tikzset{pics/carc/.style args={#1:#2:#3}{code={\draw[pic actions] (#1:#3) arc(#1:#2:#3);}}}



\usepackage{array}
\setlength{\extrarowheight}{+.1cm}
\newdimen\digitwidth
\settowidth\digitwidth{9}
\def\divrule#1#2{
\noalign{\moveright#1\digitwidth
\vbox{\hrule width#2\digitwidth}}}
























%%This is to help with formatting on future title pages.
\newenvironment{sectionOutcomes}{}{}


\title{Composition}

\begin{document}

\begin{abstract}
%
\end{abstract}
\maketitle








Our goal in Calculus is to analyze functions.


This means describing the function's behavior.

Calculus will provide procedures for obtaining the derivative of any function.  However, for compositions, we can get the same behavior information right now.






\begin{template} \textbf{\textcolor{blue!55!black}{Composition}}  \\


Let $f$ be a function with domain $D_f$. \\


Let $g$ be a function with domain $D_g$.




Then the \textbf{\textcolor{green!50!black}{composition of f and g, $f \circ g$,}} is defined as

\[
(f \circ g)(x) = f(g(x))
\] 


The value of $g$ becomes a domain number for $f$.


The composition is defined on a subset of the domain of $g$.  The composition is defined at those numbers in the domain of $g$ where the value of $g$ is in the domain of $f$.




\end{template}





\begin{sectionOutcomes}
In this section, students will 

\begin{itemize}
\item view functions as compositions.
\item deduce behavior of compositions.
\end{itemize}
\end{sectionOutcomes}

\end{document}
