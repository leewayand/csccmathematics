\documentclass{ximera}


\graphicspath{
  {./}
  {ximeraTutorial/}
  {basicPhilosophy/}
}

\newcommand{\mooculus}{\textsf{\textbf{MOOC}\textnormal{\textsf{ULUS}}}}


\usepackage{tkz-euclide}\usepackage{tikz}
\usepackage{tikz-cd}
\usetikzlibrary{arrows}
\tikzset{>=stealth,commutative diagrams/.cd,
  arrow style=tikz,diagrams={>=stealth}} %% cool arrow head
\tikzset{shorten <>/.style={ shorten >=#1, shorten <=#1 } } %% allows shorter vectors

\usetikzlibrary{backgrounds} %% for boxes around graphs
\usetikzlibrary{shapes,positioning}  %% Clouds and stars
\usetikzlibrary{matrix} %% for matrix
\usepgfplotslibrary{polar} %% for polar plots
\usepgfplotslibrary{fillbetween} %% to shade area between curves in TikZ
\usetkzobj{all}
\usepackage[makeroom]{cancel} %% for strike outs
%\usepackage{mathtools} %% for pretty underbrace % Breaks Ximera
%\usepackage{multicol}
\usepackage{pgffor} %% required for integral for loops



%% http://tex.stackexchange.com/questions/66490/drawing-a-tikz-arc-specifying-the-center
%% Draws beach ball
\tikzset{pics/carc/.style args={#1:#2:#3}{code={\draw[pic actions] (#1:#3) arc(#1:#2:#3);}}}



\usepackage{array}
\setlength{\extrarowheight}{+.1cm}
\newdimen\digitwidth
\settowidth\digitwidth{9}
\def\divrule#1#2{
\noalign{\moveright#1\digitwidth
\vbox{\hrule width#2\digitwidth}}}
























%%This is to help with formatting on future title pages.
\newenvironment{sectionOutcomes}{}{}




\title{Cooling}

\begin{document}
\begin{abstract}
  Newton's Law
\end{abstract}
\maketitle




Newton's Law of Cooling states



\begin{quote}


The rate of heat loss of a body is directly proportional to the difference in the temperatures between the body and its environment. 
\end{quote}



Newton formulated this relation as 

\[
\dot{Q} = h \, A \, (T(t) - T_{env})
\]


where


\begin{itemize}
\item $\dot{Q}$ is the rate of heat transfer out of the body.
\item $h$ is the  heat transfer coefficient.
\item $A$ is the heat transfer surface area.
\item $T$ is the temperature of the object's surface.
\item $T_{env}$ is the temperature of the environment.
\end{itemize}



In modern notation, we have

\[
\frac{dT}{dt} = r \, (T_{env} - T(t))
\]

where $r$ is the coefficient of heat transfer.\\





The derivative (rate of change) of temperature is proportional to the difference betweent the current temperature and the surrounding temperature.


\begin{observation}


Our story is a cooling story, therefore the temperature of the body is higher than the surrounding temperature.  That makes $T_{env} - T(t) < 0$, which means our derivatiev is negative, which means the temperature is decreasing, i.e. cooling.
\end{observation}



\begin{observation}


As the body cools, $T$ becomes closer to $T_{env}$, which means  $T_{env} - T(t)$ gets closer to $0$.  Therefore, $\frac{dT}{dt}$ gets closer to $0$.  The body cools slower as it cools.
\end{observation}



\begin{observation}


The body cools, however, it can't cool below the surrounding temperature. Therefore, we have

\[
\lim\limits_{t \to \infty} T(t) = T_{env}
\]

\end{observation}








\begin{example} Law of Cooling 



A freshly brewed pot of tea is removed from the stove at $200^{\circ}$.  It is set on the table where the room temperature is $72^{\circ}$ and the heat transfer coefficient is $0.025$. \\


Newton's Law of Cooling says

\[
\frac{dT}{dt} = 0.025 \, (72 - T(t))
\]


where $t$ is measured in minutes.


$\blacktriangleright$ \textbf{Frrst Minute} \\

The tea has an initial temperature of $200^{\circ}$, which means the temperature is changing approximately at a rate of 


\[
\frac{dT}{dt} = 0.025 \, (72 - 200) = 3.2 \, \text{degrees per minute}
\]


After $1$ minute the tea is at approximately $196.8^{\circ}$.




\end{example}




















\end{document}
