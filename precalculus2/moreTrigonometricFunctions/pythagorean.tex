\documentclass{ximera}


\graphicspath{
  {./}
  {ximeraTutorial/}
  {basicPhilosophy/}
}

\newcommand{\mooculus}{\textsf{\textbf{MOOC}\textnormal{\textsf{ULUS}}}}


\usepackage{tkz-euclide}\usepackage{tikz}
\usepackage{tikz-cd}
\usetikzlibrary{arrows}
\tikzset{>=stealth,commutative diagrams/.cd,
  arrow style=tikz,diagrams={>=stealth}} %% cool arrow head
\tikzset{shorten <>/.style={ shorten >=#1, shorten <=#1 } } %% allows shorter vectors

\usetikzlibrary{backgrounds} %% for boxes around graphs
\usetikzlibrary{shapes,positioning}  %% Clouds and stars
\usetikzlibrary{matrix} %% for matrix
\usepgfplotslibrary{polar} %% for polar plots
\usepgfplotslibrary{fillbetween} %% to shade area between curves in TikZ
\usetkzobj{all}
\usepackage[makeroom]{cancel} %% for strike outs
%\usepackage{mathtools} %% for pretty underbrace % Breaks Ximera
%\usepackage{multicol}
\usepackage{pgffor} %% required for integral for loops



%% http://tex.stackexchange.com/questions/66490/drawing-a-tikz-arc-specifying-the-center
%% Draws beach ball
\tikzset{pics/carc/.style args={#1:#2:#3}{code={\draw[pic actions] (#1:#3) arc(#1:#2:#3);}}}



\usepackage{array}
\setlength{\extrarowheight}{+.1cm}
\newdimen\digitwidth
\settowidth\digitwidth{9}
\def\divrule#1#2{
\noalign{\moveright#1\digitwidth
\vbox{\hrule width#2\digitwidth}}}
























%%This is to help with formatting on future title pages.
\newenvironment{sectionOutcomes}{}{}


\title{Identities}

\begin{document}

\begin{abstract}
Pythagorean Theorem
\end{abstract}
\maketitle






Below is our current diagram for our trigonometric functions.






\begin{image}
\begin{tikzpicture}[line cap=round]
  \begin{axis}[
            xmin=-1.1,xmax=1.1,ymin=-1.1,ymax=1.1,
            axis lines=center,
            width=4in,
            xtick={-1},
            ytick={-1,1},
            clip=false,
            unit vector ratio*=1 1 1,
            xlabel=$ $, ylabel=$ $,
            ticklabel style={font=\scriptsize},
            every axis y label/.style={at=(current axis.above origin),anchor=south},
            every axis x label/.style={at=(current axis.right of origin),anchor=west},
          ]        
          


          \draw [ultra thick] (axis cs:0,0) -- (axis cs:1.305,0);
          %\draw [ultra thick] (axis cs:0.766,0.643) -- (axis cs:1.305,0);
          \draw [ultra thick] (axis cs:0,1.557) -- (axis cs:1.305,0);
          \draw [ultra thick] (axis cs:0,1.557) -- (axis cs:0,0);



          \draw [thin] (axis cs:0.716,0.05) -- (axis cs:0.766,0.05);
          \draw [thin] (axis cs:0.716,0) -- (axis cs:0.716,0.05);

          \draw [thin] (axis cs:0.766,0.05) -- (axis cs:0.812,0.05);
          \draw [thin] (axis cs:0.812,0) -- (axis cs:0.812,0.05);

          \draw [thin] (axis cs:0.7,0.587) -- (axis cs:0.77,0.51);
          \draw [thin] (axis cs:0.77,0.52) -- (axis cs:0.84,0.57);





          \addplot [smooth, domain=(0:360)] ({cos(x)},{sin(x)}); %% unit circle

          \addplot [textColor] plot coordinates {(0,0) (.766,.643)}; %% 40 degrees

          \addplot [ultra thick,penColor] plot coordinates {(.766,0) (.766,.643)}; %% 40 degrees
          \addplot [ultra thick,penColor2] plot coordinates {(0,0) (.766,0)}; %% 40 degrees
          
          %\addplot [ultra thick,penColor3] plot coordinates {(1,0) (1,.839)}; %% 40 degrees          

          \addplot [textColor,smooth, domain=(0:40)] ({.15*cos(x)},{.15*sin(x)});

          \node at (axis cs:.15,.07) [anchor=west] {$\theta$};
          \node[penColor] at (axis cs:0.85,.27) {$\sin(\theta)$};
          \node[penColor2] at (axis cs:.383,0) [anchor=north] {$\cos(\theta)$};

           \node[penColor] at (axis cs:0.37,0.4) {$1$};


          \node at (axis cs:0.84, 0.5) [anchor=north] {$\theta$};


          \node[textColor] at (axis cs:0.766,-0.15)[anchor=north] {$\sec(\theta)$};

          \node at (axis cs:0.07,1.4) [anchor=north] {$\theta$};


          \node[textColor, rotate=-50] at (axis cs:1.05,0.5) {$\tan(\theta)$};
          \node[textColor, rotate=-50] at (axis cs:0.4,1.3) {$\cot(\theta)$};
          \node[textColor] at (axis cs:0,0.75)[anchor=east] {$\csc(\theta)$};





        \end{axis}
\end{tikzpicture}
\end{image}



There are many right triangles and we know that the Pythagorean Theorem is a right triangle relationship.


\begin{theorem} \textbf{\textcolor{green!50!black}{Pythagorean Theorem}}


The sum of the squares of the lengths of the legs of a right triangle equals the square of thelength of the hypotenuse.



\begin{center}
\textbf{\textcolor{red!80!black}{$\text{leg}^2 + \text{leg}^2 = \text{hypotenuse}^2$}}
\end{center} 





\end{theorem}



From our diagram, the Pythagorean Theorem tells us 




\[
\sin(\theta)^2 + \cos(\theta)^2 = 1
\]


\[
1 + \tan(\theta)^2 = \sec(\theta)^2
\]


\[
1 + \cot(\theta)^2 = \csc(\theta)^2
\]














\end{document}
