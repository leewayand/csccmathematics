\documentclass{ximera}


\graphicspath{
  {./}
  {ximeraTutorial/}
  {basicPhilosophy/}
}

\newcommand{\mooculus}{\textsf{\textbf{MOOC}\textnormal{\textsf{ULUS}}}}


\usepackage{tkz-euclide}\usepackage{tikz}
\usepackage{tikz-cd}
\usetikzlibrary{arrows}
\tikzset{>=stealth,commutative diagrams/.cd,
  arrow style=tikz,diagrams={>=stealth}} %% cool arrow head
\tikzset{shorten <>/.style={ shorten >=#1, shorten <=#1 } } %% allows shorter vectors

\usetikzlibrary{backgrounds} %% for boxes around graphs
\usetikzlibrary{shapes,positioning}  %% Clouds and stars
\usetikzlibrary{matrix} %% for matrix
\usepgfplotslibrary{polar} %% for polar plots
\usepgfplotslibrary{fillbetween} %% to shade area between curves in TikZ
\usetkzobj{all}
\usepackage[makeroom]{cancel} %% for strike outs
%\usepackage{mathtools} %% for pretty underbrace % Breaks Ximera
%\usepackage{multicol}
\usepackage{pgffor} %% required for integral for loops



%% http://tex.stackexchange.com/questions/66490/drawing-a-tikz-arc-specifying-the-center
%% Draws beach ball
\tikzset{pics/carc/.style args={#1:#2:#3}{code={\draw[pic actions] (#1:#3) arc(#1:#2:#3);}}}



\usepackage{array}
\setlength{\extrarowheight}{+.1cm}
\newdimen\digitwidth
\settowidth\digitwidth{9}
\def\divrule#1#2{
\noalign{\moveright#1\digitwidth
\vbox{\hrule width#2\digitwidth}}}
























%%This is to help with formatting on future title pages.
\newenvironment{sectionOutcomes}{}{}


\title{More Trig Functions}

\begin{document}

\begin{abstract}
%
\end{abstract}
\maketitle



There are six basic Trigonometric functions.  We have seen three of them: $\sin(\theta)$, $\cos(\theta)$, and $\tan(\theta)$. The other three are the reciprocals of these three.


\begin{itemize}
\item Secant is the reciprocal of cosine.
\item Cosecant is the reciprocal of sine.
\item Cotangent is the reciprocal of tangent.
\end{itemize}


We want to analyze all of their characteristics and features. 




\begin{itemize}
\item \textbf{\textcolor{red!80!black}{Domain}} 
\item \textbf{\textcolor{red!80!black}{Zeros}} 
item \textbf{\textcolor{red!80!black}{Continuity}} 
  \begin{itemize}
     \item \textbf{\textcolor{purple!85!blue}{discontinuities}} 
     \item \textbf{\textcolor{purple!85!blue}{singularities}} 
  \end{itemize}
\item \textbf{\textcolor{red!80!black}{End-Behavior}} 
\item \textbf{\textcolor{red!80!black}{Behavior}} 
  \begin{itemize}
     \item \textbf{\textcolor{purple!85!blue}{intervals where increasing}} 
     \item \textbf{\textcolor{purple!85!blue}{intervals where decreasing}} 
  \end{itemize}
\item \textbf{\textcolor{red!80!black}{Global Maximum and Minimum}} 
\item \textbf{\textcolor{red!80!black}{Local Maximums and Minimums}} 
\item \textbf{\textcolor{red!80!black}{Range}} 
\item \textbf{\textcolor{blue!55!black}{...and we would like a nice graph}} 
\end{itemize}







\subsection*{Learning Outcomes}

\begin{sectionOutcomes}
In this section, students will investigate

\begin{itemize}
\item $\csc(\theta) = \frac{1}{\sin(\theta)}$
\item $\sec(\theta) = \frac{1}{\cos(\theta)}$
\item $\cot(\theta) = \frac{1}{\tan(\theta)}$
\end{itemize}
\end{sectionOutcomes}












\begin{center}
\textbf{\textcolor{green!50!black}{ooooo-=-=-=-ooOoo-=-=-=-ooooo}} \\

more examples can be found by following this link\\ \link[More Examples of Trigonometric Functions]{https://ximera.osu.edu/csccmathematics/precalculus2/precalculus2/moreTrigonometricFunctions/examples/exampleList}

\end{center}








\end{document}
