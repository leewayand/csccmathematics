\documentclass{ximera}


\graphicspath{
  {./}
  {ximeraTutorial/}
  {basicPhilosophy/}
}

\newcommand{\mooculus}{\textsf{\textbf{MOOC}\textnormal{\textsf{ULUS}}}}


\usepackage{tkz-euclide}\usepackage{tikz}
\usepackage{tikz-cd}
\usetikzlibrary{arrows}
\tikzset{>=stealth,commutative diagrams/.cd,
  arrow style=tikz,diagrams={>=stealth}} %% cool arrow head
\tikzset{shorten <>/.style={ shorten >=#1, shorten <=#1 } } %% allows shorter vectors

\usetikzlibrary{backgrounds} %% for boxes around graphs
\usetikzlibrary{shapes,positioning}  %% Clouds and stars
\usetikzlibrary{matrix} %% for matrix
\usepgfplotslibrary{polar} %% for polar plots
\usepgfplotslibrary{fillbetween} %% to shade area between curves in TikZ
\usetkzobj{all}
\usepackage[makeroom]{cancel} %% for strike outs
%\usepackage{mathtools} %% for pretty underbrace % Breaks Ximera
%\usepackage{multicol}
\usepackage{pgffor} %% required for integral for loops



%% http://tex.stackexchange.com/questions/66490/drawing-a-tikz-arc-specifying-the-center
%% Draws beach ball
\tikzset{pics/carc/.style args={#1:#2:#3}{code={\draw[pic actions] (#1:#3) arc(#1:#2:#3);}}}



\usepackage{array}
\setlength{\extrarowheight}{+.1cm}
\newdimen\digitwidth
\settowidth\digitwidth{9}
\def\divrule#1#2{
\noalign{\moveright#1\digitwidth
\vbox{\hrule width#2\digitwidth}}}
























%%This is to help with formatting on future title pages.
\newenvironment{sectionOutcomes}{}{}


\title{Direction Vectors}

\begin{document}

\begin{abstract}
unit vectors
\end{abstract}
\maketitle





Just adding one dimension has added quite a bit of geometric and algebraic structure to the Cartesian plane and complex numbers over the real line and real numbers.


But really, the structure was there.  It was just too simply to see as significant.  We are elevating structure from the real line and expanding it into two dimensions.



The tick marks on our drawings of the plane axes are an example of this. The tick marks remind us of steps of size $1$ horizontally or vertically.  This again seen in our rectangular vectors.


\[ \langle 5, 7 \rangle = 5 \cdot  \langle 1, 0 \rangle + 7 \cdot \langle 0, 1 \rangle  \]









\begin{image}
\begin{tikzpicture}
  \begin{axis}[
            domain=-10:10, ymax=10, xmax=10, ymin=-10, xmin=-10,
            axis lines =center, xlabel=$t$, ylabel=$y$, grid = major, grid style={dashed},
            ytick={-10,-8,-6,-4,-2,2,4,6,8,10},
            xtick={-10,-8,-6,-4,-2,2,4,6,8,10},
            ticklabel style={font=\scriptsize},
            every axis y label/.style={at=(current axis.above origin),anchor=south},
            every axis x label/.style={at=(current axis.right of origin),anchor=west},
            axis on top
          ]
          

          \draw[black,ultra thick,->] (axis cs:0,0) -- (axis cs:5,7);

          \draw[penColor,ultra thick,->] (axis cs:0,0) -- (axis cs:0,1);
          \draw[penColor,ultra thick,->] (axis cs:0,1) -- (axis cs:0,2);
          \draw[penColor,ultra thick,->] (axis cs:0,2) -- (axis cs:0,3);
          \draw[penColor,ultra thick,->] (axis cs:0,3) -- (axis cs:0,4);
          \draw[penColor,ultra thick,->] (axis cs:0,4) -- (axis cs:0,5);
          \draw[penColor,ultra thick,->] (axis cs:0,5) -- (axis cs:0,6);
          \draw[penColor,ultra thick,->] (axis cs:0,6) -- (axis cs:0,7);



          \draw[penColor2,ultra thick,->] (axis cs:0,0) -- (axis cs:1,0);
          \draw[penColor2,ultra thick,->] (axis cs:1,0) -- (axis cs:2,0);
          \draw[penColor2,ultra thick,->] (axis cs:2,0) -- (axis cs:3,0);
          \draw[penColor2,ultra thick,->] (axis cs:3,0) -- (axis cs:4,0);
          \draw[penColor2,ultra thick,->] (axis cs:4,0) -- (axis cs:5,0);







           

  \end{axis}
\end{tikzpicture}
\end{image}





It is seen in our complex number notation.


\[   4 + 2 \, i         \]

$i$ represents a step of $1$ in the imaginary direction.







We can bring this idea to polar descriptions as well.


Each complex number can represented by a vector.  This vector can in turn be written as a scalar (number) times a \textbf{unit vector}.  A unit vector is just a vector of length $1$.



\begin{example}

\[  3 + 4 \, i = \langle 3, 4 \rangle = 5 \cdot \langle \frac{3}{5}, \frac{4}{5} \rangle     \]

\begin{image}
\begin{tikzpicture}
  \begin{axis}[
            domain=-5:5, ymax=5, xmax=5, ymin=-5, xmin=-5,
            axis lines =center, xlabel=$t$, ylabel=$y$, grid = major, grid style={dashed},
            ytick={-4,-2,2,4},
            xtick={-4,-2,2,4},
            ticklabel style={font=\scriptsize},
            every axis y label/.style={at=(current axis.above origin),anchor=south},
            every axis x label/.style={at=(current axis.right of origin),anchor=west},
            axis on top
          ]
          
          \addplot[color=penColor,fill=penColor,only marks,mark=*] coordinates{(3,4)};
          \draw[penColor,ultra thick,->] (axis cs:0,0) -- (axis cs:0.6,0.8);
          \draw[penColor,ultra thick,->] (axis cs:0.6,0.8) -- (axis cs:1.2,1.6);
          \draw[penColor,ultra thick,->] (axis cs:1.2,1.6) -- (axis cs:1.8,2.4);
          \draw[penColor,ultra thick,->] (axis cs:1.8,2.4) -- (axis cs:2.4,3.2);
          \draw[penColor,ultra thick,->] (axis cs:2.4,3.2) -- (axis cs:3,4);




           

  \end{axis}
\end{tikzpicture}
\end{image}


Every complex number can be written this way.


The unit vector is obtained by dividing the components by the length of the vector.  This length is the scalar out front.




\[     a + b \, i    =  \sqrt{a^2 + b^2}  \left(  \frac{a}{\sqrt{a^2 + b^2}}, \frac{b}{\sqrt{a^2 + b^2}} \, i \right)                      \]




\end{example}

Unit vectors as called \textbf{direction vectors}.



\begin{center}

\textbf{\Huge{\textcolor{purple!50!blue!90!black}{EVERY COMPLEX NUMBER !}}}

\end{center}



Therefore, to understand the Complex Numbers, we really need to understand the unit complex numbers - the complex numbers with modulus equal to $1$



The numbers with modulus equal to $1$ make up the unit circle.





































\end{document}
