\documentclass{ximera}


\graphicspath{
  {./}
  {ximeraTutorial/}
  {basicPhilosophy/}
}

\newcommand{\mooculus}{\textsf{\textbf{MOOC}\textnormal{\textsf{ULUS}}}}


\usepackage{tkz-euclide}\usepackage{tikz}
\usepackage{tikz-cd}
\usetikzlibrary{arrows}
\tikzset{>=stealth,commutative diagrams/.cd,
  arrow style=tikz,diagrams={>=stealth}} %% cool arrow head
\tikzset{shorten <>/.style={ shorten >=#1, shorten <=#1 } } %% allows shorter vectors

\usetikzlibrary{backgrounds} %% for boxes around graphs
\usetikzlibrary{shapes,positioning}  %% Clouds and stars
\usetikzlibrary{matrix} %% for matrix
\usepgfplotslibrary{polar} %% for polar plots
\usepgfplotslibrary{fillbetween} %% to shade area between curves in TikZ
\usetkzobj{all}
\usepackage[makeroom]{cancel} %% for strike outs
%\usepackage{mathtools} %% for pretty underbrace % Breaks Ximera
%\usepackage{multicol}
\usepackage{pgffor} %% required for integral for loops



%% http://tex.stackexchange.com/questions/66490/drawing-a-tikz-arc-specifying-the-center
%% Draws beach ball
\tikzset{pics/carc/.style args={#1:#2:#3}{code={\draw[pic actions] (#1:#3) arc(#1:#2:#3);}}}



\usepackage{array}
\setlength{\extrarowheight}{+.1cm}
\newdimen\digitwidth
\settowidth\digitwidth{9}
\def\divrule#1#2{
\noalign{\moveright#1\digitwidth
\vbox{\hrule width#2\digitwidth}}}
























%%This is to help with formatting on future title pages.
\newenvironment{sectionOutcomes}{}{}


\title{Algebraic Geometry}

\begin{document}

\begin{abstract}
%%%
\end{abstract}
\maketitle





The real number line had just one dimension, which made the idea of direction simple - two of them: left and right.  These were commonly known as the positive and negative directions.  Each number was described with two pieces of information.  Each number came with a direction and a distance.  We usually give the direction information first and then the distance.


\begin{itemize}
\item For example $-3$.  The hyphen out front means the left (negative) direction and the $3$ gives the distance.
\item For example $+5$.  The plus sign out front means the right (positive) direction and the $5$ gives the distance.
\end{itemize}
\textbf{Note:} We usually do not write the $+$, $+3 = 3$.

We represented the direction with a hyphen (for negative) to the left of the distance or the absence of the hyphen (for positive).

The Complex Numbers have many more directions, but the idea is the same.  Each complex number can be described with a direction and a distance.  This type of representation is known as \textbf{polar form}.






\subsection{Learning Outcomes}



\begin{sectionOutcomes}
In this section, students will 

\begin{itemize}
\item represent real numbers on the 2D number line in polar form.
\end{itemize}
\end{sectionOutcomes}

\end{document}
