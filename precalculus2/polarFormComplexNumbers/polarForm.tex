\documentclass{ximera}


\graphicspath{
  {./}
  {ximeraTutorial/}
  {basicPhilosophy/}
}

\newcommand{\mooculus}{\textsf{\textbf{MOOC}\textnormal{\textsf{ULUS}}}}


\usepackage{tkz-euclide}\usepackage{tikz}
\usepackage{tikz-cd}
\usetikzlibrary{arrows}
\tikzset{>=stealth,commutative diagrams/.cd,
  arrow style=tikz,diagrams={>=stealth}} %% cool arrow head
\tikzset{shorten <>/.style={ shorten >=#1, shorten <=#1 } } %% allows shorter vectors

\usetikzlibrary{backgrounds} %% for boxes around graphs
\usetikzlibrary{shapes,positioning}  %% Clouds and stars
\usetikzlibrary{matrix} %% for matrix
\usepgfplotslibrary{polar} %% for polar plots
\usepgfplotslibrary{fillbetween} %% to shade area between curves in TikZ
\usetkzobj{all}
\usepackage[makeroom]{cancel} %% for strike outs
%\usepackage{mathtools} %% for pretty underbrace % Breaks Ximera
%\usepackage{multicol}
\usepackage{pgffor} %% required for integral for loops



%% http://tex.stackexchange.com/questions/66490/drawing-a-tikz-arc-specifying-the-center
%% Draws beach ball
\tikzset{pics/carc/.style args={#1:#2:#3}{code={\draw[pic actions] (#1:#3) arc(#1:#2:#3);}}}



\usepackage{array}
\setlength{\extrarowheight}{+.1cm}
\newdimen\digitwidth
\settowidth\digitwidth{9}
\def\divrule#1#2{
\noalign{\moveright#1\digitwidth
\vbox{\hrule width#2\digitwidth}}}
























%%This is to help with formatting on future title pages.
\newenvironment{sectionOutcomes}{}{}


\title{Polar Form}

\begin{document}

\begin{abstract}
locating numbers
\end{abstract}
\maketitle





We already have several ways of representing complex numbers and will now add another.



$\blacktriangleright$ We represent complex numbers with points or dots in the Cartesian plane.




\begin{image}
\begin{tikzpicture}
  \begin{axis}[
            domain=-10:10, ymax=10, xmax=10, ymin=-10, xmin=-10,
            axis lines =center, xlabel=$t$, ylabel=$y$, grid = major,
            ytick={-10,-8,-6,-4,-2,2,4,6,8,10},
            xtick={-10,-8,-6,-4,-2,2,4,6,8,10},
            ticklabel style={font=\scriptsize},
            every axis y label/.style={at=(current axis.above origin),anchor=south},
            every axis x label/.style={at=(current axis.right of origin),anchor=west},
            axis on top
          ]
          

          \addplot[color=penColor,fill=penColor,only marks,mark=*] coordinates{(4,8)};






           

  \end{axis}
\end{tikzpicture}
\end{image}








$\blacktriangleright$ We represent complex numbers with vectors.  These are illustrated with arrows on the Cartesian plane.  We write vectors using triangular brackets for an algebraic representation.

$\langle 4, 8 \rangle$

Vectors have an arithmetic.


\begin{image}
\begin{tikzpicture}
  \begin{axis}[
            domain=-10:10, ymax=10, xmax=10, ymin=-10, xmin=-10,
            axis lines =center, xlabel=$t$, ylabel=$y$, grid = major,
            ytick={-10,-8,-6,-4,-2,2,4,6,8,10},
            xtick={-10,-8,-6,-4,-2,2,4,6,8,10},
            ticklabel style={font=\scriptsize},
            every axis y label/.style={at=(current axis.above origin),anchor=south},
            every axis x label/.style={at=(current axis.right of origin),anchor=west},
            axis on top
          ]
          

          \draw[penColor,ultra thick,->] (axis cs:0,0) -- (axis cs:4,8);






           

  \end{axis}
\end{tikzpicture}
\end{image}





$\blacktriangleright$ We represent complex numbers with a 2-dimensional sum: $4 + 8 \, i$.



The left, first, or horizontal part is called the real part and the right, second, or vertical part is called the imaginary part.




All of these representations share a common structure.  They all describe the complex number with rectangular measurements.  All of the descriptions above give horizontal (left/right) and vertical (up/down) information.


There is a different way to describe complex numbers.



\section{Polar Coordinates}

Instead of rectangular information, we could give circular information.  We could give the angle the vector makes with the positive horizontal axis together with the length or distance. 

In keeping with the circular idea, the vector resembles a radius.  

Our two pieces of information will be $r$ and $\theta$. These are known as \textbf{polar coordinates}

\begin{image}
\begin{tikzpicture}
  \begin{axis}[
            domain=-10:10, ymax=10, xmax=10, ymin=-10, xmin=-10,
            axis lines =center, xlabel=$t$, ylabel=$y$, grid = major,
            ytick={-10,-8,-6,-4,-2,2,4,6,8,10},
            xtick={-10,-8,-6,-4,-2,2,4,6,8,10},
            ticklabel style={font=\scriptsize},
            every axis y label/.style={at=(current axis.above origin),anchor=south},
            every axis x label/.style={at=(current axis.right of origin),anchor=west},
            axis on top
          ]
          
          \addplot[color=penColor,fill=penColor,only marks,mark=*] coordinates{(4,8)};
          \draw[penColor,ultra thick,->] (axis cs:0,0) -- (axis cs:4,8);
          \addplot [textColor,smooth, domain=(0:63),->] ({3*cos(x)},{3*sin(x)});
          \node at (axis cs:2,1) [anchor=east] {$\theta$};


           

  \end{axis}
\end{tikzpicture}
\end{image}


$r = \sqrt{4^2 + 8^2} = \sqrt{80} = 16\sqrt{5}$ \\
$\theta = 63.43^{\circ}$ \\



We, again, wrap the coordinates up with parentheses: $(r, \theta) = (16\sqrt{5}, 63.43^{\circ})$.

Of course, that is how we write rectanguilar coordinates as well.  The context of the situaiton will tell us how to interpret the coordinates.

And, we'll perfer radins over degrees: $(r, \theta) = (16\sqrt{5}, 1.107)$






























\end{document}
