\documentclass{ximera}


\graphicspath{
  {./}
  {ximeraTutorial/}
  {basicPhilosophy/}
}

\newcommand{\mooculus}{\textsf{\textbf{MOOC}\textnormal{\textsf{ULUS}}}}


\usepackage{tkz-euclide}\usepackage{tikz}
\usepackage{tikz-cd}
\usetikzlibrary{arrows}
\tikzset{>=stealth,commutative diagrams/.cd,
  arrow style=tikz,diagrams={>=stealth}} %% cool arrow head
\tikzset{shorten <>/.style={ shorten >=#1, shorten <=#1 } } %% allows shorter vectors

\usetikzlibrary{backgrounds} %% for boxes around graphs
\usetikzlibrary{shapes,positioning}  %% Clouds and stars
\usetikzlibrary{matrix} %% for matrix
\usepgfplotslibrary{polar} %% for polar plots
\usepgfplotslibrary{fillbetween} %% to shade area between curves in TikZ
\usetkzobj{all}
\usepackage[makeroom]{cancel} %% for strike outs
%\usepackage{mathtools} %% for pretty underbrace % Breaks Ximera
%\usepackage{multicol}
\usepackage{pgffor} %% required for integral for loops



%% http://tex.stackexchange.com/questions/66490/drawing-a-tikz-arc-specifying-the-center
%% Draws beach ball
\tikzset{pics/carc/.style args={#1:#2:#3}{code={\draw[pic actions] (#1:#3) arc(#1:#2:#3);}}}



\usepackage{array}
\setlength{\extrarowheight}{+.1cm}
\newdimen\digitwidth
\settowidth\digitwidth{9}
\def\divrule#1#2{
\noalign{\moveright#1\digitwidth
\vbox{\hrule width#2\digitwidth}}}
























%%This is to help with formatting on future title pages.
\newenvironment{sectionOutcomes}{}{}


\title{Numbers}

\begin{document}

\begin{abstract}
2D
\end{abstract}
\maketitle





\begin{center}
\textbf{\textcolor{blue!55!black}{We Have Built A Number System!}} 
\end{center}


Following how school developed the idea of the real numbers, with lines and arrows, we have built a similar system in two dimemsions.



Just like the beginning of the real numbers, we have developed this system geometrically.  


Our new system includes the real numbers.  However, the real numbers are just one slice of the new numbers. 

Our new numbers also include $\sqrt{-1}$, which the real numbers do not.  So, this is actually a larger number system than the real numbers.  In fact, the ``number line'' here is two dimensional.








The geometric description we have developed is nice and will be very useful.  But, we want arithmetic. We want an algebra language for our new numbers.


\textbf{\textcolor{blue!55!black}{$\blacktriangleright$}} That means, we want symbolic notation for the numbers. \\

\textbf{\textcolor{blue!55!black}{$\blacktriangleright$}} That means, we want symbolic notation for addition. \\

\textbf{\textcolor{blue!55!black}{$\blacktriangleright$}} That means, we want symbolic notation for multiplication. \\



\textbf{\textcolor{red!80!black}{Luckily}}, we have laid the foundation for transitioning to symbols and notation that look more like what we are used to.   We have forced the \textbf{\textcolor{purple!85!blue}{distributive property to hold.}}




Currently, we have a weird looking rule for multiplication of vectors.



\[    \langle a, b \rangle  * \langle c, d \rangle =   \langle a c - b d, b c + a d \rangle   \]



This has served its purpose, by making sure the distributive property holds.  This rule will help us move over to more familiar notation.


Instead of using triangular brackets and commas to help us separate the two dimesions, we are are switching to notation that looks like 






\begin{center}
\textbf{\textcolor{blue!55!black}{$a + b \cdot i$}} 
\end{center}




\[    \langle a, b \rangle  \rightarrow   a + b \cdot i   \]






In the next section, we will introduce this new notation.  

The one thing it will do for us is replace our weird multiplication rule the distributive property.



\begin{idea} \textbf{Sneak Peek Ahead}

\[    (a + b \cdot i) \cdot (c + d \cdot i) = a \cdot c + a \cdot d \cdot i + b \cdot i \cdot a +  b \cdot i \cdot d \cdot i \]


\[    = a \cdot c + a \cdot d \cdot i + b \cdot i \cdot c +  b \cdot d \cdot i^2 \]

\[    = a \cdot c + a \cdot d \cdot i + b \cdot i \cdot c +  b \cdot d \cdot (-1) \]

\[    = a \cdot c + a \cdot d \cdot i + b \cdot i \cdot c -  b \cdot d  \]

\[    = a \cdot c -  b \cdot d + a \cdot d \cdot i + b \cdot i \cdot c   \]

\[    = (a \cdot c -  b \cdot d) + (a \cdot d  + b \cdot c) \cdot i   \]



\end{idea}



This is the same as


\[    \langle a, b \rangle  * \langle c, d \rangle =   \langle a c - b d, b c + a d \rangle   \]



But, it looks more like the arithmetic we know.  


We want to transition over to that langugage and leave the vector langugage for when we are talking abotu geometry.


The first thing that transition will do for us is that we won't have to memorize the vector rule for multiplication any longer.  We can just use our familiar distributive property and that will automatically do it for us.























\begin{center}
\textbf{\textcolor{green!50!black}{ooooo=-=-=-=-=-=-=-=-=-=-=-=-=ooOoo=-=-=-=-=-=-=-=-=-=-=-=-=ooooo}} \\

more examples can be found by following this link\\ \link[More Examples of Representing Numbers]{https://ximera.osu.edu/csccmathematics/precalculus2/precalculus2/representingNumbers/examples/exampleList}

\end{center}





\end{document}
