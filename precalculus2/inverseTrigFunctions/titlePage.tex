\documentclass{ximera}


\graphicspath{
  {./}
  {ximeraTutorial/}
  {basicPhilosophy/}
}

\newcommand{\mooculus}{\textsf{\textbf{MOOC}\textnormal{\textsf{ULUS}}}}


\usepackage{tkz-euclide}\usepackage{tikz}
\usepackage{tikz-cd}
\usetikzlibrary{arrows}
\tikzset{>=stealth,commutative diagrams/.cd,
  arrow style=tikz,diagrams={>=stealth}} %% cool arrow head
\tikzset{shorten <>/.style={ shorten >=#1, shorten <=#1 } } %% allows shorter vectors

\usetikzlibrary{backgrounds} %% for boxes around graphs
\usetikzlibrary{shapes,positioning}  %% Clouds and stars
\usetikzlibrary{matrix} %% for matrix
\usepgfplotslibrary{polar} %% for polar plots
\usepgfplotslibrary{fillbetween} %% to shade area between curves in TikZ
\usetkzobj{all}
\usepackage[makeroom]{cancel} %% for strike outs
%\usepackage{mathtools} %% for pretty underbrace % Breaks Ximera
%\usepackage{multicol}
\usepackage{pgffor} %% required for integral for loops



%% http://tex.stackexchange.com/questions/66490/drawing-a-tikz-arc-specifying-the-center
%% Draws beach ball
\tikzset{pics/carc/.style args={#1:#2:#3}{code={\draw[pic actions] (#1:#3) arc(#1:#2:#3);}}}



\usepackage{array}
\setlength{\extrarowheight}{+.1cm}
\newdimen\digitwidth
\settowidth\digitwidth{9}
\def\divrule#1#2{
\noalign{\moveright#1\digitwidth
\vbox{\hrule width#2\digitwidth}}}
























%%This is to help with formatting on future title pages.
\newenvironment{sectionOutcomes}{}{}


\title{Inverse Trig Functions}

\begin{document}

\begin{abstract}
%
\end{abstract}
\maketitle




Sine, cosine, and tangent are functions. They connect angle measurements (degrees and radians) with ratios (numbers). Unfortunately, they are not one-to-one functions and do not have inverse functions. \\

But, we would like them to.\\

In order to get inverse functions, we need to make sine, cosine, and tangent into one-to-one functions.  To accomplish this, we'll simply restrict their domains. \\


This story will begin with a review of inverse functions in general.













\subsection*{Learning Outcomes}

\begin{sectionOutcomes}
In this section, students will 

\begin{itemize}
\item review inverse functions.
\item investigate inverse trigonometric functions.
\end{itemize}
\end{sectionOutcomes}








\begin{center}
\textbf{\textcolor{green!50!black}{ooooo-=-=-=-ooOoo-=-=-=-ooooo}} \\

more examples can be found by following this link\\ \link[More Examples of Inverse Trig Functions]{https://ximera.osu.edu/csccmathematics/precalculus2/precalculus2/inverseTrigFunctions/examples/exampleList}

\end{center}








\end{document}
