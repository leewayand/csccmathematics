\documentclass{ximera}


\graphicspath{
  {./}
  {ximeraTutorial/}
  {basicPhilosophy/}
}

\newcommand{\mooculus}{\textsf{\textbf{MOOC}\textnormal{\textsf{ULUS}}}}


\usepackage{tkz-euclide}\usepackage{tikz}
\usepackage{tikz-cd}
\usetikzlibrary{arrows}
\tikzset{>=stealth,commutative diagrams/.cd,
  arrow style=tikz,diagrams={>=stealth}} %% cool arrow head
\tikzset{shorten <>/.style={ shorten >=#1, shorten <=#1 } } %% allows shorter vectors

\usetikzlibrary{backgrounds} %% for boxes around graphs
\usetikzlibrary{shapes,positioning}  %% Clouds and stars
\usetikzlibrary{matrix} %% for matrix
\usepgfplotslibrary{polar} %% for polar plots
\usepgfplotslibrary{fillbetween} %% to shade area between curves in TikZ
\usetkzobj{all}
\usepackage[makeroom]{cancel} %% for strike outs
%\usepackage{mathtools} %% for pretty underbrace % Breaks Ximera
%\usepackage{multicol}
\usepackage{pgffor} %% required for integral for loops



%% http://tex.stackexchange.com/questions/66490/drawing-a-tikz-arc-specifying-the-center
%% Draws beach ball
\tikzset{pics/carc/.style args={#1:#2:#3}{code={\draw[pic actions] (#1:#3) arc(#1:#2:#3);}}}



\usepackage{array}
\setlength{\extrarowheight}{+.1cm}
\newdimen\digitwidth
\settowidth\digitwidth{9}
\def\divrule#1#2{
\noalign{\moveright#1\digitwidth
\vbox{\hrule width#2\digitwidth}}}
























%%This is to help with formatting on future title pages.
\newenvironment{sectionOutcomes}{}{}


\title{Basic Forms}

\begin{document}

\begin{abstract}
base e
\end{abstract}
\maketitle






\textbf{\textcolor{blue!55!black}{A Different Perspective}} 


Basic exponential functions, are those functions which \textbf{\textcolor{red!80!black}{CAN}} be represented by formulas of the form $a \cdot r^x$.  \\


We can decide whether the function is increasing or decreasing by the value of $r$ and the sign of $a$. \\



\begin{itemize}
\item $a > 0$ and $r > 1$ : increasing positive function
\item $a < 0$ and $r > 1$ : decreasing negative function  
\item $a > 0$ and $r < 1$ : decreasing positive function
\item $a < 0$ and $r < 1$ : increasing negative function  
\end{itemize}



\textbf{On the other hand,} we have the algebra rule $\frac{1}{b} = b^{-1}$.  We could think of the base of the exponential formula as always being greater than $1$, and use positive and negative exponents to switch between increasing and decreasing functions. \\


Basic exponential functions, are those functions which \textbf{\textcolor{red!80!black}{CAN}} be represented by formulas of the form $a \cdot r^{b \, x}$, where $r > 1$.  \\


In this case, we would have the following behaviors: \\


\begin{itemize}
\item $a > 0$ and $b > 0$ : increasing positive function
\item $a > 0$ and $b < 0$ : decreasing positive function
\item $a < 0$ and $b > 0$ : decreasing negative function  
\item $a < 0$ and $b < 0$ : increasing negative function  
\end{itemize}

We would decide function behavior (increasing or decreasing) by the signs of \textbf{both} leading coefficients, $a$ and $b$.

\textbf{\textcolor{red!90!darkgray}{$\blacktriangleright$}} If $a$ and $b$ are the same sign, then we have an increasing function. \\

\textbf{\textcolor{red!90!darkgray}{$\blacktriangleright$}} If $a$ and $b$ are different signs, then we have a decreasing function. \\








\section*{(e)}



In this model, we are using bases that are greater than $1$.  \\


If this is the case, then we might as well use $e$ as our base. \\


\begin{explanation}


If our formula looks like. $a \cdot r^{b \, x}$ and $r>1$, then we can use a little algebra to rewrite our formula.



Since we are only using positive bases, we know that $r > 0$.  Since $r$ is a positive real number, we know that $r$ can be written as a power of $e$.


\[
r = e^{\ln(r)}
\]



We can rewrite our formula as



\[
a \cdot r^{b \, x} = a \cdot \left( e^{\ln(r)} \right)^{b \, x} = a \cdot  e^{\ln(r) \cdot (b \, x)} = a \cdot  e^{(\ln(r) \cdot b) \, x} = a \cdot  e^{B \, x}
\]


\end{explanation}




\textbf{\textcolor{red!90!darkgray}{$\blacktriangleright$}}  Basic exponential functions, are those functions which \textbf{\textcolor{red!80!black}{CAN}} be represented by formulas of the form $A \cdot e^{B \, x}$.  \\






In this model, our basic forms to memorize would be \\






\begin{image}
\begin{tikzpicture}
   \begin{axis}[name = leftgraph, 
            domain=-10:10, ymax=10, xmax=10, ymin=-10, xmin=-10,
            axis lines =center, xlabel=$x$, ylabel={$e^x$},
            every axis y label/.style={at=(current axis.above origin),anchor=south},
            every axis x label/.style={at=(current axis.right of origin),anchor=west},
            axis on top
          ]
          
          \addplot [line width=1, gray, dashed,samples=200,domain=(-10:10),<->] {0};

          \addplot [line width=2, penColor, smooth,samples=100,domain=(-9:8), <->] {1.3^x};
          \addplot [color=penColor,only marks,mark=*] coordinates{(0,1)};

           

  \end{axis}
  \begin{axis}[at={(leftgraph.outer east)},anchor=outer west, 
            domain=-10:10, ymax=10, xmax=10, ymin=-10, xmin=-10,
            axis lines =center, xlabel=$x$, ylabel={$-e^x$},
            every axis y label/.style={at=(current axis.above origin),anchor=south},
            every axis x label/.style={at=(current axis.right of origin),anchor=west},
            axis on top
          ]
          
          \addplot [line width=1, gray, dashed,samples=200,domain=(-10:10),<->] {0};

          \addplot [line width=2, penColor, smooth,samples=100,domain=(-9:8),<->] {-(1.3^x)};
          \addplot [color=penColor,only marks,mark=*] coordinates{(0,-1)};

           

  \end{axis}
\end{tikzpicture}
\end{image}









\begin{image}
\begin{tikzpicture}
   \begin{axis}[name = leftgraph, 
            domain=-10:10, ymax=10, xmax=10, ymin=-10, xmin=-10,
            axis lines =center, xlabel=$x$, ylabel={$e^{-x}$},
            every axis y label/.style={at=(current axis.above origin),anchor=south},
            every axis x label/.style={at=(current axis.right of origin),anchor=west},
            axis on top
          ]
          
          \addplot [line width=1, gray, dashed,samples=200,domain=(-10:10),<->] {0};

          \addplot [line width=2, penColor, smooth,samples=200,domain=(-8:9), <->] {1.3^(-x)};
          \addplot [color=penColor,only marks,mark=*] coordinates{(0,1)};
           

  \end{axis}
  \begin{axis}[at={(leftgraph.outer east)},anchor=outer west, 
            domain=-10:10, ymax=10, xmax=10, ymin=-10, xmin=-10,
            axis lines =center, xlabel=$x$, ylabel={$-e^{-x}$},
            every axis y label/.style={at=(current axis.above origin),anchor=south},
            every axis x label/.style={at=(current axis.right of origin),anchor=west},
            axis on top
          ]
          
          \addplot [line width=1, gray, dashed,samples=200,domain=(-10:10),<->] {0};

          \addplot [line width=2, penColor, smooth,samples=200,domain=(-8:9),<->] {-(1.3^(-x)};
          \addplot [color=penColor,only marks,mark=*] coordinates{(0,-1)};
           

  \end{axis}
\end{tikzpicture}
\end{image}
































\begin{center}
\textbf{\textcolor{green!50!black}{ooooo=-=-=-=-=-=-=-=-=-=-=-=-=ooOoo=-=-=-=-=-=-=-=-=-=-=-=-=ooooo}} \\

more examples can be found by following this link\\ \link[More Examples of Exponential Functions]{https://ximera.osu.edu/csccmathematics/precalculus2/precalculus2/expFunctions/examples/exampleList}

\end{center}









\end{document}
