\documentclass{ximera}



\graphicspath{
  {./}
  {ximeraTutorial/}
  {basicPhilosophy/}
}

\newcommand{\mooculus}{\textsf{\textbf{MOOC}\textnormal{\textsf{ULUS}}}}


\usepackage{tkz-euclide}\usepackage{tikz}
\usepackage{tikz-cd}
\usetikzlibrary{arrows}
\tikzset{>=stealth,commutative diagrams/.cd,
  arrow style=tikz,diagrams={>=stealth}} %% cool arrow head
\tikzset{shorten <>/.style={ shorten >=#1, shorten <=#1 } } %% allows shorter vectors

\usetikzlibrary{backgrounds} %% for boxes around graphs
\usetikzlibrary{shapes,positioning}  %% Clouds and stars
\usetikzlibrary{matrix} %% for matrix
\usepgfplotslibrary{polar} %% for polar plots
\usepgfplotslibrary{fillbetween} %% to shade area between curves in TikZ
\usetkzobj{all}
\usepackage[makeroom]{cancel} %% for strike outs
%\usepackage{mathtools} %% for pretty underbrace % Breaks Ximera
%\usepackage{multicol}
\usepackage{pgffor} %% required for integral for loops



%% http://tex.stackexchange.com/questions/66490/drawing-a-tikz-arc-specifying-the-center
%% Draws beach ball
\tikzset{pics/carc/.style args={#1:#2:#3}{code={\draw[pic actions] (#1:#3) arc(#1:#2:#3);}}}



\usepackage{array}
\setlength{\extrarowheight}{+.1cm}
\newdimen\digitwidth
\settowidth\digitwidth{9}
\def\divrule#1#2{
\noalign{\moveright#1\digitwidth
\vbox{\hrule width#2\digitwidth}}}
























%%This is to help with formatting on future title pages.
\newenvironment{sectionOutcomes}{}{}




\title{Error}

\begin{document}
\begin{abstract}
within tolerance
\end{abstract}
\maketitle



\section{Error approximation}

Differentials also help us estimate error in real life settings.

\begin{example}
  The cross-section of a $250$ ml glass can be modeled by the function
  $r(x) = \frac{x^4}{3}$:
  


\begin{image}
    \begin{tikzpicture}[
        declare function = {f(\x) = (1/3)* pow(\x,4);} ]
      \begin{axis}[
          xmin =-4,xmax=4,ymax=23,ymin=-.2,
          axis lines=center, xlabel=$x$, ylabel=$y$,
          every axis y label/.style={at=(current axis.above origin),anchor=south},
          every axis x label/.style={at=(current axis.right of origin),anchor=west},
          axis on top,
        ]
        \addplot [draw=none,fill=fillp!50!white,domain=-2.65:2.65, smooth] {.8*sqrt(2.65^2-x^2)+16.8} \closedcycle;
        \addplot [draw=none,fill=fillp,domain=-2.65:2.65, smooth] {-.8*sqrt(2.65^2-x^2)+16.8} \closedcycle; 
        \addplot [draw=none,fill=white,domain=-2.7:2.7, smooth] {f(x)} \closedcycle;
        \addplot [ultra thick,penColor, smooth,domain=-2.75:2.75] {f(x)};

        \draw[penColor,very thick] (axis cs:0,16.8) ellipse (265 and 20);
        \draw[penColor,very thick] (axis cs:0,19) ellipse (275 and 20);

        \node[black] at (axis cs:2.5,3) {$y=\frac{x^4}{3}$};       
      \end{axis}
    \end{tikzpicture}
  \end{image}






  At $16.8$ cm from the base of the glass, there is a mark indicating
  when the glass is filled to $250$ ml. If the glass is filled within
  $\pm 2$ millimeters of the mark, what are the bounds on the volume?
  As a gesture of friendship, we will tell you that the volume in
  milliliters, as a function of the height of water in centimeters, $y$,
  is given by
  \[
  V(y) = \frac{2\pi y^{3/2}}{\sqrt{3}}.
  \]
  Note: If you persist in your quest to learn calculus, you will be
  able to derive the formula above like it's no-big-deal.
  \begin{explanation}
    We want to know what a small change in the height, $y$, does to the
    volume $V$.  These small changes can be modeled by the
    differentials $dV$ and $dy$. Since
    \[
    dV = V'(y) dy
    \]
    and $V'(y) = \answer[given]{\pi \sqrt{3 y}}$ we use the fact that
    $dy = \pm 0.2$ with $y=\answer[given]{16.8}$ to see
    \[
    dV = \answer[given]{\pi \sqrt{3 \cdot 16.8}} \cdot 0.2.
    \]
    Hence the volume will vary by roughly $\pm\answer[given,tolerance=.01]{4.46062}$
    milliliters.
  \end{explanation}
\end{example}










\end{document}
