\documentclass{ximera}


\graphicspath{
  {./}
  {ximeraTutorial/}
  {basicPhilosophy/}
}

\newcommand{\mooculus}{\textsf{\textbf{MOOC}\textnormal{\textsf{ULUS}}}}


\usepackage{tkz-euclide}\usepackage{tikz}
\usepackage{tikz-cd}
\usetikzlibrary{arrows}
\tikzset{>=stealth,commutative diagrams/.cd,
  arrow style=tikz,diagrams={>=stealth}} %% cool arrow head
\tikzset{shorten <>/.style={ shorten >=#1, shorten <=#1 } } %% allows shorter vectors

\usetikzlibrary{backgrounds} %% for boxes around graphs
\usetikzlibrary{shapes,positioning}  %% Clouds and stars
\usetikzlibrary{matrix} %% for matrix
\usepgfplotslibrary{polar} %% for polar plots
\usepgfplotslibrary{fillbetween} %% to shade area between curves in TikZ
\usetkzobj{all}
\usepackage[makeroom]{cancel} %% for strike outs
%\usepackage{mathtools} %% for pretty underbrace % Breaks Ximera
%\usepackage{multicol}
\usepackage{pgffor} %% required for integral for loops



%% http://tex.stackexchange.com/questions/66490/drawing-a-tikz-arc-specifying-the-center
%% Draws beach ball
\tikzset{pics/carc/.style args={#1:#2:#3}{code={\draw[pic actions] (#1:#3) arc(#1:#2:#3);}}}



\usepackage{array}
\setlength{\extrarowheight}{+.1cm}
\newdimen\digitwidth
\settowidth\digitwidth{9}
\def\divrule#1#2{
\noalign{\moveright#1\digitwidth
\vbox{\hrule width#2\digitwidth}}}
























%%This is to help with formatting on future title pages.
\newenvironment{sectionOutcomes}{}{}


\title{iRoC}

\begin{document}

\begin{abstract}
rate of change
\end{abstract}
\maketitle




By function \textbf{behavior}, we mean the rate of change.


We have two flavors of this.




\begin{fact} \textbf{\textcolor{blue!55!black}{Over an Interval}}    \\


The rate of change of the function $f$ over the interval $[a, b]$ is given by $\frac{f(b) - f(a)}{b - a}$. \\


This is our algebraic version of rate of change. This is also referred to as the \textit{average rate of change} over the interval $[a, b]$.


\end{fact}






\begin{fact} \textbf{\textcolor{blue!55!black}{At a Domain Number}}   \\


The rate of change of the function $f$ at the domain number $a$ is given by the slope of the tangent line at the point $(a, f(a))$ on the graph. \\


This rate of change is known as the \textbf{\textcolor{purple!85!blue}{instantaneous rate of change of f at a}} and is measured by the \textbf{\textcolor{blue!55!black}{derivative of f at a}}. 


This is our Calculus version of rate of change.


\end{fact}


Each is a different measurement of the change in a function's value compared to changes in the domain. \\




\begin{notation} the \textbf{\textcolor{blue!55!black}{derivative of f at a}}


\textbf{\textcolor{blue!55!black}{$iRoC_f(a)$}} or \textbf{\textcolor{blue!55!black}{$f'(a)$}}.

\end{notation}









Algebra describes the change from one domain number to another. It describes function change over an interval. Calculus takes an extreme viewpoint on this.  Calculus asks about the interval $[a,a]$. Algebra doesn't know what to do with this type of interval. The rate of change of a function at one domain number,$a$, makes no sense, algebraically. Calculus makes sense of it as the slope of a tangent line at $(a, f(a))$. \\

This gives us two types of interpretations for increasing and decreasing. \\


\textbf{\textcolor{red!80!black}{$\blacktriangleright$ Algebraic Increasing}} \\



The function, $f$, is \textbf{\textcolor{purple!85!blue}{increasing on the interval}} $[a, b]$ if


\[
f(c) \leq f(d) \, \text{ whenever } \, c \leq d \, \text{ for all } \, c, d \in [a,b]
\]









\textbf{\textcolor{red!80!black}{$\blacktriangleright$ Calculus Increasing}} \\



The function, $f$, is \textbf{\textcolor{purple!85!blue}{increasing at a}} if


\[
iRoC_f(a) > 0 \, \text{ or } \, f'(a) > 0
\]










\textbf{\textcolor{red!80!black}{$\blacktriangleright$ Algebraic Decreasing}} \\



The function, $f$, is \textbf{\textcolor{purple!85!blue}{decreasing on the interval}} $[a, b]$ if


\[
f(c) \geq f(d) \, \text{ whenever } \, c \leq d \, \text{ for all } \, c, d \in [a,b]
\]









\textbf{\textcolor{red!80!black}{$\blacktriangleright$ Calculus Decreasing}} \\



The function, $f$, is \textbf{\textcolor{purple!85!blue}{decreasing at a}} if


\[
iRoC_f(a) < 0 \, \text{ or } \, f'(a) < 0
\]








The instantaneous rate of change can have a value at each domain number, which makes it into a function.






\section{iRoC}


The instantaneous rate of change (a.k.a the derivative) gives us a formula for the slopes of tangent lines, which are rates of change.



\textbf{\textcolor{red!90!darkgray}{$\blacktriangleright$}} From vertex form, $f(x) = a (x -h)^2 + k$, we have $iRoC_f(x) = f'(x) = 2 a \, (x-h)$. \\



\textbf{\textcolor{red!90!darkgray}{$\blacktriangleright$}} From standard form, $f(x) = a \, x^2 + b \, x + c$, we have $iRoC_f(x) = f'(x) = 2 a \, x + b$. \\



















\begin{example}


Let $M(t) = -\frac{1}{2} (t - 3)^2 + 5$ \\

What is the maximum value of $M(t)$? \\



\begin{explanation}

From the vertex form, we can see a negative leading coefficient, which tells us the parabola will open down, that $M$ increases and then decreases, and that there is a maximum value. \\

From the vertex form, we can tell that the highest point on the graph is $(3, 5)$, which tells us that the maximum value of $M$ is $5$ and it occurs at $3$.



\begin{image}
\begin{tikzpicture}
  \begin{axis}[
            domain=-10:10, ymax=10, xmax=10, ymin=-10, xmin=-10,
            axis lines =center, xlabel=$t$, ylabel={$y$}, grid = major, grid style={dashed},
            ytick={-10,-8,-6,-4,-2,2,4,6,8,10},
            xtick={-10,-8,-6,-4,-2,2,4,6,8,10},
            yticklabels={$-10$,$-8$,$-6$,$-4$,$-2$,$2$,$4$,$6$,$8$,$10$}, 
            xticklabels={$-10$,$-8$,$-6$,$-4$,$-2$,$2$,$4$,$6$,$8$,$10$},
            ticklabel style={font=\scriptsize},
            every axis y label/.style={at=(current axis.above origin),anchor=south},
            every axis x label/.style={at=(current axis.right of origin),anchor=west},
            axis on top
          ]
          
          %\addplot [line width=2, penColor2, smooth,samples=100,domain=(-6:2)] {-2*x-3};
          \addplot [line width=2, penColor, smooth,samples=200,domain=(-2:8),<->] {-0.5*(x-3)^2 + 5};
          %\addplot [line width=2, penColor2, smooth,samples=200,domain=(-4:4),<->] {2*(x-1)+3};

          %\addplot[color=penColor,fill=penColor2,only marks,mark=*] coordinates{(-6,9)};
          %\addplot[color=penColor,fill=penColor2,only marks,mark=*] coordinates{(2,-7)};

          \addplot[color=penColor,fill=penColor,only marks,mark=*] coordinates{(3,5)};
          %\addplot[color=penColor,fill=penColor,only marks,mark=*] coordinates{(-0.162,0)};
          %\addplot[color=penColor,fill=penColor,only marks,mark=*] coordinates{(6.162,0)};



           

  \end{axis}
\end{tikzpicture}
\end{image}


\end{explanaiton}
\end{example}



















\begin{example}


Let $M(t) = -\frac{1}{2} (t - 3)^2 + 5$ \\

What is the maximum value of $M(t)$? \\



\begin{explanation}

From the vertex form, we have $iRoC_M(t) = -(t - 3)$ or $M'(t) = -(t - 3) = -t + 3 = 3 - t$. \\



The $iRoC$ or derivative is itself a function.  We could plot its graph. For a quadratic function, the derivative is a linear function whose graph is a line.




Graph of $y = iRoC_M(t) = M'(t) = 3 - t$







\begin{image}
\begin{tikzpicture}
  \begin{axis}[
            domain=-10:10, ymax=10, xmax=10, ymin=-10, xmin=-10,
            axis lines =center, xlabel=$t$, ylabel={$y$}, grid = major, grid style={dashed},
            ytick={-10,-8,-6,-4,-2,2,4,6,8,10},
            xtick={-10,-8,-6,-4,-2,2,4,6,8,10},
            yticklabels={$-10$,$-8$,$-6$,$-4$,$-2$,$2$,$4$,$6$,$8$,$10$}, 
            xticklabels={$-10$,$-8$,$-6$,$-4$,$-2$,$2$,$4$,$6$,$8$,$10$},
            ticklabel style={font=\scriptsize},
            every axis y label/.style={at=(current axis.above origin),anchor=south},
            every axis x label/.style={at=(current axis.right of origin),anchor=west},
            axis on top
          ]
          
          %\addplot [line width=2, penColor2, smooth,samples=100,domain=(-6:2)] {-2*x-3};
          \addplot [line width=2, penColor, smooth,samples=200,domain=(-2:8),<->] {-x+3};
          %\addplot [line width=2, penColor2, smooth,samples=200,domain=(-4:4),<->] {2*(x-1)+3};

          %\addplot[color=penColor,fill=penColor2,only marks,mark=*] coordinates{(-6,9)};
          %\addplot[color=penColor,fill=penColor2,only marks,mark=*] coordinates{(2,-7)};

          %\addplot[color=penColor,fill=penColor,only marks,mark=*] coordinates{(1,3)};
          %\addplot[color=penColor,fill=penColor,only marks,mark=*] coordinates{(-0.162,0)};
          %\addplot[color=penColor,fill=penColor,only marks,mark=*] coordinates{(6.162,0)};



           

  \end{axis}
\end{tikzpicture}
\end{image}


The derivative is positive on $(-\infty, 3)$, which tells us that the tangent lines to the parabola have positive slopes for the left half of the parabola. This tells us that $M$ is \wordChoice{\choice[correct]{increasing} \choice{decreasing}}  on $(-\infty, 3)$. \\

The derivative is negative on $(3, \infty)$, which tells us that the tangent lines to the parabola have negative slopes for the right half of the parabola. This tells us that $M$ is \wordChoice{\choice{increasing} \choice[correct]{decreasing}}  on $(3, \infty)$. \\


When the derivative equals $0$, we have a critical number. 

$3 - t = 0$ when $t=3$


Since, $M$ increases and then decreases, we know this critical number marks the location of a maximum value of $M$.  To get this maximum value, we evaluate $M$ at the critical number: $M(3) = 5$.


$M$ has a maximum value of $5$, which occurs at $3$.


\end{explanation}

\end{example}
















FOr our example, $M'(t) = -(t - 3) = -t + 3 = 3 - t$ is a function.  The values of this function give the slopes of the tangent lines to the graph of $M$. \\





\textbf{For Instance:}

The point $(1, 3)$ is on the graph of $M$.  The slope of the tangent line at $(1, 3)$ is $M'(1) = 3 - 1 = 2$.  Therefore, the equation of the tangent line is $y - 3 = 2(t-1)$.




\begin{image}
\begin{tikzpicture}
  \begin{axis}[
            domain=-10:10, ymax=10, xmax=10, ymin=-10, xmin=-10,
            axis lines =center, xlabel=$t$, ylabel={$y$}, grid = major, grid style={dashed},
            ytick={-10,-8,-6,-4,-2,2,4,6,8,10},
            xtick={-10,-8,-6,-4,-2,2,4,6,8,10},
            yticklabels={$-10$,$-8$,$-6$,$-4$,$-2$,$2$,$4$,$6$,$8$,$10$}, 
            xticklabels={$-10$,$-8$,$-6$,$-4$,$-2$,$2$,$4$,$6$,$8$,$10$},
            ticklabel style={font=\scriptsize},
            every axis y label/.style={at=(current axis.above origin),anchor=south},
            every axis x label/.style={at=(current axis.right of origin),anchor=west},
            axis on top
          ]
          
          %\addplot [line width=2, penColor2, smooth,samples=100,domain=(-6:2)] {-2*x-3};
          \addplot [line width=2, penColor, smooth,samples=200,domain=(-2:8),<->] {-0.5*(x-3)^2 + 5};
          \addplot [line width=2, penColor2, smooth,samples=200,domain=(-4:4),<->] {2*(x-1)+3};

          %\addplot[color=penColor,fill=penColor2,only marks,mark=*] coordinates{(-6,9)};
          %\addplot[color=penColor,fill=penColor2,only marks,mark=*] coordinates{(2,-7)};

          \addplot[color=penColor,fill=penColor,only marks,mark=*] coordinates{(1,3)};
          %\addplot[color=penColor,fill=penColor,only marks,mark=*] coordinates{(-0.162,0)};
          %\addplot[color=penColor,fill=penColor,only marks,mark=*] coordinates{(6.162,0)};



           

  \end{axis}
\end{tikzpicture}
\end{image}










\begin{center}
\textbf{\textcolor{green!50!black}{ooooo=-=-=-=-=-=-=-=-=-=-=-=-=ooOoo=-=-=-=-=-=-=-=-=-=-=-=-=ooooo}} \\

more examples can be found by following this link\\ \link[More Examples of Quadratic Functions]{https://ximera.osu.edu/csccmathematics/precalculus2/precalculus2/quadraticFunctions/examples/exampleList}

\end{center}




\end{document}
