\documentclass{ximera}


\graphicspath{
  {./}
  {ximeraTutorial/}
  {basicPhilosophy/}
}

\newcommand{\mooculus}{\textsf{\textbf{MOOC}\textnormal{\textsf{ULUS}}}}


\usepackage{tkz-euclide}\usepackage{tikz}
\usepackage{tikz-cd}
\usetikzlibrary{arrows}
\tikzset{>=stealth,commutative diagrams/.cd,
  arrow style=tikz,diagrams={>=stealth}} %% cool arrow head
\tikzset{shorten <>/.style={ shorten >=#1, shorten <=#1 } } %% allows shorter vectors

\usetikzlibrary{backgrounds} %% for boxes around graphs
\usetikzlibrary{shapes,positioning}  %% Clouds and stars
\usetikzlibrary{matrix} %% for matrix
\usepgfplotslibrary{polar} %% for polar plots
\usepgfplotslibrary{fillbetween} %% to shade area between curves in TikZ
\usetkzobj{all}
\usepackage[makeroom]{cancel} %% for strike outs
%\usepackage{mathtools} %% for pretty underbrace % Breaks Ximera
%\usepackage{multicol}
\usepackage{pgffor} %% required for integral for loops



%% http://tex.stackexchange.com/questions/66490/drawing-a-tikz-arc-specifying-the-center
%% Draws beach ball
\tikzset{pics/carc/.style args={#1:#2:#3}{code={\draw[pic actions] (#1:#3) arc(#1:#2:#3);}}}



\usepackage{array}
\setlength{\extrarowheight}{+.1cm}
\newdimen\digitwidth
\settowidth\digitwidth{9}
\def\divrule#1#2{
\noalign{\moveright#1\digitwidth
\vbox{\hrule width#2\digitwidth}}}
























%%This is to help with formatting on future title pages.
\newenvironment{sectionOutcomes}{}{}


\title{Missing Numbers}

\begin{document}

\begin{abstract}
need more numbers
\end{abstract}
\maketitle




We have a complete characterization of quadratic functions as far as real numbers go.  But, the story is unbalanced.  A quadratic can have $0$, $1$, or $2$ real zeros or roots.  On the other hand, we have a feeling that polynomials of degree $2$ should always have two factors and two roots.  The problem is that the real numbers don't contain the numbers we need.  The real numbers are missing numbers.


According to the quadratic formula, the quadratic function $Q(x) = x^2 + 1$ should have $\sqrt{-1}$ as a zero.  But, $\sqrt{-1}$ is not in the real numbers. 




\begin{example}  
   
  
Let $f(x) = x^2 + x + 5$


The Quadratic Formula says the zeros are 


\[
\frac{-1 \pm \sqrt{1^2 - 4 \cdot 1 \cdot 5}}{2 \cdot 1} = \frac{-1 \pm \sqrt{-19}}{2}
\]


\end{example}






\begin{example}  
   
  
Let $H(t) = 2t^2 - 3t + 7$


The Quadratic Formula says the zeros are 


\[
\frac{3 \pm \sqrt{(-3)^2 - 4 \cdot 2 \cdot 7}}{2 \cdot 2} = \frac{3 \pm \sqrt{-47}}{4}
\]


\end{example}



We need numbers like $\sqrt{-19}$ and $\sqrt{-47}$. \\









\begin{example}  
   
  
Let $m(k) = k^2 + \pi $


The Quadratic Formula says the zeros are 


\[
\sqrt{-\pi} \, \text{ and } \, -\sqrt{-\pi}
\]


\end{example}




This example suggests we need $\sqrt{-r}$ for every real number, $r$. \\





It looks like it isn't holes that need to be filled in the real numbers, but an entire new copy of the real numbers that is needed.








\begin{center}
\textbf{\textcolor{green!50!black}{ooooo=-=-=-=-=-=-=-=-=-=-=-=-=ooOoo=-=-=-=-=-=-=-=-=-=-=-=-=ooooo}} \\

more examples can be found by following this link\\ \link[More Examples of Quadratic Functions]{https://ximera.osu.edu/csccmathematics/precalculus2/precalculus2/quadraticFunctions/examples/exampleList}

\end{center}




\end{document}
