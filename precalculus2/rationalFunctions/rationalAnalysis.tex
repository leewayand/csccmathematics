\documentclass{ximera}


\graphicspath{
  {./}
  {ximeraTutorial/}
  {basicPhilosophy/}
}

\newcommand{\mooculus}{\textsf{\textbf{MOOC}\textnormal{\textsf{ULUS}}}}


\usepackage{tkz-euclide}\usepackage{tikz}
\usepackage{tikz-cd}
\usetikzlibrary{arrows}
\tikzset{>=stealth,commutative diagrams/.cd,
  arrow style=tikz,diagrams={>=stealth}} %% cool arrow head
\tikzset{shorten <>/.style={ shorten >=#1, shorten <=#1 } } %% allows shorter vectors

\usetikzlibrary{backgrounds} %% for boxes around graphs
\usetikzlibrary{shapes,positioning}  %% Clouds and stars
\usetikzlibrary{matrix} %% for matrix
\usepgfplotslibrary{polar} %% for polar plots
\usepgfplotslibrary{fillbetween} %% to shade area between curves in TikZ
\usetkzobj{all}
\usepackage[makeroom]{cancel} %% for strike outs
%\usepackage{mathtools} %% for pretty underbrace % Breaks Ximera
%\usepackage{multicol}
\usepackage{pgffor} %% required for integral for loops



%% http://tex.stackexchange.com/questions/66490/drawing-a-tikz-arc-specifying-the-center
%% Draws beach ball
\tikzset{pics/carc/.style args={#1:#2:#3}{code={\draw[pic actions] (#1:#3) arc(#1:#2:#3);}}}



\usepackage{array}
\setlength{\extrarowheight}{+.1cm}
\newdimen\digitwidth
\settowidth\digitwidth{9}
\def\divrule#1#2{
\noalign{\moveright#1\digitwidth
\vbox{\hrule width#2\digitwidth}}}
























%%This is to help with formatting on future title pages.
\newenvironment{sectionOutcomes}{}{}


\title{Analysis}

\begin{document}

\begin{abstract}
in detail
\end{abstract}
\maketitle









Rational functions are quotients or fractions of polynomials.  (Polynomials are rational functions, since they can be written in a fraction form with $1$ as the denominator.)  So, it is not surprising that analyzing rational function follows the same plan as for polynomials.
















Completely analyze $G(t) = \frac{(t+7)(t+5)(t-2)}{(t+4)^2(t-2)}$




\begin{explanation}



The domain is $\left( -\infty, \answer{-4} \right) \cup \left(\answer{-4}, \answer{2} \right) \cup \left( \answer{2}, \infty \right)$.

Let's simplify: $G(t) = \frac{(t+7)(t+5)}{(t+4)^2}$, since $t-2$ is a common factor, since $-4$ and $2$ make the denominator $0$.



The $t-2$ factor is gone in our simplified formula, but $2$ is still not in the domain.  This will visually show up as a hole in the graph.



\begin{itemize}
\item $-7$ is a root of multiplicity $\answer{1}$.  Since this multiplicity is odd, $G$ will change sign through $\answer{-7}$ and the graph will cross at $(-7,0)$.
\item $-5$ is a root of multiplicity $\answer{1}$.  Since this multiplicity is odd, $G$ will change sign through $-5$ and the graph will cross at $(-5,0)$.
\item $-4$ is a singularity of multiplicity $\answer{2}$.  The function is unbounded near $-4$.  This will show up as a vertical asymptote on the graph. Since this multiplicity is even, $G$ will not change sign across $-4$.  The graph will approach the vertical asymptote similarly on both sides.
\item Our simplified version does not have the $t-2$ factor.  Therefore, the graph will not have an intercept nor a vertical asymptote associated with this factor.  It will have a hole at $\left( 2, \answer{\frac{63}{36}} \right)$.
\end{itemize}


The end-behavior of $G$ is $\frac{t^2}{t^2}$.  Therefore, $\lim\limits_{t \to -\infty}G(t) = 1$ and $\lim\limits_{t \to \infty}G(t) = 1$.  The graph has a horizontal asymptote.




Thinking left to right on the number line, $G$ starts off near $1$, which is positive.  It changes signs across $-7$ and becomes negative. It changes signs across $-5$ and becomes positive.  It cannot change sign until $-4$.  Therefore, $G$ becomes positively unbounded on the left side of $-4$.  Across $-4$, $G$ does not change sign.  Therefore, it is again unbounded and positive on the right side of the vertical asymptote.  There are no more zeros or singularities.  So, $G$ cannot change sign again.  Now, it approaches $1$ from above.




\begin{image}
\begin{tikzpicture}
  \begin{axis}[
            domain=-10:10, ymax=10, xmax=10, ymin=-10, xmin=-10,
            axis lines =center, xlabel=$t$, ylabel={$y=G(t)$}, grid = major, grid style={dashed},
            ytick={-10,-8,-6,-4,-2,2,4,6,8,10},
            xtick={-10,-8,-6,-4,-2,2,4,6,8,10},
            yticklabels={$-10$,$-8$,$-6$,$-4$,$-2$,$2$,$4$,$6$,$8$,$10$}, 
            xticklabels={$-10$,$-8$,$-6$,$-4$,$-2$,$2$,$4$,$6$,$8$,$10$},
            ticklabel style={font=\scriptsize},
            every axis y label/.style={at=(current axis.above origin),anchor=south},
            every axis x label/.style={at=(current axis.right of origin),anchor=west},
            axis on top
          ]
          
          	\addplot [line width=1, gray, dashed,samples=200,domain=(-9:9),<->] {1};
          	\addplot [line width=1, gray, dashed,samples=200,domain=(-9:9),<->] ({-4},{x});



            \addplot [line width=5, penColor!10!background, smooth,samples=100,domain=(-8:-6)] {-(x+7)};
            \addplot [line width=5, penColor!10!background, smooth,samples=100,domain=(-5.5:-4.5)] {(x+5)};

            \addplot [line width=5, penColor!10!background, smooth,samples=100,domain=(-4.5:-4.2)] {-1/(x+4)};
            \addplot [line width=5, penColor!10!background, smooth,samples=100,domain=(-3.9:-3.5)] {1/(x+4)};



            \addplot [line width=5, penColor!10!background, smooth,samples=100,domain=(-10:-8)] {1.5)};
            \addplot [line width=5, penColor!10!background, smooth,samples=100,domain=(4:8),->] {1.5)};



          	\addplot[color=penColor,fill=penColor,only marks,mark=*] coordinates{(-7,0)};
          	\addplot[color=penColor,fill=penColor,only marks,mark=*] coordinates{(-5,0)};


           

  \end{axis}
\end{tikzpicture}
\end{image}









The graph is very suggestive that there is a local minimum somewhere around $-5.5$.  






\begin{image}
\begin{tikzpicture}
  \begin{axis}[
            domain=-10:10, ymax=10, xmax=10, ymin=-10, xmin=-10,
            axis lines =center, xlabel=$t$, ylabel={$y=G(t)$}, grid = major, grid style={dashed},
            ytick={-10,-8,-6,-4,-2,2,4,6,8,10},
            xtick={-10,-8,-6,-4,-2,2,4,6,8,10},
            yticklabels={$-10$,$-8$,$-6$,$-4$,$-2$,$2$,$4$,$6$,$8$,$10$}, 
            xticklabels={$-10$,$-8$,$-6$,$-4$,$-2$,$2$,$4$,$6$,$8$,$10$},
            ticklabel style={font=\scriptsize},
            every axis y label/.style={at=(current axis.above origin),anchor=south},
            every axis x label/.style={at=(current axis.right of origin),anchor=west},
            axis on top
          ]
          
            \addplot [line width=1, gray, dashed,samples=200,domain=(-9:9),<->] {1};
            \addplot [line width=1, gray, dashed,samples=200,domain=(-9:9),<->] ({-4},{x});


            \addplot [line width=2, penColor, smooth,samples=300,domain=(-9:-4.4),<->] {((x+7)*(x+5))/(x+4)^2};
            \addplot [line width=2, penColor, smooth,samples=300,domain=(-3.05:9),<->] {((x+7)*(x+5))/(x+4)^2};


            \addplot[color=penColor,fill=penColor,only marks,mark=*] coordinates{(-7,0)};
            \addplot[color=penColor,fill=penColor,only marks,mark=*] coordinates{(-5,0)};


           

  \end{axis}
\end{tikzpicture}
\end{image}












With some technology, we can approximate the critical number to be $-5.5$ and the local minimum to be $-0.333$.




\begin{center}
\desmos{st4dfylktg}{400}{300}
\end{center}







\begin{itemize}
\item $G$ decreases on $(-\infty, -5.5]$.
\item $G$ increases on $[-5.5, -4)$.
\item $G$ decreases on $(-4, \infty)$.
\end{itemize}



The local minimum at $-5.5$ is also a global minimum.  There is no global maximum.  There are no local maximums.


\end{explanation}


































\begin{center}
\textbf{\textcolor{green!50!black}{ooooo-=-=-=-ooOoo-=-=-=-ooooo}} \\

more examples can be found by following this link\\ \link[More Examples of Rational Functions]{https://ximera.osu.edu/csccmathematics/precalculus2/precalculus2/rationalFunctions/examples/exampleList}

\end{center}




\end{document}
