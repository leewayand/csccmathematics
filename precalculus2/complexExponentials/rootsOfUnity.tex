\documentclass{ximera}

%\usepackage{todonotes}

\newcommand{\todo}{}

\usepackage{esint} % for \oiint
\ifxake%%https://math.meta.stackexchange.com/questions/9973/how-do-you-render-a-closed-surface-double-integral
\renewcommand{\oiint}{{\large\bigcirc}\kern-1.56em\iint}
\fi


\graphicspath{
  {./}
  {ximeraTutorial/}
  {basicPhilosophy/}
  {functionsOfSeveralVariables/}
  {normalVectors/}
  {lagrangeMultipliers/}
  {vectorFields/}
  {greensTheorem/}
  {shapeOfThingsToCome/}
  {dotProducts/}
  {partialDerivativesAndTheGradientVector/}
  {../productAndQuotientRules/exercises/}
  {../normalVectors/exercisesParametricPlots/}
  {../continuityOfFunctionsOfSeveralVariables/exercises/}
  {../partialDerivativesAndTheGradientVector/exercises/}
  {../directionalDerivativeAndChainRule/exercises/}
  {../commonCoordinates/exercisesCylindricalCoordinates/}
  {../commonCoordinates/exercisesSphericalCoordinates/}
  {../greensTheorem/exercisesCurlAndLineIntegrals/}
  {../greensTheorem/exercisesDivergenceAndLineIntegrals/}
  {../shapeOfThingsToCome/exercisesDivergenceTheorem/}
  {../greensTheorem/}
  {../shapeOfThingsToCome/}
  {../separableDifferentialEquations/exercises/}
  {vectorFields/}
}

\newcommand{\mooculus}{\textsf{\textbf{MOOC}\textnormal{\textsf{ULUS}}}}

\usepackage{tkz-euclide}
\usepackage{tikz}
\usepackage{tikz-cd}
\usetikzlibrary{arrows}
\tikzset{>=stealth,commutative diagrams/.cd,
  arrow style=tikz,diagrams={>=stealth}} %% cool arrow head
\tikzset{shorten <>/.style={ shorten >=#1, shorten <=#1 } } %% allows shorter vectors

\usetikzlibrary{backgrounds} %% for boxes around graphs
\usetikzlibrary{shapes,positioning}  %% Clouds and stars
\usetikzlibrary{matrix} %% for matrix
\usepgfplotslibrary{polar} %% for polar plots
\usepgfplotslibrary{fillbetween} %% to shade area between curves in TikZ
%\usetkzobj{all}
\usepackage[makeroom]{cancel} %% for strike outs
%\usepackage{mathtools} %% for pretty underbrace % Breaks Ximera
%\usepackage{multicol}
\usepackage{pgffor} %% required for integral for loops



%% http://tex.stackexchange.com/questions/66490/drawing-a-tikz-arc-specifying-the-center
%% Draws beach ball
\tikzset{pics/carc/.style args={#1:#2:#3}{code={\draw[pic actions] (#1:#3) arc(#1:#2:#3);}}}



\usepackage{array}
\setlength{\extrarowheight}{+.1cm}
\newdimen\digitwidth
\settowidth\digitwidth{9}
\def\divrule#1#2{
\noalign{\moveright#1\digitwidth
\vbox{\hrule width#2\digitwidth}}}




% \newcommand{\RR}{\mathbb R}
% \newcommand{\R}{\mathbb R}
% \newcommand{\N}{\mathbb N}
% \newcommand{\Z}{\mathbb Z}

\newcommand{\sagemath}{\textsf{SageMath}}


%\renewcommand{\d}{\,d\!}
%\renewcommand{\d}{\mathop{}\!d}
%\newcommand{\dd}[2][]{\frac{\d #1}{\d #2}}
%\newcommand{\pp}[2][]{\frac{\partial #1}{\partial #2}}
% \renewcommand{\l}{\ell}
%\newcommand{\ddx}{\frac{d}{\d x}}

% \newcommand{\zeroOverZero}{\ensuremath{\boldsymbol{\tfrac{0}{0}}}}
%\newcommand{\inftyOverInfty}{\ensuremath{\boldsymbol{\tfrac{\infty}{\infty}}}}
%\newcommand{\zeroOverInfty}{\ensuremath{\boldsymbol{\tfrac{0}{\infty}}}}
%\newcommand{\zeroTimesInfty}{\ensuremath{\small\boldsymbol{0\cdot \infty}}}
%\newcommand{\inftyMinusInfty}{\ensuremath{\small\boldsymbol{\infty - \infty}}}
%\newcommand{\oneToInfty}{\ensuremath{\boldsymbol{1^\infty}}}
%\newcommand{\zeroToZero}{\ensuremath{\boldsymbol{0^0}}}
%\newcommand{\inftyToZero}{\ensuremath{\boldsymbol{\infty^0}}}



% \newcommand{\numOverZero}{\ensuremath{\boldsymbol{\tfrac{\#}{0}}}}
% \newcommand{\dfn}{\textbf}
% \newcommand{\unit}{\,\mathrm}
% \newcommand{\unit}{\mathop{}\!\mathrm}
% \newcommand{\eval}[1]{\bigg[ #1 \bigg]}
% \newcommand{\seq}[1]{\left( #1 \right)}
% \renewcommand{\epsilon}{\varepsilon}
% \renewcommand{\phi}{\varphi}


% \renewcommand{\iff}{\Leftrightarrow}

% \DeclareMathOperator{\arccot}{arccot}
% \DeclareMathOperator{\arcsec}{arcsec}
% \DeclareMathOperator{\arccsc}{arccsc}
% \DeclareMathOperator{\si}{Si}
% \DeclareMathOperator{\scal}{scal}
% \DeclareMathOperator{\sign}{sign}


%% \newcommand{\tightoverset}[2]{% for arrow vec
%%   \mathop{#2}\limits^{\vbox to -.5ex{\kern-0.75ex\hbox{$#1$}\vss}}}
% \newcommand{\arrowvec}[1]{{\overset{\rightharpoonup}{#1}}}
% \renewcommand{\vec}[1]{\arrowvec{\mathbf{#1}}}
% \renewcommand{\vec}[1]{{\overset{\boldsymbol{\rightharpoonup}}{\mathbf{#1}}}}

% \newcommand{\point}[1]{\left(#1\right)} %this allows \vector{ to be changed to \vector{ with a quick find and replace
% \newcommand{\pt}[1]{\mathbf{#1}} %this allows \vec{ to be changed to \vec{ with a quick find and replace
% \newcommand{\Lim}[2]{\lim_{\point{#1} \to \point{#2}}} %Bart, I changed this to point since I want to use it.  It runs through both of the exercise and exerciseE files in limits section, which is why it was in each document to start with.

% \DeclareMathOperator{\proj}{\mathbf{proj}}
% \newcommand{\veci}{{\boldsymbol{\hat{\imath}}}}
% \newcommand{\vecj}{{\boldsymbol{\hat{\jmath}}}}
% \newcommand{\veck}{{\boldsymbol{\hat{k}}}}
% \newcommand{\vecl}{\vec{\boldsymbol{\l}}}
% \newcommand{\uvec}[1]{\mathbf{\hat{#1}}}
% \newcommand{\utan}{\mathbf{\hat{t}}}
% \newcommand{\unormal}{\mathbf{\hat{n}}}
% \newcommand{\ubinormal}{\mathbf{\hat{b}}}

% \newcommand{\dotp}{\bullet}
% \newcommand{\cross}{\boldsymbol\times}
% \newcommand{\grad}{\boldsymbol\nabla}
% \newcommand{\divergence}{\grad\dotp}
% \newcommand{\curl}{\grad\cross}
%\DeclareMathOperator{\divergence}{divergence}
%\DeclareMathOperator{\curl}[1]{\grad\cross #1}
% \newcommand{\lto}{\mathop{\longrightarrow\,}\limits}

% \renewcommand{\bar}{\overline}

\colorlet{textColor}{black}
\colorlet{background}{white}
\colorlet{penColor}{blue!50!black} % Color of a curve in a plot
\colorlet{penColor2}{red!50!black}% Color of a curve in a plot
\colorlet{penColor3}{red!50!blue} % Color of a curve in a plot
\colorlet{penColor4}{green!50!black} % Color of a curve in a plot
\colorlet{penColor5}{orange!80!black} % Color of a curve in a plot
\colorlet{penColor6}{yellow!70!black} % Color of a curve in a plot
\colorlet{fill1}{penColor!20} % Color of fill in a plot
\colorlet{fill2}{penColor2!20} % Color of fill in a plot
\colorlet{fillp}{fill1} % Color of positive area
\colorlet{filln}{penColor2!20} % Color of negative area
\colorlet{fill3}{penColor3!20} % Fill
\colorlet{fill4}{penColor4!20} % Fill
\colorlet{fill5}{penColor5!20} % Fill
\colorlet{gridColor}{gray!50} % Color of grid in a plot

\newcommand{\surfaceColor}{violet}
\newcommand{\surfaceColorTwo}{redyellow}
\newcommand{\sliceColor}{greenyellow}




\pgfmathdeclarefunction{gauss}{2}{% gives gaussian
  \pgfmathparse{1/(#2*sqrt(2*pi))*exp(-((x-#1)^2)/(2*#2^2))}%
}


%%%%%%%%%%%%%
%% Vectors
%%%%%%%%%%%%%

%% Simple horiz vectors
\renewcommand{\vector}[1]{\left\langle #1\right\rangle}


%% %% Complex Horiz Vectors with angle brackets
%% \makeatletter
%% \renewcommand{\vector}[2][ , ]{\left\langle%
%%   \def\nextitem{\def\nextitem{#1}}%
%%   \@for \el:=#2\do{\nextitem\el}\right\rangle%
%% }
%% \makeatother

%% %% Vertical Vectors
%% \def\vector#1{\begin{bmatrix}\vecListA#1,,\end{bmatrix}}
%% \def\vecListA#1,{\if,#1,\else #1\cr \expandafter \vecListA \fi}

%%%%%%%%%%%%%
%% End of vectors
%%%%%%%%%%%%%

%\newcommand{\fullwidth}{}
%\newcommand{\normalwidth}{}



%% makes a snazzy t-chart for evaluating functions
%\newenvironment{tchart}{\rowcolors{2}{}{background!90!textColor}\array}{\endarray}

%%This is to help with formatting on future title pages.
\newenvironment{sectionOutcomes}{}{}



%% Flowchart stuff
%\tikzstyle{startstop} = [rectangle, rounded corners, minimum width=3cm, minimum height=1cm,text centered, draw=black]
%\tikzstyle{question} = [rectangle, minimum width=3cm, minimum height=1cm, text centered, draw=black]
%\tikzstyle{decision} = [trapezium, trapezium left angle=70, trapezium right angle=110, minimum width=3cm, minimum height=1cm, text centered, draw=black]
%\tikzstyle{question} = [rectangle, rounded corners, minimum width=3cm, minimum height=1cm,text centered, draw=black]
%\tikzstyle{process} = [rectangle, minimum width=3cm, minimum height=1cm, text centered, draw=black]
%\tikzstyle{decision} = [trapezium, trapezium left angle=70, trapezium right angle=110, minimum width=3cm, minimum height=1cm, text centered, draw=black]


\title{Roots of Unity}

\begin{document}

\begin{abstract}
unit circle
\end{abstract}
\maketitle




We have seen that every Complex number can be written as $r \cdot (\cos(\theta) + i \, \sin(\theta))$.  This is a scalar times a Complex number on the unit circle.

We have also seen that if $z = r \cdot (\cos(\theta) + i \, \sin(\theta))$, then $z^n = r^n \cdot (\cos(n\theta) + i \, \sin(n\theta))$. Raising complex numbers to powers is accomplished by raising the modulus to the power and then multiplying the angle.


$\blacktriangleright$  \textbf{Roots of Unity}

Roots or unity refer to roots or zeros of the polynomial $x^n - 1$.

Any such root of unity would be a solution to the equation $z^n = 1$, which is why they are called roots of unity.




For instance, the square roots of unity are the solutions to $z^2 = 1$.  The two solutions are $1$ and $-1$.


For instance, the $4^{th}$ roots of unity are the solutions to $z^4 = 1$.  The four solutions are $1$, $-1$, $i$, and $-i$.




$1$ is always a root of unity for any power. Then, there are $n-1$ other $n^{th}$ roots of unity.



If $z = r \cdot (\cos(\theta) + i \, \sin(\theta))$ is going to be a root of unity, then $z^n = r^n \cdot (\cos(n\theta) + i \, \sin(n\theta)) = 1$, which means $r=1$.

$n^{th}$ roots of unity all look like $\cos(\theta) + i \, \sin(\theta)$.  They all lie on the unit circle.



In addition, if $\cos(n\theta) + i \, \sin(n\theta) = 1$, then $n \theta = 2 k \pi$ (a mulitple of $2 \pi$), because $1$ is on the positive real axis.


Therefore, $\theta = \frac{2 k \pi}{n}$  












\begin{example}  $4^{th}$ roots of $1$

We need multiples of $\frac{2 \pi}{4} = \frac{\pi}{2}$.

\begin{itemize}
\item $\theta = \frac{\pi}{2}$
\item $\theta = \frac{2 \pi}{2} = \pi$
\item $\theta = \frac{3 \pi}{2}$
\item $\theta = \frac{4 \pi}{2} = 2 \pi$
\end{itemize}



The fourth roots of unity are:
\begin{itemize}
\item $\cos\left(\frac{\pi}{2}\right) + i \, \sin\left(\frac{\pi}{2}\right) = i$
\item $\cos(\pi) + i \, \sin(\pi) = -1$
\item $\cos\left(\frac{3 \pi}{2}\right) + i \, \sin\left(\frac{3 \pi}{2}\right) = -i$
\item $\cos(2 \pi) + i \, \sin(2 \pi) = 1$
\end{itemize}





The roots of unity are spread out equidistant along the unit circle.






\end{example}










\begin{example}  Cube roots of $1$

We need multiples of $\frac{2 \pi}{3}$.

\begin{itemize}
\item $\theta = \frac{2 \pi}{3}$
\item $\theta = \frac{4 \pi}{3}$
\item $\theta = \frac{6 \pi}{3} = 2 \pi$
\end{itemize}



The fourth roots of unity are:
\begin{itemize}
\item $\cos\left(\frac{2 \pi}{3}\right) + i \, \sin\left(\frac{2 \pi}{3}\right) = \frac{1}{2} + \frac{\sqrt{3}}{2} \, i$
\item $\cos\left(\frac{4 \pi}{3}\right) + i \, \sin\left(\frac{4 \pi}{3}\right) = \frac{1}{2} - \frac{\sqrt{3}}{2} \, i$
\item $\cos(2 \pi) + i \, \sin(2 \pi) = 1$
\end{itemize}



\end{example}








\begin{question} 


$\cos\left( \frac{\pi}{6} \right) + i \, \sin\left( \frac{\pi}{6} \right)$ is which root of unity?

\begin{multipleChoice}
\choice {sixth}
\choice {eighth}
\choice {tenth}
\choice[correct] {twelfeth}
\end{multipleChoice}

\end{question}







\begin{example}


Using the quadratic formula, we know that $\frac{-3 + \sqrt{11} \, i}{2}$ is a root of $x^2 + 3x + 5$. \\

We know that $\frac{-3 + \sqrt{11} \, i}{2}$ can be written in the form $r (\cos(\theta) + i \, \sin(\theta))$.

$r$ is the modulus of $\frac{-3 + \sqrt{11} \, i}{2}$.

$r = \answer{\sqrt{5}}$ 

The angle for $\frac{-3 + \sqrt{11} \, i}{2}$ is in which quadrant?


\begin{multipleChoice}
\choice {I}
\choice[correct] {II}
\choice {III}
\choice {IV}
\end{multipleChoice}


$\arctan\left( -\frac{\sqrt{11}}{3} \right)$ gives the angle, but in quadrant IV.    What must be added to $\arctan\left( -\frac{\sqrt{11}}{3} \right)$ to get $\theta$?

Add $\answer{\pi}$


\[   \frac{-3 + \sqrt{11} \, i}{2} = \sqrt{5} \left( \cos\left( \arctan\left( -\frac{\sqrt{11}}{3} \right) + \pi \right) + i \, \sin\left( \arctan\left( -\frac{\sqrt{11}}{3} \right) +\pi  \right) \right)     \]


\end{example}


As long as we are here...\\


\begin{observation}

In the example above, $x^2 + 3x + 5$ is a polynomial with real coefficients.  Therefore, if $\frac{-3 + \sqrt{11} \, i}{2}$ is a root, then so is $\frac{-3 - \sqrt{11} \, i}{2}$.



Therefore, $x^2 + 3x + 5$ factors as



\[ \left( x - \frac{-3 - \sqrt{11} \, i}{2} \right)  \left( x - \frac{-3 + \sqrt{11} \, i}{2} \right)    \]


The constant terms have to be equal, which tells us that 


\[    \left( \frac{-3 - \sqrt{11} \, i}{2} \right)  \left( \frac{-3 + \sqrt{11} \, i}{2} \right)  = 5    \]




The constant term of the polynomial is the product of the roots, which are conjugates.  The constant term is the square of the modulus of either root.





\end{observation}













\begin{center}
\textbf{\textcolor{green!50!black}{ooooo-=-=-=-ooOoo-=-=-=-ooooo}} \\

more examples can be found by following this link\\ \link[More Examples of Complex Exponentials]{https://ximera.osu.edu/csccmathematics/precalculus2/precalculus2/complexExponentials/examples/exampleList}

\end{center}



\end{document}
