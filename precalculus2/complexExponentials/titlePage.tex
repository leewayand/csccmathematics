\documentclass{ximera}


\graphicspath{
  {./}
  {ximeraTutorial/}
  {basicPhilosophy/}
}

\newcommand{\mooculus}{\textsf{\textbf{MOOC}\textnormal{\textsf{ULUS}}}}


\usepackage{tkz-euclide}\usepackage{tikz}
\usepackage{tikz-cd}
\usetikzlibrary{arrows}
\tikzset{>=stealth,commutative diagrams/.cd,
  arrow style=tikz,diagrams={>=stealth}} %% cool arrow head
\tikzset{shorten <>/.style={ shorten >=#1, shorten <=#1 } } %% allows shorter vectors

\usetikzlibrary{backgrounds} %% for boxes around graphs
\usetikzlibrary{shapes,positioning}  %% Clouds and stars
\usetikzlibrary{matrix} %% for matrix
\usepgfplotslibrary{polar} %% for polar plots
\usepgfplotslibrary{fillbetween} %% to shade area between curves in TikZ
\usetkzobj{all}
\usepackage[makeroom]{cancel} %% for strike outs
%\usepackage{mathtools} %% for pretty underbrace % Breaks Ximera
%\usepackage{multicol}
\usepackage{pgffor} %% required for integral for loops



%% http://tex.stackexchange.com/questions/66490/drawing-a-tikz-arc-specifying-the-center
%% Draws beach ball
\tikzset{pics/carc/.style args={#1:#2:#3}{code={\draw[pic actions] (#1:#3) arc(#1:#2:#3);}}}



\usepackage{array}
\setlength{\extrarowheight}{+.1cm}
\newdimen\digitwidth
\settowidth\digitwidth{9}
\def\divrule#1#2{
\noalign{\moveright#1\digitwidth
\vbox{\hrule width#2\digitwidth}}}
























%%This is to help with formatting on future title pages.
\newenvironment{sectionOutcomes}{}{}


\title{Complex Exponentials}

\begin{document}

\begin{abstract}
%
\end{abstract}
\maketitle



\begin{itemize}
\item \textbf{\textcolor{purple!85!blue}{Every}} complex number can be written in the form $a + b \, i$.
\item \textbf{\textcolor{purple!85!blue}{Every}} complex number on the unit circle can be written in the form $\cos(\theta) + \sin(\theta) \, i$.
\item \textbf{\textcolor{purple!85!blue}{Every}} complex number can be written in the form $r \cdot (\cos(\theta) + \sin(\theta) \, i)$.
\item \textbf{\textcolor{purple!85!blue}{Every}} complex number can be written in the form $(r, \theta)$.
\end{itemize}

We are adding two more to the list.



\begin{itemize}
\item \textbf{\textcolor{purple!85!blue}{Every}} complex number on the unit circle can be written in the form $e^{i \, \theta}$.
\item \textbf{\textcolor{purple!85!blue}{Every}} complex number can be written in the form $r \cdot e^{i \, \theta}$.
\end{itemize}






\subsection{Learning Outcomes}

\begin{sectionOutcomes}
In this section, students will 

\begin{itemize}
\item use the different forms to investigate the Complex Numbers
\end{itemize}
\end{sectionOutcomes}

\end{document}
