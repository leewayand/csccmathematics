\documentclass{ximera}


\graphicspath{
  {./}
  {ximeraTutorial/}
  {basicPhilosophy/}
}

\newcommand{\mooculus}{\textsf{\textbf{MOOC}\textnormal{\textsf{ULUS}}}}


\usepackage{tkz-euclide}\usepackage{tikz}
\usepackage{tikz-cd}
\usetikzlibrary{arrows}
\tikzset{>=stealth,commutative diagrams/.cd,
  arrow style=tikz,diagrams={>=stealth}} %% cool arrow head
\tikzset{shorten <>/.style={ shorten >=#1, shorten <=#1 } } %% allows shorter vectors

\usetikzlibrary{backgrounds} %% for boxes around graphs
\usetikzlibrary{shapes,positioning}  %% Clouds and stars
\usetikzlibrary{matrix} %% for matrix
\usepgfplotslibrary{polar} %% for polar plots
\usepgfplotslibrary{fillbetween} %% to shade area between curves in TikZ
\usetkzobj{all}
\usepackage[makeroom]{cancel} %% for strike outs
%\usepackage{mathtools} %% for pretty underbrace % Breaks Ximera
%\usepackage{multicol}
\usepackage{pgffor} %% required for integral for loops



%% http://tex.stackexchange.com/questions/66490/drawing-a-tikz-arc-specifying-the-center
%% Draws beach ball
\tikzset{pics/carc/.style args={#1:#2:#3}{code={\draw[pic actions] (#1:#3) arc(#1:#2:#3);}}}



\usepackage{array}
\setlength{\extrarowheight}{+.1cm}
\newdimen\digitwidth
\settowidth\digitwidth{9}
\def\divrule#1#2{
\noalign{\moveright#1\digitwidth
\vbox{\hrule width#2\digitwidth}}}
























%%This is to help with formatting on future title pages.
\newenvironment{sectionOutcomes}{}{}


\title{The Unit Circle}

\begin{document}

\begin{abstract}
%%%
\end{abstract}
\maketitle




\begin{center}
\textbf{\textcolor{red!80!black}{Every complex number can be written as a scalar multiple of a direction vector.}}
\end{center}






Direction vectors or unit vectors are vectors or length $1$, which makes the unit circle the key to understanding the Complex Numbers. \\ 



We can describe every complex number by providing a complex number on the unit circle to give the direction and then a distance to move in that direction.  \\ 


We have already described complex numbers on the unit circle with sine and cosine: $cos(\theta) + sin(\theta) \, i$.  This came from investigating the right triangle formed from points on the unit circle. \\



\begin{center}
\textbf{\textcolor{purple!85!blue}{Complex Numbers, the Unit Circle, Right Triangles, and Sine and Cosine all give the same information}}
\end{center}





\begin{sectionOutcomes}
In this section, students will 

\begin{itemize}
\item examine the unit circle via right triangles.
\item decompose vectors into perpendicular components.
\item describe components with sine and cosine.
\item describe complex numbers with this information
\end{itemize}
\end{sectionOutcomes}

\end{document}
