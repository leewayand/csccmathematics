\documentclass{ximera}


\graphicspath{
  {./}
  {ximeraTutorial/}
  {basicPhilosophy/}
}

\newcommand{\mooculus}{\textsf{\textbf{MOOC}\textnormal{\textsf{ULUS}}}}


\usepackage{tkz-euclide}\usepackage{tikz}
\usepackage{tikz-cd}
\usetikzlibrary{arrows}
\tikzset{>=stealth,commutative diagrams/.cd,
  arrow style=tikz,diagrams={>=stealth}} %% cool arrow head
\tikzset{shorten <>/.style={ shorten >=#1, shorten <=#1 } } %% allows shorter vectors

\usetikzlibrary{backgrounds} %% for boxes around graphs
\usetikzlibrary{shapes,positioning}  %% Clouds and stars
\usetikzlibrary{matrix} %% for matrix
\usepgfplotslibrary{polar} %% for polar plots
\usepgfplotslibrary{fillbetween} %% to shade area between curves in TikZ
\usetkzobj{all}
\usepackage[makeroom]{cancel} %% for strike outs
%\usepackage{mathtools} %% for pretty underbrace % Breaks Ximera
%\usepackage{multicol}
\usepackage{pgffor} %% required for integral for loops



%% http://tex.stackexchange.com/questions/66490/drawing-a-tikz-arc-specifying-the-center
%% Draws beach ball
\tikzset{pics/carc/.style args={#1:#2:#3}{code={\draw[pic actions] (#1:#3) arc(#1:#2:#3);}}}



\usepackage{array}
\setlength{\extrarowheight}{+.1cm}
\newdimen\digitwidth
\settowidth\digitwidth{9}
\def\divrule#1#2{
\noalign{\moveright#1\digitwidth
\vbox{\hrule width#2\digitwidth}}}
























%%This is to help with formatting on future title pages.
\newenvironment{sectionOutcomes}{}{}


\title{Analysis}

\begin{document}

\begin{abstract}
sine
\end{abstract}
\maketitle












\textbf{\textcolor{purple!85!blue}{Analyze  $T(r) = 3 \sin(2r - \pi) - 1$}}



\begin{template}


We are going to view this as a composition of three functions.


\[
T(r) = 3 \sin(2r - \pi) - 1 = Out(x) \circ \sin(\theta) \circ In(y)
\]


where


$Out(x) = 3x - 1$ \\

$In(y) = 2y - \pi$ \\



\end{template}



\textbf{\textcolor{blue!55!black}{Domain}}


The domain of $Out(x) = 3x - 1$ is $(-\infty, \infty)$, because $Out$ is a linear function.  \\

This includes any value from $\sin(\theta)$. That means we can use the entire domain of $\sin(\theta)$, which is $(-\infty, \infty)$.  \\


This includes any value of $In(y)$, which means we can use the entire domain of $In(y)$, which is $(-\infty, \infty)$, because $In$ is a linear function.  \\




























































\begin{example}



Analyze  $T(r) = 3 \sin(2r - \pi) - 1$





We are going to view this as a composition of three functions.


\[
T(r) = 3 \sin(2r - \pi) - 1 = Out(x) \circ \sin(\theta) \circ In(y)
\]


where


$Out(x) = 3x - 1$ \\

$In(y) = 2y - \pi$ \\







\textbf{\textcolor{blue!55!black}{Domain}}


The domain of $Out(x) = 3x - 1$ is $(-\infty, \infty)$.  This means $Out$




$Out$

$Out$




$\blacktriangleright$ \textbf{Horizontal}


One period of the sine wave occurs from where the inside equals $0$, to where the inside equals $\answer{2 \pi}$.


\begin{align*}
2r - \pi & =  0 \\
2r       & =  \pi \\
r        & =  \frac{\pi}{2}
\end{align*}



\begin{align*}
2r - \pi & =  2 \pi \\
2r       & =  3 \pi \\
r        & =  \frac{3 \pi}{2}
\end{align*}


$\sin(2r - \pi)$ would equal $0$ at $\frac{\pi}{2}$ and $\frac{3 \pi}{2}$.



$\blacktriangleright$ $\frac{\pi}{2}$ is the starting line for our one period.

$\blacktriangleright$ $\frac{3 \pi}{2}$ is the finish line for our our period.




Halfway between $\frac{\pi}{2}$ and $\frac{3 \pi}{2}$ is $\pi$ and $\sin(2r - \pi)$ would again equal $0$ there.

A quarter of the way is $\frac{3 \pi}{4}$, where $\sin(2r - \pi) = 1$.

Three-quarters of the way is $\frac{5 \pi}{4}$, where $\sin(2r - \pi) = -1$.






$\blacktriangleright$ \textbf{Vertical}


The maximum value of $\sin(2r - \pi)$ is $1$.  Therefore, the maximum value of $T(r)$ is $3 \cdot 1 - 1 = 2$, which occurs at $\answer{\frac{3 \pi}{4}}$. 




The minimum value of $\sin(2r - \pi)$ is $-1$.  Therefore, the minimum value of $T(r)$ is $3 \cdot (-1) - 1 = -4$, which occurs at $\answer{\frac{5 \pi}{4}}$. 






Graph of $y = T(r)$, for one period.



\begin{image}
\begin{tikzpicture}
  \begin{axis}[
            domain=0:7, ymax=5, xmax=7, ymin=-5, xmin=0,
            axis lines =center, xlabel={$r$}, ylabel={$y$}, grid = major, grid style={dashed},
            ytick={-4,-2,0,2,4},
            xtick={1.57, 3.142, 4.71, 6.28, 7.85},
            xticklabels={$\tfrac{\pi}{2}$, $\pi$, $\tfrac{3\pi}{2}$, $2\pi$, $\tfrac{5\pi}{2}$},
            yticklabels={$-4$,$-2$,$0$,$2$,$4$}, 
            ticklabel style={font=\scriptsize},
            every axis y label/.style={at=(current axis.above origin),anchor=south},
            every axis x label/.style={at=(current axis.right of origin),anchor=west},
            axis on top
          ]
          

            \addplot [line width=2, penColor, smooth,samples=300,domain=(1.57:4.71)] {3*sin(deg(2*x-3.14))-1};

            %\addplot [color=penColor,only marks,mark=*] coordinates{(0,0)};
            %\addplot [color=penColor,fill=white,only marks,mark=*] coordinates{(6.28,0)};
            %\addplot [color=penColor2,only marks,mark=*] coordinates{(0,1)};
            %\addplot [color=penColor2,fill=white,only marks,mark=*] coordinates{(6.28,1)};


  \end{axis}
\end{tikzpicture}
\end{image}



Now we can extend this to the whole real line.


The period is $\frac{3 \pi}{2} - \frac{\pi}{2} = \pi$.




\begin{itemize}
\item $T(r)$ has a maximum of $2$, which occurs at $\{  \frac{3 \pi}{4} \pm k \pi \, | \, k \in \mathbb{Z} \}$.

\item $T(r)$ has a minimum of $-4$, which occurs at $\{  \frac{5 \pi}{4} \pm k \pi \, | \, k \in \mathbb{Z} \}$.

\item $T(r)$ decreases on $\left[  \frac{3 \pi}{4} \pm k \pi,    \frac{5 \pi}{4} \pm k \pi    \right]$, where $k \in \mathbb{Z}$.

\item $T(r)$ increases on $\left[  \frac{5 \pi}{4} \pm k \pi,    \frac{3 \pi}{4} \pm (k+1) \pi    \right]$, where $k \in \mathbb{Z}$.
\end{itemize}






Graph of $y = T(r)$.

\begin{image}
\begin{tikzpicture}
  \begin{axis}[
            domain=0:7, ymax=5, xmax=7, ymin=-5, xmin=0,
            axis lines =center, xlabel={$r$}, ylabel={$y$}, grid = major, grid style={dashed},
            ytick={-4,-2,0,2,4},
            xtick={1.57, 3.142, 4.71, 6.28, 7.85},
            xticklabels={$\tfrac{\pi}{2}$, $\pi$, $\tfrac{3\pi}{2}$, $2\pi$, $\tfrac{5\pi}{2}$},
            yticklabels={$-4$,$-2$,$0$,$2$,$4$}, 
            ticklabel style={font=\scriptsize},
            every axis y label/.style={at=(current axis.above origin),anchor=south},
            every axis x label/.style={at=(current axis.right of origin),anchor=west},
            axis on top
          ]
          

            \addplot [line width=2, penColor, smooth,samples=300,domain=(0:6.28),<->] {3*sin(deg(2*x-3.14))-1};

            %\addplot [color=penColor,only marks,mark=*] coordinates{(0,0)};
            %\addplot [color=penColor,fill=white,only marks,mark=*] coordinates{(6.28,0)};
            %\addplot [color=penColor2,only marks,mark=*] coordinates{(0,1)};
            %\addplot [color=penColor2,fill=white,only marks,mark=*] coordinates{(6.28,1)};


  \end{axis}
\end{tikzpicture}
\end{image}



















\end{example}
































\begin{center}
\textbf{\textcolor{green!50!black}{ooooo-=-=-=-ooOoo-=-=-=-ooooo}} \\

more examples can be found by following this link\\ \link[More Examples of Trigonometric Functions]{https://ximera.osu.edu/csccmathematics/precalculus2/precalculus2/trigonometricFunctions/examples/exampleList}

\end{center}

\end{document}

