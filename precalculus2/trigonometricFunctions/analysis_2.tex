\documentclass{ximera}


\graphicspath{
  {./}
  {ximeraTutorial/}
  {basicPhilosophy/}
}

\newcommand{\mooculus}{\textsf{\textbf{MOOC}\textnormal{\textsf{ULUS}}}}


\usepackage{tkz-euclide}\usepackage{tikz}
\usepackage{tikz-cd}
\usetikzlibrary{arrows}
\tikzset{>=stealth,commutative diagrams/.cd,
  arrow style=tikz,diagrams={>=stealth}} %% cool arrow head
\tikzset{shorten <>/.style={ shorten >=#1, shorten <=#1 } } %% allows shorter vectors

\usetikzlibrary{backgrounds} %% for boxes around graphs
\usetikzlibrary{shapes,positioning}  %% Clouds and stars
\usetikzlibrary{matrix} %% for matrix
\usepgfplotslibrary{polar} %% for polar plots
\usepgfplotslibrary{fillbetween} %% to shade area between curves in TikZ
\usetkzobj{all}
\usepackage[makeroom]{cancel} %% for strike outs
%\usepackage{mathtools} %% for pretty underbrace % Breaks Ximera
%\usepackage{multicol}
\usepackage{pgffor} %% required for integral for loops



%% http://tex.stackexchange.com/questions/66490/drawing-a-tikz-arc-specifying-the-center
%% Draws beach ball
\tikzset{pics/carc/.style args={#1:#2:#3}{code={\draw[pic actions] (#1:#3) arc(#1:#2:#3);}}}



\usepackage{array}
\setlength{\extrarowheight}{+.1cm}
\newdimen\digitwidth
\settowidth\digitwidth{9}
\def\divrule#1#2{
\noalign{\moveright#1\digitwidth
\vbox{\hrule width#2\digitwidth}}}
























%%This is to help with formatting on future title pages.
\newenvironment{sectionOutcomes}{}{}


\title{Analysis}

\begin{document}

\begin{abstract}
cosine
\end{abstract}
\maketitle












\textbf{\textcolor{purple!85!blue}{Analyze  $H(x) =  2 \cos(\pi - 4x) + 1$}}





\begin{template}


We are going to view this as a composition of three functions.


\[
H = Out \circ \cos \circ In
\]


where


$Out(t) = 2t + 1$ \\


$In(y) = \pi - 4y$ \\




Along with $\cos(\theta)$, these three component functions, we get



\[
H(x) = (Out \circ \cos \circ In)(x) = Out(cos(In(x)))  = 2 \cos(\pi - 4x) + 1
\]




\end{template}

$H$ is a periodic function, since $\cos(\theta)$ is a periodic function. \\

The principal interval of $\cos(\theta)$ is $[0, 2\pi)$.  The principal interval of $H$ will be the values of $x$ that make the values of $In(x)$ run from $0$ to $2\pi$. \\


$\pi - 4x = 0$ at $x = \frac{\pi}{4}$

$\pi - 4x = 2\pi$ at $x = -\frac{\pi}{4}$

The principle interval of $H$ is $\left( -\frac{\pi}{4}, \frac{\pi}{4} \right]$. \\

The length of this interval is $\frac{\pi}{2}$, which is the period of $T$. \\


\textbf{Note:}  The principal interval is $\left( -\frac{\pi}{4}, \frac{\pi}{4} \right]$, but we are running trough it backwards.  To get the inside of $\cos(\theta)$ to move from $0$ to $2\pi$, we need $x$ to run from $\frac{\pi}{4}$ to $-\frac{\pi}{4}$.












\textbf{\textcolor{blue!55!black}{$\blacktriangleright$ desmos graph}} 
\begin{center}
\desmos{rnjncnntox}{400}{300}
\end{center}












\begin{observation}

$H$ is a periodic function with period $\frac{\pi}{2}$. \\


Therefore, our analysis will focus on the principal interval of $\left( -\frac{\pi}{4}, \frac{\pi}{4} \right]$. \\


All of the features and characteristics we discover will repeat wiht a period of $\frac{\pi}{2}$.\\


\end{observation}














\textbf{\textcolor{blue!55!black}{Domain}}


The domain of $Out(t) = 2t + 1$ is $(-\infty, \infty)$, because $Out$ is a linear function.  \\

This includes any value of $\cos(\theta)$. That means we can use the entire domain of $\cos(\theta)$, which is $(-\infty, \infty)$.  \\


This includes any value of $In(y)$, which means we can use the entire domain of $In(y)$, which is $(-\infty, \infty)$, because $In$ is a linear function.  \\


The domain of $H$ is $(-\infty, \infty)$. \\











\textbf{\textcolor{blue!55!black}{Zeros}}


Since the values of $H$ come from the values of $Out$, we are first looking for zeros of $Out$.  $Out(t) = 2t + 1$ is a linear function and has only one zero, $-\frac{1}{2}$. \\

We are looking for where $\cos(\theta) = -\frac{1}{2}$. \\


There are two numbers in $[0, 2\pi)$ where $\cos(\theta) = -\frac{1}{2}$.  They are $\frac{2\pi}{3}$ and $\frac {4\pi}{3}$.

We need the values of $y$ where $In(y) = \pi - 4y = \frac{2\pi}{3}$ and $In(y) = \pi - 4Yx = \frac{4\pi}{3}$

$y = \frac{\pi}{12}$  and   $y = -\frac{\pi}{12}$.\\

Thee are both in the principal interval for $H$. \\

[ These agree with the graph. ]


And, these repeat every $\frac{\pi}{2}$ for $H$.

















\textbf{\textcolor{blue!55!black}{Continuity}}


The component functions of the composition are linear and cosine, all continuous. \\

The composition of continuous function is continuous.

$H$ is continuous. It has no disontinuities. \\


Since the domain is $(-\infty, \infty)$, there are no singularities. \\








\textbf{\textcolor{blue!55!black}{End-Behavior}}


$H$ is a periodic function with period $\frac{\pi}{2}$. \\

It either has no end-behavior or the end-behavior is that it is periodic.

















\textbf{\textcolor{blue!55!black}{Behavior}}


We need the behavior of each of the three component funcitons and then we will compose them together. \\




$Out(t) = 2t + 1$ is a linear function with a positive leading coefficient, so it is an increasing function.\\


$In(y) = \pi - 4y$ is a linear function with a positive leading coefficient, so it is an increasing function.\\


$\cos(\theta)$ increases and decreases through the quadrants. For the principal interval, we have


\begin{itemize}
  \item decreases on $\left( 0, \frac{\pi}{2} \right)$
  \item decreases on $\left( \frac{\pi}{2}, \pi \right)$
  \item increases on $\left( \pi, \frac{3\pi}{2} \right)$
  \item increases on $\left( \frac{3\pi}{2}, 2\pi \right)$
\end{itemize}


Now to trace domains and ranges.


We need the values of $In(y) = \pi - 4y$ to be $0$, $\frac{\pi}{2}$, $\pi$, $\frac{3\pi}{2}$, and $2\pi$. \\



$\blacktriangleright$ $\pi - 4y = 0$ when $y = \frac{\pi}{4}$ \\

$\blacktriangleright$ $\pi - 4y = \frac{\pi}{2}$ when $y = \frac{\pi}{8}$ \\

$\blacktriangleright$ $\pi - 4y = \pi$ when $y = 0$ \\

$\blacktriangleright$ $\pi - 4y = \frac{3\pi}{2}$ when $y = -\frac{\pi}{8}$ \\

$\blacktriangleright$ $\pi - 4y = 2\pi$ when $y = -\frac{\pi}{4}$ \\


\textbf{Note:}  These are running from greatest to least.  That is due to the negative leading coefficent of $In(y)$. \\






Now to glue everything together. \\



\textbf{\textcolor{purple!80!black}{On $x = y \in \left( -\frac{\pi}{4}, -\frac{\pi}{8} \right)$, }}



$In(y) = \pi - 4y$ is decreasing, because $In$ is always decreasing.  The range of $In(y)$ is $\theta = In(y) \in \left(\frac{3\pi}{2}, 2\pi \right)$, where $\cos(\theta)$ is increasing. \\



The range of $\cos(\theta)$ on $\theta \in \left(\frac{3\pi}{2}, 2\pi \right)$ is $x = \cos(\theta) \in (0, 1)$, where $Out(x)$ is increasing, because $Out$ is always increasing. \\


\[
H(x) = increasing \circ increasing \circ decreasing = decreasing
\]











\textbf{\textcolor{purple!80!black}{On $x = y \in \left( -\frac{\pi}{8}, 0 \right)$, }}



$In(y) = \pi - 4y$ is decreasing, because $In$ is always decreasing.  The range of $In(y)$ is $\theta = In(y) \in \left(\pi, \frac{3\pi}{2} \right)$, where $\cos(\theta)$ is increasing. \\



The range of $\cos(\theta)$ on $\theta \in \left(\pi, \frac{3\pi}{2} \right)$ is $x = \cos(\theta) \in (-1, 0)$, where $Out(x)$ is increasing, because $Out$ is always increasing. \\


\[
H(x) = increasing \circ increasing \circ decreasing = decreasing
\]

















\textbf{\textcolor{purple!80!black}{On $x = y \in \left( 0, \frac{\pi}{8} \right)$, }}



$In(y) = \pi - 4y$ is decreasing, because $In$ is always decreasing.  The range of $In(y)$ is $\theta = In(y) \in \left(\frac{\pi}{2}, \pi \right)$, where $\cos(\theta)$ is decreasing. \\



The range of $\cos(\theta)$ on $\theta \in \left(\frac{\pi}{2}, \pi \right)$ is $x = \cos(\theta) \in (-1, 0)$, where $Out(x)$ is increasing, because $Out$ is always increasing. \\


\[
H(x) = increasing \circ decreasing \circ decreasing = increasing
\]















\textbf{\textcolor{purple!80!black}{On $x = y \in \left( \frac{\pi}{8}, \frac{\pi}{4} \right)$, }}



$In(y) = \pi - 4y$ is decreasing, because $In$ is always decreasing.  The range of $In(y)$ is $\theta = In(y) \in \left(0, \frac{\pi}{2} \right)$, where $\cos(\theta)$ is decreasing. \\



The range of $\cos(\theta)$ on $\theta \in \left(0, \frac{\pi}{2} \right)$ is $x = \cos(\theta) \in (0, 1)$, where $Out(x)$ is increasing, because $Out$ is always increasing. \\


\[
H(x) = increasing \circ decreasing \circ decreasing = increasing
\]









This behavior repeats with a period of $\frac{\pi}{2}$. \\



[ This agrees with the graph. ]





\textbf{\textcolor{blue!55!black}{$\blacktriangleright$ desmos graph}} 
\begin{center}
\desmos{mldto7vwwi}{400}{300}
\end{center}



















\textbf{\textcolor{blue!55!black}{Local Maximum and Minimum}}



$H$ is continuous on $(-\infty, \infty)$.


$\blacktriangleright$ $H$ switches from decreasing to increasing at $0$, which makes $0$ a crtical number and the location of a local minimum. \\



In our principal interval, we do not see a switch from increasing to decreasing, but $H$ is periodic.  $H$ switches from increasing to decreasing at $-\frac{\pi}{4}$ and $\frac{\pi}{4}$.  Those are one period apart, so we only need one of them.

$\blacktriangleright$ $T$ switches from increasing to decreasing at $\frac{\pi}{4}$, which makes $\frac{\pi}{4}$ a crtical number and the location of a local maximum. \\


These repeat with a period of $\frac{\pi}{2}$. \\



\[
T\left(0 \right) = 2 \cos\left(3 \cdot 0 \right) + 1 = 3
\]




\[
T\left( \frac{\pi}{4} \right) = 2 \cos\left(3 \cdot \frac{\pi}{4} \right) - 1 = -1
\]














\textbf{\textcolor{blue!55!black}{Global Maximum and Minimum}}


Since $H$ is continuous on $(-\infty, \infty)$, $0$ and $\frac{\pi}{4}$ are the only critical numbers, we also have global extrema here. \\





\[
T\left(0 \right) = 2 \cos\left(3 \cdot 0 \right) + 1 = 3
\]




\[
T\left( \frac{\pi}{4} \right) = 2 \cos\left(3 \cdot \frac{\pi}{4} \right) - 1 = -1
\]





These repeat with a period of $\frac{\pi}{2}$. \\











\textbf{\textcolor{blue!55!black}{Range}}


Since $H$ is continuous with a global maximum of $3$ and a global minimum of $-1$, the range is $[-1, 3]$.  \\ 



[ This agrees with the graph. ]


























\subsection*{Periodic}

$H$ is periodic with period $\frac{\pi}{2}$. \\



We have zeros in the principal interval as $-\frac{\pi}{12}$ and $\frac{\pi}{12}$. \\


To describe all of the zeros of $H$, we need to include all numbers which are these two numbers plus or minus and whole number of $\frac{\pi}{2}$. \\



\[
\left\{  -\frac{\pi}{12} + k  \cdot \frac{\pi}{2}  \, | \,  k \in \mathbb{Z}         \right\}
\]



\[
\left\{  \frac{\pi}{12} + k  \cdot \frac{\pi}{2}  \, | \,  k \in \mathbb{Z}         \right\}
\]




Similarly, we could describe the intervals by including $k  \cdot \frac{\pi}{2} $. \\


















\begin{center}
\textbf{\textcolor{green!50!black}{ooooo-=-=-=-ooOoo-=-=-=-ooooo}} \\

more examples can be found by following this link\\ \link[More Examples of Trigonometric Functions]{https://ximera.osu.edu/csccmathematics/precalculus2/precalculus2/trigonometricFunctions/examples/exampleList}

\end{center}

\end{document}

