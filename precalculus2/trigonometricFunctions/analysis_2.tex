\documentclass{ximera}


\graphicspath{
  {./}
  {ximeraTutorial/}
  {basicPhilosophy/}
}

\newcommand{\mooculus}{\textsf{\textbf{MOOC}\textnormal{\textsf{ULUS}}}}


\usepackage{tkz-euclide}\usepackage{tikz}
\usepackage{tikz-cd}
\usetikzlibrary{arrows}
\tikzset{>=stealth,commutative diagrams/.cd,
  arrow style=tikz,diagrams={>=stealth}} %% cool arrow head
\tikzset{shorten <>/.style={ shorten >=#1, shorten <=#1 } } %% allows shorter vectors

\usetikzlibrary{backgrounds} %% for boxes around graphs
\usetikzlibrary{shapes,positioning}  %% Clouds and stars
\usetikzlibrary{matrix} %% for matrix
\usepgfplotslibrary{polar} %% for polar plots
\usepgfplotslibrary{fillbetween} %% to shade area between curves in TikZ
\usetkzobj{all}
\usepackage[makeroom]{cancel} %% for strike outs
%\usepackage{mathtools} %% for pretty underbrace % Breaks Ximera
%\usepackage{multicol}
\usepackage{pgffor} %% required for integral for loops



%% http://tex.stackexchange.com/questions/66490/drawing-a-tikz-arc-specifying-the-center
%% Draws beach ball
\tikzset{pics/carc/.style args={#1:#2:#3}{code={\draw[pic actions] (#1:#3) arc(#1:#2:#3);}}}



\usepackage{array}
\setlength{\extrarowheight}{+.1cm}
\newdimen\digitwidth
\settowidth\digitwidth{9}
\def\divrule#1#2{
\noalign{\moveright#1\digitwidth
\vbox{\hrule width#2\digitwidth}}}
























%%This is to help with formatting on future title pages.
\newenvironment{sectionOutcomes}{}{}


\title{Analysis}

\begin{document}

\begin{abstract}
cosine
\end{abstract}
\maketitle












\begin{example}



Analyze  $g(t) = -2 \cos(\pi - 3t) + 1$


\begin{explanation}

$\blacktriangleright$ \textbf{Horizontal}


One period of the cosine wave occurs from where the inside equals $0$ to where the inside equals $2 \pi$.


\begin{align*}
\pi - 3t  & =  0 \\
\pi       & =  3t \\
\frac{\pi}{3}       & =  t
\end{align*}



\begin{align*}
\pi - 3t  & =  2 \pi \\
-\pi      & =  3t \\
\frac{-\pi}{3}       & =  t
\end{align*}



$\blacktriangleright$ $\frac{\pi}{3}$ is the starting line for our one period.

$\blacktriangleright$ $\frac{-\pi}{3}$ is the finish line for our our period.

Our wave runs backwards. \\




$\cos(\pi - 3t)$ would equal $1$ at $\frac{\pi}{3}$ and $\frac{-\pi}{3}$. 

Halfway is $0$ and $\cos(\pi - 3t)$ would equal $-1$ there.

A quarterway is $\frac{\pi}{6}$, where $\cos(\pi - 3t) = 0$.  (a zero)

Three-quarterway is $\frac{-\pi}{6}$, where $\cos(\pi - 3t) = 0$. (a zero)






$\blacktriangleright$ \textbf{Vertical}


The maximum value of $\cos(\pi - 3t)$ is $1$.  The leading coefficient is $-2$, which is negative.  Therefore, the maximum value becomes a minimum value. The minimum value of $g(t)$ is $-2 \cdot 1 + 1 = -1$, which occurs at $\frac{\pi}{3}$ and at $\frac{-\pi}{3}$. 




The minimum value of $\cos(\pi - 3t)$ is $-1$.  The leading coefficient is $-2$, which is negative.   Therefore, the minimum value becomes a maximum value.  The maximum value of $g(t)$ is $-2 \cdot (-1) + 1 = 3$, which occurs at $0$. 








Graph $y = g(t)$, for one period.

\begin{image}
\begin{tikzpicture}
  \begin{axis}[
            domain=-7:7, ymax=5, xmax=7, ymin=-5, xmin=-7,
            axis lines =center, xlabel={$r$}, ylabel={$y$}, grid = major, grid style={dashed},
            ytick={-4,-2,0,2,4},
            xtick={-7.85, -6.28, -4.71, -3.142, -1.57, 0, 1.57, 3.142, 4.71, 6.28, 7.85},
            xticklabels={$\tfrac{-5\pi}{2}$,$-2\pi$,$\tfrac{-3\pi}{2}$,$-\pi$,$\tfrac{-\pi}{2}$,$0$,$\tfrac{\pi}{2}$, $\pi$, $\tfrac{3\pi}{2}$, $2\pi$, $\tfrac{5\pi}{2}$},
            yticklabels={$-4$,$-2$,$0$,$2$,$4$}, 
            ticklabel style={font=\scriptsize},
            every axis y label/.style={at=(current axis.above origin),anchor=south},
            every axis x label/.style={at=(current axis.right of origin),anchor=west},
            axis on top
          ]
          

            \addplot [line width=2, penColor, smooth,samples=300,domain=(-1.047:1.047)] {-2*cos(deg(3.14-(3*x)))+1};

            %\addplot [color=penColor,only marks,mark=*] coordinates{(0,0)};
            %\addplot [color=penColor,fill=white,only marks,mark=*] coordinates{(6.28,0)};
            %\addplot [color=penColor2,only marks,mark=*] coordinates{(0,1)};
            %\addplot [color=penColor2,fill=white,only marks,mark=*] coordinates{(6.28,1)};


  \end{axis}
\end{tikzpicture}
\end{image}



Now we can extend this to the whole real line.


The period is $\frac{\pi}{3} - \frac{-\pi}{3} = \frac{2\pi}{3}$.




\begin{itemize}
\item $T(r)$ has a maximum of $3$, which occurs at $\{  0 \pm k \frac{2\pi}{3} \, | \, k \in \mathbb{Z} \}$.

\item $T(r)$ has a minimum of $-1$, which occurs at $\{  \frac{\pi}{3} \pm k \frac{2\pi}{3} \, | \, k \in \mathbb{Z} \}$.

\item $T(r)$ decreases on $\left[  0 \pm k \frac{2\pi}{3},    \frac{\pi}{3} \pm k \frac{2\pi}{3}    \right]$, where $k \in \mathbb{Z}$.

\item $T(r)$ increases on $\left[  \frac{\pi}{3} \pm k \frac{2\pi}{3},    0 \pm (k+1) \frac{2\pi}{3}    \right]$, where $k \in \mathbb{Z}$.
\end{itemize}






Graph $y = g(t)$.


\begin{image}
\begin{tikzpicture}
  \begin{axis}[
            domain=-7:7, ymax=5, xmax=7, ymin=-5, xmin=-7,
            axis lines =center, xlabel={$r$}, ylabel={$y$}, grid = major, grid style={dashed},
            ytick={-4,-2,0,2,4},
            xtick={-7.85, -6.28, -4.71, -3.142, -1.57, 0, 1.57, 3.142, 4.71, 6.28, 7.85},
            xticklabels={$\tfrac{-5\pi}{2}$,$-2\pi$,$\tfrac{-3\pi}{2}$,$-\pi$,$\tfrac{-\pi}{2}$,$0$,$\tfrac{\pi}{2}$, $\pi$, $\tfrac{3\pi}{2}$, $2\pi$, $\tfrac{5\pi}{2}$},
            yticklabels={$-4$,$-2$,$0$,$2$,$4$}, 
            ticklabel style={font=\scriptsize},
            every axis y label/.style={at=(current axis.above origin),anchor=south},
            every axis x label/.style={at=(current axis.right of origin),anchor=west},
            axis on top
          ]
          

            \addplot [line width=2, penColor, smooth,samples=300,domain=(-6.5:6.5),<->] {-2*cos(deg(3.14-(3*x)))+1};

            %\addplot [color=penColor,only marks,mark=*] coordinates{(0,0)};
            %\addplot [color=penColor,fill=white,only marks,mark=*] coordinates{(6.28,0)};
            %\addplot [color=penColor2,only marks,mark=*] coordinates{(0,1)};
            %\addplot [color=penColor2,fill=white,only marks,mark=*] coordinates{(6.28,1)};


  \end{axis}
\end{tikzpicture}
\end{image}




\end{explanation}




\end{example}

























\begin{center}
\textbf{\textcolor{green!50!black}{ooooo-=-=-=-ooOoo-=-=-=-ooooo}} \\

more examples can be found by following this link\\ \link[More Examples of Trigonometric Functions]{https://ximera.osu.edu/csccmathematics/precalculus2/precalculus2/trigonometricFunctions/examples/exampleList}

\end{center}

\end{document}

