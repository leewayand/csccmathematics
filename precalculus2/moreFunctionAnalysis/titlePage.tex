\documentclass{ximera}


\graphicspath{
  {./}
  {ximeraTutorial/}
  {basicPhilosophy/}
}

\newcommand{\mooculus}{\textsf{\textbf{MOOC}\textnormal{\textsf{ULUS}}}}


\usepackage{tkz-euclide}\usepackage{tikz}
\usepackage{tikz-cd}
\usetikzlibrary{arrows}
\tikzset{>=stealth,commutative diagrams/.cd,
  arrow style=tikz,diagrams={>=stealth}} %% cool arrow head
\tikzset{shorten <>/.style={ shorten >=#1, shorten <=#1 } } %% allows shorter vectors

\usetikzlibrary{backgrounds} %% for boxes around graphs
\usetikzlibrary{shapes,positioning}  %% Clouds and stars
\usetikzlibrary{matrix} %% for matrix
\usepgfplotslibrary{polar} %% for polar plots
\usepgfplotslibrary{fillbetween} %% to shade area between curves in TikZ
\usetkzobj{all}
\usepackage[makeroom]{cancel} %% for strike outs
%\usepackage{mathtools} %% for pretty underbrace % Breaks Ximera
%\usepackage{multicol}
\usepackage{pgffor} %% required for integral for loops



%% http://tex.stackexchange.com/questions/66490/drawing-a-tikz-arc-specifying-the-center
%% Draws beach ball
\tikzset{pics/carc/.style args={#1:#2:#3}{code={\draw[pic actions] (#1:#3) arc(#1:#2:#3);}}}



\usepackage{array}
\setlength{\extrarowheight}{+.1cm}
\newdimen\digitwidth
\settowidth\digitwidth{9}
\def\divrule#1#2{
\noalign{\moveright#1\digitwidth
\vbox{\hrule width#2\digitwidth}}}
























%%This is to help with formatting on future title pages.
\newenvironment{sectionOutcomes}{}{}


\title{More Function Analysis}

\begin{document}

\begin{abstract}
%
\end{abstract}
\maketitle






What do we want when analyzing a function?


$\blacktriangleright$ First, and foremost, we want an algebraic description of every detail about our function. 

\begin{itemize}
\item domain
\item zeros 
\item dicsontinuities and singularities
\item intervals of continuity
\item critical numbers
\item intervals where increasing and decreasing
\item global maximum and minimium
\item local maximums and minimums
\item end-behavior
\item limiting behavior
\end{itemize}



$\blacktriangleright$ Secondly, we would like a nice graph, including corresponding important points and auxillary graphing items.

\begin{itemize}
	\item intercepts
	\item endpoints
	\item vertical asymptotes
	\item horizontal asymptotes
\end{itemize}




We often turn to technology for a nice graph and we must remember that it is plotting individual points and may not understand the larger implications. As humans, we possess more imformation from our algebraic description and can enhance the graph with exaggerations and auxillary graphing items to help the reader understand the function better.




\subsection{Learning Outcomes}


\begin{sectionOutcomes}
In this section, students will 

\begin{itemize}
\item analyze functions from their formula.
\end{itemize}
\end{sectionOutcomes}

\end{document}
