\documentclass{ximera}

%\usepackage{todonotes}

\newcommand{\todo}{}

\usepackage{esint} % for \oiint
\ifxake%%https://math.meta.stackexchange.com/questions/9973/how-do-you-render-a-closed-surface-double-integral
\renewcommand{\oiint}{{\large\bigcirc}\kern-1.56em\iint}
\fi


\graphicspath{
  {./}
  {ximeraTutorial/}
  {basicPhilosophy/}
  {functionsOfSeveralVariables/}
  {normalVectors/}
  {lagrangeMultipliers/}
  {vectorFields/}
  {greensTheorem/}
  {shapeOfThingsToCome/}
  {dotProducts/}
  {partialDerivativesAndTheGradientVector/}
  {../productAndQuotientRules/exercises/}
  {../normalVectors/exercisesParametricPlots/}
  {../continuityOfFunctionsOfSeveralVariables/exercises/}
  {../partialDerivativesAndTheGradientVector/exercises/}
  {../directionalDerivativeAndChainRule/exercises/}
  {../commonCoordinates/exercisesCylindricalCoordinates/}
  {../commonCoordinates/exercisesSphericalCoordinates/}
  {../greensTheorem/exercisesCurlAndLineIntegrals/}
  {../greensTheorem/exercisesDivergenceAndLineIntegrals/}
  {../shapeOfThingsToCome/exercisesDivergenceTheorem/}
  {../greensTheorem/}
  {../shapeOfThingsToCome/}
  {../separableDifferentialEquations/exercises/}
  {vectorFields/}
}

\newcommand{\mooculus}{\textsf{\textbf{MOOC}\textnormal{\textsf{ULUS}}}}

\usepackage{tkz-euclide}
\usepackage{tikz}
\usepackage{tikz-cd}
\usetikzlibrary{arrows}
\tikzset{>=stealth,commutative diagrams/.cd,
  arrow style=tikz,diagrams={>=stealth}} %% cool arrow head
\tikzset{shorten <>/.style={ shorten >=#1, shorten <=#1 } } %% allows shorter vectors

\usetikzlibrary{backgrounds} %% for boxes around graphs
\usetikzlibrary{shapes,positioning}  %% Clouds and stars
\usetikzlibrary{matrix} %% for matrix
\usepgfplotslibrary{polar} %% for polar plots
\usepgfplotslibrary{fillbetween} %% to shade area between curves in TikZ
%\usetkzobj{all}
\usepackage[makeroom]{cancel} %% for strike outs
%\usepackage{mathtools} %% for pretty underbrace % Breaks Ximera
%\usepackage{multicol}
\usepackage{pgffor} %% required for integral for loops



%% http://tex.stackexchange.com/questions/66490/drawing-a-tikz-arc-specifying-the-center
%% Draws beach ball
\tikzset{pics/carc/.style args={#1:#2:#3}{code={\draw[pic actions] (#1:#3) arc(#1:#2:#3);}}}



\usepackage{array}
\setlength{\extrarowheight}{+.1cm}
\newdimen\digitwidth
\settowidth\digitwidth{9}
\def\divrule#1#2{
\noalign{\moveright#1\digitwidth
\vbox{\hrule width#2\digitwidth}}}




% \newcommand{\RR}{\mathbb R}
% \newcommand{\R}{\mathbb R}
% \newcommand{\N}{\mathbb N}
% \newcommand{\Z}{\mathbb Z}

\newcommand{\sagemath}{\textsf{SageMath}}


%\renewcommand{\d}{\,d\!}
%\renewcommand{\d}{\mathop{}\!d}
%\newcommand{\dd}[2][]{\frac{\d #1}{\d #2}}
%\newcommand{\pp}[2][]{\frac{\partial #1}{\partial #2}}
% \renewcommand{\l}{\ell}
%\newcommand{\ddx}{\frac{d}{\d x}}

% \newcommand{\zeroOverZero}{\ensuremath{\boldsymbol{\tfrac{0}{0}}}}
%\newcommand{\inftyOverInfty}{\ensuremath{\boldsymbol{\tfrac{\infty}{\infty}}}}
%\newcommand{\zeroOverInfty}{\ensuremath{\boldsymbol{\tfrac{0}{\infty}}}}
%\newcommand{\zeroTimesInfty}{\ensuremath{\small\boldsymbol{0\cdot \infty}}}
%\newcommand{\inftyMinusInfty}{\ensuremath{\small\boldsymbol{\infty - \infty}}}
%\newcommand{\oneToInfty}{\ensuremath{\boldsymbol{1^\infty}}}
%\newcommand{\zeroToZero}{\ensuremath{\boldsymbol{0^0}}}
%\newcommand{\inftyToZero}{\ensuremath{\boldsymbol{\infty^0}}}



% \newcommand{\numOverZero}{\ensuremath{\boldsymbol{\tfrac{\#}{0}}}}
% \newcommand{\dfn}{\textbf}
% \newcommand{\unit}{\,\mathrm}
% \newcommand{\unit}{\mathop{}\!\mathrm}
% \newcommand{\eval}[1]{\bigg[ #1 \bigg]}
% \newcommand{\seq}[1]{\left( #1 \right)}
% \renewcommand{\epsilon}{\varepsilon}
% \renewcommand{\phi}{\varphi}


% \renewcommand{\iff}{\Leftrightarrow}

% \DeclareMathOperator{\arccot}{arccot}
% \DeclareMathOperator{\arcsec}{arcsec}
% \DeclareMathOperator{\arccsc}{arccsc}
% \DeclareMathOperator{\si}{Si}
% \DeclareMathOperator{\scal}{scal}
% \DeclareMathOperator{\sign}{sign}


%% \newcommand{\tightoverset}[2]{% for arrow vec
%%   \mathop{#2}\limits^{\vbox to -.5ex{\kern-0.75ex\hbox{$#1$}\vss}}}
% \newcommand{\arrowvec}[1]{{\overset{\rightharpoonup}{#1}}}
% \renewcommand{\vec}[1]{\arrowvec{\mathbf{#1}}}
% \renewcommand{\vec}[1]{{\overset{\boldsymbol{\rightharpoonup}}{\mathbf{#1}}}}

% \newcommand{\point}[1]{\left(#1\right)} %this allows \vector{ to be changed to \vector{ with a quick find and replace
% \newcommand{\pt}[1]{\mathbf{#1}} %this allows \vec{ to be changed to \vec{ with a quick find and replace
% \newcommand{\Lim}[2]{\lim_{\point{#1} \to \point{#2}}} %Bart, I changed this to point since I want to use it.  It runs through both of the exercise and exerciseE files in limits section, which is why it was in each document to start with.

% \DeclareMathOperator{\proj}{\mathbf{proj}}
% \newcommand{\veci}{{\boldsymbol{\hat{\imath}}}}
% \newcommand{\vecj}{{\boldsymbol{\hat{\jmath}}}}
% \newcommand{\veck}{{\boldsymbol{\hat{k}}}}
% \newcommand{\vecl}{\vec{\boldsymbol{\l}}}
% \newcommand{\uvec}[1]{\mathbf{\hat{#1}}}
% \newcommand{\utan}{\mathbf{\hat{t}}}
% \newcommand{\unormal}{\mathbf{\hat{n}}}
% \newcommand{\ubinormal}{\mathbf{\hat{b}}}

% \newcommand{\dotp}{\bullet}
% \newcommand{\cross}{\boldsymbol\times}
% \newcommand{\grad}{\boldsymbol\nabla}
% \newcommand{\divergence}{\grad\dotp}
% \newcommand{\curl}{\grad\cross}
%\DeclareMathOperator{\divergence}{divergence}
%\DeclareMathOperator{\curl}[1]{\grad\cross #1}
% \newcommand{\lto}{\mathop{\longrightarrow\,}\limits}

% \renewcommand{\bar}{\overline}

\colorlet{textColor}{black}
\colorlet{background}{white}
\colorlet{penColor}{blue!50!black} % Color of a curve in a plot
\colorlet{penColor2}{red!50!black}% Color of a curve in a plot
\colorlet{penColor3}{red!50!blue} % Color of a curve in a plot
\colorlet{penColor4}{green!50!black} % Color of a curve in a plot
\colorlet{penColor5}{orange!80!black} % Color of a curve in a plot
\colorlet{penColor6}{yellow!70!black} % Color of a curve in a plot
\colorlet{fill1}{penColor!20} % Color of fill in a plot
\colorlet{fill2}{penColor2!20} % Color of fill in a plot
\colorlet{fillp}{fill1} % Color of positive area
\colorlet{filln}{penColor2!20} % Color of negative area
\colorlet{fill3}{penColor3!20} % Fill
\colorlet{fill4}{penColor4!20} % Fill
\colorlet{fill5}{penColor5!20} % Fill
\colorlet{gridColor}{gray!50} % Color of grid in a plot

\newcommand{\surfaceColor}{violet}
\newcommand{\surfaceColorTwo}{redyellow}
\newcommand{\sliceColor}{greenyellow}




\pgfmathdeclarefunction{gauss}{2}{% gives gaussian
  \pgfmathparse{1/(#2*sqrt(2*pi))*exp(-((x-#1)^2)/(2*#2^2))}%
}


%%%%%%%%%%%%%
%% Vectors
%%%%%%%%%%%%%

%% Simple horiz vectors
\renewcommand{\vector}[1]{\left\langle #1\right\rangle}


%% %% Complex Horiz Vectors with angle brackets
%% \makeatletter
%% \renewcommand{\vector}[2][ , ]{\left\langle%
%%   \def\nextitem{\def\nextitem{#1}}%
%%   \@for \el:=#2\do{\nextitem\el}\right\rangle%
%% }
%% \makeatother

%% %% Vertical Vectors
%% \def\vector#1{\begin{bmatrix}\vecListA#1,,\end{bmatrix}}
%% \def\vecListA#1,{\if,#1,\else #1\cr \expandafter \vecListA \fi}

%%%%%%%%%%%%%
%% End of vectors
%%%%%%%%%%%%%

%\newcommand{\fullwidth}{}
%\newcommand{\normalwidth}{}



%% makes a snazzy t-chart for evaluating functions
%\newenvironment{tchart}{\rowcolors{2}{}{background!90!textColor}\array}{\endarray}

%%This is to help with formatting on future title pages.
\newenvironment{sectionOutcomes}{}{}



%% Flowchart stuff
%\tikzstyle{startstop} = [rectangle, rounded corners, minimum width=3cm, minimum height=1cm,text centered, draw=black]
%\tikzstyle{question} = [rectangle, minimum width=3cm, minimum height=1cm, text centered, draw=black]
%\tikzstyle{decision} = [trapezium, trapezium left angle=70, trapezium right angle=110, minimum width=3cm, minimum height=1cm, text centered, draw=black]
%\tikzstyle{question} = [rectangle, rounded corners, minimum width=3cm, minimum height=1cm,text centered, draw=black]
%\tikzstyle{process} = [rectangle, minimum width=3cm, minimum height=1cm, text centered, draw=black]
%\tikzstyle{decision} = [trapezium, trapezium left angle=70, trapezium right angle=110, minimum width=3cm, minimum height=1cm, text centered, draw=black]


\title{Factoring}

\begin{document}

\begin{abstract}
Rational Roots Theorem
\end{abstract}
\maketitle




Much of our analysis of polynomial functions relies on zeros or roots. In particular, we locate local maximums and minimums where the derivative equals $0$.  Therefore, factoring is very high on our list of skills. \\

Unfortunately, factoring a random polynomial is almost impossible.  So, we need some help.


$\blacktriangleright$ First, we hope the instructor has given us a polynomial which we have a chance of factoring.  Otherwise, we really don't know what to do. \\




$\blacktriangleright$ Second, we need some good guesses. \\


We would like to factor the polynomial in order to identify zeros or roots.  But, if we had an idea of the roots, then we could guess a factor and try to factor it out of the polynomial. Then, we hope the other factor is easier to work with.\\

\begin{center}
\textbf{\textcolor{red!80!black}{So, what are good guess for roots of a polynomial?}}
\end{center}


\textbf{\textcolor{purple!85!blue}{Under certain conditions}}, we can actually get some good guesses for roots of the polynomial and thus factors.






\begin{theorem} \textbf{\textcolor{green!50!black}{Rational Roots Theorem}} 



Let $p(x) = a_n x^n + a_{n-1} x^{n-1} + \cdots + a_1 x + a_0$ be a polynomial of degree $n$ with integer coefficients, $a_i \in \mathbb{Z}$ for all $i$, such that $a_n, a_0 \ne 0$.

If $p(x)$ has any \textbf{rational roots}, then they come from this set: 
\[
\left\{ \frac{p}{q} \, | \, p \, \text{ is a factor of } \, a_0 \, \text{ and } \, q \, \text{ is a factor of } \, a_n  \right\}
\]



\end{theorem}



This set could have a lot of members, but that is better than an infinite list of possibilities.  A lot better! \\


\begin{warning} \textbf{\textcolor{red!80!black}{just Rational Roots}} \\


The Rational Roots Theorem tells us how to find rational roots, not all roots.  If the polynomial has other types of roots, then this doesn't help to find them.


\end{warning}

\textbf{Note:} If $a_0 = 0$, then the constant term is zero and the variable is a common factor of all of the terms.  It can be factored out.






\begin{example} RRT


Let $p(x) = 3\, x^3 - 16 \, x^2 + 3 \, x + 10$ \\


This is a cubic polynomial with integer coefficients and the constant term is not $0$. It meets the conditions of the RRT.


A quick glance at a graph suggests that this cubic polynomial has three real roots. The real roots might be rational numbers or they might be irrational.

The Rational Roots Theorem tells us that if $p$ has any \textbf{rational roots} then they must come from this list:


\[
\left\{  \frac{10}{1},  \frac{5}{1}, \frac{2}{1}, \frac{1}{1}, \frac{10}{3},  \frac{5}{3}, \frac{2}{3}, \frac{1}{3},   -\frac{10}{1},  -\frac{5}{1}, -\frac{2}{1}, -\frac{1}{1}, -\frac{10}{3},  -\frac{5}{3}, -\frac{2}{3}, -\frac{1}{3}                   \right\}
\]


The list is generated by the factors of $10$ : $\{ 10, 5, 2, 1, -1, -2, -5, -10 \}$ \\

over the factors of $3$ : $\{ 3, 1, -1, -3 \}$ \\


The fractions were formed from every possible combination of these sets of factors with equivalent fractions not list.  For example, $\frac{10}{1} = \frac{-10}{-1}$, so, only $\frac{10}{1}$ is listed.  All of the possible distinct fractions are listed.


We have $16$ possibilities for roots of $p$.  We can test them and see which is a root.


\begin{itemize}
\item $p(10) =  3 \cdot 10^3 - 16 \cdot 10^2 + 3 \cdot 10 + 10 = 1440 \ne 0$ \\
\item $p(5) =  3 \cdot 5^3 - 16 \cdot 5^2 + 3 \cdot 5 + 10 = 0$, therefore,  $5$ is a root\\
\end{itemize}


We now know $x-5$ is a factor of $p(x)$.  Therefore, $p(x)$ factors like $p(x) = (x-5) (a \, x^2 + b \, x + c)$.







\[
3\, x^3 - 16 \, x^2 + 3 \, x + 10  = (x-5) (a \, x^2 + b \, x + c) 
\]


If we multiply out the right side and collect like terms, then we get


\[
3\, x^3 - 16 \, x^2 + 3 \, x + 10 = \answer{a} \, x^3 + \left( \answer{b-5a} \right) x^2 + \left( \answer{c-5b} \right) x - 5c
\]



$\blacktriangleright$ Matching Constant Terms \\


$10 = -5c$, therefore $c = \answer{-2}$ \\



$\blacktriangleright$ Matching Leading Terms \\


$3 = a$ \\



$\blacktriangleright$ Matching Linear Terms \\

$3 = \answer{c-5b}$ \\

Substituting $-2$ in for $c$ gives $3 = -2 - 5b$ \\

Which tells us that $\answer{-1} = b$ \\



We now know the values of $a$, $b$, and $c$.


\[
3\, x^3 - 16 \, x^2 + 3 \, x + 10  = (x - 5) (3 x^2 -  x - 2) 
\]


We can factor this quadratic factor.

\[
3\, x^3 - 16 \, x^2 + 3 \, x + 10  = (x - 5) \left(\answer{3x + 2} \right) (x - 1) 
\]


The other roots are $-\frac{2}{3}$ and $1$. 


Or, we could have used the Quadratic Formula to obtain the roots and then used the roots to factor. \\


Of course, $-\frac{2}{3}$ and $1$ are rational numbers.  Therefore, we could have continued to use the Rational Roots Theorem.  These two fractions must appear in the list of all possible rational roots.








\end{example}














\begin{example} RRT


Let $T(y) = 2\, y^4 - 7 \, y^3 - 10 \, y^2 + 21 \, y + 12$ \\


This is a 4th degree polynomial with integer coefficients and the constant term is not $0$. It meets the conditions of the RRT.


A quick glance at its graph suggests that this 4th degree polynomial has four real roots. The real roots might be rational numbers or they might be irrational.

If $p$ has any rational roots then they must come from this list:




\[
\left\{  \frac{12}{1},  \frac{6}{1}, \frac{4}{1}, \frac{3}{1}, \frac{2}{1}, \frac{1}{1}, \frac{12}{2},  \frac{6}{2}, \frac{4}{2}, \frac{3}{2}, \frac{2}{2}, \frac{1}{2},  -\frac{12}{1},  -\frac{6}{1}, -\frac{4}{1}, -\frac{3}{1}, -\frac{2}{1}, -\frac{1}{1}, -\frac{12}{2},  -\frac{6}{2}, -\frac{4}{2}, -\frac{3}{2}, -\frac{2}{2}, -\frac{1}{2}                 \right\}
\]


The factors of $12$ are $\{ 12, 6, 4, 3, 2, 1, -1, -2, -3, -4, -6, -12 \}$ \\

over the factors of $2$ are $\{ 2, 1, -1, -2 \}$ \\


The fractions were formed from every possible combination of these sets of factors.

But it might help to clean this list up a bit.

\[
\left\{  12,  6, 4, 3, 2, 1,  \frac{3}{2},  \frac{1}{2},  -12,  -6, -4, -3, -2, -1,  -\frac{3}{2},  -\frac{1}{2}                 \right\}
\]





We have $16$ possibilities for roots of $T$.  We can test them and see which is a root.










\begin{itemize}
\item $T(12) =  2 \cdot 12^4 - 7 \cdot 12^3 - 10 \cdot 12^2 + 21 \cdot 12 + 12 = 28200$ \\
\item $T(6) =  2 \cdot 6^4 - 7 \cdot 6^3 - 10 \cdot 6^2 + 21 \cdot 6 + 12 = \answer{858}$  \\
\item $T(4) =  2 \cdot 4^4 - 7 \cdot 4^3 - 10 \cdot 4^2 + 21 \cdot 4 + 12 = \answer{0}$  \\
\end{itemize}


We now know $y-4$ is a factor of $T(y)$.  $T(y)$ factors like $T(y) = (y-4) (a \, y^3 + b \, y^2 + c \, y + d)$. \\

We can  multiply this out (or use long division) to get the cubic factor or we could try more root candidates from our list.


\begin{itemize}
\item $T(3) =  \answer{-42}$  \\
\item $T(2) =  \answer{-10}$  \\
\item $T(1) =  \answer{18}$  \\
\item $T\left(\tfrac{3}{2}\right) =  \answer{\frac{15}{2}}$  \\
\item $T\left(\tfrac{1}{2}\right) =  \answer{\frac{77}{4}}$  \\
\item $T(-12) =  \answer{51888}$  \\
\item $T(-6) =  \answer{3630}$  \\
\item $T(-4) =  \answer{728}$  \\
\item $T(-3) =  \answer{210}$  \\
\item $T(-2) =  \answer{18}$  \\
\item $T(-1) =  \answer{-10}$  \\
\item $T\left(-\tfrac{3}{2}\right) =  \answer{\frac{-33}{4}}$  \\
\item $T\left(-\tfrac{1}{2}\right) =  \answer{0}$  
\end{itemize}

There is only one other factor with a rational root and that is $y+\frac{1}{2}$, which corresponds to the root $-\frac{1}{2}$ \\



We now know $y-4$ and $y+\frac{1}{2}$ are factors of $T(y)$.  $T(y)$ factors like $T(y) = (y-4) \left(y+\frac{1}{2}\right) (a \, y^2 + b \, y + c)$. \\





\end{example}











If you are thinking that this is a long process, then you are correct.  We wouldn't bother with it, except it is so useful to have a factored polynomial.  But we might have some ways to speed up the process.  \\

It took us a lot of calculating to come up with the root $-\frac{1}{2}$.  A graph would have helped us select candidates.


\begin{center}
\desmos{yxnzawe2si}{400}{300}
\end{center}


From the graph, it appears we should select candidates near $-1.75$, $-0.5$, $1.75$, and $4$. We could then select the closet fractions from our list. This would have significantly shortened our evaluation list.  \\

It wouldn't take long to figure out that we have all of the rational roots. The other two roots must be irrational.  We'll need to begin factoring to identify these other roots. \\



$\blacktriangleright$ \textbf{\textcolor{blue!55!black}{Irrational Roots}} 


Luckily, in the previous example, we were left with a quadratic to factor.  We can do that.


\begin{explanation}



$T(y)$ factors like $T(y) = (y-4) \left(y+\frac{1}{2}\right) (a \, y^2 + b \, y + c)$. \\


Remember, $T$ was given to us in standard form

\[   T(y) = 2\, y^4 - 7 \, y^3 - 10 \, y^2 + 21 \, y + 12   \]


and the leading coefficient is $2$.  Therefore, $2$ has to be the leading coefficient in the factorization.

\[   T(y) = 2 (y-4) \left(y+\frac{1}{2}\right) (a \, y^2 + b \, y + c)   \]

This tells us that $a = 1$.


\[   T(y) = 2 (y-4) \left(y+\frac{1}{2}\right) (y^2 + b \, y + c)   \]




 It will help our thinking if we don't have fractions in our computations. So, let's multiply $2$ and $\left(y+\frac{1}{2}\right)$.



\[
T(y) = 2 (y-4) \left(y+\frac{1}{2}\right) (a \, y^2 + b \, y + c) = (y-4) (2y+1) (y^2 + b \, y + c)
\]


\[
T(y) = (2 \, y^2 - 7 \, y - 4) (y^2 + b \, y + c)
\]


Let's multiply everything out and collect like terms.


\[
T(y) = 2 y^4 + (2 b - 7) \, y^3 + (2c - 4 - 7 b) \, y^2 - (4 b + 7 c) \, y - 4 \, c
\]


\[
2\, y^4 - 7 \, y^3 - 10 \, y^2 + 21 \, y + 12 = 2 y^4 + (2 b - 7) \, y^3 + (2c - 4 - 7 b) \, y^2 - (4 b + 7 c) \, y - 4 \, c
\]


Equating terms gives us...



$12 = -4 c$, which tells us that $c =  \answer{-3}$. \\



Now that we know $a = 1$ and $c = -3$, let's update our coefficients.


$2\, y^4 - 7 \, y^3 - 10 \, y^2 + 21 \, y + 12 =$
\[
2 \, y^4 + \left( \answer{2 b - 7 } \right) \, y^3 + \left( \answer{-10 - 7 b} \right) \, y^2 - \left( \answer{4 b - 21} \right) \, y + 12
\]




Equating like terms gives us...


$-7 \, y^3 = (2 b - 7) \, y^3$, which tells us that $b = \answer{0}$. \\



\[
T(y) = 2 (y-4) \left(y+\frac{1}{2}\right) (y^2  - 3)
\]


From here we can view the quadratic as a difference of two squares and factor. 



\[
T(y) = 2 (y-4) \left(y+\frac{1}{2}\right) (y^2 - (\sqrt{3})^2)
\]



\[
T(y) = 2 (y-4) \left(y+\frac{1}{2}\right) (y - \sqrt{3}) (y + \sqrt{3})
\]


The other two roots were irrational real numbers: $-\sqrt{3}$ and $\sqrt{3}$.



\textbf{Note:} We could have used the Quadratic Formula.


\end{explanation}

If we will keep ourselves organized, we can reduce our writing of this process significantly. \\






\begin{procedure} \textbf{\textcolor{purple!85!blue}{Systems of Equations}} 



From

\[
2 \, y^4 - 7 \, y^3 - 10 \, y^2 + 21 \, y + 12 = 2 a \, y^4 + (2 b - 7 a) \, y^3 + (2 c - 4 a - 7 b) \, y^2 - (4 b + 7 c) \, y - 4 \, c
\]

we can see that 




\[
\begin{cases}
  2 &= 2 a   \\
  -7 &= 2 b - 7 a    \\
  -10 &= 2 c - 4 a - 7 b   \\
  21 &= -(4 b + 7 c)   \\
  12 &= -4 c
\end{cases}
\]


From this system of linear equations, we can see that $a = 1$ and $c = -3$.  If we substitute these values back into the system, then the systems becomes.






\[
\begin{cases}
  2 &= 2   \\
  -7 &= 2 b - 7     \\
  -10 &= -6 - 4 - 7 b   \\
  21 &= 4 b + 21   \\
  12 &= 12
\end{cases}
\]

From this system of linear equations, we can see that $b = 0$. \\


\[
T(y) = (y-4) \left(y+\frac{1}{2}\right) (y^2 - (\sqrt{3})^2)
\]



And, then we have a difference of two squares.



\[
T(y) = (y-4) \left(y+\frac{1}{2}\right) (y - \sqrt{3}) (y + \sqrt{3})
\]




\end{procedure}




This example illustrates a useful factoring fact.



\begin{fact} \textbf{\textcolor{purple!85!blue}{A Difference of Two Squares Factors}}

\[
a^2 - b^2 = (a-b)(a+b)
\]

\end{fact}



















\begin{center}
\textbf{\textcolor{green!50!black}{ooooo-=-=-=-ooOoo-=-=-=-ooooo}} \\

more examples can be found by following this link\\ \link[More Examples of Polynomial Functions]{https://ximera.osu.edu/csccmathematics/precalculus2/precalculus2/polynomialFunctions/examples/exampleList}

\end{center}





\end{document}
