\documentclass{ximera}


\graphicspath{
  {./}
  {ximeraTutorial/}
  {basicPhilosophy/}
}

\newcommand{\mooculus}{\textsf{\textbf{MOOC}\textnormal{\textsf{ULUS}}}}


\usepackage{tkz-euclide}\usepackage{tikz}
\usepackage{tikz-cd}
\usetikzlibrary{arrows}
\tikzset{>=stealth,commutative diagrams/.cd,
  arrow style=tikz,diagrams={>=stealth}} %% cool arrow head
\tikzset{shorten <>/.style={ shorten >=#1, shorten <=#1 } } %% allows shorter vectors

\usetikzlibrary{backgrounds} %% for boxes around graphs
\usetikzlibrary{shapes,positioning}  %% Clouds and stars
\usetikzlibrary{matrix} %% for matrix
\usepgfplotslibrary{polar} %% for polar plots
\usepgfplotslibrary{fillbetween} %% to shade area between curves in TikZ
\usetkzobj{all}
\usepackage[makeroom]{cancel} %% for strike outs
%\usepackage{mathtools} %% for pretty underbrace % Breaks Ximera
%\usepackage{multicol}
\usepackage{pgffor} %% required for integral for loops



%% http://tex.stackexchange.com/questions/66490/drawing-a-tikz-arc-specifying-the-center
%% Draws beach ball
\tikzset{pics/carc/.style args={#1:#2:#3}{code={\draw[pic actions] (#1:#3) arc(#1:#2:#3);}}}



\usepackage{array}
\setlength{\extrarowheight}{+.1cm}
\newdimen\digitwidth
\settowidth\digitwidth{9}
\def\divrule#1#2{
\noalign{\moveright#1\digitwidth
\vbox{\hrule width#2\digitwidth}}}
























%%This is to help with formatting on future title pages.
\newenvironment{sectionOutcomes}{}{}



\author{Lee Wayand}

\begin{document}
\begin{exercise}






\begin{question}

The rational roots theorem tells us that $\frac{3}{5}$ is a candidate for a root/zero of 

\[
p(x) 3x^4 - 2x^3 + 9x^2 + 8x - 5
\]


\begin{multipleChoice}
\choice{True}
\choice[correct]{False}
\end{multipleChoice}




\end{question}








\begin{question}

The rational roots theorem tells us that $\frac{5}{3}$ is a candidate for a root/zero of 

\[
p(x) 3x^4 - 2x^3 + 9x^2 + 8x - 5
\]


\begin{multipleChoice}
\choice[correct]{True}
\choice{False}
\end{multipleChoice}




\end{question}














\begin{question}

The rational roots theorem tells us that $\frac{5}{3}$ is a candidate for a root/zero of 

\[
p(x) 6x^7 - 2x^5 + 2x^4 + 8x^2 - 10
\]


\begin{multipleChoice}
\choice[correct]{True}
\choice{False}
\end{multipleChoice}




\end{question}













\begin{question}

The rational roots theorem tells us that $-\frac{2}{3}$ is a candidate for a root/zero of 

\[
p(x) 6x^7 - 2x^5 + 2x^4 + 8x^2 - 10
\]


\begin{multipleChoice}
\choice[correct]{True}
\choice{False}
\end{multipleChoice}




\end{question}












\begin{question}

The rational roots theorem tells us that $-\frac{5}{2}$ is a candidate for a root/zero of 

\[
p(x) 6x^7 - 2x^5 + 2x^4 + 8x^2 - 10
\]


\begin{multipleChoice}
\choice[correct]{True}
\choice{False}
\end{multipleChoice}




\end{question}











\begin{question}

The rational roots theorem tells us that $\frac{3}{2}$ is a candidate for a root/zero of 

\[
p(x) 6x^7 - 2x^5 + 2x^4 + 8x^2 - 10
\]


\begin{multipleChoice}
\choice{True}
\choice[correct]{False}
\end{multipleChoice}




\end{question}









\end{exercise}
\end{document}