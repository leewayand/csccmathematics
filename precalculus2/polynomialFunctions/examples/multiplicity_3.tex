\documentclass{ximera}


\graphicspath{
  {./}
  {ximeraTutorial/}
  {basicPhilosophy/}
}

\newcommand{\mooculus}{\textsf{\textbf{MOOC}\textnormal{\textsf{ULUS}}}}


\usepackage{tkz-euclide}\usepackage{tikz}
\usepackage{tikz-cd}
\usetikzlibrary{arrows}
\tikzset{>=stealth,commutative diagrams/.cd,
  arrow style=tikz,diagrams={>=stealth}} %% cool arrow head
\tikzset{shorten <>/.style={ shorten >=#1, shorten <=#1 } } %% allows shorter vectors

\usetikzlibrary{backgrounds} %% for boxes around graphs
\usetikzlibrary{shapes,positioning}  %% Clouds and stars
\usetikzlibrary{matrix} %% for matrix
\usepgfplotslibrary{polar} %% for polar plots
\usepgfplotslibrary{fillbetween} %% to shade area between curves in TikZ
\usetkzobj{all}
\usepackage[makeroom]{cancel} %% for strike outs
%\usepackage{mathtools} %% for pretty underbrace % Breaks Ximera
%\usepackage{multicol}
\usepackage{pgffor} %% required for integral for loops



%% http://tex.stackexchange.com/questions/66490/drawing-a-tikz-arc-specifying-the-center
%% Draws beach ball
\tikzset{pics/carc/.style args={#1:#2:#3}{code={\draw[pic actions] (#1:#3) arc(#1:#2:#3);}}}



\usepackage{array}
\setlength{\extrarowheight}{+.1cm}
\newdimen\digitwidth
\settowidth\digitwidth{9}
\def\divrule#1#2{
\noalign{\moveright#1\digitwidth
\vbox{\hrule width#2\digitwidth}}}
























%%This is to help with formatting on future title pages.
\newenvironment{sectionOutcomes}{}{}



\author{Lee Wayand}

\begin{document}
\begin{exercise}






\begin{idea} \textbf{\textcolor{blue!55!black}{Multiplicity}}


If we have a polynomial written in factored form

\[
p(x) = A (x-r_k)^{e_k} (x-r_{k-1})^{e_{k-1}}  \cdots ((x-r_1)^{e_1}
\]

then the multiplicity of the root $r_m$ is $e_m$. \\





\end{idea}






\begin{question}



The multiplicity of the root $8$ in $D(h) = -\frac{1}{3} (h - 1) (h - 8)^2 h^3 (h + 5)^4 (h - 7)^2$ is \answer{2}. \\



The graph will \wordChoice{\choice{cross} \choice[correct]{bounce back}} at the intercept $(8, 0)$.




\end{question}











\begin{question}



The multiplicity of the root $7$ in $D(h) = -\frac{1}{3} (h - 1) (h - 8)^2 h^3 (h + 5)^4 (h - 7)^2$ is \answer{2}. \\



The graph will \wordChoice{\choice{cross} \choice[correct]{bounce back}} at the intercept $(7, 0)$.




\end{question}












\begin{question}



The multiplicity of the root $1$ in $D(h) = -\frac{1}{3} (h - 1) (h - 8)^2 h^3 (h + 5)^4 (h - 7)^2$ is \answer{1}. \\



The graph will \wordChoice{\choice[correct]{cross} \choice{bounce back}} at the intercept $(1, 0)$.




\end{question}












\begin{question}



The multiplicity of the root $-5$ in $D(h) = -\frac{1}{3} (h - 1) (h - 8)^2 h^3 (h + 5)^4 (h - 7)^2$ is \answer{4}. \\



The graph will \wordChoice{\choice{cross} \choice[correct]{bounce back}} at the intercept $(-5, 0)$.




\end{question}








\end{exercise}
\end{document}