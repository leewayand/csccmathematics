\documentclass{ximera}


\graphicspath{
  {./}
  {ximeraTutorial/}
  {basicPhilosophy/}
}

\newcommand{\mooculus}{\textsf{\textbf{MOOC}\textnormal{\textsf{ULUS}}}}


\usepackage{tkz-euclide}\usepackage{tikz}
\usepackage{tikz-cd}
\usetikzlibrary{arrows}
\tikzset{>=stealth,commutative diagrams/.cd,
  arrow style=tikz,diagrams={>=stealth}} %% cool arrow head
\tikzset{shorten <>/.style={ shorten >=#1, shorten <=#1 } } %% allows shorter vectors

\usetikzlibrary{backgrounds} %% for boxes around graphs
\usetikzlibrary{shapes,positioning}  %% Clouds and stars
\usetikzlibrary{matrix} %% for matrix
\usepgfplotslibrary{polar} %% for polar plots
\usepgfplotslibrary{fillbetween} %% to shade area between curves in TikZ
\usetkzobj{all}
\usepackage[makeroom]{cancel} %% for strike outs
%\usepackage{mathtools} %% for pretty underbrace % Breaks Ximera
%\usepackage{multicol}
\usepackage{pgffor} %% required for integral for loops



%% http://tex.stackexchange.com/questions/66490/drawing-a-tikz-arc-specifying-the-center
%% Draws beach ball
\tikzset{pics/carc/.style args={#1:#2:#3}{code={\draw[pic actions] (#1:#3) arc(#1:#2:#3);}}}



\usepackage{array}
\setlength{\extrarowheight}{+.1cm}
\newdimen\digitwidth
\settowidth\digitwidth{9}
\def\divrule#1#2{
\noalign{\moveright#1\digitwidth
\vbox{\hrule width#2\digitwidth}}}
























%%This is to help with formatting on future title pages.
\newenvironment{sectionOutcomes}{}{}


\outcome{Use integral notation for both antiderivatives and definite integrals.}
\outcome{Compute definite integrals using geometry.}
\outcome{Compute definite integrals using the properties of integrals.}
\outcome{Justify the properties of definite integrals using algebra or geometry.}
\outcome{Understand how Riemann sums are used to find exact area.}
\outcome{Define net area.}
\outcome{Split the area under a curve into several pieces to aid with calculations.}
\outcome{Use symmetry to calculate definite integrals.}
\outcome{Explain geometrically why symmetry of a function simplifies calculation of some definite integrals.}


\title[Dig-In:]{The definite integral}

\begin{document}
\begin{abstract}
  Definite integrals compute net area.
\end{abstract}
\maketitle




The process of approximating areas under curves led to the notion of a Riemann sum

\[
A\approx\sum_{k=1}^n f(x_k^*)\Delta x,
\]
where $f$ is a nonnegative, continuous function on the interval $[a,b]$, and

 $x_k^*$ is a sample point for the $k^{th}$ rectangle, $k=1,2,..., n$.
 
 The limit of Riemann sums, as $n\to\infty$, gives the exact area between the curve $y=f(x)$ and the interval on the $x-$axis:
 
\[
A=\lim_{n\to\infty}\sum_{k=1}^n f(x_k^*)\Delta x.
\]
It does not matter whether we consider only right Riemann sums, or left Riemann sums, or midpoint Riemann sums, or others: the limit of any kind of Riemann sum as $n\to\infty$  is equal to the area, as long as $f$ is nonnegative and continuous on $[a,b]$. 


 
What happens when a continuous function $f$ assumes negative values  on the interval $[a,b]$?

We can still form Riemann sums and take the limit. 

The question is: What is the meaning of a Riemann sum in that case?

 \begin{example}
The graph of the function $f$ on the interval $[0,10]$ is given in the figure below.

 \begin{image}
  \begin{tikzpicture}[
      declare function = {f(\x) = 6+\x/2 - pow(\x,2)/4;}]
    \begin{axis}[  
        domain=0:10, xmin =-1,xmax=10.5,ymax=10,ymin=-15,
        width=6in,
        height=3in,xtick={0,2,4,...,10},
        xticklabels={$x_1^*=0$,$x_2^*=2$,$x_3^*=4$,$x_4^*=6$,$x_5^*=8$},
        %% ytick style={draw=none},
        %% yticklabels={},
        axis lines=center, xlabel=$x$, ylabel=$y$,
        every axis y label/.style={at=(current axis.above origin),anchor=south},
        every axis x label/.style={at=(current axis.right of origin),anchor=west},
        axis on top,
      ]
      \addplot [draw=penColor,fill=fill1] plot coordinates
               {({(0) * 2},{f(2)})
                 ({(1) * 2},{f(2) })} \closedcycle;

               \addplot [draw=penColor,fill=fill1] plot coordinates
               {({2-1) * 2},{f(4)})
                 ({(2) * 2},{f(4) })} \closedcycle;

               \addplot [draw=penColor,fill=fill1] plot coordinates
               {({3-1) * 2},{f(6)})
                 ({(3) * 2},{f(6) })} \closedcycle;

               \addplot [draw=penColor,fill=fill2] plot coordinates
               {({4-1) * 2},{f(8)})
                 ({(4) * 2},{f(8) })} \closedcycle;

               \addplot [draw=penColor,fill=fill2] plot coordinates
               {({5-1) * 2},{f(10)})
                 ({(5) * 2},{f(10) })} \closedcycle;
               
               \addplot [ultra thick,penColor, smooth] {f(x)};
              \node at (axis cs:1,7) {\large$f(2)\cdot 2$};
              \node at (axis cs:3,7) {\large$f(4)\cdot 2$};
              \node at (axis cs:5,5) {\large$f(6)\cdot 2$};
              \node at (axis cs:7,1) {\large$f(8)\cdot 2$};
              \node at (axis cs:9,1) {\large$f(10)\cdot 2$}; 
                \node at (axis cs:6.6,6.2) {\large$y=f(x)$};
    \end{axis}
  \end{tikzpicture}
\end{image}

 The figure illustrates a right Riemann sum with $n=5$ rectangles for $f$ on $[0,10]$.
\[
\sum_{k=1}^5 f(x_k^*)\Delta x= f(2)\Delta x+ f(4)\Delta x+ f(6)\Delta x+f(8)\Delta x+ f(10)\Delta x
\]

So, the Riemann sum is the sum of \textbf{signed areas} of rectangles:
rectangles that lie above the $x$-axis contribute positive values, and
rectangles that lie below the $x$-axis contribute negative values to
the Riemann sum.
\begin{align*}
  \sum_{k=1}^5 f(x_k^*)\Delta x= &\left(\underbrace{f(2)\Delta x}_{\text{nonnegative}}+  \underbrace{f(4)\Delta x}_{\text{nonnegative}}+  \underbrace{f(6)\Delta x}_{\text{nonnegative}}\right)\\
  &+ \left(\underbrace{f(8)\Delta x}_{\text{negative}}+  \underbrace{f(10)\Delta x}_{\text{negative}}\right)
\end{align*}
 \end{example}

 When we take the limit of Riemann sums, it seems that we should get that
 
 \[
\lim_{n\to\infty}\sum_{k=1}^n f(x_k^*)\Delta x= A_1+(-A_2),
\]
where the areas $A_1$ and $A_2$ are depicted in the figure below. 

In other words, the limit of Riemann sums seems to be equal to the sum of signed areas of the regions that lie entirely above  or below  the $x$-axis.
The signed area of  regions that lie above the $x$-axis is positive, and the signed area of regions that lie below the $x$-axis is negative.

This sum of signed areas is called the net area of the region  between the graph of $f$ and the interval on the $x$- axis .
 \begin{image}
  \begin{tikzpicture}[
      declare function = {f(\x) = 6+\x/2 - pow(\x,2)/4;}]
    \begin{axis}[  
        domain=0:10, xmin =-1,xmax=10.5,ymax=10,ymin=-15,
        width=6in,
        height=3in,xtick={0,2,4,...,10},
        xticklabels={0,2,4,...,10},
        %% ytick style={draw=none},
        %% yticklabels={},
        axis lines=center, xlabel=$x$, ylabel=$y$,
        every axis y label/.style={at=(current axis.above origin),anchor=south},
        every axis x label/.style={at=(current axis.right of origin),anchor=west},
        axis on top,
      ]
       \addplot [draw=none,fill=fillp,domain=0:6, smooth] {f(x)} \closedcycle;
            \addplot [draw=none,fill=fill2,domain=6:10, smooth] {f(x)} \closedcycle;
          \addplot [ultra thick,penColor, smooth] {f(x)};
            \node at (axis cs:2,2) {\huge$A_1$};
              \node at (axis cs:9,-5) {\huge$A_2$};
                \node at (axis cs:5,6.2) {\large$y=f(x)$};
        \end{axis}  
  \end{tikzpicture}
\end{image}


The limit of Riemann sums will exist for any continuous functions on the interval $[a,b]$, even if $f$ assumes negative values on $[a,b]$.
The limit of Riemann sums  gives the  \textbf{net area} of the region between the graph of $f$ and an interval on the $x$- axis.

This leads to the following definition.
\begin{definition}
\index{integral}\index{definite integral}
Let $f$ be a function which is continuous on the interval $[a,b]$. We define the \textbf{definite integral} of $f$ on $[a,b]$ by
\[
\int_a^b f(x) dx=\lim_{n\to\infty}\sum_{k=1}^n f(x_k^*)\Delta x.
\]
\end{definition}
The definite integral is a number that gives the net area of the region between the curve $y=f(x)$ and the $x$-axis on the interval $[a,b]$.  
 
\begin{example} The graph a function $f$ on the interval $[0,9]$ is given in the figure. The areas of  four regions that lie either above or below the $x$-axis are labeled in the figure.

 \begin{image}
  \begin{tikzpicture}[
      declare function = {f(\x) = (1/4)*(x-1)*(x-5)* (x-8);}]
    \begin{axis}[  
        domain=0:9, xmin =-1,xmax=9.1,ymax=10,ymin=-10,
        width=6in,
        height=3in,xtick={0,3,6,9},
        xticklabels={0,3,6,9},
        %% ytick style={draw=none},
    %% yticklabels={},
        axis lines=center, xlabel=$x$, ylabel=$y$,
        every axis y label/.style={at=(current axis.above origin),anchor=south},
        every axis x label/.style={at=(current axis.right of origin),anchor=west},
        axis on top,
      ]
        \addplot [draw=none,fill=fill2,domain=0:1, smooth] {f(x)} \closedcycle;
       \addplot [draw=none,fill=fillp,domain=1:5, smooth] {f(x)} \closedcycle;
            \addplot [draw=none,fill=fill2,domain=5:8, smooth] {f(x)} \closedcycle;
              \addplot [draw=none,fill=fillp,domain=8:9, smooth] {f(x)} \closedcycle;
          \addplot [ultra thick,penColor, smooth] {f(x)};
            \node at (axis cs:0.5,-1.2) {\large$A_1$};
              \node at (axis cs:3,1.2) {\large$A_2$};
              \node at (axis cs:6.5,-1.2) {\large$A_3$};
              \node at (axis cs:8.5,1.2) {\large$A_4$};
                \node at (axis cs:6,6.2) {\large$y=f(x)$};
        \end{axis}  
  \end{tikzpicture}
\end{image}
Consider the integral
\[
\int_0^9 f(x) dx
\]
Express the integral in terms of areas $A_1$, $A_2$, $A_3$ and $A_4$.
\begin{explanation}
\begin{align*}
    \int_0^9 f(x) dx&= -A_1+A_2-A_3+A_4\\
\end{align*}

\end{explanation}
\end{example}





















\end{document}
