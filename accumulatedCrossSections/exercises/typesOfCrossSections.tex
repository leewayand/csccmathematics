\documentclass{ximera}


\graphicspath{
  {./}
  {ximeraTutorial/}
  {basicPhilosophy/}
}

\newcommand{\mooculus}{\textsf{\textbf{MOOC}\textnormal{\textsf{ULUS}}}}


\usepackage{tkz-euclide}\usepackage{tikz}
\usepackage{tikz-cd}
\usetikzlibrary{arrows}
\tikzset{>=stealth,commutative diagrams/.cd,
  arrow style=tikz,diagrams={>=stealth}} %% cool arrow head
\tikzset{shorten <>/.style={ shorten >=#1, shorten <=#1 } } %% allows shorter vectors

\usetikzlibrary{backgrounds} %% for boxes around graphs
\usetikzlibrary{shapes,positioning}  %% Clouds and stars
\usetikzlibrary{matrix} %% for matrix
\usepgfplotslibrary{polar} %% for polar plots
\usepgfplotslibrary{fillbetween} %% to shade area between curves in TikZ
\usetkzobj{all}
\usepackage[makeroom]{cancel} %% for strike outs
%\usepackage{mathtools} %% for pretty underbrace % Breaks Ximera
%\usepackage{multicol}
\usepackage{pgffor} %% required for integral for loops



%% http://tex.stackexchange.com/questions/66490/drawing-a-tikz-arc-specifying-the-center
%% Draws beach ball
\tikzset{pics/carc/.style args={#1:#2:#3}{code={\draw[pic actions] (#1:#3) arc(#1:#2:#3);}}}



\usepackage{array}
\setlength{\extrarowheight}{+.1cm}
\newdimen\digitwidth
\settowidth\digitwidth{9}
\def\divrule#1#2{
\noalign{\moveright#1\digitwidth
\vbox{\hrule width#2\digitwidth}}}
























%%This is to help with formatting on future title pages.
\newenvironment{sectionOutcomes}{}{}


\author{Jim Talamo}
\license{Creative Commons 3.0 By-NC}


\outcome{Set up an integral or sum of integrals that gives the volume of a solid with known cross sections.}
\outcome{Explore how types cross sections affects the volume.}

\begin{document}
\begin{exercise}


The region $R$ in the first quadrant is bounded by $y=\frac{12}{x}$, $x=1$ and $x=3$.  This exercise considers three solids whose base is $R$, but whose cross sections differ from each other. 





For Solid 1, cross sections through the solid taken parallel to the $y$-axis are squares.

For Solid 2, cross sections through the solid taken parallel to the $y$-axis are equilateral triangles.

For Solid 3, cross sections through the solid taken parallel to the $y$-axis are semicircles.

\begin{exercise}
Using geometric reasoning only (i.e. without performing any calculations):

Which solid has the greatest volume?
\begin{multipleChoice}
\choice[correct]{Solid 1}
\choice{Solid 2}
\choice{Solid 3}
\end{multipleChoice}

Which solid has the least volume?
\begin{multipleChoice}
\choice{Solid 1}
\choice{Solid 2}
\choice[correct]{Solid 3}
\end{multipleChoice}

\begin{hint}
How do the areas of a square of side length $s$, and semicircle with diameter $s$, and an equilateral triangle with side length $s$ compare?  How can this be used to compare the volumes of the three given solids?
\end{hint}

\end{exercise}

\begin{exercise}
To verify your observations quantitatively, calculate the volume of each solid.

The volume of Solid 1 is $\answer{96}$ cubic units.
	
The volume of Solid 2 is $\answer{24 \sqrt{3}}$ cubic units.

The volume of Solid 3 is $\answer{12 \pi }$ cubic units.

\end{exercise}


\end{exercise}

\end{document}