\documentclass{ximera}


\graphicspath{
  {./}
  {ximeraTutorial/}
  {basicPhilosophy/}
}

\newcommand{\mooculus}{\textsf{\textbf{MOOC}\textnormal{\textsf{ULUS}}}}


\usepackage{tkz-euclide}\usepackage{tikz}
\usepackage{tikz-cd}
\usetikzlibrary{arrows}
\tikzset{>=stealth,commutative diagrams/.cd,
  arrow style=tikz,diagrams={>=stealth}} %% cool arrow head
\tikzset{shorten <>/.style={ shorten >=#1, shorten <=#1 } } %% allows shorter vectors

\usetikzlibrary{backgrounds} %% for boxes around graphs
\usetikzlibrary{shapes,positioning}  %% Clouds and stars
\usetikzlibrary{matrix} %% for matrix
\usepgfplotslibrary{polar} %% for polar plots
\usepgfplotslibrary{fillbetween} %% to shade area between curves in TikZ
\usetkzobj{all}
\usepackage[makeroom]{cancel} %% for strike outs
%\usepackage{mathtools} %% for pretty underbrace % Breaks Ximera
%\usepackage{multicol}
\usepackage{pgffor} %% required for integral for loops



%% http://tex.stackexchange.com/questions/66490/drawing-a-tikz-arc-specifying-the-center
%% Draws beach ball
\tikzset{pics/carc/.style args={#1:#2:#3}{code={\draw[pic actions] (#1:#3) arc(#1:#2:#3);}}}



\usepackage{array}
\setlength{\extrarowheight}{+.1cm}
\newdimen\digitwidth
\settowidth\digitwidth{9}
\def\divrule#1#2{
\noalign{\moveright#1\digitwidth
\vbox{\hrule width#2\digitwidth}}}
























%%This is to help with formatting on future title pages.
\newenvironment{sectionOutcomes}{}{}


\title{What is a Function?}

\begin{document}

\begin{abstract}
one additional rule
\end{abstract}
\maketitle




%\section*{What are Functions?}

Functions are special relations. While a relation is just two sets of items and some pairings between the two sets, functions satisfy one rule. 


\begin{condition} \textbf{\textcolor{green!50!black}{THE Rule}} 

\begin{itemize}
\item Functions are those relations where each domain item is paired with \textbf{\textcolor{red!70!black}{exactly}} one codomain item, or
\item Functions are those relations where each domain item occurs in \textbf{\textcolor{red!70!black}{exactly}} one pair.
\end{itemize}
\end{condition}






\begin{definition} \textbf{\textcolor{green!50!black}{Function}}

A \textbf{function} is a package of three sets

\begin{itemize}
\item One set is called the \textbf{\textcolor{purple!85!blue}{domain}}. 
\item One set is called the \textbf{\textcolor{purple!85!blue}{codomain}}.  
\item Finally, there is a third set of pairings.  Each pairing associates a member of the domain with a member of the codomain. This third set does not seem to have an official title.
\end{itemize}

\textbf{\textcolor{red!70!darkgray}{$\blacktriangleright$}} such that each domain item is included in \textbf{\textcolor{red!70!black}{one and only one}} pair.

\end{definition}




Mathematicians have a funny way of saying \textit{yes} or \textit{no} about a relation possibly being a function. If a supposed function satisfies our one and only rule, then it is said to be a \textbf{well-defined} function.  Otherwise, it is not well-defined (meaning, it is not a function).




\begin{definition} \textbf{\textcolor{green!50!black}{Well-Defined}}

A relation is a function and is said to be \textbf{well-defined}, if it satisfies the one and only rule for a function.
\end{definition}








\pdfOnly{
\textbf{$\mathcal{VIDEO}$: Introduction to Functions}
}




\begin{onlineOnly}
\begin{explanation} \textbf{Video: Introduction to Functions}

[ Click on the arrow to the right to expand for the video. ]
\begin{expandable} 

\begin{center}
\youtube{_ZPK9TeG-SI}
\end{center}

\end{expandable}

\end{explanation}
\end{onlineOnly}





\begin{example} \textit{SSN} (Social Security Number)

The \textit{SSN} function has a domain consisting of all U.S. citizens and a codomain consisting of all 9-digit numbers.  \textit{SSN} pairs a U.S. citizen in the domain with the 9-digit number in the codomain that was issued to them as their social security number. 


\textbf{$\blacktriangleright$} This function is not well-defined, because there are U.S. citizens without a social security number, like many new-born babies.   \textit{SSN} is not a function.
\end{example}


\begin{example} \textit{ReverseSSN} 

The \textit{ReverseSSN} function has a domain consisting of all 9-digit numbers and a codomain consisting of all U.S. citizens.  \textit{ReverseSSN} pairs a number in the domain with the person in the codomain who was issued the number as their social security number. 


\textbf{$\blacktriangleright$} This function is not well-defined, because there are 9-digit numbers that have not been issued as social security numbers.  \textit{ReverseSSN} is not a function.
\end{example}






\begin{example} \textit{RationalAdd} 

The \textit{RationalAdd} function has a domain consisting of all rational numbers and a codomain consisting of all integers.  \textit{RationalAdd} pairs a rational number in the domain with an integer in the codomain according to the following procedure: 

\begin{quote}

\textbf{\textcolor{blue!55!black}{Since a chosen domain number is a rational number, it can be represented as $\tfrac{A}{B}$, where $A$ and $B$ are integers. This domain rational number is paired with $A+B$ in the codomain.}} 
\end{quote}
 
\textbf{$\blacktriangleright$} This function is not well-defined.  

For instance, there is a rational number that can be represented as $\tfrac{2}{3}$ and thus it would be paired with $2 + 3 = 5$.  However, this same rational number can also be represented as $\tfrac{4}{6}$ and thus would be paired with $4 + 6 = 10$.  This violates the one and only rule we have for functions.  


This particular rational number from the domain is paired with both $5$ and $10$. That isn't allowed for a function. Each domain item must be paired with exactly one codomain number. \textit{RationalAdd} is not well-defined and not a function.

\hfill\break 

\begin{warning}
When function values depend on how you write and represent numbers, that is a pretty good clue that the function may not be well-defined.
\end{warning}
\end{example}

\hfill\break 

\begin{example} \textit{SuperBowlWinner}  

The \textit{SuperBowlWinner} function has a domain consisting of the NFL Superbowls played and a codomain consisting of all of the NFL teams that have existed.  \textit{SuperBowlWinner} pairs a Superbowl in the domain with the NFL team in the codomain that won that Superbowl.  


\textbf{$\blacktriangleright$} This function is well-defined. Each Superbowl has exactly one winner.
\end{example}



Whew!  We finally found a well-defined function. That one little rule actually narrows the pool of relations quite a bit.  By focusing our investigation on only functions, this rule will help us study relationships between all kinds of measurements (our main goal). 












\subsection*{Function Notation}

Now that we are only investigating functions, rather than all relations, we discover some opportunities to help our communication.  For instance, it turns out that when you select a domain item in a function, you have automatically selected a codomain item.

\textbf{Our Rule:} Each domain item is paired with \textbf{\textcolor{purple!85!blue}{EXACTLY}} one codomain item.  Not 0. Not 2.  Not 3.  \textbf{\textcolor{purple!85!blue}{EXACTLY 1!}}


Thus, when you pick a domain item, you have automatically selected its partner in the codomain.  We could certainly hunt down this partner inside the codomain.  But, we can also just talk about 






\begin{center}
``\textit{\textbf{\textcolor{blue!55!black}{the domain member's partner in the codomain}} }.'' 
\end{center}



We have notation for this thought. 


\newpage


\begin{notation} \textbf{\textcolor{red!50!green}{Function Notation}}  


Let $functionName$ be the name of a function. \\
Let $d$ represent a domain item of $functionName$.


Then, ``\textbf{\textcolor{blue!55!black}{$functionName(d)$}}'' represents $d$'s partner in the codomain. 


\textbf{\textcolor{red!70!black}{Language:}} \\
$functionName(d)$ is \textbf{\textcolor{blue!55!black}{pronounced}} as ``functionName \textbf{\textcolor{purple!85!blue}{OF}} $d$''.


$(d, functionName(d))$ is a pair in $functionName$. 


$functionName(d)$ is called ``the \textbf{\textcolor{purple!85!blue}{VALUE}} of functionName \textbf{\textcolor{purple!85!blue}{AT}} $d$''. \\
$functionName(d)$ is called ``the \textbf{\textcolor{purple!85!blue}{IMAGE}} of $d$ under functionName''. 



\end{notation}







\begin{example} \textit{SuperBowlWinner}


\textit{SuperBowlWinner}(Superbowl 13) represents the winning team from Superbowl 13, which is the Pittsburgh Steelers.  The following equality is communicating this with an equation.

\begin{itemize}
\item \textit{SuperBowlWinner}(Superbowl 13) = Pittsburgh Steelers  
\end{itemize}



The pair (Superbowl 13, \textit{SuperBowlWinner}(Superbowl 13)) is the pair (Superbowl 13, Pittsburgh Steelers). \

\end{example}

\textbf{Note:} The one and only rule for functions doesn't refer to the codomain.

Not every NFL team has won a Superbowl. As of 2025, the Cleveland Browns had not won a Superbowl.  So, there are codomain items that are not partnered with a domain item.  This is not true of the domain.  Every domain item is paired with a codomain item.  That is our one and only function rule.  Every domain item appears in exactly one pair.  But not every codomain item must appear in a pair.  This is significant and deserves some language.


\begin{definition} \textbf{\textcolor{green!50!black}{Range}} \\

The \textbf{range} of a function is the collection of items in the codomain which are paired with some item in the domain.  The range is the collection of function values.
\end{definition}

The range and codomain are different sets.  The range and codomain could be equal sets or the range could be a proper subset of the codomain. \\

\[  Range \subseteq Codomain\]

\begin{itemize}
\item The Cleveland Browns are not in the range of \textit{SuperBowlWinner}.  
\item The Pittsburgh Steelers are in the range of \textit{SuperBowlWinner}.
\end{itemize}











\begin{notation}  \textbf{\textcolor{red!50!green}{Symbols for Sets}}  

\begin{itemize}
\item $\subseteq$ is the symbol for \textbf{is a subset of}.
\item $\subset$ is the symbol for \textbf{is a proper subset of}.
\item $\in$ is the symbol for \textbf{is a member of}.
\item $\mathbb{R}$ is the symbol for \textbf{the real numbers}.
\item $\mathbb{Q}$ is the symbol for \textbf{the rational numbers}.
\item $\mathbb{Z}$ is the symbol for \textbf{the integers}.
\item $\mathbb{N}$ is the symbol for \textbf{the natural numbers}.
\end{itemize}


For example, $\pi \in \mathbb{R}$, $\tfrac{2}{3} \in \mathbb{Q}$, $-3 \in \mathbb{Z}$, $5 \in \mathbb{N}$. \\

For Example,  $\mathbb{N} \subseteq \mathbb{Z} \subseteq \mathbb{Q} \subseteq \mathbb{R}$ \\

For Example,  $\mathbb{N} \subset \mathbb{Z} \subset \mathbb{Q} \subset \mathbb{R}$


\end{notation}


\hfill\break
\hfill\break


\begin{example} Membership


Let $D = \{ 3, 5, 7, 8, 10, 11, 13, 15, 16 \}$  \\
Let $T = \{ 3, 10, 11, 16 \}$  \\
Let $R = \{ 3, 6, 10, 11, 16 \}$   \\
Let $W = \{ 3, 8, 10 \}$   \\


\[   3 \in D  \]
\[   3 \in T  \]
\[   3 \in R  \]
\[   3 \in W  \]
\[   T \subset D \]
\[   T \subseteq D \]
\[   D \subseteq D \]
\[   W \nsubseteq R \]
\[   8 \notin R  \]


\end{example}





\newpage


\begin{question}


Let $D = \{ 3, 5, 7, 8, 10, 11, 13, 15, 16 \}$  \\
Let $T = \{ 3, 10, 11, 16 \}$  \\
Let $R = \{ 3, 6, 10, 11, 16 \}$   \\
Let $W = \{ 3, 8, 10 \}$   \\


Select all of the true statements.

\begin{selectAll}
	\choice [correct]{$6 \in R$}
	\choice [correct]{$10 \in T$}
	\choice {$15 \in W$}	
	\choice {$13 \in R$}
	\choice [correct] {$6 \notin D$}
	\choice [correct]{$\{ 3, 6 \} \subset R$}
	\choice {$\{ 3, 6 \} \subset D$}
	\choice [correct] {$W \subset D$}
\end{selectAll}


\end{question}



\begin{example} \textit{Starring}

Let's name the following function \textit{Starring}.

domain = \{ Casablanca, House of Wax,  The Godfather, Lawrence of Arabia, Toy Story, The Fly \}

codomain = \{ Marlon Brando, Vincent Price, Humphrey Bogart, Peter O'Toole, Harrison Ford, Al Pacino \}

pairs = \{ (The Godfather, Marlon Brando), (House of Wax, Vincent Price), (Casablanca, Humphrey Bogart), (Lawrence of Arabia, Peter O'Toole), (The Fly, Vincent Price) \} 





\begin{question}

Is \textit{Starring} well-defined?

\begin{multipleChoice}
	\choice{Yes}
	\choice[correct]{No}
\end{multipleChoice}
\begin{feedback}
Remember that every domain element must be used in exactly one ordered pair. Toy Story is in the domain and not in any pair.
\end{feedback}

\end{question}



\end{example}














\begin{example} \textit{StarringAgain}

Let's name the following function \textit{StarringAgain}.

domain = \{ Casablanca, House of Wax,  The Godfather, Lawrence of Arabia, The Fly \}

codomain = \{ Marlon Brando, Vincent Price, Humphrey Bogart, Peter O'Toole, Harrison Ford, Al Pacino \}

pairs = \{ (The Godfather, Marlon Brando), (House of Wax, Vincent Price), (Casablanca, Humphrey Bogart), (Lawrence of Arabia, Peter O'Toole), (The Fly, Vincent Price) \} 





\begin{question}

Is \textit{StarringAgain} well-defined?

\begin{multipleChoice}
	\choice[correct]{Yes}
	\choice{No}
\end{multipleChoice}
\begin{feedback}
Every domain member is in exactly one pair.
\end{feedback}

\end{question}







\begin{question}

Which codomain items are not in the range of \textit{StarringAgain}?

\begin{selectAll}
	\choice{Marlon Brando}
	\choice{Vincent Price}
	\choice{Humphrey Bogart}
	\choice{Peter O'Toole}
	\choice[correct]{Harrison Ford}
	\choice[correct]{Al Pacino}
\end{selectAll}

\end{question}





\end{example}







We can extend our idea of value and image from single items to sets and also think of a function in reverse. 






\begin{definition} \textbf{\textcolor{green!50!black}{Image}} and \textbf{\textcolor{green!50!black}{Preimage}} 


Let $f$ be a function with domain $D$, range $R$, and codomain $C$. \\
Let $S$ be a subset of the domain $D$, $S \subseteq D$ . \\
Let $T$ be a subset of the codomain $C$, $T \subseteq C$ . \\


The \textbf{image} of $S$ under $f$ is the collection of function values of all members of $S$.

\[
f(S) = \{ f(s) \, | \, s \in S   \} \subseteq R \subseteq C
\]




The \textbf{preimage} of $T$ under $f$ is the collection of domain members whose function values are inside $T$.

\[
f^{-1}(T) = \{ d \in D \, | \, f(d) \in T  \} \subseteq D
\]


The preimage is also known as the \textbf{inverse image}.

$f^{-1}(T)$ is the symbol for the reverse of $f$. 

$f^{-1}(T)$ pairs subsets of the codomain with subsets of the domain.  It is just reading the function $f$ backwards or in reverse.



\end{definition}





\newpage



\begin{example} Reverse


Define the function $H$ by these sets: 


\begin{itemize}
	\item  $D = \{ 1, 2, 3, 4, 5, 6, 7, 8 \}$
	\item  $R = \{ A, B, C, D \}$
	\item  $pairs = \{ (1, A), (2, B), (3, A), (3, C), (4, C), (5, A), (6, B), (7, A), (8, C)    \}$
\end{itemize}


\[  H(\{ 1, 2 \}) = \{ A, B \}  \]

\[  H^{-1}(\{ A, B \}) = \{ 1, 2, 3, 5, 6, 7 \}  \]

\[  H^{-1}(\{ C \}) = \{ 3, 4, 8 \}  \]

\[  H^{-1}(\{ D \}) = \emptyset  \]

\[  H^{-1}(\{ C, D \}) = \{ 3, 4, 8 \}  \]

\[  H(\{ 1, 3, 5 \}) = \{ A \}  \]


\end{example}








\begin{notation} \textbf{\textcolor{red!50!green}{Reverse}}  

Mathematics is full of concepts and skills which have a feeling that you are applying something.

Take multiplication, for example.  We can think of multiplication by $3$ as ``doing'' something.  This thought automatically comes with the reverse thought.  We might think of reversing multiplication by $3$ as multiplying by $\frac{1}{3}$ and we have alternative exponential notation available, $3^{-1} = \frac{1}{3}$. 


With the addition operation, we think of the opposite of $4$ as $-4 = -1 \cdot 4$. 


Mathematics has many ideas that involve reverse, opposite, backwards, undo, inverse, back up, turn around, etc. In one way or another, $-1$ seems to be included in the notation for this reverse action.  


Functions have a feeling of beginning with a domain number and pairing it with a range number.  And, naturally, we have the reverse idea. 

Our symbol for the reverse of the function $f$, is $f^{-1}$.  It begins with range items and pairs them with their domain partners.  This reversing of function pairs is certainly a relation, because, well, everything is a relation.  Whether or not this reverse action results in a function is a topic for later in the course.



\end{notation}



\begin{warning} \textbf{\textcolor{red!70!black}{-1}}


The $-1$ superscript has different meanings depending on the context. 


Sometimes $f^{-1}(x)$ and $f^{-1}(x)$ do not mean the same thing.


\begin{itemize}
\item $f^{-1}(x)$ might mean the reciprocal, $\frac{1}{f(x)}$.  (Like, $3^{-1} = \frac{1}{3}$.)
\item $f^{-1}(x)$ might mean the inverse function (switch domain and codomain)
\end{itemize}


Context is everything! 

We will also encounter shorthand notation for all kinds of mathematics. 

Mathematics is a language that people use to communicate.  We need to keep this in mind.

\end{warning}




Lastly, we should establish when two functions are equal. 








\begin{definition} \textbf{\textcolor{green!50!black}{Equality}}

Two functions are \textbf{equal} provided they satisfy all three of the following:

\begin{itemize}
\item they have the same domain.
\item they have the same range.
\item they have the same pairs.
\end{itemize}

\end{definition}
Equality uses the range instead of the codomain, because we are mostly focused on the pairs.  However, there are situations where the codomain is important and returns to the story.  We'll point these situations out when we encounter them.













\begin{onlineOnly}
\begin{center}
\textbf{\textcolor{green!50!black}{ooooo-=-=-=-ooOoo-=-=-=-ooooo}} \\

more examples can be found by following this link\\ \link[More Examples of Functions]{https://ximera.osu.edu/csccmathematics/precalculus/precalculus/functions/examples/exampleList}

\end{center}
\end{onlineOnly}




\end{document}
