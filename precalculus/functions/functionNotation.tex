\documentclass{ximera}


\graphicspath{
  {./}
  {ximeraTutorial/}
  {basicPhilosophy/}
}

\newcommand{\mooculus}{\textsf{\textbf{MOOC}\textnormal{\textsf{ULUS}}}}


\usepackage{tkz-euclide}\usepackage{tikz}
\usepackage{tikz-cd}
\usetikzlibrary{arrows}
\tikzset{>=stealth,commutative diagrams/.cd,
  arrow style=tikz,diagrams={>=stealth}} %% cool arrow head
\tikzset{shorten <>/.style={ shorten >=#1, shorten <=#1 } } %% allows shorter vectors

\usetikzlibrary{backgrounds} %% for boxes around graphs
\usetikzlibrary{shapes,positioning}  %% Clouds and stars
\usetikzlibrary{matrix} %% for matrix
\usepgfplotslibrary{polar} %% for polar plots
\usepgfplotslibrary{fillbetween} %% to shade area between curves in TikZ
\usetkzobj{all}
\usepackage[makeroom]{cancel} %% for strike outs
%\usepackage{mathtools} %% for pretty underbrace % Breaks Ximera
%\usepackage{multicol}
\usepackage{pgffor} %% required for integral for loops



%% http://tex.stackexchange.com/questions/66490/drawing-a-tikz-arc-specifying-the-center
%% Draws beach ball
\tikzset{pics/carc/.style args={#1:#2:#3}{code={\draw[pic actions] (#1:#3) arc(#1:#2:#3);}}}



\usepackage{array}
\setlength{\extrarowheight}{+.1cm}
\newdimen\digitwidth
\settowidth\digitwidth{9}
\def\divrule#1#2{
\noalign{\moveright#1\digitwidth
\vbox{\hrule width#2\digitwidth}}}
























%%This is to help with formatting on future title pages.
\newenvironment{sectionOutcomes}{}{}


\title{Reading and Writing}

\begin{document}

\begin{abstract}
language
\end{abstract}
\maketitle





\begin{warning} \textbf{\textcolor{red!70!black}{Function Notation}}  \\



Function notation can look a heck of a lot like the distributive property. \\


\begin{center}

\textbf{\textcolor{purple!85!blue}{Context is Everything}}

\end{center}


Mathematics is a language.  We use it to communicate.  Mathematics has symbols that we arrange to communicate information, just like letters in the alphabet. \\

And, just like letters forming words, we over use them.  We use the same arrangment of letters to mean different ideas. \\

\textbf{Word Example:} bat \\

Is a bat flying around in a cave or are your trying to hit a baseball with a bat? \\


Same in mathematics.\\



\textbf{Mathematics Example:} $f(a + b)$ \\

\begin{itemize}
	\item $f(a + b) = f(a) + f(b)$
	\item $f(a + b) \ne f(a) + f(b)$
\end{itemize}


It depends on what $f$ is representing. \\

If $f$ is the name of a function, then $f(a + b) \ne f(a) + f(b)$.  On the other hand, if $f$ is the name of a variable, then this is just multiplication and we can distribute.



\end{warning}







This issue is not going away. \\



You are learning a new language.  You are learning a new alphabet.  You are learning new words. \\

You are learning to communicate with this new language about new concepts and ideas and skills.  \\


You are learning how to read. \\


It is going to take a while.  But, you can't even begin until you understand that you are reading and that the context affects the meaning of the words, just like in any language.










\begin{onlineOnly}
\begin{center}
\textbf{\textcolor{green!50!black}{ooooo-=-=-=-ooOoo-=-=-=-ooooo}} \\

more examples can be found by following this link\\ \link[More Examples of Functions]{https://ximera.osu.edu/csccmathematics/precalculus/precalculus/functions/examples/exampleList}

\end{center}
\end{onlineOnly}




\end{document}
