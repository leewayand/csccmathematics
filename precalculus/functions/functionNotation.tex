\documentclass{ximera}


\graphicspath{
  {./}
  {ximeraTutorial/}
  {basicPhilosophy/}
}

\newcommand{\mooculus}{\textsf{\textbf{MOOC}\textnormal{\textsf{ULUS}}}}


\usepackage{tkz-euclide}\usepackage{tikz}
\usepackage{tikz-cd}
\usetikzlibrary{arrows}
\tikzset{>=stealth,commutative diagrams/.cd,
  arrow style=tikz,diagrams={>=stealth}} %% cool arrow head
\tikzset{shorten <>/.style={ shorten >=#1, shorten <=#1 } } %% allows shorter vectors

\usetikzlibrary{backgrounds} %% for boxes around graphs
\usetikzlibrary{shapes,positioning}  %% Clouds and stars
\usetikzlibrary{matrix} %% for matrix
\usepgfplotslibrary{polar} %% for polar plots
\usepgfplotslibrary{fillbetween} %% to shade area between curves in TikZ
\usetkzobj{all}
\usepackage[makeroom]{cancel} %% for strike outs
%\usepackage{mathtools} %% for pretty underbrace % Breaks Ximera
%\usepackage{multicol}
\usepackage{pgffor} %% required for integral for loops



%% http://tex.stackexchange.com/questions/66490/drawing-a-tikz-arc-specifying-the-center
%% Draws beach ball
\tikzset{pics/carc/.style args={#1:#2:#3}{code={\draw[pic actions] (#1:#3) arc(#1:#2:#3);}}}



\usepackage{array}
\setlength{\extrarowheight}{+.1cm}
\newdimen\digitwidth
\settowidth\digitwidth{9}
\def\divrule#1#2{
\noalign{\moveright#1\digitwidth
\vbox{\hrule width#2\digitwidth}}}
























%%This is to help with formatting on future title pages.
\newenvironment{sectionOutcomes}{}{}


\title{Reading and Writing}

\begin{document}

\begin{abstract}
language
\end{abstract}
\maketitle


\begin{itemize}
\item Science is the search and identification of structure. \\

\item Mathematics is a language. We use it to describe and communicate our discoveries.
\end{itemize}

As a language, mathematics has all of the trappings, idioms, shorthand, and homographs as any other language. In Precalculus and Calculus, function notation requires our fullest attention.  \\


\begin{warning} \textbf{\textcolor{red!70!black}{Function Notation}}  \\



Function notation can look a heck of a lot like the distributive property. \\


\begin{center}

\textbf{\textcolor{purple!85!blue}{Context is Everything}}

\end{center}


Mathematics is a language.  We use it to communicate.  Mathematics has symbols that we arrange to communicate information, just like letters in the alphabet. \\

And, just like letters forming words, we over use these symbols in mathematics.  We use the same arrangment of symbols to mean different ideas. \\

\textbf{Word Example:} bat \\

Is a bat flying around in a cave or are your trying to hit a baseball with a bat? \\


Same in mathematics.\\



\textbf{Mathematics Example:} $f(a + b)$ \\

\begin{itemize}
	\item $f(a + b) = f(a) + f(b)$
	\item $f(a + b) \ne f(a) + f(b)$
\end{itemize}


It depends on what $f$ is representing. \\

If $f$ is the name of a function, then $f(a + b) \ne f(a) + f(b)$.  On the other hand, if $f$ is the name of a variable, then this is just multiplication and we can distribute.



\end{warning}







This issue is not going away. \\


We are overly familiar with functions named ``$f$'' and ``$g$''.  It is going to get more confusing with expressions like $a(b+c)$.




You are learning a new language.  You are learning a new alphabet.  You are learning new words. \\

You are learning to communicate with this new language about new concepts and ideas and skills.  \\


You are learning how to read. \\


It is going to take a while.  But, you can't even begin until you understand that you are reading and writing, and that the context affects the meaning of the words, just like in any language.



In Precalculus and Calculus, $f(x)$ or $g(t)$ or $h(y)$ or $a(b)$ is going to be function notation.  That is the default expectation you should have when reading mathematics. However, not always.


Even if we are expecting function notation to be the default interpretation, we still need to be aware of what symbols are actively being used as functions, variables, parameters, and constants.


\subsection*{Scope}

In computer programming, there is an idea called \textbf{scope}. Scope dictates how symbols are to be interpreted at any point in the program. Mathematics has the same idea.

When writing mathematics, the same symbols will be used at different places and mean different things.  Probably, this is most often seen when numbering homework or test questions.  Once a homework question has concluded and the next question begins, all definitions from the previous question are lost. 

\begin{example} Scope



\begin{model} Question 1

Let $f(x) = 7x - 5$  

Let $g = 4$

	\begin{model} Part (a)

	Let $y(t) = 9 - 5t$

	Let $k = 6$
		\begin{enumerate}
			\item $f(1) = 2$
			\item $g(3) = g \cdot 3 = 4 \cdot 3 = 12$
			\item $h(6) = h \cdot 6 = 6h$
			\item $w g = w \cdot 4 = 4w$
		\end{enumerate}
	\end{model} 


	\begin{model} Part (b)

	Let $h(x) = 3x + 6$

	Let $w = 7$.

		\begin{enumerate}
			\item $f(3) = 17$
			\item $h(g) = h(4) = 18$
			\item $y(w) = y \cdot 7 = 7y$
		\end{enumerate}
	\end{model} 

\end{model}



\end{example}




Scope tells us not to automatically assume a particular symbol ALWAYS represents the same thing. \\





\textbf{\textcolor{blue!55!black}{$\blacktriangleright$ Variables}}

We should not be using the same symbol in the same scope to represent more than one thing.

For instance: defining $f(x) = \sqrt{7-x}$ and $g(x) = \ln(x)$ is not a good idea.

It is unclear if $x$ can equal $-4$ or not.




\textbf{\textcolor{blue!55!black}{$\blacktriangleright$ Fraction Bars}}

This idea spills over into all of our notation.

Using slanted fraction bars is just miscommunication.




\begin{example} Fraction Bars

The order of operations tells us that $8+20/5+2$ means $8 + \frac{20}{5} + 2 = 14$.

However, all too often students mean $\frac{8+20}{5+2} $.

It is confusing.

\end{example}



One of the main learning objectives of this course is \textbf{\textcolor{blue!55!black}{rigor}}.


We want to use proper and precise mathematics to communicate our reasoning.

Why? So, that we can communicate!  Learning is going to be difficult if people don't say things correctly.

How would you have learned how to read, if people wrote words with the wrong letters in the wrong order and just said ``well, you know what I meant''?






























\begin{onlineOnly}
\begin{center}
\textbf{\textcolor{green!50!black}{ooooo-=-=-=-ooOoo-=-=-=-ooooo}} \\

more examples can be found by following this link\\ \link[More Examples of Functions]{https://ximera.osu.edu/csccmathematics/precalculus/precalculus/functions/examples/exampleList}

\end{center}
\end{onlineOnly}




\end{document}
