\documentclass{ximera}


\graphicspath{
  {./}
  {ximeraTutorial/}
  {basicPhilosophy/}
}

\newcommand{\mooculus}{\textsf{\textbf{MOOC}\textnormal{\textsf{ULUS}}}}


\usepackage{tkz-euclide}\usepackage{tikz}
\usepackage{tikz-cd}
\usetikzlibrary{arrows}
\tikzset{>=stealth,commutative diagrams/.cd,
  arrow style=tikz,diagrams={>=stealth}} %% cool arrow head
\tikzset{shorten <>/.style={ shorten >=#1, shorten <=#1 } } %% allows shorter vectors

\usetikzlibrary{backgrounds} %% for boxes around graphs
\usetikzlibrary{shapes,positioning}  %% Clouds and stars
\usetikzlibrary{matrix} %% for matrix
\usepgfplotslibrary{polar} %% for polar plots
\usepgfplotslibrary{fillbetween} %% to shade area between curves in TikZ
\usetkzobj{all}
\usepackage[makeroom]{cancel} %% for strike outs
%\usepackage{mathtools} %% for pretty underbrace % Breaks Ximera
%\usepackage{multicol}
\usepackage{pgffor} %% required for integral for loops



%% http://tex.stackexchange.com/questions/66490/drawing-a-tikz-arc-specifying-the-center
%% Draws beach ball
\tikzset{pics/carc/.style args={#1:#2:#3}{code={\draw[pic actions] (#1:#3) arc(#1:#2:#3);}}}



\usepackage{array}
\setlength{\extrarowheight}{+.1cm}
\newdimen\digitwidth
\settowidth\digitwidth{9}
\def\divrule#1#2{
\noalign{\moveright#1\digitwidth
\vbox{\hrule width#2\digitwidth}}}
























%%This is to help with formatting on future title pages.
\newenvironment{sectionOutcomes}{}{}


\title{Polynomial and Rational}

\begin{document}

\begin{abstract}
%Stuff can go here later if we want!
\end{abstract}
\maketitle




The \textbf{Elementary Functions} is a collection of functions for which we know a lot about their algebraic properties.

They include 

\begin{itemize}
\item constant, linear, quadratic, cubic, and other polynomial functions
\item rational functions
\item root or radical functions
\item exponential and logarithmic functions
\item trigonometric functions
\item absolute value, greatest integer, other individual piecewise defined functions
\end{itemize}








Our goal in this course is to map out our own plan for analyzing functions.  Our starting point is the elementary functions. However, we also want to analyze new functions made from combinations of the elementary functions.  We will add, subtract, and multiply, them; create quotients of them; glue pieces of them together; repeat them; and other concoctions we will think up.

We want exact analysis.  We want to know exact information.  We want exact values and exact descriptions where these occur in the domain. Our overall plan includes a lot of algebra to help us track down exact values.  


This plan will continue through Calculus.  We will continue to develop algebraic tools and reasoning to make more exact values possible.

In this section, we focus on polynomial functions and rational functions.  How much can we describe exactly? How much do we need to approximate? How does Calculus help our analysis?


What do we want to know when we analyze functions?



We want to know the 
\begin{itemize}
     \item \textbf{\textcolor{red!80!black}{Domain}} 
     \item \textbf{\textcolor{red!80!black}{Zeros}} 
     \item \textbf{\textcolor{red!80!black}{Continuity}} 
\begin{itemize}
     \item \textbf{\textcolor{purple!85!blue}{discontinuities}} 
     \item \textbf{\textcolor{purple!85!blue}{singularities}} 
\end{itemize}
     \item \textbf{\textcolor{red!80!black}{End-Behavior}} 
     \item \textbf{\textcolor{red!80!black}{Behavior}} 
\begin{itemize}
     \item \textbf{\textcolor{purple!85!blue}{intervals where increasing}} 
     \item \textbf{\textcolor{purple!85!blue}{intervals where decreasing}} 
\end{itemize}
     \item \textbf{\textcolor{red!80!black}{Global Maximum and Minimum}} 
     \item \textbf{\textcolor{red!80!black}{Local Maximums and Minimums}} 
     \item \textbf{\textcolor{red!80!black}{Range}} 
     \item \textbf{\textcolor{blue!55!black}{...and we would like a nice graph}} 
\end{itemize}










\subsection*{Learning Outcomes}

\begin{sectionOutcomes}
After completing this section, students should 

\begin{itemize}
\item have a plan for analyzing polynomial and rational functions.
\end{itemize}
\end{sectionOutcomes}





\begin{center}
\textbf{\textcolor{green!50!black}{ooooo-=-=-=-ooOoo-=-=-=-ooooo}} \\

more examples can be found by following this link\\ \link[More Examples of Poynomial and Rational functions]{https://ximera.osu.edu/csccmathematics/precalculus/precalculus/polynomialFunctions/examples/exampleList}

\end{center}




\end{document}

