\documentclass{ximera}


\graphicspath{
  {./}
  {ximeraTutorial/}
  {basicPhilosophy/}
}

\newcommand{\mooculus}{\textsf{\textbf{MOOC}\textnormal{\textsf{ULUS}}}}


\usepackage{tkz-euclide}\usepackage{tikz}
\usepackage{tikz-cd}
\usetikzlibrary{arrows}
\tikzset{>=stealth,commutative diagrams/.cd,
  arrow style=tikz,diagrams={>=stealth}} %% cool arrow head
\tikzset{shorten <>/.style={ shorten >=#1, shorten <=#1 } } %% allows shorter vectors

\usetikzlibrary{backgrounds} %% for boxes around graphs
\usetikzlibrary{shapes,positioning}  %% Clouds and stars
\usetikzlibrary{matrix} %% for matrix
\usepgfplotslibrary{polar} %% for polar plots
\usepgfplotslibrary{fillbetween} %% to shade area between curves in TikZ
\usetkzobj{all}
\usepackage[makeroom]{cancel} %% for strike outs
%\usepackage{mathtools} %% for pretty underbrace % Breaks Ximera
%\usepackage{multicol}
\usepackage{pgffor} %% required for integral for loops



%% http://tex.stackexchange.com/questions/66490/drawing-a-tikz-arc-specifying-the-center
%% Draws beach ball
\tikzset{pics/carc/.style args={#1:#2:#3}{code={\draw[pic actions] (#1:#3) arc(#1:#2:#3);}}}



\usepackage{array}
\setlength{\extrarowheight}{+.1cm}
\newdimen\digitwidth
\settowidth\digitwidth{9}
\def\divrule#1#2{
\noalign{\moveright#1\digitwidth
\vbox{\hrule width#2\digitwidth}}}
























%%This is to help with formatting on future title pages.
\newenvironment{sectionOutcomes}{}{}


\title{Rational Functions}

\begin{document}

\begin{abstract}
fractions of polynomials
\end{abstract}
\maketitle








Rational functions are fractions of polynomials


\begin{definition}\textbf{\textcolor{green!50!black}{Rational Functions}}

A rational function is any function that can be represented as a quotient of two polynomials 

\[   \frac{ a_n x^n + a_{n-1} x^{n-1} + \cdots + a_3 x^3 + a_2 x^2 + a_1 x + a_0  } { b_m x^m + b_{m-1} x^{m-1} + \cdots + b_3 x^3 + b_2 x^2 + b_1 x + b_0 }   \]



where the $a_k$ and $b_k$ are real numbers and $a_n \ne 0$ and $b_m \ne 0$.





\end{definition}




Again, we prefer polynomials in factored form.  Therefore, usually, our first step is to transform the formula for the rational function to look like



\[   \frac{ (x-r_n)(x-r_{n-1})  \cdots (x-r_2)(x-r_1)  } { (x-s_m)(x-s_{m-1})  \cdots (x-s_2)(x-s_1) }   \]




And, again, we will be able to obtain a product of exclusively linear factors with the addition of complex numbers.  With real numbers we can get these products to consist of a mixture of linear and irreducible quadratics.










\begin{example}


The graph of $y = H(w) = \frac{(w-1)}{(w+3) (w-4)} $


\begin{image}
\begin{tikzpicture} 
  \begin{axis}[
            domain=-10:10, ymax=10, xmax=10, ymin=-10, xmin=-10,
            axis lines =center, xlabel=$w$, ylabel=$y$,
            ytick={-10,-8,-6,-4,-2,2,4,6,8,10},
            xtick={-10,-8,-6,-4,-2,2,4,6,8,10},
            ticklabel style={font=\scriptsize},
            every axis y label/.style={at=(current axis.above origin),anchor=south},
            every axis x label/.style={at=(current axis.right of origin),anchor=west},
            axis on top
          ]

          \addplot [line width=1, gray, dashed, domain=(-9.5:9.5),<->] ({-3},{x});
          \addplot [line width=1, gray, dashed, domain=(-9.5:9.5),<->] ({4},{x});

          \addplot [line width=2, penColor, smooth, domain=(-9:-3.1),<->] {(x-1)/((x+3)*(x-4))};
          \addplot [line width=2, penColor, smooth, domain=(-2.9:3.9),<->] {(x-1)/((x+3)*(x-4))};
          \addplot [line width=2, penColor, smooth, domain=(4.1:9),<->] {(x-1)/((x+3)*(x-4))};


  \end{axis}
\end{tikzpicture}
\end{image}

\end{example}








\begin{example}


The graph of $y = v(r) = \frac{(r-1)}{(r+3) (r-4)^2} $


\begin{image}
\begin{tikzpicture} 
  \begin{axis}[
            domain=-10:10, ymax=10, xmax=10, ymin=-10, xmin=-10,
            axis lines =center, xlabel=$r$, ylabel=$y$,
            ytick={-10,-8,-6,-4,-2,2,4,6,8,10},
            xtick={-10,-8,-6,-4,-2,2,4,6,8,10},
            ticklabel style={font=\scriptsize},
            every axis y label/.style={at=(current axis.above origin),anchor=south},
            every axis x label/.style={at=(current axis.right of origin),anchor=west},
            axis on top
          ]

          \addplot [line width=1, gray, dashed, domain=(-9.5:9.5),<->] ({-3},{x});
          \addplot [line width=1, gray, dashed, domain=(-9.5:9.5),<->] ({4},{x});

          \addplot [line width=2, penColor, smooth, samples=200, domain=(-9:-3.02),<->] {(x-1)/((x+3)*(x-4)^2)};
          \addplot [line width=2, penColor, smooth, samples=200, domain=(-2.98:3.77),<->] {(x-1)/((x+3)*(x-4)^2)};
          \addplot [line width=2, penColor, smooth, samples=200, domain=(4.25:9),<->] {(x-1)/((x+3)*(x-4)^2)};


  \end{axis}
\end{tikzpicture}
\end{image}

\end{example}








While polynomials only combined factors with positive integer powers, rational functions include factors with negative integer powers.


We can pull $H(w)$ apart to see these power-like functions.


\[    H(w) = \frac{(w-1)}{(w+3) (w-4)}  = \frac{4}{7} \cdot \frac{1}{w+3}  + \frac{3}{7} \cdot \frac{1}{w-4} =  \frac{4}{7} (w+3)^{-1}  + \frac{3}{7} (w-4)^{-1} \]


Strictly speaking, these are not power functions, but more like shifted power functions - in the denominator.









\subsection*{Continuity}


Rational functions are continuous everywhere on their domain.  They have singularities at zeros of their denominators.  

Graphs of rational functions might have vertical asymptotes at these singularities, where the function grows (positively or negatively) without bound.  As we can see in the examples above, the sign of the function can switch across a singularity, or not.  This has to do with the power of the corresponding factor, which we will investigate later.













\begin{example}


The graph of $y = B(c) = \frac{(c-1)(c-5)}{5(c+3)} $


\begin{image}
\begin{tikzpicture} 
  \begin{axis}[
            domain=-20:20, ymax=20, xmax=20, ymin=-20, xmin=-20,
            axis lines =center, xlabel=$c$, ylabel=$y$,
            ytick={-20,-16,-12,-8,-4,4,8,12,16,20},
            xtick={-20,-16,-12,-8,-4,4,8,12,16,20},
            ticklabel style={font=\scriptsize},
            every axis y label/.style={at=(current axis.above origin),anchor=south},
            every axis x label/.style={at=(current axis.right of origin),anchor=west},
            axis on top
          ]


          \addplot [line width=1, gray, dashed, domain=(-19.5:19.5),<->] ({-3},{x});
          \addplot [line width=1, gray, dashed, domain=(-19.5:19.5),<->] {0.2*x-1.8};


          \addplot [line width=2, penColor, smooth, samples=200, domain=(-19:-3.4),<->] {((x-1)*(x-5))/(5*(x+3))};
          \addplot [line width=2, penColor, smooth, samples=200, domain=(-2.9:19),<->] {((x-1)*(x-5))/(5*(x+3))};



  \end{axis}
\end{tikzpicture}
\end{image}

\end{example}




In addition to horizontal and vertical asymptotes, graphs of rational functions can have \textbf{oblique} asymptotes, which describe the end-behavior.












The graph of a rational function need not have an asymptote at a zero of the denominator.


\begin{example}


The graph of $y = h(t) = \frac{(t-1)(t-5)}{(t-5)(t+3)} $ \\

$5$ is a zero of the polynomial in the denominator.  However, $t=5$ is not an asymptote in the graph.


\begin{image}
\begin{tikzpicture} 
  \begin{axis}[
            domain=-20:20, ymax=20, xmax=20, ymin=-20, xmin=-20,
            axis lines =center, xlabel=$t$, ylabel=$y$,
            ytick={-20,-16,-12,-8,-4,4,8,12,16,20},
            xtick={-20,-16,-12,-8,-4,4,8,12,16,20},
            ticklabel style={font=\scriptsize},
            every axis y label/.style={at=(current axis.above origin),anchor=south},
            every axis x label/.style={at=(current axis.right of origin),anchor=west},
            axis on top
          ]
 
          \addplot [line width=1, gray, dashed, domain=(-19.5:19.5),<->] ({-3},{x});
          \addplot [line width=1, gray, dashed, domain=(-19.5:19.5),<->] ({x},{1});

          \addplot [line width=2, penColor, smooth, samples=200, domain=(-19:-3.4),<->] {(x-1)/(x+3)};
          \addplot [line width=2, penColor, smooth, samples=200, domain=(-2.9:19),<->] {(x-1)/(x+3)};


          \addplot[color=penColor,fill=white,only marks,mark=*] coordinates{(5,0.5)}; 
                 

  \end{axis}
\end{tikzpicture}
\end{image}




This is because $5$ is also a zero of the polynomial in the numerator (with the same exponent).


\[ y = h(t) = \frac{(t-1)(t-5)}{(t-5)(t+3)} = \frac{(t-1)}{(t+3)}  \, \text{ for } \,  t \ne 5  \]

\end{example}




































The example illustrates why we prefer polynomial and rational functions in factored form.  It helps us make decisions on zeros and singularities. 

Once we get to Calculus, we will see that the derivatives of polynomials and rational functions are again polynomials and rational functions. We will want to factor the derivatives as well to help us decide where the function is increasing and decreasing (behavior). \\




Therefore, usually our first step is to transform any rational function to look like



\[ p(x) =   \frac{ a (x-r_n)(x-r_{n-1})  \cdots (x-r_2)(x-r_1)  } { b (x-s_m)(x-s_{m-1})  \cdots (x-s_2)(x-s_1) }   \]




And, again, we will be able to obtain a product of linear factors with the addition of complex numbers.  With real numbers we can get these products to consist only of linear and irreducible quadratics.  Therefore, we will leave factors of irreducible quadratics to the next course. \\

As our first step, let's consider rational functions that do factor into linear factors with real numbers. \\




Finally, we would like to clean up the factors and group them together






\[ p(x) =   \frac{ a (x-r_n)^{e_n} (x-r_{n-1})^{e_{n-1}}  \cdots (x-r_2)^{e_2} (x-r_1)^{e_1}  } { b (x-s_m)^{f_m} (x-s_{m-1})^{f_{m-1}}  \cdots (x-s_2)^{f_2} (x-s_1)^{f_1} }   \]






\subsection*{Reduced Form}

\textbf{Shared Roots}

Suppose the numerator and denominater share a root, $r_i = s_j$.  Then we can reduce the expression but we must remember that $s_j$ is not in the domain.


With that reduction in mind, let's assume that we have reduced the rational expression and there are no shared roots between the numerator and denominator. \\




\textbf{No Shared Roots}

In this case, all of the roots are distinct

\begin{itemize}
\item $r_i \ne s_j$ for all possible $i$ and $j$
\item $r_i \ne r_j$ for all possible $i$ and $j$
\item $s_i \ne s_j$ for all possible $i$ and $j$
\end{itemize}


With this we can analyze our rational function from the perspective of the roots.



$\blacktriangleright$ \textbf{Numerator}



The numerator is a polynomial.  Each factor gives a root or zero of the function, which corresponds to an intercept on the graph. \\

If the multiplicity of the root is odd, then the function changes sign and the graph crosses the axis at this intercept.  \\
If the multiplicity of the root is even, then the graph does not cross the axis at this intercept and the function maintains its sign. 






$\blacktriangleright$ \textbf{Denominator}


The denominator is a polynomial.  Each factor gives a root or zero of the denominator, which corresponds to a singularity of the rational function and a vertical asymptote on the graph. \\


The factors in the denominator still affect the sign of the function values.


If the multiplicity of the singularity is odd, then the function changes sign and the graph jumps to the other end of the vertical asymptote.  \\
If the multiplicity of the singularity is even, then the function does not change sign and the graph does not jump to the other end of the vertical asymptote. 















\begin{example} Rational Function


The graph of $y = H(w) = \frac{(w-1)}{(w+3) (w-4)} $





\begin{image}
\begin{tikzpicture} 
  \begin{axis}[
            domain=-10:10, ymax=10, xmax=10, ymin=-10, xmin=-10,
            axis lines =center, xlabel=$w$, ylabel=$y$,
            ytick={-10,-8,-6,-4,-2,2,4,6,8,10},
            xtick={-10,-8,-6,-4,-2,2,4,6,8,10},
            ticklabel style={font=\scriptsize},
            every axis y label/.style={at=(current axis.above origin),anchor=south},
            every axis x label/.style={at=(current axis.right of origin),anchor=west},
            axis on top
          ]
          
          \addplot [line width=2, penColor, smooth, domain=(-9:-3.1),<->] {(x-1)/((x+3)*(x-4))};
          \addplot [line width=2, penColor, smooth, domain=(-2.9:3.9),<->] {(x-1)/((x+3)*(x-4))};
          \addplot [line width=2, penColor, smooth, domain=(4.1:9),<->] {(x-1)/((x+3)*(x-4))};

          \addplot [line width=1, gray, dashed, domain=(-9.5:9.5),<->] ({-3},{x});
          \addplot [line width=1, gray, dashed, domain=(-9.5:9.5),<->] ({4},{x});

           

  \end{axis}
\end{tikzpicture}
\end{image}





The numerator has the factor $w-1$, which has multiplicity $\answer{1}$, odd.  Therefore $H$ has $\answer{1}$ as a root and the graph has $(1,0)$ as its only intercept, the only place where the graph crosses the horizontal $w$-axis.


The denominator has two factors $w+3$ and $w-4$.  Both have \wordChoice{\choice[correct]{odd} \choice {even}}  multiplicity, therefore, the function changes sign over these singularities and the graph jumps to the other end of the vertical asymptotes.


The degree of the denominator is larger than the degree of the numerator, therefore the $w$-axis is a horizontal asymptote.



The end-behavior is



\[       \lim_{w \to -\infty} H(w) = \answer{0}   \, \text{ and } \,    \lim_{w \to \infty} H(w) = \answer{0}               \]





$\blacktriangleright$ \textbf{Vertical Asymptotes} 





The factor $w+3$ corresponds to the vertical asymptote described by $w = -3$.


Near $-3$, on either side of $-3$, the factors $w-1$ and $w-4$ do not change sign. 



\begin{itemize}
\item The value of $w-1$ is near $-4$ (negative), when $w$ is near $-3$.
\item The value of $w-4$ is near $-7$ (negative), when $w$ is near $-3$.
\end{itemize}

Both factors take on negative values when $w$ is near $-3$.  

Near $-3$, the value of the numerator of $H$ is near $(-4)(-7) = 28$, when $w$ is near $-3$.  \\





The factor $w+3$ does change sign, since its multiplicity is odd.

\begin{itemize}
\item On the left side, where $w < -3$, we have $w+3<0$, negative.
\item On the right side, where $w > -3$, we have $w+3>0$, positive.
\end{itemize}


Therefore, on the left side of $-3$, $H(w) = \frac{neg}{neg \, \cdot \, neg} = negative$.  \\

Plus, the denominator is approaching $0$ making the whole fraction get bigger and bigger negatively.

We can see this in the graph.  The graph moves down the left side of the $w=-3$ vertical asymptote.

Since this singularity has an odd multiplicity, we also know that $H$ changes sign across this vertical asymptote.


We can now walk through the sign changes of $H$.

\begin{itemize}
\item $w = 1$ is a zero of $H$ with odd multipicity.  The sign of $H$ changes across this zero. The graph crosses the $w$-axis at the $(1, 0)$ intercept.
\item $w = 4$ has an odd singularity. The sign of $H$ changes across this singularity.  The graph jumps to the other infinity across the vertical asymptote.
\end{itemize}



We can now categorize the singularity behavior for $H$.

\begin{itemize}
\item $\lim\limits_{w \to -3^-} H(w) = -\infty$
\item $\lim\limits_{w \to -3^+} H(w) = \infty$
\item $\lim\limits_{w \to 4^-} H(w) = -\infty$
\item $\lim\limits_{w \to 4^+} H(w) = \infty$
\end{itemize}

$H$ has no global or local maximums or minimums.





\textbf{Range:} \\


On the interval $(-3, 4)$, $H$ is continuous.

\begin{itemize}
\item $\lim\limits_{w \to -3^+} H(w) = \infty$
\item $\lim\limits_{w \to 4^-} H(w) = -\infty$
\end{itemize}

That gives us a range for $H$ of $(-\infty, \infty )$.

\end{example}







\subsection*{Sign Changes}

For rational functions, the sign can change only at a zero of odd multiplicity or a singularity of odd multiplicity. This is helpful when graphing. These requirements can force the graph to go in particular directions.

And, since the derivative of a rational function is also a rational function, multiplicities are also helpful when determining where a rational function is increasing or decreasing, because the multiplicities of the derivative will tell us whent he derivative changes sign. \\









\subsection*{End-Behavior}



The end-behavior of a rational function depends on the leading terms of the numerator and denominator.





\textbf{\textcolor{red!90!darkgray}{$\blacktriangleright$}} If $n > m$




\[   \lim\limits_{x \to \pm\infty}  \frac{ a_n x^n + a_{n-1} x^{n-1} + \cdots + a_3 x^3 + a_2 x^2 + a_1 x + a_0  } { b_m x^m + b_{m-1} x^{m-1} + \cdots + b_3 x^3 + b_2 x^2 + b_1 x + b_0 }   = \lim\limits_{x \to \pm\infty} \frac{ a_n x^n } { b_m x^m } = \pm \infty
\]











\textbf{\textcolor{red!90!darkgray}{$\blacktriangleright$}} If $n < m$




\[  \lim\limits_{x \to \pm\infty} \frac{ a_n x^n + a_{n-1} x^{n-1} + \cdots + a_3 x^3 + a_2 x^2 + a_1 x + a_0  } { b_m x^m + b_{m-1} x^{m-1} + \cdots + b_3 x^3 + b_2 x^2 + b_1 x + b_0 }   = \lim\limits_{x \to \pm\infty} \frac{ a_n x^n } { b_m x^m } = 0
\]











\textbf{\textcolor{red!90!darkgray}{$\blacktriangleright$}} If $n = m$




\[  \lim\limits_{x \to \pm\infty} \frac{ a_n x^n + a_{n-1} x^{n-1} + \cdots + a_3 x^3 + a_2 x^2 + a_1 x + a_0  } { b_m x^m + b_{m-1} x^{m-1} + \cdots + b_3 x^3 + b_2 x^2 + b_1 x + b_0 }   =  \lim\limits_{x \to \pm\infty} \frac{ a_n x^n } { b_m x^m } = \frac{ a_n} { b_m }
\]















\begin{center}
\textbf{\textcolor{green!50!black}{ooooo-=-=-=-ooOoo-=-=-=-ooooo}} \\

more examples can be found by following this link\\ \link[More Examples of Poynomial and Rational functions]{https://ximera.osu.edu/csccmathematics/precalculus/precalculus/polynomialFunctions/examples/exampleList}

\end{center}







\end{document}
