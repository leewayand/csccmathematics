\documentclass{ximera}

%\usepackage{todonotes}

\newcommand{\todo}{}

\usepackage{esint} % for \oiint
\ifxake%%https://math.meta.stackexchange.com/questions/9973/how-do-you-render-a-closed-surface-double-integral
\renewcommand{\oiint}{{\large\bigcirc}\kern-1.56em\iint}
\fi


\graphicspath{
  {./}
  {ximeraTutorial/}
  {basicPhilosophy/}
  {functionsOfSeveralVariables/}
  {normalVectors/}
  {lagrangeMultipliers/}
  {vectorFields/}
  {greensTheorem/}
  {shapeOfThingsToCome/}
  {dotProducts/}
  {partialDerivativesAndTheGradientVector/}
  {../productAndQuotientRules/exercises/}
  {../normalVectors/exercisesParametricPlots/}
  {../continuityOfFunctionsOfSeveralVariables/exercises/}
  {../partialDerivativesAndTheGradientVector/exercises/}
  {../directionalDerivativeAndChainRule/exercises/}
  {../commonCoordinates/exercisesCylindricalCoordinates/}
  {../commonCoordinates/exercisesSphericalCoordinates/}
  {../greensTheorem/exercisesCurlAndLineIntegrals/}
  {../greensTheorem/exercisesDivergenceAndLineIntegrals/}
  {../shapeOfThingsToCome/exercisesDivergenceTheorem/}
  {../greensTheorem/}
  {../shapeOfThingsToCome/}
  {../separableDifferentialEquations/exercises/}
  {vectorFields/}
}

\newcommand{\mooculus}{\textsf{\textbf{MOOC}\textnormal{\textsf{ULUS}}}}

\usepackage{tkz-euclide}
\usepackage{tikz}
\usepackage{tikz-cd}
\usetikzlibrary{arrows}
\tikzset{>=stealth,commutative diagrams/.cd,
  arrow style=tikz,diagrams={>=stealth}} %% cool arrow head
\tikzset{shorten <>/.style={ shorten >=#1, shorten <=#1 } } %% allows shorter vectors

\usetikzlibrary{backgrounds} %% for boxes around graphs
\usetikzlibrary{shapes,positioning}  %% Clouds and stars
\usetikzlibrary{matrix} %% for matrix
\usepgfplotslibrary{polar} %% for polar plots
\usepgfplotslibrary{fillbetween} %% to shade area between curves in TikZ
%\usetkzobj{all}
\usepackage[makeroom]{cancel} %% for strike outs
%\usepackage{mathtools} %% for pretty underbrace % Breaks Ximera
%\usepackage{multicol}
\usepackage{pgffor} %% required for integral for loops



%% http://tex.stackexchange.com/questions/66490/drawing-a-tikz-arc-specifying-the-center
%% Draws beach ball
\tikzset{pics/carc/.style args={#1:#2:#3}{code={\draw[pic actions] (#1:#3) arc(#1:#2:#3);}}}



\usepackage{array}
\setlength{\extrarowheight}{+.1cm}
\newdimen\digitwidth
\settowidth\digitwidth{9}
\def\divrule#1#2{
\noalign{\moveright#1\digitwidth
\vbox{\hrule width#2\digitwidth}}}




% \newcommand{\RR}{\mathbb R}
% \newcommand{\R}{\mathbb R}
% \newcommand{\N}{\mathbb N}
% \newcommand{\Z}{\mathbb Z}

\newcommand{\sagemath}{\textsf{SageMath}}


%\renewcommand{\d}{\,d\!}
%\renewcommand{\d}{\mathop{}\!d}
%\newcommand{\dd}[2][]{\frac{\d #1}{\d #2}}
%\newcommand{\pp}[2][]{\frac{\partial #1}{\partial #2}}
% \renewcommand{\l}{\ell}
%\newcommand{\ddx}{\frac{d}{\d x}}

% \newcommand{\zeroOverZero}{\ensuremath{\boldsymbol{\tfrac{0}{0}}}}
%\newcommand{\inftyOverInfty}{\ensuremath{\boldsymbol{\tfrac{\infty}{\infty}}}}
%\newcommand{\zeroOverInfty}{\ensuremath{\boldsymbol{\tfrac{0}{\infty}}}}
%\newcommand{\zeroTimesInfty}{\ensuremath{\small\boldsymbol{0\cdot \infty}}}
%\newcommand{\inftyMinusInfty}{\ensuremath{\small\boldsymbol{\infty - \infty}}}
%\newcommand{\oneToInfty}{\ensuremath{\boldsymbol{1^\infty}}}
%\newcommand{\zeroToZero}{\ensuremath{\boldsymbol{0^0}}}
%\newcommand{\inftyToZero}{\ensuremath{\boldsymbol{\infty^0}}}



% \newcommand{\numOverZero}{\ensuremath{\boldsymbol{\tfrac{\#}{0}}}}
% \newcommand{\dfn}{\textbf}
% \newcommand{\unit}{\,\mathrm}
% \newcommand{\unit}{\mathop{}\!\mathrm}
% \newcommand{\eval}[1]{\bigg[ #1 \bigg]}
% \newcommand{\seq}[1]{\left( #1 \right)}
% \renewcommand{\epsilon}{\varepsilon}
% \renewcommand{\phi}{\varphi}


% \renewcommand{\iff}{\Leftrightarrow}

% \DeclareMathOperator{\arccot}{arccot}
% \DeclareMathOperator{\arcsec}{arcsec}
% \DeclareMathOperator{\arccsc}{arccsc}
% \DeclareMathOperator{\si}{Si}
% \DeclareMathOperator{\scal}{scal}
% \DeclareMathOperator{\sign}{sign}


%% \newcommand{\tightoverset}[2]{% for arrow vec
%%   \mathop{#2}\limits^{\vbox to -.5ex{\kern-0.75ex\hbox{$#1$}\vss}}}
% \newcommand{\arrowvec}[1]{{\overset{\rightharpoonup}{#1}}}
% \renewcommand{\vec}[1]{\arrowvec{\mathbf{#1}}}
% \renewcommand{\vec}[1]{{\overset{\boldsymbol{\rightharpoonup}}{\mathbf{#1}}}}

% \newcommand{\point}[1]{\left(#1\right)} %this allows \vector{ to be changed to \vector{ with a quick find and replace
% \newcommand{\pt}[1]{\mathbf{#1}} %this allows \vec{ to be changed to \vec{ with a quick find and replace
% \newcommand{\Lim}[2]{\lim_{\point{#1} \to \point{#2}}} %Bart, I changed this to point since I want to use it.  It runs through both of the exercise and exerciseE files in limits section, which is why it was in each document to start with.

% \DeclareMathOperator{\proj}{\mathbf{proj}}
% \newcommand{\veci}{{\boldsymbol{\hat{\imath}}}}
% \newcommand{\vecj}{{\boldsymbol{\hat{\jmath}}}}
% \newcommand{\veck}{{\boldsymbol{\hat{k}}}}
% \newcommand{\vecl}{\vec{\boldsymbol{\l}}}
% \newcommand{\uvec}[1]{\mathbf{\hat{#1}}}
% \newcommand{\utan}{\mathbf{\hat{t}}}
% \newcommand{\unormal}{\mathbf{\hat{n}}}
% \newcommand{\ubinormal}{\mathbf{\hat{b}}}

% \newcommand{\dotp}{\bullet}
% \newcommand{\cross}{\boldsymbol\times}
% \newcommand{\grad}{\boldsymbol\nabla}
% \newcommand{\divergence}{\grad\dotp}
% \newcommand{\curl}{\grad\cross}
%\DeclareMathOperator{\divergence}{divergence}
%\DeclareMathOperator{\curl}[1]{\grad\cross #1}
% \newcommand{\lto}{\mathop{\longrightarrow\,}\limits}

% \renewcommand{\bar}{\overline}

\colorlet{textColor}{black}
\colorlet{background}{white}
\colorlet{penColor}{blue!50!black} % Color of a curve in a plot
\colorlet{penColor2}{red!50!black}% Color of a curve in a plot
\colorlet{penColor3}{red!50!blue} % Color of a curve in a plot
\colorlet{penColor4}{green!50!black} % Color of a curve in a plot
\colorlet{penColor5}{orange!80!black} % Color of a curve in a plot
\colorlet{penColor6}{yellow!70!black} % Color of a curve in a plot
\colorlet{fill1}{penColor!20} % Color of fill in a plot
\colorlet{fill2}{penColor2!20} % Color of fill in a plot
\colorlet{fillp}{fill1} % Color of positive area
\colorlet{filln}{penColor2!20} % Color of negative area
\colorlet{fill3}{penColor3!20} % Fill
\colorlet{fill4}{penColor4!20} % Fill
\colorlet{fill5}{penColor5!20} % Fill
\colorlet{gridColor}{gray!50} % Color of grid in a plot

\newcommand{\surfaceColor}{violet}
\newcommand{\surfaceColorTwo}{redyellow}
\newcommand{\sliceColor}{greenyellow}




\pgfmathdeclarefunction{gauss}{2}{% gives gaussian
  \pgfmathparse{1/(#2*sqrt(2*pi))*exp(-((x-#1)^2)/(2*#2^2))}%
}


%%%%%%%%%%%%%
%% Vectors
%%%%%%%%%%%%%

%% Simple horiz vectors
\renewcommand{\vector}[1]{\left\langle #1\right\rangle}


%% %% Complex Horiz Vectors with angle brackets
%% \makeatletter
%% \renewcommand{\vector}[2][ , ]{\left\langle%
%%   \def\nextitem{\def\nextitem{#1}}%
%%   \@for \el:=#2\do{\nextitem\el}\right\rangle%
%% }
%% \makeatother

%% %% Vertical Vectors
%% \def\vector#1{\begin{bmatrix}\vecListA#1,,\end{bmatrix}}
%% \def\vecListA#1,{\if,#1,\else #1\cr \expandafter \vecListA \fi}

%%%%%%%%%%%%%
%% End of vectors
%%%%%%%%%%%%%

%\newcommand{\fullwidth}{}
%\newcommand{\normalwidth}{}



%% makes a snazzy t-chart for evaluating functions
%\newenvironment{tchart}{\rowcolors{2}{}{background!90!textColor}\array}{\endarray}

%%This is to help with formatting on future title pages.
\newenvironment{sectionOutcomes}{}{}



%% Flowchart stuff
%\tikzstyle{startstop} = [rectangle, rounded corners, minimum width=3cm, minimum height=1cm,text centered, draw=black]
%\tikzstyle{question} = [rectangle, minimum width=3cm, minimum height=1cm, text centered, draw=black]
%\tikzstyle{decision} = [trapezium, trapezium left angle=70, trapezium right angle=110, minimum width=3cm, minimum height=1cm, text centered, draw=black]
%\tikzstyle{question} = [rectangle, rounded corners, minimum width=3cm, minimum height=1cm,text centered, draw=black]
%\tikzstyle{process} = [rectangle, minimum width=3cm, minimum height=1cm, text centered, draw=black]
%\tikzstyle{decision} = [trapezium, trapezium left angle=70, trapezium right angle=110, minimum width=3cm, minimum height=1cm, text centered, draw=black]


\title{Optimization}

\begin{document}

\begin{abstract}
minimum and maximum
\end{abstract}
\maketitle



When using functions to analyze situations, we are often interested in the maximum and minimum values that a function takes on. The maximum or minimum value the function \textit{assumes}. These are often called \textbf{extreme} values. We would like to know the extreme values and where, in the domain, they occur.



There are two views on extreme values.

\begin{definition} \textbf{\textcolor{green!50!black}{Global Extreme Values}} 

\begin{itemize}
\item The \textbf{global maximum value} is the greatest value of the function.  

\[  f(c) \text{ is a global maximum if } f(x) \leq f(c) \text{ for all } x \text{ in the domain of } f \]

\item The \textbf{global minimum value} is the least value of the function.  

\[  f(c) \text{ is a global minimum if } f(c) \leq f(x) \text{ for all } x \text{ in the domain of } f \]
\end{itemize}

A function has at most, one maximum value and one minimum value, but they can occur at multiple domain numbers.

Global extrema are also called \textbf{absolute} extrema.

\end{definition}




\begin{warning}

\begin{itemize}
\item A global maximum may occur at multiple domain numbers.  
\item A function can also not have a maximum value.
\end{itemize}

\begin{itemize}
\item A global minimum may occur at multiple domain numbers.  
\item A function can also not have a minimum value.
\end{itemize}

\end{warning}

















In contrast to a single maximum or minimum value, a function may also have values which are the greatest value in their own little neighborhood of the domain, but not neccessarily the greatest overall value.  We see these visually encoded as tops of hills and bottom of valleys on the graph.  A function may have many of these \textbf{local} or \textbf{relative} extrema.






\begin{definition} \textbf{\textcolor{green!50!black}{Local Extreme Values}} 
\begin{itemize}
\item $f(c)$ is a \textbf{local maximum value} of the function $f$, if there exists an $\epsilon > 0$ such that $f(x) \leq f(c)$ for all domain numbers, $x$, within a distance of $\epsilon$ from $c$. 

\[  f(c) \text{ is a local maximum if } f(x) \leq f(c) \text{ for all } \{ x \in domain \, | \, x \in (c - \epsilon, c + \epsilon) \} \text{ for some } \epsilon > 0 \]







\item $f(c)$ is a \textbf{local minimum value} of the function $f$, if there exists an $\epsilon > 0$ such that $f(c) \leq f(x)$ for all domain numbers, $x$, within a distance of $\epsilon$ from $c$. 



\[  f(c) \text{ is a local minimum if } f(c) \leq f(x) \text{ for all } \{ x \in domain \, | \, x \in (c - \epsilon, c + \epsilon) \} \text{ for some } \epsilon > 0 \]
\end{itemize}

Local extrema are also called \textbf{relative} extrema.

\end{definition}





\textbf{Note:} By default, all global extrema are automatically local extrema.






\begin{example} Extrema \\

Let $g(k)$ be a function.  The graph of $y= g(k)$ is displayed below. 

\begin{image}
\begin{tikzpicture}
     \begin{axis}[
            	domain=-10:10, ymax=10, xmax=10, ymin=-10, xmin=-10,
            	axis lines =center, xlabel=$k$, ylabel=$y$,
                ytick={-10,-8,-6,-4,-2,2,4,6,8,10},
                xtick={-10,-8,-6,-4,-2,2,4,6,8,10},
                ticklabel style={font=\scriptsize},
            	every axis y label/.style={at=(current axis.above origin),anchor=south},
            	every axis x label/.style={at=(current axis.right of origin),anchor=west},
            	axis on top,
          		]

        
        \addplot [draw=penColor, very thick, smooth, domain=(-8:5), ->] {0.05*(x+7)*(x+2)*(x-3)};
        %\addplot [draw=penColor, very thick, smooth, domain=(-1:3)] {0.5*x-1};
        %\addplot [draw=penColor, very thick, smooth, domain=(4:8)] {-2*x+10};


        \addplot[color=penColor,fill=penColor,only marks,mark=*] coordinates{(-8,-3.3)};
        %\addplot[color=penColor,fill=white,only marks,mark=*] coordinates{(-3,3)};


    \end{axis}
\end{tikzpicture}
\end{image}

\begin{itemize}
\item $g$ has no global maximum.
\item The global minimum of $g$ is $-3.3$, which occurs at $-8$.
\item $g$ has a local minimum of $-3.3$, which occurs at $-8$ and a local minimum of $-2.3$, which occurs at $0.9$.
\item $g$ has a local maximum of $2.4$, which occurs at $-4.9$
\end{itemize}

\end{example}






Let's run through the idea of a local extrema for the previous example. \\


\begin{explanation} \textit{Local Explanations}




$\blacktriangleright$ $g$ has a local minimum of $-2.3$, which occurs at $0.9$. \\


Let $\epsilon = 0.3$.  Then, $\epsilon > 0$.   \\
$-2.3 = g(0.9) \leq g(k)$ for all $k$ in the domain within a distance of $0.3$ from $0.9$, which would be the interval $\left(\answer{0.6}, \answer{1.2}\right)$.




$\blacktriangleright$ $g$ has a local maximum of $2.4$, which occurs at $-4.9$. \\






Let $\epsilon = 0.4$.  Then, $\epsilon > 0$.  \\
$2.4 = g(k) \leq g(-4.9)$ for all $k$ in the domain within a distance of $0.4$ from $-4.9$, which would be the interval $\left(\answer{-5.3}, \answer{-4.5}\right)$.




$\blacktriangleright$ $g$ has a local minimum of $-3.3$, which occurs at $-8$. \\





Let $\epsilon = 0.5$.  Then, $\epsilon > 0$. \\

$-3.3 = g(-8) \leq g(k)$ for all $k$ in the domain within a distance of $0.5$ from $-8$, which would be the interval $\left[\answer{-8.5}, \answer{-7.5}\right)$.


\end{explanation}


\textbf{Note:} As the previous example illustrates, global extrema are also local extrema. \\
\textbf{Note:} We picked $0.3$, $0.4$, and $0.5$ for $\epsilon$, but we could have selected any small, positive numbers that work. In fact, once one positive number works, then any lesser positive number will also work. \\



\begin{example} Extrema \\

Let $B(w)$ be a function.  The graph of $y = B(w)$ is displayed below. 

\begin{image}
\begin{tikzpicture}
     \begin{axis}[
            	domain=-10:10, ymax=10, xmax=10, ymin=-10, xmin=-10,
            	axis lines =center, xlabel=$w$, ylabel=$y$,
                ytick={-10,-8,-6,-4,-2,2,4,6,8,10},
                xtick={-10,-8,-6,-4,-2,2,4,6,8,10},
                ticklabel style={font=\scriptsize},
            	every axis y label/.style={at=(current axis.above origin),anchor=south},
            	every axis x label/.style={at=(current axis.right of origin),anchor=west},
            	axis on top,
          		]

        
        \addplot [draw=penColor, very thick, smooth, domain=(-8:4), <->] {0.01*(x+4)*(x+7)*(x+2)*(x-3)};
        %\addplot [draw=penColor, very thick, smooth, domain=(-1:3)] {0.5*x-1};
        %\addplot [draw=penColor, very thick, smooth, domain=(4:8)] {-2*x+10};


        %\addplot[color=penColor,fill=penColor,only marks,mark=*] coordinates{(-8,4.3)};
        %\addplot[color=penColor,fill=white,only marks,mark=*] coordinates{(-3,3)};


    \end{axis}
\end{tikzpicture}
\end{image}

\begin{itemize}
\item $B$ has no global maximum.
\item The global minimum of $B$ is $-2.3$, which occurs at $1.4$.
\item $B$ has a local minimum of $-2.3$, which occurs at $1.4$ and a local minimum of $-0.7$, which occurs at $-5.9$.
\item $B$ has a local maximum of $0.25$, which occurs at $-3$
\end{itemize}

\end{example}







\begin{explanation}
Local Explanations for $B(w)$ \\

Let $0 < \epsilon = 0.6$, then   $-2.3 = B(1.4) \leq B(w)$ for all $w$ in the domain within a distance of $\answer{0.6}$ from $\answer{1.4}$, which would be the interval $(0.8, 2)$.

Let $0 < \epsilon = 0.25$, then   $-0.7 = B(-5.9) \leq B(w)$ for all $w$ in the domain within a distance of $\answer{0.25}$ from $\answer{-5.9}$, which would be the interval $(-6.15, -5.65)$.

Let $0 < \epsilon = 0.5$, then  $0.25 = B(w) \leq B(-3)$ for all $w$ in the domain within a distance of $\answer{0.5}$ from $\answer{-3}$, which would be the interval $(-3.5, -2.5)$.
\end{explanation}







\begin{example} Extrema \\

Let $T(p)$ be a function.  The graph of $y=  T(p)$ is displayed below. 

\begin{image}
\begin{tikzpicture}
     \begin{axis}[
            	domain=-10:10, ymax=10, xmax=10, ymin=-10, xmin=-10,
            	axis lines =center, xlabel=$p$, ylabel=$y$,
            	every axis y label/.style={at=(current axis.above origin),anchor=south},
            	every axis x label/.style={at=(current axis.right of origin),anchor=west},
            	axis on top,
          		]

        
        \addplot [draw=penColor, very thick, smooth, domain=(-8:-1)] {-x};
        \addplot [draw=penColor, very thick, smooth, domain=(-1:3)] {0.5*x-1};
        \addplot [draw=penColor, very thick, smooth, domain=(3:8), ->] {-2*x+10};


        \addplot[color=penColor,fill=white,only marks,mark=*] coordinates{(-8,8)};
        \addplot[color=penColor,fill=penColor,only marks,mark=*] coordinates{(-8,-2)};

        \addplot[color=penColor,fill=white,only marks,mark=*] coordinates{(-1,1)};
        \addplot[color=penColor,fill=white,only marks,mark=*] coordinates{(-1,-1.5)};
        \addplot[color=penColor,fill=penColor,only marks,mark=*] coordinates{(-1,6)};

        \addplot[color=penColor,fill=white,only marks,mark=*] coordinates{(3,0.5)};
        \addplot[color=penColor,fill=white,only marks,mark=*] coordinates{(3,4)};
        \addplot[color=penColor,fill=penColor,only marks,mark=*] coordinates{(3,-6)};

    \end{axis}
\end{tikzpicture}
\end{image}

\begin{itemize}
\item $T$ has no global maximum.
\item $T$ has no global minimum.
\item $T$ has a local minimum of $-2$, which occurs at $-8$.
\item $T$ has a local minimum of $\answer{-6}$, which occurs at $3$.
\item $T$ has a local maximum of $\answer{6}$, which occurs at $-1$.
\end{itemize}

\end{example}





It appears that $8$ would have been the global maximum of $T$ occurring at $-8$, however, the point $(-8, 8)$ is missing from the graph and $8$ is not in the range.  



\begin{idea}  \textbf{\textcolor{blue!55!black}{``Next'' Real Number}}


The real numbers have a special property which makes Calculus work. \\

We can illustrate this property with the number $8$. \\

\begin{explanation}

There is no real number ``just below $8$''.  \\


If you choose any real number, $r$, below $8$ to be the possible candidate for the previous number, then there is the number $\frac{8+r}{2}$. 

\[  
r < \frac{8+r}{2} < 8
\]  

You cannot identify a specific number as the ``previous'' number, so there isn't one. \\

This is the idea behind space and open intervals.


\end{explanation}

\end{idea}



\begin{explanation} \textit{Local Explanations}


Let $0 < \epsilon = \answer{0.5}$.  $T(-8) \leq T(p)$ for all $p$ in the domain within a distance of $0.5$ from $-8$, which would be the interval $[-8, -7.5)$.

Let $0 < \epsilon = \answer{0.4}$.  $T(3) \leq T(p)$ for all $p$ in the domain within a distance of $0.4$ from $3$, which would be the interval $(3.4, 2.6)$.

Let $0 < \epsilon = \answer{0.3}$.  $T(p) \leq T(-1)$ for all $p$ in the domain within a distance of $0.3$ from $-1$, which would be the interval $(-1.3, -0.7)$.

\end{explanation}





\begin{idea} \textbf{\textcolor{blue!55!black}{Next}}



Inside the natural numbers there is the idea of ``next''.  Each natural number has a next number. \\


However, the real numbers does not have a ``next'' property.  There is no ``next'' number for any real number.


Let $r$ be a real number. \\

Supposed $N$ is the ``next'' real number.  Then we have a problem, because the real number $\frac{r+N}{2}$ is between $r$ and $N$.  Therefore, $N$ cannot be the ``next'' real number.  There is no ``next'' real number. \\


A consequence of this is that each open interval contains an infinite number of numbers. \\



\textbf{Note: } This is not true for closed intervals.  For example, the closed interval $[3, 3]$ contains only one number.  This is one reason we talk about open intervals.



\end{idea}





\textbf{\textcolor{red!70!black}{$\blacktriangleright$ Thinking Ahead}} \\


Linear functions have the same rate of change over any interval. \\

That is not quite true.  They have a constant rate of change over an interval of the form $[a, b]$, where $a \ne b$. \\

As a mental exercise, if a linear function has the exact same rate of change over any interval $[a, b]$, then doesn't it feel like the linear function should have the same rate of change over the interval $[a, a]$? \\

Just thinking.






\begin{center}
\textbf{\textcolor{green!50!black}{ooooo-=-=-=-ooOoo-=-=-=-ooooo}} \\

more examples can be found by following this link\\ \link[More Examples of Visual Behavior]{https://ximera.osu.edu/csccmathematics/precalculus1/precalculus1/visualBehavior/examples/exampleList}

\end{center}






\end{document}
