\documentclass{ximera}

%\usepackage{todonotes}

\newcommand{\todo}{}

\usepackage{esint} % for \oiint
\ifxake%%https://math.meta.stackexchange.com/questions/9973/how-do-you-render-a-closed-surface-double-integral
\renewcommand{\oiint}{{\large\bigcirc}\kern-1.56em\iint}
\fi


\graphicspath{
  {./}
  {ximeraTutorial/}
  {basicPhilosophy/}
  {functionsOfSeveralVariables/}
  {normalVectors/}
  {lagrangeMultipliers/}
  {vectorFields/}
  {greensTheorem/}
  {shapeOfThingsToCome/}
  {dotProducts/}
  {partialDerivativesAndTheGradientVector/}
  {../productAndQuotientRules/exercises/}
  {../normalVectors/exercisesParametricPlots/}
  {../continuityOfFunctionsOfSeveralVariables/exercises/}
  {../partialDerivativesAndTheGradientVector/exercises/}
  {../directionalDerivativeAndChainRule/exercises/}
  {../commonCoordinates/exercisesCylindricalCoordinates/}
  {../commonCoordinates/exercisesSphericalCoordinates/}
  {../greensTheorem/exercisesCurlAndLineIntegrals/}
  {../greensTheorem/exercisesDivergenceAndLineIntegrals/}
  {../shapeOfThingsToCome/exercisesDivergenceTheorem/}
  {../greensTheorem/}
  {../shapeOfThingsToCome/}
  {../separableDifferentialEquations/exercises/}
  {vectorFields/}
}

\newcommand{\mooculus}{\textsf{\textbf{MOOC}\textnormal{\textsf{ULUS}}}}

\usepackage{tkz-euclide}
\usepackage{tikz}
\usepackage{tikz-cd}
\usetikzlibrary{arrows}
\tikzset{>=stealth,commutative diagrams/.cd,
  arrow style=tikz,diagrams={>=stealth}} %% cool arrow head
\tikzset{shorten <>/.style={ shorten >=#1, shorten <=#1 } } %% allows shorter vectors

\usetikzlibrary{backgrounds} %% for boxes around graphs
\usetikzlibrary{shapes,positioning}  %% Clouds and stars
\usetikzlibrary{matrix} %% for matrix
\usepgfplotslibrary{polar} %% for polar plots
\usepgfplotslibrary{fillbetween} %% to shade area between curves in TikZ
%\usetkzobj{all}
\usepackage[makeroom]{cancel} %% for strike outs
%\usepackage{mathtools} %% for pretty underbrace % Breaks Ximera
%\usepackage{multicol}
\usepackage{pgffor} %% required for integral for loops



%% http://tex.stackexchange.com/questions/66490/drawing-a-tikz-arc-specifying-the-center
%% Draws beach ball
\tikzset{pics/carc/.style args={#1:#2:#3}{code={\draw[pic actions] (#1:#3) arc(#1:#2:#3);}}}



\usepackage{array}
\setlength{\extrarowheight}{+.1cm}
\newdimen\digitwidth
\settowidth\digitwidth{9}
\def\divrule#1#2{
\noalign{\moveright#1\digitwidth
\vbox{\hrule width#2\digitwidth}}}




% \newcommand{\RR}{\mathbb R}
% \newcommand{\R}{\mathbb R}
% \newcommand{\N}{\mathbb N}
% \newcommand{\Z}{\mathbb Z}

\newcommand{\sagemath}{\textsf{SageMath}}


%\renewcommand{\d}{\,d\!}
%\renewcommand{\d}{\mathop{}\!d}
%\newcommand{\dd}[2][]{\frac{\d #1}{\d #2}}
%\newcommand{\pp}[2][]{\frac{\partial #1}{\partial #2}}
% \renewcommand{\l}{\ell}
%\newcommand{\ddx}{\frac{d}{\d x}}

% \newcommand{\zeroOverZero}{\ensuremath{\boldsymbol{\tfrac{0}{0}}}}
%\newcommand{\inftyOverInfty}{\ensuremath{\boldsymbol{\tfrac{\infty}{\infty}}}}
%\newcommand{\zeroOverInfty}{\ensuremath{\boldsymbol{\tfrac{0}{\infty}}}}
%\newcommand{\zeroTimesInfty}{\ensuremath{\small\boldsymbol{0\cdot \infty}}}
%\newcommand{\inftyMinusInfty}{\ensuremath{\small\boldsymbol{\infty - \infty}}}
%\newcommand{\oneToInfty}{\ensuremath{\boldsymbol{1^\infty}}}
%\newcommand{\zeroToZero}{\ensuremath{\boldsymbol{0^0}}}
%\newcommand{\inftyToZero}{\ensuremath{\boldsymbol{\infty^0}}}



% \newcommand{\numOverZero}{\ensuremath{\boldsymbol{\tfrac{\#}{0}}}}
% \newcommand{\dfn}{\textbf}
% \newcommand{\unit}{\,\mathrm}
% \newcommand{\unit}{\mathop{}\!\mathrm}
% \newcommand{\eval}[1]{\bigg[ #1 \bigg]}
% \newcommand{\seq}[1]{\left( #1 \right)}
% \renewcommand{\epsilon}{\varepsilon}
% \renewcommand{\phi}{\varphi}


% \renewcommand{\iff}{\Leftrightarrow}

% \DeclareMathOperator{\arccot}{arccot}
% \DeclareMathOperator{\arcsec}{arcsec}
% \DeclareMathOperator{\arccsc}{arccsc}
% \DeclareMathOperator{\si}{Si}
% \DeclareMathOperator{\scal}{scal}
% \DeclareMathOperator{\sign}{sign}


%% \newcommand{\tightoverset}[2]{% for arrow vec
%%   \mathop{#2}\limits^{\vbox to -.5ex{\kern-0.75ex\hbox{$#1$}\vss}}}
% \newcommand{\arrowvec}[1]{{\overset{\rightharpoonup}{#1}}}
% \renewcommand{\vec}[1]{\arrowvec{\mathbf{#1}}}
% \renewcommand{\vec}[1]{{\overset{\boldsymbol{\rightharpoonup}}{\mathbf{#1}}}}

% \newcommand{\point}[1]{\left(#1\right)} %this allows \vector{ to be changed to \vector{ with a quick find and replace
% \newcommand{\pt}[1]{\mathbf{#1}} %this allows \vec{ to be changed to \vec{ with a quick find and replace
% \newcommand{\Lim}[2]{\lim_{\point{#1} \to \point{#2}}} %Bart, I changed this to point since I want to use it.  It runs through both of the exercise and exerciseE files in limits section, which is why it was in each document to start with.

% \DeclareMathOperator{\proj}{\mathbf{proj}}
% \newcommand{\veci}{{\boldsymbol{\hat{\imath}}}}
% \newcommand{\vecj}{{\boldsymbol{\hat{\jmath}}}}
% \newcommand{\veck}{{\boldsymbol{\hat{k}}}}
% \newcommand{\vecl}{\vec{\boldsymbol{\l}}}
% \newcommand{\uvec}[1]{\mathbf{\hat{#1}}}
% \newcommand{\utan}{\mathbf{\hat{t}}}
% \newcommand{\unormal}{\mathbf{\hat{n}}}
% \newcommand{\ubinormal}{\mathbf{\hat{b}}}

% \newcommand{\dotp}{\bullet}
% \newcommand{\cross}{\boldsymbol\times}
% \newcommand{\grad}{\boldsymbol\nabla}
% \newcommand{\divergence}{\grad\dotp}
% \newcommand{\curl}{\grad\cross}
%\DeclareMathOperator{\divergence}{divergence}
%\DeclareMathOperator{\curl}[1]{\grad\cross #1}
% \newcommand{\lto}{\mathop{\longrightarrow\,}\limits}

% \renewcommand{\bar}{\overline}

\colorlet{textColor}{black}
\colorlet{background}{white}
\colorlet{penColor}{blue!50!black} % Color of a curve in a plot
\colorlet{penColor2}{red!50!black}% Color of a curve in a plot
\colorlet{penColor3}{red!50!blue} % Color of a curve in a plot
\colorlet{penColor4}{green!50!black} % Color of a curve in a plot
\colorlet{penColor5}{orange!80!black} % Color of a curve in a plot
\colorlet{penColor6}{yellow!70!black} % Color of a curve in a plot
\colorlet{fill1}{penColor!20} % Color of fill in a plot
\colorlet{fill2}{penColor2!20} % Color of fill in a plot
\colorlet{fillp}{fill1} % Color of positive area
\colorlet{filln}{penColor2!20} % Color of negative area
\colorlet{fill3}{penColor3!20} % Fill
\colorlet{fill4}{penColor4!20} % Fill
\colorlet{fill5}{penColor5!20} % Fill
\colorlet{gridColor}{gray!50} % Color of grid in a plot

\newcommand{\surfaceColor}{violet}
\newcommand{\surfaceColorTwo}{redyellow}
\newcommand{\sliceColor}{greenyellow}




\pgfmathdeclarefunction{gauss}{2}{% gives gaussian
  \pgfmathparse{1/(#2*sqrt(2*pi))*exp(-((x-#1)^2)/(2*#2^2))}%
}


%%%%%%%%%%%%%
%% Vectors
%%%%%%%%%%%%%

%% Simple horiz vectors
\renewcommand{\vector}[1]{\left\langle #1\right\rangle}


%% %% Complex Horiz Vectors with angle brackets
%% \makeatletter
%% \renewcommand{\vector}[2][ , ]{\left\langle%
%%   \def\nextitem{\def\nextitem{#1}}%
%%   \@for \el:=#2\do{\nextitem\el}\right\rangle%
%% }
%% \makeatother

%% %% Vertical Vectors
%% \def\vector#1{\begin{bmatrix}\vecListA#1,,\end{bmatrix}}
%% \def\vecListA#1,{\if,#1,\else #1\cr \expandafter \vecListA \fi}

%%%%%%%%%%%%%
%% End of vectors
%%%%%%%%%%%%%

%\newcommand{\fullwidth}{}
%\newcommand{\normalwidth}{}



%% makes a snazzy t-chart for evaluating functions
%\newenvironment{tchart}{\rowcolors{2}{}{background!90!textColor}\array}{\endarray}

%%This is to help with formatting on future title pages.
\newenvironment{sectionOutcomes}{}{}



%% Flowchart stuff
%\tikzstyle{startstop} = [rectangle, rounded corners, minimum width=3cm, minimum height=1cm,text centered, draw=black]
%\tikzstyle{question} = [rectangle, minimum width=3cm, minimum height=1cm, text centered, draw=black]
%\tikzstyle{decision} = [trapezium, trapezium left angle=70, trapezium right angle=110, minimum width=3cm, minimum height=1cm, text centered, draw=black]
%\tikzstyle{question} = [rectangle, rounded corners, minimum width=3cm, minimum height=1cm,text centered, draw=black]
%\tikzstyle{process} = [rectangle, minimum width=3cm, minimum height=1cm, text centered, draw=black]
%\tikzstyle{decision} = [trapezium, trapezium left angle=70, trapezium right angle=110, minimum width=3cm, minimum height=1cm, text centered, draw=black]


\title{Change}

\begin{document}

\begin{abstract}
increasing and decreasing
\end{abstract}
\maketitle



We use functions to compare information(measurements). One aspect we are interest in is how one measurement changes compared to changes in the other. 


\begin{itemize}
\item When the supply of pineapples goes up does the price go down?
\item When medicine dosage goes up does the pain go down?
\item When the speed limit goes up does the number of accidents go up?
\end{itemize}


Our words for this comparison are \textbf{\textcolor{purple!85!blue}{increase}} and \textbf{\textcolor{purple!85!blue}{decrease}}.


\begin{itemize}
\item If the price of pineapples goes down when the supply of pineapples goes up, then we say the price is decreasing with respect to the supply. \\
\item If the pain goes down when the dosage goes up, then we say that pain decreases with respect to dosage. \\
\item If the number of accidents goes up when the speed limit goes up, then we say that accidents increases with respect to speed limit.
\end{itemize}




\begin{definition} \textbf{\textcolor{green!50!black}{Increase on a Set}} \\

The function $f$ increases on the set $S$, if for EVERY pair of numbers $a < b \in S$, we have  $f(a) < f(b)$. \\

When the domain values increase, the function values also increase.  When the domain values decrease, the function values also decrease.  \\


\textbf{\textcolor{purple!85!blue}{The domain and range change in the same way.}}



\end{definition}


\textbf{Note:} You cannot test the endpoints of an interval to show a function is increasing.  It has to increase for EVERY pair of numbers in the interval.








\begin{definition} \textbf{\textcolor{green!50!black}{Decrease on a Set}} \\

The function $f$ decreases on the set $S$, if for EVERY pair of numbers such that $a < b \in S$, we have  $f(a) > f(b)$. \\

When the domain values increase, the function values decrease. When the domain values decrease, the function values increase.  \\



\textbf{\textcolor{purple!85!blue}{The domain and range change oppositely.}}



\end{definition}






\textbf{Note:} You cannot test the endpoints of an interval to show a function is decreasing.  It has to decrease for EVERY pair of numbers in the interval. \\










\begin{warning} \textbf{\textcolor{red!70!black}{Language}}

The words \textbf{increase} and \textbf{decrease} are used in two different ways. \\


\textbf{\textcolor{blue!55!black}{$\blacktriangleright$ A List of Numbers:}} The domain and range of a function are essentially a list of numbers. When we present a list of numbers, we line them up one after the other and that gives them a natural ordering.  Each number sits at a position before or after other numbers. \\

As we step through the positions, the values of the numbers get greater, or get lesser, or stay the same. \\ 

\begin{itemize}
    \item If the value of the numbers get greater as we step through the positions of the numbers, then we say the list (sequence) of numbers \textbf{increases}.  
    \item If the value of the numbers get lesser as we step through the positions of the numbers, then we say the list (sequence) of numbers \textbf{decreases}. 
\end{itemize}


For lists of numbers (sequences), we use the words increase and decrease as a comparison between the change in the values of the numbers compared to the change in their position in the list. \\


We can extend this idea to a comparison of function values and domain numbers, essentially two lists. \\

\textbf{\textcolor{blue!55!black}{$\blacktriangleright$ A Function:}} 

A function is also kind of like a list of numbers with positions.  The domain numbers play the role of positions and the function values play the role of list numbers sitting in those positions. \\

We compare the change in the function numbers with the change in the domain numbers. \\

\begin{itemize}
    \item If the function values get greater as we step through the domain numbers, then we say the function \textbf{increases}.  
    \item If the function values get lesser as we step through the domain numbers, then we say the function \textbf{decreases}.  
\end{itemize}

We have a tendency to read left to right with graphs, but that is only half the story.  Everything can be read right to left. \\


\begin{itemize}
    \item If the function values get greater as we step backwards through the domain numbers, then we say the function \textbf{decreases}.  
    \item If the function values get lesser as we step backwards through the domain numbers, then we say the function \textbf{increases}.  
\end{itemize}


This tells us that increasing and decreasing are two-way comparisons.

\begin{itemize}
    \item \textbf{\textcolor{purple!85!blue}{If the function values and domain numbers change in the same way (both get greater or lesser), then we say the function \textbf{increases}}}.  
    \item \textbf{\textcolor{purple!85!blue}{If the function values and domain numbers change in opposite ways (one greater and one lesser), then we say the function \textbf{decreases}}}. 
\end{itemize}






\end{warning}















Graphically, increasing appears as dots to the right being higher than dots to the left.  Decreasing appears as dots on the curve being lower to the right. \\


Or, the reverse....


Increasing also appears as dots to the left being lower than dots to the right.  \\

Decreasing also appears as dots on the curve being higher to the left. \\








\begin{example} Increasing and Decreasing \\

Let $g(k)$ be a function.  The graph of $y= g(k)$ is displayed below. 

\begin{image}
\begin{tikzpicture}
     \begin{axis}[
            	domain=-10:10, ymax=10, xmax=10, ymin=-10, xmin=-10,
            	axis lines =center, xlabel=$k$, ylabel=$y$,
                ytick={-10,-8,-6,-4,-2,2,4,6,8,10},
                xtick={-10,-8,-6,-4,-2,2,4,6,8,10},
                ticklabel style={font=\scriptsize},
            	every axis y label/.style={at=(current axis.above origin),anchor=south},
            	every axis x label/.style={at=(current axis.right of origin),anchor=west},
            	axis on top,
          		]

        
        \addplot [draw=penColor, very thick, smooth, domain=(-8:-3)] {-x};
        \addplot [draw=penColor, very thick, smooth, domain=(-1:3)] {0.5*x-1};
        \addplot [draw=penColor, very thick, smooth, domain=(4:8)] {-2*x+10};


        \addplot[color=penColor,fill=white,only marks,mark=*] coordinates{(-8,8)};
        \addplot[color=penColor,fill=white,only marks,mark=*] coordinates{(-3,3)};

        \addplot[color=penColor,fill=penColor,only marks,mark=*] coordinates{(-1,-1.5)};
        \addplot[color=penColor,fill=white,only marks,mark=*] coordinates{(3,0.5)};

        \addplot[color=penColor,fill=penColor,only marks,mark=*] coordinates{(4,2)};
        \addplot[color=penColor,fill=white,only marks,mark=*] coordinates{(8,-6)};

    \end{axis}
\end{tikzpicture}
\end{image}

\begin{itemize}
\item $g$ is \wordChoice{\choice{increasing} \choice[correct]{decreasing} \choice{neither}} on the interval $(-8,-3)$.  
\item $g$ is \wordChoice{\choice[correct]{increasing} \choice{decreasing} \choice{neither}} on the interval $[-1,3)$. 
\item $g$ is \wordChoice{\choice{increasing} \choice[correct]{decreasing} \choice{neither}} on the interval $[4, 8)$. 
\end{itemize}
\end{example}


\begin{question}
In the example above, $g$ is decreasing on $(-8,-3)$.  Is $g$ decreasing on any and every subinterval of $(-8,-3)$. 
\begin{multipleChoice}
\choice [correct]{Yes}
\choice {No}
\end{multipleChoice}
\end{question}








\begin{example} Decreasing \\

Let $m(t)$ be a function.  The graph of $y = m(t)$ is displayed below. 

\begin{image}
\begin{tikzpicture}
     \begin{axis}[
            	domain=-10:10, ymax=10, xmax=10, ymin=-10, xmin=-10,
            	axis lines =center, xlabel=$t$, ylabel=$y$,
                ytick={-10,-8,-6,-4,-2,2,4,6,8,10},
                xtick={-10,-8,-6,-4,-2,2,4,6,8,10},
                ticklabel style={font=\scriptsize},
            	every axis y label/.style={at=(current axis.above origin),anchor=south},
            	every axis x label/.style={at=(current axis.right of origin),anchor=west},
            	axis on top,
          		]

        
        \addplot [draw=penColor, very thick, smooth, domain=(-8:-3)] {-x};
        %\addplot [draw=penColor, very thick, smooth, domain=(-1:3)] {0.5*x-1};
        \addplot [draw=penColor, very thick, smooth, domain=(-3:8)] {-x+3};


        \addplot[color=penColor,fill=white,only marks,mark=*] coordinates{(-8,8)};
        \addplot[color=penColor,fill=white,only marks,mark=*] coordinates{(-3,3)};

        %\addplot[color=penColor,fill=penColor,only marks,mark=*] coordinates{(-1,-1.5)};
        %\addplot[color=penColor,fill=white,only marks,mark=*] coordinates{(3,0.5)};

        \addplot[color=penColor,fill=penColor,only marks,mark=*] coordinates{(-3,6)};
        \addplot[color=penColor,fill=white,only marks,mark=*] coordinates{(8,-5)};

    \end{axis}
\end{tikzpicture}
\end{image}


\begin{itemize}
\item $m$ is \wordChoice{\choice{increasing} \choice[correct]{decreasing} \choice{neither}} on the interval $(-8,-3)$.  
\item $m$ is \wordChoice{\choice{increasing} \choice[correct]{decreasing} \choice{neither}} on the interval $[-3, 8)$. 
\end{itemize}



However, $m$ is not decreasing on the interval $(-8, 8)$.  To show that a function is not decreasing on an interval, we merely have to give ONE \textbf{counterexample}. That is we need to come up with ONE pair of numbers $a < b$ in the interval where $f(a) \ngtr f(b)$.  Let's select $a = -4$ and $b = -3$.

\[ -4 < -3    \text{ but, }   f(-4) = 4  \ngtr f(-3) = 6 \]




\end{example}












\begin{idea} \textbf{\textcolor{green!50!black}{Counterexample}} \\

A \textbf{counterexample} is one example where all of the conditions of a statement are met, yet the conclusion is not true.  \\

One single counterexample shows a statement to be false.


\end{idea}





\begin{example} Decreasing \\

Let $N(z)$ be a function.  The graph of $y = N(z)$ is displayed below. 

\begin{image}
\begin{tikzpicture}
     \begin{axis}[
            	domain=-10:10, ymax=10, xmax=10, ymin=-10, xmin=-10,
            	axis lines =center, xlabel=$z$, ylabel=$y$,
                xtick={-10,-8,-6,-4,-2,2,4,6,8,10},
                xticklabels={$-10$,$-8$,$-6$,$-4$,$-2$,$2$,$4$,$6$,$8$,$10$},
                ticklabel style={font=\scriptsize},
            	every axis y label/.style={at=(current axis.above origin),anchor=south},
            	every axis x label/.style={at=(current axis.right of origin),anchor=west},
            	axis on top,
          		]

        
        \addplot [draw=penColor, very thick, smooth, domain=(-8:-3), <-] {(x+7)*(x+2)};
        \addplot [draw=penColor, very thick, smooth, domain=(-3:2)] {-x};
        \addplot [draw=penColor, very thick, smooth, domain=(2:8)] {-0.25*x-4.5};


        \addplot[color=penColor,fill=white,only marks,mark=*] coordinates{(-3,-4)};
        \addplot[color=penColor,fill=white,only marks,mark=*] coordinates{(-3,3)};
        \addplot[color=penColor,fill=penColor,only marks,mark=*] coordinates{(-3,5)};
       

        \addplot[color=penColor,fill=penColor,only marks,mark=*] coordinates{(2,-5)};
        \addplot[color=penColor,fill=white,only marks,mark=*] coordinates{(2,-2)};
        \addplot[color=penColor,fill=penColor,only marks,mark=*] coordinates{(8,-8)};
        \addplot[color=penColor,fill=white,only marks,mark=*] coordinates{(8,-6.5)};

    \end{axis}
\end{tikzpicture}
\end{image}







\begin{question}
Select all of the correct statements
\begin{selectAll}
    \choice [correct]{$N$ is decreasing on the interval $(-3,2)$}
    \choice [correct]{$N$ is decreasing on the interval $(2,8)$}
    \choice [correct]{$N$ is decreasing on the interval $(-3,8)$}
    \choice [correct]{$N$ is decreasing on the interval $[-3,8)$}
    \choice [correct]{$N$ is decreasing on the interval $(-3,8]$}
    \choice [correct]{$N$ is decreasing on the interval $[-3,8]$}
\end{selectAll}



\end{question}




\end{example}
































\subsection*{Measuring Rates}

Comparing how the connected values in the domain and range change is called a \textbf{rate}. Our symbol for a change in an amount is a Greek uppercase delta, $\Delta$.  And, we use fractions to quantify the comparison of changes.

\[ rate = \frac{\Delta \, pineapples}{\Delta \, price}\]




\begin{itemize}
\item If the changes in pineapples and price are both positive, then this rate is positive.
\item If the changes in pineapples and price are both negative, then this rate is again positive.
\item If the changes in pineapples and price are of different sign, one increases while the other decreases, then this rate is negative.
\end{itemize}


Using rates, we can compare and measure changes in function values over a domain interval.


\begin{definition}  \textbf{\textcolor{green!50!black}{Rate of Change}} \\
The \textbf{rate-of-change} of a function $f$ across $[a, b]$ in the domain is given by

\[  \frac{f(b) - f(a)}{b - a}  \]



This is also referred to as the average rate of change across the interval.

\end{definition}

The rate-of-change measures how the function values change relative to changes in the domain.









\begin{example} Rates \\

Let $V(x)$ be a function.  The graph of $y = V(x)$ is displayed below. 

\begin{image}
\begin{tikzpicture}
     \begin{axis}[
            	domain=-10:10, ymax=10, xmax=10, ymin=-10, xmin=-10,
            	axis lines =center, xlabel=$x$, ylabel=$y$,
                ytick={-10,-8,-6,-4,-2,2,4,6,8,10},
                xtick={-10,-8,-6,-4,-2,2,4,6,8,10},
                ticklabel style={font=\scriptsize},
            	every axis y label/.style={at=(current axis.above origin),anchor=south},
            	every axis x label/.style={at=(current axis.right of origin),anchor=west},
            	axis on top,
          		]

        
        \addplot [draw=penColor, very thick, smooth, domain=(-8:-3), <-] {(x+7)*(x+2)};
        \addplot [draw=penColor, very thick, smooth, domain=(-3:2)] {-x};
        \addplot [draw=penColor, very thick, smooth, domain=(2:8)] {-0.25*x-4.5};


        \addplot[color=penColor,fill=white,only marks,mark=*] coordinates{(-3,-4)};
        \addplot[color=penColor,fill=white,only marks,mark=*] coordinates{(-3,3)};
        \addplot[color=penColor,fill=penColor,only marks,mark=*] coordinates{(-3,5)};
       

        \addplot[color=penColor,fill=penColor,only marks,mark=*] coordinates{(2,-5)};
        \addplot[color=penColor,fill=white,only marks,mark=*] coordinates{(2,-2)};
        \addplot[color=penColor,fill=penColor,only marks,mark=*] coordinates{(8,-8)};
        \addplot[color=penColor,fill=white,only marks,mark=*] coordinates{(8,-6.5)};

    \end{axis}
\end{tikzpicture}
\end{image}



From the graph, we can estimate that over the interval $[-7, -5]$, $V$ had a rate-of-change of $\frac{-6 - 0}{-5 - (-7)} = \frac{-6}{2} = -3$.


From the graph, we can estimate that over the interval $[1, 8]$, $V$ had a rate-of-change of $\frac{-8 - (-1)}{8 - 1} = \frac{-7}{7} = -1$.



\end{example}
























All of this might sound familiar. You have probably seen this when working with slopes of lines, $\frac{rise}{run}$.  The \textit{rise} is the vertical change (change in the value of the function). The \textit{run} is the horizontal change (change in the domain).

$\blacktriangleright$ Rate-of-change is a function idea that we will connect up to the geometric idea of slope.





The rate of change over an interval is an overall measurement.  It compares the changes occuring over the whole interval. It compares final values to initial values. 











\begin{example} 



Let $M(t)$ be a function.  The graph of $y = M(t)$ is displayed below. 

\begin{image}
\begin{tikzpicture}
     \begin{axis}[
                domain=-10:10, ymax=10, xmax=10, ymin=-10, xmin=-10,
                axis lines =center, xlabel=$t$, ylabel=$y$,
                ytick={-10,-8,-6,-4,-2,2,4,6,8,10},
                xtick={-10,-8,-6,-4,-2,2,4,6,8,10},
                ticklabel style={font=\scriptsize},
                every axis y label/.style={at=(current axis.above origin),anchor=south},
                every axis x label/.style={at=(current axis.right of origin),anchor=west},
                axis on top,
                ]

        
        \addplot [draw=penColor, very thick, smooth, domain=(-8:-3), <-] {(x+8)*(x+2)+2};
        \addplot [draw=penColor, very thick, smooth, domain=(-3:2)] {-x};
        \addplot [draw=penColor, very thick, smooth, domain=(2:8)] {0.25*x + 1};


        \addplot[color=penColor,fill=white,only marks,mark=*] coordinates{(-3,-3)};
        \addplot[color=penColor,fill=white,only marks,mark=*] coordinates{(-3,3)};
        \addplot[color=penColor,fill=penColor,only marks,mark=*] coordinates{(-3,5)};
       

        \addplot[color=penColor,fill=penColor,only marks,mark=*] coordinates{(2,1.5)};
        \addplot[color=penColor,fill=white,only marks,mark=*] coordinates{(2,-2)};
        \addplot[color=penColor,fill=penColor,only marks,mark=*] coordinates{(8,-1)};
        \addplot[color=penColor,fill=white,only marks,mark=*] coordinates{(8,3)};

    \end{axis}
\end{tikzpicture}
\end{image}


\begin{question}
The rate of change of $M$ over the interval $[-5,5]$ is
    \begin{multipleChoice}
        \choice [correct]{positive}
        \choice {negative}
    \end{multipleChoice}
\end{question}



\begin{question}
The rate of change of $M$ over the interval $[-2,1]$ is
    \begin{multipleChoice}
        \choice {positive}
        \choice [correct]{negative}
    \end{multipleChoice}
\end{question}


\end{example}




\begin{question}
If the rate of change of a function is positive over the interval $[a,b]$ then the rate of change over any subinterval must also be positive.
    \begin{multipleChoice}
        \choice {True}
        \choice [correct]{False}
    \end{multipleChoice}
\end{question}



\subsection*{Constant Rate of Change}


Linear functions are functions that can be represented with formulas of the form $L(x) = A \, x + B$. \\


Given an interval, we can measure the change of al inear function over this interval. \\












\begin{example} 



Let $L(x)$ be the linear function $L(x) = \frac{1}{2} x - 4$.  The graph of $y = L(x)$ is displayed below. 

\begin{image}
\begin{tikzpicture}
     \begin{axis}[
                domain=-10:10, ymax=10, xmax=10, ymin=-10, xmin=-10,
                axis lines =center, xlabel=$x$, ylabel=$y$,
                ytick={-10,-8,-6,-4,-2,2,4,6,8,10},
                xtick={-10,-8,-6,-4,-2,2,4,6,8,10},
                ticklabel style={font=\scriptsize},
                every axis y label/.style={at=(current axis.above origin),anchor=south},
                every axis x label/.style={at=(current axis.right of origin),anchor=west},
                axis on top,
                ]


        \addplot [draw=penColor, very thick, smooth, domain=(-8:8), <->] {0.5*x - 4};



    \end{axis}
\end{tikzpicture}
\end{image}




The rate of change of $L$ over the interval $[-6, 0]$ is

\[
\frac{L(0) - L(-6)}{0 - (-6)} = \frac{-4 - (-7)}{6} = \frac{3}{6} = \frac{1}{2}
\]





The rate of change of $L$ over the interval $[-4, 6]$ is

\[
\frac{L(6) - L(-4)}{6 - (-4)} = \frac{-1 - (-6)}{10} = \frac{5}{10} = \frac{1}{2}
\]






The rate of change of $L$ over the interval $[-2, 2]$ is

\[
\frac{L(2) - L(-2)}{2 - (-2)} = \frac{-3 - (-5)}{4} = \frac{2}{4} = \frac{1}{2}
\]




The rate of change of $L$ over the interval $[8, 10]$ is

\[
\frac{L(10) - L(8)}{10 - 8} = \frac{1 - 0}{2} = \frac{1}{2}
\]




\end{example}



Linear functions have a special property that other function do not have.  No matter what interval you select, the rate of change over that interval is always the same. \\



Let $L$ be the linear function in the example above, $L(x) = \frac{1}{2} x - 4$. 

Let $a$ and $b$ be any distinct real numbers. \\

The rate of change of $L$ over the interval $[a, b]$ is


\[
\frac{L(b) - L(a)}{b - a} = \frac{\left( \frac{b}{2} - 4 \right) - \left( \frac{a}{2} - 4 \right)}{b-a} 
= \frac{\left( \frac{b - a}{2} \right)}{b-a} = \frac{1}{2}
\]



\textbf{\textcolor{blue!55!black}{$\blacktriangleright$}} Linear functions have a constant rate of change, which measures the constant slope of the corresponding line. \\



We will use this fact to help us understand how function values change.



\textbf{\textcolor{red!70!black}{A Peek Ahead...}}



Later in Calculus, we will more closely examine sequences.  We will describe and represent them as functions.  Their domains will be the natural numbers, rather than the real numbers. \\


That will bring all of our versions of increasing and decreasing together under one umbrella.
































\begin{onlineOnly}
\begin{center}
\textbf{\textcolor{green!50!black}{ooooo-=-=-=-ooOoo-=-=-=-ooooo}} \\

more examples can be found by following this link\\ \link[More Examples of Visual Behavior]{https://ximera.osu.edu/csccmathematics/precalculus/precalculus/visualBehavior/examples/exampleList}

\end{center}
\end{onlineOnly}








\end{document}
