\documentclass{ximera}


\graphicspath{
  {./}
  {ximeraTutorial/}
  {basicPhilosophy/}
}

\newcommand{\mooculus}{\textsf{\textbf{MOOC}\textnormal{\textsf{ULUS}}}}


\usepackage{tkz-euclide}\usepackage{tikz}
\usepackage{tikz-cd}
\usetikzlibrary{arrows}
\tikzset{>=stealth,commutative diagrams/.cd,
  arrow style=tikz,diagrams={>=stealth}} %% cool arrow head
\tikzset{shorten <>/.style={ shorten >=#1, shorten <=#1 } } %% allows shorter vectors

\usetikzlibrary{backgrounds} %% for boxes around graphs
\usetikzlibrary{shapes,positioning}  %% Clouds and stars
\usetikzlibrary{matrix} %% for matrix
\usepgfplotslibrary{polar} %% for polar plots
\usepgfplotslibrary{fillbetween} %% to shade area between curves in TikZ
\usetkzobj{all}
\usepackage[makeroom]{cancel} %% for strike outs
%\usepackage{mathtools} %% for pretty underbrace % Breaks Ximera
%\usepackage{multicol}
\usepackage{pgffor} %% required for integral for loops



%% http://tex.stackexchange.com/questions/66490/drawing-a-tikz-arc-specifying-the-center
%% Draws beach ball
\tikzset{pics/carc/.style args={#1:#2:#3}{code={\draw[pic actions] (#1:#3) arc(#1:#2:#3);}}}



\usepackage{array}
\setlength{\extrarowheight}{+.1cm}
\newdimen\digitwidth
\settowidth\digitwidth{9}
\def\divrule#1#2{
\noalign{\moveright#1\digitwidth
\vbox{\hrule width#2\digitwidth}}}
























%%This is to help with formatting on future title pages.
\newenvironment{sectionOutcomes}{}{}


\title{As Input}

\begin{document}

\begin{abstract}
composition
\end{abstract}
\maketitle





\begin{example}



Analyze $f(t) = 3 \sin(2t - \pi) - 2$


\textbf{\textcolor{red!75!green}{explanation}} 


$\blacktriangleright$ \textbf{One period} 


Start:  $2t - \pi = 0$, which occurs at $t = \frac{\pi}{2}$where $f = 3 \cdot 0 - 2 = -2$  


Finish:  $2t - \pi = 2 \pi$, which occurs at $t = \frac{3 \pi}{2}$, where $f = 3 \cdot 0 - 2 = -2$ 


Period = $\frac{3 \pi}{2} - \frac{\pi}{2} = \pi$


Halfway between $\frac{\pi}{2}$ and $\frac{3\pi}{2}$ is $\pi$.  


$\sin(2t - \pi)$ equals $0$ at $\pi$, where $f = 3 \cdot 0 - 2 = -2$.


A quarter of the way from Start to Finish is $\frac{3 \pi}{4}$, where $f = 3 \cdot 1 - 2 = 1$


The three-quarter mark is $\frac{5 \pi}{4}$, where $f = 3 \cdot -1 - 2 = -5$





\begin{image}
\begin{tikzpicture}
  \begin{axis}[
            domain=0:7, ymax=6, xmax=7, ymin=-6, xmin=0,
            axis lines =center, xlabel={t}, ylabel={$y$}, grid = major, grid style={dashed},
            ytick={-5,-4,-3,-2,-1,0,1,2,3,4,5},
            xtick={0, 1.57, 3.142, 4.71, 6.28},
            xticklabels={$0$, $\tfrac{\pi}{2}$, $\pi$, $\tfrac{3\pi}{2}$, $2\pi$},
            yticklabels={$-5$,$-4$,$-3$,$-2$,$-1$,$0$,$1$,$2$,$3$,$4$,$5$}, 
            ticklabel style={font=\scriptsize},
            every axis y label/.style={at=(current axis.above origin),anchor=south},
            every axis x label/.style={at=(current axis.right of origin),anchor=west},
            axis on top
          ]
          

            \addplot [line width=2, penColor, smooth,samples=300,domain=(1.57:4.71)] {3*sin(deg(2*x-3.14))-2};
            %\addplot [line width=2, penColor2, smooth,samples=300,domain=(-10:10),<->] {cos(deg(x)};


            %\node[penColor] at (axis cs:3.5,1.1) [anchor=north] {$\sin(\theta)$};
            %\node[penColor2] at (axis cs:1.4,1.3) [anchor=north] {$\cos(\theta)$};



  \end{axis}
\end{tikzpicture}
\end{image}



$\blacktriangleright$ \textbf{Then, extend periodically} 








\begin{image}
\begin{tikzpicture}
  \begin{axis}[
            domain=0:7, ymax=6, xmax=7, ymin=-6, xmin=0,
            axis lines =center, xlabel={t}, ylabel={$y$}, grid = major, grid style={dashed},
            ytick={-5,-4,-3,-2,-1,0,1,2,3,4,5},
            xtick={0, 1.57, 3.142, 4.71, 6.28},
            xticklabels={$0$, $\tfrac{\pi}{2}$, $\pi$, $\tfrac{3\pi}{2}$, $2\pi$},
            yticklabels={$-5$,$-4$,$-3$,$-2$,$-1$,$0$,$1$,$2$,$3$,$4$,$5$}, 
            ticklabel style={font=\scriptsize},
            every axis y label/.style={at=(current axis.above origin),anchor=south},
            every axis x label/.style={at=(current axis.right of origin),anchor=west},
            axis on top
          ]
          

            \addplot [line width=2, penColor, smooth,samples=300,domain=(0.2:6.5),<->] {3*sin(deg(2*x-3.14)) -2};
            %\addplot [line width=2, penColor2, smooth,samples=300,domain=(-10:10),<->] {cos(deg(x)};


            %\node[penColor] at (axis cs:3.5,1.1) [anchor=north] {$\sin(\theta)$};
            %\node[penColor2] at (axis cs:1.4,1.3) [anchor=north] {$\cos(\theta)$};



  \end{axis}
\end{tikzpicture}
\end{image}




We can calculate some additional exact values at the easy angles. \\



Our principle interval was $\left[ \frac{\pi}{2}, \frac{3\pi}{2}   \right]$, which has a length of $\pi$. 





$\blacktriangleright$ $30^{\circ} = \frac{\pi}{6}$ is $\frac{1}{12}$ of the normal basic interval $[0, 2\pi]$.  


$\frac{1}{12}$ of our period here would be $\frac{1}{12} \cdot \pi = \frac{\pi}{12}$.

Therefore, if we move $\frac{\pi}{12}$ from $\frac{\pi}{2}$, then we should be at the angle where we normally get $\sin(\frac{\pi}{6}) = \frac{1}{2}$.


$\frac{\pi}{2} + \frac{\pi}{12}  = \frac{7\pi}{12}$.

Let's evaluate $f\left( \frac{7\pi}{12} \right)$ and see if we get $\frac{\pi}{6}$ inside the sine function.

\[
f\left( \frac{7\pi}{12} \right) = 3 \sin\left(2 \left(\frac{7\pi}{12} \right) - \pi \right) - 2 = 3 \sin\left( \frac{2\pi}{12} \right) - 2  = 3 \sin\left( \frac{\pi}{6} \right) - 2 = 3 \cdot \frac{1}{2} - 2 = -\frac{1}{2}
\]






\begin{image}
\begin{tikzpicture}
  \begin{axis}[
            domain=0:7, ymax=6, xmax=7, ymin=-6, xmin=0,
            axis lines =center, xlabel={$\theta$}, ylabel={$y$}, grid = major, grid style={dashed},
            ytick={-5,-4,-3,-2,-1,0,1,2,3,4,5},
            xtick={0, 1.57, 3.142, 4.71, 6.28},
            xticklabels={$0$, $\tfrac{\pi}{2}$, $\pi$, $\tfrac{3\pi}{2}$, $2\pi$},
            yticklabels={$-5$,$-4$,$-3$,$-2$,$-1$,$0$,$1$,$2$,$3$,$4$,$5$}, 
            ticklabel style={font=\scriptsize},
            every axis y label/.style={at=(current axis.above origin),anchor=south},
            every axis x label/.style={at=(current axis.right of origin),anchor=west},
            axis on top
          ]
          

            \addplot [line width=2, penColor, smooth,samples=300,domain=(1.57:4.71)] {3*sin(deg(2*x-3.14))-2};
            %\addplot [line width=2, penColor2, smooth,samples=300,domain=(-10:10),<->] {cos(deg(x)};


            %\node[penColor] at (axis cs:3.5,1.1) [anchor=north] {$\sin(\theta)$};
            %\node[penColor2] at (axis cs:1.4,1.3) [anchor=north] {$\cos(\theta)$};

            \addplot[color=penColor,fill=penColor,only marks,mark=*] coordinates{(1.833,-0.5)};



  \end{axis}
\end{tikzpicture}
\end{image}









$\blacktriangleright$ $60^{\circ} = \frac{\pi}{3}$ is $\frac{1}{6}$ of the normal basic interval $[0, 2\pi]$.  

$\frac{1}{6}$ of our period here would be $\frac{1}{6} \cdot \pi = \frac{\pi}{6}$.

Therefore, if we move $\frac{\pi}{6}$ from $\frac{\pi}{2}$, then we should be at the angle where we get $\sin(\frac{\pi}{3}) = \frac{\sqrt{3}}{2}$.


$\frac{\pi}{2} + \frac{\pi}{6} = \frac{4\pi}{6} =  \frac{2\pi}{3}$.

Let's evaluate $f\left( \frac{2\pi}{3} \right)$ and see if we get $\frac{\pi}{3}$ inside the sine. \\

\[
f\left( \frac{2\pi}{3} \right) = 3 \sin\left(2 \left(\frac{2\pi}{3} \right) - \pi \right) - 2 = 3 \sin\left( \frac{\pi}{3} \right) - 2  =  3 \cdot \frac{\sqrt{3}}{2} - 2 = \frac{3\sqrt{3}-4}{2} \approx 0.5980762114
\]










\begin{image}
\begin{tikzpicture}
  \begin{axis}[
            domain=0:7, ymax=6, xmax=7, ymin=-6, xmin=0,
            axis lines =center, xlabel={t}, ylabel={$y$}, grid = major, grid style={dashed},
            ytick={-5,-4,-3,-2,-1,0,1,2,3,4,5},
            xtick={0, 1.57, 3.142, 4.71, 6.28},
            xticklabels={$0$, $\tfrac{\pi}{2}$, $\pi$, $\tfrac{3\pi}{2}$, $2\pi$},
            yticklabels={$-5$,$-4$,$-3$,$-2$,$-1$,$0$,$1$,$2$,$3$,$4$,$5$}, 
            ticklabel style={font=\scriptsize},
            every axis y label/.style={at=(current axis.above origin),anchor=south},
            every axis x label/.style={at=(current axis.right of origin),anchor=west},
            axis on top
          ]
          

            \addplot [line width=2, penColor, smooth,samples=300,domain=(1.57:4.71)] {3*sin(deg(2*x-3.14))-2};
            %\addplot [line width=2, penColor2, smooth,samples=300,domain=(-10:10),<->] {cos(deg(x)};


            %\node[penColor] at (axis cs:3.5,1.1) [anchor=north] {$\sin(\theta)$};
            %\node[penColor2] at (axis cs:1.4,1.3) [anchor=north] {$\cos(\theta)$};

            \addplot[color=penColor,fill=penColor,only marks,mark=*] coordinates{(1.833,-0.5)};
            \addplot[color=penColor,fill=penColor,only marks,mark=*] coordinates{(2.094,0.59)};



  \end{axis}
\end{tikzpicture}
\end{image}










\end{example}






\begin{summary} Sine and Cosine 


Sine and Cosine are basically shifts of each other.  They follow the same periodic patterns. 

They both oscillate between $-1$ and $1$, which makes their range $[-1,1]$. 

They both have their maximum value of $1$ and their minimum value of $-1$ at the top, bottom, or sides of the unit circle. 

When one has maximum or minimum, the other has a zero. 

\end{summary}


When we analyze sine and cosine functions, we usually start by locating when the inside is $0$, $\frac{\pi}{2}$, $\pi$, $\frac{3\pi}{2}$, or $2\pi$.  This tells us when the function is experiencing a maximum, minimum, or zero.

Then we move to the easy angles ($\frac{\pi}{6}$, $\frac{\pi}{4}$, and $\frac{\pi}{3}$) to round out the shape. 





\textbf{\textcolor{red!80!black}{$\blacktriangleright$}} This is how we analyze compositions involving sine and cosine.





























































Now that we can move around the circle and obtain values of sine and cosine, we can use these function values as input to additional functions, via composition. \\




Let's examine 
\[ g(x) = \frac{1}{2} (\sin(x) - 3)^2 - 4 \]


We can view this as a composition: $\sin(x)$ has been composed with $P(t) = \frac{1}{2} (t - 3)^2 - 4$.



$P(t) = \frac{1}{2} (t - 3)^2 - 4$ is a quadratic function whose graph is a parabola.






\begin{image}
\begin{tikzpicture}
  \begin{axis}[
            domain=-10:10, ymax=10, xmax=10, ymin=-10, xmin=-10,
            axis lines =center, xlabel=$t$, ylabel={$y=P(t)$}, grid = major, grid style={dashed},
            ytick={-10,-8,-6,-4,-2,2,4,6,8,10},
            xtick={-10,-8,-6,-4,-2,2,4,6,8,10},
            yticklabels={$-10$,$-8$,$-6$,$-4$,$-2$,$2$,$4$,$6$,$8$,$10$}, 
            xticklabels={$-10$,$-8$,$-6$,$-4$,$-2$,$2$,$4$,$6$,$8$,$10$},
            ticklabel style={font=\scriptsize},
            every axis y label/.style={at=(current axis.above origin),anchor=south},
            every axis x label/.style={at=(current axis.right of origin),anchor=west},
            axis on top
          ]
          

			\addplot [line width=2, penColor, smooth,samples=200,domain=(-2.2:8.2),<->] {0.5*(x-3)^2 - 4};

          	%\addplot[color=penColor,fill=penColor2,only marks,mark=*] coordinates{(-6,9)};
          	%\addplot[color=penColor,fill=penColor2,only marks,mark=*] coordinates{(2,-7)};


  \end{axis}
\end{tikzpicture}
\end{image}



We are replacing $t$ with $\sin(x)$ to get $g(x) = \frac{1}{2} (\sin(x) - 3)^2 - 4$.

$\sin(x)$ has values from $-1$ to $1$ and those values will be going in for $t$ into $P(t) = \frac{1}{2} (t - 3)^2 - 4$. 

Therefore, we want to look at $P(t)$ on the interval $[-1,1]$. 





\begin{image}
\begin{tikzpicture}
  \begin{axis}[
            domain=-10:10, ymax=10, xmax=10, ymin=-10, xmin=-10,
            axis lines =center, xlabel=$t$, ylabel={$y=P(t)$}, grid = major, grid style={dashed},
            ytick={-10,-8,-6,-4,-2,2,4,6,8,10},
            xtick={-10,-8,-6,-4,-2,2,4,6,8,10},
            yticklabels={$-10$,$-8$,$-6$,$-4$,$-2$,$2$,$4$,$6$,$8$,$10$}, 
            xticklabels={$-10$,$-8$,$-6$,$-4$,$-2$,$2$,$4$,$6$,$8$,$10$},
            ticklabel style={font=\scriptsize},
            every axis y label/.style={at=(current axis.above origin),anchor=south},
            every axis x label/.style={at=(current axis.right of origin),anchor=west},
            axis on top
          ]
          

			\addplot [line width=2, penColor, smooth,samples=200,domain=(-1:1)] {0.5*(x-3)^2 - 4};

          	%\addplot[color=penColor,fill=penColor2,only marks,mark=*] coordinates{(-6,9)};
          	%\addplot[color=penColor,fill=penColor2,only marks,mark=*] coordinates{(2,-7)};


  \end{axis}
\end{tikzpicture}
\end{image}


On this interval $P$ is decreasing, which means the maximum occurs on the left and the minimum on the right.

As $\sin(x)$ moves from $-1$ to $1$, $P(t)$ moves from $P(-1)=4$ to $P(1)=-2$. 


To get $\sin(x)$ to move from $-1$ to $1$, $x$ needs to start at $-\frac{\pi}{2}$ and proceed to $\frac{\pi}{2}$.








\begin{image}
\begin{tikzpicture}
  \begin{axis}[
            domain=-10:10, ymax=10, xmax=10, ymin=-10, xmin=-10,
            axis lines =center, xlabel=$x$, ylabel={$y=g(x)$}, grid = major, grid style={dashed},
            ytick={-10,-8,-6,-4,-2,2,4,6,8,10},
            xtick={-10,-8,-6,-4,-2,2,4,6,8,10},
            yticklabels={$-10$,$-8$,$-6$,$-4$,$-2$,$2$,$4$,$6$,$8$,$10$}, 
            xticklabels={$-10$,$-8$,$-6$,$-4$,$-2$,$2$,$4$,$6$,$8$,$10$},
            ticklabel style={font=\scriptsize},
            every axis y label/.style={at=(current axis.above origin),anchor=south},
            every axis x label/.style={at=(current axis.right of origin),anchor=west},
            axis on top
          ]
          

			\addplot [line width=2, penColor, smooth,samples=200,domain=(-1.570:1.570)] {0.5*(sin(deg(x))-3)^2 - 4};

          	%\addplot[color=penColor,fill=penColor2,only marks,mark=*] coordinates{(-6,9)};
          	%\addplot[color=penColor,fill=penColor2,only marks,mark=*] coordinates{(2,-7)};


  \end{axis}
\end{tikzpicture}
\end{image}





As $x$ moves from $\frac{\pi}{2}$ to $\frac{3\pi}{2}$, $\sin(x)$ to moves from $1$ to $-1$. This piece of the parabola is traced backwards as $x$ moves forward.








\begin{image}
\begin{tikzpicture}
  \begin{axis}[
            domain=-10:10, ymax=10, xmax=10, ymin=-10, xmin=-10,
            axis lines =center, xlabel=$x$, ylabel={$y=g(x)$}, grid = major, grid style={dashed},
            ytick={-10,-8,-6,-4,-2,2,4,6,8,10},
            xtick={-10,-8,-6,-4,-2,2,4,6,8,10},
            yticklabels={$-10$,$-8$,$-6$,$-4$,$-2$,$2$,$4$,$6$,$8$,$10$}, 
            xticklabels={$-10$,$-8$,$-6$,$-4$,$-2$,$2$,$4$,$6$,$8$,$10$},
            ticklabel style={font=\scriptsize},
            every axis y label/.style={at=(current axis.above origin),anchor=south},
            every axis x label/.style={at=(current axis.right of origin),anchor=west},
            axis on top
          ]
          

			\addplot [line width=2, penColor, smooth,samples=200,domain=(-1.570:4.71)] {0.5*(sin(deg(x))-3)^2 - 4};

          	%\addplot[color=penColor,fill=penColor2,only marks,mark=*] coordinates{(-6,9)};
          	%\addplot[color=penColor,fill=penColor2,only marks,mark=*] coordinates{(2,-7)};


  \end{axis}
\end{tikzpicture}
\end{image}



And, this repeats.












\begin{image}
\begin{tikzpicture}
  \begin{axis}[
            domain=-10:10, ymax=10, xmax=10, ymin=-10, xmin=-10,
            axis lines =center, xlabel=$x$, ylabel={$y=g(x)$}, grid = major, grid style={dashed},
            ytick={-10,-8,-6,-4,-2,2,4,6,8,10},
            xtick={-10,-8,-6,-4,-2,2,4,6,8,10},
            yticklabels={$-10$,$-8$,$-6$,$-4$,$-2$,$2$,$4$,$6$,$8$,$10$}, 
            xticklabels={$-10$,$-8$,$-6$,$-4$,$-2$,$2$,$4$,$6$,$8$,$10$},
            ticklabel style={font=\scriptsize},
            every axis y label/.style={at=(current axis.above origin),anchor=south},
            every axis x label/.style={at=(current axis.right of origin),anchor=west},
            axis on top
          ]
          

			\addplot [line width=2, penColor, smooth,samples=200,domain=(-9.5:9.5),<->] {0.5*(sin(deg(x))-3)^2 - 4};

          	%\addplot[color=penColor,fill=penColor2,only marks,mark=*] coordinates{(-6,9)};
          	%\addplot[color=penColor,fill=penColor2,only marks,mark=*] coordinates{(2,-7)};


  \end{axis}
\end{tikzpicture}
\end{image}


$g(x)$ has a maximum value whenever $t=-1$ and a minimum value whenever $t=1$. 

This corresponds to when $x=\frac{-\pi}{2} \pm 2k\pi$ and $=\frac{\pi}{2} \pm 2k\pi$. 


$g(x)$ has a maximum value of 


\begin{align*}
g\left( -\frac{\pi}{2} \right) &= \frac{1}{2} \left(\left( -\frac{\pi}{2} \right) - 3 \right)^2 - 4 \\
                               &= \frac{1}{2} (-1 - 3)^2 - 4 \\
                               &= \frac{1}{2} \cdot 16 - 4 \\
                               &=  4
\end{align*}





$g(x)$ has a minimum value of
\begin{align*}
g\left( \frac{\pi}{2} \right) &= \frac{1}{2} \left(\left( \frac{\pi}{2} \right) - 3 \right)^2 - 4 \\
                               &= \frac{1}{2} (1 - 3)^2 - 4 \\
                               &= \frac{1}{2} \cdot 4 - 4 \\
                               &=  -2
\end{align*}



$g$ is decreasing on 
\[
\left[ \frac{-\pi}{2} + 2k\pi, \frac{\pi}{2} + 2k\pi \right], \text{ where } \,  k \in \mathbb{Z}
\]




$g$ is increasing on 
\[
\left[ \frac{\pi}{2} + 2k\pi, \frac{3\pi}{2} + 2k\pi \right], \text{ where } \,  k \in \mathbb{Z}
\]






\begin{center}
\textbf{\textcolor{green!50!black}{ooooo-=-=-=-ooOoo-=-=-=-ooooo}} \\

more examples can be found by following this link\\ \link[More Examples of Trigonometric Functions]{https://ximera.osu.edu/csccmathematics/precalculus/precalculus/trigFunctions/examples/exampleList}

\end{center}





\end{document}

