\documentclass{ximera}


\graphicspath{
  {./}
  {ximeraTutorial/}
  {basicPhilosophy/}
}

\newcommand{\mooculus}{\textsf{\textbf{MOOC}\textnormal{\textsf{ULUS}}}}


\usepackage{tkz-euclide}\usepackage{tikz}
\usepackage{tikz-cd}
\usetikzlibrary{arrows}
\tikzset{>=stealth,commutative diagrams/.cd,
  arrow style=tikz,diagrams={>=stealth}} %% cool arrow head
\tikzset{shorten <>/.style={ shorten >=#1, shorten <=#1 } } %% allows shorter vectors

\usetikzlibrary{backgrounds} %% for boxes around graphs
\usetikzlibrary{shapes,positioning}  %% Clouds and stars
\usetikzlibrary{matrix} %% for matrix
\usepgfplotslibrary{polar} %% for polar plots
\usepgfplotslibrary{fillbetween} %% to shade area between curves in TikZ
\usetkzobj{all}
\usepackage[makeroom]{cancel} %% for strike outs
%\usepackage{mathtools} %% for pretty underbrace % Breaks Ximera
%\usepackage{multicol}
\usepackage{pgffor} %% required for integral for loops



%% http://tex.stackexchange.com/questions/66490/drawing-a-tikz-arc-specifying-the-center
%% Draws beach ball
\tikzset{pics/carc/.style args={#1:#2:#3}{code={\draw[pic actions] (#1:#3) arc(#1:#2:#3);}}}



\usepackage{array}
\setlength{\extrarowheight}{+.1cm}
\newdimen\digitwidth
\settowidth\digitwidth{9}
\def\divrule#1#2{
\noalign{\moveright#1\digitwidth
\vbox{\hrule width#2\digitwidth}}}
























%%This is to help with formatting on future title pages.
\newenvironment{sectionOutcomes}{}{}


\title{Tangent}

\begin{document}

\begin{abstract}
quotient
\end{abstract}
\maketitle











Tangent is the quotient of sine and cosine.

\[   \tan(\theta)  =  \frac{\sin(\theta)}{\cos(\theta)}  \]


\begin{itemize}
\item \textbf{zeros:} $\tan(\theta)$ has a zero everywhere that $\sin(\theta)$ has a zero:  all of the whole-$\pi$'s.
\item \textbf{singluaries:} $\tan(\theta)$ has a singularity everywhere that $\cos(\theta)$ has a zero:  all of the half-$\pi$'s.
\end{itemize}

This means intercepts at the whole-$\pi$'s and vertical asymptotes at the half-$\pi$'s.








\begin{image}
\begin{tikzpicture} 
  \begin{axis}[
            domain=-10:10, ymax=10, xmax=10, ymin=-10, xmin=-10,
            xtick={-4.9, -1.7, 1.7, 4.9}, 
            xticklabels={$\frac{-3\pi}{2}$, $\frac{-\pi}{2}$, $\frac{\pi}{2}$, $\frac{3\pi}{2}$},
            axis lines =center,  xlabel={$\theta$}, ylabel=$y$,
            ticklabel style={font=\scriptsize},
            every axis y label/.style={at=(current axis.above origin),anchor=south},
            every axis x label/.style={at=(current axis.right of origin),anchor=west},
            axis on top
          ]
          
            \addplot [line width=1, gray, dashed,samples=100,domain=(-10:10), <->] ({-4.71},{x});
            \addplot [line width=1, gray, dashed,samples=100,domain=(-10:10), <->] ({-1.57},{x});
            \addplot [line width=1, gray, dashed,samples=100,domain=(-10:10), <->] ({1.57},{x});
            \addplot [line width=1, gray, dashed,samples=100,domain=(-10:10), <->] ({4.71},{x});

            \addplot [line width=2, penColor, smooth,samples=100,domain=(-1.47:1.47), <->] {tan(deg(x))};
            \addplot [line width=2, penColor, smooth,samples=100,domain=(-4.61:-1.67), <->] {tan(deg(x))};
            \addplot [line width=2, penColor, smooth,samples=100,domain=(1.67:4.61), <->] {tan(deg(x))};

      \addplot[color=penColor,fill=penColor,only marks, mark size=1pt, mark=*] coordinates{(-9,5) (-8,5) (-7,5) (7,5) (8,5) (9,5)};


            

           

  \end{axis}
\end{tikzpicture}
\end{image}




\begin{itemize}
\item Tangent is always increasing.
\item Tangent has no maximums or minimum, since it is unbounded near the half-$\pi$'s.
\item The period for tangent is $\pi$.
\end{itemize}



















\textbf{\textcolor{purple!85!blue}{Analyze $K(t) = -\tan\left( \frac{t}{2} + \pi \right)$









\begin{idea}




The usual interval people look at tangent is $\left(  \frac{-\pi}{2}, \frac{\pi}{2} \right )$.  This interval has singularities on both ends, which signal vertical asymptotes on the graph. $\frac{-\pi}{2}$ and $\frac{\pi}{2}$ are what we need the inside of our function to equal.


\begin{align*}
\frac{t}{2} + \pi   & = \frac{-\pi}{2}   \\
\frac{t}{2}   & = \frac{-3\pi}{2}   \\
t             & = \answer{-3 \pi}
\end{align*}





\begin{align*}
\frac{t}{2} + \pi   & = \frac{\pi}{2}   \\
\frac{t}{2}   & = \frac{-\pi}{2}   \\
t             & = \answer{-\pi}
\end{align*}




The period is $\answer{2 \pi}$.  One of the arms of this tangent function sits on the interval $[ -3 \pi, - \pi ]$.



Then the $-1$ coefficient flips the graph upside-down.
















\begin{image}
\begin{tikzpicture} 
  \begin{axis}[
            domain=-15:15, ymax=10, xmax=15, ymin=-10, xmin=-15,
            xtick={-9.4, -3.1, 3.1, 9.4}, 
            xticklabels={$-3\pi$, $-\pi$, $\pi$, $3\pi$},
            axis lines =center,  xlabel={$t$}, ylabel=$y$,
            ticklabel style={font=\scriptsize},
            every axis y label/.style={at=(current axis.above origin),anchor=south},
            every axis x label/.style={at=(current axis.right of origin),anchor=west},
            axis on top
          ]
          
            \addplot [line width=2, penColor, smooth,samples=100,domain=(-9.2:-3.25), <->] {-tan(deg(0.5*x+3.14))};
            %\addplot [line width=2, penColor, smooth,samples=100,domain=(-4.61:-1.67), <->] {tan(deg(x))};
            %\addplot [line width=2, penColor, smooth,samples=100,domain=(1.67:4.61), <->] {tan(deg(x))};

      \addplot[color=penColor,fill=penColor,only marks, mark size=1pt, mark=*] coordinates{(-14,5) (-13,5) (-12,5) (12,5) (13,5) (14,5)};


            \addplot [line width=1, gray, dashed,samples=100,domain=(-10:10), <->] ({-9.4},{x});
            \addplot [line width=1, gray, dashed,samples=100,domain=(-10:10), <->] ({-3.1},{x});
            %\addplot [line width=1, gray, dashed,samples=100,domain=(-10:10), <->] ({1.57},{x});
            %\addplot [line width=1, gray, dashed,samples=100,domain=(-10:10), <->] ({4.71},{x});

           

  \end{axis}
\end{tikzpicture}
\end{image}

There are no maximums or minimums.

$K(t)$ is always decreasing.

The zeros are $\{  2 k \pi \, | \, k \in \mathbb{Z}    \}$


With these ideas, we can write an algebraic analysis.\\


\end{idea}












\begin{template}


We are going to view this as a composition of three functions.


\[
K = Out \circ \tan \circ In
\]


where


$Out(x) = -x$ \\


$In(y) = \frac{y}{2} + \pi$ \\




Along with $\tan(\theta)$, these three component functions, we get



\[
K(t) = (Out \circ \tan \circ In)(t) = Out(tan(In(t)))  = -\tan\left( \frac{t}{2} + \pi \right)
\]




\end{template}

$K$ is a periodic function, since $\tan(\theta)$ is a periodic function. \\

The principal interval of $\tan(\theta)$ is $\left( -\frac{\pi}{2}, \frac{\pi}{2}  \right)$.  The principal interval of $K$ will be the values of $t$ that make the values of $In(y)$ run from $-\frac{\pi}{2}$ to $\frac{\pi}{2}$. \\


$\frac{y}{2} + \pi = -\frac{\pi}{2}$ at $y = -3\pi$

$\frac{y}{2} + \pi = \frac{\pi}{2}$ at $y = -\pi$

The principle interval of $T$ is $\left( -3\pi, \pi \right)$. \\

The length of this interval is $2\pi$, which is the period of $K$. \\












\textbf{\textcolor{blue!55!black}{$\blacktriangleright$ desmos graph}} 
\begin{center}
\desmos{pqjnzks9b2}{400}{300}
\end{center}











\begin{observation}

$K$ is a periodic function with period $2\pi$. \\


Therefore, our analysis will focus on the principal interval of $\left( -3\pi, -\pi \right)$. \\


All of the features and characteristics we discover will repeat with a period of $2\pi$.\\


\end{observation}









\textbf{\textcolor{blue!55!black}{Domain}}


The domain of $Out(x) = -x$ is $(-\infty, \infty)$, because $Out$ is a linear function.  \\

This includes any value of $\tan(\theta)$. That means we can use the entire domain of $\tan(\theta)$, which excludes $\frac{\pi}{2} + n \pi$ where $n \in \mathbb{Z}$. \\


This would be $-\pi$ plus or minus any whole numbers of $2\pi$.

We need to exclude  all $\pi + 2n\pi$ where $n \in \mathbb{Z}$. \\




The domain of $T$ is $(-\infty, \infty)$ except $\left\{  \pi + 2n\pi \, | \,   n \in \mathbb{Z}  \right\}$. \\




We will be examining $K(t)$ on $(-3\pi, -\pi)$. \\














\textbf{\textcolor{blue!55!black}{Zeros}}


Since the values of $K$ come from the values of $Out$, we are first looking for zeros of $Out$.  $Out(x) = -x$ is a linear function and has only one zero, $0$. \\

We are looking for where $x = \tan(\theta) = 0$. \\


The only number in our principal interval where this happens is $\theta = 0$. \\



We need the values of $y$ where $\theta = In(y) = \frac{y}{2} + \pi = 0$. \\


$y = -2\pi$  

[ These agree with the graph. ]


And, these repeat every $2\pi$ for $K$.




















\textbf{\textcolor{blue!55!black}{Continuity}}


The component functions of the composition are linear and tangent, all continuous. \\

The composition of continuous function is continuous.

$K$ is continuous. It has no disontinuities. \\


However, $K$ does have singularities. In our principal interval, $K$ has singularities at $-3\pi$ and $\pi$.  These repeat witha period of $2\pi$.


The singularities of $K$ are $\{ (2n+1)\pi  \, | \, n \in \mathbb{Z}  \}$. \\



We also need the behvaior of $K$ at these singularities. \\



\[
\lim\limits_{y to -3\pi^+} In(y) = \lim\limits_{y \to -3\pi^+} \frac{y}{2} + \pi = \left( \frac {5\pi}{2} \right)^+
\]




\[
\lim\limits_{\theta \to \left( \tfrac {5\pi}{2} \right)^+} \tan(\theta) = -\infty
\]




\[
\lim\limits_{x \to -\infty} Out(x) = \lim\limits_{x to -\infty} (-x) = \infty
\]





\[
\lim\limits_{x \to -3\pi^+} K(x) = \infty
\]


Tangent has the opposite behavior on the other side of the principal interval. 






\[
\lim\limits_{x to -\pi^-} K(x) = -\infty
\]




\textbf{Notes:} \\


$\blacktriangleright$ Comparing $K(t)$ to $\tan(t)$, the inside has been multiplied by $\frac{1}{2}$.  This would indicate the perioid is doubled from $\pi$ to $2\pi$, which has happened.  

$\blacktriangleright$ Additionally, an $\pi$ hs been added to the inside.  This would indicate a horizontal shift left, which has happened.














\textbf{\textcolor{blue!55!black}{End-Behavior}}


$K$ is a periodic function with period $2\pi$. \\

It either has no end-behavior or the end-behavior is that it is periodic.











\textbf{\textcolor{blue!55!black}{Behavior}}


We need the behavior of each of the three component funcitons and then we will compose them together. \\




$Out(x) = -x$ is a linear function with a negative leading coefficient, so it is an decreasing function.\\


$In(y) = \frac{y}{2} + \pi$ is a linear function with a positive leading coefficient, so it is an increasing function.\\


$\tan(\theta)$ is an increasing function.


Each of the component functions is monomtonic function, which means $K$ will not change behavior. \\



\[
K(t) = negative \circ increasin \circ increasing = decreasing
\]




[ This agrees with the graph. ]












\textbf{\textcolor{blue!55!black}{Global Maximum and Minimum}}


The singularity behavior has already shown us that $K$ is unbounded. There is no global maximum or minimum.  














\textbf{\textcolor{blue!55!black}{Local Maximum and Minimum}}


$K$ is monotonic (its bahavior doesn't change) and it is unbounded.  There are no local minimums or minimums. 










\textbf{\textcolor{blue!55!black}{Range}}


Since $T$ is continuous and the singularity behavior has already shown us that $K$ is unbounded, then the range is $(-\infty, \infty)$. \\ 



[ This agrees with the graph. ]






















\begin{center}
\textbf{\textcolor{green!50!black}{ooooo-=-=-=-ooOoo-=-=-=-ooooo}} \\

more examples can be found by following this link\\ \link[More Examples of Trigonometric Functions]{https://ximera.osu.edu/csccmathematics/precalculus/precalculus/trigFunctions/examples/exampleList}

\end{center}




\end{document}

