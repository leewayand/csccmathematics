\documentclass{ximera}

%\usepackage{todonotes}

\newcommand{\todo}{}

\usepackage{esint} % for \oiint
\ifxake%%https://math.meta.stackexchange.com/questions/9973/how-do-you-render-a-closed-surface-double-integral
\renewcommand{\oiint}{{\large\bigcirc}\kern-1.56em\iint}
\fi


\graphicspath{
  {./}
  {ximeraTutorial/}
  {basicPhilosophy/}
  {functionsOfSeveralVariables/}
  {normalVectors/}
  {lagrangeMultipliers/}
  {vectorFields/}
  {greensTheorem/}
  {shapeOfThingsToCome/}
  {dotProducts/}
  {partialDerivativesAndTheGradientVector/}
  {../productAndQuotientRules/exercises/}
  {../normalVectors/exercisesParametricPlots/}
  {../continuityOfFunctionsOfSeveralVariables/exercises/}
  {../partialDerivativesAndTheGradientVector/exercises/}
  {../directionalDerivativeAndChainRule/exercises/}
  {../commonCoordinates/exercisesCylindricalCoordinates/}
  {../commonCoordinates/exercisesSphericalCoordinates/}
  {../greensTheorem/exercisesCurlAndLineIntegrals/}
  {../greensTheorem/exercisesDivergenceAndLineIntegrals/}
  {../shapeOfThingsToCome/exercisesDivergenceTheorem/}
  {../greensTheorem/}
  {../shapeOfThingsToCome/}
  {../separableDifferentialEquations/exercises/}
  {vectorFields/}
}

\newcommand{\mooculus}{\textsf{\textbf{MOOC}\textnormal{\textsf{ULUS}}}}

\usepackage{tkz-euclide}
\usepackage{tikz}
\usepackage{tikz-cd}
\usetikzlibrary{arrows}
\tikzset{>=stealth,commutative diagrams/.cd,
  arrow style=tikz,diagrams={>=stealth}} %% cool arrow head
\tikzset{shorten <>/.style={ shorten >=#1, shorten <=#1 } } %% allows shorter vectors

\usetikzlibrary{backgrounds} %% for boxes around graphs
\usetikzlibrary{shapes,positioning}  %% Clouds and stars
\usetikzlibrary{matrix} %% for matrix
\usepgfplotslibrary{polar} %% for polar plots
\usepgfplotslibrary{fillbetween} %% to shade area between curves in TikZ
%\usetkzobj{all}
\usepackage[makeroom]{cancel} %% for strike outs
%\usepackage{mathtools} %% for pretty underbrace % Breaks Ximera
%\usepackage{multicol}
\usepackage{pgffor} %% required for integral for loops



%% http://tex.stackexchange.com/questions/66490/drawing-a-tikz-arc-specifying-the-center
%% Draws beach ball
\tikzset{pics/carc/.style args={#1:#2:#3}{code={\draw[pic actions] (#1:#3) arc(#1:#2:#3);}}}



\usepackage{array}
\setlength{\extrarowheight}{+.1cm}
\newdimen\digitwidth
\settowidth\digitwidth{9}
\def\divrule#1#2{
\noalign{\moveright#1\digitwidth
\vbox{\hrule width#2\digitwidth}}}




% \newcommand{\RR}{\mathbb R}
% \newcommand{\R}{\mathbb R}
% \newcommand{\N}{\mathbb N}
% \newcommand{\Z}{\mathbb Z}

\newcommand{\sagemath}{\textsf{SageMath}}


%\renewcommand{\d}{\,d\!}
%\renewcommand{\d}{\mathop{}\!d}
%\newcommand{\dd}[2][]{\frac{\d #1}{\d #2}}
%\newcommand{\pp}[2][]{\frac{\partial #1}{\partial #2}}
% \renewcommand{\l}{\ell}
%\newcommand{\ddx}{\frac{d}{\d x}}

% \newcommand{\zeroOverZero}{\ensuremath{\boldsymbol{\tfrac{0}{0}}}}
%\newcommand{\inftyOverInfty}{\ensuremath{\boldsymbol{\tfrac{\infty}{\infty}}}}
%\newcommand{\zeroOverInfty}{\ensuremath{\boldsymbol{\tfrac{0}{\infty}}}}
%\newcommand{\zeroTimesInfty}{\ensuremath{\small\boldsymbol{0\cdot \infty}}}
%\newcommand{\inftyMinusInfty}{\ensuremath{\small\boldsymbol{\infty - \infty}}}
%\newcommand{\oneToInfty}{\ensuremath{\boldsymbol{1^\infty}}}
%\newcommand{\zeroToZero}{\ensuremath{\boldsymbol{0^0}}}
%\newcommand{\inftyToZero}{\ensuremath{\boldsymbol{\infty^0}}}



% \newcommand{\numOverZero}{\ensuremath{\boldsymbol{\tfrac{\#}{0}}}}
% \newcommand{\dfn}{\textbf}
% \newcommand{\unit}{\,\mathrm}
% \newcommand{\unit}{\mathop{}\!\mathrm}
% \newcommand{\eval}[1]{\bigg[ #1 \bigg]}
% \newcommand{\seq}[1]{\left( #1 \right)}
% \renewcommand{\epsilon}{\varepsilon}
% \renewcommand{\phi}{\varphi}


% \renewcommand{\iff}{\Leftrightarrow}

% \DeclareMathOperator{\arccot}{arccot}
% \DeclareMathOperator{\arcsec}{arcsec}
% \DeclareMathOperator{\arccsc}{arccsc}
% \DeclareMathOperator{\si}{Si}
% \DeclareMathOperator{\scal}{scal}
% \DeclareMathOperator{\sign}{sign}


%% \newcommand{\tightoverset}[2]{% for arrow vec
%%   \mathop{#2}\limits^{\vbox to -.5ex{\kern-0.75ex\hbox{$#1$}\vss}}}
% \newcommand{\arrowvec}[1]{{\overset{\rightharpoonup}{#1}}}
% \renewcommand{\vec}[1]{\arrowvec{\mathbf{#1}}}
% \renewcommand{\vec}[1]{{\overset{\boldsymbol{\rightharpoonup}}{\mathbf{#1}}}}

% \newcommand{\point}[1]{\left(#1\right)} %this allows \vector{ to be changed to \vector{ with a quick find and replace
% \newcommand{\pt}[1]{\mathbf{#1}} %this allows \vec{ to be changed to \vec{ with a quick find and replace
% \newcommand{\Lim}[2]{\lim_{\point{#1} \to \point{#2}}} %Bart, I changed this to point since I want to use it.  It runs through both of the exercise and exerciseE files in limits section, which is why it was in each document to start with.

% \DeclareMathOperator{\proj}{\mathbf{proj}}
% \newcommand{\veci}{{\boldsymbol{\hat{\imath}}}}
% \newcommand{\vecj}{{\boldsymbol{\hat{\jmath}}}}
% \newcommand{\veck}{{\boldsymbol{\hat{k}}}}
% \newcommand{\vecl}{\vec{\boldsymbol{\l}}}
% \newcommand{\uvec}[1]{\mathbf{\hat{#1}}}
% \newcommand{\utan}{\mathbf{\hat{t}}}
% \newcommand{\unormal}{\mathbf{\hat{n}}}
% \newcommand{\ubinormal}{\mathbf{\hat{b}}}

% \newcommand{\dotp}{\bullet}
% \newcommand{\cross}{\boldsymbol\times}
% \newcommand{\grad}{\boldsymbol\nabla}
% \newcommand{\divergence}{\grad\dotp}
% \newcommand{\curl}{\grad\cross}
%\DeclareMathOperator{\divergence}{divergence}
%\DeclareMathOperator{\curl}[1]{\grad\cross #1}
% \newcommand{\lto}{\mathop{\longrightarrow\,}\limits}

% \renewcommand{\bar}{\overline}

\colorlet{textColor}{black}
\colorlet{background}{white}
\colorlet{penColor}{blue!50!black} % Color of a curve in a plot
\colorlet{penColor2}{red!50!black}% Color of a curve in a plot
\colorlet{penColor3}{red!50!blue} % Color of a curve in a plot
\colorlet{penColor4}{green!50!black} % Color of a curve in a plot
\colorlet{penColor5}{orange!80!black} % Color of a curve in a plot
\colorlet{penColor6}{yellow!70!black} % Color of a curve in a plot
\colorlet{fill1}{penColor!20} % Color of fill in a plot
\colorlet{fill2}{penColor2!20} % Color of fill in a plot
\colorlet{fillp}{fill1} % Color of positive area
\colorlet{filln}{penColor2!20} % Color of negative area
\colorlet{fill3}{penColor3!20} % Fill
\colorlet{fill4}{penColor4!20} % Fill
\colorlet{fill5}{penColor5!20} % Fill
\colorlet{gridColor}{gray!50} % Color of grid in a plot

\newcommand{\surfaceColor}{violet}
\newcommand{\surfaceColorTwo}{redyellow}
\newcommand{\sliceColor}{greenyellow}




\pgfmathdeclarefunction{gauss}{2}{% gives gaussian
  \pgfmathparse{1/(#2*sqrt(2*pi))*exp(-((x-#1)^2)/(2*#2^2))}%
}


%%%%%%%%%%%%%
%% Vectors
%%%%%%%%%%%%%

%% Simple horiz vectors
\renewcommand{\vector}[1]{\left\langle #1\right\rangle}


%% %% Complex Horiz Vectors with angle brackets
%% \makeatletter
%% \renewcommand{\vector}[2][ , ]{\left\langle%
%%   \def\nextitem{\def\nextitem{#1}}%
%%   \@for \el:=#2\do{\nextitem\el}\right\rangle%
%% }
%% \makeatother

%% %% Vertical Vectors
%% \def\vector#1{\begin{bmatrix}\vecListA#1,,\end{bmatrix}}
%% \def\vecListA#1,{\if,#1,\else #1\cr \expandafter \vecListA \fi}

%%%%%%%%%%%%%
%% End of vectors
%%%%%%%%%%%%%

%\newcommand{\fullwidth}{}
%\newcommand{\normalwidth}{}



%% makes a snazzy t-chart for evaluating functions
%\newenvironment{tchart}{\rowcolors{2}{}{background!90!textColor}\array}{\endarray}

%%This is to help with formatting on future title pages.
\newenvironment{sectionOutcomes}{}{}



%% Flowchart stuff
%\tikzstyle{startstop} = [rectangle, rounded corners, minimum width=3cm, minimum height=1cm,text centered, draw=black]
%\tikzstyle{question} = [rectangle, minimum width=3cm, minimum height=1cm, text centered, draw=black]
%\tikzstyle{decision} = [trapezium, trapezium left angle=70, trapezium right angle=110, minimum width=3cm, minimum height=1cm, text centered, draw=black]
%\tikzstyle{question} = [rectangle, rounded corners, minimum width=3cm, minimum height=1cm,text centered, draw=black]
%\tikzstyle{process} = [rectangle, minimum width=3cm, minimum height=1cm, text centered, draw=black]
%\tikzstyle{decision} = [trapezium, trapezium left angle=70, trapezium right angle=110, minimum width=3cm, minimum height=1cm, text centered, draw=black]


\title{Sine}

\begin{document}

\begin{abstract}
analysis
\end{abstract}
\maketitle












\textbf{\textcolor{purple!85!blue}{Analyze  $T(r) =  2 \sin(3r - \pi) - 1$}}



\begin{template}


We are going to view this as a composition of three functions.


\[
T = Out \circ \sin \circ In
\]


where


$Out(x) = 2x - 1$ \\


$In(y) = 3y - \pi$ \\




Along with $\sin(\theta)$, these three component functions, we get



\[
T(r) = (Out \circ \sin \circ In)(r) = Out(sin(In(r)))  = 2 \sin(3r - \pi) - 1
\]




\end{template}

$T$ is a periodic function, since $\sin(\theta)$ is a periodic function. \\

The principal interval of $\sin(\theta)$ is $[0, 2\pi)$.  The principal interval of $T$ will be the values of $r$ that make the values of $In(y)$ run from $0$ to $2\pi$. \\


$3y - \pi = 0$ at $y = \frac{\pi}{3}$

$3y - \pi = 2\pi$ at $y = \pi$

The principle interval of $T$ is $\left[ \frac{\pi}{3}, \pi \right)$. \\

The length of this interval is $\frac{2\pi}{3}$, which is the period of $T$. \\









\textbf{\textcolor{blue!55!black}{$\blacktriangleright$ desmos graph}} 
\begin{center}
\desmos{b9yiwlxg2n}{400}{300}
\end{center}





\begin{observation}

$T$ is a periodic function with period $\frac{2\pi}{3}$. \\


Therefore, our analysis will focus on the principal interval of $\left[ \frac{\pi}{3}, \pi \right)$. \\


All of the features and characteristics we discover will repeat wiht a period of $\frac{2\pi}{3}$.\\


\end{observation}









\textbf{\textcolor{blue!55!black}{Domain}}


The domain of $Out(x) = 2x - 1$ is $(-\infty, \infty)$, because $Out$ is a linear function.  \\

This includes any value of $\sin(\theta)$. That means we can use the entire domain of $\sin(\theta)$, which is $(-\infty, \infty)$.  \\


This includes any value of $In(y)$, which means we can use the entire domain of $In(y)$, which is $(-\infty, \infty)$, because $In$ is a linear function.  \\


The domain of $T$ is $(-\infty, \infty)$. \\








\textbf{\textcolor{blue!55!black}{Zeros}}


Since the values of $T$ come from the values of $Out$, we are first looking for zeros of $Out$.  $Out(x) = 2x - 1$ is a linear function and has only one zero, $\frac{1}{2}$. \\

We are looking for where $\sin(\theta) = \frac{1}{2}$. \\


There are two numbers in $[0, 2\pi)$ where $\sin(\theta) = \frac{1}{2}$.  They are $\frac{\pi}{6}$ and $\frac {5\pi}{6}$.

We need the values of $y$ where $In(y) = 3y - \pi = \frac{\pi}{6}$ and $In(y) = 3y - \pi = \frac{5\pi}{6}$

$y = \frac{7\pi}{18}$  and   $y = \frac{11\pi}{18}$

[ These agree with the graph. ]


And, these repeat every $\frac{2\pi}{3}$ for $T$.












\textbf{\textcolor{blue!55!black}{Continuity}}


The component functions of the composition are linear and sine, all continuous. \\

The composition of continuous function is continuous.

$T$ is continuous. It has no disontinuities. \\


Since the domain is $(-\infty, \infty)$, there are no singularities. \\








\textbf{\textcolor{blue!55!black}{End-Behavior}}


$T$ is a periodic function with period $\frac{2\pi}{3}$. \\

It either has no end-behavior or the end-behavior is that it is periodic.












\textbf{\textcolor{blue!55!black}{Behavior}}


We need the behavior of each of the three component funcitons and then we will compose them together. \\




$Out(x) = 2x - 1$ is a linear function with a positive leading coefficient, so it is an increasing function.\\


$In(y) = 3y - \pi$ is a linear function with a positive leading coefficient, so it is an increasing function.\\


$\sin(\theta)$ increases and decreases through the quadrants. For the principal interval, we have


\begin{itemize}
  \item increases on $\left( 0, \frac{\pi}{2} \right)$
  \item decreases on $\left( \frac{\pi}{2}, \pi \right)$
  \item decreases on $\left( \pi, \frac{3\pi}{2} \right)$
  \item increases on $\left( \frac{3\pi}{2}, 2\pi \right)$
\end{itemize}


Now to trace domains and ranges.


We need the values of $In(y) = 3y - \pi$ to be $0$, $\frac{\pi}{2}$, $\pi$, $\frac{3\pi}{2}$, and $2\pi$. \\



$\blacktriangleright$ $3y - \pi = 0$ when $y = \frac{\pi}{3}$ \\

$\blacktriangleright$ $3y - \pi = \frac{\pi}{2}$ when $y = \frac{\pi}{2}$ \\

$\blacktriangleright$ $3y - \pi = \pi$ when $y = \frac{2\pi}{3}$ \\

$\blacktriangleright$ $3y - \pi = \frac{3\pi}{2}$ when $y = \frac{5\pi}{6}$ \\

$\blacktriangleright$ $3y - \pi = 2\pi$ when $y = \pi$ \\





Now to glue everything together. \\



\textbf{\textcolor{purple!80!black}{On $r = y \in \left( \frac{\pi}{3}, \frac{\pi}{2} \right)$, }}



$In(y) = 3y - \pi$ is increasing.  The range of $In(y)$ is $\theta = In(y) \in \left(0, \frac{\pi}{2} \right)$, where $\sin(\theta)$ is increasing. \\



The range of $\sin(\theta)$ on $\theta \in \left(0, \frac{\pi}{2} \right)$ is $x = \sin(\theta) \in (0, 1)$, where $Out(x)$ is increasing, because $Out$ is always increasing. \\


\[
T(r) = increasing \circ increasing \circ increasing = increasing
\]








\textbf{\textcolor{purple!80!black}{On $r = y \in \left( \frac{\pi}{2}, \frac{2\pi}{3} \right)$, }}




$In(y) = 3y - \pi$ is increasing.  The range of $In(y)$ is $\theta = In(y) \in \left(\frac{\pi}{2}, \pi \right)$, where $\sin(\theta)$ is decreasing. \\



The range of $\sin(\theta)$ on $\theta \in \left( \frac{\pi}{2}, \pi \right)$ is $x = \sin(\theta) \in (0, 1)$, where $Out(x)$ is increasing, because $Out$ is always increasing. \\


\[
T(r) = increasing \circ decreasing \circ increasing = decreasing
\]












\textbf{\textcolor{purple!80!black}{On $r = y \in \left( \frac{2\pi}{3}, \frac{5\pi}{6} \right)$, }}




$In(y) = 3y - \pi$ is increasing.  The range of $In(y)$ is $\theta = In(y) \in \left(\pi, \frac{3\pi}{2} \right)$, where $\sin(\theta)$ is decreasing. \\



The range of $\sin(\theta)$ on $\theta \in \left( \pi, \frac{3\pi}{2} \right)$ is $x = \sin(\theta) \in (-1, 0)$, where $Out(x)$ is increasing, because $Out$ is always increasing. \\


\[
T(r) = increasing \circ decreasing \circ increasing = decreasing
\]










\textbf{\textcolor{purple!80!black}{On $r = y \in \left( \frac{5\pi}{6}, \pi \right)$, }}




$In(y) = 3y - \pi$ is increasing.  The range of $In(y)$ is $\theta = In(y) \in \left( \frac{3\pi}{2}, 2\pi \right)$, where $\sin(\theta)$ is increasing. \\



The range of $\sin(\theta)$ on $\theta \in \left( \frac{3\pi}{2}, 2\pi \right)$ is $x = \sin(\theta) \in (-1, 0)$, where $Out(x)$ is increasing, because $Out$ is always increasing. \\


\[
T(r) = increasing \circ increasing \circ increasing = increasing
\]







This behavior repeats with a period of $\frac{2\pi}{3}$. \\



[ This agrees with the graph. ]





\textbf{\textcolor{blue!55!black}{$\blacktriangleright$ desmos graph}} 
\begin{center}
\desmos{bntu7gqkni}{400}{300}
\end{center}



















\textbf{\textcolor{blue!55!black}{Local Maximum and Minimum}}



$T$ is continuous on $(-\infty, \infty)$.


$\blacktriangleright$ $T$ switches from increasing to decreasing at $\frac{\pi}{2}$, which makes $\frac{\pi}{2}$ a crtical number and the location of a local maximum. \\




$\blacktriangleright$ $T$ switches from decreasing to increasing at $\frac{5\pi}{6}$, which makes $\frac{5\pi}{6}$ a crtical number and the location of a local minimum. \\


These repeat with a period of $\frac{2\pi}{3}$. \\



\[
T\left( \frac{\pi}{2} \right) = 2 \sin\left(3 \cdot \frac{\pi}{2} \right) - 1 = -3
\]




\[
T\left( \frac{5\pi}{6} \right) = 2 \sin\left(3 \cdot \frac{5\pi}{6} \right) - 1 = 1
\]












\textbf{\textcolor{blue!55!black}{Global Maximum and Minimum}}


Since $T$ is continuous on $(-\infty, \infty)$, $\frac{\pi}{2}$ and $\frac{5\pi}{6}$ are the only critical numbers, we also have global extrema here. \\




\[
T\left( \frac{\pi}{2} \right) = 2 \sin\left(3 \cdot \frac{\pi}{2} \right) - 1 = -3
\]




\[
T\left( \frac{5\pi}{6} \right) = 2 \sin\left(3 \cdot \frac{5\pi}{6} \right) - 1 = 1
\]




These repeat with a period of $\frac{2\pi}{3}$. \\









\textbf{\textcolor{blue!55!black}{Range}}


Since $T$ is continuous with a global maximum of $1$ and a global minimum of $-1$, the range is $[-3, 1]$.  \\ 



[ This agrees with the graph. ]
























\subsection*{Periodic}

$T$ is periodic with period $\frac{2\pi}{3}$. \\



We have zeros in the principal interval as $\frac{7\pi}{18}$ and $\frac{11\pi}{18}$. \\


To describe all of the zeros of $T$, we need to include all numbers which are these two numbers plus or minus and whole number of $\frac{2\pi}{3}$. \\



\[
\left\{  \frac{7\pi}{18} + k  \cdot \frac{2\pi}{3}  \, | \,  k \in \mathbb{Z}         \right\}
\]



\[
\left\{  \frac{11\pi}{18} + k  \cdot \frac{2\pi}{3}  \, | \,  k \in \mathbb{Z}         \right\}
\]




Similarly, we could describe the intervals by including $k  \cdot \frac{2\pi}{3} $. \\














\begin{center}
\textbf{\textcolor{green!50!black}{ooooo-=-=-=-ooOoo-=-=-=-ooooo}} \\

more examples can be found by following this link\\ \link[More Examples of Trigonometric Functions]{https://ximera.osu.edu/csccmathematics/precalculus/precalculus/trigFunctions/examples/exampleList}

\end{center}


\end{document}

