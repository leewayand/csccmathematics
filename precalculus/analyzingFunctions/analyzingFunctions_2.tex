\documentclass{ximera}


\graphicspath{
  {./}
  {ximeraTutorial/}
  {basicPhilosophy/}
}

\newcommand{\mooculus}{\textsf{\textbf{MOOC}\textnormal{\textsf{ULUS}}}}


\usepackage{tkz-euclide}\usepackage{tikz}
\usepackage{tikz-cd}
\usetikzlibrary{arrows}
\tikzset{>=stealth,commutative diagrams/.cd,
  arrow style=tikz,diagrams={>=stealth}} %% cool arrow head
\tikzset{shorten <>/.style={ shorten >=#1, shorten <=#1 } } %% allows shorter vectors

\usetikzlibrary{backgrounds} %% for boxes around graphs
\usetikzlibrary{shapes,positioning}  %% Clouds and stars
\usetikzlibrary{matrix} %% for matrix
\usepgfplotslibrary{polar} %% for polar plots
\usepgfplotslibrary{fillbetween} %% to shade area between curves in TikZ
\usetkzobj{all}
\usepackage[makeroom]{cancel} %% for strike outs
%\usepackage{mathtools} %% for pretty underbrace % Breaks Ximera
%\usepackage{multicol}
\usepackage{pgffor} %% required for integral for loops



%% http://tex.stackexchange.com/questions/66490/drawing-a-tikz-arc-specifying-the-center
%% Draws beach ball
\tikzset{pics/carc/.style args={#1:#2:#3}{code={\draw[pic actions] (#1:#3) arc(#1:#2:#3);}}}



\usepackage{array}
\setlength{\extrarowheight}{+.1cm}
\newdimen\digitwidth
\settowidth\digitwidth{9}
\def\divrule#1#2{
\noalign{\moveright#1\digitwidth
\vbox{\hrule width#2\digitwidth}}}
























%%This is to help with formatting on future title pages.
\newenvironment{sectionOutcomes}{}{}


\title{Analyzing}

\begin{document}

\begin{abstract}
describe everything
\end{abstract}
\maketitle







$\blacktriangleright$ \textbf{\textcolor{red!80!black}{Reasoning:}} Reasoning is a logical explanation that describes our conclusions, how we arrived at those conclusions, and why we think those conclusions are correct. \\

Analysis is not a list of conclusions. We are not looking for such a list. \\

We are looking for the thought process that arrived at the list of conclusions. \\











\begin{example}

\textbf{\textcolor{purple!85!blue}{Completely analyze}} \\


\[   L(x) = \frac{5}{1 + 3 \, e^{-\tfrac{x}{2}}} \]




\textbf{Note:} $L(x)$ is the composition of a rational function with an exponential function.

$L = f \circ g$ where

\[  f(y) =  \frac{5}{1 + 3 \, y}  \, \text{ and } \, g(k) = e^{-\tfrac{k}{2}}    \]




\textbf{\textcolor{blue!55!black}{$\blacktriangleright$ Domain}} 

$e^{-\tfrac{x}{2}}$ is an exponential function with base $e > 1$.  It has a positive leading coefficient and the leading coefficient of the inside linear function is negative. \\



The range of the exponential function, $e^{-\tfrac{x}{2}}$ is $(0, \infty)$. The range does not include \wordChoice{\choice{positive} \choice[correct]{negative}}  numbers. \\



Since, the range of $g(k)$ does not include negative numbers, it does not include $-\frac{1}{3}$, which is the zero of the denominaotr of $f(y)$.  The denominator of $L(x)$ cannot equal $0$.  \\


 That makes the domain of $L$ all real numbers, $(-\infty, \infty)$.














\textbf{\textcolor{blue!55!black}{$\blacktriangleright$ Zeros}} 



$L$ is a quotient, therefore the zeros of $L$ would be the zeros of the numerator (which are not also zeros of the denominator).  However, the numerator is a constant function.  It has no zeros. Therefore, $L$ has no zeros.










\textbf{\textcolor{blue!55!black}{$\blacktriangleright$ Continuity}} 


The denominator of $L$ is a shifted exponential function, so it is continuous everywhere. The numerator is a constant function, so it is continuous everywhere.  $L$ is the quotient of two continuous functions.  Therefore, it is continuous. \\


Since the denominator is never equal to $0$, there are no singularities. \\














\textbf{\textcolor{blue!55!black}{$\blacktriangleright$ End-Behavior}} 


The inside function is $e^{-\tfrac{x}{2}}$, which is an exponential function with base $e > 1$.  It has a positive leading coefficient and the leading coefficient of the inside linear function is negative. \\

That means $g(k)$ is a positive decreasing exponential function. \\



\[  
\lim\limits_{k \to -\infty}g(k) = \lim\limits_{k \to -\infty}e^{-\tfrac{k}{2}} = \infty
\]




\[  
\lim\limits_{k \to \infty}g(k) = \lim\limits_{k \to \infty}e^{-\tfrac{k}{2}} = 0
\]




To get the end-behavior of $L(x)$, we need to see what $f(y)$ is doing when $g(k)$ is behaving as above. \\





\[  
\lim\limits_{x \to -\infty}L(x) = \lim\limits_{y \to \infty}f(y) = \lim\limits_{y \to \infty}\frac{5}{1 + 3 y} = 0
\]

This is because $f(y)$ is a rational function with the degree of the denominator greater than the degree of the numerator. \\






\[  
\lim\limits_{x \to \infty}L(x) = \lim\limits_{y \to 0}f(y) = \frac{5}{1 + 3 (0)} = 5
\]



This is because $f$ is continuous, which means the limit as $y$ approaches $0$ is just the value of the function. \\












\textbf{\textcolor{blue!55!black}{$\blacktriangleright$ Behavior}} 
\textbf{Rate-of-Change}  
\textbf{Increasing and Decreasing}   



The inside function, $e^{-\tfrac{x}{2}}$, is an exponential function with base $e > 1$.  It has a positive leading coefficient and the leading coefficient of the inside linear function is negative. \\


$e^{-\tfrac{x}{2}}$ is a strictly \wordChoice{\choice{increasing} \choice[correct]{decreasing}}  function 


The outside function is a rational function with a constant positive numerator. \\

The denominator of the fraction is positve and decreasing, which means it is getting smaller. \\

That makes $f$ an increasing function. \\


$L$ is a composition of two increasing functions. \\


$L(x)$ is a strictly \wordChoice{\choice[correct]{increasing} \choice{decreasing}}  function.






\textbf{\textcolor{blue!55!black}{$\blacktriangleright$ Extrema}} 


As a strictly increasing function on $(-\infty, \infty)$, $L$ cannot have global or local maximums or minimums. \\










\textbf{\textcolor{blue!55!black}{$\blacktriangleright$ Range}} 



$L$ is a continuous function, never negative, and increasing. In addition, we nkow that 





\[  
\lim\limits_{x \to -\infty}L(x) = \lim\limits_{y \to \infty}f(y) = \lim\limits_{y \to \infty}\frac{5}{1 + 3 y} = 0
\]




\[  
\lim\limits_{x \to \infty}L(x) = \lim\limits_{y \to 0}f(y) = \frac{5}{1 + 3 (0)} = 5
\]








The range of $L$ is $(0, 5)$.













\textbf{\textcolor{blue!55!black}{$\blacktriangleright$ A Graph}} 





The end-behavior tells us that the graph has horizontal asymptotes.










\begin{image}
\begin{tikzpicture}
  \begin{axis}[
            domain=-10:10, ymax=6, xmax=10, ymin=-1, xmin=-10,
            axis lines =center, xlabel=$x$, ylabel={$y=L(x)$}, grid = major, grid style={dashed},
            ytick={1,2,3,4,5,6},
            xtick={-10,-8,-6,-4,-2,2,4,6,8,10},
            yticklabels={$1$,$2$,$3$,$4$,$5$,$6$}, 
            xticklabels={$-10$,$-8$,$-6$,$-4$,$-2$,$2$,$4$,$6$,$8$,$10$},
            ticklabel style={font=\scriptsize},
            every axis y label/.style={at=(current axis.above origin),anchor=south},
            every axis x label/.style={at=(current axis.right of origin),anchor=west},
            axis on top
          ]
          
          \addplot [line width=1, gray, dashed,samples=100,domain=(-10:10),<->] {5};
          \addplot [line width=1, gray, dashed,samples=100,domain=(-10:10),<->] {0};

            %\addplot [line width=2, penColor, smooth,samples=100,domain=(-9:9)] {5/(1 + 3 * e^(-x/2))};
          \addplot [line width=2, penColor, smooth,samples=100,domain=(-9:9),<->] {5/(1 + 3 * 2.7182^(-x/2))};

           




           

  \end{axis}
\end{tikzpicture}
\end{image}




\end{example}





\subsection*{with Calculus}

\textbf{\textcolor{red!70!black}{A Peek Ahead...}}




Calculus will give us a formula for the derivative.

\[  L'(x) =   \frac{-15}{2} \cdot \frac{e^{-\tfrac{x}{2}}}{\left(1+3 e^{-\tfrac{x}{2}}\right)^2}    \]


This derivative has no zeros, which tells us that $L(x)$ has no critical numbers.  $L'(x)$ is always positive, which tells us that $L(x)$ is always increasing.












\begin{center}
\textbf{\textcolor{green!50!black}{ooooo-=-=-=-ooOoo-=-=-=-ooooo}} \\

more examples can be found by following this link\\ \link[More Examples of Analyzing Functions]{https://ximera.osu.edu/csccmathematics/precalculus2/precalculus2/analyzingFunctions/examples/exampleList}

\end{center}








\end{document}
