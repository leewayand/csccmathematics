\documentclass{ximera}

%\usepackage{todonotes}

\newcommand{\todo}{}

\usepackage{esint} % for \oiint
\ifxake%%https://math.meta.stackexchange.com/questions/9973/how-do-you-render-a-closed-surface-double-integral
\renewcommand{\oiint}{{\large\bigcirc}\kern-1.56em\iint}
\fi


\graphicspath{
  {./}
  {ximeraTutorial/}
  {basicPhilosophy/}
  {functionsOfSeveralVariables/}
  {normalVectors/}
  {lagrangeMultipliers/}
  {vectorFields/}
  {greensTheorem/}
  {shapeOfThingsToCome/}
  {dotProducts/}
  {partialDerivativesAndTheGradientVector/}
  {../productAndQuotientRules/exercises/}
  {../normalVectors/exercisesParametricPlots/}
  {../continuityOfFunctionsOfSeveralVariables/exercises/}
  {../partialDerivativesAndTheGradientVector/exercises/}
  {../directionalDerivativeAndChainRule/exercises/}
  {../commonCoordinates/exercisesCylindricalCoordinates/}
  {../commonCoordinates/exercisesSphericalCoordinates/}
  {../greensTheorem/exercisesCurlAndLineIntegrals/}
  {../greensTheorem/exercisesDivergenceAndLineIntegrals/}
  {../shapeOfThingsToCome/exercisesDivergenceTheorem/}
  {../greensTheorem/}
  {../shapeOfThingsToCome/}
  {../separableDifferentialEquations/exercises/}
  {vectorFields/}
}

\newcommand{\mooculus}{\textsf{\textbf{MOOC}\textnormal{\textsf{ULUS}}}}

\usepackage{tkz-euclide}
\usepackage{tikz}
\usepackage{tikz-cd}
\usetikzlibrary{arrows}
\tikzset{>=stealth,commutative diagrams/.cd,
  arrow style=tikz,diagrams={>=stealth}} %% cool arrow head
\tikzset{shorten <>/.style={ shorten >=#1, shorten <=#1 } } %% allows shorter vectors

\usetikzlibrary{backgrounds} %% for boxes around graphs
\usetikzlibrary{shapes,positioning}  %% Clouds and stars
\usetikzlibrary{matrix} %% for matrix
\usepgfplotslibrary{polar} %% for polar plots
\usepgfplotslibrary{fillbetween} %% to shade area between curves in TikZ
%\usetkzobj{all}
\usepackage[makeroom]{cancel} %% for strike outs
%\usepackage{mathtools} %% for pretty underbrace % Breaks Ximera
%\usepackage{multicol}
\usepackage{pgffor} %% required for integral for loops



%% http://tex.stackexchange.com/questions/66490/drawing-a-tikz-arc-specifying-the-center
%% Draws beach ball
\tikzset{pics/carc/.style args={#1:#2:#3}{code={\draw[pic actions] (#1:#3) arc(#1:#2:#3);}}}



\usepackage{array}
\setlength{\extrarowheight}{+.1cm}
\newdimen\digitwidth
\settowidth\digitwidth{9}
\def\divrule#1#2{
\noalign{\moveright#1\digitwidth
\vbox{\hrule width#2\digitwidth}}}




% \newcommand{\RR}{\mathbb R}
% \newcommand{\R}{\mathbb R}
% \newcommand{\N}{\mathbb N}
% \newcommand{\Z}{\mathbb Z}

\newcommand{\sagemath}{\textsf{SageMath}}


%\renewcommand{\d}{\,d\!}
%\renewcommand{\d}{\mathop{}\!d}
%\newcommand{\dd}[2][]{\frac{\d #1}{\d #2}}
%\newcommand{\pp}[2][]{\frac{\partial #1}{\partial #2}}
% \renewcommand{\l}{\ell}
%\newcommand{\ddx}{\frac{d}{\d x}}

% \newcommand{\zeroOverZero}{\ensuremath{\boldsymbol{\tfrac{0}{0}}}}
%\newcommand{\inftyOverInfty}{\ensuremath{\boldsymbol{\tfrac{\infty}{\infty}}}}
%\newcommand{\zeroOverInfty}{\ensuremath{\boldsymbol{\tfrac{0}{\infty}}}}
%\newcommand{\zeroTimesInfty}{\ensuremath{\small\boldsymbol{0\cdot \infty}}}
%\newcommand{\inftyMinusInfty}{\ensuremath{\small\boldsymbol{\infty - \infty}}}
%\newcommand{\oneToInfty}{\ensuremath{\boldsymbol{1^\infty}}}
%\newcommand{\zeroToZero}{\ensuremath{\boldsymbol{0^0}}}
%\newcommand{\inftyToZero}{\ensuremath{\boldsymbol{\infty^0}}}



% \newcommand{\numOverZero}{\ensuremath{\boldsymbol{\tfrac{\#}{0}}}}
% \newcommand{\dfn}{\textbf}
% \newcommand{\unit}{\,\mathrm}
% \newcommand{\unit}{\mathop{}\!\mathrm}
% \newcommand{\eval}[1]{\bigg[ #1 \bigg]}
% \newcommand{\seq}[1]{\left( #1 \right)}
% \renewcommand{\epsilon}{\varepsilon}
% \renewcommand{\phi}{\varphi}


% \renewcommand{\iff}{\Leftrightarrow}

% \DeclareMathOperator{\arccot}{arccot}
% \DeclareMathOperator{\arcsec}{arcsec}
% \DeclareMathOperator{\arccsc}{arccsc}
% \DeclareMathOperator{\si}{Si}
% \DeclareMathOperator{\scal}{scal}
% \DeclareMathOperator{\sign}{sign}


%% \newcommand{\tightoverset}[2]{% for arrow vec
%%   \mathop{#2}\limits^{\vbox to -.5ex{\kern-0.75ex\hbox{$#1$}\vss}}}
% \newcommand{\arrowvec}[1]{{\overset{\rightharpoonup}{#1}}}
% \renewcommand{\vec}[1]{\arrowvec{\mathbf{#1}}}
% \renewcommand{\vec}[1]{{\overset{\boldsymbol{\rightharpoonup}}{\mathbf{#1}}}}

% \newcommand{\point}[1]{\left(#1\right)} %this allows \vector{ to be changed to \vector{ with a quick find and replace
% \newcommand{\pt}[1]{\mathbf{#1}} %this allows \vec{ to be changed to \vec{ with a quick find and replace
% \newcommand{\Lim}[2]{\lim_{\point{#1} \to \point{#2}}} %Bart, I changed this to point since I want to use it.  It runs through both of the exercise and exerciseE files in limits section, which is why it was in each document to start with.

% \DeclareMathOperator{\proj}{\mathbf{proj}}
% \newcommand{\veci}{{\boldsymbol{\hat{\imath}}}}
% \newcommand{\vecj}{{\boldsymbol{\hat{\jmath}}}}
% \newcommand{\veck}{{\boldsymbol{\hat{k}}}}
% \newcommand{\vecl}{\vec{\boldsymbol{\l}}}
% \newcommand{\uvec}[1]{\mathbf{\hat{#1}}}
% \newcommand{\utan}{\mathbf{\hat{t}}}
% \newcommand{\unormal}{\mathbf{\hat{n}}}
% \newcommand{\ubinormal}{\mathbf{\hat{b}}}

% \newcommand{\dotp}{\bullet}
% \newcommand{\cross}{\boldsymbol\times}
% \newcommand{\grad}{\boldsymbol\nabla}
% \newcommand{\divergence}{\grad\dotp}
% \newcommand{\curl}{\grad\cross}
%\DeclareMathOperator{\divergence}{divergence}
%\DeclareMathOperator{\curl}[1]{\grad\cross #1}
% \newcommand{\lto}{\mathop{\longrightarrow\,}\limits}

% \renewcommand{\bar}{\overline}

\colorlet{textColor}{black}
\colorlet{background}{white}
\colorlet{penColor}{blue!50!black} % Color of a curve in a plot
\colorlet{penColor2}{red!50!black}% Color of a curve in a plot
\colorlet{penColor3}{red!50!blue} % Color of a curve in a plot
\colorlet{penColor4}{green!50!black} % Color of a curve in a plot
\colorlet{penColor5}{orange!80!black} % Color of a curve in a plot
\colorlet{penColor6}{yellow!70!black} % Color of a curve in a plot
\colorlet{fill1}{penColor!20} % Color of fill in a plot
\colorlet{fill2}{penColor2!20} % Color of fill in a plot
\colorlet{fillp}{fill1} % Color of positive area
\colorlet{filln}{penColor2!20} % Color of negative area
\colorlet{fill3}{penColor3!20} % Fill
\colorlet{fill4}{penColor4!20} % Fill
\colorlet{fill5}{penColor5!20} % Fill
\colorlet{gridColor}{gray!50} % Color of grid in a plot

\newcommand{\surfaceColor}{violet}
\newcommand{\surfaceColorTwo}{redyellow}
\newcommand{\sliceColor}{greenyellow}




\pgfmathdeclarefunction{gauss}{2}{% gives gaussian
  \pgfmathparse{1/(#2*sqrt(2*pi))*exp(-((x-#1)^2)/(2*#2^2))}%
}


%%%%%%%%%%%%%
%% Vectors
%%%%%%%%%%%%%

%% Simple horiz vectors
\renewcommand{\vector}[1]{\left\langle #1\right\rangle}


%% %% Complex Horiz Vectors with angle brackets
%% \makeatletter
%% \renewcommand{\vector}[2][ , ]{\left\langle%
%%   \def\nextitem{\def\nextitem{#1}}%
%%   \@for \el:=#2\do{\nextitem\el}\right\rangle%
%% }
%% \makeatother

%% %% Vertical Vectors
%% \def\vector#1{\begin{bmatrix}\vecListA#1,,\end{bmatrix}}
%% \def\vecListA#1,{\if,#1,\else #1\cr \expandafter \vecListA \fi}

%%%%%%%%%%%%%
%% End of vectors
%%%%%%%%%%%%%

%\newcommand{\fullwidth}{}
%\newcommand{\normalwidth}{}



%% makes a snazzy t-chart for evaluating functions
%\newenvironment{tchart}{\rowcolors{2}{}{background!90!textColor}\array}{\endarray}

%%This is to help with formatting on future title pages.
\newenvironment{sectionOutcomes}{}{}



%% Flowchart stuff
%\tikzstyle{startstop} = [rectangle, rounded corners, minimum width=3cm, minimum height=1cm,text centered, draw=black]
%\tikzstyle{question} = [rectangle, minimum width=3cm, minimum height=1cm, text centered, draw=black]
%\tikzstyle{decision} = [trapezium, trapezium left angle=70, trapezium right angle=110, minimum width=3cm, minimum height=1cm, text centered, draw=black]
%\tikzstyle{question} = [rectangle, rounded corners, minimum width=3cm, minimum height=1cm,text centered, draw=black]
%\tikzstyle{process} = [rectangle, minimum width=3cm, minimum height=1cm, text centered, draw=black]
%\tikzstyle{decision} = [trapezium, trapezium left angle=70, trapezium right angle=110, minimum width=3cm, minimum height=1cm, text centered, draw=black]


\title{Radical Analysis}

\begin{document}

\begin{abstract}
examples
\end{abstract}
\maketitle









We are building a library of the elemntary functions.  The idea is to use the library to list characteristics, features, and aspects of all functions within each category.  \\

That way, if we can identify the type of function we have, then we get free information when analyzing functions. \\

The category becomes our reasoning. \\



\begin{center}

\textbf{\textcolor{red!70!black}{These are ``CAN'' questions.}} \\

\end{center}




\textbf{\textcolor{purple!85!blue}{CAN}}the formula we are given be rewritten as one of the official standard forms for each category? \\











\begin{formula} \textbf{\textcolor{blue!55!black}{Radical/Root Functions}} 

A \textbf{radical} or \textbf{root function} is any function that \textbf{\textcolor{purple!85!blue}{CAN}} be represented with a formula of the form  

\[   rad(x) = A \sqrt[n]{B \, x + C} + D =  A (B \, x + C)^{\tfrac{1}{n}} + D    \]

where the $A$, $B$, $C$, and $D$ are real numbers and $A \ne 0$ and $B \ne 0$.

\end{formula}














\begin{example}

Comnpletely (Algebraically) analyze

\[
R(y) = -3 \sqrt{5-2y} + 4
\]









\textbf{\textcolor{blue!55!black}{$\blacktriangleright$ Category}} \\



$R(y) = -3 \sqrt{5-2y} + 4$ matches our official template, $A \sqrt[n]{B \, x + C} + D $, which makes it a radical function.  





\textbf{\textcolor{blue!55!black}{$\blacktriangleright$ Domain}} \\


$R(y)$ is an even root, which means its natural domain is all real numbers that make the inside nonegative.

The inside is $5-2y$, which is a linear function with a negative leading coefficient.  That tells us that it is nonnegative for numbers less than its zero.  Its zero is $\frac{5}{2}$.\\

The domain of $R$ is $\left( -\infty, \frac{5}{2} \right]$\\






\textbf{\textcolor{blue!55!black}{$\blacktriangleright$ Zeros}} \\

\[
R(y) = -3 \sqrt{5-2y} + 4 = 0
\]

\begin{align*}
-3 \sqrt{5-2y} + 4 &= 0 \\
\sqrt{5-2y} &= \frac{4}{3} \\
5 - 2y &= \frac{16}{9} \\
5 - \frac{16}{9} &= 2y  \\
\frac{1}{2} \left( 5 - \frac{16}{9} \right) &= y  
\end{align*}


$\frac{1}{2} \left( 5 - \frac{16}{9} \right) \approx 1.6111$, which agrees with the graph. \\














\textbf{\textcolor{blue!55!black}{$\blacktriangleright$ Continuity}} \\

$R$ is continuous, since it is a root function. \\






\textbf{\textcolor{blue!55!black}{$\blacktriangleright$ End-Behavior}} \\


As an even root, $R$ only has end-behavior in one direction. It is unbounded as $y$ approaches $-\infty$. \\

Since the leading coefficient is $-3 < 0$, R is unbounded negatively. \\


\[
\lim\limits_{y \to -\infty} R(y) = -\infty
\]












\textbf{\textcolor{blue!55!black}{$\blacktriangleright$ Behavior}} \\


A root or radical function is always increasing or always decreasing. \\


Since both the leading coefficient ($-3$) and the leading coefficient of the linear funciton inside the radical ($-2$) are neagtive, $R$ is increasing. \\

This agrees with the end-behavior. \\






\textbf{\textcolor{blue!55!black}{$\blacktriangleright$ Global Maximum and Minimum}} \\



The end-behavior tells us that there is no global minimum. And, since $R$ is always increasing, its global maximum value must occur at $\frac{5}{2}$


\[
R\left( \frac{5}{2}\right) = -3 (0) + 4 = 4
\]









\textbf{\textcolor{blue!55!black}{$\blacktriangleright$ Local Maximum and Minimum}} \\



As a radical function, $R$ has a single local maximum or minimum, which is the same as the global maximum or minimum. \\

Therefore, the only local maximum is $4$, which occurs at $\frac{5}{2}$ \\





\textbf{\textcolor{blue!55!black}{$\blacktriangleright$ Range}} \\


\begin{itemize}
\item $R$ is continuous
\item $\lim\limits_{y \to -\infty} R(y) = -\infty$
\item The global maximum is $4$
\end{itemize}


The range of $R$ is $(-\infty, 4]$










\textbf{\textcolor{blue!55!black}{$\blacktriangleright$ Graph}} \\




The graph of $z = R(y) = -3 \sqrt{5-2y} + 4$


\begin{image}
\begin{tikzpicture} 
  \begin{axis}[
            domain=-10:10, ymax=10, xmax=10, ymin=-10, xmin=-10,
            axis lines =center, xlabel=$y$, ylabel=$z$,
            ytick={-10,-8,-6,-4,-2,2,4,6,8,10},
            xtick={-10,-8,-6,-4,-2,2,4,6,8,10},
            ticklabel style={font=\scriptsize},
            every axis y label/.style={at=(current axis.above origin),anchor=south},
            every axis x label/.style={at=(current axis.right of origin),anchor=west},
            axis on top
          ]
          
          	%\addplot [line width=2, penColor, smooth, domain=(-9:0),<->] {(x-1)/((x+3)*(x-4))};
          	\addplot [line width=2, penColor, smooth, samples=200,domain=(-7:2.5),<-] {-3*sqrt(5-2*x)+4};
   
 			\addplot[color=penColor,fill=penColor,only marks,mark=*] coordinates{(2.5,4)};

           

  \end{axis}
\end{tikzpicture}
\end{image}





\begin{center}
\desmos{umcu66v9sr}{400}{300}
\end{center}


\end{example}























































\begin{center}
\textbf{\textcolor{green!50!black}{ooooo-=-=-=-ooOoo-=-=-=-ooooo}} \\

more examples can be found by following this link\\ \link[More Examples of Analysis]{https://ximera.osu.edu/csccmathematics/precalculus/precalculus/radicalFunctions/examples/exampleList}

\end{center}






\end{document}
