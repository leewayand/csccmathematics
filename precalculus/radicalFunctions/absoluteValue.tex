\documentclass{ximera}


\graphicspath{
  {./}
  {ximeraTutorial/}
  {basicPhilosophy/}
}

\newcommand{\mooculus}{\textsf{\textbf{MOOC}\textnormal{\textsf{ULUS}}}}


\usepackage{tkz-euclide}\usepackage{tikz}
\usepackage{tikz-cd}
\usetikzlibrary{arrows}
\tikzset{>=stealth,commutative diagrams/.cd,
  arrow style=tikz,diagrams={>=stealth}} %% cool arrow head
\tikzset{shorten <>/.style={ shorten >=#1, shorten <=#1 } } %% allows shorter vectors

\usetikzlibrary{backgrounds} %% for boxes around graphs
\usetikzlibrary{shapes,positioning}  %% Clouds and stars
\usetikzlibrary{matrix} %% for matrix
\usepgfplotslibrary{polar} %% for polar plots
\usepgfplotslibrary{fillbetween} %% to shade area between curves in TikZ
\usetkzobj{all}
\usepackage[makeroom]{cancel} %% for strike outs
%\usepackage{mathtools} %% for pretty underbrace % Breaks Ximera
%\usepackage{multicol}
\usepackage{pgffor} %% required for integral for loops



%% http://tex.stackexchange.com/questions/66490/drawing-a-tikz-arc-specifying-the-center
%% Draws beach ball
\tikzset{pics/carc/.style args={#1:#2:#3}{code={\draw[pic actions] (#1:#3) arc(#1:#2:#3);}}}



\usepackage{array}
\setlength{\extrarowheight}{+.1cm}
\newdimen\digitwidth
\settowidth\digitwidth{9}
\def\divrule#1#2{
\noalign{\moveright#1\digitwidth
\vbox{\hrule width#2\digitwidth}}}
























%%This is to help with formatting on future title pages.
\newenvironment{sectionOutcomes}{}{}


\title{Absolute Value}

\begin{document}

\begin{abstract}
traits
\end{abstract}
\maketitle





\begin{definition} \textbf{\textcolor{green!50!black}{Absolute Functions}}

Absolute functions are those functions that \textbf{\textcolor{purple!85!blue}{can}} be represented by formulas of the form


\[      AV(x) = A \cdot | B \, x + C| + D   \]

where $A$, $B$, $C$, and $D$ are real numbers, with $A \ne B$ and $B \ne 0$ is a nonzero real number.


\end{definition}


\textbf{Note:}  In the template for absolute value functions, There is a leading coefficient for the function and there is a leading coefficient for the linear function inside the absolute value bars. \\






\subsection*{Domain}

The domain of every absolute value function is all real numbers.





\subsection*{Zeros}

Absolute value functions have two zeros or no zeros.





\subsection*{Continuity}

Absolute value functions are continuous functions.








\subsection*{End-Behavior}


The end-behavior is the same on both sides for an absolute value function. The sign of the leading coefficient determines which way it goes.\\


If the leading coefficent is positive, then 
\[
\lim_{x \to -\infty} AV(x) = \infty 
\]
 
\[
\lim_{x \to \infty} AV(x) = \infty 
\]







If the leading coefficent is negative, then 
\[
\lim_{x \to -\infty} AV(x) = -\infty 
\]
 
\[
\lim_{x \to \infty} AV(x) = -\infty 
\]











\subsection*{Behavior}



Formulas for absolute value functions look like

\[     AV(x) =    A  | B \, x + C | + D           \]

where $A \ne 0$, $B \ne 0$, $C$, and $D$ are real numbers.





The critical number for an absolute value function is the domain number that makes the inside of the absolute value bars equal to $0$.


\[
B \, x + C = 0
\]


The absolute value function increases on one side of the critical number and decreases on the other.  The sign of the leading coefficient determines which way it goes. 




\begin{itemize}
  \item If the leading coefficent is postive, then decreasing switching to increasing.
  \item If the leading coefficent is negative, then increasing switching to decreasing.
\end{itemize}











\subsection*{Global Maximum and Minimum}


Absolute value function have either a global maximum or a global minimum.  The sign of the leading coefficient tells us which.







\begin{itemize}
  \item If the leading coefficent is postive, then there is a global minimum.
  \item If the leading coefficent is negative, then there is a global maximum.
\end{itemize}














\subsection*{Local Maximums and Minimums}


Absolute value function have either one local maximum or one local minimum.  It is the same as the global extrema.

















\subsection*{Range}


The range of an abvsolute value function is unbounded in one direction.


\begin{itemize}
  \item If the leading coefficent is postive, then range is of the form $[m, \infty)$.
  \item If the leading coefficent is negative, then range is of the form $(-\infty, m]$.
\end{itemize}











\begin{center}
\textbf{\textcolor{green!50!black}{ooooo-=-=-=-ooOoo-=-=-=-ooooo}} \\

more examples can be found by following this link\\ \link[More Examples of Analysis]{https://ximera.osu.edu/csccmathematics/precalculus/precalculus/radicalFunctions/examples/exampleList}

\end{center}





\end{document}
