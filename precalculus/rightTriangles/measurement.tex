\documentclass{ximera}


\graphicspath{
  {./}
  {ximeraTutorial/}
  {basicPhilosophy/}
}

\newcommand{\mooculus}{\textsf{\textbf{MOOC}\textnormal{\textsf{ULUS}}}}


\usepackage{tkz-euclide}\usepackage{tikz}
\usepackage{tikz-cd}
\usetikzlibrary{arrows}
\tikzset{>=stealth,commutative diagrams/.cd,
  arrow style=tikz,diagrams={>=stealth}} %% cool arrow head
\tikzset{shorten <>/.style={ shorten >=#1, shorten <=#1 } } %% allows shorter vectors

\usetikzlibrary{backgrounds} %% for boxes around graphs
\usetikzlibrary{shapes,positioning}  %% Clouds and stars
\usetikzlibrary{matrix} %% for matrix
\usepgfplotslibrary{polar} %% for polar plots
\usepgfplotslibrary{fillbetween} %% to shade area between curves in TikZ
\usetkzobj{all}
\usepackage[makeroom]{cancel} %% for strike outs
%\usepackage{mathtools} %% for pretty underbrace % Breaks Ximera
%\usepackage{multicol}
\usepackage{pgffor} %% required for integral for loops



%% http://tex.stackexchange.com/questions/66490/drawing-a-tikz-arc-specifying-the-center
%% Draws beach ball
\tikzset{pics/carc/.style args={#1:#2:#3}{code={\draw[pic actions] (#1:#3) arc(#1:#2:#3);}}}



\usepackage{array}
\setlength{\extrarowheight}{+.1cm}
\newdimen\digitwidth
\settowidth\digitwidth{9}
\def\divrule#1#2{
\noalign{\moveright#1\digitwidth
\vbox{\hrule width#2\digitwidth}}}
























%%This is to help with formatting on future title pages.
\newenvironment{sectionOutcomes}{}{}


\title{Measurement}

\begin{document}

\begin{abstract}
rate, length, and time
\end{abstract}
\maketitle




\begin{center}
\textbf{\textcolor{purple!85!blue}{Distance = Rate $\cdot$ Time}} 
\end{center}

This relationship is extremely common in our situations.  We might see it as



\begin{center}
\textbf{\textcolor{purple!85!blue}{Length = Rate $\cdot$ Time}} 
\end{center}



However, these give the different perspective from what we are studying.  These give an accumulation perspective. You  know the elapsed time you can calculate the accumulated distance.  Our focus, right now, is on rates of change. 


The rate is our focus and it is a comparison measurement between \textbf{\textcolor{red!80!black}{change in length}} and \textbf{\textcolor{red!80!black}{change in time}}.



\[
\text{rate} = \frac{\Delta \text{length}}{\Delta \text{time}} = \frac{\Delta \text{distance}}{\Delta \text{time}}
\]

\textbf{Note:} $\Delta$ is our shorthand notation for ``change in''. \






\subsection*{Airplane}

An airplane is flying overhead on a level flight path, 5 miles above the ground.  The plane is travelling at a constant speed and will travel directly over a tracking station. The tracking station's radar antennea measures the distance from the station to the plane.






\textbf{\textcolor{purple!85!blue}{Step 1: A Picture}}


\begin{image}
\includegraphics{pics/plane_1.png}
\end{image}




\textbf{\textcolor{purple!85!blue}{Step 2: Identify Geometric Objects}}



\begin{image}
\includegraphics{pics/plane_2.png}
\end{image}







\textbf{\textcolor{purple!85!blue}{Step 3: Identify Lengths}}

$\blacktriangleright$ We have three pertinent lengths:

\begin{itemize}
\item $D$ is the distance between the station and the plane.
\item $F$ is the flight distance between the plane the point directly above the station.
\item $H$ is the height of the flight path above the station.
\end{itemize}

All of these are functions of time, $t$: $D(t)$, $F(t)$, and $H(t)$.


\begin{question} 


Which distance measurement is a constant function?

\begin{multipleChoice}
\choice {$D$}
\choice {$F$}
\choice[correct] {$H$}
\end{multipleChoice}

\end{question}



\begin{question} 


Is $D(t)$ an increasing or decreasing function with respect to $t$?

\begin{multipleChoice}
\choice {Increasing}
\choice[correct] {Decreasing}
\end{multipleChoice}

\end{question}



\begin{question} 


Is $F(t)$ an increasing or decreasing function with respoect to $t$?

\begin{multipleChoice}
\choice {Increasing}
\choice[correct] {Decreasing}
\end{multipleChoice}

\end{question}










\textbf{\textcolor{purple!85!blue}{Step 4: Relationships}}


An airplane is flying overhead on a level flight path, 5 miles above the ground.  The plane is travelling at a constant speed and will travel directly over a tracking station. The tracking station's radar antennea measures the distance from the station to the plane. If the distance between the station and the plane is decreasing at a rate of $350$ miles per hour when that distance is $10$ miles, then what is the speed of the plane?


\begin{image}
\includegraphics{pics/plane_2.png}
\end{image}





\begin{question} 


What geometric shape connects these measurements?

\begin{multipleChoice}
\choice {Rectangle}
\choice[correct] {Right Triangle}
\choice {Circle}
\end{multipleChoice}

\end{question}







\begin{question} 


What geometric relationship connects these measurements?

\begin{multipleChoice}
\choice[correct] {Pythagorean Theorem}
\choice {Circumference}
\choice {Similar Triangles}
\end{multipleChoice}

\end{question}






\begin{question} 


Which relationship is suggested by the diagram?

\begin{multipleChoice}
\choice {$D^2 + F^2 = H^2$}
\choice[correct] {$H^2 + F^2 = D^2$}
\choice {$H^2 + D^2 = F^2$}
\end{multipleChoice}

\end{question}






\begin{question} 


$350$ miles per hour is the rate of change of which measurement?

\begin{multipleChoice}
\choice[correct] {$D$}
\choice {$F$}
\choice {$H$}
\end{multipleChoice}

\end{question}



















\subsection*{Angles}



We have other measurements, besides length, that might be changing in a situation.




\[
\text{rate} = \frac{\Delta \text{angle}}{\Delta \text{time}} 
\]

\textbf{Note:} $\Delta \text{angle}$ might be measured in degrees or radians. Calculus will use radians.






\subsection*{Airplane}

An airplane is flying overhead on a level flight path, 5 miles above the ground.  The plane is travelling at a constant speed and will travel directly over a tracking station. The tracking station's radar antennea measures the distance from the station to the plane. If the distance between the station and the plane is decreasing at a rate of $350$ miles per hour when that distance is $10$ miles, then what is the speed of the plane?









\textbf{\textcolor{purple!85!blue}{Step 1: A Picture}}


\begin{image}
\includegraphics{pics/plane_1.png}
\end{image}




\textbf{\textcolor{purple!85!blue}{Step 2: Identify Measurements}}



\begin{image}
\includegraphics{pics/plane_3.png}
\end{image}










$\blacktriangleright$ We have two pertinent angles:

Angles $\alpha$ and $\beta$ both are functions of $t$: $\alpha(t)$ and $\beta(t)$.




\begin{itemize}
\item $\alpha$ is the angle of elevation from the station to the plane.
\item $\beta$ is the angle of depression from the plane to the station.
\end{itemize}








\begin{question} 


Is $\alpha(t)$ an increasing or decreasing function with respect to $t$?

\begin{multipleChoice}
\choice {Increasing}
\choice[correct] {Decreasing}
\end{multipleChoice}

\end{question}








\begin{question} 


Is $\beta(t)$ an increasing or decreasing function with respect to $t$?

\begin{multipleChoice}
\choice[correct] {Increasing}
\choice {Decreasing}
\end{multipleChoice}

\end{question}







These angles are related to the sides of the triangle through sine and cosine.










\begin{question} 


Which expression represents $\sin(\alpha)$?

\begin{multipleChoice}
\choice {$\frac{F}{H}$}
\choice {$\frac{D}{F}$}
\choice[correct] {$\frac{F}{D}$}
\choice {$\frac{H}{D}$}
\end{multipleChoice}

\end{question}









\begin{question} 


Which expression represents $\cos(\beta)$?

\begin{multipleChoice}
\choice {$\frac{F}{H}$}
\choice {$\frac{D}{F}$}
\choice[correct] {$\frac{F}{D}$}
\choice {$\frac{H}{D}$}
\end{multipleChoice}

\end{question}




























\subsection*{Volume}







We can also measure volume and how it changes.



\[
\text{rate} = \frac{\Delta \text{volume}}{\Delta \text{time}} 
\]








\subsection*{Conical Tank}

A conical tank (with vertex down) is 12 feet across and 14 fee deep. If water is flowing into the tank at a rate of 9 cubic feet per minute, find the rate of change of the depth of the water when the water is 8 feet deep.






\textbf{\textcolor{purple!85!blue}{Step 1: A Picture}}


\begin{image}
\includegraphics{pics/cone_1.png}
\end{image}




\textbf{\textcolor{purple!85!blue}{Step 2: Identify Measurements}}






\begin{image}
\includegraphics{pics/cone_3.png}
\end{image}






$\blacktriangleright$ We begin with two pertinent lengths that remain constant throughout the story:

\begin{itemize}
\item $R$ is the radius of the base of the conical tank.
\item $H$ is the height of the conical tank.
\end{itemize}

These are constant functions of time, $t$: $R(t)$ and $H(t)$.







\begin{question} 


As water pours into the tank, what shape does the water take?

\begin{multipleChoice}
\choice {Triangle}
\choice {Sphere}
\choice[correct] {Cone}
\choice {Rectangular Prism}
\end{multipleChoice}

\end{question}









\begin{question} 


As water pours into the tank, how does the height of the water change?

\begin{multipleChoice}
\choice[correct] {Increases}
\choice {Decreases}
\choice {Remains Constant}
\end{multipleChoice}

\end{question}







\begin{question} 


As water pours into the tank, how does the radius of the water base change?

\begin{multipleChoice}
\choice[correct] {Increases}
\choice {Decreases}
\choice {Remains Constant}
\end{multipleChoice}

\end{question}









\textbf{\textcolor{purple!85!blue}{Step 3: Water Measurements}}


As water pours into the tank the water level rises.  The water also forms a cone and the height and radius both increase.  








\begin{image}
\includegraphics{pics/cone_4.png}
\end{image}





$\blacktriangleright$ We need to represent these measurements as well:

\begin{itemize}
\item $r$ is the radius of the base of the conical water.
\item $h$ is the height of the conical water.
\end{itemize}

These are increasing functions of time, $t$: $r(t)$ and $h(t)$.










\textbf{\textcolor{purple!85!blue}{Step 4: Geometric Relationships}}


Our measurements are vertical and horizontal measurements.  If we look at these from the side of the tank, they make a right triangle.  In fact, they make two similar right triangles.






\begin{image}
\includegraphics{pics/cone_5.png}
\end{image}




$\blacktriangleright$ Similar Triangles



\[
\frac{H}{R} = \frac{h}{r}
\]


From this relationship, we can get expressions for $h$ or $r$ in terms of the others. 



\[
h = \frac{r \, H}{R} \, \text{ and } \,  r = \frac{h \, R}{H}
\]







$\blacktriangleright$ Volume


We are also given information about the volume.  Therefore, we should be thinking about the volume of cones.

\[
V = \frac{1}{3} \pi \, r^2 \, h
\]







We are just modeling here, which means collecting as many descriptions and relationships as we can. 




We can make a model for the volume of the water in terms of $h$:

\[  V(h) = \frac{1}{3}  \pi \, \left( \frac{h \, R}{H} \right)^2 \, h    \]



We can make a model for the volume of the water in terms of $r$:

\[  V(r) = \frac{1}{3}  \pi \, r^2 \, \frac{r \, H}{R}    \]





With these models we can consider the original question.  We'll leave this for later.




\subsection*{Tides}





The following data was collected recording the height of tides in Austraila (NSW Dept of Education).




\[
\begin{array}{ll}
\text{Time (hrs)}  & \text{Height (m)}  \\
0      &  1.63  \\
6.4    &  0.64  \\
12.2   &  1.36  \\
18.1   &  0.53  \\
24.9   &  1.69  \\
31.4   &  0.58  \\
37.1   &  1.32
\end{array}
\]


As expected, the heights seem to be periodic.  This suggests that a good model might be a sine or cosine function.  Since there seems to be a maximum height at time $0$, a cosine model is a good choice.




In the following DESMOS app, select choices for $A$, $B$, $C$, and $D$ to fit the model to the data.

\begin{center}
\desmos{9pmszixak2}{400}{300}
\end{center}


The best model is approximately

\[   H(t) = 0.495 \cos(0.515 t) 1.135       \]







\begin{image}
\includegraphics{pics/tides.png}
\end{image}


However, it doesn't do such a good job. It looks like every other high tide height is missed.


If we graph more of the data, we can see this better.





\begin{image}
\includegraphics{pics/tides2.png}
\end{image}



There appear to be two sinusoidal waves of different heights.  

A better model might involve two cosine functions. 


Our current model seems to be doing half the job.  It appears we need to shorten every other wave.  We can do this with another cosine function with twice the period. 



When the model was a single cosine function, it appear that the period was approximately $12.2$ hours, which gives a frequency of $\frac{2 \pi}{12.2} = \frac{\pi}{6.1} = 0.515$, which was the coefficient inside the cosine function.  



\begin{question}  Period

If we switch the model to two cosine functions, then the period of the second cosine would be $\answer[tolerance=0.1]{24.4}$ hours.


\end{question}




\begin{question}  Period

If the period is $24.4$ hours, then the frequency is $\frac{2 \pi}{\answer{24.4}} = \frac{\pi}{\answer{12.2}}$.


\end{question}









Our new model begins to look like:

\[   H(t) = A \cos\left( \frac{\pi \, t}{6.1} \right) + B \cos\left( \frac{\pi \, t}{12.2} \right) + C       \]


We have three unknowns. 

We need three data points. 


We'll select the first three peaks/troughs:  $(0, 1.63)$, $(6.4, 0.64)$, and $(12.2, 1.36)$. 



\textbf{\textcolor{blue!55!black}{$\blacktriangleright$}} $first(t) = \cos\left( \frac{\pi \, t}{6.1} \right)$ is the faster piece. 


\textbf{\textcolor{blue!55!black}{$\blacktriangleright$}} $second(t) = \cos\left( \frac{\pi \, t}{12.2} \right)$ is the slower piece. 




The second cosine function is slower function and has double the period as the first.  They both start at maximums at $t=0$.  Both functions have a value of $1$. 

Then, the first cosine function arrive at its minimum of $-1$ at $t=6.4$, however, the second function is slower and is at a zero when $t=6.4$.  

At $t=12.2$ hours the first cosine is at it maximum of $1$ and the second function has arrived at its minimum of $-1$.






\begin{itemize}
\item At $(0, 1.63)$  we have $1.63 = A + B + C$
\item At$(6.4, 0.64)$ we have $0.64 = -A + C$
\item At $(12.2, 1.36)$  we have $1.36 = A - B + C$
\end{itemize}




From these three equations, we can see that $1.63 + 1.36 = 2 \, A + 2 \, C$   or $1.495 = A + C$.  \\

This and the second equation tell us that $0.64 + 1.495 = 2 \, C$ or $C = 1.068$. 

We can put this in for $C$, in the second equation: $0.64 = -A + 1.068$ or $A = 0.4275$.

And, from all of this we get $B = 0.1345$.






\begin{center}
\desmos{ufauqfzfgk}{400}{300}
\end{center}

Much better.
























\begin{center}
\textbf{\textcolor{green!50!black}{ooooo-=-=-=-ooOoo-=-=-=-ooooo}} \\

more examples can be found by following this link\\ \link[More Examples of Modeling]{https://ximera.osu.edu/csccmathematics/precalculus2/precalculus2/modeling/examples/exampleList}

\end{center}






\end{document}
