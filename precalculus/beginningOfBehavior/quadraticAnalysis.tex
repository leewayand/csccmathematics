\documentclass{ximera}


\graphicspath{
  {./}
  {ximeraTutorial/}
  {basicPhilosophy/}
}

\newcommand{\mooculus}{\textsf{\textbf{MOOC}\textnormal{\textsf{ULUS}}}}


\usepackage{tkz-euclide}\usepackage{tikz}
\usepackage{tikz-cd}
\usetikzlibrary{arrows}
\tikzset{>=stealth,commutative diagrams/.cd,
  arrow style=tikz,diagrams={>=stealth}} %% cool arrow head
\tikzset{shorten <>/.style={ shorten >=#1, shorten <=#1 } } %% allows shorter vectors

\usetikzlibrary{backgrounds} %% for boxes around graphs
\usetikzlibrary{shapes,positioning}  %% Clouds and stars
\usetikzlibrary{matrix} %% for matrix
\usepgfplotslibrary{polar} %% for polar plots
\usepgfplotslibrary{fillbetween} %% to shade area between curves in TikZ
\usetkzobj{all}
\usepackage[makeroom]{cancel} %% for strike outs
%\usepackage{mathtools} %% for pretty underbrace % Breaks Ximera
%\usepackage{multicol}
\usepackage{pgffor} %% required for integral for loops



%% http://tex.stackexchange.com/questions/66490/drawing-a-tikz-arc-specifying-the-center
%% Draws beach ball
\tikzset{pics/carc/.style args={#1:#2:#3}{code={\draw[pic actions] (#1:#3) arc(#1:#2:#3);}}}



\usepackage{array}
\setlength{\extrarowheight}{+.1cm}
\newdimen\digitwidth
\settowidth\digitwidth{9}
\def\divrule#1#2{
\noalign{\moveright#1\digitwidth
\vbox{\hrule width#2\digitwidth}}}
























%%This is to help with formatting on future title pages.
\newenvironment{sectionOutcomes}{}{}


\title{Quadratic Analysis}

\begin{document}

\begin{abstract}
behavior
\end{abstract}
\maketitle




 
 



\section*{Quadratic Analysis}

What do we want to know when we analyze any function?

We want to know the 
\begin{itemize}
     \item \textbf{\textcolor{red!80!black}{Domain}} 
     \item \textbf{\textcolor{red!80!black}{Zeros}} 
     \item \textbf{\textcolor{red!80!black}{Continuity}} 
\begin{itemize}
     \item \textbf{\textcolor{purple!85!blue}{discontinuities}} 
     \item \textbf{\textcolor{purple!85!blue}{singularities}} 
\end{itemize}
     \item \textbf{\textcolor{red!80!black}{End-Behavior}} 
     \item \textbf{\textcolor{red!80!black}{Behavior}} 
\begin{itemize}
     \item \textbf{\textcolor{purple!85!blue}{intervals where increasing}} 
     \item \textbf{\textcolor{purple!85!blue}{intervals where decreasing}} 
\end{itemize}
     \item \textbf{\textcolor{red!80!black}{Global Maximum and Minimum}} 
     \item \textbf{\textcolor{red!80!black}{Local Maximums and Minimums}} 
     \item \textbf{\textcolor{red!80!black}{Range}} 
     \item \textbf{\textcolor{blue!55!black}{...and we would like a nice graph}} 
\end{itemize}

We want all of this information for quadratic functions and we want exact information, not approximations. \\

Remember, we are beginning with graphical analysis, because that is a familiar jumping off point for students.  But, graphs are inherently inaccurate tools.  That isn't what we want. We are taking our familiarity with graphical descriptions and moving them over to algebraic descriptions, because algebra is our exact tool. \\

Linear and quadratic functions are our first bridges to exactness. \\


Quadratic functions are those functions which can be described with formulas like

\begin{itemize}
     \item $A \, x^2 + B \, x + C$
     \item $A \, (x - H)^2 + K$
     \item $A \, (x - r_1) (x-r_2)$
\end{itemize}

For a quadratic function, much of the analysis information is connected to vertex of the graph, which is why we like the vertex form for a formula, which is why we like completing the square.  But, we can get all of our information from the standard or factored forms as well. \\

We want to keep the graph in our heads, but translate to function and algebraic reasoning. \\





\subsection*{Domain} 

All quadratic functions are defined for all real numbers.  Their natural domain is $\mathbb{R}$. \\

If you can identify a function as quadratic, then you automatically know its domain. \\











\subsection*{Zeros}


The quadratic formula gives the zeros of a quadratic function, when we have the \textit{standard form}. The quadratic formula gives the solutions to the quadratic equation

\[
a \, t^2 + b \, t + c = 0
\]

\[ t  =   \frac{-b \pm \sqrt{b^2 - 4 a c}}{2a}      \]



which we can separate into 



\[ t  =   \frac{-b}{2a} \pm \frac{\sqrt{b^2 - 4 a c}}{2a}      \]


The $\pm$ shows that the zeros are symmetric about $\frac{-b}{2a}$, which means that the intercepts are symmetric about the vertical line $t = \frac{-b}{2a}$. \\

Intercepts are $\left(  \frac{-b - \sqrt{b^2 - 4 a c}}{2a}, 0 \right)$ and $\left(  \frac{-b + \sqrt{b^2 - 4 a c}}{2a}, 0 \right)$.





\begin{image}
\begin{tikzpicture}
     \begin{axis}[
                domain=-10:10, ymax=10, xmax=10, ymin=-10, xmin=-10,
                axis lines =center, xlabel=$t$, ylabel=$y$,
                ytick={-10,-8,-6,-4,-2,2,4,6,8,10},
                xtick={-10,-8,-6,-4,-2,2,4,6,8,10},
                ticklabel style={font=\scriptsize},
                every axis y label/.style={at=(current axis.above origin),anchor=south},
                every axis x label/.style={at=(current axis.right of origin),anchor=west},
                axis on top,
                ]


        \addplot [draw=penColor, very thick, smooth, domain=(1:5),<->] {-2*(x-3)^2 + 4};
        \addplot [line width=1, gray, dashed,samples=100,domain=(-9.5:9.5)] ({3},{x});
        %\addplot [color=penColor2,only marks,mark=*] coordinates{(3,4)};

        \addplot [color=penColor2,only marks,mark=*] coordinates{(1.585,0)};
        \addplot [color=penColor2,only marks,mark=*] coordinates{(4.414,0)};
        


        %\node[penColor] at (axis cs:5,4) {$(h, k))$};
        %\node[penColor] at (axis cs:5,-9) {$-0.5 x^2 - 5 x + 15.5$};



    \end{axis}
\end{tikzpicture}
\end{image}

\textbf{Graph Description:} Cartesian plane. Horizontal axis labeled $t$. Vertical axis labeled $y$. Parabola opening down with vertex at $(3,4)$ and $t$-intercepts at $(1.6,0)$ and $(4.4,0)$. There is a dashed vertical line through the vertex. \\



Working from a formula to a graph, we can see that if we have the zeros or roots, like from factoring or the quadratic formula, then we have the intercepts, and the \textbf{line of symmetry} must run in the middle. \\


On the other hand, if we know the coordinates of the intercepts of a parabola, then we know the zeros or roots, which means we know the factors of the quadratic.




\begin{idea}

zeros (roots), factors, and intercepts all describe the same information.


\end{idea}






\subsection*{Continuity}

Quadratic functions are continuous functions.  They have no discontinuities or singularities. \\












\subsection*{End-Behavior}

Quadratic functions have the same end-behavior on both sides, which is given by the sign of the leading coefficient. \\


End-behavior describes what the function is doing out in the ``tails'' of the domain.  That is where the domain numbers are really really really really big positively or negatively. \\

For quadratic functions, $f(x) = a \, x^2 + b \, x + c$, when $x$ is really really really really big positively or negatively, then the leading term ``dominates'' the other two terms.  The whole function behaves just like $a \, x^2$. \\

That tells us that the whole function will become unbounded. \\


\begin{itemize}
     \item Quadratic functions become unbounded positively if the leading coefficient is positive.
     \item Quadratic functions become unbounded negatively if the leading coefficient is negative.
\end{itemize}



\begin{idea} \textbf{\textcolor{red!80!black}{A Peek Ahead}}

We will want some mathematical notation for end-behavior.  \textbf{Limit notation} will be introduced later and that will be our way of algebraically describing end-behavior. \\




\begin{example}

Let $g(t) = -2 \, t^2 + 7 \, t - 3$. \\


The end-behavior will be described as 

\[
\lim\limits_{t \to -\infty} g(t) = -\infty
\]


\[
\lim\limits_{t \to \infty} g(t) = -\infty
\]

\end{example}




\end{idea}

















\subsection*{Behavior}



\textbf{\textcolor{blue!55!black}{Increasing and Decreasing}}






The graph vividly suggests that quadratic functions switch from increasing to decreasing (or vice versa) at the ``vertex'' number in the domain, which is called the \textit{critical number}.


\begin{itemize}
\item If the leading coefficient is positive, (then the graph is opening up) then the quadratic function is \\

decreasing on $\left( -\infty, \frac{-b}{2a} \right)$ and increasing on $\left( \frac{-b}{2a}, \infty \right)$

\item If the leading coefficient is negative, (then graph is opening down) then the quadratic function is \\

increasing on $\left( -\infty, \frac{-b}{2a} \right)$ and decreasing on $\left( \frac{-b}{2a}, \infty \right)$
\end{itemize}





Increasing and decreasing refer to the rate of change.


\begin{itemize}
\item Increasing is a positive rate of change. (The domain and function values change in the same way.)
\item Decreasing is a negative rate of change. (The domain and function values change in the opposite way.)
\end{itemize}



Now, we can replace our graphical intuition with algebraic rigor. \\ 

We have seen if we write a quadratic function as $f(x) = a (x - h)^2 + k$, then the instantaneous rate of change of $f$ is the linear function $iRoC_f(x) = 2 a (x - h)$. The values of $iRoC$ are the slopes of the lines tangent to the parabola.


Since $iRoC_f$ is a linear function, its graph is a line.


Here is a graph of both the parabola for $f$ and the line for $iRoC_f = f'$.







\begin{image}
\begin{tikzpicture}
     \begin{axis}[
                domain=-10:10, ymax=10, xmax=10, ymin=-10, xmin=-10,
                axis lines =center, xlabel=$x$, ylabel=$y$,
                ytick={-10,-8,-6,-4,-2,2,4,6,8,10},
                xtick={-10,-8,-6,-4,-2,2,4,6,8,10},
                ticklabel style={font=\scriptsize},
                every axis y label/.style={at=(current axis.above origin),anchor=south},
                every axis x label/.style={at=(current axis.right of origin),anchor=west},
                axis on top,
                ]


        \addplot [draw=penColor, very thick, smooth, domain=(1:5),<->] {2*(x-3)^2 - 4};
        \addplot [line width=1, gray, dashed,samples=100,domain=(-9.5:9.5)] ({3},{x});
        \addplot [color=penColor,only marks,mark=*] coordinates{(3,-4)};
        
        \addplot [draw=penColor2, very thick, smooth, domain=(1:5),<->] {4*(x-3)};

        \node[penColor] at (axis cs:5,-4) {$(h, k)$};
        %\node[penColor] at (axis cs:5,-9) {$-0.5 x^2 - 5 x + 15.5$};



    \end{axis}
\end{tikzpicture}
\end{image}
\textbf{Graph Description:} Cartesian plane. Horizontal axis labeled $x$. Vertical axis labeled $y$. Parabola opening up with vertex at $(3,-4)$ and $x$-intercepts at $(1.6,0)$ and $(4.4,0)$. There is a dashed vertical line through the vertex. There is a line of positive slope going through $(3,0)$.\\



Our linear rate of change function now informs us about the behavior of $f$. \\





\begin{conclusion}  \textbf{\textcolor{green!50!black}{Behavior}}

$\blacktriangleright$ When the instantaneous rate of change function is negative, $iRoC_f(x) < 0$, then $f(x)$ is decreasing. \\

$\blacktriangleright$ When the instantaneous rate of change function is positive, $iRoC_f(x) > 0$, then $f(x)$ is increasing. \\

$\blacktriangleright$ When the instantaneous rate of change function is zero, $iRoC_f(x) = 0$, then $f(x)$ is neither increasing nor decreasing and the graph of $y = f(x)$ is flat. 


We can say the exat same thing with the word ``derivative''. \\



$\blacktriangleright$ When the derivative is negative, $f'(x) < 0$, then $f(x)$ is decreasing. \\

$\blacktriangleright$ When the derivative is positive, $f'(x) > 0$, then $f(x)$ is increasing. \\

$\blacktriangleright$ When the derivative is zero, $f'(x) = 0$, then $f(x)$ is neither increasing nor decreasing and the graph of $y = f(x)$ is flat or horizontal. 



\end{conclusion}


As we can see, the behavior of our function, $f(x)$, can change drastically where $iRoC_f(x) = 0$.  Such domain numbers deserve a special name.



\begin{definition} \textbf{\textcolor{green!50!black}{Critical Number}}  


Let $f$ be a function. Let $x_0$ be a number in the domain of $f$ such that $iRoC_f(x_0) = f'(x) = 0$ or $iRoC_f(x_0) = f'(x)$ does not exist.

Then $x_0$ is called a \textbf{critical number}.


\end{definition}

\textbf{Note: } Domain numbers where $iRoC_f$ or $f'$ doesn't exist are also places where a function's behavior can change drastically.

\textbf{Note: } Singularities are not in the domain. However, a function's behavior can also change across singularities. (Quadratics don't have singularities.)


\begin{procedure} \textbf{\textcolor{blue!75!black}{iRoC for Quadratic Functions in Vertex Form}} 



Let $Q$ be a quadratic function.

Then $Q(x) = a (x - h)^2 + k$, for some $a$, $h$, and $k$ with $a \ne 0$. \\

Then, $iRoC_Q(x) = 2 a (x - h)$. \\


\textbf{Procedure:} It appears that the $2$ in the exponent has slid down in front and is multiplying the leading coefficient while the constant term, $k$, has been removed.



\end{procedure}



We have a procedure for obtaining the $iRoC$ of a quadratic function, when the formula is in vertex form (completed square form). \\

\textbf{\textcolor{blue!55!black}{$\blacktriangleright$}} What about standard form? \\
We can get standard form by multiplying out vertex form. \\



\begin{align*}
Q(x) & = a (x - h)^2 + k \\
     & = a \, x^2 - 2 \, a \, h \, x + a \, h^2 \, k \\
     & = a \, x^2  + (- 2 \, a \, h) x + (a \, h^2 \, k) 
\end{align*}


\begin{align*}
iRoC_Q(x) &= 2 a (x - h) \\
          & = 2 \, a \, x - 2 \, a \, h  \\
\end{align*}


Comparing this to standard form, $a \, x^2 + b \, x + c$, tells us that $-2ah = b$. \\


\begin{procedure} \textbf{\textcolor{blue!75!black}{iRoC for Quadratic Functions in Standard Form}} 



Let $Q$ be a quadratic function.

Then $Q(x) = a \, x^2 + b \, x + c$, for some $a$, $b$, and $c$ with $a \ne 0$. \\

Then, $iRoC_Q(x) = 2 \, a \, x + b$. \\


\textbf{Procedure:}. It appears that the $2$ in the exponent has slid down in front and is multiplying the leading coefficient and just the linear coefficent remains. 



\end{procedure}

There is probably an overall pattern going on here, which will be revealed in Calculus. \\












\subsection*{Maximums and Minimums}




The maximum and minimum values of a quadratic function, $f$, are visually encoded in the coordinates of the highest or lowest points on the  graph, which is the vertex of the parabola.


\begin{itemize}
     \item Depending on the sign of the leading coefficient, the maximum or minimum value of $f(x) = a \, x^2 + b \, x + c$ occurs at $\frac{-b}{2a}$. 


     \item Depending on the sign of the leading coefficient, the maximum or minimum value of $f(x) = a \, (x - h)^2 + k$ is $k$ and occurs at $h = \frac{-b}{2a}$. 
\end{itemize}

Either way, the maximum or minimum value is $f\left( \frac{-b}{2a} \right)$




We can also look at the linear $iRoC_f(x)$ function.  Where $iRoC_f(x) = 0$ is where the vertex is located, which encodes the maximum or minimum value of $f$.





\begin{align*}
iRoC_Q(x)       &= 0  \\
2 \, a \, x + b  & = 0  \\
x     &=  \frac{-b}{2a}
\end{align*}




\textbf{\textcolor{blue!55!black}{$\blacktriangleright$ $\frac{-b}{2a}$ is the critical number for a quadratic function given in standard form: $a \, x^2 + b \, x + c$.}}  



\textbf{\textcolor{blue!55!black}{$\blacktriangleright$ $h$ is the critical number for a quadratic function given in vertex form: $a \, (x - h)^2 + c$.}}  



\textbf{\textcolor{blue!55!black}{$\blacktriangleright$ $\frac{r_1 + r_2}{2}$ is the critical number for a quadratic function given in factored form: $a \, (x - r_1) (x - r_2)$.}}  

















\subsection*{Range}

$\blacktriangleright$  \textbf{Vertex Form} \\

The graph of a quadratic function is a parabola, which is easily connected to the completed square form of the formula (vertex form).

Below is the graph of $y = f(x) = a (x - h)^2 + k$, with $a$, and $h$, and $k$ all real numbers and $a > 0$. The extreme point is called the \textbf{vertex}. If $a<0$, then the parabola opens downward and the extreme point is at the top.

\begin{image}
\begin{tikzpicture}
     \begin{axis}[
                domain=-10:10, ymax=10, xmax=10, ymin=-10, xmin=-10,
                axis lines =center, xlabel=$x$, ylabel=$y$,
                ytick={-10,-8,-6,-4,-2,2,4,6,8,10},
                xtick={-10,-8,-6,-4,-2,2,4,6,8,10},
                ticklabel style={font=\scriptsize},
                every axis y label/.style={at=(current axis.above origin),anchor=south},
                every axis x label/.style={at=(current axis.right of origin),anchor=west},
                axis on top,
                ]


        \addplot [draw=penColor, very thick, smooth, domain=(1:5),<->] {2*(x-3)^2 - 4};
        \addplot [line width=1, gray, dashed,samples=100,domain=(-9.5:9.5)] ({3},{x});
        \addplot [color=penColor2,only marks,mark=*] coordinates{(3,-4)};
        


        \node[penColor] at (axis cs:5,-4) {$(h, k)$};
        %\node[penColor] at (axis cs:5,-9) {$-0.5 x^2 - 5 x + 15.5$};



    \end{axis}
\end{tikzpicture}
\end{image}

\textbf{Graph Description:} Cartesian plane. Horizontal axis labeled $x$. Vertical axis labeled $y$. Parabola opening up with vertex labeled ``(h,k)'' at $(3,-4)$. There is a dashed vertical line through the vertex.  \\



The vertex visually encodes the  minimum (or maximum) value of the function. 




If $a<0$, then everything is reveresed.







\begin{image}
\begin{tikzpicture}
     \begin{axis}[
                domain=-10:10, ymax=10, xmax=10, ymin=-10, xmin=-10,
                axis lines =center, xlabel=$x$, ylabel=$y$, 
                ytick={-10,-8,-6,-4,-2,2,4,6,8,10},
                xtick={-10,-8,-6,-4,-2,2,4,6,8,10},
                ticklabel style={font=\scriptsize},
                every axis y label/.style={at=(current axis.above origin),anchor=south},
                every axis x label/.style={at=(current axis.right of origin),anchor=west},
                axis on top,
                ]


        \addplot [draw=penColor, very thick, smooth, domain=(1:5),<->] {-2*(x-3)^2 + 4};
        \addplot [line width=1, gray, dashed,samples=100,domain=(-9.5:9.5)] ({3},{x});
        \addplot [color=penColor2,only marks,mark=*] coordinates{(3,4)};
        


        \node[penColor] at (axis cs:5,4) {$(h, k)$};
        %\node[penColor] at (axis cs:5,-9) {$-0.5 x^2 - 5 x + 15.5$};



    \end{axis}
\end{tikzpicture}
\end{image}

\textbf{Graph Description:} Cartesian plane. Horizontal axis labeled $x$. Vertical axis labeled $y$. Parabola opening down with vertex labeled ``(h'k)' at $(3,4)$. There is a dashed vertical line through the vertex.  \\









We can see from the formula $f(x) = a (x - h)^2 + k$, that since $(x - h)$ is squared, and thus nonnegative, the range of $f$ depends on the sign of $a$, the leading coefficient. \\


\begin{itemize}
     \item When the leading coefficient is positive, the values of $f(x)$ are greater than or equal to $k$. This corresponds to the graph opening up.  The only way to get the least value possible for $f$ is to select $x = h$. That corresponds to the vertex $(k, h)$. \\


     \item When the leading coefficient is negative, the values of $f(x)$ are less than or equal to $k$. This corresponds to the graph opening down.  The only way to get the greatest value possible for $f$ is to select $x = h$. That corresponds to the vertex $(k, h)$.  \\
\end{itemize}



We can see that the implied range of a quadratic comes in two types.  

\begin{itemize}
\item The range could be all real numbers greater than or equal to some particular number:  $\{ r \in \textbf{R} \, | \, r \geq k \} = [k, \infty)$.
\item The range could be all real numbers less than or equal to some particular number:  $\{ r \in \textbf{R} \, | \, r \leq k \} = (-\infty, k]$.
\end{itemize}






$\blacktriangleright$  \textbf{Standard Form} \\

$\frac{-b}{2 \, a}$ is the critical number for a quadratic given in standard form, $f(x) = a \, x^2 + b \, x + c$. \\

The maximum or minium value will be 

\[
f\left( \frac{-b}{2 \, a} \right)
\]

\begin{itemize}
     \item If the leading coefficient is negative, then the range is $\left(-\infty, f\left( \frac{-b}{2 \, a} \right)\right]$ \\
     \item If the leading coefficient is positive, then the range is $\left[f\left( \frac{-b}{2 \, a} \right), \infty \right)$ \\
\end{itemize}














\begin{onlineOnly}
\begin{center}
\textbf{\textcolor{green!50!black}{ooooo-=-=-=-ooOoo-=-=-=-ooooo}} \\

more examples can be found by following this link\\ \link[More Examples of Function Behavior]{https://ximera.osu.edu/csccmathematics/precalculus/precalculus/beginningOfBehavior/examples/exampleList}

\end{center}
\end{onlineOnly}





\end{document}




