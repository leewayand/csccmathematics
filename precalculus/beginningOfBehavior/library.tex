\documentclass{ximera}


\graphicspath{
  {./}
  {ximeraTutorial/}
  {basicPhilosophy/}
}

\newcommand{\mooculus}{\textsf{\textbf{MOOC}\textnormal{\textsf{ULUS}}}}


\usepackage{tkz-euclide}\usepackage{tikz}
\usepackage{tikz-cd}
\usetikzlibrary{arrows}
\tikzset{>=stealth,commutative diagrams/.cd,
  arrow style=tikz,diagrams={>=stealth}} %% cool arrow head
\tikzset{shorten <>/.style={ shorten >=#1, shorten <=#1 } } %% allows shorter vectors

\usetikzlibrary{backgrounds} %% for boxes around graphs
\usetikzlibrary{shapes,positioning}  %% Clouds and stars
\usetikzlibrary{matrix} %% for matrix
\usepgfplotslibrary{polar} %% for polar plots
\usepgfplotslibrary{fillbetween} %% to shade area between curves in TikZ
\usetkzobj{all}
\usepackage[makeroom]{cancel} %% for strike outs
%\usepackage{mathtools} %% for pretty underbrace % Breaks Ximera
%\usepackage{multicol}
\usepackage{pgffor} %% required for integral for loops



%% http://tex.stackexchange.com/questions/66490/drawing-a-tikz-arc-specifying-the-center
%% Draws beach ball
\tikzset{pics/carc/.style args={#1:#2:#3}{code={\draw[pic actions] (#1:#3) arc(#1:#2:#3);}}}



\usepackage{array}
\setlength{\extrarowheight}{+.1cm}
\newdimen\digitwidth
\settowidth\digitwidth{9}
\def\divrule#1#2{
\noalign{\moveright#1\digitwidth
\vbox{\hrule width#2\digitwidth}}}
























%%This is to help with formatting on future title pages.
\newenvironment{sectionOutcomes}{}{}


\title{Library}

\begin{document}

\begin{abstract}
categories
\end{abstract}
\maketitle



We are building a library of Elementary Functions.

We will use our library of functions quite often as our reason for conclusions within our analysis.

We have our first two library entries.











\begin{formula} \textbf{\textcolor{green!50!black}{Linear Functions}} \\
\textbf{Linear functions} are functions which \textbf{\textcolor{red!70!black}{CAN}} be described with a formula of the form

\[  L(t) = a \, t + b   \]

where $a$ and $b$ are real numbers.



\begin{itemize}
\item $a$ is called the \textbf{leading coefficient} 
\item $a \, t$ is called the \textbf{linear term} 
\item $b$ is called the \textbf{constant term} 
\end{itemize}



\end{formula}











\begin{formula} \textbf{\textcolor{green!50!black}{Quadratic Functions}} \\
\textbf{Quadratic functions} are functions which \textbf{\textcolor{red!70!black}{CAN}} be described with a formula of the form

\[  Q(t) = a \, t^2 + b \, t + c  \]

where $a$, $b$, and $c$ (called \textbf{coefficients}) are real numbers with $a \ne 0$



\begin{itemize}
\item $a$ is called the \textbf{leading coefficient} 
\item $a \, t^2$ is called the \textbf{leading or quadratic term} 
\item $b \, t$ is called the \textbf{linear term} 
\item $c$ is called the \textbf{constant term} 
\end{itemize}


This is officially known as the \textbf{standard form}.

\end{formula}


These are our official library forms.



These are \textbf{\textcolor{red!70!black}{CAN}} questions.


If a formula \textbf{CAN} be written like $a \, t + b$, then the function is linear. 


If a formula \textbf{CAN} be written like $a \, t^2 + b \, t + c$, then the function is quadratic. 








\begin{example} Equivalent


$3(x-2) + 9$ is equivalent to $3x + 3$, so it is linear.




$\pi(7 - 5x) + 8x$ is equivalent to $(-5\pi + 8) x + 7\pi$, so it is linear.



$\frac{3+7x}{5}+2x$ is equivalent to $\frac{17}{5} x + \frac{3}{5}$, so it is linear.

\end{example}












\begin{example} Equivalent


$3(x-2)^2 + 9$ is equivalent to $3 x^2 - 12 x + 21$, so it is quadratic.




$2(x-1)(x+5)$ is equivalent to $2 x^2 + 8 x - 10$, so it is quadratic.




\end{example}



























\begin{onlineOnly}
\begin{center}
\textbf{\textcolor{green!50!black}{ooooo-=-=-=-ooOoo-=-=-=-ooooo}} \\

more examples can be found by following this link\\ \link[More Examples of Function Behavior]{https://ximera.osu.edu/csccmathematics/precalculus/precalculus/beginningOfBehavior/examples/exampleList}

\end{center}
\end{onlineOnly}












\end{document}
