\documentclass{ximera}


\graphicspath{
  {./}
  {ximeraTutorial/}
  {basicPhilosophy/}
}

\newcommand{\mooculus}{\textsf{\textbf{MOOC}\textnormal{\textsf{ULUS}}}}


\usepackage{tkz-euclide}\usepackage{tikz}
\usepackage{tikz-cd}
\usetikzlibrary{arrows}
\tikzset{>=stealth,commutative diagrams/.cd,
  arrow style=tikz,diagrams={>=stealth}} %% cool arrow head
\tikzset{shorten <>/.style={ shorten >=#1, shorten <=#1 } } %% allows shorter vectors

\usetikzlibrary{backgrounds} %% for boxes around graphs
\usetikzlibrary{shapes,positioning}  %% Clouds and stars
\usetikzlibrary{matrix} %% for matrix
\usepgfplotslibrary{polar} %% for polar plots
\usepgfplotslibrary{fillbetween} %% to shade area between curves in TikZ
\usetkzobj{all}
\usepackage[makeroom]{cancel} %% for strike outs
%\usepackage{mathtools} %% for pretty underbrace % Breaks Ximera
%\usepackage{multicol}
\usepackage{pgffor} %% required for integral for loops



%% http://tex.stackexchange.com/questions/66490/drawing-a-tikz-arc-specifying-the-center
%% Draws beach ball
\tikzset{pics/carc/.style args={#1:#2:#3}{code={\draw[pic actions] (#1:#3) arc(#1:#2:#3);}}}



\usepackage{array}
\setlength{\extrarowheight}{+.1cm}
\newdimen\digitwidth
\settowidth\digitwidth{9}
\def\divrule#1#2{
\noalign{\moveright#1\digitwidth
\vbox{\hrule width#2\digitwidth}}}
























%%This is to help with formatting on future title pages.
\newenvironment{sectionOutcomes}{}{}


\title{Quadratic Forms}

\begin{document}

\begin{abstract}
standard vertex factored
\end{abstract}
\maketitle



We have seen that the trajectories of projectiles are described by quadratic equations.  In fact, quadratic equations and formulas are used everywhere.


\begin{itemize}
\item \link[101 Uses of a Quadratic Equation Part 1]{https://plus.maths.org/content/101-uses-quadratic-equation}
\item \link[101 Uses of a Quadratic Equation Part 2]{https://plus.maths.org/content/101-uses-quadratic-equation-part-ii}
\end{itemize}







\subsection*{Quadratic Functions}


\begin{definition} \textbf{\textcolor{green!50!black}{Quadratic Functions}} \\
\textbf{Quadratic functions} are functions which can be described with a formula of the form

\[  Q(t) = a \, t^2 + b \, t + c  \]

where $a$, $b$, and $c$ (called \textbf{coefficients}) are real numbers with $a \ne 0$



\begin{itemize}
\item $a$ is called the \textbf{leading coefficient} 
\item $a \, t^2$ is called the \textbf{leading or quadratic term} 
\item $b \, t$ is called the \textbf{linear term} 
\item $c$ is called the \textbf{constant term} 
\end{itemize}


This is officialy known as the \textbf{standard form}.

\end{definition}


Quadratic functions describe relationships between many different measurements. \\

For us, heading to Calculus, quadratics are the simplest complex functions.  We want to understand rates of change and linear functions just don't change their behavior. Quadratics change their behavior. \\


Being the simplest complex functions makes quadratics our favorite for constructing compositions and illustrating the chain rule (Calculus). \\

We will see shortly, that involving quadratics in our investigation of rates of change will make their zeros of the utmost importance. \\


So, we would like to know everything about the zeros of quadratic functions. \\



To this end, we have three viewpoints into the inner workings of quadratic functions.



\textbf{\textcolor{red!70!darkgray}{$\blacktriangleright$ Standard Form}} Every quadratic function can be written as a sum, $Q(t) = a \, t^2 + b \, t + c$, with $a \ne 0$. \\



\textbf{\textcolor{red!70!darkgray}{$\blacktriangleright$ Vertex Form}} Every quadratic function can be written with one occurrence of the variable via a square, $Q(t) = a \, (t - h)^2 + k$, with $a \ne 0$. \\



\textbf{\textcolor{red!70!darkgray}{$\blacktriangleright$ Factored Form}} Every quadratic function can be written as a product, $Q(t) = a \, (t - r_1)(t - r_2)$, where $r_1$ and $r_2$ are the real zeros and $a \ne 0$ - provided the quadratic function has two real zeros. \\











When analyzing a quadratic function (or any function), we want to know two basic pieces of information: where is it and how is it changing? 


\begin{itemize}
\item ``Where is it?'' includes the function's zeros, maximum value, and minimum value.
\item ``How is it changing?'' includes increasing and decreasing.
\end{itemize}



For a quadratic, all of this information is directly connected to its zeros.  So, we have many perspectives and viewpoints of a quadratic function's zeros (also called roots).


















\subsection*{Standard Form}






The standard form of a quadratic function looks like \textbf{$Q(t) = a \, t^2 + b \, t + c$, with $a \ne 0$}.  

From this form the zeros are given by the \textbf{quadratic formula}.





\begin{definition} \textbf{\textcolor{green!50!black}{The Quadratic Formula}} \\



If $a \, t^2 + b \, t + c = 0$ with $a \ne 0$, then


\[ t    = \frac{-b + \sqrt{b^2 - 4 a c}}{2a}  = - \frac{b}{2 a} + \frac{\sqrt{b^2 - 4 a c}}{2a}     \]

or

\[ t   =    \frac{-b - \sqrt{b^2 - 4 a c}}{2a}   = - \frac{b}{2 a}  -\frac{\sqrt{b^2 - 4 a c}}{2a}   \]



Shorthand: 
\[ t  =   \frac{-b \pm \sqrt{b^2 - 4 a c}}{2a}      \]


\end{definition}







\begin{example}


Solve $3 \, x^2 + 4 \, x - 5 = 0$ \\


\[
x = \frac{-4 \pm \sqrt{4^2 - 4 (3)(-5)}}{2(3)} = \frac{-4 \pm \sqrt{76}}{6}
\]

\[
= \frac{-4 \pm \sqrt{4 \cdot 19}}{6} \frac{-4 \pm \sqrt{4} \cdot \sqrt{19}}{6} = \frac{-4 \pm 2\sqrt{19}}{6}
\]

\[
= \frac{2}{2} \cdot \frac{-2 \pm \sqrt{19}}{3} = \frac{-2 \pm \sqrt{19}}{3}
\]



\textbf{checking...} \\


\[
3 \, \left( \frac{-2 - \sqrt{19}}{3} \right)^2 + 4 \, \left( \frac{-2 - \sqrt{19}}{3} \right) - 5 
\]

\[
=  3 \, \left( \frac{4 + 4\sqrt{19}+19}{9} \right) + 4 \, \left( \frac{-2 - \sqrt{19}}{3} \right) - 5
\]


\[
= \frac{12 + 12\sqrt{19} + 57 - 24 - 12\sqrt{19} - 45}{9} = \frac{0}{9} = 0
\]

\end{example}

The quadratic formula tells us that a quadratic function has two roots.  There are several possibilities.




\textbf{\textcolor{blue!55!black}{$\blacktriangleright$ Possible Root Types}} 


The quadratic formula tells us that there are three possibilities for types of zeros or roots.


\begin{itemize}
\item two real roots
\item one real root
\item no real roots (two complex roots)
\end{itemize}















\begin{example} \textit{Two Real Solutions} 

Solve $4 t^2 - 4 t - 8 = 0$ \\


\textbf{explanation}


To use the Quadratic Formula the quadratic equation has to have everything on one side and $0$ on the other.  Our equation is already in that form.

Apply the Quadratic Formula




To use the Quadratic Formula, we match our quadratic to the general quadratic template:

\[
4 t^2 - 4 t - 8 = 0 = a \, t^2 + b \, t + c
\]


\begin{itemize}
\item Matching the leading terms tells us that $a = 4$.
\item Matching the linear terms tells us that $b = -4$.
\item Matching the constant terms tells us that $c = -8$.
\end{itemize}

These are the substitutions we need to perform in the Quadratic Formula.


















\[   t = \frac{-(-4) \pm \sqrt{(-4)^2 - 4 (4) (-8)}}{2 (4)}            \]

\[   t = \frac{4 \pm \sqrt{16 + 128}}{8}            \]

\[   t = \frac{4 \pm \sqrt{\answer{144}}}{8}            \]

\[   t = \frac{4 \pm \answer{12}}{8}            \]




Either $t = \frac{4 + 12}{8} $ or $t = \frac{4 - 12}{8} $

Either $t = \frac{16}{8} $ or $t = \frac{-8}{8} $

Either $t = \answer{2}$ or $t = \answer{-1}$



\end{example}









\begin{example} \textit{One Real Solution}

Solve $2 x^2 - 12x + 21 = 3$ \\


\textbf{explanation}


First, get everything to one side and $0$ on the other side.



\[  2 x^2 - 12x + 18 = 0  \]

Apply the Quadratic Formula


\[   t = \frac{-(-12) \pm \sqrt{(-12)^2 - 4 (2) (18)}}{2 (2)}            \]



\[   t = \frac{12 \pm \sqrt{144 - 144}}{4}            \]

\[   t = \frac{12 \pm \answer{0}}{4}            \]

If you add or subtract $0$, you get the same result.

\[   t = \frac{12}{4}   = 3         \]






\end{example}








\begin{example} \textit{No Real Solutions}

Solve $2 m^2 - 12m + 21 = 1$ \\


\textbf{explanation}


First, get everything to one side and $0$ on the other side.



\[  2 m^2 - 12m + 20 = 0  \]

Apply the Quadratic Formula


\[   m = \frac{-(-12) \pm \sqrt{(-12)^2 - 4 (2) (20)}}{2 (2)}            \]

\[   m = \frac{12 \pm \sqrt{144 - 160}}{4}            \]

\[   m = \frac{12 \pm \sqrt{\answer{-16}}}{4}            \]



The real numbers don't hold a number equal to $\sqrt{-16}$.  Therefore, there are no real solutions.





\end{example}








In each of the examples above the number of solutions was determined by $\sqrt{b^2 - 4 a c}$.  The inside of the square root, $b^2 - 4 a c$, is called the \textbf{discriminant} and its sign tells us how many real solutions the equation has.


\begin{itemize}
\item If $b^2 - 4 a c$ \, \wordChoice{\choice[correct]{$>$}\choice{$=$}\choice{$<$}} \, $0$, then there are two distinct real solutions.
\item If $b^2 - 4 a c$ \, \wordChoice{\choice{$>$}\choice[correct]{$=$}\choice{$<$}} \, $0$, then there is one real solutions.
\item If $b^2 - 4 a c$ \, \wordChoice{\choice{$>$}\choice{$=$}\choice[correct]{$<$}} \, $0$, then there are no distinct real solutions.

\end{itemize}








\textbf{\textcolor{purple!85!blue}{$\blacktriangleright$ Where does the Quadratic Formula come from?}} 

To answer this, we need a different form for a quadratic function, the \textbf{vertex form}.









































































\subsection*{Vertex Form}







The vertext form of a quadratic function, $f(x) = A \, (t-B)^2 + C$ presents some function information quickly.


For instance, the maximum or minimum value is the constant term, $C$.  This extreme valus of the function occurs in the domain at $B$. \\


So, it can be advantageous, to convert from the standard form to the vertex form.  This process is called \textbf{completing the square}. \\




\textbf{\textcolor{blue!55!black}{Completing the Square (Algebraic Viewpoint)}} \\

Completing the square is an algebraic procedure.  Later in this section, we will take a functional viewpoint of this (which will be more direct than the procedure below). \\



Zeros of linear functions were easy to identify.  Simply apply a little algebra to the equation to get the variable by itself on one side of the equation.  But, this is not always possible with equations of the form $0 = a \, t^2 + b \, t + c $, because there are two occurrences of the variable and they have different degrees. \\


It will take a bit of Algebra to combine them into just one occurrence. This procedure is called \textbf{completing the square}. \\



\begin{idea}


The idea is to rearrange  $0 = a \, t^2 + b \, t + c$ to look like $0 = A \, (t-B)^2 + C$.  Then there will only be  one occurrence of the $t$ and we can solve for it.


To accomplish this, we need to investigate $A \, (t-B)^2 + C$. For the moment, let's multiply it back out.


\begin{align*}
A \, (t-B)^2 + C & = A \, (t-B)(t-B) + C \\
& = A \, (t^2 - \answer{2 t B} + B^2) + C  \\
& = A \, t^2 - \answer{2 t A B} + A \, B^2 + C
\end{align*}

This last line, $A \, t^2 - 2 t A B + A \, B^2 + C$ should turn out to be $a \, t^2 + b \, t + c$, our original quadratic.

\end{idea}






We want


\[ A \, t^2 - 2 t A B + A \, B^2 + C = a \, t^2 + b \, t + c \]



The only way that is going to happen is if $A = a$. And, as long as we know $A = a$, we can factor it out and work with what is left.

Then, let's pretend our first step is factoring out $A$ or $a$ and let's start over. \\






\textbf{\textcolor{blue!75!black}{Step 1}} \\

 Factor out the leading coefficent.   In this way we can pretend that we had a \textbf{monic} quadratic from the beginning.  That means we can pretend the leading coefficient was $1$.


Starting over...starting with a leading coefficient of $1$...


\textbf{\textcolor{blue!75!black}{Step 2}} \\


We want to complete the square on $t^2 + b \, t + c$.  When the leading coefficent is $1$, then we call this quadratic \textbf{monic}.



\begin{align*}
(t-B)^2 + C & = (t - B)(t - B) + C \\
& = t^2 - 2 \, t \, B + B^2 + C  
\end{align*}


This last line $t^2 - 2 \, t \, B + B^2 + C$ should turn out to be $t^2 + b \, t + c$.

We want


\[   t^2 - 2 \, t \, B + B^2 + C = t^2 + b \, t + c   \]



\begin{explanation}


The leading term is $1$ in both (since we factored out $A$). Now, the linear terms need to be the same: $ - 2 \, t \, B = \answer{b t}$.  That tells us that $B = \answer{\frac{b}{-2}}$.

\end{explanation}


To complete the square with a monic quadratic, we need $B = \frac{b}{-2} = -\frac{b}{2}$.  Let's put that in.


\[ t^2 - 2 \, t \, B + B^2 + C = t^2 - 2 t \left( -\frac{b}{2} \right) + \left( -\frac{b}{2} \right)^2 + C = t^2 + b \, t + \left(-\frac{b}{2}\right)^2 + C \]


We want this to be $t^2 + b \, t + c$. \\

leading terms match...check \\
linear terms match...check \\


\textbf{\textcolor{blue!75!black}{Step 3}} \\


Now there is a mess for the constant term.  We have $\left( -\frac{b}{2}\right )^2 + C$, when we just wanted $c$.  \\

Then let's just pick $C$ to be $-\left( -\frac{b}{2}\right )^2 + c = -\left(\frac{b}{2}\right)^2 + c$



Whew!

Let's see some examples








\begin{example} \textit{Completing the Square}



Completing the square for $2 \, t^2 + 4 \, t + 6$.


\textbf{explanation}

$\blacktriangleright$ Factor out the leading coefficent, $2$.\\

\[     2 \, t^2 + 4 \, t + 6 = 2 \left( \answer{t^2 + 2 \, t + 3} \right)   \] 


Now we have a monic to work with inside the parentheses. \\


$\blacktriangleright$ Let's move inside the parentheses.

\[ t^2 + 2 \, t + 3 \]

Take half of the linear coefficient, $\frac{2}{2} = 1$, square that $1^2 = 1$, and add and subtract it, so that we have just added $0$ to the expression and not changed its value.


\[ t^2 + 2 \, t + 1 - 1 +3 \]


Now, group.

\[ (t^2 + 2 \, t + 1) - 1 + 3 \]

the grouped part is a square.

\[ \left( \answer{t+1} \right)^2 - 1 + 3 \]

\[ (t+1)^2 + 2 \]

Remember, this was inside parentheses.

\[     2 \, t^2 + 4 \, t + 6 = 2 (t^2 + 2 \, t + 3)  = 2 ((t+1)^2 + 2) =  2 (t+1)^2 + \answer{4}\] 


$2 (t+1)^2 + 4$ is the completed square form of $2 \, t^2 + 4 \, t + 6$ \\


$2 (t+1)^2 + 4 = 2 \, t^2 + 4 \, t + 6$





\end{example}














The whole point was to change from an expression with two occurences of the variable to an expression with only one occurence of the variable. \\

This makes solving for zeros much easier. \\

Instead of solving $2 \, t^2 + 4 \, t + 6 = 0$, we can solve $2 (t+1)^2 + 4 = 0$.


\[
2 (t+1)^2 + 4 = 0
\]


\[
(t+1)^2 = -2
\]


From this equation, we can see that this function has no zeros. We wouldn't have been able to guess the factors. \\


As we saw earlier, quadratic functions can have two real zeros, one real zero, or no real zeros. \\








\begin{example} \textit{Two Real Solutions}

Solve $4  t^2 - 4 \, t - 8 = 0$ \\


\textbf{explanation}


First complete the square.



\[ 4 \, t^2 -  4 \, t - 8 = 0 \]

\[ \answer{4} (t^2 - t - 2) = 0 \]



Half of the linear coefficient is $\answer{-\frac{1}{2}}$. \\

The square of that is $\answer{\frac{1}{4}}$. \\

We will be adding and subtracting $\answer{\frac{1}{4}}$ inside the parentheses.




\[ 4 \, (t^2 - t + \frac{1}{4} - \frac{1}{4} - 2)  \]


\[ 4 \, \left(\left(t - \frac{1}{2}\right)^2 - \frac{1}{4} - 2\right)  \]


\[ 4 \, \left(\left(t - \frac{1}{2}\right)^2 - \frac{1}{4} - 2\right)  \]

\[ 4 \, \left( \answer{t - \frac{1}{2}} \right)^2 - 1 - 8  \]

\[ 4 \, \left(t - \frac{1}{2}\right)^2 - 9  \]


One occurrence of $t$, good. Now to get $t$ by itself.

\[ 4 \, \left(t - \frac{1}{2}\right)^2 - 9 = 0  \]

\[ 4 \, \left(t - \frac{1}{2}\right)^2 = 9  \]

\[  \left(t - \frac{1}{2}\right)^2 = \answer{\frac{9}{4}}  \]

We know something special here. The only way this can happen is if \\



either   $\answer{t - \frac{1}{2}} = \frac{3}{2}$  or  $\answer{t - \frac{1}{2}} = -\frac{3}{2}$ \\

The first choice gives us $t = 2$ and the second choice gives us $t = -1$




Let's check those solutions.

\begin{itemize}
\item $4 (2)^2 - 4 (2) - 8 = 0$ ... check
\item $4 (-1)^2 - 4 (-1) - 8 = 0$ ... check
\end{itemize}



We have already seen that a quadratic equation can have at most two solutions.  So, we must have all of the solutions.




\end{example}









\begin{example} \textit{One Real Solution}

Solve $2 \, x^2 - 12 \, x + 21 = 3$ \\


\textbf{explanation}


First, get everything to one side and $0$ on the other side.



\[  2 \, x^2 - 12 \, x + \answer{18} = 0  \]

Factor out $a$, which is $2$ here.

\[  2 \left( \answer{x^2 - 6x + 9} \right) = 0  \]


Square half of $b$, $\left(\frac{-6}{2}\right) = (-3)^2 = 9$.  But, $9$ is already there.  This must already be a square.



\[  2 \left( \answer{x - 3} \right)^2 = 0  \]


Now, get $x$ by itself.

\[  2 (x - 3)^2 = 0  \]

\[  (x - 3)^2 = 0  \]


Unlike $\frac{9}{4}$ in the previous example, the only number you can square and get $0$ is $0$.

\[  x - 3 = 0  \]

\[  x = 3  \]


This equation only has one solution.



\end{example}



The graph of $y = 2 \, x^2 - 12\, x + 18$ would be a parabola touching the horizontal axis at only one point (its vertex).




\begin{idea} \textbf{\textcolor{red!80!black}{Double Root}} \\

If the vertex of the parabola is the only intercept, then the corresponding quadratic function has a double root.


\end{idea}



There is another viewpoint on the example above.  We arrived at this equation

\[  (x - 3)^2 = 0  \]

which could be viewed as 

\[  (x - 3) (x - 3) = 0  \]


If we proceed with the zero product property we would create two new equations.  One for each factor.

Either $x - 3 = 0$   or $x - 3 = 0$. \\

Either $x = 3$ or $x = 3$.  \\

$3$ is the only solution to the equation, but the equation has this solution twice. \textbf{\textcolor{red!80!black}{A double root}}.


The previous two examples illustrated that quadratic equations can have two or one solutions.  And, a quadratic equation can have no real solutions.










\begin{example} \textit{No Real Solutions}

Solve $2 \, x^2 - 12 \, x + 21 = 1$ \\


\textbf{explanation}


First, get everything to one side and $0$ on the other side.



\[  2 \, x^2 - 12 \, x + 20 = 0  \]

Factor out the leading coefficient.

\[  2 \left( \answer{x^2 - 6x + 10} \right) = 0  \]


Square half of $b$, $\left(\frac{-6}{2}\right)^2 = (-3)^2$ and $(-3)^2 = 9$.  We could add and subtract $9$ or we can change the $10$ to be $9+1$.  All we are trying to do is to see a $9$.



\[  2 (x^2 - 6 \, x + 9 + 1) = 0  \]


\[  2 ((x-3)^2 + 1) = 0  \]


\[  2 (x-3)^2 + 2 = 0  \]

In the real numbers, $2 (x-3)^2$ is never negative, since we have a square.  This cannot be added to $2$, to get $0$.  \\


This equation has no real solutions.



\end{example}



The above example illustrates that there must be numbers missing from the real numbers.  We normally expect equations to have solutions. 


$\blacktriangleright$  \textbf{\textcolor{purple!85!blue}{A Peek Ahead}} 



It feels like that equation should have a solution.  


All we needed was for $(x-3)^2 = -1$ 

There are no real numebrs that will do this.  The square of a real number cannot be negative. 




We can see that the real numbers are not enough for all of our equations.  Eventually, we will fill in the missing pieces with the \textbf{Complex Numbers}.  They will include $\sqrt{-1}$.  Then our last example will have two complex solutions.  


\begin{itemize}
\item Our first example had two real solutions.  
\item Our second example had two identical solutions.  
\item This third example has two distinct complex solutions. 
\end{itemize}



$\blacktriangleright$ All quadratics will have two solutions...eventually. \\

We'll fill in the holes in the second course.









For now, we are staying inside the real numbers.



For now, we note that there are no \textbf{real} solutions in the third example.
































\textbf{\textcolor{blue!55!black}{Completing the Square (Conceptual Viewpoint)}} \\






The previous viewpoint was an algebraic viewpoint.  \\

It presented a step-by-step procedure for applying algebra that would transform a quadratic polynomial in standard form to one in verxtex form.  The procedure gather terms that could be rewritten as a square.  \\

The procedure began with the standard form, $A \, x^2 + B \ x + C$, and the result of the procedure was an equivalent form, $A (x - h)^2 + k$, called the vertex form. \\










The whole point was to change from an expression with two occurences of the variable to an expression with only one occurence of the variable.

This is helpful in graphing and helpful in solving quadratic equations.


However, the procedure isn't the point of all of this.  It is just a procedure to obtain the vertex form.  If we take a functional viewpoint instead of an algebraic viewpoint, we can get there much faster.










Our goal is to convert $a \, x^2 + b \, x + c$ into $a \, (x - h)^2 + k$. \\

Let's think of these as function.


\begin{itemize}
\item $S(x) = a \, x^2 + b \, x + c$.
\item $V(x) = a \, (x - h)^2 + k$.
\end{itemize}



The $a$ is the leading coefficient in both forms.  Therefore, we only need $h$ and $k$ in terms of $a$, $b$, and $c$. \\




\textbf{\textcolor{blue!55!black}{$\blacktriangleright$}}  The extrema \\



\begin{itemize}
\item From the standard form, we know that the vertex's first coordinate is $\frac{-b}{2 a}$.
\item From the vertex form, we know that the vertex's first coordinate is $h$.
\end{itemize}


These must be equal.

\[
h = \frac{-b}{2 a}
\]



If we evaluate each function at $\frac{-b}{2 a}$, we must get the same value.  \\

For $\frac{-b}{2 a}  = h$, we have 

\[
V\left( \frac{-b}{2 a} \right) = V(h) = a (h - h)^2 + k = k
\]

Therefore, if we evaluate $S(x) = a \, x^2 + b \, x + c$ at $\frac{-b}{2 a}$, we must also get $k$.



\[
k = S\left( \frac{-b}{2 a} \right) = a \left( \frac{-b}{2 a} \right)^2 + b \left( \frac{-b}{2 a} \right) + c 
\]







This makes the example above much quicker.





\begin{example} \textit{Vertex Form}



Convert $2 \, t^2 + 4 \, t + 6$ into vertex form, i.e. complete the square.


\textbf{explanation}


$S(t) = 2 \, t^2 + 4 \, t + 6$


\[
\frac{-b}{2 a} = \frac{-4}{2 \cdot 2} = -1
\]


$S(-1) = 2 \, (-1)^2 + 4 \, (-1) + 6 = 4$



$a (x - h)^2 + k = 2 \, (t+1)^2 + 4$

which is what we got through the procedure of completing the square.





\end{example}




These examples give us a procedure for developing the quadratic formula. \\




















\textbf{\textcolor{purple!85!blue}{$\blacktriangleright$ Where does the Quadratic Formula come from?}} 


The discussion for the vertex form of a quadrtaic function tells us that the extreme value of the function happens at $-\frac{b}{2a}$.

This means that to get the extreme value of the function, we need to evaluate at $-\frac{b}{2a}$. We saw that this value is $c - \frac{b^2}{4a}$


All together the general form of the vertex for looks like














\[ a\left(t + \frac{b}{2 a}\right)^2 + c - \frac{b^2}{4 a}  = 0\]

TO get the quadratic formula, we solve this equation for $t$.



\begin{explanation}
\[ a \left(t + \frac{b}{2 a} \right)^2  = \frac{b^2}{4 a} - c\]

\[ \left(t + \frac{b}{2 a} \right)^2  = \frac{b^2}{4 a^2} - \frac{c}{a}\]

\[ \left(t + \frac{b}{2 a} \right)^2  = \frac{\answer{b^2 - 4 a c}}{4 a^2} \]
\end{explanation}


Either 


\[ t + \frac{b}{2 a}  = \sqrt{\frac{b^2 - 4 a c}{4 a^2}}  = \frac{\sqrt{b^2 - 4 a c}}{| 2a |}   \]

or


\[ t + \frac{b}{2 a}  = -\sqrt{\frac{b^2 - 4 a c}{4 a^2}} = -\frac{\sqrt{b^2 - 4 a c}}{| 2a |}    \]



\begin{itemize}
\item If $a > 0$, then $| 2a | = 2a$.
\item If $a < 0$, then $| 2a | = -2a$.
\end{itemize}

Either way, we still get one negative and one positive fraction.  Therefore, we can drop the absolute value signs.  





Either 


\[ t + \frac{b}{2 a}  = \sqrt{\frac{b^2 - 4 a c}{4 a^2}}  = \frac{\sqrt{b^2 - 4 a c}}{2a}   \]

or


\[ t + \frac{b}{2 a}  = -\sqrt{\frac{b^2 - 4 a c}{4 a^2}} = -\frac{\sqrt{b^2 - 4 a c}}{2a}    \]










And, finally \\


\begin{conclusion}



Either 


\[ t   = - \frac{b}{2 a} + \frac{\sqrt{b^2 - 4 a c}}{2a}  = \frac{-b + \sqrt{b^2 - 4 a c}}{2a}      \]

or


\[ t  = - \frac{b}{2 a}  -\frac{\sqrt{b^2 - 4 a c}}{2a} =    \frac{-b - \sqrt{b^2 - 4 a c}}{2a}      \]

\end{conclusion}






People generally shorthand these two separate solutions as



\[ t  =   \frac{-b \pm \sqrt{b^2 - 4 a c}}{2a}      \]


This is known as \textbf{The Quadratic Formula}.


It gives the zeros or roots of a quadratic function where the standard form is 

\[
a \, t^2 + b \, t + c
\]



Now that we know what the zeros are of a quadratic function, we can factor it.




























\subsection*{Factored Form}










We have seen that there are at most two real solutions to $a \, t^2 + b \, t + c = 0$, with $a \ne 0$. 

Let's look at this from a function viewpoint.  The quadratic function  $Q(t) = a \, t^2 + b \, t + c$ has at most two real zeros.  

The quadratic formula gives explicit expressions for these roots.


\[   \frac{-b + \sqrt{b^2 - 4 a c}}{2a}     \text{ and }    \frac{-b - \sqrt{b^2 - 4 a c}}{2a}   \]





$\blacktriangleright$  Are there other quadratic equations, besides $a \, t^2 + b \, t + c = 0$, that have these two solutions?















Perhaps, there are other quadratic functions, which have these two zeros. Let's create a quadratic function from these two zeros.



\[ f(t) =  \left(t - \frac{-b + \sqrt{b^2 - 4 a c}}{2a}\right)   \left(t -  \frac{-b - \sqrt{b^2 - 4 a c}}{2a}\right)   \]


If we multiply this out, we get



\[ f(t) =   t^2 \, - \, \frac{-b + \sqrt{b^2 - 4 a c}}{2a} t \, - \, \frac{-b - \sqrt{b^2 - 4 a c}}{2a}  t \, + \, \left(\frac{-b + \sqrt{b^2 - 4 a c}}{2a}\right) \left(\frac{-b - \sqrt{b^2 - 4 a c}}{2a}\right) \]


\[ f(t) = t^2  \, - \, 2 \cdot \frac{-b}{2a} + \left(    \frac{b^2 - (b^2 - 4 a c)}{4 a^2}     \right)        \]


\[ f(t) = t^2  \, + \frac{b}{a} + \left(    \frac{4 a c}{4 a^2}     \right)        \]

\[ f(t) = t^2  \, +  \frac{b}{a} + \left(    \frac{c}{a}     \right)        \]


This quadratic function has the same two zeros as $Q(t)$.  


\textbf{Note:} If we multiply this by $a$, we get 


\[ a \, f(t) = a \, t^2 + b \, t + c \]

which is $Q(t)$.


$\blacktriangleright$ If two quadratics have the same two roots, then they must be multiples of each other.






We have discovered a new expression or formula or form for our quadratic functions.





\[ f(t) =  \left(t - \frac{-b + \sqrt{b^2 - 4 a c}}{2a}\right)   \left(t -  \frac{-b - \sqrt{b^2 - 4 a c}}{2a}\right)   \]



















The real zeros can be obtained via the quadratic formula - provided the discrimanant is positve.  If the discriminant is $0$, then $r_1 = r_2$ and we get a square.  If the discriminant is negative, then $Q(t)$ doesn't factor with real numbers.  


The process of writing the function as a product is called \textbf{factoring}. $(t - r_1)$  and $(t - r_2)$ are the \textbf{factors}.



\begin{example} \textit{Two Real Solutions} 

Factor $Q(t) = 4 \, t^2 - 4 \, t - 8$ 


\textbf{explanation}


The leading coefficient is $a=4$ and the quadratic formula gives us $2$ and $-1$ as zeros.  

That gives us a \textbf{factorization} of $Q(t)$:



\[    Q(t) = 4 \, t^2 - 4 \, t - 8 =  4 (t-2)(t-(-1))    = 4 (t-2)(t+1)         \]



\end{example}



The quadratic formula always works.  But, many times it is slow with lots of steps and reducing.  Sometimes it is just easier to guess the factors.  ``Easier'' usually means integers.




















\subsection*{Guessing Factors}


Let's begin with a quadratic function written as a sum: $g(x) = a \, x^2 + b \, x + c$, a.k.a \textbf{standard form}.  


\textbf{\textcolor{blue!75!black}{Step 1:}} If there are any common numeric factors among the three coefficients then factor them out.


\textbf{\textcolor{blue!75!black}{Step 2:}} We are looking for a factorization $g(x) = a \, x^2 + b \, x + c = (A \, x + B)(C \, x + D)$.

for this to happen, we need

\begin{itemize}
\item $a = A \cdot C $
\item $b = A\cdot D + B \cdot C$
\item $c = B \cdot D$
\end{itemize}

Therefore, look for pairs of factors of $a$ and pairs of factors of $c$.






\textbf{\textcolor{blue!75!black}{Step 3:}} Step through your pairs of factors and look for $b = A \cdot D + B \cdot C$.










\begin{example} \textit{Factoring}

Factor $k(x) = 2 \, x^2 - 5 \, x - 12$.


\textbf{explanation}


First, the coefficients, $2$, $-5$, and $-21$, have no common factors.


Second, $2$ is a prime number, therefore, it's factors are $2 \cdot 1$ or $-2 \cdot -1$.

From this, we know there are only two possibilities. The factorization looks like $(2 x \, + \, ?) (x \, + \, ?)$ or $(-2 x \, + \, ?) (-x \, + \, ?)$.






$-12$ factors as 
\begin{itemize}
\item $-12 \cdot \answer{1}$
\item $12 \cdot \answer{-1}$
\item $-6 \cdot \answer{2}$
\item $6 \cdot \answer{-2}$
\item $-4 \cdot \answer{3}$
\item $4 \cdot \answer{-3}$  
\end{itemize}



Let's start stepping through our list.

The middle term is $-5 \, x$.  That's our goal. It seems like it might be difficult to get $5$ from $12$, let's skip $12$ in our list and begin stepping through our list with $-6$ and $2$.

\begin{itemize}

\item $-12 = -6 \cdot 2$ gives $(2 x + (-6)) (x + 2) = \answer{2 x^2 - 2 x - 12}$
\item $-12 = 6 \cdot -2$ gives $(2 x + 6) (x - 2) = \answer{2 x^2 + 2 x -12}$
\item $-12 = -3 \cdot 4$ gives $(2 x + (-3)) (x + 4) = \answer{2 x^2 + 5 x - 12}$
\item $-12 = 3 \cdot -4$ gives $(2 x + 3) (x + (-4)) = \answer{2 x^2 - 5 x -12}$
\end{itemize}

$2 x^2 - 5 x - 12$ factors as $(2 x + 3) (x + (-4)) = (2 x + 3)(x - 4)$



\end{example}

As you gain more experience, this process becomes a mental process, which means it is fast. \\


Always give factoring a try, before moving onto the quadratic formula. \\






\begin{example} \textit{Factoring}

Factor $p(y) = 6 \, y^2 - 16 \, y - 22$.


\textbf{explanation}


First, the coefficients have a common factor.  Factor out $2$ to get $p(y) = 2 \left( \answer{3 y^2 - 8 y - 11} \right)$.


Now factor $3 \, y^2 - 8 \, y - 11$.



$3$ is a prime number, therefore, it's factors are $3 \cdot 1$ or $-3 \cdot -1$. However, $-3 \cdot -1$ is actually redundant.  If we picked $-3 \cdot -1$, then we could just factor out $-1$ from both and get back to $3 \cdot 1$.


$3 \, y^2 - 8 \, y - 11 = (3y \, \pm \, ?) (y \, \pm \, ?) $


$11$ is also a prime number. We only need to think about $11$ and $1$. And, since we need $-11$, then we know to pick opposite signs for $11$ and $1$.



\begin{itemize}
\item $(3y + 11) (y - 1) = \answer{3 y^2 + 8 y - 11}$
\end{itemize}

Everything is correct except the sign of $-8$. That tells us we are almost correct. We just need to reverse the signs.


\begin{itemize}
\item $(3y - 11) (y + 1) = 3 \, y^2 - 8 \, y - 11$
\end{itemize}





\end{example}


``Easier'' usually means integers.  

All of the factorizations above involved integers, because we are fast with those. \\



If the roots of the quadratic involve square roots, then we probably are not guessing them.  In any case, the quadratic formula will find all of the roots.






















































\begin{onlineOnly}
\begin{center}
\textbf{\textcolor{green!50!black}{ooooo-=-=-=-ooOoo-=-=-=-ooooo}} \\

more examples can be found by following this link\\ \link[More Examples of Function Behavior]{https://ximera.osu.edu/csccmathematics/precalculus/precalculus/beginningOfBehavior/examples/exampleList}

\end{center}
\end{onlineOnly}












\end{document}
