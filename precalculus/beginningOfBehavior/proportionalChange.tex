\documentclass{ximera}


\graphicspath{
  {./}
  {ximeraTutorial/}
  {basicPhilosophy/}
}

\newcommand{\mooculus}{\textsf{\textbf{MOOC}\textnormal{\textsf{ULUS}}}}


\usepackage{tkz-euclide}\usepackage{tikz}
\usepackage{tikz-cd}
\usetikzlibrary{arrows}
\tikzset{>=stealth,commutative diagrams/.cd,
  arrow style=tikz,diagrams={>=stealth}} %% cool arrow head
\tikzset{shorten <>/.style={ shorten >=#1, shorten <=#1 } } %% allows shorter vectors

\usetikzlibrary{backgrounds} %% for boxes around graphs
\usetikzlibrary{shapes,positioning}  %% Clouds and stars
\usetikzlibrary{matrix} %% for matrix
\usepgfplotslibrary{polar} %% for polar plots
\usepgfplotslibrary{fillbetween} %% to shade area between curves in TikZ
\usetkzobj{all}
\usepackage[makeroom]{cancel} %% for strike outs
%\usepackage{mathtools} %% for pretty underbrace % Breaks Ximera
%\usepackage{multicol}
\usepackage{pgffor} %% required for integral for loops



%% http://tex.stackexchange.com/questions/66490/drawing-a-tikz-arc-specifying-the-center
%% Draws beach ball
\tikzset{pics/carc/.style args={#1:#2:#3}{code={\draw[pic actions] (#1:#3) arc(#1:#2:#3);}}}



\usepackage{array}
\setlength{\extrarowheight}{+.1cm}
\newdimen\digitwidth
\settowidth\digitwidth{9}
\def\divrule#1#2{
\noalign{\moveright#1\digitwidth
\vbox{\hrule width#2\digitwidth}}}
























%%This is to help with formatting on future title pages.
\newenvironment{sectionOutcomes}{}{}


\title{Proportional Change}

\begin{document}

\begin{abstract}
linear
\end{abstract}
\maketitle




There are many relationships between measurements that exhibit proportional changes.  







\subsection*{Proportional Changes}




\begin{example} Speed


Suppose a car is traveling on the highway at a constant speed of $60 \, mph$.


$\blacktriangleright$ Whenever the distance measurement \textbf{\textcolor{purple!85!blue}{changes}} by $60 \, miles$, the time measurement \textbf{\textcolor{purple!85!blue}{changes}} by $1 \, hour$. \\
$\blacktriangleright$ Whenever the time measurement \textbf{\textcolor{purple!85!blue}{changes}} by $1 \, hour$, the distance measurement \textbf{\textcolor{purple!85!blue}{changes}} by $60 \, miles$. \\




\begin{itemize}
\item When the time changes by $2$ hours, the distance changes by $\answer{120}$ miles. \\
\item When the time changes by $5$ hours, the distance changes by $\answer{300}$ miles. \\
\item When the time changes by $0.5$ hours, the distance changes by $\answer{30}$ miles. \\
\end{itemize}



We have a constant conversion factor of $\frac{60 \, miles}{1 \, hour}$ for converting time changes into distance changes. \\






\begin{itemize}
\item When the distance changes by $2$ miles, the time changes by $\answer{\frac{1}{30}}$ hours. \\
\item When the distance changes by $5$ miles, the time changes by $\answer{\frac{1}{12}}$ hours. \\
\item When the distance changes by $0.5$ miles, the time changes by $\answer{\frac{1}{120}}$ hours. \\
\end{itemize}



We have a constant conversion factor of $\frac{1 \, hour}{60 \, mile}$ for converting distance changes into time changes. \\




$\blacktriangleright$ One viewpoint is that, in this situation, $60 \, miles = 1 hour$.  When one change occurs, the other must occur as well.





\end{example} 









\begin{notation}  \textbf{\textcolor{red!80!black}{$\Delta$}} \\


Mathematics has a shorthand symbol for ``change''.  It is an uppercase Greek delta: $\Delta$

\end{notation}

\textbf{\textcolor{purple!85!blue}{$\blacktriangleright$}}  In the previous example, it might be more appropriate to say \textbf{\textcolor{purple!85!blue}{$\Delta 60 \, miles = \Delta 1 hour$}} . It is the change in the measurements that is proportional.




\begin{example} Temperature


Temperature can be measured in degrees Fahrenheit or degrees Celsius and these measurements change proportionally. 


$\blacktriangleright$ Whenever the Fahrenheit measurement changes by $9^{\circ}$, the Celsius measurement changes by $5^{\circ}$. \\
$\blacktriangleright$ Whenever the Celsius measurement changes by $5^{\circ}$, the Fahrenheit measurement changes by $9^{\circ}$. \\



Water freezes at $0^{\circ}$ C and $32^{\circ}$ F.  From here, if the Celsius measurement changes by $100^{\circ}$, then the Fahrenheit measurement changes by $180^{\circ}$.  $18$ each of $5$'s and $9$'s. We get water boiling at $100^{\circ}$ celsius and $212^{\circ}$ degrees.


In this situation, $\frac{\Delta 5^{\circ}C}{\Delta 9^{\circ}F}$ or just $\frac{5^{\circ}C}{9^{\circ}F}$ is the conversion factor from Fahrenheit change to Celsius change, and $\frac{9^{\circ}F}{5^{\circ}C}$ is the conversion factor from Celsius to Fahrenheit.






\end{example} 
















\subsection*{Rates}


Those conversions are examples of \textbf{rates}. 








\begin{definition} \textbf{\textcolor{green!50!black}{Rates}} \\

A \textbf{rate} is a ratio between two quantities with different units.   \\



We usually represent rates with factions, although we also use phrases involving ``per'' or equalities.

\end{definition}


For us, rates measure how fast one quantity changes compared to the change in another quantity.











\begin{example} Speed


Suppose a car is traveling on the highway at a constant speed of $60 \, mph$.


$\blacktriangleright$ Whenever the distance measurement changes by $60 \, miles$, the time measurement changes by $1 \, hour$. \\
$\blacktriangleright$ Whenever the time measurement changes by $1 \, hour$, the distance measurement changes by $60 \, miles$. \\




If we compare these related measurement changes in a rate, then we get

\[
\frac{\Delta Distance}{\Delta Time} = \frac{60 \, miles}{1 \, hour} = 60 \,miles \, per \, hour
\]


\end{example} 



This example gives us the equation for linear motion: $Distance = Rate \times Time$, which we could represent here with a function.

\[
D(t) = R \cdot t =  \frac{60 \, miles}{1 \, hour} \cdot t
\]









\begin{example} Temperature


Temperature can be measured in degrees Fahrenheit or degrees Celsius and these measurements have a function relationship. This relationship has a special property.


$\blacktriangleright$ Whenever the Fahrenheit measurement changes by $9^{\circ}$, the Celsius measurement changes by $5^{\circ}$. \\
$\blacktriangleright$ Whenever the Celsius measurement changes by $5^{\circ}$, the Fahrenheit measurement changes by $9^{\circ}$. \\



Water freezes at $0^{\circ}$ C and $32^{\circ}$ F.  From here, if the Celsius measurement changes by $100^{\circ}$, then the Fahrenheit measurement changes by $180^{\circ}$.  $20$ each of $5$'s and $9$'s. We get water boiling at $100^{\circ}$ celsius and $212^{\circ}$ degrees.


If we compare these related measurement changes in a rate, then we get

\[
\frac{\Delta Degrees}{\Delta Celsius} = \frac{9^{\circ}F}{5^{\circ}C} \, \text{ or } \, \frac{\Delta Celsius}{\Delta Degrees} = \frac{5^{\circ}C}{9^{\circ}F}
\]


\end{example} 


This second example gives us the conversion between Fahrenheit change and Celsius change, which we could express with a function.  We just have to remember that $0^{\circ}C = 32^{\circ}F$.

\[
F(C) = 32 + \frac{9}{5} \cdot C
\]


The formula converts each change of $5$ in $C$ into a change of $9$ for $F$.







\begin{definition} \textbf{\textcolor{green!50!black}{Constant Rate of Change}} \\


A rate is a comparison of how measurements \textbf{\textcolor{purple!85!blue}{CHANGE}}. A rate is not a comparison of the measurement values, but a comparison of how they change. \\


Two measurements share a \textbf{constant rate of change} of $\tfrac{A}{B}$ if whenever measurement \#1 changes by $A$, then measurement \#2 changes by $B$ and vice versa, and this rate is not dependent on the amount of each measurement present.  The rate is the same throughout the situation.  It is a constant.




\end{definition}











\begin{definition} \textbf{\textcolor{green!50!black}{Linear Functions}} \\

Linear functions are those functions where the domain and range share a constant rate of change.  

\end{definition}


Each linear function has its own constant rate of change. \\


Suppose $L$ is a linear function.  Let $a$ and $b$ be numbers in the domain of $L$.  Then $L(a)$ and $L(b)$ are the corresponding range values.

Since $L$ is a linear function, we know that $\frac{L(b) - L(a)}{b - a} = constant$.  And, this works for ANY two domain numbers.

Otherwise, it is not a linear function.







\begin{example} \textit{Constant Rate of Change}


Suppose $f$ is a function, which contains the pairs $(3, 7)$, $(5, 17)$, and $(6, 23)$.


The rate-of-change from $3$ to $5$ is $\answer{5}$.

The rate-of-change from $5$ to $6$ is $\answer{6}$.

\begin{question} 
$f$ is a linear function.
\begin{multipleChoice}
\choice {True}
\choice [correct]{False}
\end{multipleChoice}
\end{question}

\end{example}


Somewhere in history, $m$ became a popular choice for the constant rate of change of a linear function.



\[
\frac{L(b) - L(a)}{b - a} = m
\]





No matter which two numbers you select from the domain of $L$, the rate of change always turns out to be $m$.  Each linear function has its own $m$ - its own constant rate of change.






\begin{model} \textit{Mileage vs. Weight}

\textbf{How does the weight of a car affect its gas mileage?} \\



$392$ makes and models of automobiles manufactured between 1970 and 1982 were considered.  

\begin{itemize}
\item How many gallons of gasoline were used to travel $100$ miles?   (gallons)
\item What is the weight of the car?    ($1000$ pounds)
\end{itemize}


These pairings are plotted below.




\begin{center}
\desmos{4ysvvryemv}{400}{300}
\end{center}


We can see from the graph that this is not the graph of a function.  It is a data plot.\\


Our goal is to make predictions about other cars from this data.  Therefore, we would like a \textbf{\textcolor{purple!85!blue}{model}} for this data.  That is, we would like a function that approximates the data relationship.  The easiest model is a linear model.


A linear model would be a linear function whose graph does the best job a line can do of pretending to be the dots.  The line should pass through the ``middle of the dots''.

In the DESMOS window above, change the values of ``$m$'' and ``$b$'' to find such a line. \\

When you have your line, turn on the model found by DESMOS by clicking on the circle next to $y_1 \sim a_1 x + a_2$.


Source:   \textit{regressit.com}
\end{model}



\begin{question}


DESMOS found the linear model $y_1 = 0.450808 x_1 + 0.820606$ \\


The units of $x_1$ are $1000 \, lbs$. \\
The units of $y_1$ are $gallons$.



What are the units of $0.450808$?

\begin{multipleChoice}
\choice {$gallons$}
\choice {$pounds$}
\choice[correct] {$\frac{gallon}{pound}$}
\choice {$\frac{pound}{gallon}$}
\end{multipleChoice}


\end{question}








\begin{question}


DESMOS found the linear model $y_1 = 0.450808 x_1 + 0.820606$ \\


The units of $x_1$ are $1000 \, lbs$. \\
The units of $y_1$ are $gallons$.



Which is the correct statement?

\begin{multipleChoice}
\choice[correct] {For every $1000 \, pounds$ added to a car, an extra $0.450808 \, gallons$ of gas is needed to travel $100 \, miles$.}
\choice {Every added gallon of gas can carry $4508.08 \, pounds$ an additional $100 \, miles$.}
\end{multipleChoice}


\end{question}





























\section*{Linear Functions}

















$\blacktriangleright$  Linear functions are those functions where the domain and range share a constant rate of change.  





Each linear function has its own constant rate of change. \\


Suppose $L$ is a linear function.  Let $a$ and $b$ be numbers in the domain of $L$.  Then $L(a)$ and $L(b)$ are the corresponding range values.

Since $L$ is a linear function, we know that $\frac{L(b) - L(a)}{b - a} = constant$.  And, this works for \textbf{\textcolor{purple!85!blue}{ANY}} two domain numbers.

Otherwise, it is not a linear function.







Somewhere in history, $m$ became a popular choice to represent the constant rate of change of a linear function.



\[
\frac{L(b) - L(a)}{b - a} = m
\]





No matter which two numbers you select from the domain of $L$, the rate of change always turns out to be $m$.  Each linear function has its own $m$ - its own constant rate of change.












\subsection*{A Formula}  


Let $L$ be a linear function.  That means it has is own constant rate of change.  Let's call it $m$. \\

Let $(a, b)$ be one specific pair in $L$.\\


Let $(x, y)$ represent any other pair (all other pairs) in the function $L$. 

\begin{formula} \textbf{\textcolor{blue!55!black}{Slope}} 


Then the rate of change from $(a,b)$ to $(x, y)$ must equal $m$.


\[  \frac{y - \answer{b}}{x-\answer{a}} = m \]
\end{formula}


Clearing the denominator gives us the point-slope form for a line.






\begin{formula} \textbf{\textcolor{blue!55!black}{Point-Slope Form}} 


In practice, in addition to the slope, $m$, we almost always have a point on the line that is not an intercept $(a,b)$. Describing the other points on the line is much easier with the point-slope form of a line.


\[  (y - b) = m (x - a) \]
\end{formula}






Since $(a, b)$ is a pair in $L$, we know that $b = L(a)$.  And, since $(x, y)$ represents any other pair in $L$, we know that $y = L(x)$.  Replacing these in the equation for constant slope gives


\[  \frac{L(x) - L(a)}{x-a} = m \]


Clearing the denominator gives


\[  L(x) - L(a) = m (x-a) + L(a)     \]

Solving this for $L(x)$ gives

\[  L(x) = m (x-a) + L(a)     \]



\textbf{Note:} In advanced mathematics, a similar idea called \textit{linear maps} are required to include the pair $(0,0)$.  This would result in $L(0) = 0$. \\  
This is not required for Calculus. \\




\begin{example} \textit{A pair and a rate-of-change}


Suppose $W$ is a linear function with constant rate of change equal to $5$ and $(3, -1)$ is one pair in $W$.  \\

Create a formula for $W$.

\begin{explanation}

The template $L(x) = m (x-a) + L(a)$ tell us that a formula for $W$ looks like 


\[  W(x) = 5 (x-3) + \left(\answer{-1}\right)     \]


\[  W(x) = 5 (x-3) - 1     \]

We could multiply this out and collect like terms and obtain the equivalent formula


\[  W(x) = \answer{5}x - \answer{16}   \]

\end{explanation}

Perhaps, we do not like $x$ as the variable for our formula.  Perhaps $v$ suits our situation better.

\[  W(v) = 5v - 16   \]


Or, $k$.

\[  W(k) = 5k - 16   \]


Or, $A$.

\[  W(A) = 5A - 16   \]


It is always advantageous to select a variable that is a nice reminder of the domain measurement.

\end{example}









\begin{example} \textit{Two Pairs}


Suppose $g$ is a linear function containing $(0, 6)$ and $(-2, -5)$.

Then the template: $L(x) = m (x-a) + L(a)$ will need a rate of change



\[  m = \frac{-5 - 6}{\answer{-2} - \answer{0}} = \frac{-11}{-2} = \frac{11}{2}  \]

A formula for $g$ is


\[  g(t) = \frac{11}{2} (t-0) + 6     \]


\[  g(t) = \frac{11}{2} t + 6    \]



\end{example}










\subsection*{A Graph}




A linear function has a line as its graph.  The line includes a point for each pair in the function.  And, since it is a line, only two points are needed to draw the graph.  Any two distinct points will do.





\begin{example} \textit{A Line}


Let $g(k) = \frac{k}{2} - 4$ be a linear function with its natural domain. Draw a graph of $g$.


\begin{explanation}
Let's select two random domain numbers: $-4$ and $6$.  The function values at these domain numbers are $g(-4) = \answer{-6}$ and $g(6) = \answer{-1}$.  Therefore, the points $(-4, -6)$ and $(6, -1)$ are on the graph, which is a line.  We'll plot the two points and draw the line through them.


Below is the graph of $y=g(k)$.


\begin{image}
\begin{tikzpicture}
     \begin{axis}[
            	domain=-10:10, ymax=10, xmax=10, ymin=-10, xmin=-10,
            	axis lines =center, xlabel=$k$, ylabel=$y$, grid = major,
                ytick={-10,-8,-6,-4,-2,2,4,6,8,10},
                xtick={-10,-8,-6,-4,-2,2,4,6,8,10},
                ticklabel style={font=\scriptsize},
            	every axis y label/.style={at=(current axis.above origin),anchor=south},
            	every axis x label/.style={at=(current axis.right of origin),anchor=west},
            	axis on top,
          		]

        
        \addplot [draw=penColor, very thick, smooth, domain=(-8:8),<->] {0.5*x-4};

        \addplot[color=penColor,fill=penColor,only marks,mark=*] coordinates{(-4,-6)};
        \addplot[color=penColor,fill=penColor,only marks,mark=*] coordinates{(6,-1)};


    \end{axis}
\end{tikzpicture}
\end{image}


\end{explanation}

\end{example}










\begin{example} \textit{A Line}


Let $B(t)$ be a linear function.    Below is the graph of $y = B(t)$. From the graph obtain a formula for $B$.


\begin{image}
\begin{tikzpicture}
     \begin{axis}[
            	domain=-10:10, ymax=10, xmax=10, ymin=-10, xmin=-10,
            	axis lines =center, xlabel=$t$, ylabel=$y$, grid = major,
                ytick={-10,-8,-6,-4,-2,2,4,6,8,10},
                xtick={-10,-8,-6,-4,-2,2,4,6,8,10},
                ticklabel style={font=\scriptsize},
            	every axis y label/.style={at=(current axis.above origin),anchor=south},
            	every axis x label/.style={at=(current axis.right of origin),anchor=west},
            	axis on top,
          		]

        
        \addplot [draw=penColor, very thick, smooth, domain=(-3:8),<->] {(3/2)*x-5};

        %\addplot[color=penColor,fill=penColor,only marks,mark=*] coordinates{(-4,-6)};
        %\addplot[color=penColor,fill=penColor,only marks,mark=*] coordinates{(6,-1)};


    \end{axis}
\end{tikzpicture}
\end{image}

\begin{explanation}

From the graph, we can approximate the points $(0, -5)$ and $(3.3, 0)$.  These give a slope of

\[  slope = \frac{0 - \left(\answer{-5}\right)}{\answer{3.3} - 0} = \frac{5}{3.3} (\approx 1.5)     \]


This would give the formula $B(t) = \frac{5}{3.3} (t - 0) - 5 = \frac{5}{3.3} t - 5$

\end{explanation}

\end{example}


We used the point $(0, -5)$ to create the equation.  We could also have used $(3.3,0)$.



$B(t) = \frac{5}{3.3} (t - 3.3) - 0 = \frac{5}{3.3} t - \frac{5}{3.3}\cdot 3.3 = \frac{5}{3.3} t -  5$ \\


Same equation.  You can use any point on the line.















\subsection*{Linear Equations}


\begin{definition} \textbf{\textcolor{green!50!black}{Linear Equation}} \\


A \textbf{linear equation in $x_1$ and $x_2$} is an equation that is equivalent to 

\[
a \, x_1 + b \, x_2 + c = 0
\]


$x_1$ and $x_2$ are the \textbf{variables}.

$a$ and $b$ are constants called \textbf{coefficients}.

$c$ is called the \textbf{constant term}.

\end{definition}










\begin{definition} \textbf{\textcolor{green!50!black}{Solution}} \\


A \textbf{solution} to the linear equation  $a \, x_1 + b \, x_2 + c = 0$ is a pair of numbers that \textbf{satisfy} the equation.


\end{definition}

\begin{notation}


We often write solution pairs as an ordered pair: $(A, B)$. \\


One of the solution numbers is designated for $x_1$ and one is designated for $x_2$.  (That is what ordered means.) Upon substituting these numbers into the equation for $x_1$ and $x_2$, the resulting statement is a true statement, i.e. the equation is \textit{satisfied}. \\

If the order is not understood, then we might write $(x_1, x_2) = (A, B)$.


\end{notation}





With the order understood, each solution pair can be interpreted as coordinates for a point on the Cartesian plane and plotted as a dot.




\begin{example}






Consider the linear equation  $3 t + 2 y = 12$.


Let's select $t$ to be the first variable and $y$ the second variable.

With this agreement, $(4, 0)$ is a solution to the equation.   $(4,0)$ can be viewed as a point and plotted as a dot on the Cartesian plane.



\begin{image}
\begin{tikzpicture}
     \begin{axis}[
                domain=-10:10, ymax=10, xmax=10, ymin=-10, xmin=-10,
                axis lines =center, xlabel=$t$, ylabel=$y$, grid = major,
                ytick={-10,-8,-6,-4,-2,2,4,6,8,10},
                xtick={-10,-8,-6,-4,-2,2,4,6,8,10},
                ticklabel style={font=\scriptsize},
                every axis y label/.style={at=(current axis.above origin),anchor=south},
                every axis x label/.style={at=(current axis.right of origin),anchor=west},
                axis on top,
                ]

        
        %\addplot [draw=penColor, very thick, smooth, domain=(-3:8),<->] {(3/2)*x-5};

         \addplot[color=penColor,fill=penColor,only marks,mark=*] coordinates{(4,0)};


    \end{axis}
\end{tikzpicture}
\end{image}







If we plot a dot for each and every solution pair for the equation, then we obtain the graph of the equation.  






\begin{image}
\begin{tikzpicture}
     \begin{axis}[
                domain=-10:10, ymax=10, xmax=10, ymin=-10, xmin=-10,
                axis lines =center, xlabel=$t$, ylabel=$y$, grid = major,
                ytick={-10,-8,-6,-4,-2,2,4,6,8,10},
                xtick={-10,-8,-6,-4,-2,2,4,6,8,10},
                ticklabel style={font=\scriptsize},
                every axis y label/.style={at=(current axis.above origin),anchor=south},
                every axis x label/.style={at=(current axis.right of origin),anchor=west},
                axis on top,
                ]

        
        \addplot [draw=penColor, very thick, smooth, domain=(-2:8),<->] {(-1.5)*(x-4)};

         \addplot[color=penColor,fill=penColor,only marks,mark=*] coordinates{(4,0)};


    \end{axis}
\end{tikzpicture}
\end{image}

The graph of a linear equation is a line, which can be drawn using only two points from two solution pairs.














\end{example}

\textbf{\textcolor{purple!85!blue}{Geometry:}} Graphs of lines don't care which axis is ``vertical'' and which axis is ``horizontal''.  The graph is just a collection of points whose coordinates satisfy the linear equation.  \\
\textbf{\textcolor{purple!85!blue}{Analysis:}} The situation changes once we wish to interpret one of the variables as a function of the other variable. Once we designate one of the variables as the dependent variable or the function, then its axis becomes the ``vertical'' axis.  The other axis measures the independent variable and represents the domain of the function. The domain is measured horizontally in teh graph. \\


\textbf{Note:} This all works for all lines, except for vertical lines. All lines are graphs of linear functions, except vertical lines. \\



\begin{explanation}
If we interpret the first or left variable as representing domain values and the second or right variable as representing function values, then linear equations describe linear functions.  We can solve the equation for the function variable to obtain a formula.



In the example above, $3 t + 2 y = 12$ can be rewritten as $y = \answer{\frac{-3}{2}} \, t + \answer{6}$.

To emphasize that we are now thinking in terms of functions, we might write $y(t) = \tfrac{-3}{2} t + 6$.




We could just as easily have chosen $y$ as the domain variable and $t$ as the function variable.  In this case, $3 t + 2 y = 12$ can be rewritten as $t = \answer{\frac{-2}{3}} \, y + \answer{4}$.  We might write this as $t(y) = \tfrac{-2}{3} y + 4$




In either case, the function formula matches the ``$y = m \, x + b$'' template.  $\answer{m}$ is the slope of the line as well as the rate of change of the function.


\end{explanation}



























\begin{center}
\textbf{\textcolor{green!50!black}{ooooo-=-=-=-ooOoo-=-=-=-ooooo}} \\

more examples can be found by following this link\\ \link[More Examples of Linear Functions]{https://ximera.osu.edu/csccmathematics/precalculus1/precalculus1/linearFunctions/examples/exampleList}

\end{center}





\end{document}
