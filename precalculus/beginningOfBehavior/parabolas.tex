\documentclass{ximera}


\graphicspath{
  {./}
  {ximeraTutorial/}
  {basicPhilosophy/}
}

\newcommand{\mooculus}{\textsf{\textbf{MOOC}\textnormal{\textsf{ULUS}}}}


\usepackage{tkz-euclide}\usepackage{tikz}
\usepackage{tikz-cd}
\usetikzlibrary{arrows}
\tikzset{>=stealth,commutative diagrams/.cd,
  arrow style=tikz,diagrams={>=stealth}} %% cool arrow head
\tikzset{shorten <>/.style={ shorten >=#1, shorten <=#1 } } %% allows shorter vectors

\usetikzlibrary{backgrounds} %% for boxes around graphs
\usetikzlibrary{shapes,positioning}  %% Clouds and stars
\usetikzlibrary{matrix} %% for matrix
\usepgfplotslibrary{polar} %% for polar plots
\usepgfplotslibrary{fillbetween} %% to shade area between curves in TikZ
\usetkzobj{all}
\usepackage[makeroom]{cancel} %% for strike outs
%\usepackage{mathtools} %% for pretty underbrace % Breaks Ximera
%\usepackage{multicol}
\usepackage{pgffor} %% required for integral for loops



%% http://tex.stackexchange.com/questions/66490/drawing-a-tikz-arc-specifying-the-center
%% Draws beach ball
\tikzset{pics/carc/.style args={#1:#2:#3}{code={\draw[pic actions] (#1:#3) arc(#1:#2:#3);}}}



\usepackage{array}
\setlength{\extrarowheight}{+.1cm}
\newdimen\digitwidth
\settowidth\digitwidth{9}
\def\divrule#1#2{
\noalign{\moveright#1\digitwidth
\vbox{\hrule width#2\digitwidth}}}
























%%This is to help with formatting on future title pages.
\newenvironment{sectionOutcomes}{}{}


\title{Parabolas}

\begin{document}

\begin{abstract}
quadratic graphs
\end{abstract}
\maketitle





We have three forms for quadratic functions (or equations):



$\blacktriangleright$ $S(x) = A \, x^2 + B \, x + C$  with $A \ne 0$ : \textbf{\textcolor{blue!55!black}{Standard Form}}. \\


$\blacktriangleright$ $F(t) = A \, (t - r_1)(t - r_2)$  with $A \ne 0$ : \textbf{\textcolor{blue!55!black}{Factored Form}}. \\


$\blacktriangleright$ $V(d) = A \, (d - h)^2 + k$  with $A \ne 0$ : \textbf{\textcolor{blue!55!black}{Vertex Form}}. \\




Vertex form comes from completing the square.  It gets its name from the graph of quadratic functions.









\subsection*{Graphs of Quadratic Functions}



Quadratic functions all have parabolas for graphs.




\begin{image}
\begin{tikzpicture}
     \begin{axis}[
                domain=-10:10, ymax=10, xmax=10, ymin=-10, xmin=-10,
                axis lines =center, xlabel=$x$, ylabel=$y$,
                ytick={-10,-8,-6,-4,-2,2,4,6,8,10},
            	xtick={-10,-8,-6,-4,-2,2,4,6,8,10},
            	ticklabel style={font=\scriptsize},
                every axis y label/.style={at=(current axis.above origin),anchor=south},
                every axis x label/.style={at=(current axis.right of origin),anchor=west},
                axis on top,
                ]



        \addplot [draw=penColor, very thick, smooth, domain=(-7:9),<->] {-0.25*(x+4)*(x-6)};
        %\addplot [line width=1, gray, dashed,samples=100,domain=(-9.5:9.5)] ({3},{x});
        


        \addplot [color=penColor,only marks,mark=*] coordinates{(1,6.25)};
        \node[penColor] at (axis cs:2,8) {vertex};
        %\node[penColor] at (axis cs:4,1.5) {$(h, k)$};
        %\node[penColor] at (axis cs:5,-9) {$-0.5 x^2 - 5 x + 15.5$};



    \end{axis}
\end{tikzpicture}
\end{image}


\textbf{Graph Description:} Parabola opening down. Top point (vertex) at $(1,6.5)$. Intercepts are $(-4,0)$ and $(6,0)$.








\begin{image}
\begin{tikzpicture}
     \begin{axis}[
                domain=-10:10, ymax=10, xmax=10, ymin=-10, xmin=-10,
                axis lines =center, xlabel=$x$, ylabel=$y$,
                ytick={-10,-8,-6,-4,-2,2,4,6,8,10},
            	xtick={-10,-8,-6,-4,-2,2,4,6,8,10},
            	ticklabel style={font=\scriptsize},
                every axis y label/.style={at=(current axis.above origin),anchor=south},
                every axis x label/.style={at=(current axis.right of origin),anchor=west},
                axis on top,
                ]



        \addplot [draw=penColor, very thick, smooth, domain=(-9:7),<->] {0.25*(x+6)*(x-4)};
        %\addplot [line width=1, gray, dashed,samples=100,domain=(-9.5:9.5)] ({3},{x});
        


        \addplot [color=penColor,only marks,mark=*] coordinates{(-1,-6.25)};
        \node[penColor] at (axis cs:-3,-7) {vertex};
        %\node[penColor] at (axis cs:4,1.5) {$(h, k)$};
        %\node[penColor] at (axis cs:5,-9) {$-0.5 x^2 - 5 x + 15.5$};



    \end{axis}
\end{tikzpicture}
\end{image}

\textbf{Graph Description:} Parabola opening up. Bottom point (vertex) at $(-1,-6.5)$. Intercepts are $(-6,0)$ and $(4,0)$.








The extreme point on a parabola is called the \textbf{vertex}.  It is the lowest or highest point on the parabola, depending on whether the parabola opens up or down. 

This can be seen from the vertex form of the formula.




\[
V(x) = A \, (x - h)^2 + k
\]

The squared term, $A \, (x - h)^2$ has the same sign as $A$, except when it equals $0$.  That happens at $h$.  When $x = h$, then $V(h) = k$, which is either the minimum or maximum value of $V$.  The vertex is the graphical representation of the the extreme value of the quadratic function and where this extrema occurs in the domain.


In addition, the intercepts represent the zeros of the quadratic function and we have seen there can be $0$, $1$, or $2$ real zeros for a quadratic function.  Therefore, there can be $0$, $1$, or $2$ intercepts for a parabola.











\begin{image}
\begin{tikzpicture}
     \begin{axis}[
                domain=-10:10, ymax=10, xmax=10, ymin=-10, xmin=-10,
                axis lines =center, xlabel=$x$, ylabel=$y$,
                ytick={-10,-8,-6,-4,-2,2,4,6,8,10},
            	xtick={-10,-8,-6,-4,-2,2,4,6,8,10},
            	ticklabel style={font=\scriptsize},
                every axis y label/.style={at=(current axis.above origin),anchor=south},
                every axis x label/.style={at=(current axis.right of origin),anchor=west},
                axis on top,
                ]



        \addplot [draw=penColor, very thick, smooth, domain=(-9:7),<->] {0.25*(x+6)*(x-4)};
        %\addplot [line width=1, gray, dashed,samples=100,domain=(-9.5:9.5)] ({3},{x});
        


        \addplot [color=penColor,only marks,mark=*] coordinates{(-1,-6.25)};
        \addplot [color=penColor,only marks,mark=*] coordinates{(-6,0)};
        \addplot [color=penColor,only marks,mark=*] coordinates{(4,0)};

        \node[penColor] at (axis cs:-3,-7.25) {$(h, k)$};
        \node[penColor] at (axis cs:-4.5,1) {$(r_1, 0)$};
        \node[penColor] at (axis cs:6,1) {$(r_2, 0)$};



    \end{axis}
\end{tikzpicture}
\end{image}












\begin{example}   
Analysis


Analyze $f(x) = x^2 + 4 \, x - 5$ with its natural domain    \\ 




\textbf{\textcolor{red!75!green}{explanation}} 





\textbf{Category}


$f$ is a quadratic function, since its formula matches the standard form, $A \, x^2 + B \, x + C$. \\






\textbf{Domain}

$f$ is a quadratic function, therefore its domain is $(-\infty, \infty)$. \\



\textbf{Zeros}


We can use the quadratic formula to get the zeros.

\[
\frac{-4 \pm \sqrt{4^2 - 4(1)(-5)}}{2(1)} = \frac{-4 \pm \sqrt{36}}{2} = \frac{-4 \pm 6}{2}
\]

This give $-5$ and $1$ as the zeros.

This gives us the factored for : $f(x) = (x+5)(x-1)$.


It might have been quick to use the distributive property to factor the quadratic instead of using the quadratic formula. 

From the factored form, we can also see that $-5$ and $1$ are the zeros of $f(x)$.  \\


\textbf{Continuity}

$f$ is a quadratic function, therefore it is continuous. Quadratic functions do not have singularities.\\



\textbf{End-Behavior}

$f$ is a quadratic function, with a positive leading coefficient. Therefore, its end-behavior is unbounded positively in both directions. \\

\[
\lim\limits_{x \to -\infty} f(x) = \infty 
\]


\[
\lim\limits_{x \to \infty} f(x) = \infty 
\]




\textbf{Behavior (increasing and decreasing)}

$f$ is a quadratic function, with a positive leading coefficient. Therefore, it will decrease and then increase. It switches behavior at the domain number for the vertex, which is


\[
-\frac{b}{2a} = -\frac{4}{2(1)} = -2
\]

($-2$ is called the \textbf{critical number} of $f$.)



$f$ decreases on $(-\infty, -2)$ and increases on $(-2, \infty )$. \\




\textbf{Global and Local Maximum and Minimum}

For a quadratic function, the global and local extreme values are the same. \\



Since $f$ decreases on $(-\infty, -2)$ and increases on $(-2, \infty )$, we have a global minimum at $-2$.   The minimum value is $f(-2) = -9$. This is also a local minimum and is the only local extrema.\\


$\lim\limits_{x \to -\infty} f(x) = \infty$ tells us that there is no global maximum. \\




\textbf{Range}

$f$ is continuous, with a global minimum and unbounded positively. \\

The range is $[-9, \infty)$. \\



\textbf{\textcolor{purple!85!blue}{A Nice Graph}}

$f$ is a quadratic function and will have a parabola for a graph.


From any of the three forms, we can see that the leading coefficient is $1$, which is positive.  Thus, our parabola will open up.


From the factored form, we can see that $-5$ and $1$ are the zeros of $f(x)$.  These will be represented by the intercepts $(-5, 0)$ and $(1,0)$.



From the vertex form, we can see that the lowest point of the parabola will be $(-2, 9)$.  Or, the function $f$ has a global minimum value of $-9$, which occurs at $-2$ in the domain.









\begin{image}
\begin{tikzpicture}
     \begin{axis}[
                domain=-10:10, ymax=10, xmax=10, ymin=-10, xmin=-10,
                axis lines =center, xlabel=$x$, ylabel=$y$,
                ytick={-10,-8,-6,-4,-2,2,4,6,8,10},
            	xtick={-10,-8,-6,-4,-2,2,4,6,8,10},
            	ticklabel style={font=\scriptsize},
                every axis y label/.style={at=(current axis.above origin),anchor=south},
                every axis x label/.style={at=(current axis.right of origin),anchor=west},
                axis on top,
                ]



        \addplot [draw=penColor, very thick, smooth, domain=(-6.12:2.12),<->] {(x+5)*(x-1)};
        %\addplot [line width=1, gray, dashed,samples=100,domain=(-9.5:9.5)] ({3},{x});
        


        \addplot [color=penColor,only marks,mark=*] coordinates{(-2,-9)};
        \addplot [color=penColor,only marks,mark=*] coordinates{(-5,0)};
        \addplot [color=penColor,only marks,mark=*] coordinates{(1,0)};




    \end{axis}
\end{tikzpicture}
\end{image}





\textbf{\textcolor{blue!55!black}{$\blacktriangleright$ desmos graph}} 
\begin{center}
\desmos{rt8qwmnoqt}{400}{300}
\end{center}




\end{example}















\begin{onlineOnly}
\begin{center}
\textbf{\textcolor{green!50!black}{ooooo-=-=-=-ooOoo-=-=-=-ooooo}} \\

more examples can be found by following this link\\ \link[More Examples of Function Behavior]{https://ximera.osu.edu/csccmathematics/precalculus/precalculus/beginningOfBehavior/examples/exampleList}

\end{center}
\end{onlineOnly}



  



\end{document}
