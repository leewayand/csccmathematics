\documentclass{ximera}


\graphicspath{
  {./}
  {ximeraTutorial/}
  {basicPhilosophy/}
}

\newcommand{\mooculus}{\textsf{\textbf{MOOC}\textnormal{\textsf{ULUS}}}}


\usepackage{tkz-euclide}\usepackage{tikz}
\usepackage{tikz-cd}
\usetikzlibrary{arrows}
\tikzset{>=stealth,commutative diagrams/.cd,
  arrow style=tikz,diagrams={>=stealth}} %% cool arrow head
\tikzset{shorten <>/.style={ shorten >=#1, shorten <=#1 } } %% allows shorter vectors

\usetikzlibrary{backgrounds} %% for boxes around graphs
\usetikzlibrary{shapes,positioning}  %% Clouds and stars
\usetikzlibrary{matrix} %% for matrix
\usepgfplotslibrary{polar} %% for polar plots
\usepgfplotslibrary{fillbetween} %% to shade area between curves in TikZ
\usetkzobj{all}
\usepackage[makeroom]{cancel} %% for strike outs
%\usepackage{mathtools} %% for pretty underbrace % Breaks Ximera
%\usepackage{multicol}
\usepackage{pgffor} %% required for integral for loops



%% http://tex.stackexchange.com/questions/66490/drawing-a-tikz-arc-specifying-the-center
%% Draws beach ball
\tikzset{pics/carc/.style args={#1:#2:#3}{code={\draw[pic actions] (#1:#3) arc(#1:#2:#3);}}}



\usepackage{array}
\setlength{\extrarowheight}{+.1cm}
\newdimen\digitwidth
\settowidth\digitwidth{9}
\def\divrule#1#2{
\noalign{\moveright#1\digitwidth
\vbox{\hrule width#2\digitwidth}}}
























%%This is to help with formatting on future title pages.
\newenvironment{sectionOutcomes}{}{}


\author{Lee Wayand}

\begin{document}
\begin{exercise}  





Below is the graph of $k=h(g)$.  

\begin{image}
\includegraphics{../../pics/func_graphs/f12.png}
\end{image}









\begin{question} 


What is the domain of $h$?\\


\begin{multipleChoice}
\choice {$[3, 9]$}
\choice {$(3, 9)$}
\choice {$\{ 5 \} \cup (3, \infty)$}
\choice [correct]{$[3, \infty)$}
\end{multipleChoice}

\end{question}






\begin{question} 


What is the range of $h$?\\


\begin{multipleChoice}
\choice [correct]{$(-\infty, 1) \cup \{ 5 \}$}
\choice {$(-\infty, 3) \cup \{ 5 \}$}
\choice {$[3, \infty)$}
\choice {$(3, \infty)$}
\end{multipleChoice}


\end{question}









\begin{question} 



\[  h(4) = \answer[tolerance=0.25]{0}  \]

\end{question}










\begin{question} 



$h(g)$ is a decreasing function.
\begin{multipleChoice}
\choice [correct]{True}
\choice {False}
\end{multipleChoice}

\end{question}









\begin{question} 



The maximum of $h(g)$ is $\answer[tolerance=0.25]{5}$.   \\

The minimum of $h(g)$ is $\answer{DNE}$.
 

\end{question}






\begin{question} 


The maximum of $h(g)$ occurs at $\answer[tolerance=0.25]{3}$.   

\end{question}









\end{exercise}
\end{document}