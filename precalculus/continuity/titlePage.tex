\documentclass{ximera}


\graphicspath{
  {./}
  {ximeraTutorial/}
  {basicPhilosophy/}
}

\newcommand{\mooculus}{\textsf{\textbf{MOOC}\textnormal{\textsf{ULUS}}}}


\usepackage{tkz-euclide}\usepackage{tikz}
\usepackage{tikz-cd}
\usetikzlibrary{arrows}
\tikzset{>=stealth,commutative diagrams/.cd,
  arrow style=tikz,diagrams={>=stealth}} %% cool arrow head
\tikzset{shorten <>/.style={ shorten >=#1, shorten <=#1 } } %% allows shorter vectors

\usetikzlibrary{backgrounds} %% for boxes around graphs
\usetikzlibrary{shapes,positioning}  %% Clouds and stars
\usetikzlibrary{matrix} %% for matrix
\usepgfplotslibrary{polar} %% for polar plots
\usepgfplotslibrary{fillbetween} %% to shade area between curves in TikZ
\usetkzobj{all}
\usepackage[makeroom]{cancel} %% for strike outs
%\usepackage{mathtools} %% for pretty underbrace % Breaks Ximera
%\usepackage{multicol}
\usepackage{pgffor} %% required for integral for loops



%% http://tex.stackexchange.com/questions/66490/drawing-a-tikz-arc-specifying-the-center
%% Draws beach ball
\tikzset{pics/carc/.style args={#1:#2:#3}{code={\draw[pic actions] (#1:#3) arc(#1:#2:#3);}}}



\usepackage{array}
\setlength{\extrarowheight}{+.1cm}
\newdimen\digitwidth
\settowidth\digitwidth{9}
\def\divrule#1#2{
\noalign{\moveright#1\digitwidth
\vbox{\hrule width#2\digitwidth}}}
























%%This is to help with formatting on future title pages.
\newenvironment{sectionOutcomes}{}{}


\title{Continuity}

\begin{document}

\begin{abstract}
%Stuff can go here later if we want!
\end{abstract}
\maketitle






The study of functions falls into two general camps: the weird parts and the nice parts.  

The nice parts fit together nicely. Expectations hold true. Predictions are easy. Models made with nice functions are easier to analyze and interpret.


The weird parts are where we learn just what the real numbers can do.  The weird parts are difficult to predict. Analysis of weird parts means not trusting your intuition.  Your descriptions must be deliberate and specific about what exactly is going on and what is not.  Of course, mathematicians find the weird parts more interesting.  It is even more interesting when a function is constructed only of weird parts.


For Precalculus, weird begins with discontinunities and singularities.  What exactly is going on at a discontinuity or singularity?


To investigate discontinuities and singularities, we need to put our ideas of \textbf{closeness} into action.




\subsection*{Learning Outcomes}


\begin{sectionOutcomes}
In this section, students will 

\begin{itemize}
\item study removable discontinuities.
\item study jump discontinuities.
\item study asymptotic discontinuities.

\end{itemize}
\end{sectionOutcomes}
















\begin{center}
\textbf{\textcolor{green!50!black}{ooooo-=-=-=-ooOoo-=-=-=-ooooo}} \\

more examples can be found by following this link\\ \link[More Examples of Continuity]{https://ximera.osu.edu/csccmathematics/precalculus/precalculus/continuity/examples/exampleList}

\end{center}






\end{document}
