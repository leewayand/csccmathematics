\documentclass{ximera}


\graphicspath{
  {./}
  {ximeraTutorial/}
  {basicPhilosophy/}
}

\newcommand{\mooculus}{\textsf{\textbf{MOOC}\textnormal{\textsf{ULUS}}}}


\usepackage{tkz-euclide}\usepackage{tikz}
\usepackage{tikz-cd}
\usetikzlibrary{arrows}
\tikzset{>=stealth,commutative diagrams/.cd,
  arrow style=tikz,diagrams={>=stealth}} %% cool arrow head
\tikzset{shorten <>/.style={ shorten >=#1, shorten <=#1 } } %% allows shorter vectors

\usetikzlibrary{backgrounds} %% for boxes around graphs
\usetikzlibrary{shapes,positioning}  %% Clouds and stars
\usetikzlibrary{matrix} %% for matrix
\usepgfplotslibrary{polar} %% for polar plots
\usepgfplotslibrary{fillbetween} %% to shade area between curves in TikZ
\usetkzobj{all}
\usepackage[makeroom]{cancel} %% for strike outs
%\usepackage{mathtools} %% for pretty underbrace % Breaks Ximera
%\usepackage{multicol}
\usepackage{pgffor} %% required for integral for loops



%% http://tex.stackexchange.com/questions/66490/drawing-a-tikz-arc-specifying-the-center
%% Draws beach ball
\tikzset{pics/carc/.style args={#1:#2:#3}{code={\draw[pic actions] (#1:#3) arc(#1:#2:#3);}}}



\usepackage{array}
\setlength{\extrarowheight}{+.1cm}
\newdimen\digitwidth
\settowidth\digitwidth{9}
\def\divrule#1#2{
\noalign{\moveright#1\digitwidth
\vbox{\hrule width#2\digitwidth}}}
























%%This is to help with formatting on future title pages.
\newenvironment{sectionOutcomes}{}{}


\title{Behavior}

\begin{document}

\begin{abstract}
to algebra
\end{abstract}
\maketitle



Everything gets more complicated the deeper you look. 


Mathematics is no different. 


We are just getting started with the derivative and the more we examine it, the more complicated it will become. 

We are ready for the next layer to this story. 




\subsection*{So far...}


Our definition of the derivative has been a graphical definition.

\begin{idea} \textbf{\textcolor{blue!55!black}{Slope of Tangent Line}} 

Let $f(x)$ be a function with domain $D$. \\
Let $a \in D$ be a domain number. 

Then $f(x)$ has a graph. 

There are two possibilities:  


\textbf{\textcolor{blue!55!black}{1)}} \textbf{the graph has a tangent line at $(a, f(a))$}

\textbf{\textcolor{blue!55!black}{2)}} \textbf{the graph does not have a tangent line at $(a, f(a))$}




If the graph has a tangent line at $(a, f(a))$ and this tangent line has a slope, then


\begin{center}

$f'(a)$ = the slope of the tangent line at $(a, f(a))$,

\end{center}



Otherwise, we say that $f'(a)$ does not exist (DNE).






\end{idea}

There are several ways in which $f'(a)$ might not exist. 



\textbf{\textcolor{blue!55!black}{Vertical Tangent Line:}} 

There may be a tangent line, but the tangent line is vertical and thus has no slope. \\
$f(x) = 4 \sqrt[3]{x-1}$ is an example.


\begin{image}
\begin{tikzpicture}
  \begin{axis}[
            domain=-10:10, ymax=10, xmax=10, ymin=-10, xmin=-10,
            axis lines =center, xlabel=$x$, ylabel=$y$, grid = major,
            ytick={-10,-8,-6,-4,-2,2,4,6,8,10},
            xtick={-10,-8,-6,-4,-2,2,4,6,8,10},
            ticklabel style={font=\scriptsize},
            every axis y label/.style={at=(current axis.above origin),anchor=south},
            every axis x label/.style={at=(current axis.right of origin),anchor=west},
            axis on top
          ]
          
			\addplot [line width=1, gray, dashed,samples=200,domain=(-10:10),<->] ({1},{x});
          	\addplot [line width=2, penColor, smooth,samples=200,domain=(-8:1),<-] {-4*((1-x)^0.333)};
          	\addplot [line width=2, penColor, smooth,samples=200,domain=(1:8),->] {4*((x-1)^0.333)};

          
            %\addplot [line width=1, gray, dashed,samples=200,domain=(-10:10),<->] ({x},{x});


          %\addplot[color=penColor,fill=penColor,only marks,mark=*] coordinates{(-2,-4)};


           

  \end{axis}
\end{tikzpicture}
\end{image}


This graph has a tangent line at $(1,0)$.  However, the tangent line is vertical, which means it doesn't have slope.  

Therefore, $f(1)$ does not exist.







\textbf{\textcolor{blue!55!black}{Corners:}} 

There may not be a tangent line, like at a corner. \\
$f(x) = | x - 2 | - 3$ is an example.


\begin{image}
\begin{tikzpicture}
  \begin{axis}[
            domain=-10:10, ymax=10, xmax=10, ymin=-10, xmin=-10,
            axis lines =center, xlabel=$x$, ylabel=$y$, grid = major,
            ytick={-10,-8,-6,-4,-2,2,4,6,8,10},
            xtick={-10,-8,-6,-4,-2,2,4,6,8,10},
            ticklabel style={font=\scriptsize},
            every axis y label/.style={at=(current axis.above origin),anchor=south},
            every axis x label/.style={at=(current axis.right of origin),anchor=west},
            axis on top
          ]
          

          \addplot [line width=2, penColor, smooth,samples=200,domain=(-8:8),<->] {abs(x-2)-3};
          %\addplot [line width=2, penColor, smooth,samples=200,domain=(0:8),->] {4*((x-1)^0.333)};

          %\addplot [line width=1, gray, dashed,samples=200,domain=(-10:10),<->] ({1},{x});
            %\addplot [line width=1, gray, dashed,samples=200,domain=(-10:10),<->] ({x},{x});


          %\addplot[color=penColor,fill=penColor,only marks,mark=*] coordinates{(-2,-4)};


           

  \end{axis}
\end{tikzpicture}
\end{image}


This graph does not have a tangent line at $(2,-3)$.  When you approach $(2,-3)$ from the right or left, you get two different lines that describe the graph upon approaching.  They don't match. 

Therefore, $f(2)$ does not exist.

















\textbf{\textcolor{blue!55!black}{Discontinuity:}} 


There might be a discontinuity. 
For example,


\[
f(x) = 
\begin{cases}
  -x &\text{if $x<-1$,}\\
  x-4 &\text{if $x\ge -1$}.
\end{cases}
\]


\begin{image}
\begin{tikzpicture}
  \begin{axis}[
            domain=-10:10, ymax=10, xmax=10, ymin=-10, xmin=-10,
            axis lines =center, xlabel=$x$, ylabel=$y$, grid = major,
            ytick={-10,-8,-6,-4,-2,2,4,6,8,10},
            xtick={-10,-8,-6,-4,-2,2,4,6,8,10},
            ticklabel style={font=\scriptsize},
            every axis y label/.style={at=(current axis.above origin),anchor=south},
            every axis x label/.style={at=(current axis.right of origin),anchor=west},
            axis on top
          ]
          

          \addplot [line width=2, penColor, smooth,samples=200,domain=(-8:-1),<-] {-x};
          \addplot [line width=2, penColor, smooth,samples=200,domain=(-1:8),->] {x-4};
          %\addplot [line width=2, penColor, smooth,samples=200,domain=(0:8),->] {4*((x-1)^0.333)};

          %\addplot [line width=1, gray, dashed,samples=200,domain=(-10:10),<->] ({1},{x});
            %\addplot [line width=1, gray, dashed,samples=200,domain=(-10:10),<->] ({x},{x});


          \addplot[color=penColor,fill=penColor,only marks,mark=*] coordinates{(-1,-5)};
          \addplot[color=penColor,fill=white,only marks,mark=*] coordinates{(-1,1)};


           

  \end{axis}
\end{tikzpicture}
\end{image}


This graph does not have a tangent line at $(-1,-5)$.  When you approach $(-1,-5)$ from the right or left, you get two different lines that describe the graph upon approaching.  They don't match. 

Therefore, $f(-1)$ does not exist. 







\begin{idea} \textbf{\textcolor{blue!55!black}{Critical Numbers}} 



We use the derivative to establish a function's behavior: where a function is increasing or decreasing, or where local maximums and minimums occur.


Critical numbers are domain numbers where the derivative equals $0$ of does not exist. 

Critical numbers are candidates for locations of maximums and minimums.


\end{idea}



However, there are other places where a function might exhibit a maximum or minimum. 

For instances, at an endpoint. 










\textbf{\textcolor{blue!55!black}{An Endpoint:}} 


When we write our domains in reduced interval notation, the ends of our intervals might be included in the domain. These are not discontinuities, but they might be locations of extreme values of the function. 


\[
f(x) = x-4 \, \text{ on } \, [-1, \infty)
\]


\begin{image}
\begin{tikzpicture}
  \begin{axis}[
            domain=-10:10, ymax=10, xmax=10, ymin=-10, xmin=-10,
            axis lines =center, xlabel=$x$, ylabel=$y$, grid = major,
            ytick={-10,-8,-6,-4,-2,2,4,6,8,10},
            xtick={-10,-8,-6,-4,-2,2,4,6,8,10},
            ticklabel style={font=\scriptsize},
            every axis y label/.style={at=(current axis.above origin),anchor=south},
            every axis x label/.style={at=(current axis.right of origin),anchor=west},
            axis on top
          ]
          

          %\addplot [line width=2, penColor, smooth,samples=200,domain=(-8:-1),<-] {-x};
          \addplot [line width=2, penColor, smooth,samples=200,domain=(-1:8),->] {x-4};
          %\addplot [line width=2, penColor, smooth,samples=200,domain=(0:8),->] {4*((x-1)^0.333)};

          %\addplot [line width=1, gray, dashed,samples=200,domain=(-10:10),<->] ({1},{x});
            %\addplot [line width=1, gray, dashed,samples=200,domain=(-10:10),<->] ({x},{x});


          \addplot[color=penColor,fill=penColor,only marks,mark=*] coordinates{(-1,-5)};
          %\addplot[color=penColor,fill=white,only marks,mark=*] coordinates{(-1,1)};


           

  \end{axis}
\end{tikzpicture}
\end{image}


This graph has a tangent line at $(-1,-5)$.  

The line $y = x - 4$ is a tangent line to this graph at $(-1,-5)$ . 

Therefore, $f'(-1)$ exists and $f'(-1) = 1$.  

$-1$ is not a critical number. 


\begin{center}
\textbf{\textcolor{red!80!black}{We would prefer that $-1$ be a critical number for this function.}}
\end{center}














\subsection*{Revision}


With this example as motivation, we are going to revise our definition of the derivative. 


When we say ``derivative'', we will insist that we are talking both sides.  ``Derivative'' will mean a tangent line on both sides and they match up.  We can accomplish this in the domain by insisting that there is domain space around the domain number corresponding to the tangent point. 




\begin{definition} (from earlier) \textbf{\textcolor{green!50!black}{Derivative}}  

Let $f(x)$ be a function with domain $D$. 
Let $a \in D$ be a domain number. 



For $f'(a)$ to exist, we need two things:

\begin{itemize}
\item There is an open interval inside the domain, which contains $a$.
\item The graph has a tangent line at $(a, f(a))$ and this tangent line has a slope.
\end{itemize}



\begin{center}

$f'(a)$ = the slope of the tangent line at $(a, f(a))$,

\end{center}



Otherwise, we say that $f'(a)$ does not exist (DNE).






\end{definition}


Now, endpoints can be critical numbers, because they do not have an open interval around them in the domain.  That automatically means the derivative does not exist there. 






\begin{itemize}
\item This gives a more succinct way of talking about changing function behavior. 
\item This gives a more succinct way of talking about locations of extreme values. 
\end{itemize}




\begin{idea} \textbf{\textcolor{blue!55!black}{Behavior}} 


A function's behavior can change at \textbf{critical numbers} or \textbf{singularities}.






\end{idea}


\begin{example}


Suppose the function $g$ is defined on the domain $(-7, -1) \cup \{ 3 \} \cup (5, 10]$.

$3$ is automatically a critical number, because there is no open interval containing $3$ that is entirely inside the domain.

$10$ is automatically a critical number, because there is no open interval containing $10$ that is entirely inside the domain.



Singletons and endpoints are automatically critical numbers and candidates for locations of extreme values of function. 




In addition, with no more information, other than the domain, we would keep $-7$, $-1$, and $5$ as possible singularities. 




\end{example}



Calculus has further revisions.


Calculus will want more detail about endpoints.  Rather than just saying that they are critical numbers, Calculus will define left and right derivatives.  This will give us better ways to describe function behavior. 

















































Algebra's strength is in identifying zeros. 


This is largely due to the Zero Property Property, which says that 



\[
\text{ If } \, a \cdot b = 0 \, \text{ then either } a = 0 \, \text{ or } \, b = 0
\]


Unless you know of some handy property of a function, our procedure often begin with ``get eveything on one side and $0$ on the other and factor''.




Contrst that to increasing and decreasing, which do not involve equations to solve.  They are comparisons of movement or change between the range and domain. 

Our algebra is not that good at such comparisons. 


The derivative rephrases this comparison of change back into algebra, where we have methods. 





\begin{template}

Let $f$ be a function. 

Then the derivative is denoted as $f'$. 


$f$ is increasing on intervals where $f'$ is positive. 


$f$ is decreasing on intervals where $f'$ is negative. 


\end{template}

This allows us to bring our algebraic tools to the question of function behavior. 











\begin{example} Quadratics 


Any quadratic can be written in the form $a \, x^2 + b \, x + c$. 

The derivative is given by $2a \, x + b$.


The derivative switches signs at $\frac{-b}{2a}$, which is the only critical. It corresponds to the first coordinate of the vertex. 



\textbf{\textcolor{red!70!darkgray}{$\blacktriangleright$}} If $a < 0$, then 

\begin{itemize}
\item $f' > 0$ on $\left( -\infty, \frac{-b}{2a} \right)$ and $f$ is increasing on $\left( -\infty, \frac{-b}{2a} \right)$. \\
\item $f' < 0$ on $\left( \frac{-b}{2a}, -\infty \right)$ and $f$ is decreasing on $\left( \frac{-b}{2a}, \infty \right)$. 
\end{itemize}







\textbf{\textcolor{red!70!darkgray}{$\blacktriangleright$}} If $a > 0$, then 

\begin{itemize}
\item $f' < 0$ on $\left( -\infty, \frac{-b}{2a} \right)$ and $f$ is decreasing on $\left( -\infty, \frac{-b}{2a} \right)$. \\
\item $f' > 0$ on $\left( \frac{-b}{2a}, -\infty \right)$ and $f$ is increasing on $\left( \frac{-b}{2a}, \infty \right)$. 
\end{itemize}






\end{example}







































The sign of the derivative tells us where the original function is increasing or decreasing.  Therefore, we are very interested in where the derivative changes sign.

This can happen at three types of numbers. 


\textbf{\textcolor{red!70!darkgray}{$\blacktriangleright$}} The derivative can change signs across a zero. 


\textbf{\textcolor{red!70!darkgray}{$\blacktriangleright$}} The derivative can change signs across a discontiuity. 


\textbf{\textcolor{red!70!darkgray}{$\blacktriangleright$}}  The derivative can change signs across a singularity. 



\begin{center}
The operative word here is \textbf{\textcolor{red!80!black}{CAN}}.
\end{center}




\begin{template}  \textbf{\textcolor{blue!55!black}{Critical Numbers}}  

Let $f$ be a function with derivative, $f'$. 


A \textbf{\textcolor{green!50!black}{critical number}} of $f$ is a domain number of $f$ where $f' = 0$ or $f'$ does not exist.


\end{template}

Functions \textbf{\textcolor{red!80!black}{CAN}} switch behavior at critical numbers. 








\begin{template}  \textbf{\textcolor{blue!55!black}{Singularities}}  

Let $f$ be a function. 

A \textbf{\textcolor{green!50!black}{singularity}} of $f$ is a non-domain number where $f$ has weird behavior.



\begin{explanation}

We have expressed somewhat stable descriptions for discontiuities, but not for singularities.  Calculus will help us with this.
\end{explanation}


\end{template}

Functions \textbf{\textcolor{red!80!black}{CAN}} switch behavior at singularities. 




Hunting down zeros, discontinuities, and singularities have risen to the top of our to-do list.  They are the crux to function behavior. 
































\begin{center}
\textbf{\textcolor{green!50!black}{ooooo-=-=-=-ooOoo-=-=-=-ooooo}} \\

more examples can be found by following this link\\ \link[More Examples of the Derivative]{https://ximera.osu.edu/csccmathematics/precalculus/precalculus/theDerivative/examples/exampleList}

\end{center}








\end{document}
