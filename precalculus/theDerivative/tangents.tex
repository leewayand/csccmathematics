\documentclass{ximera}


\graphicspath{
  {./}
  {ximeraTutorial/}
  {basicPhilosophy/}
}

\newcommand{\mooculus}{\textsf{\textbf{MOOC}\textnormal{\textsf{ULUS}}}}


\usepackage{tkz-euclide}\usepackage{tikz}
\usepackage{tikz-cd}
\usetikzlibrary{arrows}
\tikzset{>=stealth,commutative diagrams/.cd,
  arrow style=tikz,diagrams={>=stealth}} %% cool arrow head
\tikzset{shorten <>/.style={ shorten >=#1, shorten <=#1 } } %% allows shorter vectors

\usetikzlibrary{backgrounds} %% for boxes around graphs
\usetikzlibrary{shapes,positioning}  %% Clouds and stars
\usetikzlibrary{matrix} %% for matrix
\usepgfplotslibrary{polar} %% for polar plots
\usepgfplotslibrary{fillbetween} %% to shade area between curves in TikZ
\usetkzobj{all}
\usepackage[makeroom]{cancel} %% for strike outs
%\usepackage{mathtools} %% for pretty underbrace % Breaks Ximera
%\usepackage{multicol}
\usepackage{pgffor} %% required for integral for loops



%% http://tex.stackexchange.com/questions/66490/drawing-a-tikz-arc-specifying-the-center
%% Draws beach ball
\tikzset{pics/carc/.style args={#1:#2:#3}{code={\draw[pic actions] (#1:#3) arc(#1:#2:#3);}}}



\usepackage{array}
\setlength{\extrarowheight}{+.1cm}
\newdimen\digitwidth
\settowidth\digitwidth{9}
\def\divrule#1#2{
\noalign{\moveright#1\digitwidth
\vbox{\hrule width#2\digitwidth}}}
























%%This is to help with formatting on future title pages.
\newenvironment{sectionOutcomes}{}{}


\title{Tangent Lines}

\begin{document}

\begin{abstract}
zooming in
\end{abstract}
\maketitle


 
 


\subsection*{Tangent Lines}

The story about tangent lines we told involved parabolas, but it extends to any curve.






Suppose we have the cubic function $f(x) = \frac{1}{10} (3x - 2)(x+5)(x+3)$. 

This graph contains the point $(0,-3)$.


The line $y=\frac{29}{10}x-3$  goes through the point.




\begin{image}
\begin{tikzpicture}
  \begin{axis}[
            domain=-8:6, ymax=10, xmax=10, ymin=-10, xmin=-10,
            axis lines =center, xlabel=$x$, ylabel=$y$, grid = major,
            ytick={-20,-18,-16,-14,-12,-10,-8,-6,-4,-2,2,4,6,8,10,12,14,16,18,20},
            xtick={-10,-8,-6,-4,-2,2,4,6,8,10},
            ticklabel style={font=\scriptsize},
            every axis y label/.style={at=(current axis.above origin),anchor=south},
            every axis x label/.style={at=(current axis.right of origin),anchor=west},
            axis on top
          ]
          

          \addplot [line width=2, penColor2, smooth,domain=(-6.2:1.5),<->] {0.1*(3*x-2)*(x+5)*(x+3)};

          \addplot [line width=1, penColor, smooth,domain=(-2:4),<->] {2.9*x-3};

 

  \end{axis}
\end{tikzpicture}
\end{image}





The curve and line have a special relationship, which we can see by zooming in around the point.






\textbf{\textcolor{blue!55!black}{$\blacktriangleright$ desmos graph}} 
\begin{center}
\desmos{j3rltvwd0l}{400}{300}
\end{center}





\textbf{\textcolor{blue!55!black}{$\blacktriangleright$ desmos graph}} 
\begin{center}
\desmos{el01gfmoiv}{400}{300}
\end{center}





\textbf{\textcolor{blue!55!black}{$\blacktriangleright$ desmos graph}} 
\begin{center}
\desmos{sbfbfh7oqx}{400}{300}
\end{center}






\textbf{\textcolor{blue!55!black}{$\blacktriangleright$ desmos graph}} 
\begin{center}
\desmos{izri7rfpqf}{400}{300}
\end{center}





\begin{image}
\begin{tikzpicture}
  \begin{axis}[
            domain=-0.02:0.02, ymax=-2.9, xmax=0.02, ymin=-3.1, xmin=-0.02,
            axis lines =center, xlabel=$x$, ylabel=$y$, grid = major,
            ytick={-3.05,-3,-2.95},
            xtick={-0.02,-0.01,0.01,0.02},
            ticklabel style={font=\scriptsize},
            every axis y label/.style={at=(current axis.above origin),anchor=south},
            every axis x label/.style={at=(current axis.right of origin),anchor=west},
            axis on top
          ]
          

          \addplot [line width=2, penColor2, smooth,domain=(-0.02:0.02),<->] {0.1*(3*x-2)*(x+5)*(x+3)};

          \addplot [line width=1, penColor, smooth,domain=(-0.02:0.02),<->] {2.9*x-3};

 

  \end{axis}
\end{tikzpicture}
\end{image}


As you zoom in, the graph slowly looks more and more like the line. 


The line does the best job of approximating the graph at the point $(0,-3)$. 


And, this is the only line that will work for the point $(0,-3)$.  Any other line, besides this one, will have a permanant nonzero angle between the graph and the line.  

The graph and this particular line have the same ``slope'' at the point. 


This line is called the \textbf{\textcolor{blue!55!black}{tangent line}} for $W(t) = \frac{1}{10} (3x - 2)(x+5)(x+3)$ at the point $(0,-3)$.


Each point on the graph of the function has its own unique tangent line with its own slope.




\begin{definition} (from earlier) \textbf{\textcolor{green!50!black}{Tangent Line}}


Suppose $C$ is a curve, like the graph of a function. 

Let $(a, b)$ be a point on the curve.

The line tangent to the curve, $C$, at the point $(a, b)$ is the line that goes through the point $(a, b)$ and does the best job of approximating the curve at the point $(a, b)$.

This line is called a \textbf{tangent line} or a \textbf{tangent} at the point $(a, b)$.




\end{definition}


Since ``slope'' is a characteristic of a line and a curve is not a line, curves do not have a slope as we know that word. 

However, it is easy to see that curves have something that seems a lot like a slope at points. We will adopt the slope of the tangent as the slope of the curve at a point. 

As the cubic curve above demonstrates, curves might have different slopes at different points. 

We want to quantize this idea of a slope for a curve at a point. 

If we can quantize it, then we can build a function for it and use that to measure a rate of change for our curve. This is the first step toward Calculus. It will take us a while to quantize this idea of slope for a curve in a logical tool. A tool we will call the \textbf{derivative}.

The way Calculus gets to this tangent line is through secant lines on either side of the tangent point.




































































\subsection*{Intervals and Secants}



Continuing with our cubic function $W(t) = \frac{1}{10} (3x - 2)(x+5)(x+3)$ and the tangent point $(0,-3)$ and the tangent line $y=\frac{29}{10}x-3$   through the tangent point, we are going to investigate pairs of secant lines.







\begin{image}
\begin{tikzpicture}
  \begin{axis}[
            domain=-8:6, ymax=10, xmax=10, ymin=-10, xmin=-10,
            axis lines =center, xlabel=$x$, ylabel=$y$, grid = major,
            ytick={-20,-18,-16,-14,-12,-10,-8,-6,-4,-2,2,4,6,8,10,12,14,16,18,20},
            xtick={-10,-8,-6,-4,-2,2,4,6,8,10},
            ticklabel style={font=\scriptsize},
            every axis y label/.style={at=(current axis.above origin),anchor=south},
            every axis x label/.style={at=(current axis.right of origin),anchor=west},
            axis on top
          ]
          

          \addplot [line width=2, penColor2, smooth,domain=(-6.2:1.5),<->] {0.1*(3*x-2)*(x+5)*(x+3)};

          \addplot [line width=1, penColor, smooth,domain=(-2:4),<->] {2.9*x-3};

 

  \end{axis}
\end{tikzpicture}
\end{image}




Let $h > 0$ be a small positive real number.

With this value of $h$, we can consider two intervals: $(-h,0)$ and $(0,h)$.

We image the two secant lines:

One: The secant line through  $(-h,0)$ and $(0,-3)$.

\[
\frac{f(0) - f(-h)}{0-(-h)} x - 3 = \frac{-3-f(-h)}{h} x - 3
\]



Two: The secant line through  $(0,-3)$ and $(h,f(h))$.

\[
\frac{f(h) - f(0)}{h-0} x - 3 = \frac{f(h)+3}{h} x - 3
\]


















\begin{image}
\begin{tikzpicture}
  \begin{axis}[
            domain=-8:6, ymax=10, xmax=10, ymin=-10, xmin=-10,
            axis lines =center, xlabel=$x$, ylabel=$y$, grid = major,
            ytick={-20,-18,-16,-14,-12,-10,-8,-6,-4,-2,2,4,6,8,10,12,14,16,18,20},
            xtick={-10,-8,-6,-4,-2,2,4,6,8,10},
            ticklabel style={font=\scriptsize},
            every axis y label/.style={at=(current axis.above origin),anchor=south},
            every axis x label/.style={at=(current axis.right of origin),anchor=west},
            axis on top
          ]
          

          \addplot [line width=2, penColor2, smooth,domain=(-6.2:1.5),<->] {0.1*(3*x-2)*(x+5)*(x+3)};

          \addplot [line width=1, penColor, smooth,domain=(-2:4),<->] {2.9*x-3};

 


          \addplot [line width=1, penColor3, smooth,domain=(-6:0),<->] {x-3};
          \addplot [line width=1, penColor3, smooth,domain=(0:2),<->] {5.4*x-3};

          \addplot[color=black,only marks,mark=*] coordinates{(0,-3)}; 
          \addplot[color=penColor3,only marks,mark=*] coordinates{(-1,-4)}; 
          \addplot[color=penColor3,only marks,mark=*] coordinates{(1,2.4)}; 


  \end{axis}
\end{tikzpicture}
\end{image}






\begin{onlineOnly}
\textbf{\textcolor{blue!55!black}{$\blacktriangleright$ desmos graph}} 
\begin{center}
\desmos{vilqspjxsf}{400}{300}
\end{center}
\end{onlineOnly}




This general set up works for any positive value of $h$.

However, we want the slope of the tangent line.

We want $h=0$.

Unfortunately, if $h=0$, then we would be lokoing at secant lines that go through $(0,-3)$ and $(0,-3)$.


The slope of our secant lines would be $\frac{0}{0}$, which has no meaning.







We have an idea that will help.

We have the idea of \textbf{\textcolor{blue!55!black}{expected values}}.




What is the rate of change \textbf{\textcolor{red!90!darkgray}{AT}} a point? 


It is the expected value coming from the slopes of the secant lines as $h$ tends to $0$.






\begin{idea}

You cannot calculate the rate of change over an interval with $0$ length.  

Therefore, we are inventing an interpretation for the rate of change at a point. 

We are calling our invention the \textit{instantaneous rate of change}.


\end{idea}











\subsection*{Tangent Lines}

Suppose $f$ is a function.  The graph of $f$ is the collection of points whose coordinates are pairs in $f$.  That is, their coordinates look like $(d, f(d))$.

Suppose $a$ and $b$ are two distinct domain numbers of $f$.  Then the line through $(a, f(a))$ and $(b, f(b))$ is called a secant line.  The slope of this secant line equals the rate of change of $f$ over the interval $[a, b]$.

Tangent lines are degenerate secants. Secant lines need two points on the graph of a function.  A tangent line is a secant line where the two points are the same point. Tangent lines are secant lines created from an interval of length $0$.  But tangent lines are still lines.  They still have a slope.


\textbf{\textcolor{red!90!darkgray}{$\blacktriangleright$}} If the slope of a secant line corresponds to the function's rate of change over an interval, then the slope of a tangent line corresponds to the function's rate of change at a single number.


We are calling this slope of the tangent line the \textbf{\textcolor{purple!85!blue}{instantaneous rate of change}} at the domain number corresponding to the tangent point.




\begin{definition} (from earlier) \textbf{\textcolor{green!50!black}{Instantaneous Rate of Change}}  


Let $f$ be a function. Let $a$ be a number in the domain of $f$.

If the graph of $y = f(x)$ has a non-vertical tangent line at the point $(a, f(a))$, then the slope of this tangent line is the \textbf{instantaneous rate of change} of $f$ \textbf{at} a.

\end{definition}

















\subsection*{the Derivative}

Let $f(x)$ be any function.

Let $(x_0, y_0)$ be a point on the graph of $y = f(x)$.

We have three possibilities:


\begin{itemize}
\item The graph has a tangent line at $(x_0, y_0)$ with a slope.
\item The graph has a tangent line at $(x_0, y_0)$ without a slope.
\item The graph does not have a tangent line at $(x_0, y_0)$.
\end{itemize}



If the point has a tangent line with a slope, which we are calling the instantaneous rate of change of $Q$ at $x_0$, then we can create a new function from this.



\begin{idea} \textbf{\textcolor{blue!55!black}{the Derivative Function}}  


Let $f(x)$ be any function.

Define a new function called \textbf{the derivative of f}, denoted by $f'(x)$.


It domain is all of the domain values, $a$, of $f$ where the tangent line at $(a, f(a))$ has a slope.

\[
f'(a) = \text{slope of tangent line} at (a, f(a))
\]


\end{idea}


We now have a function for measuring the instantaneous rate of change of any function at any domain number (where there is a tangent line). 


\



\begin{notation} \textbf{\textcolor{blue!55!black}{Language}} 


We are just beginning our investigation of function behavior, so we have given our new function a very descriptive name, \textbf{\textcolor{blue!55!black}{Instantaneous Rate of Change}},  \textbf{\textcolor{blue!55!black}{$iRoC$}}. 


Calculus uses the name \textbf{\textcolor{blue!55!black}{Derivative}} and it uses a little prime sign as the notation.



\[
iRoC_f(x) = f'(x)
\]


People pronounce $f'(x)$ as ``f prime of $x$''.


\end{notation}

We will use both names.  This will keep the idea fresh in our heads and also give us experience with Calculus notation.






\begin{notation} \textbf{\textcolor{red!70!black}{A Peek Ahead}}


We are beginning our investigation of function analysis at the beginning.  Our formulas have one variable.  But, later, in Calculus, it may not be so clear on what the variable is or what the function is. \\

For this reason, Calculus introduces more notation that is very explicit on this matter.



\[
iRoC_f(x) = f'(x) = \frac{df}{dx}
\]


People pronounce $\frac{df}{dx}$ as ``dfdx''.  They just say the letters.


\end{notation}











\begin{conclusion}  \textbf{\textcolor{green!50!black}{Behavior}}

$\blacktriangleright$ When the instantaneous rate of change function is negative, $iRoC_f(x) < 0$, then $f(x)$ is decreasing. \\

$\blacktriangleright$ When the instantaneous rate of change function is positive, $iRoC_f(x) > 0$, then $f(x)$ is increasing. \\

$\blacktriangleright$ When the instantaneous rate of change function is zero, $iRoC_f(x) = 0$, then $f(x)$ is neither increasing nor decreasing and the graph of $y = f(x)$ is flat. 


We can say the exat same thing with the word ``derivative''. \\



$\blacktriangleright$ When the derivative is negative, $f'(x) < 0$, then $f(x)$ is decreasing. \\

$\blacktriangleright$ When the derivative is positive, $f'(x) > 0$, then $f(x)$ is increasing. \\

$\blacktriangleright$ When the derivative is zero, $f'(x) = 0$, then $f(x)$ is neither increasing nor decreasing and the graph of $y = f(x)$ is flat or horizontal. 



\end{conclusion}


As we can see, the behavior of our function, $f(x)$, can change drastically where $iRoC_f(x) = 0$.  Such domain numbers deserve a special name.



\begin{definition} \textbf{\textcolor{green!50!black}{Critical Number}}  


Let $f$ be a function. Let $x_0$ be a number in the domain of $f$ such that $iRoC_f(x_0) = f'(x) = 0$ or $iRoC_f(x_0) = f'(x)$ does not exist.

Then $x_0$ is called a \textbf{critical number}.


\end{definition}

\textbf{Note: } Domain numbers where $iRoC_f$ or $f'$ doesn't exist are also places where a function's behavior can change drastically.

\textbf{Note: } Singularities are not in the domain. However, a function's behavior can also change across singularities. (Quadratics don't have singularities.)













\begin{center}
\textbf{\textcolor{green!50!black}{ooooo-=-=-=-ooOoo-=-=-=-ooooo}} \\

more examples can be found by following this link\\ \link[More Examples of the Derivative]{https://ximera.osu.edu/csccmathematics/precalculus/precalculus/theDerivative/examples/exampleList}

\end{center}





\end{document}




