\documentclass{ximera}


\graphicspath{
  {./}
  {ximeraTutorial/}
  {basicPhilosophy/}
}

\newcommand{\mooculus}{\textsf{\textbf{MOOC}\textnormal{\textsf{ULUS}}}}


\usepackage{tkz-euclide}\usepackage{tikz}
\usepackage{tikz-cd}
\usetikzlibrary{arrows}
\tikzset{>=stealth,commutative diagrams/.cd,
  arrow style=tikz,diagrams={>=stealth}} %% cool arrow head
\tikzset{shorten <>/.style={ shorten >=#1, shorten <=#1 } } %% allows shorter vectors

\usetikzlibrary{backgrounds} %% for boxes around graphs
\usetikzlibrary{shapes,positioning}  %% Clouds and stars
\usetikzlibrary{matrix} %% for matrix
\usepgfplotslibrary{polar} %% for polar plots
\usepgfplotslibrary{fillbetween} %% to shade area between curves in TikZ
\usetkzobj{all}
\usepackage[makeroom]{cancel} %% for strike outs
%\usepackage{mathtools} %% for pretty underbrace % Breaks Ximera
%\usepackage{multicol}
\usepackage{pgffor} %% required for integral for loops



%% http://tex.stackexchange.com/questions/66490/drawing-a-tikz-arc-specifying-the-center
%% Draws beach ball
\tikzset{pics/carc/.style args={#1:#2:#3}{code={\draw[pic actions] (#1:#3) arc(#1:#2:#3);}}}



\usepackage{array}
\setlength{\extrarowheight}{+.1cm}
\newdimen\digitwidth
\settowidth\digitwidth{9}
\def\divrule#1#2{
\noalign{\moveright#1\digitwidth
\vbox{\hrule width#2\digitwidth}}}
























%%This is to help with formatting on future title pages.
\newenvironment{sectionOutcomes}{}{}


\title{From the Sides}

\begin{document}

\begin{abstract}
curve pretending
\end{abstract}
\maketitle
 
 



\subsection*{Extending Beyond Quadratics} 


Tangent lines are lines that are tangent to a curve or graph at a tangent point. There must be a point on the curve. At that tangent point, the curve or graph is behaving in some manner, which the tangent line is modeling.  

The tangent line models the curve only for a short distance, because the curve probably pulls away from the line as you get further from the tangent point.

\begin{idea} \textbf{\textcolor{blue!55!black}{Tangent Line}}


A tangent line is a line that does the best job of pretending to be the curve at a single point.
\end{idea}


\textbf{Best job} means the line shares the tangent point with the curve and the slope of the line is the same as the ``slope'' of the curve at the tangent point. If you zoomed in on the graph at the tangent point, then the curve would slowly begin to look just like the line. Thee canonly be one tangent line.

This is a graphical concept and we are hesitant to use graphs for analysis, because they are inherently inaccurate. So, we will be diligently working to translate this graphical idea over to algebraic language. 


For quadratics, we have a formula for the $iRoC$ or $f'$ (the derivative of $f$), which gives the exact slope of each tangent line along the parabola.  But, for a random function, how do you get a tangent line for its graph without knowing the slope ahead of time? 


To do this, we are returning to our idea of a secant line and sliding it over to a tangent point. 








\textbf{\textcolor{blue!55!black}{$\blacktriangleright$ desmos graph}} 
\begin{center}
\desmos{tdkxeewwcj}{400}{300}
\end{center}


The idea is that we can select two points on the graph and a secant line will run through them. Since we have two points, we can calculate the slope of the secant line. We can keep selecting two points that get closer to the tangent point. The slopes of the secant lines shoulsd be getting closer to the slope of the tangent line. 




\textbf{\textcolor{red!90!darkgray}{$\blacktriangleright$}} There are two behaviors happening as we shift a secant line over to the tangent point. The graph has its own behavior and the moving secant lines have another behavior. 

The graph is behaving (there is a pattern) and the secant lines are behaving (there is a pattern). Hopefully, these seprate behaviors will converge to the same behavior at the tangent point.



\begin{itemize}
\item \textbf{First}, the points on the curve or graph itself are approaching the tangent point, from \textbf{both sides}. The graph is connecting up to the point. There is no break. The function is continuous at the corresponding domain number.\\
\item \textbf{Second}, the secant lines are smoothly turning into the tangent line at the tangent point. 
\end{itemize}



We want to keep track of these two behaviors and see if they agree at the tangent line. 


\begin{warning} \textbf{\textcolor{red!80!black}{Both Sides}}

Both sides means both sides of the tangent point. 

The graph points on both sides are approaching the tangent point. 

The secant lines on both sides are approaching the tangent line




\end{warning}
\textbf{Later}, we will also consider the situation where the graph only has one side. Right now we are investigating situations which have both sides.







\textbf{\textcolor{red!70!black}{$\blacktriangleright$ Both Sides}}

It will help our investigation of both sides, if we allow both sides to operate independently. 


We are going to help our calculation a little bit by giving each side its own secant line.  


We are to select two points for our secant lines.  We are going to select one of these points to just be the tangent point. 

We'll just do this twice. 


One on the left of our tangent point.


\textbf{\textcolor{blue!55!black}{$\blacktriangleright$ desmos graph}} 
\begin{center}
\desmos{r0u5uerknf}{400}{300}
\end{center}




And, one on the right side of our tangent point.


\textbf{\textcolor{blue!55!black}{$\blacktriangleright$ desmos graph}} 
\begin{center}
\desmos{qmfjm4iebx}{400}{300}
\end{center}



By watching two secant lines approaching the tangent line, it will be easier to spot some important structure. 






\begin{idea}   \textbf{\textcolor{blue!55!black}{Special Secant Lines}}


To help us watch the moving secant lines, we are going to use special secant lines. 

We are using secant lines that also go through the tangent point. 

We need two points on the graph for a secant. One of the points will be the tangent point.  The second point will be another point on the graph. 

To move the secant line, we will select graph points that are getting ever closer to the tangent point. 

We can do this on either side of the tangent line. 

This will help our eyes.


\end{idea}


\textbf{Note:} We are considering the situation where we have both sides, which means the domain number corresponding to the tangent line is inside an open interval in the domain. 






















\subsection*{Two Sides : Points and Secants Agree}


The setup: 

\begin{itemize}
     \item Let $f(t)$ be a function and $t_0$ a domain number. \\

     \item $t_0$ is inside an open interval in the domain, $t_0 \in (a, b) \subset Domain$. (Because, we need two sides.) \\

     \item The graph point $(t_0 , f(t_0))$ is our tangent point on the graph.
\end{itemize}


Our secant lines are special secant lines. Rather than a secant line running through two points on either side of the tangent point, our secant lines will use the tangent point as one of their points. 



Our first situation is the nicest. The points on the graph approach the tangent point $(t_0 , f(t_0))$ and the secant lines (from both sides) approach the same line, which is the tangent line.



\textbf{\textcolor{blue!55!black}{Example}}  




Here is the graph of the function $f(x) = (x - 3)^2 - 4$ and the point $(4, f(4))= (4, -3)$. 

\begin{image}
\begin{tikzpicture}
     \begin{axis}[
                domain=-10:10, ymax=10, xmax=10, ymin=-6, xmin=-6,
                axis lines =center, xlabel=$x$, ylabel=$y$,
                ytick={-6,-4,-2,2,4,6,8,10},
                xtick={-6,-4,-2,2,4,6,8,10},
                ticklabel style={font=\scriptsize},
                every axis y label/.style={at=(current axis.above origin),anchor=south},
                every axis x label/.style={at=(current axis.right of origin),anchor=west},
                axis on top,
                ]


        \addplot [draw=penColor, very thick, smooth, domain=(0:6),<->] {(x-3)^2 - 4};

        \addplot [color=penColor2,only marks,mark=*] coordinates{(4,-3)};
        


        %\node[penColor] at (axis cs:5,-4) {$(h, k)$};
        %\node[penColor] at (axis cs:5,-9) {$-0.5 x^2 - 5 x + 15.5$};



    \end{axis}
\end{tikzpicture}
\end{image}

A parabola and a point on the parabola. 


\textbf{First}, the points on the graph are approaching the point $(4, -3)$ on both sides.  There is no break in the curve. $f$ is continuous at $4$.


The soon to be tangent point is $(4, -3)$ and a tangent line runs through it are shown together below.







\begin{image}
\begin{tikzpicture}
     \begin{axis}[
                domain=-10:10, ymax=10, xmax=10, ymin=-6, xmin=-6,
                axis lines =center, xlabel=$x$, ylabel=$y$,
                ytick={-6,-4,-2,2,4,6,8,10},
                xtick={-6,-4,-2,2,4,6,8,10},
                ticklabel style={font=\scriptsize},
                every axis y label/.style={at=(current axis.above origin),anchor=south},
                every axis x label/.style={at=(current axis.right of origin),anchor=west},
                axis on top,
                ]


        \addplot [draw=penColor, very thick, smooth, domain=(0:6),<->] {(x-3)^2 - 4};

        \addplot [color=penColor,only marks,mark=*] coordinates{(4,-3)};
        
        \addplot [draw=penColor2, very thick, smooth, domain=(3:8),<->] {2*(x-4)-3};

        %\node[penColor] at (axis cs:5,-4) {$(h, k)$};
        %\node[penColor] at (axis cs:5,-9) {$-0.5 x^2 - 5 x + 15.5$};



    \end{axis}
\end{tikzpicture}
\end{image}



The tangent line does the best job of modeling the curve right at the tangent point. If you were to zoom in closely to the point $(4, -3)$, the line and the curve would look the same.


As it turns out, we know how to get the slope of lines tangent to a parabola.  Use the derivative of $f$ or $iRoC_f$.   

\[ 
f'(x)=iRoC_f(x) = 2 (x - 3)
\] 

In this example, the slope of this tangent line is $f'(3) = 2 (4-3) = 2$ 

Using the point-slope form of a line, we can get an equation for the tangent line, $y - (-3)=2 (x-4)$. That gives us a tangent line described by the equation $y = 2 (x - 4) - 3$.











\textbf{Second}, For every other point on the parabola, there is a line through that point and the tangent point.  These are called \textit{secant lines}. These secant lines run through approaching points and the tangent point. These secant lines smoothly turn into the tangent line at the tangent point...on both sides.


Here are the secant lines through the tangent point and the approaching points $(3, -4)$, $(3.3, -3.91)$, $(3.7, -3.51)$, $(4.3, -2.31)$, $(4.7, -1.11)$, and $(5, 0)$.



\begin{image}
\begin{tikzpicture}
     \begin{axis}[
                domain=-10:10, ymax=10, xmax=10, ymin=-6, xmin=-6,
                axis lines =center, xlabel=$x$, ylabel=$y$,
                ytick={-6,-4,-2,2,4,6,8,10},
                xtick={-6,-4,-2,2,4,6,8,10},
                ticklabel style={font=\scriptsize},
                every axis y label/.style={at=(current axis.above origin),anchor=south},
                every axis x label/.style={at=(current axis.right of origin),anchor=west},
                axis on top,
                ]


        \addplot [draw=penColor, very thick, smooth, domain=(0:6),<->] {(x-3)^2 - 4};

        \addplot [color=penColor,only marks,mark=*] coordinates{(4,-3)};
        
       


        \addplot [draw=penColor4, very thick, smooth, domain=(1:5)] {(x-4)-3};
        \addplot [draw=penColor4, very thick, smooth, domain=(1:5)] {1.3*(x-4)-3};
        \addplot [draw=penColor4, very thick, smooth, domain=(1:5)] {1.7*(x-4)-3};

        \addplot [draw=penColor5, very thick, smooth, domain=(3:8)] {2.3*(x-4)-3};
        \addplot [draw=penColor5, very thick, smooth, domain=(3:8)] {2.7*(x-4)-3};
        \addplot [draw=penColor5, very thick, smooth, domain=(1:8)] {3*(x-4)-3};


          \addplot [draw=penColor2, very thick, smooth, domain=(3:8),<->] {2*(x-4)-3};



    \end{axis}
\end{tikzpicture}
\end{image}





The two secant lines converge to the tangent line, which you can convince yourself in this desmos graph. 





\textbf{\textcolor{blue!55!black}{$\blacktriangleright$ desmos graph}} 
\begin{center}
\desmos{unodj41seg}{400}{300}
\end{center}







The secant lines on the left smoothly turn into the tangent line as you move the left secant point to the right. The secant lines on the right smoothly turn into the tangent line as you move the secand point to the left. 



This is the best situation. 

The graph is nice at the tangent line and the secants lines on both sides smoothly turn into the tange line at the tangent point. 



Now, let's take a look at kinks in the story. 





















\subsection*{Two Sides : Points Agree and Secants Disagree}


Suppose our tangent point is $(t_0 , f(t_0))$ and $t_0$ is inside an open interval in the domain, $t_0 \in (a, b) \subset Domain$. 


Suppose the points on the graph approach the point $(t_0 , f(t_0))$, however the secant lines (from both sides) do not approach the same line.



\textbf{\textcolor{blue!55!black}{Example}}  




Here is the graph of the function $f(x) = | x - 3 | - 4$ and the point $(3, f(3))= (3, -4)$. 

\begin{image}
\begin{tikzpicture}
     \begin{axis}[
                domain=-10:10, ymax=10, xmax=10, ymin=-6, xmin=-6,
                axis lines =center, xlabel=$x$, ylabel=$y$,
                ytick={-6,-4,-2,2,4,6,8,10},
                xtick={-6,-4,-2,2,4,6,8,10},
                ticklabel style={font=\scriptsize},
                every axis y label/.style={at=(current axis.above origin),anchor=south},
                every axis x label/.style={at=(current axis.right of origin),anchor=west},
                axis on top,
                ]

        \addplot [draw=penColor, very thick, smooth, domain=(3:9),->] {x-7};
        \addplot [draw=penColor, very thick, smooth, domain=(-5:3),<-] {-x-1};

        \addplot [color=penColor2,only marks,mark=*] coordinates{(3,-4)};
        


        %\node[penColor] at (axis cs:5,-4) {$(h, k)$};
        %\node[penColor] at (axis cs:5,-9) {$-0.5 x^2 - 5 x + 15.5$};



    \end{axis}
\end{tikzpicture}
\end{image}

In this situation, the graph is an absolute value ``Vee'' and its corner point will be used as the tangent point. 


\textbf{First}, the points on the graph are approaching the corner point on both sides.  There is no break in the curve. $f$ is continuous at $3$.


The hopeful tangent point is $(3, -4)$. 


If there is a tangent line, then the secant lines should smoothly turn into the tangent line, on both sides.  


The other way of thinking of this is that if the secant lines on both sides do not smoothly turn into the same line, then there cannot be a tangent line. 

That is the case here. 








\textbf{Second}, the secant lines through approaching points do not smoothly turn into a common line at the tangent point...\textbf{on both sides}.


Here are the secant lines through the tangent point and the approaching points $(2, -3)$, $(2.3, -3.3)$, $(2.7, -3.7)$, $(3.3, -3.7)$, $(3.7, -3.3)$, and $(4, -3)$.





\begin{image}
\begin{tikzpicture}
     \begin{axis}[
                domain=-10:10, ymax=10, xmax=10, ymin=-6, xmin=-6,
                axis lines =center, xlabel=$x$, ylabel=$y$,
                ytick={-6,-4,-2,2,4,6,8,10},
                xtick={-6,-4,-2,2,4,6,8,10},
                ticklabel style={font=\scriptsize},
                every axis y label/.style={at=(current axis.above origin),anchor=south},
                every axis x label/.style={at=(current axis.right of origin),anchor=west},
                axis on top,
                ]


          \addplot [draw=penColor, very thick, smooth, domain=(3:9),->] {(x-7)};
        \addplot [draw=penColor, very thick, smooth, domain=(-5:3),<-] {(-x-1)};

          \addplot [color=penColor2,only marks,mark=*] coordinates{(3,-4)};
        
          \addplot [draw=penColor4, very thick, smooth, domain=(0:5)] {-1*(x-3)-4};

          \addplot [draw=penColor5, very thick, smooth, domain=(2:6)] {(x-3)-4};



    \end{axis}
\end{tikzpicture}
\end{image}








\textbf{\textcolor{blue!55!black}{$\blacktriangleright$ desmos graph}} 
\begin{center}
\desmos{sjnzjidxwf}{400}{300}
\end{center}











The secant lines on the left smootly turn into a line as you move to the right. The secant lines on the right smoothly turn into a line as you move to the left. 


\begin{center}

\textbf{\textcolor{blue!55!black}{BUT!  They DO NOT converge to the same line!}}

\end{center}





There is no common line that the secant lines are approaching, on both sides. 

There is no tangent line at the pont $(3,-4)$. 

There is no slope of a tangent line, since there is no tangent line. 

$f$ has no derivative value at $3$.  

$f'(3)$ does not exist. 

$3$ is a number in the domain, but $f'(3)$ does not exist. 

($3$ is a critical number.)

































\subsection*{Two Sides : Points Agree and Secants Agree, but No Slope}


In this situation, our tangent point is $(t_0 , f(t_0))$ and $t_0$ is inside an open interval in the domain, $t_0 \in (a, b) \subset Domain$. 


The points on the graph approach the point $(t_0 , f(t_0))$, and the secant lines (from both sides) approach the same line, but it is a vertical line. 



\textbf{\textcolor{blue!55!black}{Example}}  




Here is the graph of the function $f(x) = 4 \, \sqrt[3]{x-2}$ and the point $(2, f(2))= (2, 0)$. 


A cube root and a point on the graph. 




\begin{image}
\begin{tikzpicture}
     \begin{axis}[
                domain=-10:10, ymax=10, xmax=10, ymin=-10, xmin=-10,
                axis lines =center, xlabel=$x$, ylabel=$y$,
                ytick={-10,-8,-6,-4,-2,2,4,6,8,10},
                xtick={-10,-8,-6,-4,-2,2,4,6,8,10},
                ticklabel style={font=\scriptsize},
                every axis y label/.style={at=(current axis.above origin),anchor=south},
                every axis x label/.style={at=(current axis.right of origin),anchor=west},
                axis on top,
                ]


        \addplot [draw=penColor, very thick, smooth, samples=300, domain=(2:9),->] {4*(x-2)^(0.3333)};
        \addplot [draw=penColor, very thick, smooth, samples=300, domain=(-5:2),<-] {-4*(2-x)^(0.3333)};

        \addplot [color=penColor2,only marks,mark=*] coordinates{(2,0)};
        




    \end{axis}
\end{tikzpicture}
\end{image}














\textbf{First}, the points on the graph are approaching the point on both sides.  There is no break in the curve. $f$ is continuous at $3$.


The hopeful tangent point is $(2, 0)$ 

There is a tangent line.




\begin{image}
\begin{tikzpicture}
     \begin{axis}[
                domain=-10:10, ymax=10, xmax=10, ymin=-10, xmin=-10,
                axis lines =center, xlabel=$x$, ylabel=$y$,
                ytick={-10,-8,-6,-4,-2,2,4,6,8,10},
                xtick={-10,-8,-6,-4,-2,2,4,6,8,10},
                ticklabel style={font=\scriptsize},
                every axis y label/.style={at=(current axis.above origin),anchor=south},
                every axis x label/.style={at=(current axis.right of origin),anchor=west},
                axis on top,
                ]


        \addplot [draw=penColor, very thick, smooth, samples=300, domain=(2:9),->] {4*(x-2)^(0.3333)};
        \addplot [draw=penColor, very thick, smooth, samples=300, domain=(-5:2),<-] {-4*(2-x)^(0.3333)};

        \addplot [color=penColor2,only marks,mark=*] coordinates{(2,0)};
        

          \addplot [line width=1, penColor2, smooth,samples=200,domain=(-6:6)] ({2},{x});


    \end{axis}
\end{tikzpicture}
\end{image}












The tangent line does the best job of modeling the curve right at the tangent point. 

Here, the tangent line is described by the equation $x = 2$. It is a vertical line. 











\textbf{Second}, the secant lines running through approaching points smoothly turn into the tangent line at the tangent point...\textbf{on both sides}.


Here are the secant lines through the tangent point and the approaching points $(1, -4)$, $(1.3, -3.91)$, $(1.7, -3.51)$, $(2.3, -2.31)$, $(2.7, -1.11)$, and $(3, 0)$.



\begin{image}
\begin{tikzpicture}
     \begin{axis}[
                domain=-10:10, ymax=10, xmax=10, ymin=-10, xmin=-10,
                axis lines =center, xlabel=$x$, ylabel=$y$,
                ytick={-10,-8,-6,-4,-2,2,4,6,8,10},
                xtick={-10,-8,-6,-4,-2,2,4,6,8,10},
                ticklabel style={font=\scriptsize},
                every axis y label/.style={at=(current axis.above origin),anchor=south},
                every axis x label/.style={at=(current axis.right of origin),anchor=west},
                axis on top,
                ]



         \addplot [draw=penColor, very thick, smooth, samples=300, domain=(2:9),->] {4*(x-2)^(0.3333)};
        \addplot [draw=penColor, very thick, smooth, samples=300, domain=(-5:2),<-] {-4*(2-x)^(0.3333)};

        \addplot [color=penColor2,only marks,mark=*] coordinates{(2,0)};
        

          \addplot [line width=1, penColor2, smooth,samples=200,domain=(-6:6)] ({2},{x});
        
       


        \addplot [draw=penColor4, very thick, smooth, domain=(-2:4)] {2.5*(x-2)};
        \addplot [draw=penColor4, very thick, smooth, domain=(-2:4)] {4*(x-2)};
        \addplot [draw=penColor4, very thick, smooth, domain=(-2:4)] {5*(x-2)};

        \addplot [draw=penColor5, very thick, smooth, domain=(1:5)] {5*(x-2)};
        \addplot [draw=penColor5, very thick, smooth, domain=(1:5)] {4*(x-2)};
        \addplot [draw=penColor5, very thick, smooth, domain=(1:5)] {2.5*(x-2)};





    \end{axis}
\end{tikzpicture}
\end{image}






You can convince yourself in this desmo graph. 

\textbf{\textcolor{blue!55!black}{$\blacktriangleright$ desmos graph}} 
\begin{center}
\desmos{oakpkmuxfo}{400}{300}
\end{center}







The secant lines on the left smoothly turn into the tangent line as you move the secant point to the right. The secant lines on the right smoothly turn into the tangent line as you move the secant point to the left. 



However, the tangent line is a vertical line.  It has no slope.


There is no slope of a tangent line. 

$f$ has no derivative value at $2$.  

$f'(2)$ does not exist. 

$2$ is a number in the domain, but $f'(2)$ does not exist. 

($2$ is a critical number.)



























\subsection*{Two Sides : Points Disagree and Secants Disagree}





A function with a discontiuity has a break in the graph, which means the points on the graph do not approach the same point.  This automatically results in the derivative not existing. 








\begin{image}
\begin{tikzpicture}
     \begin{axis}[
                domain=-10:10, ymax=10, xmax=10, ymin=-6, xmin=-6,
                axis lines =center, xlabel=$x$, ylabel=$y$,
                ytick={-6,-4,-2,2,4,6,8,10},
                xtick={-6,-4,-2,2,4,6,8,10},
                ticklabel style={font=\scriptsize},
                every axis y label/.style={at=(current axis.above origin),anchor=south},
                every axis x label/.style={at=(current axis.right of origin),anchor=west},
                axis on top,
                ]


        \addplot [draw=penColor, very thick, smooth, domain=(0:4),<-] {(x-3)^2 - 4};
        \addplot [draw=penColor, very thick, smooth, domain=(4:6),->] {(x-3)^2 + 1};

        \addplot [color=penColor,only marks,mark=*] coordinates{(4,-3)};
        \addplot [color=penColor,fill=white,only marks,mark=*] coordinates{(4,2)};
        
       





    \end{axis}
\end{tikzpicture}
\end{image}











We can also follow the approaching secant lines.  Remember, the secant lines all go through the prospective tangent point and another point on the graph at.  This throws the secant lines way off. 




\begin{image}
\begin{tikzpicture}
     \begin{axis}[
                domain=-10:10, ymax=10, xmax=10, ymin=-6, xmin=-6,
                axis lines =center, xlabel=$x$, ylabel=$y$,
                ytick={-6,-4,-2,2,4,6,8,10},
                xtick={-6,-4,-2,2,4,6,8,10},
                ticklabel style={font=\scriptsize},
                every axis y label/.style={at=(current axis.above origin),anchor=south},
                every axis x label/.style={at=(current axis.right of origin),anchor=west},
                axis on top,
                ]


        \addplot [draw=penColor, very thick, smooth, domain=(0:4),<-] {(x-3)^2 - 4};
        \addplot [draw=penColor, very thick, smooth, domain=(4:6),->] {(x-3)^2 + 1};

        \addplot [color=penColor,only marks,mark=*] coordinates{(4,-3)};
        \addplot [color=penColor,fill=white,only marks,mark=*] coordinates{(4,2)};
        
       


        \addplot [draw=penColor4, very thick, smooth, domain=(3:8)] {19*(x-4)-3};
        \addplot [draw=penColor4, very thick, smooth, domain=(3:8)] {9.8*(x-4)-3};
        \addplot [draw=penColor4, very thick, smooth, domain=(3:8)] {8*(x-4)-3};




        \addplot [draw=penColor5, very thick, smooth, domain=(1:5)] {-4};
        \addplot [draw=penColor5, very thick, smooth, domain=(1:5)] {0.6*(x-3.3)-3.82};
        \addplot [draw=penColor5, very thick, smooth, domain=(1:5)] {1.2*(x-3.6)-3.64};


       



    \end{axis}
\end{tikzpicture}
\end{image}






You can experiment for yourself in this desmo graph. 

\textbf{\textcolor{blue!55!black}{$\blacktriangleright$ desmos graph}} 
\begin{center}
\desmos{90rjh21df1}{400}{300}
\end{center}






The secant lines on the two sides are approaching their own lines, but they are not smoothly turning to agree on a common line. 


In this case, we say that there is no tangent line. 

And, if there is no tangent line, then there is no slope of the tangent line, then the derivative has no value here. 







So, a derivative implies that there are two sides and the two sides are agreeing. 



But, there is no need to just throw everything else away.  We can extend our idea of derivative to include just one side. 






\subsection*{Summary}


A tangent point is a point on the graph, which means it corresponds to a domain number for the function. 


\textbf{\textcolor{blue!55!black}{$\blacktriangleright$}}  The function could be continuous at this domain number. The graph could have a tangent line at this point and the secant lines on both sides might converge to this tangent line.  In this case, we have a value of the derivative at this domain number. This is our best scenario. 

\begin{center}
\textbf{\textcolor{blue!55!black}{The domain number is a critical number.}}
\end{center}


\textbf{\textcolor{blue!55!black}{$\blacktriangleright$}}  The function could be continuous at this domain number.  However, the graph might not have a tangent line at this point because the secant lines on BOTH sides might not BOTH converge to this tangent line.  In this case, we do not have a value of the derivative at this domain number. 

\begin{center}
\textbf{\textcolor{blue!55!black}{The domain number is a critical number.}}
\end{center}

\textbf{\textcolor{blue!55!black}{$\blacktriangleright$}}  The function could be continuous at this domain number. The graph could have a tangent line at this point and the secant lines on both sides might converge to this tangent line.  However, the tangent line is vertical, so it doesn't have a slope. In this case, we do not have a value of the derivative at this domain number. 

\begin{center}
\textbf{\textcolor{blue!55!black}{The domain number is a critical number.}}
\end{center}

\textbf{\textcolor{blue!55!black}{$\blacktriangleright$}}  This domain number might be a discontinuity of the function. In this situation, the secant lines on both sides cannot both converge to the same line.  There is no tangent line, which means there is no slope for the tangent line, which means there is no value for the derivative for this domain number. 

\begin{center}
\textbf{\textcolor{blue!55!black}{The domain number is a critical number.}}
\end{center}













































\subsection*{Extending Beyond Quadratics} 


 
 

Tangent lines are lines that are tangent to a curve or graph at a tangent point. There must be a point on the curve. At that tangent point, the curve or graph is behaving in some manner, which the tangent line is modeling.  

\begin{idea} \textbf{\textcolor{blue!55!black}{Tangent}}

A tangent line is a line that does the best job of pretending to be the curve at a single point.
\end{idea}
\textbf{Best job} means the line shares the tangent point with the curve and the slope of the line is the same as the ``slope'' of the curve at the tangent point. 





This is possible even at an endpoint on a graph.





\subsection*{Endpoints}



In this case, we have a restricted domain for our function, which gives us a graph with an endpoint. 




\begin{image}
\begin{tikzpicture}
     \begin{axis}[
                domain=-10:10, ymax=10, xmax=10, ymin=-6, xmin=-6,
                axis lines =center, xlabel=$x$, ylabel=$y$,
                ytick={-6,-4,-2,2,4,6,8,10},
                xtick={-6,-4,-2,2,4,6,8,10},
                ticklabel style={font=\scriptsize},
                every axis y label/.style={at=(current axis.above origin),anchor=south},
                every axis x label/.style={at=(current axis.right of origin),anchor=west},
                axis on top,
                ]


        \addplot [draw=penColor, very thick, smooth, domain=(0:4),<-] {(x-3)^2 - 4};


        \addplot [color=penColor,only marks,mark=*] coordinates{(4,-3)};


          %\addplot [draw=penColor2, very thick, smooth, domain=(3:8),<->] {2*(x-4)-3};




    \end{axis}
\end{tikzpicture}
\end{image}


This point $(4, -3)$ is an endpoint on the graph.

However, there still might be a line that models one side of the graph. 

A tangent line, well, a one-sided tangent line. 










\begin{image}
\begin{tikzpicture}
     \begin{axis}[
                domain=-10:10, ymax=10, xmax=10, ymin=-6, xmin=-6,
                axis lines =center, xlabel=$x$, ylabel=$y$,
                ytick={-6,-4,-2,2,4,6,8,10},
                xtick={-6,-4,-2,2,4,6,8,10},
                ticklabel style={font=\scriptsize},
                every axis y label/.style={at=(current axis.above origin),anchor=south},
                every axis x label/.style={at=(current axis.right of origin),anchor=west},
                axis on top,
                ]


        \addplot [draw=penColor, very thick, smooth, domain=(0:4),<-] {(x-3)^2 - 4};


        \addplot [color=penColor,only marks,mark=*] coordinates{(4,-3)};


          \addplot [draw=penColor2, very thick, smooth, domain=(3:8),<->] {2*(x-4)-3};




    \end{axis}
\end{tikzpicture}
\end{image}


We have a tangent line.  It is a one-sided tangent line. 




Secant lines on the left approach the tangent line.









\begin{image}
\begin{tikzpicture}
     \begin{axis}[
                domain=-10:10, ymax=10, xmax=10, ymin=-6, xmin=-6,
                axis lines =center, xlabel=$x$, ylabel=$y$,
                ytick={-6,-4,-2,2,4,6,8,10},
                xtick={-6,-4,-2,2,4,6,8,10},
                ticklabel style={font=\scriptsize},
                every axis y label/.style={at=(current axis.above origin),anchor=south},
                every axis x label/.style={at=(current axis.right of origin),anchor=west},
                axis on top,
                ]


        \addplot [draw=penColor, very thick, smooth, domain=(0:4),<-] {(x-3)^2 - 4};

        \addplot [color=penColor,only marks,mark=*] coordinates{(4,-3)};
        
       


        \addplot [draw=penColor4, very thick, smooth, domain=(1:5)] {(x-4)-3};
        \addplot [draw=penColor4, very thick, smooth, domain=(1:5)] {1.3*(x-4)-3};
        \addplot [draw=penColor4, very thick, smooth, domain=(1:5)] {1.7*(x-4)-3};



          \addplot [draw=penColor2, very thick, smooth, domain=(3:8),<->] {2*(x-4)-3};



        %\node[penColor] at (axis cs:5,-4) {$(h, k)$};
        %\node[penColor] at (axis cs:5,-9) {$-0.5 x^2 - 5 x + 15.5$};



    \end{axis}
\end{tikzpicture}
\end{image}







\textbf{\textcolor{blue!55!black}{$\blacktriangleright$ desmos graph}} 
\begin{center}
\desmos{12gvt2psre}{400}{300}
\end{center}







The secant lines on the left smoothly turn into the tangent line as you move to the right. 


There are no secant lines on the right, so we don't have to worry about them. 




In this case, we have a one-side tangent line.  

We have a one-side derivative. 


For the function above, we have a \textbf{left derivative}.












Similarly, we might have a one-side tangent line on the right.  

We would have a one-side derivative. 


We would have a \textbf{right derivative}.






\subsection*{The Derivative (Two-Sided)}





\begin{definition} \textbf{\textcolor{green!50!black}{the Derivative Function}}  (two-sided)


Suppose we have a really nice situation.


\begin{itemize}
\item We have a function, $f$, 
\item We have a domain number, $t_0$, 
\item This domain number is inside an open interval inside the domain.  $t_0 \in (a, b) \subset Domain$. 
\item The secant lines on both sides of $(t_0, f(t_0))$ are smoothly approaching the same line from both sides.
\end{itemize}

Then, there is a tangent line to the graph of $f$ and the secant lines on both sides are smoothly turning into this tangent line at the tangent point $(t_0, f(t_0))$. 

The slope of this tangent line is the value of the \textbf{derivative} of $f$ at $t_0$.

\[
iRoC(t_0) =f'(t_0) = \text{slope of the corresponding tangent line}
\]


\begin{itemize}
  \item If the tangent line doesn't have a slope, then the derivative has no value there. The derivative does not exist at $t_0$.

  \item If the secant lines on both sides are not smoothly agreeing, then there is no tangent line at $(t_0, f(t_0))$ and the derivative does not exist at $t_0$.   
\end{itemize}


By collecting all of the domain nmumbers where the slope of the corresponding tangent line exists, we can create a new function called \textbf{the derivative of $f$}.

This function pairs up domain numbers with the slopes of tangent lines.

It is our tool for measuring instantaneous rate of change at a domain number.




\end{definition}

\textbf{Note}: The term ``derivative" implies a two-sided derivative.  Otherwise, we would say `right' or `left'. 




\begin{warning}

Saying the word ``derivative'' implies two sides. 

For the derivative to exist at a domain number, the corresponding point on the graph must have secants approaching on both sides. 


This means that end-point numbers for domain intervals are automatically critical numbers.  The derivative cannot exist at end-points, even if there is a one-sided tangent line, because ``derivative'' means two sides. 

\end{warning}




\subsection*{The Left or Right Derivative (One-Sided)}


So, we have two separate situations. 


In the two-sided situaiton, we have a domain number inside an open interval in the domain, so that there are two sides. Since, it is a domain number, we know there is a point on the graph of the function and the graph is on both sides.  We can examine if there is a tangent line or not and if it has a slope or not. 

A different situation involves a domain number that is the end of an interval in the domain.  The domain number only has one side in the domain.  This automatically disqualifies the function from having a derivative value at this domain number.  The domain number is a critical number.  However, this is still a nice situation, which we would like to describe. 

We would still like to describe one-sided situations. 

There can still be a tangent line, a one-sided tangent line. That tangent line can still have a slope. 

So, we also have a one-sided derivative, which is like half of a derivative. 











\begin{definition} (from earlier) \textbf{\textcolor{green!50!black}{the Left Derivative}}  (one-sided) 


Suppose we have a really nice one-sided situation.


\begin{itemize}
\item We have a function, $f$, 
\item We have and a domain number, $t_0$, 
\item This domain number is inside an interval (half open) inside the domain of the form  $t_0 \in (a, t_0] \subset Domain$. 
\item The secant lines on the left side of $(t_0, f(t_0))$ are smootlhy approaching the same line.
\end{itemize}

Then, there is a tangent line to the graph of $f$ and the secant lines on the left side are smoothly turning into this tangent line at the tangent point $(t_0, f(t_0))$. 

The slope of this tangent line is the value of the \textbf{left derivative of $f$ at $t_0$}. 

We have a function called the left derivative of $f$ or the left $iRoC_f$.

\[
iRoC_{f_{-}}(t) =f'_{-}(t) = \text{slope of tangent line}
\]

The left derivative is symbolized with a subscript of ``-'' with the derivative name. 
 

\end{definition}






\begin{definition} (from earlier) \textbf{\textcolor{green!50!black}{Right Derivative}}  (one-sided) 


Suppose we have a really nice one-sided situation.


\begin{itemize}
\item We have a function, $f$, 
\item We have and a domain number, $t_0$, 
\item This domain number is inside an interval (half open) inside the domain of the form  $t \in [t_0, b) \subset Domain$. 
\item The secant lines on the right side of $(t_0, f(t_0))$ are smoothly approaching the same line.
\end{itemize}

Then, there is a tangent line to the graph of $f$ and the secant lines on the right side are smoothly turning into this tangent line at the tangent point $(t_0, f(t_0))$. 

The slope of this tangent line is the value of the \textbf{right derivative of $f$ at $t_0$}.


We have a function called the right derivative of $f$ or the right $iRoC_f$. 

\[
iRoC_{f_{+}}(t) =f'_{+}(t) = \text{slope of tangent line}
\]


The right derivative is symbolized with a subscript of ``+'' with the derivative name. \

\end{definition}



\textbf{Note:} It is possible that we have a domain number where there is both a left and right derivative value and they are different values.  In this case, there would be no value of the derivative.  If there is a derivative value, then the left and right sides must match.  We saw this with an absolute value function.





\subsection*{No Side}


If the domain number is isolated, meaning no domain numbers immediately to the left or right, then there just is no derivative there. 

The term derivative implies that there is some sort of approaching.






















\begin{center}
\textbf{\textcolor{green!50!black}{ooooo-=-=-=-ooOoo-=-=-=-ooooo}} \\

more examples can be found by following this link\\ \link[More Examples of the Derivative]{https://ximera.osu.edu/csccmathematics/precalculus/precalculus/theDerivative/examples/exampleList}

\end{center}






\end{document}




