\documentclass{ximera}


\graphicspath{
  {./}
  {ximeraTutorial/}
  {basicPhilosophy/}
}

\newcommand{\mooculus}{\textsf{\textbf{MOOC}\textnormal{\textsf{ULUS}}}}


\usepackage{tkz-euclide}\usepackage{tikz}
\usepackage{tikz-cd}
\usetikzlibrary{arrows}
\tikzset{>=stealth,commutative diagrams/.cd,
  arrow style=tikz,diagrams={>=stealth}} %% cool arrow head
\tikzset{shorten <>/.style={ shorten >=#1, shorten <=#1 } } %% allows shorter vectors

\usetikzlibrary{backgrounds} %% for boxes around graphs
\usetikzlibrary{shapes,positioning}  %% Clouds and stars
\usetikzlibrary{matrix} %% for matrix
\usepgfplotslibrary{polar} %% for polar plots
\usepgfplotslibrary{fillbetween} %% to shade area between curves in TikZ
\usetkzobj{all}
\usepackage[makeroom]{cancel} %% for strike outs
%\usepackage{mathtools} %% for pretty underbrace % Breaks Ximera
%\usepackage{multicol}
\usepackage{pgffor} %% required for integral for loops



%% http://tex.stackexchange.com/questions/66490/drawing-a-tikz-arc-specifying-the-center
%% Draws beach ball
\tikzset{pics/carc/.style args={#1:#2:#3}{code={\draw[pic actions] (#1:#3) arc(#1:#2:#3);}}}



\usepackage{array}
\setlength{\extrarowheight}{+.1cm}
\newdimen\digitwidth
\settowidth\digitwidth{9}
\def\divrule#1#2{
\noalign{\moveright#1\digitwidth
\vbox{\hrule width#2\digitwidth}}}
























%%This is to help with formatting on future title pages.
\newenvironment{sectionOutcomes}{}{}


\title{Function of Change}

\begin{document}

\begin{abstract}
rate as a relationship
\end{abstract}
\maketitle






\section*{Rate of Change Over an Interval}





Consider the function $Q(x) = \frac{(x+7)(x-5)}{5}$








\begin{image}
\begin{tikzpicture}
  \begin{axis}[
            domain=-10:10, ymax=10, xmax=10, ymin=-10, xmin=-10,
            axis lines =center, xlabel=$x$, ylabel={$y=Q(x)$}, grid = major,
            ytick={-10,-8,-6,-4,-2,2,4,6,8,10},
            xtick={-10,-8,-6,-4,-2,2,4,6,8,10},
            yticklabels={$-10$,$-8$,$-6$,$-4$,$-2$,$2$,$4$,$6$,$8$,$10$}, 
            xticklabels={$-10$,$-8$,$-6$,$-4$,$-2$,$2$,$4$,$6$,$8$,$10$},
            ticklabel style={font=\scriptsize},
            every axis y label/.style={at=(current axis.above origin),anchor=south},
            every axis x label/.style={at=(current axis.right of origin),anchor=west},
            axis on top
          ]
          
          %\addplot [line width=2, penColor2, smooth,samples=100,domain=(-6:2)] {-2*x-3};
            \addplot [line width=2, penColor, smooth,samples=100,domain=(-9:8),<->] {.2*(x+7)*(x-5)};

          %\addplot[color=penColor,fill=penColor2,only marks,mark=*] coordinates{(-6,9)};
          %\addplot[color=penColor,fill=penColor2,only marks,mark=*] coordinates{(2,-7)};



           

  \end{axis}
\end{tikzpicture}
\end{image}




The rate-of-change over the interval $[-3, 4]$ is 
\[  \frac{Q(4)-Q(-3)}{4-(-3)} = \frac{4.2}{7} = 0.6      \]


This is the slope of the secant line through the points $(-3, Q(-3))$ and $(4, Q(4))$.











\begin{image}
\begin{tikzpicture}
  \begin{axis}[
            domain=-10:10, ymax=10, xmax=10, ymin=-10, xmin=-10,
            axis lines =center, xlabel=$x$, ylabel={$y=Q(x)$}, grid = major,
            ytick={-10,-8,-6,-4,-2,2,4,6,8,10},
            xtick={-10,-8,-6,-4,-2,2,4,6,8,10},
            yticklabels={$-10$,$-8$,$-6$,$-4$,$-2$,$2$,$4$,$6$,$8$,$10$}, 
            xticklabels={$-10$,$-8$,$-6$,$-4$,$-2$,$2$,$4$,$6$,$8$,$10$},
            ticklabel style={font=\scriptsize},
            every axis y label/.style={at=(current axis.above origin),anchor=south},
            every axis x label/.style={at=(current axis.right of origin),anchor=west},
            axis on top
          ]
          
            \addplot [line width=2, penColor2, smooth,samples=100,domain=(-6:8),<->] {0.6*(x+3)-6.4};
            \addplot [line width=2, penColor, smooth,samples=100,domain=(-9:8),<->] {.2*(x+7)*(x-5)};

          %\addplot[color=penColor,fill=penColor2,only marks,mark=*] coordinates{(-6,9)};
          %\addplot[color=penColor,fill=penColor2,only marks,mark=*] coordinates{(2,-7)};



           

  \end{axis}
\end{tikzpicture}
\end{image}




This rate-of-change is the \textbf{constant} rate-of-change that would be needed in order for a linear funciton to match the endpoint values of the function.  Of course, we can see from the graph that this rate-of-change varies depending on the interval selected. 

$[-3, 4]$ would not be a good interval to use if you were interested in the rate-of-change near $-3$.

$[-3, -2.5]$ would probably give a better interval. 






The rate-of-change over the interval $[-3, -2.5]$ is 
\[  \frac{Q(-2.5)-Q(-3)}{-2.5-(-3)} = \frac{-0.35}{0.5} = -0.7      \]


This is the slope of the secant line through the points $(-3, Q(-3))$ and $(-2.5, Q(-2.5))$.











\begin{image}
\begin{tikzpicture}
  \begin{axis}[
            domain=-10:10, ymax=10, xmax=10, ymin=-10, xmin=-10,
            axis lines =center, xlabel=$x$, ylabel={$y=Q(x)$}, grid = major,
            ytick={-10,-8,-6,-4,-2,2,4,6,8,10},
            xtick={-10,-8,-6,-4,-2,2,4,6,8,10},
            yticklabels={$-10$,$-8$,$-6$,$-4$,$-2$,$2$,$4$,$6$,$8$,$10$}, 
            xticklabels={$-10$,$-8$,$-6$,$-4$,$-2$,$2$,$4$,$6$,$8$,$10$},
            ticklabel style={font=\scriptsize},
            every axis y label/.style={at=(current axis.above origin),anchor=south},
            every axis x label/.style={at=(current axis.right of origin),anchor=west},
            axis on top
          ]
          
            \addplot [line width=2, penColor2, smooth,samples=100,domain=(-6:8),<->] {-0.7*(x+3)-6.4};
            \addplot [line width=2, penColor, smooth,samples=100,domain=(-9:8),<->] {.2*(x+7)*(x-5)};

          %\addplot[color=penColor,fill=penColor2,only marks,mark=*] coordinates{(-6,9)};
          %\addplot[color=penColor,fill=penColor2,only marks,mark=*] coordinates{(2,-7)};



           

  \end{axis}
\end{tikzpicture}
\end{image}




That is looking more like a tangent line, than a secant.





\begin{idea}
\textbf{\textcolor{green!50!black}{Tangent Line}}

A tangent line to a graph at a point, is first a line. Second it intersects the graph at a given point, called the \textbf{point of tangency} or the \textbf{tangent point}. Third, it does the best job of pretending to be the graph \textbf{\textcolor{red!70!darkgray}{at that point}}. \\




The tangent point below is $\left( -3, -\frac{32}{5} \right)$. If you zoom in on the intersection point (a lot), then the parabola will slowly appear to become a line. The tangent line is the line the graph is becoming. 






\begin{center}
\desmos{rufcna7r3b}{400}{300}
\end{center}



It is like the tangent line and the graph have the same slope at the tangent point.

\end{idea}










\section*{Rate of Change Over a Single Number?}





If we make a really small interval at $-3$, then the slope of the secant line should give a good approximation of the tangent line.  The slope of the secant line should approximate the slope of the tangent line.




The rate-of-change over the interval $[-3, -3+h]$ is 
\[  \frac{Q(-3+h)-Q(-3)}{(-3+h)-(-3)} = \frac{Q(-3+h)-Q(-3)}{h} =  \frac{\frac{(-3+h+7)(-3+h-5)}{5}+64}{h}   \]


This is the slope of the secant line through the points $(-3, Q(-3))$ and $(-3+h, Q(-3+h))$, which is very near the tangent line at $(-3, Q(-3))$, when $h$ is small.



By moving the value of $h$ close to $0$, we can get a good estimate of the slope of the tangent line.









\begin{center}
\desmos{di7qdhdfnn}{400}{300}
\end{center}




The slope of the line tangent to the graph at $(-3, -6.4)$ looks to be about $-0.8$.

Of course $-3$ was nothing special.  We can get slopes of tangent lines at any point on the graph. Let's move around the graph and record the tangent line slopes.  We'll record visually.

The slope of the tangent line at $-3$ is $-0.8$, so we'll plot the point $(-3,-0.8)$ to record this information.  This will be a visual encoding of the tangent slope for the domain number $-3$.





\begin{image}
\begin{tikzpicture}
  \begin{axis}[
            domain=-10:10, ymax=10, xmax=10, ymin=-10, xmin=-10,
            axis lines =center, xlabel=$x$, ylabel={$y=Q(x)$}, grid = major,
            ytick={-10,-8,-6,-4,-2,2,4,6,8,10},
            xtick={-10,-8,-6,-4,-2,2,4,6,8,10},
            yticklabels={$-10$,$-8$,$-6$,$-4$,$-2$,$2$,$4$,$6$,$8$,$10$}, 
            xticklabels={$-10$,$-8$,$-6$,$-4$,$-2$,$2$,$4$,$6$,$8$,$10$},
            ticklabel style={font=\scriptsize},
            every axis y label/.style={at=(current axis.above origin),anchor=south},
            every axis x label/.style={at=(current axis.right of origin),anchor=west},
            axis on top
          ]
          
            %\addplot [line width=2, penColor2, smooth,samples=100,domain=(-6:8),<->] {-0.7*(x+3)-6.4};
            \addplot [line width=2, penColor, smooth,samples=100,domain=(-9:8),<->] {.2*(x+7)*(x-5)};

          	\addplot[color=penColor2,fill=penColor2,only marks,mark=*] coordinates{(-3,-0.8)};




           

  \end{axis}
\end{tikzpicture}
\end{image}


We could do this for a bunch of domain numbers.

\begin{enumerate}
\item select a domain number
\item select a very tiny interval beginning at that domain number
\item calculate the rate-of-change over the tiny interval
\item plot a point on the graph.  First coordinate is the domain number.  Second coordinate is the rate-of-change.
\end{enumerate}







\begin{image}
\begin{tikzpicture}
  \begin{axis}[
            domain=-10:10, ymax=10, xmax=10, ymin=-10, xmin=-10,
            axis lines =center, xlabel=$x$, ylabel={$y=Q(x)$}, grid = major,
            ytick={-10,-8,-6,-4,-2,2,4,6,8,10},
            xtick={-10,-8,-6,-4,-2,2,4,6,8,10},
            yticklabels={$-10$,$-8$,$-6$,$-4$,$-2$,$2$,$4$,$6$,$8$,$10$}, 
            xticklabels={$-10$,$-8$,$-6$,$-4$,$-2$,$2$,$4$,$6$,$8$,$10$},
            ticklabel style={font=\scriptsize},
            every axis y label/.style={at=(current axis.above origin),anchor=south},
            every axis x label/.style={at=(current axis.right of origin),anchor=west},
            axis on top
          ]
          
            %\addplot [line width=2, penColor2, smooth,samples=100,domain=(-6:8),<->] {-0.7*(x+3)-6.4};
            \addplot [line width=2, penColor, smooth,samples=100,domain=(-9:8),<->] {.2*(x+7)*(x-5)};

          	\addplot[color=penColor2,fill=penColor2,only marks,mark=*] coordinates{(-5,-1.6)};
          	\addplot[color=penColor2,fill=penColor2,only marks,mark=*] coordinates{(-3,-0.8)};
          	\addplot[color=penColor2,fill=penColor2,only marks,mark=*] coordinates{(-1,0)};
          	\addplot[color=penColor2,fill=penColor2,only marks,mark=*] coordinates{(1,0.8)};
          	\addplot[color=penColor2,fill=penColor2,only marks,mark=*] coordinates{(3,1.6)};
          	\addplot[color=penColor2,fill=penColor2,only marks,mark=*] coordinates{(5,2.4)};




           

  \end{axis}
\end{tikzpicture}
\end{image}






If we do this for every domain number, then we will have created a new function.  Every domain number will be paired with the slope of the tangent line at the corresponding point.

We could graph this ``slope function''.







\begin{center}
\desmos{xv5dnwhbku}{400}{300}
\end{center}




This function, whose values are the rates-of-change of another function, is called the \textbf{derivative} of the other function, because it was derived from the original function.




$\blacktriangleright$ Let $f$ be a function.  Then the derivative of $f$ is denoted as $f'$. People pronounce this as ``\textit{f prime}''. \\



\begin{definition} \textbf{\textcolor{green!50!black}{The Derivative - The Slope Function}} 

Let $f$ be a function.

Then $f'$ represents the \textbf{derivative of $f$}.


$f'(a) = $ the slope of the tangent line at $(a, f(a))$, on the graph of $f$.




\end{definition}



Through the slope of the tangent line, we have invented a way of talking about the \textbf{\textcolor{purple!85!blue}{instantaneous rate of change}} of a function at a number, rather than over an interval. \\









\begin{example}



The function $Q(x) = \frac{(x+7)(x-5)}{5}$ is a quadratic and has a parabola for a graph.  The vertex, $(-1, \tfrac{-36}{5})$, of the parabola has a horizontal tangent line, $y=\tfrac{-36}{5}$.











\begin{image}
\begin{tikzpicture}
  \begin{axis}[
            domain=-10:10, ymax=10, xmax=10, ymin=-10, xmin=-10,
            axis lines =center, xlabel=$x$, ylabel={$y=Q(x)$}, grid = major,
            ytick={-10,-8,-6,-4,-2,2,4,6,8,10},
            xtick={-10,-8,-6,-4,-2,2,4,6,8,10},
            yticklabels={$-10$,$-8$,$-6$,$-4$,$-2$,$2$,$4$,$6$,$8$,$10$}, 
            xticklabels={$-10$,$-8$,$-6$,$-4$,$-2$,$2$,$4$,$6$,$8$,$10$},
            ticklabel style={font=\scriptsize},
            every axis y label/.style={at=(current axis.above origin),anchor=south},
            every axis x label/.style={at=(current axis.right of origin),anchor=west},
            axis on top
          ]
          
            %\addplot [line width=2, penColor2, smooth,samples=100,domain=(-6:8),<->] {-0.7*(x+3)-6.4};
            \addplot [line width=2, penColor, smooth,samples=100,domain=(-9:8),<->] {.2*(x+7)*(x-5)};

            \addplot [line width=2, penColor2, smooth,samples=100,domain=(-9:8),<->] {-7.2};

           

  \end{axis}
\end{tikzpicture}
\end{image}



The slope of this tangent line is $0$.  Therefore, $Q'(-1) = 0$.



\end{example}





















\begin{example} The Derivative





Graph of $y = f(x) = e^{\tfrac{x}{5}}$







\begin{image}
\begin{tikzpicture}
  \begin{axis}[
            domain=-10:10, ymax=10, xmax=10, ymin=-10, xmin=-10,
            axis lines =center, xlabel=$x$, ylabel={$y=f(x)$}, grid = major,
            ytick={-10,-8,-6,-4,-2,2,4,6,8,10},
            xtick={-10,-8,-6,-4,-2,2,4,6,8,10},
            yticklabels={$-10$,$-8$,$-6$,$-4$,$-2$,$2$,$4$,$6$,$8$,$10$}, 
            xticklabels={$-10$,$-8$,$-6$,$-4$,$-2$,$2$,$4$,$6$,$8$,$10$},
            ticklabel style={font=\scriptsize},
            every axis y label/.style={at=(current axis.above origin),anchor=south},
            every axis x label/.style={at=(current axis.right of origin),anchor=west},
            axis on top
          ]
          
            %\addplot [line width=2, penColor2, smooth,samples=100,domain=(-6:8),<->] {-0.7*(x+3)-6.4};
            \addplot [line width=2, penColor, smooth,samples=100,domain=(-9:9),<->] {e^(0.2*x)};


           

  \end{axis}
\end{tikzpicture}
\end{image}




The value of $f'(4)$ is
\begin{multipleChoice}
\choice[correct]{positive}
\choice{negative}
\end{multipleChoice}





\end{example}










\begin{example}



Here is a graph of $y = g(x) = \sin(x)$ and $g'(x)$.






\begin{center}
\desmos{yp5zobfigx}{400}{300}
\end{center}



The derivative of $\sin(x)$ is
\begin{multipleChoice}
\choice[correct]{$\cos(x)$}
\choice{$\sin(x)$}
\choice{$-\cos(x)$}
\choice{$-\sin(x)$}
\end{multipleChoice}


\end{example}




















































Consider the function $k(t) = t \, \sqrt{5-t}$ with its natural or implied domain.

Here is the graph of $y = k(t)$.








\begin{image}
\begin{tikzpicture}
  \begin{axis}[
            domain=-10:10, ymax=10, xmax=10, ymin=-10, xmin=-10,
            axis lines =center, xlabel=$t$, ylabel=$y$, grid = major,
            ytick={-10,-8,-6,-4,-2,2,4,6,8,10},
            xtick={-10,-8,-6,-4,-2,2,4,6,8,10},
            ticklabel style={font=\scriptsize},
            every axis y label/.style={at=(current axis.above origin),anchor=south},
            every axis x label/.style={at=(current axis.right of origin),anchor=west},
            axis on top
          ]
          

      \addplot [line width=2, penColor, smooth,samples=200,domain=(-3:5),<-] {x*sqrt(5-x)};


      \addplot[color=penColor,fill=penColor,only marks,mark=*] coordinates{(5,0)};
           

  \end{axis}
\end{tikzpicture}
\end{image}





We can see that $k(t)$ increases on $(-\infty, c]$ and then decreases on $[c,5]$, where $c$ is a real number around $3$.

\begin{explanation}
All of the tangent lines at $(a, k(a))$, where $a \in (-\infty, c)$ have \wordChoice{\choice[correct]{positive} \choice{negative}}  slopes, which means $k'(a) > 0$. \\

All of the tangent lines at $(a, k(a))$, where $a \in (-c, \infty)$ have \wordChoice{\choice{positive} \choice[correct]{negative}}  slopes, which means $k'(a) < 0$. \\
\end{explanation}


$\blacktriangleright$ \textbf{\textcolor{blue!55!black}{What is the value of $c$?}} \\

\begin{explanation}


\begin{itemize}
\item The highest point on the graph has a horizontal tangent line, which has a slope of $\answer{0}$.  
\item Therefore, $k'(t)$ will be $\answer{0}$ at the domain number corresponding to this highest point.
\end{itemize}




\[   k'(t) = \sqrt{5-t} + t \cdot \frac{1}{2} \cdot \frac{-1}{\sqrt{5-t}}    \]

(You don't know how to get the formula for the derivative yet.  That is a topic for Calculus.)

We are looking for a number, $c$, around $3$, such that 


\[   k'(c) = \sqrt{5-c} + c \cdot \frac{-1}{2 \sqrt{5-c}}  = 0  \]

We have a common factor of $5-c$.  The least power is $-\frac{1}{2}$.  We use the distributive property to factor this out.


\[  \frac{1}{\sqrt{5-c}} \left( (5-c) - \frac{c}{2} \right) = 0  \]




We need 


\[  \left( (5-c) - \frac{c}{2} \right) = 0  \]

\[  5 - \frac{3c}{2}  = 0  \]


\[  c = \answer{\frac{10}{3}}  \]


\end{explanation}


Therefore, $k(t)$ increases on $\left(-\infty, \frac{10}{3}\right]$ and then decreases on $\left[\frac{10}{3},5\right]$.








\begin{explanation}
In the previous example, instead of factoring out a common factor with the least power, we could have gotten a common denominator and create a single fraction.\\



\[   k'(c) = \sqrt{5-c} + c \cdot \frac{-1}{2 \sqrt{5-c}}  = 0  \]


\[   k'(c) = \frac{5-c}{\sqrt{5-c}} + c \cdot \frac{-1}{2 \sqrt{5-c}}  = 0  \]


\[   k'(c) = \frac{2(5-c)}{2 \sqrt{5-c}} +  \frac{-c}{2 \sqrt{5-c}}  = 0  \]


\[   k'(c) = \frac{2(5-c)}{2 \sqrt{5-c}} -  \frac{c}{2 \sqrt{5-c}}  = 0  \]


\[   k'(c) = \frac{2(5-c)-c}{2 \sqrt{5-c}}  = 0  \]

\[   k'(c) = \frac{10 - 3c}{2 \sqrt{5-c}}   = 0  \]



Now, we have a fraction equal to $0$, which happens when the numerator equals $0$. \\

\[  10 - 3c = 0  \]

\[  c = \frac{10}{3}  \]






\end{explanation}








Places in the domain of $f$, where $f' = 0$ are critically important to function analysis. The function switches its type of growth at these numbers.  Such numbers in the domain are called \textbf{\textcolor{purple!85!blue}{critical numbers}}.


Functions can also change their growth across other weird places - like spikes and corners and discontinuities.  The graph will not have a tangent line at the corresponding points. And, $f'$ will not have a value.




Therfore, we are collecting all of these types of domain numbers and calling them \textbf{\textcolor{purple!85!blue}{critical numbers}}.
















\section*{Critical Numbers}




\begin{definition} \textbf{\textcolor{green!50!black}{Critical Number}}

Let $f: D \mapsto R$ be a function. \\
$c \in D$ is a \textbf{critical number} if  either $f'(c) = 0$ or $f'(c)$ doesn't exist.


\end{definition}


\textbf{Note:}  Sometimes these are called \textbf{critical points}, especially when you investigate multi-variate functions.  But, we are just getting started.  So, we are trying to separate \textit{numbers} in the domain from \textit{points} on the graph. \\



\begin{center}


\textbf{\textcolor{purple!85!blue}{We call them critical numbers.}}

\end{center}


\textbf{Note:} You will learn how to get formulas for derivatives in Calculus.  In this course, you will be given the derivative formula, when needed. \\















\begin{example}

Consider the function $H(v) = 0.1(v+6)(v+1)(v-4)$.

Here is the graph of $y = H(v)$.








\begin{image}
\begin{tikzpicture}
  \begin{axis}[
            domain=-10:10, ymax=10, xmax=10, ymin=-10, xmin=-10,
            axis lines =center, xlabel=$v$, ylabel=$y$, grid = major,
            ytick={-10,-8,-6,-4,-2,2,4,6,8,10},
            xtick={-10,-8,-6,-4,-2,2,4,6,8,10},
            ticklabel style={font=\scriptsize},
            every axis y label/.style={at=(current axis.above origin),anchor=south},
            every axis x label/.style={at=(current axis.right of origin),anchor=west},
            axis on top
          ]
          

      \addplot [line width=2, penColor, smooth,samples=200,domain=(-7:5),<->] {0.1*(x+6)*(x+1)*(x-4)};


      %\addplot[color=penColor,fill=penColor,only marks,mark=*] coordinates{(5,0)};
           

  \end{axis}
\end{tikzpicture}
\end{image}




We can see that $H(v)$ increases, then decreases, then increases.  


The derivative is $H'(v) = 0.3 v^2 + 0.6 v - 2.2$. \\

Let's clean that up a bit, by factoring out $0.1$. $H'(v) = 0.1 \left( \answer{3 v^2 + 6 v - 22} \right)$. \\



The critical numbers are when $H'(v) = 0.1(3 v^2 + 6 v - 22) =0$ or when $3 v^2 + 6 v - 22 =0$,   a quadratic. \\


The only factors of $22$ are $\{ 1, 2, 11, 22    \}$.  None of those (or their negatives) are adding up to $6$.  So, we will use the quadratic formula.



\[  v = \frac{-6 \pm \sqrt{6^2 - 4 \cdot 3 \cdot (-22)}}{2 \cdot 3}  = \frac{-6 \pm \sqrt{300}}{6}    \]


\[  v = \frac{-6 \pm 10\sqrt{3}}{6}  =  \frac{-3 \pm 5\sqrt{3}}{3} \]



We have two solutions


\[  v =  \frac{-3 + 5\sqrt{3}}{3}  \approx  \answer[tolerance=0.01]{1.887} \]


\[  v =  \frac{-3 - 5\sqrt{3}}{3} \approx   \answer[tolerance=0.01]{-3.887} \]




The approximations are to give us an idea of the size of the numbers. However, we always prefer to quote the exact value.

\begin{itemize}
\item $H(v)$ increases on $\left( -\infty, \answer{\frac{-3 - 5\sqrt{3}}{3}}   \right]$.
\item $H(v)$ decreases on $\left[ \frac{-3 - 5\sqrt{3}}{3} , \frac{-3 + 5\sqrt{3}}{3}   \right]$.
\item $H(v)$ increases on $\left[ \answer{\frac{-3 + 5\sqrt{3}}{3}}, \infty  \right)$.
\end{itemize}



\end{example}





Isn't $1.887$ good enough?  \textbf{NO!}  

What about $1.886751346$? \textbf{NO!}  

What about $1.886751345948128822545743902509787278238008756350634380093$? \textbf{NO!}  

We want to be exact, when we can be exact.  That is why we are bringing our Algebra tools with us.  Otherwise, we would just use graphs all of the time.

\textbf{\textcolor{red!90!darkgray}{$\blacktriangleright$}} Our goal is to be precise with our descriptions of functions.


We will encounter many functions whose graphs are misleading - because they have no choice but to be inaccurate.

\textbf{\textcolor{red!90!darkgray}{$\blacktriangleright$}} We want our descriptions to be exact and offer nothing to question.









\section*{Graphs are Inherently Inaccurate}

Sorry.  There is nothing you can do about this. We draw graphs so that we can see them, which means they are too wide with huge dots all over the place. They are just inaccurate tools.  We still like them.





\begin{example}


Let $K(t) = (2t-3)e^{t-5}$.


Graph of $y = K(t)$.



\begin{image}
\begin{tikzpicture}
  \begin{axis}[
            domain=-10:10, ymax=10, xmax=10, ymin=-10, xmin=-10,
            axis lines =center, xlabel=$t$, ylabel=$y$, grid = major,
            ytick={-10,-8,-6,-4,-2,2,4,6,8,10},
            xtick={-10,-8,-6,-4,-2,2,4,6,8,10},
            ticklabel style={font=\scriptsize},
            every axis y label/.style={at=(current axis.above origin),anchor=south},
            every axis x label/.style={at=(current axis.right of origin),anchor=west},
            axis on top
          ]
          

      \addplot [line width=2, penColor, smooth,samples=200,domain=(-9:5.3),<->] {(2*x-3)*2.718^(x-5)};


      %\addplot[color=penColor,fill=penColor,only marks,mark=*] coordinates{(5,0)};
           

  \end{axis}
\end{tikzpicture}
\end{image}



It is easy to see that $K(t)$ is always positive and increasing. \\

\textbf{\textcolor{red!70!black}{Except, that is false.}} \\

We can easily see from the formula that $K$ has $\frac{3}{2}$ as a zero.\\


In Calculus, we will see that the derivative is $K'(t) = (2t-1)e^{t-5}$.   $K$ has $\frac{1}{2}$ as a critical number and the derivative changes sign over this critical number since the multiplicity is $1$, odd.

$K'(t)$ is negative on $\left( -\infty, \frac{1}{2} \right)$ and $K(t)$ is decreasing on $\left( -\infty, \frac{1}{2} \right)$.

$K'(t)$ is positive on $\left(\frac{1}{2}, \infty \right)$ and $K(t)$ is increasing on $\left(\frac{1}{2}, \infty \right)$.


The graph is totally misleading. \\


However, now that our algebra as cleared the way, we can zoom in right where our algebra tells us.





\begin{image}
\begin{tikzpicture}
  \begin{axis}[
            domain=-1.5:1.5, ymax=0.25, xmax=1.5, ymin=-0.25, xmin=-1.5,
            axis lines =center, xlabel=$t$, ylabel=$y$, grid = major,
            ytick={-0.2,-0.1,0.1,0.2},
            xtick={-1.25,-1,-0.75,-0.5,-0.25,0.25,0.5,0.75,1,1.25},
            ticklabel style={font=\scriptsize},
            every axis y label/.style={at=(current axis.above origin),anchor=south},
            every axis x label/.style={at=(current axis.right of origin),anchor=west},
            axis on top
          ]
          

      \addplot [line width=2, penColor, smooth,samples=300,domain=(-1.5:1.5)] {(2*x-3)*2.718^(x-5)};


      %\addplot[color=penColor,fill=penColor,only marks,mark=*] coordinates{(5,0)};
           

  \end{axis}
\end{tikzpicture}
\end{image}




\end{example}


















\begin{example}


Let $g(x) = \frac{1}{(x-1.95)(x-2.05)}$.


DESMOS graph of $y = g(x)$.



\begin{center}
\desmos{kloinmeoyi}{400}{300}
\end{center}



It is easy to see that the graph has $x=2$ as a vertical asymptote. \\

\textbf{\textcolor{red!70!black}{Except, that is false.}} \\

We can easily see from the formula that the natural domain of $K$ is $(\infty, 1.95) \cup (1.95, 2.05) \cup (2.05, \infty)$.\\

There are two vertical asymptotes and there is a third middle piece of the graph on $(1.95, 2.05)$. Where is it?


A calculator says $g(2)=-400$. \\

The third middle piece of the graph is way below what we see.  

\begin{itemize}
\item Can you find it in the DESMOS graph?
\item Can you get DESMOS to show is parabolic shape?
\end{itemize}




\end{example}


Sorry but graphs are not completely trustworthy.  \\


Graphs are wonderful!  They give us a big, overall, global view of the function.  They illustrate trends and characteristics and features.


\begin{center}
\textbf{\textcolor{red!70!black}{We just don't trust graphs.}}
\end{center}



\begin{center}
\textbf{\textcolor{blue!55!black}{We trust our algebra.}}
\end{center}



We still use graphs, but we trust our algebra.



























\begin{center}
\textbf{\textcolor{green!50!black}{ooooo-=-=-=-ooOoo-=-=-=-ooooo}} \\

more examples can be found by following this link\\ \link[More Examples of the Derivative]{https://ximera.osu.edu/csccmathematics/precalculus/precalculus/theDerivative/examples/exampleList}

\end{center}















\end{document}