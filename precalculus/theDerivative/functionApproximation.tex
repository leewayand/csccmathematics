\documentclass{ximera}


\graphicspath{
  {./}
  {ximeraTutorial/}
  {basicPhilosophy/}
}

\newcommand{\mooculus}{\textsf{\textbf{MOOC}\textnormal{\textsf{ULUS}}}}


\usepackage{tkz-euclide}\usepackage{tikz}
\usepackage{tikz-cd}
\usetikzlibrary{arrows}
\tikzset{>=stealth,commutative diagrams/.cd,
  arrow style=tikz,diagrams={>=stealth}} %% cool arrow head
\tikzset{shorten <>/.style={ shorten >=#1, shorten <=#1 } } %% allows shorter vectors

\usetikzlibrary{backgrounds} %% for boxes around graphs
\usetikzlibrary{shapes,positioning}  %% Clouds and stars
\usetikzlibrary{matrix} %% for matrix
\usepgfplotslibrary{polar} %% for polar plots
\usepgfplotslibrary{fillbetween} %% to shade area between curves in TikZ
\usetkzobj{all}
\usepackage[makeroom]{cancel} %% for strike outs
%\usepackage{mathtools} %% for pretty underbrace % Breaks Ximera
%\usepackage{multicol}
\usepackage{pgffor} %% required for integral for loops



%% http://tex.stackexchange.com/questions/66490/drawing-a-tikz-arc-specifying-the-center
%% Draws beach ball
\tikzset{pics/carc/.style args={#1:#2:#3}{code={\draw[pic actions] (#1:#3) arc(#1:#2:#3);}}}



\usepackage{array}
\setlength{\extrarowheight}{+.1cm}
\newdimen\digitwidth
\settowidth\digitwidth{9}
\def\divrule#1#2{
\noalign{\moveright#1\digitwidth
\vbox{\hrule width#2\digitwidth}}}
























%%This is to help with formatting on future title pages.
\newenvironment{sectionOutcomes}{}{}


\title{Approximating Functions}

\begin{document}

\begin{abstract}
tangent lines
\end{abstract}
\maketitle








What are we supposed to do with a function like this?

\[  C(x) =  \frac{(\sin(2x)+2)^{|x|}}{5(x^2+1)}       \]




\begin{image}
\begin{tikzpicture}
  \begin{axis}[
            domain=-10:10, ymax=10, xmax=10, ymin=-10, xmin=-10,
            axis lines =center, xlabel=$x$, ylabel={$y=C(x)$}, grid = major,
            ytick={-10,-8,-6,-4,-2,2,4,6,8,10},
            xtick={-10,-8,-6,-4,-2,2,4,6,8,10},
            yticklabels={$-10$,$-8$,$-6$,$-4$,$-2$,$2$,$4$,$6$,$8$,$10$}, 
            xticklabels={$-10$,$-8$,$-6$,$-4$,$-2$,$2$,$4$,$6$,$8$,$10$},
            ticklabel style={font=\scriptsize},
            every axis y label/.style={at=(current axis.above origin),anchor=south},
            every axis x label/.style={at=(current axis.right of origin),anchor=west},
            axis on top
          ]
          
          %\addplot [line width=2, penColor2, smooth,samples=100,domain=(-6:2)] {-2*x-3};
            \addplot [line width=2, penColor2, smooth,samples=300,domain=(-10:9.5)] {(sin(deg(2*x))+2)^abs(x)/(5*(x^2+1))};

          %\addplot[color=penColor,fill=penColor2,only marks,mark=*] coordinates{(-6,9)};
          %\addplot[color=penColor,fill=penColor2,only marks,mark=*] coordinates{(2,-7)};

          %\addplot[color=penColor2,fill=white,only marks,mark=*] coordinates{(2,-4.5)};
          %\addplot[color=penColor2,fill=white,only marks,mark=*] coordinates{(8,6)};


           

  \end{axis}
\end{tikzpicture}
\end{image}









The algebra is beyond us. 




The formula is too difficult to manipulate with algebra.  So, we concentrate on approximations. 


The function is too complicated to consider as a whole. So, we think of it in pieces - small pieces. 






If we are only interested in approximating small pieces of the function, then we can use a replacement function that does a pretty good job of approximating this function over a small interval - a replacement that is easier to work with.

And, our favorite functions are linear functions.


$C(x)$ appears to be linear-ish on the interval $(6.6, 7.0)$. Our plan is to create a linear function that does a pretty good job of approximating $C(x)$ on the interval. Graphically, that means a tangent line around $6.8$.  We need two data for this.  We need $C(6.8)$ and $C'(6.8)$.





Out linear approximation will be $approxC(x) = C'(6.8)(x-6.8) + C(6.8)$.

Of course, it will only be useful on $(6.6, 7.0)$, if that. 




We'll need the derivative (which Calculus will. provide): 
\textbf{Note:} On the interval $(6.6, 7.0)$, $| x | = x$.

\[  
C'(x) = \frac{(\sin(2x)+2)^x (\ln(\sin(2x)+2) + \frac{2 x \cos(2x)}{\sin(2x)+2})}{5(x^2+1)} - \frac{2x(\sin(2x)+2)^x}{5(x^2+1)^2}
\]




That's terrible.  But we only need it for $6.8$ and we are approximating, for which decimal numbers are handy.

$C'(6.8) \approx 17.132$


$C(6.8) \approx 5.359$


That gives $approxC(x) \approx 17.132(x-6.8)+5.359$







\begin{image}
\begin{tikzpicture}
  \begin{axis}[
            domain=-10:10, ymax=10, xmax=10, ymin=-10, xmin=-10,
            axis lines =center, xlabel=$x$, ylabel={$y=C(x)$}, grid = major,
            ytick={-10,-8,-6,-4,-2,2,4,6,8,10},
            xtick={-10,-8,-6,-4,-2,2,4,6,8,10},
            yticklabels={$-10$,$-8$,$-6$,$-4$,$-2$,$2$,$4$,$6$,$8$,$10$}, 
            xticklabels={$-10$,$-8$,$-6$,$-4$,$-2$,$2$,$4$,$6$,$8$,$10$},
            ticklabel style={font=\scriptsize},
            every axis y label/.style={at=(current axis.above origin),anchor=south},
            every axis x label/.style={at=(current axis.right of origin),anchor=west},
            axis on top
          ]
          
          %\addplot [line width=2, penColor2, smooth,samples=100,domain=(-6:2)] {-2*x-3};
            \addplot [line width=2, penColor, smooth,samples=300,domain=(-10:9.5)] {(sin(deg(2*x))+2)^abs(x)/(5*(x^2+1))};

          \addplot [line width=2, penColor2, smooth,samples=100,domain=(6.2:7.2)] {17*(x-6.8)+5.36};
          \addplot[color=penColor2,fill=penColor2,only marks,mark=*] coordinates{(6.8,5.36)};

           

  \end{axis}
\end{tikzpicture}
\end{image}












That is pretty good on $(6.6, 7.0)$.






\begin{image}
\begin{tikzpicture}
  \begin{axis}[
            domain=-10:10, ymax=10, xmax=10, ymin=-10, xmin=-10,
            axis lines =center, xlabel=$x$, ylabel={$y=C(x)$}, grid = major,
            ytick={-10,-8,-6,-4,-2,2,4,6,8,10},
            xtick={-10,-8,-6,-4,-2,2,4,6,8,10},
            yticklabels={$-10$,$-8$,$-6$,$-4$,$-2$,$2$,$4$,$6$,$8$,$10$}, 
            xticklabels={$-10$,$-8$,$-6$,$-4$,$-2$,$2$,$4$,$6$,$8$,$10$},
            ticklabel style={font=\scriptsize},
            every axis y label/.style={at=(current axis.above origin),anchor=south},
            every axis x label/.style={at=(current axis.right of origin),anchor=west},
            axis on top
          ]
          
          %\addplot [line width=2, penColor2, smooth,samples=100,domain=(-6:2)] {-2*x-3};
            \addplot [line width=2, penColor, smooth,samples=300,domain=(6.3:7.3)] {(sin(deg(2*x))+2)^abs(x)/(5*(x^2+1))};

          \addplot [line width=2, penColor2, smooth,samples=100,domain=(6.3:7.3)] {17*(x-6.8)+5.36};
          \addplot[color=penColor2,fill=penColor2,only marks,mark=*] coordinates{(6.8,5.36)};

           

  \end{axis}
\end{tikzpicture}
\end{image}







\begin{image}
\begin{tikzpicture}
  \begin{axis}[
            domain=6.2:7.4, ymax=10, xmax=7.4, ymin=-10, xmin=6.2,
            axis lines =center, xlabel=$x$, ylabel={$y=C(x)$}, grid = major,
            ytick={-10,-8,-6,-4,-2,2,4,6,8,10},
            xtick={6.3,6.5,6.7,6.9,7.1,7.3},
            yticklabels={$-10$,$-8$,$-6$,$-4$,$-2$,$2$,$4$,$6$,$8$,$10$}, 
            xticklabels={$6.3$,$6.5$,$6.7$,$6.9$,$7.1$,$7.3$},
            ticklabel style={font=\scriptsize},
            every axis y label/.style={at=(current axis.above origin),anchor=south},
            every axis x label/.style={at=(current axis.right of origin),anchor=west},
            axis on top
          ]
          
          %\addplot [line width=2, penColor2, smooth,samples=100,domain=(-6:2)] {-2*x-3};
            \addplot [line width=2, penColor, smooth,samples=300,domain=(6.3:7.3)] {(sin(deg(2*x))+2)^abs(x)/(5*(x^2+1))};

          \addplot [line width=2, penColor2, smooth,samples=100,domain=(6.3:7.3)] {17*(x-6.8)+5.36};
          \addplot[color=penColor2,fill=penColor2,only marks,mark=*] coordinates{(6.8,5.36)};

           

  \end{axis}
\end{tikzpicture}
\end{image}








Now we can approximate $C(x) \approx approxC(x) \approx 17.132(x-6.8)+5.359$ on $(6.6, 7.0)$.  That would save a lot of time and energy and not give up too much accuracy - as long as we stay in this interval.






However, these types of functions are for later courses.  We are just trying to figure out the concept here.  So, let's bring our investigation back down to our elementary functions.





\section*{Linear Approximations}


Approximating complex functions with linear functions is an important aspect of analysis, even if it is over a small interval.







\begin{example} Square Roots



Approximate $\sqrt{4.5}$ with a decimal expansion.

\textbf{\textcolor{red!75!green}{explanation}} 

First, graph $y = s(t) = \sqrt{t}$.


\begin{image}
\begin{tikzpicture}
  \begin{axis}[
            domain=-10:10, ymax=10, xmax=10, ymin=-10, xmin=-10,
            axis lines =center, xlabel=$t$, ylabel={$y=s(t)$}, grid = major,
            ytick={-10,-8,-6,-4,-2,2,4,6,8,10},
            xtick={-10,-8,-6,-4,-2,2,4,6,8,10},
            yticklabels={$-10$,$-8$,$-6$,$-4$,$-2$,$2$,$4$,$6$,$8$,$10$}, 
            xticklabels={$-10$,$-8$,$-6$,$-4$,$-2$,$2$,$4$,$6$,$8$,$10$},
            ticklabel style={font=\scriptsize},
            every axis y label/.style={at=(current axis.above origin),anchor=south},
            every axis x label/.style={at=(current axis.right of origin),anchor=west},
            axis on top
          ]
          
          	
			\addplot [line width=2, penColor, smooth,samples=200,domain=(0:9),->] {sqrt(x)};
			\addplot[color=penColor,fill=penColor,only marks,mark=*] coordinates{(0,0)};
			%\addplot [line width=2, penColor2, smooth,samples=100,domain=(6.2:7.2)] {17*(x-6.8)+5.36};
			%\addplot[color=penColor2,fill=penColor2,only marks,mark=*] coordinates{(6.8,5.36)};

  \end{axis}
\end{tikzpicture}
\end{image}


$4.5$ simply doesn't work well with the square root function.  However, $4$ does and $4$ is near $4.5$.


Let's create a linear approximation to $s(t)$ at $4$.

Courtesy of Calculus: the derivative of $s(t)$ is $s'(t) = \frac{1}{2\sqrt{t}}$.

We have 

\begin{itemize}
\item $s(4) = \sqrt{4} = 2$
\item $s'(4) = \frac{1}{2\sqrt{4}} = \frac{1}{4}$
\end{itemize}



The linear approximation is $L(t) = \frac{1}{4}(t-4)+2$










\begin{image}
\begin{tikzpicture}
  \begin{axis}[
            domain=-10:10, ymax=10, xmax=10, ymin=-10, xmin=-10,
            axis lines =center, xlabel=$t$, ylabel={$y=s(t)$}, grid = major,
            ytick={-10,-8,-6,-4,-2,2,4,6,8,10},
            xtick={-10,-8,-6,-4,-2,2,4,6,8,10},
            yticklabels={$-10$,$-8$,$-6$,$-4$,$-2$,$2$,$4$,$6$,$8$,$10$}, 
            xticklabels={$-10$,$-8$,$-6$,$-4$,$-2$,$2$,$4$,$6$,$8$,$10$},
            ticklabel style={font=\scriptsize},
            every axis y label/.style={at=(current axis.above origin),anchor=south},
            every axis x label/.style={at=(current axis.right of origin),anchor=west},
            axis on top
          ]
          
          	
			\addplot [line width=2, penColor, smooth,samples=200,domain=(0:9),->] {sqrt(x)};
			\addplot[color=penColor,fill=penColor,only marks,mark=*] coordinates{(0,0)};
			\addplot [line width=2, penColor2, smooth,samples=200,domain=(-6:9.5),<->] {0.25*(x-4)+2};

  \end{axis}
\end{tikzpicture}
\end{image}

Now we'll use our linear approximation to get an estimate to $\sqrt{4.5} = s(4.5)$.



\[   \sqrt{4.5} \approx   \frac{1}{4}(4.5-4)+2    =   \frac{0.5}{4}+2 =    2.125   \]

Compare this to what a calculator gives: $2.121320344$.  Not bad for just a linear function.





\end{example}








\section*{Polygonal Lines}


We might even approximate a function over a wider interval, just use more linear functions (i.e. line segments).





Graph of $y = p(k) = -k^2 + 7$.




\begin{image}
\begin{tikzpicture}
  \begin{axis}[
            domain=-10:10, ymax=10, xmax=10, ymin=-10, xmin=-10,
            axis lines =center, xlabel=$k$, ylabel={$y=p(k)$}, grid = major,
            ytick={-10,-8,-6,-4,-2,2,4,6,8,10},
            xtick={-10,-8,-6,-4,-2,2,4,6,8,10},
            yticklabels={$-10$,$-8$,$-6$,$-4$,$-2$,$2$,$4$,$6$,$8$,$10$}, 
            xticklabels={$-10$,$-8$,$-6$,$-4$,$-2$,$2$,$4$,$6$,$8$,$10$},
            ticklabel style={font=\scriptsize},
            every axis y label/.style={at=(current axis.above origin),anchor=south},
            every axis x label/.style={at=(current axis.right of origin),anchor=west},
            axis on top
          ]
          
          	
			\addplot [line width=2, penColor, smooth,samples=200,domain=(-4:4),<->] {-(x^2)+7};
			%\addplot[color=penColor,fill=penColor,only marks,mark=*] coordinates{(0,0)};
			%\addplot [line width=2, penColor2, smooth,samples=200,domain=(-6:9.5),<->] {0.25*(x-4)+2};

  \end{axis}
\end{tikzpicture}
\end{image}




Let's approximate $p(k) = -k^2+7$ with five linear functions, corresponding to five tangent lines tangent where $k=-2, -1, 0, 1, 2$.

$p(k)$ is a quadratic function, therefore, we know that $p'(k) = -2k$.  This will get us rate of change or slope.  Together with the point of tangency, we can get a linear model of $p(k)$.


\begin{model}

\begin{itemize}
\item $p(-2) = 3$ and $p'(-2) = 4$ give $p(k) \approx 4(k+2)+3$
\item $p(-1) = 6$ and $p'(-1) = 2$ give $p(k) \approx \answer{2(k+1)+6}$
\item $p(0) = 7$ and $p'(0) = 0$   give $p(k) \approx \answer{7}$
\item $p(1) = 6$ and $p'(1) = -2$   give  $ p(k) \approx \answer{-2(k-1)+6}$
\item $p(2) = 3$ and $p'(2) = -4$   give $p(k) \approx -4(k-2)+3$
\end{itemize}

\end{model}










\begin{image}
\begin{tikzpicture}
  \begin{axis}[
            domain=-10:10, ymax=10, xmax=10, ymin=-10, xmin=-10,
            axis lines =center, xlabel=$k$, ylabel={$y=p(k)$}, grid = major,
            ytick={-10,-8,-6,-4,-2,2,4,6,8,10},
            xtick={-10,-8,-6,-4,-2,2,4,6,8,10},
            yticklabels={$-10$,$-8$,$-6$,$-4$,$-2$,$2$,$4$,$6$,$8$,$10$}, 
            xticklabels={$-10$,$-8$,$-6$,$-4$,$-2$,$2$,$4$,$6$,$8$,$10$},
            ticklabel style={font=\scriptsize},
            every axis y label/.style={at=(current axis.above origin),anchor=south},
            every axis x label/.style={at=(current axis.right of origin),anchor=west},
            axis on top
          ]
          
          	
			\addplot [line width=2, penColor, smooth,samples=200,domain=(-4:4),<->] {-(x^2)+7};
			%\addplot[color=penColor,fill=penColor,only marks,mark=*] coordinates{(0,0)};
			\addplot [line width=2, penColor2, smooth,samples=200,domain=(-6:-1),<->] {4*(x+2)+3};
			\addplot [line width=2, penColor2, smooth,samples=200,domain=(-6:0),<->] {2*(x+1)+6};
			\addplot [line width=2, penColor2, smooth,samples=200,domain=(-6:6),<->] {7};
			\addplot [line width=2, penColor2, smooth,samples=200,domain=(0:6),<->] {-2*(x-1)+6};
			\addplot [line width=2, penColor2, smooth,samples=200,domain=(1:6),<->] {-4*(x-2)+3};

  \end{axis}
\end{tikzpicture}
\end{image}



It looks like maybe these will approximate $p(k)$ on $(-3, 3)$.








\begin{image}
\begin{tikzpicture}
  \begin{axis}[
            domain=-4:4, ymax=8, xmax=4, ymin=-6, xmin=-4,
            axis lines =center, xlabel=$k$, ylabel={$y=p(k)$}, grid = major,
            ytick={-6,-4,-2,2,4,6,8},
            xtick={-3,-2,-1,1,2,3},
            yticklabels={$-6$,$-4$,$-2$,$2$,$4$,$6$,$8$}, 
            xticklabels={$-3$,$-2$,$-1$,$1$,$2$,$3$},
            ticklabel style={font=\scriptsize},
            every axis y label/.style={at=(current axis.above origin),anchor=south},
            every axis x label/.style={at=(current axis.right of origin),anchor=west},
            axis on top
          ]
          
            
      \addplot [line width=2, penColor, smooth,samples=200,domain=(-4:4),<->] {-(x^2)+7};
      %\addplot[color=penColor,fill=penColor,only marks,mark=*] coordinates{(0,0)};
      \addplot [line width=2, penColor2, smooth,samples=200,domain=(-4:-1),<->] {4*(x+2)+3};
      \addplot [line width=2, penColor2, smooth,samples=200,domain=(-4:0),<->] {2*(x+1)+6};
      \addplot [line width=2, penColor2, smooth,samples=200,domain=(-4:4),<->] {7};
      \addplot [line width=2, penColor2, smooth,samples=200,domain=(0:4),<->] {-2*(x-1)+6};
      \addplot [line width=2, penColor2, smooth,samples=200,domain=(1:4),<->] {-4*(x-2)+3};

  \end{axis}
\end{tikzpicture}
\end{image}



















Now we need domains for each piece.  We'll decide the domains by locating the intersections of the lines.




$\blacktriangleright$  Intersection of $4(k+2)+3$ and $2(k+1)+6$.

\begin{procedure}

$4(k+2)+3 = 2(k+1)+6 $

$4k + 8 + 3 = \answer{2k + 2 + 6}$

$2k = \answer{-3}$

$k=-\frac{3}{2}$

\end{procedure}


$\blacktriangleright$  Intersection of $2(k+1)+6$ and $7$.

\begin{procedure}

$2(k+1)+6 = 7$

$2k + 2 + 6 = 7$

$2k = \answer{-1}$

$k = -\frac{1}{2}$

\end{procedure}




$\blacktriangleright$  By symmetry we can see that the other two intersections happen when $k = \frac{1}{2}$  and $k = \frac{3}{2}$.



Our approximating piecewise linear function is 



\[
p(k) \approx L(k) = 
\begin{cases}
  4(k+2)+3       &          \text{ on } \,     \left(-3, -\frac{3}{2}\right]   \\
  2(k+1)+6       &          \text{ on } \,     \left(-\frac{3}{2}, -\frac{1}{2}\right]   \\
  7              &          \text{ on } \,     \left(-\frac{1}{2}, \frac{1}{2}\right]   \\
  -2(k-1)+6      &          \text{ on } \,     \left(\frac{1}{2}, \frac{3}{2}\right]   \\
  -4(k-2)+3      &          \text{ on } \,     \left(\frac{3}{2}, 3\right)   
\end{cases}
\]










\begin{image}
\begin{tikzpicture}
  \begin{axis}[
            domain=-10:10, ymax=10, xmax=10, ymin=-10, xmin=-10,
            axis lines =center, xlabel=$k$, ylabel={$y=L(k)$}, grid = major,
            ytick={-10,-8,-6,-4,-2,2,4,6,8,10},
            xtick={-10,-8,-6,-4,-2,2,4,6,8,10},
            yticklabels={$-10$,$-8$,$-6$,$-4$,$-2$,$2$,$4$,$6$,$8$,$10$}, 
            xticklabels={$-10$,$-8$,$-6$,$-4$,$-2$,$2$,$4$,$6$,$8$,$10$},
            ticklabel style={font=\scriptsize},
            every axis y label/.style={at=(current axis.above origin),anchor=south},
            every axis x label/.style={at=(current axis.right of origin),anchor=west},
            axis on top
          ]
          
          	
			\addplot [line width=2, penColor, smooth,samples=200,domain=(-4:4),<->] {-(x^2)+7};
			%\addplot[color=penColor,fill=penColor,only marks,mark=*] coordinates{(0,0)};
			\addplot [line width=2, penColor2, smooth,samples=200,domain=(-3:-1.5)] {4*(x+2)+3};
			\addplot [line width=2, penColor2, smooth,samples=200,domain=(-1.5:-0.5)] {2*(x+1)+6};
			\addplot [line width=2, penColor2, smooth,samples=200,domain=(-0.5:0.5)] {7};
			\addplot [line width=2, penColor2, smooth,samples=200,domain=(0.5:1.5)] {-2*(x-1)+6};
			\addplot [line width=2, penColor2, smooth,samples=200,domain=(1.5:3)] {-4*(x-2)+3};

  \end{axis}
\end{tikzpicture}
\end{image}




Pretty good.











\begin{image}
\begin{tikzpicture}
  \begin{axis}[
            domain=-4:4, ymax=8, xmax=4, ymin=-2, xmin=-4,
            axis lines =center, xlabel=$k$, ylabel={$y=L(k)$}, grid = major,
            ytick={-2,2,4,6,8},
            xtick={-4,-2,2,4},
            yticklabels={$-10$,$-8$,$-6$,$-4$,$-2$,$2$,$4$,$6$,$8$,$10$}, 
            xticklabels={$-10$,$-8$,$-6$,$-4$,$-2$,$2$,$4$,$6$,$8$,$10$},
            ticklabel style={font=\scriptsize},
            every axis y label/.style={at=(current axis.above origin),anchor=south},
            every axis x label/.style={at=(current axis.right of origin),anchor=west},
            axis on top
          ]
          
          	
			\addplot [line width=2, penColor, smooth,samples=200,domain=(-4:4),<->] {-(x^2)+7};
			%\addplot[color=penColor,fill=penColor,only marks,mark=*] coordinates{(0,0)};
			\addplot [line width=2, penColor2, smooth,samples=200,domain=(-3:-1.5)] {4*(x+2)+3};
			\addplot [line width=2, penColor2, smooth,samples=200,domain=(-1.5:-0.5)] {2*(x+1)+6};
			\addplot [line width=2, penColor2, smooth,samples=200,domain=(-0.5:0.5)] {7};
			\addplot [line width=2, penColor2, smooth,samples=200,domain=(0.5:1.5)] {-2*(x-1)+6};
			\addplot [line width=2, penColor2, smooth,samples=200,domain=(1.5:3)] {-4*(x-2)+3};

  \end{axis}
\end{tikzpicture}
\end{image}



Linear approximations do better on straight parts of the graph rather than parts with more curve to them.  Calculus will give us a way of measuring this curvature.



















































Suppose a car is travelling at $5 \, mph = \frac{5 \, miles}{1 \, hour}$. \\

This means every hour it moves another $5 \, miles$. \\

\begin{itemize}
\item If the car travels for $2 \, hours$, then it moves $5 \, miles + 5 \, miles = 10 \, miles$, or $\frac{5 \, miles}{1 \, hour} \cdot 2 \, hours = 10 \, miles$

\item If the car travels for $5 \, hours$, then it moves $\frac{5 \, miles}{1 \, hour} \cdot 5 \, hours = 25 \, miles$

\item If the car travels for $\frac{1}{2} \, hour$, then it moves $\frac{5 \, miles}{1 \, hour} \cdot \frac{1}{2} \, hours = 2.5 \,miles$
\end{itemize}


The way we calculate distance is by mulitplying the rate times the length of time.

We can model this graphically.










\section*{Area = Accumulated Distance}



$\blacktriangleright$ \textbf{A Constant Rate}



Our car's rate is a constant function: $r(t) = 5$. \\








\begin{image}
\begin{tikzpicture}
  \begin{axis}[
            domain=-10:10, ymax=10, xmax=10, ymin=-10, xmin=-10,
            axis lines =center, xlabel=$t$, ylabel={$y=r(t)$}, grid = major,
            ytick={-10,-8,-6,-4,-2,2,4,6,8,10},
            xtick={-10,-8,-6,-4,-2,2,4,6,8,10},
            yticklabels={$-10$,$-8$,$-6$,$-4$,$-2$,$2$,$4$,$6$,$8$,$10$}, 
            xticklabels={$-10$,$-8$,$-6$,$-4$,$-2$,$2$,$4$,$6$,$8$,$10$},
            ticklabel style={font=\scriptsize},
            every axis y label/.style={at=(current axis.above origin),anchor=south},
            every axis x label/.style={at=(current axis.right of origin),anchor=west},
            axis on top
          ]
          
            
      \addplot [line width=2, penColor, smooth,samples=200,domain=(0:9),->] {5};
      %\addplot[color=penColor,fill=penColor,only marks,mark=*] coordinates{(0,0)};


  \end{axis}
\end{tikzpicture}
\end{image}



Since time is measured by the horizontal axis. our $distance = rate \times time$ calculation can be represented visually as the rectangular area under the rate curve.








\begin{image}
\begin{tikzpicture}
  \begin{axis}[
            domain=-10:10, ymax=10, xmax=10, ymin=-10, xmin=-10,
            axis lines =center, xlabel=$t$, ylabel={$y=r(t)$}, grid = major, grid style={dashed},
            ytick={-10,-8,-6,-4,-2,2,4,6,8,10},
            xtick={-10,-8,-6,-4,-2,2,4,6,8,10},
            yticklabels={$-10$,$-8$,$-6$,$-4$,$-2$,$2$,$4$,$6$,$8$,$10$}, 
            xticklabels={$-10$,$-8$,$-6$,$-4$,$-2$,$2$,$4$,$6$,$8$,$10$},
            ticklabel style={font=\scriptsize},
            every axis y label/.style={at=(current axis.above origin),anchor=south},
            every axis x label/.style={at=(current axis.right of origin),anchor=west},
            axis on top
          ]
          
            

            \draw[gray,fill=gray] (axis cs:0,0) -- (axis cs:0,5) -- (axis cs:1,5) -- (axis cs:1,0) -- (axis cs:0,0);
      \addplot [line width=2, penColor, smooth,samples=200,domain=(0:9),->] {5};

      %\addplot [name path=darkerA,domain=0:1,draw=none] {5};
      %\addplot [name path=darkerB,domain=0:1,draw=none] {0};
      %\addplot [blue!50!black!50] fill between[of=darkerA and darkerB];



      %\draw[gray,fill=gray] (axis cs:0,0) -- (axis cs:0,5) -- (axis cs:1,5) -- (axis cs:1,0) -- (axis cs:0,0);



  \end{axis}
\end{tikzpicture}
\end{image}


The area of the rectangle represents the distance travelled. \\

The height of the rectangle is the height of the graph, which is the value of the rate, $5 \, mph$.

The width of the rectangle is a time measurement. In the graph above, this is $1 \, hour$.

The area is height times width, which gives $area = 5 \, mph \times 1 \, hour = 5 \, miles$ \\



Let $D(t) = $ the accumulated miles travelled after $t \, hours$.

Graphically, this is the area of the rectangle below the graph from $0$ to $t$.  


\begin{model}

The formula would be $D(t) = \answer{5 t}$ miles.

\end{model}








\begin{image}
\begin{tikzpicture}
  \begin{axis}[
            domain=-10:10, ymax=10, xmax=10, ymin=-10, xmin=-10,
            axis lines =center, xlabel=$t$, ylabel={$y=D(t)$}, grid = major, grid style={dashed},
            ytick={-10,-8,-6,-4,-2,2,4,6,8,10},
            xtick={-10,-8,-6,-4,-2,2,4,6,8,10},
            yticklabels={$-10$,$-8$,$-6$,$-4$,$-2$,$2$,$4$,$6$,$8$,$10$}, 
            xticklabels={$-10$,$-8$,$-6$,$-4$,$-2$,$2$,$4$,$6$,$8$,$10$},
            ticklabel style={font=\scriptsize},
            every axis y label/.style={at=(current axis.above origin),anchor=south},
            every axis x label/.style={at=(current axis.right of origin),anchor=west},
            axis on top
          ]
          
            

      \addplot [line width=2, penColor, smooth,samples=200,domain=(0:9), ->] {5};

      \addplot [line width=2, penColor2, smooth,samples=200,domain=(0:2), ->] {5*x};



  \end{axis}
\end{tikzpicture}
\end{image}



$D(t)$ is the accumulated distance travelled in $t \, hours$, which makes $r(t)$ its derivative.

\[    D'(t) = r(t)        \]


\[    (5 t)' = 5     \]





$\blacktriangleright$ \textbf{A Linear Rate}



This time our rate function is a linear function: $r(t) = 2 t$. \\

Our car goes faster as it goes farther.






\begin{image}
\begin{tikzpicture}
  \begin{axis}[
            domain=-1:10, ymax=10, xmax=10, ymin=-1, xmin=-1,
            axis lines =center, xlabel=$t$, ylabel={$y=r(t)$}, grid = major,
            ytick={2,4,6,8,10},
            xtick={2,4,6,8,10},
            yticklabels={$2$,$4$,$6$,$8$,$10$}, 
            xticklabels={$2$,$4$,$6$,$8$,$10$},
            ticklabel style={font=\scriptsize},
            every axis y label/.style={at=(current axis.above origin),anchor=south},
            every axis x label/.style={at=(current axis.right of origin),anchor=west},
            axis on top
          ]
          
            
      \addplot [line width=2, penColor, smooth,samples=200,domain=(0:5),->] {2*x};
      %\addplot[color=penColor,fill=penColor,only marks,mark=*] coordinates{(0,0)};


  \end{axis}
\end{tikzpicture}
\end{image}



$Distance$ is still represented visually as the area under the rate curve.


The area is a triangle this time. 




\begin{image}
\begin{tikzpicture}
  \begin{axis}[
            domain=-1:10, ymax=10, xmax=10, ymin=-1, xmin=-1,
            axis lines =center, xlabel=$t$, ylabel={$y=r(t)$}, grid = major,
            ytick={2,4,6,8,10},
            xtick={2,4,6,8,10},
            yticklabels={$2$,$4$,$6$,$8$,$10$}, 
            xticklabels={$2$,$4$,$6$,$8$,$10$},
            ticklabel style={font=\scriptsize},
            every axis y label/.style={at=(current axis.above origin),anchor=south},
            every axis x label/.style={at=(current axis.right of origin),anchor=west},
            axis on top
          ]
          
            \draw[black,fill=lightgray] (axis cs:0,0) -- (axis cs:3.7,7.4) -- (axis cs:3.7,0) -- (axis cs:0,0);
      \addplot [line width=2, penColor, smooth,samples=200,domain=(0:5),->] {2*x};
      %\addplot[color=penColor,fill=penColor,only marks,mark=*] coordinates{(0,0)};
      \node at (axis cs:3.7,-0.5) [black] {$t$};


  \end{axis}
\end{tikzpicture}
\end{image}



 Over the interval $[0,t]$, the area is $D(t) = \frac{1}{2} \cdot t \cdot (2 t) = t^2$



$D(t) = t^2$ is the accumulated distance travelled in $t \, hours$, which makes $2 t$ its derivative.



\[    (t^2)' = 2 t     \]












\begin{image}
\begin{tikzpicture}
  \begin{axis}[
            domain=-1:10, ymax=10, xmax=10, ymin=-1, xmin=-1,
            axis lines =center, xlabel=$t$, ylabel={$y=r(t)$}, grid = major,
            ytick={2,4,6,8,10},
            xtick={2,4,6,8,10},
            yticklabels={$2$,$4$,$6$,$8$,$10$}, 
            xticklabels={$2$,$4$,$6$,$8$,$10$},
            ticklabel style={font=\scriptsize},
            every axis y label/.style={at=(current axis.above origin),anchor=south},
            every axis x label/.style={at=(current axis.right of origin),anchor=west},
            axis on top
          ]
          
            
      \addplot [line width=2, penColor, smooth,samples=200,domain=(0:5),->] {2*x};
      \addplot [line width=2, penColor2, smooth,samples=200,domain=(0:3),->] {x^2};
      %\addplot[color=penColor,fill=penColor,only marks,mark=*] coordinates{(0,0)};


  \end{axis}
\end{tikzpicture}
\end{image}




























\section*{Accumulation}



Below is the graph of $y = p(k) = -k^2 + 7$, approximated with

\[
L(k) = 
\begin{cases}
  4(k+2)+3       &          \text{ on } \,     \left(-3, -\frac{3}{2}\right]   \\
  2(k+1)+6       &          \text{ on } \,     \left(-\frac{3}{2}, -\frac{1}{2}\right]   \\
  7              &          \text{ on } \,     \left(-\frac{1}{2}, \frac{1}{2}\right]   \\
  -2(k-1)+6      &          \text{ on } \,     \left(\frac{1}{2}, \frac{3}{2}\right]   \\
  -4(k-2)+3      &          \text{ on } \,     \left(\frac{3}{2}, 3\right)   
\end{cases}
\]


$L(k)$ is a piecewise linear function, which means we can calculate area using rectangles and triangles.


\begin{image}
\begin{tikzpicture}
  \begin{axis}[
            domain=-4:4, ymax=8, xmax=4, ymin=-1, xmin=-4,
            axis lines =center, xlabel=$k$, ylabel={$y=L(k)$}, grid = major,
            ytick={2,4,6,8},
            xtick={-4,-2,2,4},
            yticklabels={$2$,$4$,$6$,$8$}, 
            xticklabels={$-4$,$-2$,$2$,$4$},
            ticklabel style={font=\scriptsize},
            every axis y label/.style={at=(current axis.above origin),anchor=south},
            every axis x label/.style={at=(current axis.right of origin),anchor=west},
            axis on top
          ]
          
            \draw[black,fill=gray] (axis cs:-2.75,0) -- (axis cs:-1.5,5) -- (axis cs:-1.5,0) -- (axis cs:-2.75,0);
            \draw[black,fill=lightgray] (axis cs:-1.5,0) -- (axis cs:-1.5,5) -- (axis cs:-0.5,7) -- (axis cs:-0.5,0) -- (axis cs:-1.5,0);
            \draw[black,fill=gray] (axis cs:-0.5,0) -- (axis cs:-0.5,7) -- (axis cs:0.5,7) -- (axis cs:0.5,0) -- (axis cs:-0.5,0);
            \draw[black,fill=lightgray] (axis cs:1.5,0) -- (axis cs:1.5,5) -- (axis cs:0.5,7) -- (axis cs:0.5,0) -- (axis cs:1.5,0);
            \draw[black,fill=gray] (axis cs:2.75,0) -- (axis cs:1.5,5) -- (axis cs:1.5,0) -- (axis cs:2.75,0);
            
      \addplot [line width=2, penColor, smooth,samples=200,domain=(-4:4),<->] {-(x^2)+7};
      %\addplot[color=penColor,fill=penColor,only marks,mark=*] coordinates{(0,0)};
      \addplot [line width=2, penColor2, smooth,samples=200,domain=(-3:-1.5)] {4*(x+2)+3};
      \addplot [line width=2, penColor2, smooth,samples=200,domain=(-1.5:-0.5)] {2*(x+1)+6};
      \addplot [line width=2, penColor2, smooth,samples=200,domain=(-0.5:0.5)] {7};
      \addplot [line width=2, penColor2, smooth,samples=200,domain=(0.5:1.5)] {-2*(x-1)+6};
      \addplot [line width=2, penColor2, smooth,samples=200,domain=(1.5:3)] {-4*(x-2)+3};

  \end{axis}
\end{tikzpicture}
\end{image}


The shaded regions are made of triangles, trapezoids, and rectangles.  And, we know their formulas from Geometry class.


\begin{center}

area = triangle + trapezoid + rectangle + trapezoid + triangle
\end{center}



\[ \text{Area} = \frac{1}{2} \cdot 1.25 \cdot L(-1.5) + \frac{1}{2} \cdot (L(-1.5) + L(-0.5)) \cdot 1 + 1 \cdot L(0.5) + \frac{1}{2} \cdot (L(1.5) + L(0.5)) \cdot 1  +    \frac{1}{2} \cdot 1.25 \cdot L(1.5)    \]



\[ \text{Area} = \frac{1}{2} \cdot 1.25 \cdot 5 + \frac{1}{2} \cdot 12 \cdot 1 + 1 \cdot 7 + \frac{1}{2} \cdot 12 \cdot 1  +    \frac{1}{2} \cdot 1.25 \cdot 5    \]


\[ \text{Area} = 25.25    \]


This is the area under $y = L(x)$, so it is the approximate area under $y = p(k) = -k^2 + 7$ from $k = -3$ to $k = 3$.







































\begin{center}
\textbf{\textcolor{green!50!black}{ooooo-=-=-=-ooOoo-=-=-=-ooooo}} \\

more examples can be found by following this link\\ \link[More Examples of the Derivative]{https://ximera.osu.edu/csccmathematics/precalculus/precalculus/theDerivative/examples/exampleList}

\end{center}






\end{document}
