\documentclass{ximera}


\graphicspath{
  {./}
  {ximeraTutorial/}
  {basicPhilosophy/}
}

\newcommand{\mooculus}{\textsf{\textbf{MOOC}\textnormal{\textsf{ULUS}}}}


\usepackage{tkz-euclide}\usepackage{tikz}
\usepackage{tikz-cd}
\usetikzlibrary{arrows}
\tikzset{>=stealth,commutative diagrams/.cd,
  arrow style=tikz,diagrams={>=stealth}} %% cool arrow head
\tikzset{shorten <>/.style={ shorten >=#1, shorten <=#1 } } %% allows shorter vectors

\usetikzlibrary{backgrounds} %% for boxes around graphs
\usetikzlibrary{shapes,positioning}  %% Clouds and stars
\usetikzlibrary{matrix} %% for matrix
\usepgfplotslibrary{polar} %% for polar plots
\usepgfplotslibrary{fillbetween} %% to shade area between curves in TikZ
\usetkzobj{all}
\usepackage[makeroom]{cancel} %% for strike outs
%\usepackage{mathtools} %% for pretty underbrace % Breaks Ximera
%\usepackage{multicol}
\usepackage{pgffor} %% required for integral for loops



%% http://tex.stackexchange.com/questions/66490/drawing-a-tikz-arc-specifying-the-center
%% Draws beach ball
\tikzset{pics/carc/.style args={#1:#2:#3}{code={\draw[pic actions] (#1:#3) arc(#1:#2:#3);}}}



\usepackage{array}
\setlength{\extrarowheight}{+.1cm}
\newdimen\digitwidth
\settowidth\digitwidth{9}
\def\divrule#1#2{
\noalign{\moveright#1\digitwidth
\vbox{\hrule width#2\digitwidth}}}
























%%This is to help with formatting on future title pages.
\newenvironment{sectionOutcomes}{}{}


\title{Algebra}

\begin{document}

\begin{abstract}
exponential form
\end{abstract}
\maketitle










The template for the formula of the basic exponential function looks like



\[  A \cdot r^x   \, \text{ with } \,  A, r \in \mathbb{R} \, | \,  A \ne 0, \, r > 0   \]




Exponential functions are those functions that \textbf{\textcolor{red!80!black}{CAN}} be represented by a formula of the form $A \cdot r^x$.
















\begin{example}  Exponential Functions \\



$f(x) = 7 \cdot 2^{3x+1} $ represents an exponential function, because this formula can be rewritten in the form $a \cdot r^x$.   \\


\begin{explanation}


\[
f(x) = 7 \cdot 2^{3x+1}
\]


\[
f(x) = 7 \cdot 2^{3x}  \, 2^1
\]

\[
f(x) = 7 \cdot 2 \, 2^{3x} 
\]

\[
f(x) = 7 \cdot 2 \, (2^3)^x 
\]

\[
f(x) = 14 \cdot 8^x 
\]


\end{explanation}

\end{example}






















\begin{example}  Exponential Functions \\



\[ g(t) = \frac{5 \cdot 3^{8 t + 9}}{7 \cdot 3^{6 t + 4}} \]

represents an exponential function, because this formula can be rewritten in the form $a \cdot r^t$.   \\


\begin{explanation}


\[ g(t) = \frac{5 \cdot 3^{8 t + 9}}{7 \cdot 3^{6 t + 4}} \]



\[ 
g(t) = \frac{5}{7} \cdot \frac{3^{8 t + 9}}{3^{6 t + 4}} 
\]



\[ 
g(t) = \frac{5}{7} \cdot 3^{(8 t + 9)-(6 t + 4)}
\]




\[ 
g(t) = \frac{5}{7} \cdot 3^{2 t + 5}
\]


\[ 
g(t) = \frac{5}{7} \cdot 3^5 \cdot 3^{2 t}
\]



\[ 
g(t) = \left( \frac{5}{7} \cdot 3^5 \right) \cdot (3^2)^t
\]




\[ 
g(t) = \left( \frac{5}{7} \cdot 3^5 \right) \cdot 9^t
\]








\end{explanation}

\end{example}


















\begin{example}  Exponential Functions \\



$H(k) = 6 \sqrt{e^{8 k + 6}} $ represents an exponential function, because this formula can be rewritten in the form $a \cdot r^x$.   \\


\begin{explanation}


\[
H(k) = 6 \, \sqrt{e^{8 k + 6}}
\]


\[
H(k) = 6 \, \left( e^{8 k + 6} \right)^{\tfrac{1}{2}}
\]

\[
H(k) = 6 \, \left( e^{(8 k + 6 ) \cdot \tfrac{1}{2}} \right)
\]

\[
H(k) = 6 \, \left( e^{(4 k + 3)} \right)
\]

\[
H(k) = \left( 6 \cdot \answer{e^3} \right) \left( \answer{e^4} \right)^k
\]


\end{explanation}

\end{example}

















\begin{example}  Exponential Functions \\



$W(m) = 5^{2 m} \cdot 7^{3 m}$ represents an exponential function, because this formula can be rewritten in the form $a \, r^x$.   \\


\begin{explanation}


\[
W(m) = 5^{2 m} \cdot 7^{3 m}
\]


\[
W(m) = \left( 5^2 \right)^m \cdot \left( 7^3 \right)^m
\]

\[
W(m) = \left( 5^2 \cdot 7^3 \right)^m 
\]




\end{explanation}

\end{example}














The template for the formula of the basic shifted exponential function looks like



\[  A \cdot r^x + D   \, \text{ with } \,  A, D, r \in \mathbb{R} \, | \,  A \ne 0, \, r > 0   \]




However, this is not how we usually encounter exponential and shifted exponential functions.  


\begin{template}

We usually encounter exponential and shifted exponential functions in a more general form. \\




\[  A \cdot r^{B \, x + C}     \]


\[  A \cdot r^{B \, x + C} + D    \]



\end{template}




$r$ is called the \textbf{base}. \\
$A$ is called the \textbf{leading coefficient} of the function. \\
$B$ is the \textbf{leading coefficient} of the linear function in the exponent. \\
$D$ is called the \textbf{constant term}. \\














\begin{center}
\textbf{\textcolor{green!50!black}{ooooo-=-=-=-ooOoo-=-=-=-ooooo}} \\

more examples can be found by following this link\\ \link[More Examples of Exponential Functions]{https://ximera.osu.edu/csccmathematics/precalculus2/precalculus2/expFunctions/examples/exampleList}

\end{center}









\end{document}
