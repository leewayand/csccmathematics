\documentclass{ximera}


\graphicspath{
  {./}
  {ximeraTutorial/}
  {basicPhilosophy/}
}

\newcommand{\mooculus}{\textsf{\textbf{MOOC}\textnormal{\textsf{ULUS}}}}


\usepackage{tkz-euclide}\usepackage{tikz}
\usepackage{tikz-cd}
\usetikzlibrary{arrows}
\tikzset{>=stealth,commutative diagrams/.cd,
  arrow style=tikz,diagrams={>=stealth}} %% cool arrow head
\tikzset{shorten <>/.style={ shorten >=#1, shorten <=#1 } } %% allows shorter vectors

\usetikzlibrary{backgrounds} %% for boxes around graphs
\usetikzlibrary{shapes,positioning}  %% Clouds and stars
\usetikzlibrary{matrix} %% for matrix
\usepgfplotslibrary{polar} %% for polar plots
\usepgfplotslibrary{fillbetween} %% to shade area between curves in TikZ
\usetkzobj{all}
\usepackage[makeroom]{cancel} %% for strike outs
%\usepackage{mathtools} %% for pretty underbrace % Breaks Ximera
%\usepackage{multicol}
\usepackage{pgffor} %% required for integral for loops



%% http://tex.stackexchange.com/questions/66490/drawing-a-tikz-arc-specifying-the-center
%% Draws beach ball
\tikzset{pics/carc/.style args={#1:#2:#3}{code={\draw[pic actions] (#1:#3) arc(#1:#2:#3);}}}



\usepackage{array}
\setlength{\extrarowheight}{+.1cm}
\newdimen\digitwidth
\settowidth\digitwidth{9}
\def\divrule#1#2{
\noalign{\moveright#1\digitwidth
\vbox{\hrule width#2\digitwidth}}}
























%%This is to help with formatting on future title pages.
\newenvironment{sectionOutcomes}{}{}


\title{Reciprocals}

\begin{document}

\begin{abstract}
upsidedown and backwards
\end{abstract}
\maketitle




\section*{Reciprocals}


\textbf{\textcolor{red!70!black}{$\blacktriangleright$}} The \textbf{reciprocal} of the fraction $\frac{A}{B}$ is $\frac{B}{A}$.





\textbf{Note:} If we say $\frac{A}{B}$ is a fraction, then we are automatically saying that $B \ne 0$, because the denominator of a fraction cannot equal $0$. 

$A$ might equal $0$.  The numerator of a fraction can equal $0$. 


However, then the reciprocal of $\frac{A}{B}$ would not be a fraction, because its denoninator would be $A$, which equals $0$.



\begin{warning}

The reciprocal of any fraction is again a fraction, except for $0$. 

Any fraction representing $0$ does not have a reciprocal.


\end{warning}



\subsection*{Size}



There are only two numbers that equal their own reciprocals.



$\blacktriangleright$ The only fractions that represent $1$ are fractions where the numerator and denominator are equal.  They are of the form $\frac{A}{A}$.  The reciprocal of such a fraction would be $\frac{A}{A}$, which equals $1$ again.


$\blacktriangleright$ The only fractions that represent $1$ are fractions where the numerator and denominator are opposites.  They are of the form $\frac{-A}{A}$.  The reciprocal of such a fraction would be $\frac{A}{-A}$, which equals $-1$ again.



$-1$ and $1$ are the only numbers tha equal their own reciprocal. \\





$\blacktriangleright$ If a fraction has a small value, then it is of the form $\frac{small}{big}$.  Its reciprocal would look like $\frac{big}{small}$, which has a large value. \\



$\blacktriangleright$ Conversely, if a fraction has a big value, then it is of the form $\frac{big}{small}$.  Its reciprocal would look like $\frac{small}{big}$, which has a small value. \\




This helps us think about zeros and singularities.



\begin{exmaple}

Consider the linear function $f(x) = \frac{5x - 10}{3}$. \\

Its only zero is $2$. \\

Since the leading coefficient is positive, we have


\[
\lim\limits_{x \to 2^-}f(x) = 0^-
\]


As $x$ approaches $2$ from the negative side, $f(x)$ approaches $0$ from the negative side.





\[
\lim\limits_{x \to 2^+}f(x) = 0^+
\]


As $x$ approaches $2$ from the positive side, $f(x)$ approaches $0$ from the positive side.


\textbf{\textcolor{blue!55!black}{What happens with the reciprocal function?}} \\

The reciprocal of $f(x)$ is another function.  Let's call it $g$.  Let's select $t$ for its variable.


\[
g(t) = \frac{3}{5t - 10}
\]


The domain of $g$ does not include $2$, since that would make hte denomintor equal to $0$. \\


Now what happens as $t$ approaches $2$? \\

As $t$ approaches $2$ from the negative side, the numerator stays constant at $3$, but the denominator approach $0$ from negative side.  The fraction for $g$ starts to look like $\frac{big}{small}$, which has a big value. \\

The numerator is positive and the denominator will be negative.  That makes the whole fraction negative.  \\


Let $\epsilon$ be a very small positive number. On the interval $(2 - \epsilon, 2)$, $g$ is unbounded negatively. \\

$2$ is a singularity of $g$.  There is a vertical asymptote on the graph. \\




On the other side. On the interval $(2, 2 + \epsilon)$, $g$ is unbounded positively. \\










\textbf{\textcolor{blue!55!black}{$\blacktriangleright$ desmos graph}} 
\begin{center}
\desmos{mmilwksy8w}{400}{300}
\end{center}





\end{example}



\subsection*{Sign}


If a fraction, $\frac{A}{B}$ represents a negative value, then the sign of the numerator and denominator are opposite. \\

That means the signs of the numerator and denominator of the reciprocal are also opposite.  The reciprocal is also negative. \\


If a fraction, $\frac{A}{B}$ represents a positive value, then the sign of the numerator and denominator are the same. \\

That means the signs of the numerator and denominator of the reciprocal are also the same.  The reciprocal is also positive.








\begin{fact}

A function and its reciprocal have the same sign.


\end{fact}






















\subsection*{Behavior}



Conside the two quotient funcitons from the previous example.


\[
f(x) = \frac{5x - 10}{3}
\]




\[
g(t) = \frac{3}{5t - 10}
\]



$f(x)$ is a linear function with a positive leading coeffcient.  So, it is an increasing function. \\


The fraction for $f$ gets greater and greater. \\

That means its reciprocal gets lesser and lesser. \\

$g$ is a decreasing function.




\begin{fact}

A function and its reciprocal have opposite behaviors.


\end{fact}

If we know the behavior of a function, then we automatically know the behavior of its reciprocal. \\





\textbf{\textcolor{purple!80!black}{Sine}} 

We know the behavior of $\sin(\theta)$, which means we also know the behavior of its reciprocal $\frac{1}{\sin(\theta)}$. \\


We know the zeros of $\sin(\theta)$, which means we also know the singularities of its reciprocal $\frac{1}{\sin(\theta)}$. \\







\textbf{\textcolor{purple!80!black}{Cosine}} 

We know the behavior of $\cos(\theta)$, which means we also know the behavior of its reciprocal $\frac{1}{\cos(\theta)}$. \\


We know the zeros of $\cos(\theta)$, which means we also know the singularities of its reciprocal $\frac{1}{\cos(\theta)}$. \\










\textbf{\textcolor{purple!80!black}{Tangent}} 

We know the behavior of $\tan(\theta)$, which means we also know the behavior of its reciprocal $\frac{1}{\tan(\theta)}$. \\


We know the zeros of $\tan(\theta)$, which means we also know the singularities of its reciprocal $\frac{1}{\tan(\theta)}$. \\





We know the singularities of $\tan(\theta)$, which means we also know the zeros of its reciprocal $\frac{1}{\tan(\theta)}$. \\










\begin{center}
\textbf{\textcolor{green!50!black}{ooooo-=-=-=-ooOoo-=-=-=-ooooo}} \\

more examples can be found by following this link\\ \link[More Examples of Trigonometric Functions]{https://ximera.osu.edu/csccmathematics/precalculus/precalculus/moreTrigFunctions/examples/exampleList}

\end{center}







\end{document}
