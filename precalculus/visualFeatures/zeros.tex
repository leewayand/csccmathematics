\documentclass{ximera}


\graphicspath{
  {./}
  {ximeraTutorial/}
  {basicPhilosophy/}
}

\newcommand{\mooculus}{\textsf{\textbf{MOOC}\textnormal{\textsf{ULUS}}}}


\usepackage{tkz-euclide}\usepackage{tikz}
\usepackage{tikz-cd}
\usetikzlibrary{arrows}
\tikzset{>=stealth,commutative diagrams/.cd,
  arrow style=tikz,diagrams={>=stealth}} %% cool arrow head
\tikzset{shorten <>/.style={ shorten >=#1, shorten <=#1 } } %% allows shorter vectors

\usetikzlibrary{backgrounds} %% for boxes around graphs
\usetikzlibrary{shapes,positioning}  %% Clouds and stars
\usetikzlibrary{matrix} %% for matrix
\usepgfplotslibrary{polar} %% for polar plots
\usepgfplotslibrary{fillbetween} %% to shade area between curves in TikZ
\usetkzobj{all}
\usepackage[makeroom]{cancel} %% for strike outs
%\usepackage{mathtools} %% for pretty underbrace % Breaks Ximera
%\usepackage{multicol}
\usepackage{pgffor} %% required for integral for loops



%% http://tex.stackexchange.com/questions/66490/drawing-a-tikz-arc-specifying-the-center
%% Draws beach ball
\tikzset{pics/carc/.style args={#1:#2:#3}{code={\draw[pic actions] (#1:#3) arc(#1:#2:#3);}}}



\usepackage{array}
\setlength{\extrarowheight}{+.1cm}
\newdimen\digitwidth
\settowidth\digitwidth{9}
\def\divrule#1#2{
\noalign{\moveright#1\digitwidth
\vbox{\hrule width#2\digitwidth}}}
























%%This is to help with formatting on future title pages.
\newenvironment{sectionOutcomes}{}{}


\title{Zeros}

\begin{document}

\begin{abstract}
intercepts
\end{abstract}
\maketitle



Let $f$ be a function with its natural domain. 

As we saw before, a domain number $b$ is a \textbf{\textcolor{blue!55!black}{zero}} of $f$, if $f(b) = 0$. 


The point corresponding to the zero $b$ is $(b, f(b)) = (b, 0)$, an intercept on the graph. 






Here is the complete graph of the function $G(x)$. 


\begin{image}
\begin{tikzpicture}
     \begin{axis}[
               domain=-10:10, ymax=10, xmax=10, ymin=-10, xmin=-10,
               axis lines =center, xlabel=$x$, ylabel=$y$,
                ytick={-10,-8,-6,-4,-2,2,4,6,8,10},
                xtick={-10,-8,-6,-4,-2,2,4,6,8,10},
                yticklabels={$-10$,$-8$,$-6$,$-4$,$-2$,$2$,$4$,$6$,$8$,$10$}, 
                xticklabels={$-10$,$-8$,$-6$,$-4$,$-2$,$2$,$4$,$6$,$8$,$10$},
                ticklabel style={font=\scriptsize},
               every axis y label/.style={at=(current axis.above origin),anchor=south},
               every axis x label/.style={at=(current axis.right of origin),anchor=west},
               axis on top,
                    ]

        
        \addplot [draw=penColor, very thick, smooth, domain=(-6:3), <->] {1/(x-3) + 2};
        \addplot [draw=penColor, very thick, smooth, domain=(3:8), ->] {1/(x-3) + 2};

        \addplot [line width=0.5, gray, dashed,samples=100,domain=(-9:9)] ({3},{x});
        \addplot [line width=0.5, gray, dashed,samples=100,domain=(-9:9)] ({x},{2});

        \addplot[color=penColor,only marks,mark=*] coordinates{(3.2,7)}; 
        %\addplot[color=penColor,only marks,mark=*] coordinates{(8,2.2)}; 


    \end{axis}



\end{tikzpicture}
\end{image}


The graph has one intercept, which means that $G$ has one zero, which appears to be $2.4$. 

\[  G(2.4) = 0  \]




The graph has another intercept: $(0, 1.7)$. This tells us that $G(0)=1.7$.  However, this is not really useful information in function analysis. 

$(0, 1.7)$ might be useful information if the function was modeling some situation, so that the domain was an applied domain.



If this function was a model for some timed event, then perhaps $t = 0$, would be important to interpret back to the situation, as initial information.  That would be left for the interpretation of the model. 


When analyzing a function, we are mostly interested in its zeros, which correspond to intercepts on the horizontal axis.





















\begin{example} Zeros








Here is the complete graph of the function $G(x)$. 


\begin{image}
\begin{tikzpicture}
     \begin{axis}[
               domain=-10:10, ymax=10, xmax=10, ymin=-10, xmin=-10,
               axis lines =center, xlabel=$x$, ylabel=$y$,
                ytick={-10,-8,-6,-4,-2,2,4,6,8,10},
                xtick={-10,-8,-6,-4,-2,2,4,6,8,10},
                yticklabels={$-10$,$-8$,$-6$,$-4$,$-2$,$2$,$4$,$6$,$8$,$10$}, 
                xticklabels={$-10$,$-8$,$-6$,$-4$,$-2$,$2$,$4$,$6$,$8$,$10$},
                ticklabel style={font=\scriptsize},
               every axis y label/.style={at=(current axis.above origin),anchor=south},
               every axis x label/.style={at=(current axis.right of origin),anchor=west},
               axis on top,
                    ]

        
        \addplot [draw=penColor, very thick, smooth, domain=(-6:3), <->] {1/(x-3) + 2};
        \addplot [draw=penColor, very thick, smooth, domain=(3:8), ->] {1/(x-3) + 2};

        \addplot [line width=0.5, gray, dashed,samples=100,domain=(-9:9)] ({3},{x});
        \addplot [line width=0.5, gray, dashed,samples=100,domain=(-9:9)] ({x},{2});

        \addplot[color=penColor,only marks,mark=*] coordinates{(3.2,7)}; 
        %\addplot[color=penColor,only marks,mark=*] coordinates{(8,2.2)}; 

        \addplot[color=penColor,only marks,mark=*] coordinates{(7,0)};
        \addplot[color=penColor,fill=white, only marks,mark=*] coordinates{(7,2.5)};


    \end{axis}



\end{tikzpicture}
\end{image}
The graph appears to have two intercepts, which means the function $g$ has two zeros.




\begin{itemize}
\item Intercepts of graph: $(3,0)$ and $(7,0)$.
\item Zeros of $g$: $3$ and $7$.
\end{itemize}



\textbf{Note:} The intercepts are points on the graph.  They are not zeros of the function.  Function zeros are domain numbers.  The intercepts on the graph visually encode a function zero as the first coordinate of the intercept.



\textbf{Note:} The $y$-intercept has nothing to do with function zeros.


\end{example}


























\begin{example} Zeros





Let $P(w)$ be a function.  The graph of $y = P(w)$ is displayed below. 

\begin{image}
\begin{tikzpicture}
     \begin{axis}[
               domain=-10:10, ymax=10, xmax=10, ymin=-10, xmin=-10,
               axis lines =center, xlabel=$w$, ylabel=$y$,
                ytick={-10,-8,-6,-4,-2,2,4,6,8,10},
                xtick={-10,-8,-6,-4,-2,2,4,6,8,10},
                yticklabels={$-10$,$-8$,$-6$,$-4$,$-2$,$2$,$4$,$6$,$8$,$10$}, 
                xticklabels={$-10$,$-8$,$-6$,$-4$,$-2$,$2$,$4$,$6$,$8$,$10$},
                ticklabel style={font=\scriptsize},
               every axis y label/.style={at=(current axis.above origin),anchor=south},
               every axis x label/.style={at=(current axis.right of origin),anchor=west},
               axis on top,
                    ]

        
        \addplot [draw=penColor, very thick, smooth, domain=(-8:-4)] {-(x+8)*(x+5)};
        \addplot [draw=penColor, very thick, smooth, domain=(-4:4)] {sin(deg(2*x)) + 4};
        \addplot [draw=penColor, very thick, smooth, domain=(4:8)] {-2*x+10};


        \addplot[color=penColor,fill=white,only marks,mark=*] coordinates{(-8,0)};
        \addplot[color=penColor,fill=white,only marks,mark=*] coordinates{(-4,-4)};

        \addplot[color=penColor,fill=penColor,only marks,mark=*] coordinates{(-4,3.01) (-8,4)};
        \addplot[color=penColor,fill=white,only marks,mark=*] coordinates{(4,4.989)};

        \addplot[color=penColor,fill=white,only marks,mark=*] coordinates{(4,2)};
        \addplot[color=penColor,fill=white,only marks,mark=*] coordinates{(8,-6)};

    \end{axis}
\end{tikzpicture}
\end{image}

The graph appears to have two intercepts, not three. This means the function $P$ has two zeros.




\begin{itemize}
\item Intercepts of graph: $(-5,0)$ and $(5,0)$.
\item Zeros of $P$: $-5$ and $5$.
\end{itemize}



\textbf{Note:} The intercepts are points on the graph.  They are not zeros of the function.  Function zeros are domain numbers.  The intercepts on the graph visually encode a function zero as the first coordinate of the intercept.

\textbf{Note:} $-8$ is in the domain. However, there is an open/hollow dot at $(-8,0)$. which means it is not there.  The dot for $-8$ is $(-8,4)$.  $-8$ is not a zero of $P$.


\textbf{Note:} The $y$-intercept has nothing to do with function zeros.


\end{example}






























\begin{example} Hidden Zeros





Let $K(z)$ be a function.  The graph of $y = K(z)$ is displayed below. 

\begin{image}
\begin{tikzpicture}
     \begin{axis}[
               domain=-10:10, ymax=10, xmax=10, ymin=-10, xmin=-10,
               axis lines =center, xlabel=$z$, ylabel=$y$,
                ytick={-10,-8,-6,-4,-2,2,4,6,8,10},
                xtick={-10,-8,-6,-4,-2,2,4,6,8,10},
                yticklabels={$-10$,$-8$,$-6$,$-4$,$-2$,$2$,$4$,$6$,$8$,$10$}, 
                xticklabels={$-10$,$-8$,$-6$,$-4$,$-2$,$2$,$4$,$6$,$8$,$10$},
                ticklabel style={font=\scriptsize},
               every axis y label/.style={at=(current axis.above origin),anchor=south},
               every axis x label/.style={at=(current axis.right of origin),anchor=west},
               axis on top,
                    ]

        
        \addplot [draw=penColor, very thick, smooth, domain=(-8:-4),->] {-(x+8)*(x+5)+5};
        %\addplot [draw=penColor, very thick, smooth, domain=(-4:4)] {sin(deg(2*x)) + 4};
        %\addplot [draw=penColor, very thick, smooth, domain=(4:6),->] {-2*x+15};


        \addplot[color=penColor,fill=white,only marks,mark=*] coordinates{(-8,5)};
        %\addplot[color=penColor,fill=white,only marks,mark=*] coordinates{(-4,-4)};


    \end{axis}
\end{tikzpicture}
\end{image}

The graph appears to have no intercepts.  That is because the whole graph is not drawn.  The arrow tells us that the graph continues to go down to the right.

That means there is an intercept around $(-3.8,0)$.


$-3.8$ is the only zero of $K(z)$.


\end{example}



















\begin{example} Hidden Zeros





Let $S(\theta)$ be a function.  The graph of $y = S(\theta)$ is displayed below. 

\begin{image}
\begin{tikzpicture}
     \begin{axis}[
               domain=-10:10, ymax=10, xmax=10, ymin=-10, xmin=-10,
               axis lines =center, xlabel=$\theta$, ylabel=$y$,
                ytick={-10,-8,-6,-4,-2,2,4,6,8,10},
                xtick={-10,-8,-6,-4,-2,2,4,6,8,10},
                yticklabels={$-10$,$-8$,$-6$,$-4$,$-2$,$2$,$4$,$6$,$8$,$10$}, 
                xticklabels={$-10$,$-8$,$-6$,$-4$,$-2$,$2$,$4$,$6$,$8$,$10$},
                ticklabel style={font=\scriptsize},
               every axis y label/.style={at=(current axis.above origin),anchor=south},
               every axis x label/.style={at=(current axis.right of origin),anchor=west},
               axis on top,
                    ]

        
        %\addplot [draw=penColor, very thick, smooth, domain=(-8:-4),->] {-(x+8)*(x+5)+5};
        \addplot [draw=penColor, very thick, smooth, domain=(-8:8),<->] {3*sin(deg(x*1.57))};
        %\addplot [draw=penColor, very thick, smooth, domain=(4:6),->] {-2*x+15};


        %\addplot[color=penColor,fill=white,only marks,mark=*] coordinates{(-8,5)};
        %\addplot[color=penColor,fill=white,only marks,mark=*] coordinates{(-4,-4)};


    \end{axis}
\end{tikzpicture}
\end{image}

The graph appears to have an infinite number of intercepts.  

They occur at even integer.



The zeros of $S(\theta)$ are


\[
\theta \in \{ 2k \, \text{ | } \,  k \in \mathbb{Z}    \}
\]


\end{example}

























\begin{onlineOnly}
\begin{center}
\textbf{\textcolor{green!50!black}{ooooo-=-=-=-ooOoo-=-=-=-ooooo}} \\

more examples can be found by following this link\\ \link[More Examples of Visual Features]{https://ximera.osu.edu/csccmathematics/precalculus/precalculus/visualFeatures/examples/exampleList}

\end{center}
\end{onlineOnly}











\end{document}
