\documentclass{ximera}


\graphicspath{
  {./}
  {ximeraTutorial/}
  {basicPhilosophy/}
}

\newcommand{\mooculus}{\textsf{\textbf{MOOC}\textnormal{\textsf{ULUS}}}}


\usepackage{tkz-euclide}\usepackage{tikz}
\usepackage{tikz-cd}
\usetikzlibrary{arrows}
\tikzset{>=stealth,commutative diagrams/.cd,
  arrow style=tikz,diagrams={>=stealth}} %% cool arrow head
\tikzset{shorten <>/.style={ shorten >=#1, shorten <=#1 } } %% allows shorter vectors

\usetikzlibrary{backgrounds} %% for boxes around graphs
\usetikzlibrary{shapes,positioning}  %% Clouds and stars
\usetikzlibrary{matrix} %% for matrix
\usepgfplotslibrary{polar} %% for polar plots
\usepgfplotslibrary{fillbetween} %% to shade area between curves in TikZ
\usetkzobj{all}
\usepackage[makeroom]{cancel} %% for strike outs
%\usepackage{mathtools} %% for pretty underbrace % Breaks Ximera
%\usepackage{multicol}
\usepackage{pgffor} %% required for integral for loops



%% http://tex.stackexchange.com/questions/66490/drawing-a-tikz-arc-specifying-the-center
%% Draws beach ball
\tikzset{pics/carc/.style args={#1:#2:#3}{code={\draw[pic actions] (#1:#3) arc(#1:#2:#3);}}}



\usepackage{array}
\setlength{\extrarowheight}{+.1cm}
\newdimen\digitwidth
\settowidth\digitwidth{9}
\def\divrule#1#2{
\noalign{\moveright#1\digitwidth
\vbox{\hrule width#2\digitwidth}}}
























%%This is to help with formatting on future title pages.
\newenvironment{sectionOutcomes}{}{}


\title{End-Behavior}

\begin{document}

\begin{abstract}
settling down
\end{abstract}
\maketitle


$(-\infty, \infty)$ is very long. \\


When we think of analyzing functions, we envision three pieces of the real line. \\





There is the \textbf{\textcolor{red!80!black}{left tail}}, which is the part with very very very very large negative numbers. There is the \textbf{\textcolor{red!80!black}{right tail}}, which is the part with very very very very large positive numbers. Then, there is the \textbf{\textcolor{purple!85!blue}{middle}} part. \\

All of the individual characteristics of a function happen in the middle. \\

That is, each function has its own middle bubble (subset of the domain) where all of the important locations live.  The middle bubble contains the zeros, the discontinuities, the singularities, the critical numbers, the maximums, the minimums, the everything.  Once we are outside the middle part of the domain, the function settles down into a consistent pattern of behavior and maintains that predictable behavior all the way to $-\infty$ or $\infty$ (in the domain). \\

This settled down pattern in the \textbf{\textcolor{red!80!black}{tails}} is called the \textbf{\textcolor{red!80!black}{end-behavior}} of the function. \\


Graphs use graphical symbols to help describe this end-behavior.\\


Here is the complete graph of the function $G(x)$. \\


\begin{image}
\begin{tikzpicture}
     \begin{axis}[
               domain=-10:10, ymax=10, xmax=10, ymin=-10, xmin=-10,
               axis lines =center, xlabel=$x$, ylabel=$y$,
                ytick={-10,-8,-6,-4,-2,2,4,6,8,10},
                xtick={-10,-8,-6,-4,-2,2,4,6,8,10},
                yticklabels={$-10$,$-8$,$-6$,$-4$,$-2$,$2$,$4$,$6$,$8$,$10$}, 
                xticklabels={$-10$,$-8$,$-6$,$-4$,$-2$,$2$,$4$,$6$,$8$,$10$},
                ticklabel style={font=\scriptsize},
               every axis y label/.style={at=(current axis.above origin),anchor=south},
               every axis x label/.style={at=(current axis.right of origin),anchor=west},
               axis on top,
                    ]

        
        \addplot [draw=penColor, very thick, smooth, domain=(-6:3), <->] {1/(x-3) + 2};
        \addplot [draw=penColor, very thick, smooth, domain=(3:8), ->] {1/(x-3) + 2};

        \addplot [line width=0.5, gray, dashed,samples=100,domain=(-9:9)] ({3},{x});
        \addplot [line width=0.5, gray, dashed,samples=100,domain=(-9:9)] ({x},{2});

        \addplot[color=penColor,only marks,mark=*] coordinates{(3.2,7)}; 
        %\addplot[color=penColor,only marks,mark=*] coordinates{(8,2.2)}; 


    \end{axis}



\end{tikzpicture}
\end{image}



In the graph above, we see a vertical asymptote (singularity).  This is inside the middle bubble.  It is not a part of end-behavior. \\

In the graph above, we see an intercept (zero).  This is inside the middle bubble.  It is not a part of end-behavior. \\

The horizontal asymptote is communicating end-behavior.  It tells us that as we move further out in the domain, on either side, the function approaches the value $2$.




We also have algebraic notation that describes the end-behavior.


\[
\lim\limits_{x \to -\infty} G(x) = 2
\]



\[
\lim\limits_{x \to \infty} G(x) = 2
\]



\begin{notation} Limits


``lim'' is shorthand for limit.  


Underneath ``lim'' we show which tail we are talking about. 


$\blacktriangleright$ If we are talking about the left tail then we write $variable \to -\infty$.

$\blacktriangleright$ If we are talking about the right tail then we write $variable \to \infty$.

All of that fits underneath ``lim''.



The function name is written next, then $=$, then the expected value is written.



The expected value could be 

\begin{itemize}
\item a number
\item $\infty$, indicating that the funciton is unbounded positively
\item $-\infty$, indicating that the funciton is unbounded negatively
\item DNE, indicating that the function is doing none of those
\end{itemize}


\end{notation}












Here is the complete graph of the function $H(t)$. \\


\begin{image}
\begin{tikzpicture}
     \begin{axis}[
            	domain=-10:10, ymax=10, xmax=10, ymin=-10, xmin=-10,
            	axis lines =center, xlabel=$t$, ylabel=${y=H(t)}$,
                ytick={-10,-8,-6,-4,-2,2,4,6,8,10},
                xtick={-10,-8,-6,-4,-2,2,4,6,8,10},
                yticklabels={$-10$,$-8$,$-6$,$-4$,$-2$,$2$,$4$,$6$,$8$,$10$}, 
                xticklabels={$-10$,$-8$,$-6$,$-4$,$-2$,$2$,$4$,$6$,$8$,$10$},
                ticklabel style={font=\scriptsize},
            	every axis y label/.style={at=(current axis.above origin),anchor=south},
            	every axis x label/.style={at=(current axis.right of origin),anchor=west},
            	axis on top,
          		]

        
        \addplot [draw=penColor, very thick, smooth, domain=(-6:3), <->] {1/(x-3) + 2};
        \addplot [draw=penColor, very thick, smooth, domain=(3:8),->] {1/(x-3) + 2};

        \addplot [line width=0.5, gray, dashed,samples=100,domain=(-9:9)] ({3},{x});
        %\addplot [line width=0.5, gray, dashed,samples=100,domain=(-9:9)] ({x},{2});

        \addplot[color=penColor,only marks,mark=*] coordinates{(3.2,7)}; 
        %\addplot[color=penColor,only marks,mark=*] coordinates{(8,2.2)}; 


    \end{axis}



\end{tikzpicture}
\end{image}



The graph of $H$ does not have horizontal asymptote.  That is meaningful.  The fact that the horizontal asymptote is not included in the drawing tells us that the function does not settle down, getting closer to $2$.  Otherwise, there would have been a horizontal asymptote. (This is an example of a missing symbol communicating meaning.)\\

From this, we deduce that as we move out in the domain towards $\infty$ (the right tail), the value of $H$ grows bigger and bigger - unbounded negatively. \\

We also deduce that as we move out in the domain towards $-\infty$ (the left tail), the value of $H$ grows bigger and bigger - unbounded positively. \\


The graphs suggests that this unbounded growth might be very slow, but $H$ does become unbounded.





\[
\lim\limits_{t \to -\infty} H(t) = \infty
\]



\[
\lim\limits_{t \to \infty} H(t) = -\infty
\]




\begin{warning} Bigger and Bigger


Bigger and bigger does not mean the limit is $\infty$.


For example,  $E(x) = 7 - e^{-x}$ is an increasing function.  Its values get bigger and bigger as $x$ moves out through the right tail. But, it does not become unbounded. Its values get closer and closer to $7$.





\[
\lim\limits_{x \to \infty} E(x) = 7
\]



\end{warning}







\begin{example} Limiting Behavior

Here is a complete graph of $K(v)$. \\


\begin{image}
\begin{tikzpicture}
     \begin{axis}[
                domain=-10:10, ymax=10, xmax=10, ymin=-10, xmin=-10,
                axis lines =center, xlabel=$v$, ylabel=${y=K(v)}$,
                ytick={-10,-8,-6,-4,-2,2,4,6,8,10},
                xtick={-10,-8,-6,-4,-2,2,4,6,8,10},
                yticklabels={$-10$,$-8$,$-6$,$-4$,$-2$,$2$,$4$,$6$,$8$,$10$}, 
                xticklabels={$-10$,$-8$,$-6$,$-4$,$-2$,$2$,$4$,$6$,$8$,$10$},
                ticklabel style={font=\scriptsize},
                every axis y label/.style={at=(current axis.above origin),anchor=south},
                every axis x label/.style={at=(current axis.right of origin),anchor=west},
                axis on top,
                ]

        
        \addplot [draw=penColor, very thick, smooth, domain=(-6:10), ->] {10*sin(deg(x))/((x+10)^1.2) + e^(-x-5) + 2};
        %\addplot [draw=penColor, very thick, smooth, domain=(3:8), <->] {1/(x-3) + 2};

        %\addplot [line width=1, gray, dashed,samples=100,domain=(-9:9)] ({3},{x});
        \addplot [line width=0.5, gray, dashed,samples=100,domain=(1:9)] ({x},{2});
        \addplot[color=penColor,only marks,mark=*] coordinates{(-6,5.25)}; 


    \end{axis}



\end{tikzpicture}
\end{image}




The domain of the function $K$ is $[-6, \infty)$. Therefore, it only has one end-behavior.  As we move out in the domain towards $\infty$, $K$ approaches $2$.  The value of $K$ continues to oscillate above and below $2$ as it moves cloaser and closer to the value of $2$. 





\[
\lim\limits_{v \to \infty} K(v) = 2
\]





$K$ does not have an end-behavior as the domain moves towards $-\infty$, because the domain doesn't move towards $-\infty$ (no left tail).





\[
\lim\limits_{v \to \-infty} K(v) = DNE
\]





\end{example}















\begin{example} No End-Behavior

Here is a complete graph of $p(x)$. \\


\begin{image}
\begin{tikzpicture}
     \begin{axis}[
                domain=-10:10, ymax=10, xmax=10, ymin=-10, xmin=-10,
                axis lines =center, xlabel=$x$, ylabel=${y=p(x)}$,
                ytick={-10,-8,-6,-4,-2,2,4,6,8,10},
                xtick={-10,-8,-6,-4,-2,2,4,6,8,10},
                yticklabels={$-10$,$-8$,$-6$,$-4$,$-2$,$2$,$4$,$6$,$8$,$10$}, 
                xticklabels={$-10$,$-8$,$-6$,$-4$,$-2$,$2$,$4$,$6$,$8$,$10$},
                ticklabel style={font=\scriptsize},
                every axis y label/.style={at=(current axis.above origin),anchor=south},
                every axis x label/.style={at=(current axis.right of origin),anchor=west},
                axis on top,
                ]

        
        \addplot [draw=penColor, very thick, smooth, domain=(-6:10)] {10*sin(deg(x))/((x+10)^1.2) + e^(-x-5) + 2};
        %\addplot [draw=penColor, very thick, smooth, domain=(3:8), <->] {1/(x-3) + 2};

        %\addplot [line width=1, gray, dashed,samples=100,domain=(-9:9)] ({3},{x});
        %\addplot [line width=0.5, gray, dashed,samples=100,domain=(1:9)] ({x},{2});
        \addplot[color=penColor,only marks,mark=*] coordinates{(-6,5.25)}; 
        \addplot[color=penColor,only marks,mark=*] coordinates{(10,1.8)}; 


    \end{axis}



\end{tikzpicture}
\end{image}




The domain of the function $p$ is $[-6, 10]$. Therefore, it has no end-behavior.  The domain doesn't move towards $-\infty$ or $\infty$. The domain is bounded.

\end{example}





























\subsection*{Language for End-Behavior}


We have language, symbols, and notation for end-behavior. \\



\textbf{\textcolor{blue!55!black}{$\blacktriangleright$ Graphical Language}} 


We use graphic symbols on our graphs to communicate to readers how the function behaves in the two tails of the domain.  These are horizontal asymptotes and are drawn as dashed horizontal lines on the graph.\\

If there is a horizontal dashed line, that is telling the reader that the function's values will eventually get close to a constant value and stay there. \\

If there is no horizontal dashed asymptote, then the function's values \textbf{do not} eventually get close to a constant value and stay there. It doesn't matter what is ``looks like''.

The horizontal asymptotes tells the reader something and its absence also tells the reader something. \\








\textbf{\textcolor{blue!55!black}{$\blacktriangleright$ Algebraic Language}} 


We use algebraic symbols/notation to communicate the same information to reader.  This algebraic language is called \textbf{limits}. \\


When the graph $y = f(x)$ of the function $f$ includes the horizontal asymptote $y = y_0$ in the right tail of the graph, that is translated as


\[
\lim\limits_{x \to \infty}f(x) = y_0
\]





When the graph $y = f(x)$ of the function $f$ includes the horizontal asymptote $y = y_0$ in the left tail of the graph, that is translated as


\[
\lim\limits_{x \to -\infty}f(x) = y_0
\]





\begin{example}

Here is the complete graph of the function $R(m)$.





\begin{image}
\begin{tikzpicture}
     \begin{axis}[
               domain=-10:10, ymax=10, xmax=10, ymin=-10, xmin=-10,
               axis lines =center, xlabel=$m$, ylabel=$y$,
                ytick={-10,-8,-6,-4,-2,2,4,6,8,10},
                xtick={-10,-8,-6,-4,-2,2,4,6,8,10},
                yticklabels={$-10$,$-8$,$-6$,$-4$,$-2$,$2$,$4$,$6$,$8$,$10$}, 
                xticklabels={$-10$,$-8$,$-6$,$-4$,$-2$,$2$,$4$,$6$,$8$,$10$},
                ticklabel style={font=\scriptsize},
               every axis y label/.style={at=(current axis.above origin),anchor=south},
               every axis x label/.style={at=(current axis.right of origin),anchor=west},
               axis on top,
                    ]

        
        \addplot [draw=penColor, very thick, smooth, domain=(-9:9), <->] {(3-4*(2.7^(-x)))/(1+2.7^(-x))};
        

        %\addplot [line width=0.5, gray, dashed,samples=100,domain=(-9:9)] ({3},{x});
        \addplot [line width=0.5, gray, dashed,samples=100,domain=(-9:9)] ({x},{2});
        \addplot [line width=0.5, gray, dashed,samples=100,domain=(-9:9)] ({x},{2});

        %\addplot[color=penColor,only marks,mark=*] coordinates{(3.2,7)}; 
        %\addplot[color=penColor,only marks,mark=*] coordinates{(8,2.2)}; 


    \end{axis}



\end{tikzpicture}
\end{image}


The horizontal asymptotes graphically give the same information as 



\[
\lim\limits_{m \to \infty}R(m) = 3
\]



\[
\lim\limits_{m \to -\infty}R(m) = -4
\]



\end{example}
























\begin{example}



Here is the complete graph of the function $W(p)$. \\


\begin{image}
\begin{tikzpicture}
     \begin{axis}[
               domain=-10:10, ymax=10, xmax=10, ymin=-10, xmin=-10,
               axis lines =center, xlabel=$p$, ylabel=${y=W(p)}$,
                ytick={-10,-8,-6,-4,-2,2,4,6,8,10},
                xtick={-10,-8,-6,-4,-2,2,4,6,8,10},
                yticklabels={$-10$,$-8$,$-6$,$-4$,$-2$,$2$,$4$,$6$,$8$,$10$}, 
                xticklabels={$-10$,$-8$,$-6$,$-4$,$-2$,$2$,$4$,$6$,$8$,$10$},
                ticklabel style={font=\scriptsize},
               every axis y label/.style={at=(current axis.above origin),anchor=south},
               every axis x label/.style={at=(current axis.right of origin),anchor=west},
               axis on top,
                    ]

        
        \addplot [draw=penColor, very thick, smooth, domain=(-9:9), <->] {1+4*cos(deg(x))};
        %\addplot [draw=penColor, very thick, smooth, domain=(3:8),->] {1/(x-3) + 2};

        %\addplot [line width=0.5, gray, dashed,samples=100,domain=(-9:9)] ({3},{x});
        %\addplot [line width=0.5, gray, dashed,samples=100,domain=(-9:9)] ({x},{2});

        %\addplot[color=penColor,only marks,mark=*] coordinates{(3.2,7)}; 
        %\addplot[color=penColor,only marks,mark=*] coordinates{(8,2.2)}; 


    \end{axis}



\end{tikzpicture}
\end{image}



This function does not become unbounded.  It does not settle down and approach a constant value. Instead, it continues to oscillate through both tails



\[
\lim\limits_{p \to \infty}W(p) = DNE
\]



\[
\lim\limits_{p \to -\infty}W(p) = DNE
\]







\end{example}













\begin{remark} Discontinuties and Singularities


Graphs have vertical asymptotes, just like they have horizontal asymptotes.

Graphs also have holes and jumps.


All of these involve expected values, which do not agree with actual values.


We would like to describe these discrepencies.


Limits will be our language for these as well.



\end{remark}













































\begin{onlineOnly}
\begin{center}
\textbf{\textcolor{green!50!black}{ooooo-=-=-=-ooOoo-=-=-=-ooooo}} \\

more examples can be found by following this link\\ \link[More Examples of Visual Features]{https://ximera.osu.edu/csccmathematics/precalculus/precalculus/visualFeatures/examples/exampleList}

\end{center}
\end{onlineOnly}









\end{document}
