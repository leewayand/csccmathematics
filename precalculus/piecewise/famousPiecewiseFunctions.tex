\documentclass{ximera}

%\usepackage{todonotes}

\newcommand{\todo}{}

\usepackage{esint} % for \oiint
\ifxake%%https://math.meta.stackexchange.com/questions/9973/how-do-you-render-a-closed-surface-double-integral
\renewcommand{\oiint}{{\large\bigcirc}\kern-1.56em\iint}
\fi


\graphicspath{
  {./}
  {ximeraTutorial/}
  {basicPhilosophy/}
  {functionsOfSeveralVariables/}
  {normalVectors/}
  {lagrangeMultipliers/}
  {vectorFields/}
  {greensTheorem/}
  {shapeOfThingsToCome/}
  {dotProducts/}
  {partialDerivativesAndTheGradientVector/}
  {../productAndQuotientRules/exercises/}
  {../normalVectors/exercisesParametricPlots/}
  {../continuityOfFunctionsOfSeveralVariables/exercises/}
  {../partialDerivativesAndTheGradientVector/exercises/}
  {../directionalDerivativeAndChainRule/exercises/}
  {../commonCoordinates/exercisesCylindricalCoordinates/}
  {../commonCoordinates/exercisesSphericalCoordinates/}
  {../greensTheorem/exercisesCurlAndLineIntegrals/}
  {../greensTheorem/exercisesDivergenceAndLineIntegrals/}
  {../shapeOfThingsToCome/exercisesDivergenceTheorem/}
  {../greensTheorem/}
  {../shapeOfThingsToCome/}
  {../separableDifferentialEquations/exercises/}
  {vectorFields/}
}

\newcommand{\mooculus}{\textsf{\textbf{MOOC}\textnormal{\textsf{ULUS}}}}

\usepackage{tkz-euclide}
\usepackage{tikz}
\usepackage{tikz-cd}
\usetikzlibrary{arrows}
\tikzset{>=stealth,commutative diagrams/.cd,
  arrow style=tikz,diagrams={>=stealth}} %% cool arrow head
\tikzset{shorten <>/.style={ shorten >=#1, shorten <=#1 } } %% allows shorter vectors

\usetikzlibrary{backgrounds} %% for boxes around graphs
\usetikzlibrary{shapes,positioning}  %% Clouds and stars
\usetikzlibrary{matrix} %% for matrix
\usepgfplotslibrary{polar} %% for polar plots
\usepgfplotslibrary{fillbetween} %% to shade area between curves in TikZ
%\usetkzobj{all}
\usepackage[makeroom]{cancel} %% for strike outs
%\usepackage{mathtools} %% for pretty underbrace % Breaks Ximera
%\usepackage{multicol}
\usepackage{pgffor} %% required for integral for loops



%% http://tex.stackexchange.com/questions/66490/drawing-a-tikz-arc-specifying-the-center
%% Draws beach ball
\tikzset{pics/carc/.style args={#1:#2:#3}{code={\draw[pic actions] (#1:#3) arc(#1:#2:#3);}}}



\usepackage{array}
\setlength{\extrarowheight}{+.1cm}
\newdimen\digitwidth
\settowidth\digitwidth{9}
\def\divrule#1#2{
\noalign{\moveright#1\digitwidth
\vbox{\hrule width#2\digitwidth}}}




% \newcommand{\RR}{\mathbb R}
% \newcommand{\R}{\mathbb R}
% \newcommand{\N}{\mathbb N}
% \newcommand{\Z}{\mathbb Z}

\newcommand{\sagemath}{\textsf{SageMath}}


%\renewcommand{\d}{\,d\!}
%\renewcommand{\d}{\mathop{}\!d}
%\newcommand{\dd}[2][]{\frac{\d #1}{\d #2}}
%\newcommand{\pp}[2][]{\frac{\partial #1}{\partial #2}}
% \renewcommand{\l}{\ell}
%\newcommand{\ddx}{\frac{d}{\d x}}

% \newcommand{\zeroOverZero}{\ensuremath{\boldsymbol{\tfrac{0}{0}}}}
%\newcommand{\inftyOverInfty}{\ensuremath{\boldsymbol{\tfrac{\infty}{\infty}}}}
%\newcommand{\zeroOverInfty}{\ensuremath{\boldsymbol{\tfrac{0}{\infty}}}}
%\newcommand{\zeroTimesInfty}{\ensuremath{\small\boldsymbol{0\cdot \infty}}}
%\newcommand{\inftyMinusInfty}{\ensuremath{\small\boldsymbol{\infty - \infty}}}
%\newcommand{\oneToInfty}{\ensuremath{\boldsymbol{1^\infty}}}
%\newcommand{\zeroToZero}{\ensuremath{\boldsymbol{0^0}}}
%\newcommand{\inftyToZero}{\ensuremath{\boldsymbol{\infty^0}}}



% \newcommand{\numOverZero}{\ensuremath{\boldsymbol{\tfrac{\#}{0}}}}
% \newcommand{\dfn}{\textbf}
% \newcommand{\unit}{\,\mathrm}
% \newcommand{\unit}{\mathop{}\!\mathrm}
% \newcommand{\eval}[1]{\bigg[ #1 \bigg]}
% \newcommand{\seq}[1]{\left( #1 \right)}
% \renewcommand{\epsilon}{\varepsilon}
% \renewcommand{\phi}{\varphi}


% \renewcommand{\iff}{\Leftrightarrow}

% \DeclareMathOperator{\arccot}{arccot}
% \DeclareMathOperator{\arcsec}{arcsec}
% \DeclareMathOperator{\arccsc}{arccsc}
% \DeclareMathOperator{\si}{Si}
% \DeclareMathOperator{\scal}{scal}
% \DeclareMathOperator{\sign}{sign}


%% \newcommand{\tightoverset}[2]{% for arrow vec
%%   \mathop{#2}\limits^{\vbox to -.5ex{\kern-0.75ex\hbox{$#1$}\vss}}}
% \newcommand{\arrowvec}[1]{{\overset{\rightharpoonup}{#1}}}
% \renewcommand{\vec}[1]{\arrowvec{\mathbf{#1}}}
% \renewcommand{\vec}[1]{{\overset{\boldsymbol{\rightharpoonup}}{\mathbf{#1}}}}

% \newcommand{\point}[1]{\left(#1\right)} %this allows \vector{ to be changed to \vector{ with a quick find and replace
% \newcommand{\pt}[1]{\mathbf{#1}} %this allows \vec{ to be changed to \vec{ with a quick find and replace
% \newcommand{\Lim}[2]{\lim_{\point{#1} \to \point{#2}}} %Bart, I changed this to point since I want to use it.  It runs through both of the exercise and exerciseE files in limits section, which is why it was in each document to start with.

% \DeclareMathOperator{\proj}{\mathbf{proj}}
% \newcommand{\veci}{{\boldsymbol{\hat{\imath}}}}
% \newcommand{\vecj}{{\boldsymbol{\hat{\jmath}}}}
% \newcommand{\veck}{{\boldsymbol{\hat{k}}}}
% \newcommand{\vecl}{\vec{\boldsymbol{\l}}}
% \newcommand{\uvec}[1]{\mathbf{\hat{#1}}}
% \newcommand{\utan}{\mathbf{\hat{t}}}
% \newcommand{\unormal}{\mathbf{\hat{n}}}
% \newcommand{\ubinormal}{\mathbf{\hat{b}}}

% \newcommand{\dotp}{\bullet}
% \newcommand{\cross}{\boldsymbol\times}
% \newcommand{\grad}{\boldsymbol\nabla}
% \newcommand{\divergence}{\grad\dotp}
% \newcommand{\curl}{\grad\cross}
%\DeclareMathOperator{\divergence}{divergence}
%\DeclareMathOperator{\curl}[1]{\grad\cross #1}
% \newcommand{\lto}{\mathop{\longrightarrow\,}\limits}

% \renewcommand{\bar}{\overline}

\colorlet{textColor}{black}
\colorlet{background}{white}
\colorlet{penColor}{blue!50!black} % Color of a curve in a plot
\colorlet{penColor2}{red!50!black}% Color of a curve in a plot
\colorlet{penColor3}{red!50!blue} % Color of a curve in a plot
\colorlet{penColor4}{green!50!black} % Color of a curve in a plot
\colorlet{penColor5}{orange!80!black} % Color of a curve in a plot
\colorlet{penColor6}{yellow!70!black} % Color of a curve in a plot
\colorlet{fill1}{penColor!20} % Color of fill in a plot
\colorlet{fill2}{penColor2!20} % Color of fill in a plot
\colorlet{fillp}{fill1} % Color of positive area
\colorlet{filln}{penColor2!20} % Color of negative area
\colorlet{fill3}{penColor3!20} % Fill
\colorlet{fill4}{penColor4!20} % Fill
\colorlet{fill5}{penColor5!20} % Fill
\colorlet{gridColor}{gray!50} % Color of grid in a plot

\newcommand{\surfaceColor}{violet}
\newcommand{\surfaceColorTwo}{redyellow}
\newcommand{\sliceColor}{greenyellow}




\pgfmathdeclarefunction{gauss}{2}{% gives gaussian
  \pgfmathparse{1/(#2*sqrt(2*pi))*exp(-((x-#1)^2)/(2*#2^2))}%
}


%%%%%%%%%%%%%
%% Vectors
%%%%%%%%%%%%%

%% Simple horiz vectors
\renewcommand{\vector}[1]{\left\langle #1\right\rangle}


%% %% Complex Horiz Vectors with angle brackets
%% \makeatletter
%% \renewcommand{\vector}[2][ , ]{\left\langle%
%%   \def\nextitem{\def\nextitem{#1}}%
%%   \@for \el:=#2\do{\nextitem\el}\right\rangle%
%% }
%% \makeatother

%% %% Vertical Vectors
%% \def\vector#1{\begin{bmatrix}\vecListA#1,,\end{bmatrix}}
%% \def\vecListA#1,{\if,#1,\else #1\cr \expandafter \vecListA \fi}

%%%%%%%%%%%%%
%% End of vectors
%%%%%%%%%%%%%

%\newcommand{\fullwidth}{}
%\newcommand{\normalwidth}{}



%% makes a snazzy t-chart for evaluating functions
%\newenvironment{tchart}{\rowcolors{2}{}{background!90!textColor}\array}{\endarray}

%%This is to help with formatting on future title pages.
\newenvironment{sectionOutcomes}{}{}



%% Flowchart stuff
%\tikzstyle{startstop} = [rectangle, rounded corners, minimum width=3cm, minimum height=1cm,text centered, draw=black]
%\tikzstyle{question} = [rectangle, minimum width=3cm, minimum height=1cm, text centered, draw=black]
%\tikzstyle{decision} = [trapezium, trapezium left angle=70, trapezium right angle=110, minimum width=3cm, minimum height=1cm, text centered, draw=black]
%\tikzstyle{question} = [rectangle, rounded corners, minimum width=3cm, minimum height=1cm,text centered, draw=black]
%\tikzstyle{process} = [rectangle, minimum width=3cm, minimum height=1cm, text centered, draw=black]
%\tikzstyle{decision} = [trapezium, trapezium left angle=70, trapezium right angle=110, minimum width=3cm, minimum height=1cm, text centered, draw=black]


\title{Famous Piecewise}

\begin{document}

\begin{abstract}
examples
\end{abstract}
\maketitle



We have many famous functions, which are piecewise defined function.  We have already seen the step function.  It acts as a mathematical on/off switch.

Perhaps more famous are the absolute value function and greatest integer function.







\begin{example} \textit{Absolute Value Function} \\

The Absolute Value function is made from pieces of two linear functions: $L_1(x) = x$ and $L_2(x) = -x$. 

\begin{itemize}
\item From $L_1$ we'll take pairs with nonnegative domain values.
\item From $L_2$ we'll take pairs with negative domain values.
\end{itemize}

The traditional way of notating the absolute value function is with little vertical bars.


\[
|x| = 
\begin{cases}
  -x & \text{ if }  x < 0 \\
  x & \text{ if } x \geq 0
\end{cases}
\]


or

\[
|x| = 
\begin{cases}
  -x & (-\infty, 0) \\
  x & [0, \infty)
\end{cases}
\]


Graph of $y = |x|$.
\begin{image}
\begin{tikzpicture}
  \begin{axis}[
            domain=-10:10, ymax=10, xmax=10, ymin=-10, xmin=-10,
            axis lines =center, xlabel=$x$, ylabel=$y$,
            ytick={-10,-8,-6,-4,-2,2,4,6,8,10},
            xtick={-10,-8,-6,-4,-2,2,4,6,8,10},
            ticklabel style={font=\scriptsize},
            every axis y label/.style={at=(current axis.above origin),anchor=south},
            every axis x label/.style={at=(current axis.right of origin),anchor=west},
            axis on top
          ]
          
  \addplot [draw=penColor,very thick,smooth,domain=(-6:0),<-] {-x};
  \addplot [draw=penColor,very thick,smooth,domain=(0:6),->] {x};


    \end{axis}
\end{tikzpicture}
\end{image}


\end{example}

The formula for absolute value is not very complicated.  However, absolute value involves different formulas used in different situations.  Our algebra doesn't like this. The first step when working algebraically with absolute value signs is getting rid of the absolute value's vertical bars.  They don't work well with our algebra. For instance, they do not always distribute over addition of subtraction.


\begin{question}
$| -3 + 4 | = | -3 | + | 4 |$
  \begin{multipleChoice}
    \choice {True}
    \choice [correct] {False}
  \end{multipleChoice}
\end{question}


\begin{question}
$| 5 - 7 | = | 5 | - | 7 |$
  \begin{multipleChoice}
    \choice {True}
    \choice [correct] {False}
  \end{multipleChoice}
\end{question}


Sometimes this works, but not as a general rule.


When working algebraically, the first step is to convert the absolute value signs into a piecewise defined function. Then we can identify the formula we need in our algebraic expression.  We can see from the formula's domain column that the formulas switch at $0$. Working algebraically with absolute value means first identifying zeros.



\begin{example}


Rewrite $G(t) = | 3t - 5 |$ as a piecewise defined function that doesn't include absolute value bars.


\begin{explanation}
The first step is to identify where $3t - 5$ is positive and negative, which means identifying the inside linear function's zeros.



$3t-5=0$ when $t=\frac{5}{3}$.


\begin{itemize}
\item $3t - 5 < 0$ when $t<\tfrac{5}{3}$, which means $G(t) = -(3t - 5)$ when $t<\tfrac{5}{3}$.
\item $3t - 5 > 0$ when $t>\tfrac{5}{3}$, which means $G(t) = 3t - 5$ when $t>\tfrac{5}{3}$.
\end{itemize}





\[
G(t) = 
\begin{cases}
  -(3t-5) &\text{on $\left( -\infty, \tfrac{5}{3} \right)$} \\
  3t-5 &\text{on $\left[ \tfrac{5}{3}, \infty \right)$}
\end{cases}
\]

\end{explanation}




\end{example}




\begin{example}


Solve $| 3t - 5 | = 7$

\begin{explanation}

We have two scenarios here.

The first situation is when $t < \frac{5}{3}$.  The equation then looks like $-(3t - 5) = 7$.

\begin{align*}
-(3t - 5) & = 7 \\
3t - 5 & = -7  \\
3t & = -2  \\
t & = \frac{-2}{3} = -\frac{2}{3}
\end{align*}



The second situation is when $t > \frac{5}{3}$.  The equation then looks like $3t - 5 = 7$.

\begin{align*}
3t - 5 & = 7 \\
3t  & = 12  \\
t & = 4
\end{align*}


The solution set is $\{  -\frac{2}{3}, 4 \}$.


\end{explanation}

\end{example}






\begin{example} \textit{Greatest Integer Function} \\
Also known as the \textit{floor function}, the \textit{Greatest Integer function} (\textit{GIF}) pairs a domain number with the greatest integer less than or equal to the domain number.

If you think of a horizontal number line, then you start at the domain number and move down (or left) until you arrive at an integer.  If your domain number is already an integer, then you don't need to go anywhere.


Like the absolute value function, the greatest integer function has its own symbol:  $GIF(x) = \lfloor x \rfloor$. The little feet on the vertical bars remind us to go down.

\begin{itemize}
\item $GIF(4.5) = \answer{4}$
\item $\lfloor 2 \rfloor = 2$
\item $GIF(0.5) = 0$
\item $GIF(0) = 0$
\item $\lfloor -4.5 \rfloor = \answer{-5}$
\item $GIF(-7) = -7$
\end{itemize}




Let $N$ be any integer. On the interval $[N, N+1)$, we have $GIF(x) = N$, a constant function.

\textit{GIF} is made from an infinite number of pieces of constant functions.






The graph of $y = \lfloor x\rfloor$ is seen below, except the steps keep going up to the right and down to the left.  The extra 3 dots are a graphical symbol telling us that the pattern continues.
\begin{image}
\begin{tikzpicture}
  \begin{axis}[
            domain=-3:5,
            width=6in,
            height=4in,
            axis lines =middle, xlabel=$x$, ylabel=$y$,
            every axis y label/.style={at=(current axis.above origin),anchor=south},
            every axis x label/.style={at=(current axis.right of origin),anchor=west},
            clip=false,
            %axis on top,
          ]
          \addplot [textColor, very thin, domain=(0:2.3)] {0}; % puts the axis back, axis on top clobbers our open holes
          \addplot [textColor, very thin] plot coordinates {(0,0) (0,2)}; % puts the axis back, axis on top clobbers our open holes
          \addplot [very thick, penColor, domain=(-2:-1)] {-2};
          \addplot [very thick, penColor, domain=(-1:0)] {-1};
          \addplot [very thick, penColor, domain=(0:1)] {0};
          \addplot [very thick, penColor, domain=(1:2)] {1};
          \addplot [very thick, penColor, domain=(2:3)] {2};
          \addplot [very thick, penColor, domain=(3:4)] {3};
          \addplot[color=penColor,fill=penColor,only marks,mark=*] coordinates{(-2,-2)};  %% closed hole          
          \addplot[color=penColor,fill=penColor,only marks,mark=*] coordinates{(-1,-1)};  %% closed hole          
          \addplot[color=penColor,fill=penColor,only marks,mark=*] coordinates{(0,0)};  %% closed hole          
          \addplot[color=penColor,fill=penColor,only marks,mark=*] coordinates{(1,1)};  %% closed hole          
          \addplot[color=penColor,fill=penColor,only marks,mark=*] coordinates{(2,2)};  %% closed hole  
          \addplot[color=penColor,fill=penColor,only marks,mark=*] coordinates{(3,3)};  %% closed hole                  
          \addplot[color=penColor,fill=background,only marks,mark=*] coordinates{(-1,-2)};  %% open hole
          \addplot[color=penColor,fill=background,only marks,mark=*] coordinates{(0,-1)};  %% open hole
          \addplot[color=penColor,fill=background,only marks,mark=*] coordinates{(1,0)};  %% open hole
          \addplot[color=penColor,fill=background,only marks,mark=*] coordinates{(2,1)};  %% open hole
          \addplot[color=penColor,fill=background,only marks,mark=*] coordinates{(3,2)};  %% open hole
          \addplot[color=penColor,fill=background,only marks,mark=*] coordinates{(4,3)};  %% open hole

          \addplot[color=penColor,fill=penColor,only marks,mark=*] coordinates{(3.7,3.5) (3.8,3.6) (3.9,3.7)};  %% 3 dots     
          \addplot[color=penColor,fill=penColor,only marks,mark=*] coordinates{(-1.7,-2.5) (-1.8,-2.6) (-1.9,-2.7)};  %% 3 dots     


        \end{axis}
\end{tikzpicture}
\end{image}




\end{example}

Notice that we have many domain numbers with the same function value. That is ok. To be a function, we just need that each domain number only has one function value.  The \textit{GIF} passes this test.






\[
GIF(x) = 
\begin{cases}
   etc. &    \\
  -2 & \text{ on } [-2, -1) \\
  -1 & \text{ on } [-1, 0) \\ 
   0 & \text{ on } [0, 1) \\
  1 & \text{ on } [1, 2) \\
   2 & \text{ on } [2, 3) \\
  etc. &  
\end{cases}
\]


\[
GIF(x) = N \, \text{ on } \, [N, N+1) \, \text{ where } \, N \in \mathbb{Z}
\]


\begin{question}
  Calculate:
  \[
  \lfloor 2.4 \rfloor
  \begin{prompt}
    =\answer{2}
  \end{prompt}
  \]
  \end{question}

  \begin{question}
  Calculate:
  \[
  \lfloor -2.4 \rfloor
  \begin{prompt}
    =\answer{-3}
  \end{prompt}
  \]
\end{question}





Notice that both the absolute value function and the greatest integer function pass the so-called \textit{vertical line test}.

\begin{theorem} \textbf{\textcolor{green!50!black}{Vertical Line Test}}   \\


The curve with points whose coordinates satisfy $y=f(x)$ represents $y$ as a function of $x$ on a set $S$ if and only if the vertical line $x=a$ intersects the curve $y=f(x)$ at exactly one point for every $a \in S$. This is called the \textbf{vertical line test}.
\end{theorem}





















\begin{center}
\textbf{\textcolor{green!50!black}{ooooo-=-=-=-ooOoo-=-=-=-ooooo}} \\

more examples can be found by following this link\\ \link[More Examples of Piecewise-Defined Functions]{https://ximera.osu.edu/csccmathematics/precalculus1/precalculus1/piecewise/examples/exampleList}

\end{center}









\end{document}
