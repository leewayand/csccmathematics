\documentclass{ximera}


\graphicspath{
  {./}
  {ximeraTutorial/}
  {basicPhilosophy/}
}

\newcommand{\mooculus}{\textsf{\textbf{MOOC}\textnormal{\textsf{ULUS}}}}


\usepackage{tkz-euclide}\usepackage{tikz}
\usepackage{tikz-cd}
\usetikzlibrary{arrows}
\tikzset{>=stealth,commutative diagrams/.cd,
  arrow style=tikz,diagrams={>=stealth}} %% cool arrow head
\tikzset{shorten <>/.style={ shorten >=#1, shorten <=#1 } } %% allows shorter vectors

\usetikzlibrary{backgrounds} %% for boxes around graphs
\usetikzlibrary{shapes,positioning}  %% Clouds and stars
\usetikzlibrary{matrix} %% for matrix
\usepgfplotslibrary{polar} %% for polar plots
\usepgfplotslibrary{fillbetween} %% to shade area between curves in TikZ
\usetkzobj{all}
\usepackage[makeroom]{cancel} %% for strike outs
%\usepackage{mathtools} %% for pretty underbrace % Breaks Ximera
%\usepackage{multicol}
\usepackage{pgffor} %% required for integral for loops



%% http://tex.stackexchange.com/questions/66490/drawing-a-tikz-arc-specifying-the-center
%% Draws beach ball
\tikzset{pics/carc/.style args={#1:#2:#3}{code={\draw[pic actions] (#1:#3) arc(#1:#2:#3);}}}



\usepackage{array}
\setlength{\extrarowheight}{+.1cm}
\newdimen\digitwidth
\settowidth\digitwidth{9}
\def\divrule#1#2{
\noalign{\moveright#1\digitwidth
\vbox{\hrule width#2\digitwidth}}}
























%%This is to help with formatting on future title pages.
\newenvironment{sectionOutcomes}{}{}


\title{Recursive Functions}

\begin{document}

\begin{abstract}
formula of formulas
\end{abstract}
\maketitle



So far, our formulas have been arithmetic instructions involving domain numbers. We have also seen functions defined by repeating values. We can stretch the idea of repeating to recursive values.  That is function values that depend on other values of the same function.









\begin{example}  Factorial 



Domain is all whole numbers. 

Range is all natural numbers. 




\[
Factorial(n) = 
\begin{cases}
  1 & n = 0 \\
  1 & n = 1 \\ 
  n \cdot Factorial(n-1) & 1 < n
\end{cases}
\]

The traditional symbol for the factorial function is ``!''.


\begin{itemize}
  \item $0! = 1$
  \item $1! = 1$
  \item $n! = n \cdot (n-1) \cdot (n-2) \cdots 3 \cdot 2 \cdot 1$
\end{itemize}





\begin{question} 


Write these factorials in the standard place value notation.


\begin{itemize}
\item $3! = \answer{6}$
\item $4! = \answer{24}$
\item $5! = \answer{120}$
\end{itemize}


\end{question}



\end{example}





The values of the factorial function depend on previous values of the factorial function. \\












\begin{example}  Fibonacci


Domain is $\mathbb{N}$. 

Range is $\mathbb{N}$. 



\[
Fibonacci(n) = 
\begin{cases}
  1 & n = 1 \\
  1 & n = 2 \\ 
  Fibonacci(n-1) + Fibonacci(n-2) & 2 < n
\end{cases}
\]


\begin{itemize}
\item $Fibonacci(2) = Fibonacci(1) + Fibonacci(0) = 1 + 1 = 2$
\item $Fibonacci(3) = Fibonacci(2) + Fibonacci(1) = 2 + 1 = 3$
\item $Fibonacci(4) = Fibonacci(3) + Fibonacci(2) = 3 + 2 = 5$
\end{itemize}




\begin{question} 


\begin{itemize}
\item $Fibonacci(5) = \answer{8}$
\item $Fibonacci(6) = \answer{13}$
\item $Fibonacci(7) = \answer{21}$
\end{itemize}


\end{question}



\end{example}

The values of the Fibonacci function depend on previous values of the Fibonacci function.





Recursively defined functions mimic periodic functions.  Both use ``previous'' values.  Periodic functions repeat the ``previous'' values, while recursive functions use these ``previous'' values within more computation. \\


Equations such as

\begin{itemize}
\item $Factorial(n) = n \cdot Factorial(n-1)$
\item $Fibonacci(n) = Fibonacci(n-1) + Fibonacci(n-2)$
\end{itemize}

are called \textbf{functional equations}.  They define how function values relate to other function values.




















\begin{onlineOnly}
\begin{center}
\textbf{\textcolor{green!50!black}{ooooo-=-=-=-ooOoo-=-=-=-ooooo}} \\

more examples can be found by following this link\\ \link[More Examples of Piecewise-Defined Functions]{https://ximera.osu.edu/csccmathematics/precalculus/precalculus/piecewise/examples/exampleList}

\end{center}
\end{onlineOnly}








\end{document}
