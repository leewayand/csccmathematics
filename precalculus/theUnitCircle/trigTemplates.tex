\documentclass{ximera}


\graphicspath{
  {./}
  {ximeraTutorial/}
  {basicPhilosophy/}
}

\newcommand{\mooculus}{\textsf{\textbf{MOOC}\textnormal{\textsf{ULUS}}}}


\usepackage{tkz-euclide}\usepackage{tikz}
\usepackage{tikz-cd}
\usetikzlibrary{arrows}
\tikzset{>=stealth,commutative diagrams/.cd,
  arrow style=tikz,diagrams={>=stealth}} %% cool arrow head
\tikzset{shorten <>/.style={ shorten >=#1, shorten <=#1 } } %% allows shorter vectors

\usetikzlibrary{backgrounds} %% for boxes around graphs
\usetikzlibrary{shapes,positioning}  %% Clouds and stars
\usetikzlibrary{matrix} %% for matrix
\usepgfplotslibrary{polar} %% for polar plots
\usepgfplotslibrary{fillbetween} %% to shade area between curves in TikZ
\usetkzobj{all}
\usepackage[makeroom]{cancel} %% for strike outs
%\usepackage{mathtools} %% for pretty underbrace % Breaks Ximera
%\usepackage{multicol}
\usepackage{pgffor} %% required for integral for loops



%% http://tex.stackexchange.com/questions/66490/drawing-a-tikz-arc-specifying-the-center
%% Draws beach ball
\tikzset{pics/carc/.style args={#1:#2:#3}{code={\draw[pic actions] (#1:#3) arc(#1:#2:#3);}}}



\usepackage{array}
\setlength{\extrarowheight}{+.1cm}
\newdimen\digitwidth
\settowidth\digitwidth{9}
\def\divrule#1#2{
\noalign{\moveright#1\digitwidth
\vbox{\hrule width#2\digitwidth}}}
























%%This is to help with formatting on future title pages.
\newenvironment{sectionOutcomes}{}{}


\title{Categories}

\begin{document}

\begin{abstract}
Trig Templates
\end{abstract}
\maketitle





We are building a library of the elemntary functions.  The idea is to use the library to list characteristics, features, and aspects of all functions within each category.  

That way, if we can identify the type of function we have, then we get free information when analyzing functions. 

The category becomes our reasoning. 



\begin{center}

\textbf{\textcolor{red!70!black}{These are ``CAN'' questions.}} 

\end{center}




\textbf{\textcolor{purple!85!blue}{CAN}} the formula we are given be rewritten as one of the official standard forms for each category? 







\section*{Official Templates}


These elementary function categories are our \textbf{first choice}.  If a function can be represented by one of these standard forms, then we want to describe the function as one of these elementary functions.  That gives us the most information. 



Our general templates are really compositions of basic functions with linear functions.









\begin{formula} \textbf{\textcolor{blue!55!black}{Basic Sine Functions}}

There is one \textbf{basic sine function}. 

\[     S(x) = \sin(x)           \]


\end{formula}

Our category of sine functions is built from composing this basic sine function with linear functions


Let $L_{out}(t) = A t + D$

Let $L_{in}(v) = B v + C$


\[
(L_{out} \, \circ \, \sin \, \circ L_{in})(x) = A \sin(B \, x + C) + D 
\]


\begin{formula} \textbf{\textcolor{blue!55!black}{General Sine Functions}}

A \textbf{sine function} is any function that \textbf{\textcolor{purple!85!blue}{CAN}} be represented with a formula of the form

\[     S(x) =    A \sin(B \, x + C) + D           \]

where $A$, $B$, $C$, and $D$ are real numbers and $A \ne 0$, $B \ne 0$.


\end{formula}












\begin{formula} \textbf{\textcolor{blue!55!black}{Basic Cosine Functions}}

There is one \textbf{basic cosine function}. 

\[     Cx) = \cos(x)           \]


\end{formula}

Our category of cosine functions is built from composing this basic sine function with linear functions


Let $L_{out}(t) = A t + D$

Let $L_{in}(v) = B v + C$


\[
(L_{out} \, \circ \, \cos \, \circ L_{in})(x) = A \cos(B \, x + C) + D 
\]


\begin{formula} \textbf{\textcolor{blue!55!black}{General Cosine Functions}}

A \textbf{cosine function} is any function that \textbf{\textcolor{purple!85!blue}{CAN}} be represented with a formula of the form

\[     C(x) =    A \cos(B \, x + C) + D           \]

where $A$, $B$, $C$, and $D$ are real numbers and $A \ne 0$, $B \ne 0$.


\end{formula}




The same composition with linear functions gives the general categories for tangent, secant, cosecant, and cotangent.












\begin{formula} \textbf{\textcolor{blue!55!black}{General Tangent Functions}}

A \textbf{tangent function} is any function that \textbf{\textcolor{purple!85!blue}{CAN}} be represented with a formula of the form

\[     T(x) =    A \tan(B \, x + C) + D           \]

where $A$, $B$, $C$, and $D$ are real numbers and $A \ne 0$, $B \ne 0$.


\end{formula}








\begin{formula} \textbf{\textcolor{blue!55!black}{General Secant Functions}}

A \textbf{secant function} is any function that \textbf{\textcolor{purple!85!blue}{CAN}} be represented with a formula of the form

\[     S(x) =    A \sec(B \, x + C) + D           \]

where $A$, $B$, $C$, and $D$ are real numbers and $A \ne 0$, $B \ne 0$.


\end{formula}








\begin{formula} \textbf{\textcolor{blue!55!black}{General Cosecant Functions}}

A \textbf{cosecant function} is any function that \textbf{\textcolor{purple!85!blue}{CAN}} be represented with a formula of the form

\[     C(x) =    A \csc(B \, x + C) + D           \]

where $A$, $B$, $C$, and $D$ are real numbers and $A \ne 0$, $B \ne 0$.


\end{formula}








\begin{formula} \textbf{\textcolor{blue!55!black}{General Cotangent Functions}}

A \textbf{cotangent function} is any function that \textbf{\textcolor{purple!85!blue}{CAN}} be represented with a formula of the form

\[     C(x) =    A \cot(B \, x + C) + D           \]

where $A$, $B$, $C$, and $D$ are real numbers and $A \ne 0$, $B \ne 0$.


\end{formula}







\subsection{Reciprocals}



Cotangent, secant, and cosecant were defined as reciprocals of tangent, cosine, and secant.



\[
\cot(x) = \frac{1}{\tan(x)}
\]




\[
\sec(x) = \frac{1}{\cos(x)}
\]



\[
\csc(x) = \frac{1}{\sin(x)}
\]





However, this does not hold for the general forms.



It is not true that the reciprocal of any sine function is a secant function.



\begin{example} Reciprocals


Let $f(x) = \sin(x) + 2$.

This is a sine function, becasue it matches our template, $A \sin(B x + C) + D$.

The reciprocal of $f$ is another function, $g(t) = \frac{1}{\sin(t) + 2}$.

However, this is not a cosecant function.

\textbf{\textcolor{red!75!green}{explanation}} 


The claim is that $g(t) = \frac{1}{\sin(t) + 2}$ is not a cosecant function.

It cannot be written in the form $A \csc(B t + C) + D$.


This is shown by assuming it is a cosecant function and running into a contradiction.


Assume $g(t)$ is cosecant function.


\[
\frac{1}{\sin(t) + 2} = A \csc(B t + C) + D = A \csc(B t + C) + D
\]


The reason that this cannot happen is that $\sin(t) + 2$ can never equal $0$. That means that the domain of $\frac{1}{\sin(t) + 2}$ is al lreal numbers.

However, the domain of $A \csc(B t + C) + D$ can never be all real numbers.


As long as $B \ne 0$, then $B t + C = 0$ has a solution.  That solution is a singularity for the cosecant function and so is not in the domain of $A \csc(B t + C) + D$.

One function includes a real number in its domain that is not in the other domain.  They cannot be equal functions.




\end{example}

The problems is the added constant term.  Added constants are trouble for denominators, because fractions cannot be split at addition in the denominator.




If just stay with $0$ added constant terms, then the reciprocal of sine functions are cosecant functions.



\[
\frac{1}{A \sin(B t + C)} = \frac{1}{A} \csc(B t + C) 
\]




\begin{warning} Reciprocals


Reciprocals of trigonometric functions are not always trigonometric functions.


Since, we use function categories as our reasoning for many characteristics, we must be careful.


\end{warning}










































\begin{center}
\textbf{\textcolor{green!50!black}{ooooo-=-=-=-ooOoo-=-=-=-ooooo}} \\

more examples can be found by following this link\\ \link[More Examples of Trigonometric Functions]{https://ximera.osu.edu/csccmathematics/precalculus2/precalculus2/moreTrigonometricFunctions/examples/exampleList}

\end{center}







\end{document}
