\documentclass{ximera}


\graphicspath{
  {./}
  {ximeraTutorial/}
  {basicPhilosophy/}
}

\newcommand{\mooculus}{\textsf{\textbf{MOOC}\textnormal{\textsf{ULUS}}}}


\usepackage{tkz-euclide}\usepackage{tikz}
\usepackage{tikz-cd}
\usetikzlibrary{arrows}
\tikzset{>=stealth,commutative diagrams/.cd,
  arrow style=tikz,diagrams={>=stealth}} %% cool arrow head
\tikzset{shorten <>/.style={ shorten >=#1, shorten <=#1 } } %% allows shorter vectors

\usetikzlibrary{backgrounds} %% for boxes around graphs
\usetikzlibrary{shapes,positioning}  %% Clouds and stars
\usetikzlibrary{matrix} %% for matrix
\usepgfplotslibrary{polar} %% for polar plots
\usepgfplotslibrary{fillbetween} %% to shade area between curves in TikZ
\usetkzobj{all}
\usepackage[makeroom]{cancel} %% for strike outs
%\usepackage{mathtools} %% for pretty underbrace % Breaks Ximera
%\usepackage{multicol}
\usepackage{pgffor} %% required for integral for loops



%% http://tex.stackexchange.com/questions/66490/drawing-a-tikz-arc-specifying-the-center
%% Draws beach ball
\tikzset{pics/carc/.style args={#1:#2:#3}{code={\draw[pic actions] (#1:#3) arc(#1:#2:#3);}}}



\usepackage{array}
\setlength{\extrarowheight}{+.1cm}
\newdimen\digitwidth
\settowidth\digitwidth{9}
\def\divrule#1#2{
\noalign{\moveright#1\digitwidth
\vbox{\hrule width#2\digitwidth}}}
























%%This is to help with formatting on future title pages.
\newenvironment{sectionOutcomes}{}{}


\title{Cosine}

\begin{document}

\begin{abstract}
analysis
\end{abstract}
\maketitle








As we have seen, the cosine function is just the $x$-coordinates of points on the unit circle viewed as a function of a central angle.



The coordinates of points on the Unit Circle are $( \cos(\theta), \sin(\theta) )$.


\begin{image}
\begin{tikzpicture}
  \begin{axis}[
            xmin=-1.1,xmax=1.1,ymin=-1.1,ymax=1.1,
            axis lines=center,
            width=4in,
            xtick={-1,1},
            ytick={-1,1},
            clip=false,
            unit vector ratio*=1 1 1,
            xlabel=$x$, ylabel=$y$,
            every axis y label/.style={at=(current axis.above origin),anchor=south},
            every axis x label/.style={at=(current axis.right of origin),anchor=west},
          ]        
          \addplot [smooth, domain=(0:360)] ({cos(x)},{sin(x)}); %% unit circle

          \addplot [textColor] plot coordinates {(0,0) (0.766,0.643)}; %% 40 degrees

          \addplot [ultra thick,penColor] plot coordinates {(0.766,0) (0.766,0.643)}; %% 40 degrees
          \addplot [ultra thick,penColor2] plot coordinates {(0,0) (0.766,0)}; %% 40 degrees
          
          %\addplot [ultra thick,penColor3] plot coordinates {(1,0) (1,.839)}; %% 40 degrees          

          \addplot [textColor,smooth, domain=(0:40)] ({0.15*cos(x)},{0.15*sin(x)});
          %\addplot [very thick,penColor] plot coordinates {(0,0) (.766,.643)}; %% sector
          %\addplot [very thick,penColor] plot coordinates {(0,0) (1,0)}; %% sector
          %\addplot [very thick, penColor, smooth, domain=(0:40)] ({cos(x)},{sin(x)}); %% sector
          \node at (axis cs:0.15,0.07) [anchor=west] {$\theta$};
          \node[penColor, rotate=-90] at (axis cs:0.84,0.322) {$\sin(\theta)$};
          \node[penColor2] at (axis cs:0.383,0) [anchor=north] {$\cos(\theta)$};
          %\node[penColor3, rotate=-90] at (axis cs:1.06,.322) {$\tan(\theta)$};

          \addplot[color=black,fill=black,only marks,mark=*] coordinates{(0.766,0.643)};


        \end{axis}
\end{tikzpicture}
\end{image}



If we watch how the $x$-coordinates change as you travel counterclockwise around the unit circle, then we will see exactly how the cosine function behaves.






\subsection*{Basic Cosine Function}



The basic sine function is $\cos(\theta)$.

Its characteristics exactly follow the characteristics of the $x$-coordinates of points on the unit circle.






\textbf{\textcolor{blue!55!black}{Domain}}

The domain of $\cos(\theta)$ is $(-\infty, \infty)$.



\textbf{\textcolor{blue!55!black}{Zeros}}


The zeros of $\cos(\theta)$ are the angles where the $x$-coordinates equal $0$.  These are the angles correspond to the points $(0,1)$ and $(0,-1)$.

That would be an infinite collection of angles.

\[
\{ \, \frac{\pi}{2} + k \pi\, \text{ | } \,  k \in \mathbb{Z} \,  \}
\]



\textbf{\textcolor{blue!55!black}{Continuity}}


$\cos(\theta)$ is a continuous function with no singularities.




\textbf{\textcolor{blue!55!black}{End-Behavior}}



Since cosine is the $x$-coordinate of points on the unit circle, it just keeps oscillating between $-1$ and $1$.

$\cos(\theta)$ has no end-behaviot other than oscillating.




\textbf{\textcolor{blue!55!black}{Behavior}}


Following the $x$-coordinates for one cycle around the unit circle, we can see the behavior of $\cos(\theta)$.



\begin{itemize}
\item On $\left( 0, \frac{\pi}{2} \right)$, $\cos(\theta)$ decreases from $1$ to $0$.
\item On $\left( \frac{\pi}{2}, \pi \right)$, $\cos(\theta)$ decreases from $0$ to $-1$.
\item On $\left( \pi, \frac{3\pi}{2} \right)$, $\cos(\theta)$ increases from $-1$ to $0$.
\item On $\left( \frac{3\pi}{2}, 2\pi \right)$, $\cos(\theta)$ increases from $0$ to $1$.
\end{itemize}


Since $\cos(\theta)$ is periodic with period $2\pi$, this behavior keeps repeating.





\textbf{\textcolor{blue!55!black}{Global Maximum and Minimum}}



$\cos(\theta)$ is a continuous function and switches from decreasing to increasing at $\pi$.  


That makes $\cos(\pi) = -1$ the global minimum, which occurs at $\pi$.


Since $\cos(\theta)$ is periodic with period $2\pi$, this minimum keeps occurring.


The minimum value of $-1$ occurs on the set 

\[
\{ \, \pi + 2k \pi \, \text{ | } \,  k \in \mathbb{Z} \,  \}
\]





Remembering that cosine is periodic, we locate the maximum value of $0$ and $2\pi$.



$\cos(\theta)$ is a continuous function and switches from increasing to decreasing at $0$.  


That makes $\cos(0) = 1$ the global maximum, which occurs at $0$.




Since $\cos(\theta)$ is periodic with period $2\pi$, this maximum keeps occurring.


The maximum value of $1$ occurs on the set 

\[
\{ \, 2k \pi \, \text{ | } \,  k \in \mathbb{Z} \,  \}
\]








\textbf{\textcolor{blue!55!black}{Local Maximum and Minimum}}



The global extrema are automatically local extrema and they are the only ones, since there are no other critical numbers.







\textbf{\textcolor{blue!55!black}{Range}}


$\cos(\theta)$ is a continuous function with a maximum of $1$ and a minimum of $-1$.

Therefore the range is $[-1,1]$.














































\begin{center}
\textbf{\textcolor{green!50!black}{ooooo-=-=-=-ooOoo-=-=-=-ooooo}} \\

more examples can be found by following this link\\ \link[More Examples of Trigonometric Functions]{https://ximera.osu.edu/csccmathematics/precalculus/precalculus/theUnitCircle/examples/exampleList}

\end{center}


\end{document}

