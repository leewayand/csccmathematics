\documentclass{ximera}


\graphicspath{
  {./}
  {ximeraTutorial/}
  {basicPhilosophy/}
}

\newcommand{\mooculus}{\textsf{\textbf{MOOC}\textnormal{\textsf{ULUS}}}}


\usepackage{tkz-euclide}\usepackage{tikz}
\usepackage{tikz-cd}
\usetikzlibrary{arrows}
\tikzset{>=stealth,commutative diagrams/.cd,
  arrow style=tikz,diagrams={>=stealth}} %% cool arrow head
\tikzset{shorten <>/.style={ shorten >=#1, shorten <=#1 } } %% allows shorter vectors

\usetikzlibrary{backgrounds} %% for boxes around graphs
\usetikzlibrary{shapes,positioning}  %% Clouds and stars
\usetikzlibrary{matrix} %% for matrix
\usepgfplotslibrary{polar} %% for polar plots
\usepgfplotslibrary{fillbetween} %% to shade area between curves in TikZ
\usetkzobj{all}
\usepackage[makeroom]{cancel} %% for strike outs
%\usepackage{mathtools} %% for pretty underbrace % Breaks Ximera
%\usepackage{multicol}
\usepackage{pgffor} %% required for integral for loops



%% http://tex.stackexchange.com/questions/66490/drawing-a-tikz-arc-specifying-the-center
%% Draws beach ball
\tikzset{pics/carc/.style args={#1:#2:#3}{code={\draw[pic actions] (#1:#3) arc(#1:#2:#3);}}}



\usepackage{array}
\setlength{\extrarowheight}{+.1cm}
\newdimen\digitwidth
\settowidth\digitwidth{9}
\def\divrule#1#2{
\noalign{\moveright#1\digitwidth
\vbox{\hrule width#2\digitwidth}}}
























%%This is to help with formatting on future title pages.
\newenvironment{sectionOutcomes}{}{}


\title{Cosine}

\begin{document}

\begin{abstract}
analysis
\end{abstract}
\maketitle








As we have seen, the cosine function is just the $x$-coordinates of points on the unit circle viewed as a function of a central angle.



The coordinates of points on the Unit Circle are $( \cos(\theta), \sin(\theta) )$.


\begin{image}
\begin{tikzpicture}
  \begin{axis}[
            xmin=-1.1,xmax=1.1,ymin=-1.1,ymax=1.1,
            axis lines=center,
            width=4in,
            xtick={-1,1},
            ytick={-1,1},
            clip=false,
            unit vector ratio*=1 1 1,
            xlabel=$x$, ylabel=$y$,
            every axis y label/.style={at=(current axis.above origin),anchor=south},
            every axis x label/.style={at=(current axis.right of origin),anchor=west},
          ]        
          \addplot [smooth, domain=(0:360)] ({cos(x)},{sin(x)}); %% unit circle

          \addplot [textColor] plot coordinates {(0,0) (0.766,0.643)}; %% 40 degrees

          \addplot [ultra thick,penColor] plot coordinates {(0.766,0) (0.766,0.643)}; %% 40 degrees
          \addplot [ultra thick,penColor2] plot coordinates {(0,0) (0.766,0)}; %% 40 degrees
          
          %\addplot [ultra thick,penColor3] plot coordinates {(1,0) (1,.839)}; %% 40 degrees          

          \addplot [textColor,smooth, domain=(0:40)] ({0.15*cos(x)},{0.15*sin(x)});
          %\addplot [very thick,penColor] plot coordinates {(0,0) (.766,.643)}; %% sector
          %\addplot [very thick,penColor] plot coordinates {(0,0) (1,0)}; %% sector
          %\addplot [very thick, penColor, smooth, domain=(0:40)] ({cos(x)},{sin(x)}); %% sector
          \node at (axis cs:0.15,0.07) [anchor=west] {$\theta$};
          \node[penColor, rotate=-90] at (axis cs:0.84,0.322) {$\sin(\theta)$};
          \node[penColor2] at (axis cs:0.383,0) [anchor=north] {$\cos(\theta)$};
          %\node[penColor3, rotate=-90] at (axis cs:1.06,.322) {$\tan(\theta)$};

          \addplot[color=black,fill=black,only marks,mark=*] coordinates{(0.766,0.643)};


        \end{axis}
\end{tikzpicture}
\end{image}



If we watch how the $x$-coordinates change as you travel counterclockwise around the unit circle, then we will see exactly how the cosine function behaves.






\subsection*{Basic Cosine Function}



The basic cosine function is $\cos(\theta)$.

Its characteristics exactly follow the characteristics of the $x$-coordinates of points on the unit circle.






\textbf{\textcolor{blue!55!black}{Domain}}

The domain of $\cos(\theta)$ is $(-\infty, \infty)$.

A point can be rotated counterclockwise and clockwise.



\textbf{\textcolor{blue!55!black}{Zeros}}


The zeros of $\cos(\theta)$ are the angles where the $x$-coordinates equal $0$.  These are the angles correspond to the points $(0,1)$ and $(0,-1)$.

That would be an infinite collection of angles.

\[
\left\{ \, \frac{\pi}{2} + k \pi\, \text{ | } \,  k \in \mathbb{Z} \,  \right\}
\]



\textbf{\textcolor{blue!55!black}{Continuity}}


$\cos(\theta)$ is a continuous function with no singularities.




\textbf{\textcolor{blue!55!black}{End-Behavior}}



Since cosine is the $x$-coordinate of points on the unit circle, it just keeps oscillating between $-1$ and $1$.

$\cos(\theta)$ has no end-behavior other than oscillating.




\textbf{\textcolor{blue!55!black}{Behavior}}


Following the $x$-coordinates for one cycle around the unit circle, we can see the behavior of $\cos(\theta)$.



\begin{itemize}
\item On $\left( 0, \frac{\pi}{2} \right)$, $\cos(\theta)$ decreases from $1$ to $0$.
\item On $\left( \frac{\pi}{2}, \pi \right)$, $\cos(\theta)$ decreases from $0$ to $-1$.
\item On $\left( \pi, \frac{3\pi}{2} \right)$, $\cos(\theta)$ increases from $-1$ to $0$.
\item On $\left( \frac{3\pi}{2}, 2\pi \right)$, $\cos(\theta)$ increases from $0$ to $1$.
\end{itemize}


Since $\cos(\theta)$ is periodic with period $2\pi$, this behavior keeps repeating.





\textbf{\textcolor{blue!55!black}{Global Maximum and Minimum}}



$\cos(\theta)$ is a continuous function and switches from decreasing to increasing at $\pi$.  


That makes $\cos(\pi) = -1$ the global minimum, which occurs at $\pi$.


Since $\cos(\theta)$ is periodic with period $2\pi$, this minimum keeps occurring.


The minimum value of $-1$ occurs on the set 

\[
\{ \, \pi + 2k \pi \, \text{ | } \,  k \in \mathbb{Z} \,  \}
\]





Remembering that cosine is periodic, we locate the maximum value of $0$ and $2\pi$.



$\cos(\theta)$ is a continuous function and switches from increasing to decreasing at $0$.  


That makes $\cos(0) = 1$ the global maximum, which occurs at $0$.




Since $\cos(\theta)$ is periodic with period $2\pi$, this maximum keeps occurring.


The maximum value of $1$ occurs on the set 

\[
\{ \, 2k \pi \, \text{ | } \,  k \in \mathbb{Z} \,  \}
\]








\textbf{\textcolor{blue!55!black}{Local Maximum and Minimum}}



The global extrema are automatically local extrema and they are the only ones, since there are no other critical numbers (switching behavior).







\textbf{\textcolor{blue!55!black}{Range}}


$\cos(\theta)$ is a continuous function with a maximum of $1$ and a minimum of $-1$.

Therefore the range is $[-1,1]$.














\begin{idea}

The idea is that the basic cosine function describes the $x$-coordinate as you travel counterclockwise around the unit circle.

The characteristics of the basic cosine function hold steady during each quadrant, changing at the angles $0$, $\frac{\pi}{2}$, $\pi$, $\frac{3\pi}{2}$, and $2\pi$. From there, they keep repeating.

This angle is whatever is inside the parentheses in $\cos(angle)$.


\end{idea}






























\begin{example} Cosine

Describe the behavior of   $f(x) = \cos\left( 2 x + \frac{\pi}{2} \right)$



\textbf{\textcolor{red!75!green}{explanation}} 



The ``angle'' is $2 x + \frac{\pi}{2}$.

We want to know when $2 x + \frac{\pi}{2}$ has the values $0$, $\frac{\pi}{2}$, $\pi$, $\frac{3\pi}{2}$, and $2\pi$.






\[
\begin{array}{l|l}
\text{angle}                  & x  \\ \hline
2 x + \frac{\pi}{2} = 0        & x = -\frac{\pi}{4}  \\
2 x + \frac{\pi}{2} = \frac{\pi}{2}      &  x = 0 \\
2 x + \frac{\pi}{2} = \pi      & x = \frac{\pi}{4}  \\
2 x + \frac{\pi}{2} = \frac{3\pi}{2}     & x = \frac{\pi}{2}   \\
2 x + \frac{\pi}{2} = 2\pi     & x = \frac{3\pi}{4}
\end{array}
\]


We now know which values of $x$ define our quadrants.





\[
\begin{array}{l|l}
\text{Quadrant}                  & x\text{-Interval}  \\  \hline
I        & x \in  \left( -\frac{\pi}{4}, 0 \right)  \\
II       & x \in  \left( 0, \frac{\pi}{4} \right) \\
III      & x \in  \left( \frac{\pi}{4}, \frac{\pi}{2} \right)  \\
IV       & x \in  \left( \frac{\pi}{2}, \frac{3\pi}{4} \right)   \\
\end{array}
\]



Our principal interval is $\left( -\frac{\pi}{4}, \frac{3\pi}{4} \right)$.

That makes the period $\frac{3\pi}{4} - \left( -\frac{\pi}{4} \right) = \pi$.


Now we just remember what the basic cosine function does in each quadrant. 



\textbf{\textcolor{blue!55!black}{Behavior}} 



\textbf{\textcolor{blue!55!black}{$\blacktriangleright$ (I):}} On $\left( -\frac{\pi}{4}, 0 \right)$, $f(x)$ decreases from $1$ to $0$.



\textbf{\textcolor{blue!55!black}{$\blacktriangleright$ (II):}} On $\left( 0, \frac{\pi}{4} \right)$, $f(x)$ decreases from $0$ to $-1$.



\textbf{\textcolor{blue!55!black}{$\blacktriangleright$ (III):}} On $\left( \frac{\pi}{4}, \frac{\pi}{2} \right)$, $f(x)$ increases from $-1$ to $0$.


\textbf{\textcolor{blue!55!black}{$\blacktriangleright$ (IV):}} On $\left( \frac{\pi}{2}, \frac{3\pi}{4} \right)$, $f(x)$ increases from $0$ to $1$.



All of this repeats with a period of $\pi$.












\textbf{\textcolor{blue!55!black}{$\blacktriangleright$ desmos graph}} 
\begin{center}
\desmos{z1dzx293bd}{400}{300}
\end{center}






\textbf{\textcolor{blue!55!black}{Extrema}} 


On our principal interval, $f(x)$ has a global and local maximum of $1$ at $-\frac{\pi}{4}$.


On our principal interval, $f(x)$ has a global and local minimum of $-1$ at $\frac{\pi}{4}$.





\textbf{\textcolor{blue!55!black}{Range}} 

Since cosine is continuous, the extreme values give a range of $[-1,1]$.


\end{example}




\begin{warning}


We must distinguish between the variable of the function and the angle of sine.


Our analysis follows the angle around the unit circle through the four quadrants.

\end{warning}

























\begin{example} Negative Leading Coefficient Inside

Describe the behavior of   $g(t) = \cos\left( \frac{\pi}{4} - 3 t \right)$



\textbf{\textcolor{red!75!green}{explanation}} 



The ``angle'' is $\frac{\pi}{4} - 3 t$.

We want to know when $\frac{\pi}{4} - 3 t$ has the values $0$, $\frac{\pi}{2}$, $\pi$, $\frac{3\pi}{2}$, and $2\pi$.






\[
\begin{array}{l|l}
\text{angle}                  & t  \\ \hline
\frac{\pi}{4} - 3 t = 0        & t = \frac{\pi}{12}  \\
\frac{\pi}{4} - 3 t = \frac{\pi}{2}      &  t = -\frac{\pi}{12} \\
\frac{\pi}{4} - 3 t = \pi      & t = -\frac{\pi}{4}  \\
\frac{\pi}{4} - 3 t = \frac{3\pi}{2}     & t = -\frac{5\pi}{12}   \\
\frac{\pi}{4} - 3 t = 2\pi     & t = -\frac{7\pi}{12}
\end{array}
\]


We now know which values of $t$ define our quadrants, but they are running backwards.  

That is because of the negative coefficient in $-2 t$.


We just have to remember that we write interval notation as the numbers appear on the number line.  They are not written backwards.  

We write interval notation from left to right, but our angles are running backwards of that.  That reverses the behavior of $g(t)$.


The first quadrant begins at $\frac{\pi}{12}$ and ends at $-\frac{\pi}{12}$.

Therefore, our basic cosine function is $1$ at $\frac{\pi}{12}$ and it is $0$ at $-\frac{\pi}{12}$.

But, we write our interval notation like  $\left( -\frac{\pi}{12}, \frac{\pi}{12} \right)$.

Our function will be $0$ on the left and $1$ on the right.  $g(t)$ will be increasing.





\[
\begin{array}{l|l}
\text{Quadrant}                  & t\text{-Interval}  \\ \hline
I        & t \in  \left( -\frac{\pi}{12}, \frac{\pi}{12} \right)  \\
II       & t \in  \left( -\frac{\pi}{4}, -\frac{\pi}{12} \right) \\
III      & t \in  \left( -\frac{5\pi}{12}, -\frac{\pi}{4} \right)  \\
IV       & t \in  \left( -\frac{7\pi}{12}, -\frac{5\pi}{12} \right)   \\
\end{array}
\]



Our principal interval is $\left( -\frac{7\pi}{12}, \frac{\pi}{12} \right)$.

That makes the period $\frac{\pi}{12} - \left( -\frac{7\pi}{12} \right) = \frac{\pi}{12} -\frac{7\pi}{12} = \frac{8\pi}{12} = \frac{2\pi}{3}$.


Now we just remember what the basic cosine function does in each quadrant. 

We just read it backwards.

The basic cosine function values are 
\[
1 \rightarrow 0 \rightarrow -1 \rightarrow 0 \rightarrow 1
\]


$g(t)$ does the same thing.  It just does it from right to left:

\[
1 \leftarrow 0 \leftarrow -1 \leftarrow 0 \leftarrow 1
\]


But, we still read left to right.  So, $g(t)$ begins on the left at $1$, decreases to $0$, decreases to $-1$, increases to $0$, then increases to $1$.










\textbf{\textcolor{blue!55!black}{Behavior}} 



\textbf{\textcolor{blue!55!black}{$\blacktriangleright$ (I):}} On $\left( -\frac{\pi}{12}, \frac{\pi}{12} \right)$, $g(t)$ increases from $0$ to $1$.



\textbf{\textcolor{blue!55!black}{$\blacktriangleright$ (II):}} On $\left( -\frac{\pi}{4}, \frac{\pi}{4} \right)$, $g(t)$ increases from $-1$ to $0$.



\textbf{\textcolor{blue!55!black}{$\blacktriangleright$(III):}} On $\left( -\frac{5\pi}{12}, -\frac{\pi}{4} \right)$, $g(t)$ decreases from $0$ to $-1$.


\textbf{\textcolor{blue!55!black}{$\blacktriangleright$(IV):}} On $\left( -\frac{7\pi}{12}, -\frac{5\pi}{12} \right)$, $g(t)$ decreases from $1$ to $0$.



All of this repeats with a period of $\frac{2\pi}{3}$.









\textbf{\textcolor{blue!55!black}{$\blacktriangleright$ desmos graph}} 
\begin{center}
\desmos{tjeuggk3k2}{400}{300}
\end{center}




We can reverse the order of our intervals to make it follow the graph better.

On $\left( -\frac{7\pi}{12}, -\frac{5\pi}{12} \right)$, $g(t)$ decreases.

On $\left( -\frac{5\pi}{12}, -\frac{\pi}{4} \right)$, $g(t)$ decreases.

On $\left( -\frac{\pi}{4}, \frac{\pi}{4} \right)$, $g(t)$ increases.

On $\left( -\frac{\pi}{12}, \frac{\pi}{12} \right)$, $g(t)$ increases.








\textbf{\textcolor{blue!55!black}{Extrema}} 


On our principal interval, $g(t)$ has a global and local maximum of $1$ at $-\frac{7\pi}{12}$.


On our principal interval, $g(t)$ has a global and local minimum of $-1$ at $-\frac{\pi}{4}$.



\textbf{\textcolor{blue!55!black}{Range}} 


Since cosine is continuous, the extreme values give a range of $[-1,1]$.





\end{example}

















\textbf{\textcolor{red!70!black}{Note:}}  In the example above, the inside of the cosine function is a linear function with a negative leading coefficent.  That has reversed the behavior from the basic cosine function.





\textbf{\textcolor{blue!55!black}{$\blacktriangleright$}} In addition to the leading coefficient of the inner linear function, the cosine funciton itself has a leading coefficient.  When it is negative, the behavior reverses again.































\begin{example} Negative Leading Coefficient

Describe the behavior of   $H(k) = -2 \cos\left( \frac{k}{3} - \pi \right)$



\textbf{\textcolor{red!75!green}{explanation}} 



\textbf{Step 1)}  We'll take this in two steps.  First, the inside.  Then, the outside.


First, we'll just think of $\cos\left( \frac{k}{3} - \pi \right)$.  Then, we'll bring in the $-2$.


Let $f(k) = \cos\left( \frac{k}{3} - \pi \right)$.


The ``angle'' is $\frac{k}{3} - \pi$.

We want to know when $\frac{k}{3} - \pi$ has the values $0$, $\frac{\pi}{2}$, $\pi$, $\frac{3\pi}{2}$, and $2\pi$.






\[
\begin{array}{l|l}
\text{angle}                  & k  \\ \hline
\frac{k}{3} - \pi = 0        & k = 3 \pi  \\
\frac{k}{3} - \pi = \frac{\pi}{2}      &  k = \frac{9\pi}{2} \\
\frac{k}{3} - \pi = \pi      & k = 6 \pi  \\
\frac{k}{3} - \pi = \frac{3\pi}{2}     & k = \frac{15\pi}{2}   \\
\frac{k}{3} - \pi = 2\pi     & k = 9 \pi
\end{array}
\]


We now know which values of $k$ define our quadrants.





\[
\begin{array}{l|l}
\text{Quadrant}                  & k\text{-Interval}  \\  \hline
I        & k \in  \left( 3\pi, \frac{9\pi}{2} \right)  \\
II       & k \in  \left( \frac{9\pi}{2}, 6 \pi \right) \\
III      & k \in  \left( 6 \pi, \frac{15\pi}{2} \right)  \\
IV       & k \in  \left( \frac{15\pi}{2}, 9 \pi \right)   \\
\end{array}
\]



Our principal interval is $\left( 3 \pi, 9 \pi \right)$.

That makes the period $9 \pi - 3 \pi = 6 \pi$.


Now we just remember what the basic sine function does in each quadrant and multiply by $-2$. 



\textbf{\textcolor{blue!55!black}{Behavior}} 



\textbf{\textcolor{blue!55!black}{$\blacktriangleright$ (I):}} On $\left( 3 \pi, \frac{9\pi}{2} \right)$, $f(k)$ decreases from $1$ to $0$.



\textbf{\textcolor{blue!55!black}{$\blacktriangleright$ (II):}} On $\left( \frac{9\pi}{2}, 6 \pi \right)$, $f(k)$ decreases from $0$ to $-1$.



\textbf{\textcolor{blue!55!black}{$\blacktriangleright$ (III):}} On $\left( 6 \pi, \frac{15\pi}{2} \right)$, $f(k)$ increases from $-1$ to $0$.


\textbf{\textcolor{blue!55!black}{$\blacktriangleright$ (IV):}} On $\left( \frac{15\pi}{2}, 9 \pi \right)$, $f(k)$ increases from $0$ to $1$.



All of this repeats with a period of $6 \pi$.






\textbf{Step 2)}



Now we'll bring in $-2$ as the leading coefficent.



\textbf{Remember:} The negative leading coefficient reverses all of the behavior.










\textbf{\textcolor{blue!55!black}{Behavior of $H(k)$}} 




\textbf{\textcolor{blue!55!black}{$\blacktriangleright$ (I):}} On $\left( 3 \pi, \frac{9\pi}{2} \right)$, $f(k)$ decreases from $1$ to $0$, so $H(k)$ increases $-2$ to $0$.



\textbf{\textcolor{blue!55!black}{$\blacktriangleright$ (II):}} On $\left( \frac{9\pi}{2}, 6 \pi \right)$, $f(k)$ decreases from $0$ to $-1$, so $H(k)$ increases $0$ to $2$.



\textbf{\textcolor{blue!55!black}{$\blacktriangleright$ (III):}} On $\left( 6 \pi, \frac{15\pi}{2} \right)$, $f(k)$ increases from $-1$ to $0$, so $H(k)$ decreases $2$ to $0$.


\textbf{\textcolor{blue!55!black}{$\blacktriangleright$ (IV):}} On $\left( \frac{15\pi}{2}, 9 \pi \right)$, $f(k)$ increases from $0$ to $1$, so $H(k)$ decreases $0$ to $-2$.





All of this repeats with a period of $6 \pi$.









\textbf{\textcolor{blue!55!black}{$\blacktriangleright$ desmos graph}} 
\begin{center}
\desmos{xljum2jpck}{400}{300}
\end{center}






\textbf{\textcolor{blue!55!black}{Extrema}} 




$-2 \sin\left( \frac{k}{3} - \pi \right)$ has a maximum value of $2$ and a minimum value of $-2$, which occur at $6 \pi$ and $3 \pi$, respectively.





On our principal interval, $H(k)$ has a global and local maximum of $2$ at $6 \pi$.  This is where the function switches from increasing to decreasing.


On our principal interval, $H(k)$ has a global and local maximum of $-2$ at $3 \pi$.  This is where the function switches from decreasing to increasing.




\end{example}































\begin{formula} \textbf{\textcolor{blue!55!black}{Basic Cosine Functions}}

A \textbf{Basic Cosine Function} is any function that \textbf{\textcolor{purple!85!blue}{CAN}} be represented with a formula of the form

\[     C(x) =    \cos(x)           \]




\end{formula}











\begin{formula} \textbf{\textcolor{blue!55!black}{Cosine Functions}}

A \textbf{Cosine Function} is any function that \textbf{\textcolor{purple!85!blue}{CAN}} be represented with a formula of the form

\[     C(x) =    A \cos(B \, x + C) + D           \]

where $A$, $B$, $C$, and $D$ are real numbers, and $A, B \ne 0$.


\end{formula}





















\begin{center}
\textbf{\textcolor{green!50!black}{ooooo-=-=-=-ooOoo-=-=-=-ooooo}} \\

more examples can be found by following this link\\ \link[More Examples of Unit Circle Functions]{https://ximera.osu.edu/csccmathematics/precalculus/precalculus/theUnitCircle/examples/exampleList}

\end{center}



\end{document}

