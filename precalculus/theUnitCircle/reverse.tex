\documentclass{ximera}


\graphicspath{
  {./}
  {ximeraTutorial/}
  {basicPhilosophy/}
}

\newcommand{\mooculus}{\textsf{\textbf{MOOC}\textnormal{\textsf{ULUS}}}}


\usepackage{tkz-euclide}\usepackage{tikz}
\usepackage{tikz-cd}
\usetikzlibrary{arrows}
\tikzset{>=stealth,commutative diagrams/.cd,
  arrow style=tikz,diagrams={>=stealth}} %% cool arrow head
\tikzset{shorten <>/.style={ shorten >=#1, shorten <=#1 } } %% allows shorter vectors

\usetikzlibrary{backgrounds} %% for boxes around graphs
\usetikzlibrary{shapes,positioning}  %% Clouds and stars
\usetikzlibrary{matrix} %% for matrix
\usepgfplotslibrary{polar} %% for polar plots
\usepgfplotslibrary{fillbetween} %% to shade area between curves in TikZ
\usetkzobj{all}
\usepackage[makeroom]{cancel} %% for strike outs
%\usepackage{mathtools} %% for pretty underbrace % Breaks Ximera
%\usepackage{multicol}
\usepackage{pgffor} %% required for integral for loops



%% http://tex.stackexchange.com/questions/66490/drawing-a-tikz-arc-specifying-the-center
%% Draws beach ball
\tikzset{pics/carc/.style args={#1:#2:#3}{code={\draw[pic actions] (#1:#3) arc(#1:#2:#3);}}}



\usepackage{array}
\setlength{\extrarowheight}{+.1cm}
\newdimen\digitwidth
\settowidth\digitwidth{9}
\def\divrule#1#2{
\noalign{\moveright#1\digitwidth
\vbox{\hrule width#2\digitwidth}}}
























%%This is to help with formatting on future title pages.
\newenvironment{sectionOutcomes}{}{}


\title{Reverse}

\begin{document}

\begin{abstract}
values to angles
\end{abstract}
\maketitle










The sine function associates every real number to a number in the interval $[-1,1]$.  This association comes form the unit circle.


Given an angle, we can interpret this as a counterclockwise angle measurement from the positive $x$-axis. There is exactly one point on the unit circle positioned at that angle. The value of sine is the $y$-coordinate of that point.

We can go the other way.



\textbf{\textcolor{blue!55!black}{$\blacktriangleright$ Reverse:}} Given a number between $-1$ and $1$, we can find interpret this number as a $y$ coordinate of a point on the unit circle.  There is an angle associated with this point on the unit circle.  That angle has a counterclockwise measurement from the positive $x$-axis.   In fact, it has an infinite number of such angles.


Given a number in the interval $[-1,1]$, there are an infinite number of angles whose sine is that number.





\textbf{\textcolor{blue!55!black}{$\blacktriangleright$}} This means the reverse is not a function.  It breaks the one and only function rule.


That doesn't mean it isn't useful.  It is very useful!  For one thing, this helps us solve trigonometric equations, which means zeros of functions.


















\begin{example} Zeros


What are all of the zeros of $f(x) = 2 \sin(3x - \pi) + 1$.


\textbf{\textcolor{red!75!green}{explanation}} 




\[
f(x) = 2 \sin(3x - \pi) + 1 = 0
\]



\[
\sin(3x - \pi) = -\frac{1}{2}
\]



$3x - \pi$ is an angle whose sine is $-\frac{1}{2}$.

We have encountered such angles.  Since the value of sine is negative here, we are looking for ``$\frac{\pi}{6}$'' angles in quadrants III and IV.






\begin{image}
\begin{tikzpicture}
  \begin{axis}[
            xmin=-1.1,xmax=1.1,ymin=-1.1,ymax=1.1,
            axis lines=center,
            width=4in,
            xtick={-1,1},
            ytick={-1,1},
            clip=false,
            unit vector ratio*=1 1 1,
            xlabel=$x$, ylabel=$y$,
            every axis y label/.style={at=(current axis.above origin),anchor=south},
            every axis x label/.style={at=(current axis.right of origin),anchor=west},
          ]        
          \addplot [smooth, domain=(0:360)] ({cos(x)},{sin(x)}); %% unit circle

          \addplot [textColor] plot coordinates {(0,0) (0.866,-0.5)}; %% 30 degrees
          \addplot [textColor] plot coordinates {(0,0) (-0.866,-0.5)};

          %\addplot [ultra thick,penColor] plot coordinates {(0.766,0) (0.766,0.643)}; %% 40 degrees
          %\addplot [ultra thick,penColor2] plot coordinates {(0,0) (0.766,0)}; %% 40 degrees
          
          %\addplot [ultra thick,penColor3] plot coordinates {(1,0) (1,.839)}; %% 40 degrees          

          \addplot [textColor,smooth, domain=(0:30)] ({0.15*cos(x)},{-0.15*sin(x)});
           \addplot [textColor,smooth, domain=(0:30)] ({-0.15*cos(x)},{-0.15*sin(x)});
          %\addplot [very thick,penColor] plot coordinates {(0,0) (.766,.643)}; %% sector
          %\addplot [very thick,penColor] plot coordinates {(0,0) (1,0)}; %% sector
          %\addplot [very thick, penColor, smooth, domain=(0:40)] ({cos(x)},{sin(x)}); %% sector
          \node at (axis cs:0.15,-0.07) [anchor=west] {$\tfrac{\pi}{6}$};
          \node at (axis cs:-0.15,-0.07) [anchor=east] {$\tfrac{\pi}{6}$};
          %\node[penColor, rotate=-90] at (axis cs:0.84,0.322) {$\sin(\theta)$};
          \node[penColor] at (axis cs:0.866,-0.5) [anchor=west] {$\left( \frac{\sqrt{3}}{2}, -\frac{1}{2} \right)$};
          %\node[penColor3, rotate=-90] at (axis cs:1.06,.322) {$\tan(\theta)$};

          \addplot[color=black,fill=black,only marks,mark=*] coordinates{(0.866,-0.5)};
          \addplot[color=black,fill=black,only marks,mark=*] coordinates{(-0.866,-0.5)};


        \end{axis}
\end{tikzpicture}
\end{image}










Two such angles are the angles $\frac{7\pi}{6}$ and $\frac{11\pi}{6}$.


The other angles are full circles from these.

\[
\frac{7\pi}{6} + 2 \pi k \, \text{ and } \, \frac{11\pi}{6} + 2 \pi k \, \text{ where } \, k \in \mathbb{Z}
\]


These are the values of the angles.

$\frac{7\pi}{6} + 2 \pi k$ and $\frac{11\pi}{6} + 2 \pi k$ are the values of $3x - \pi$.



\[ 
3 x - \pi = \frac{7\pi}{6} + 2 \pi k \, \text{ where } \, k \in \mathbb{Z}
\]


\[ 
3 x = \frac{13\pi}{6} + 2 \pi k  \, \text{ where } \, k \in \mathbb{Z}
\]


\[ 
x = \frac{13\pi}{18} + \frac{2}{3} \pi k  \, \text{ where } \, k \in \mathbb{Z}
\]


and



\[ 
x = \frac{17\pi}{18} + \frac{2}{3} \pi k  \, \text{ where } \, k \in \mathbb{Z}
\]











\textbf{\textcolor{blue!55!black}{$\blacktriangleright$ desmos graph}} 
\begin{center}
\desmos{sthiebjcu1}{400}{300}
\end{center}









\end{example}














































\begin{example} Zeros


What are all of the zeros of $G(t) = -2 \cos\left (\frac{\pi}{4} - 2 t \right) + \sqrt{3}$.


\textbf{\textcolor{red!75!green}{explanation}} 




\[
G(t) = -2 \cos\left (\frac{\pi}{4} - 2 t \right) + \sqrt{3} = 0
\]



\[
\cos\left (\frac{\pi}{4} - 2 t \right) = \frac{\sqrt{3}}{2}
\]



$\frac{\pi}{4} - 2 t$ is an angle whose cosine is $\frac{\sqrt{3}}{2}$.

We have encountered such angles.  Since the value of cosine is positive here, we are looking for ``$\frac{\pi}{6}$'' angles in quadrants I and IV.






\begin{image}
\begin{tikzpicture}
  \begin{axis}[
            xmin=-1.1,xmax=1.1,ymin=-1.1,ymax=1.1,
            axis lines=center,
            width=4in,
            xtick={-1,1},
            ytick={-1,1},
            clip=false,
            unit vector ratio*=1 1 1,
            xlabel=$x$, ylabel=$y$,
            every axis y label/.style={at=(current axis.above origin),anchor=south},
            every axis x label/.style={at=(current axis.right of origin),anchor=west},
          ]        
          \addplot [smooth, domain=(0:360)] ({cos(x)},{sin(x)}); %% unit circle

          \addplot [textColor] plot coordinates {(0,0) (0.866,-0.5)}; %% 30 degrees
          \addplot [textColor] plot coordinates {(0,0) (0.866,0.5)};

          %\addplot [ultra thick,penColor] plot coordinates {(0.766,0) (0.766,0.643)}; %% 40 degrees
          %\addplot [ultra thick,penColor2] plot coordinates {(0,0) (0.766,0)}; %% 40 degrees
          
          %\addplot [ultra thick,penColor3] plot coordinates {(1,0) (1,.839)}; %% 40 degrees          

          \addplot [textColor,smooth, domain=(0:30)] ({0.15*cos(x)},{-0.15*sin(x)});
           \addplot [textColor,smooth, domain=(0:30)] ({0.15*cos(x)},{0.15*sin(x)});
          %\addplot [very thick,penColor] plot coordinates {(0,0) (.766,.643)}; %% sector
          %\addplot [very thick,penColor] plot coordinates {(0,0) (1,0)}; %% sector
          %\addplot [very thick, penColor, smooth, domain=(0:40)] ({cos(x)},{sin(x)}); %% sector
          \node at (axis cs:0.15,-0.07) [anchor=west] {$\tfrac{\pi}{6}$};
          \node at (axis cs:0.15,0.07) [anchor=east] {$\tfrac{\pi}{6}$};
          %\node[penColor, rotate=-90] at (axis cs:0.84,0.322) {$\sin(\theta)$};
          \node[penColor] at (axis cs:0.866,0.5) [anchor=west] {$\left( \frac{\sqrt{3}}{2}, \frac{1}{2} \right)$};
          %\node[penColor3, rotate=-90] at (axis cs:1.06,.322) {$\tan(\theta)$};

          \addplot[color=black,fill=black,only marks,mark=*] coordinates{(0.866,-0.5)};
          \addplot[color=black,fill=black,only marks,mark=*] coordinates{(0.866,0.5)};


        \end{axis}
\end{tikzpicture}
\end{image}










Two such angles are the angles $\frac{\pi}{6}$ and $\frac{11\pi}{6}$.


The other angles are full circles from these.

\[
\frac{\pi}{6} + 2 \pi k \, \text{ and } \, \frac{11\pi}{6} + 2 \pi k \, \text{ where } \, k \in \mathbb{Z}
\]


These are the values of the angles.

$\frac{\pi}{6} + 2 \pi k$ and $\frac{11\pi}{6} + 2 \pi k$ are the values of $\frac{\pi}{4} - 2 t$.



\[ 
\frac{\pi}{4} - 2 t = \frac{\pi}{6} + 2 \pi k \, \text{ where } \, k \in \mathbb{Z}
\]


\[ 
- 2 t = -\frac{\pi}{12} + 2 \pi k  \, \text{ where } \, k \in \mathbb{Z}
\]


\[ 
t = \frac{\pi}{24} +  \pi k  \, \text{ where } \, k \in \mathbb{Z}
\]


and


\[ 
t = -\frac{19\pi}{24} +  \pi k  \, \text{ where } \, k \in \mathbb{Z}
\]












\textbf{\textcolor{blue!55!black}{$\blacktriangleright$ desmos graph}} 
\begin{center}
\desmos{kzhi2htulq}{400}{300}
\end{center}









\end{example}


















The reverses of sine and cosine are not functions, because the same number in $[-1,1]$ is connected to multiple angles.

We will mold these into functions later in the course, by restricting the range.  Restricting the range will us into compliance with the one and only function rule.  These functions will be called \textbf{arcsine} and \textbf{arccosine}.







More useful is the reverse of sine or cosine is the reverse of tangent.


Suppose we are given a random point in the Cartesian plane: $(a,b)$.  Just like points on the unit circle, this point is on a line from the origin and thus at an angle from the positive $x$-axis.  What is this angle?


Let's call this angle $\theta$.

The point $(a,b)$ has a distance from the origin.  Let's call this distance $r$.


Then the coordinates of this point are

\begin{itemize}
\item $a = r \cos(\theta)$
\item $b = r \sin(\theta)$
\end{itemize}

which gives us


\begin{itemize}
\item $\frac{a}{r} = \cos(\theta)$
\item $\frac{b}{r} = \sin(\theta)$
\end{itemize}



which give us


\[
\frac{b}{a} = \frac{\tfrac{b}{r}}{\tfrac{a}{r}} = \frac{\sin(\theta)}{\cos(\theta)} = \tan(\theta)
\]



We can get the angle directly from the coordinates through the tangent function.





















\begin{example} Angles of Points


What is the angle for $(-12, 12 \sqrt{3})$?


\textbf{\textcolor{red!75!green}{explanation}} 



Let's call this angle $\theta$.


$\theta$ is an angle in the second quadrant whose tangent is $\frac{12 \sqrt{3}}{-12} = -\sqrt{3}$.


The value of tangent is negative here, which is what should happen for an angle in quandrant II.



We have encountered such an angle. 



$\tan\left( \frac{2 \pi}{3} \right) = -\sqrt{3}$.


There are an infinite number of such angles, separated by multiples of $\pi$.


\[
\frac{2 \pi}{3} + 2 k \pi \, \text{ where } \, k \in \mathbb{Z}
\]




\end{example}
















































\begin{center}
\textbf{\textcolor{green!50!black}{ooooo-=-=-=-ooOoo-=-=-=-ooooo}} \\

more examples can be found by following this link\\ \link[More Examples of Unit Circle Functions]{https://ximera.osu.edu/csccmathematics/precalculus/precalculus/theUnitCircle/examples/exampleList}

\end{center}




\end{document}

