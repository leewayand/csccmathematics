\documentclass{ximera}


\graphicspath{
  {./}
  {ximeraTutorial/}
  {basicPhilosophy/}
}

\newcommand{\mooculus}{\textsf{\textbf{MOOC}\textnormal{\textsf{ULUS}}}}


\usepackage{tkz-euclide}\usepackage{tikz}
\usepackage{tikz-cd}
\usetikzlibrary{arrows}
\tikzset{>=stealth,commutative diagrams/.cd,
  arrow style=tikz,diagrams={>=stealth}} %% cool arrow head
\tikzset{shorten <>/.style={ shorten >=#1, shorten <=#1 } } %% allows shorter vectors

\usetikzlibrary{backgrounds} %% for boxes around graphs
\usetikzlibrary{shapes,positioning}  %% Clouds and stars
\usetikzlibrary{matrix} %% for matrix
\usepgfplotslibrary{polar} %% for polar plots
\usepgfplotslibrary{fillbetween} %% to shade area between curves in TikZ
\usetkzobj{all}
\usepackage[makeroom]{cancel} %% for strike outs
%\usepackage{mathtools} %% for pretty underbrace % Breaks Ximera
%\usepackage{multicol}
\usepackage{pgffor} %% required for integral for loops



%% http://tex.stackexchange.com/questions/66490/drawing-a-tikz-arc-specifying-the-center
%% Draws beach ball
\tikzset{pics/carc/.style args={#1:#2:#3}{code={\draw[pic actions] (#1:#3) arc(#1:#2:#3);}}}



\usepackage{array}
\setlength{\extrarowheight}{+.1cm}
\newdimen\digitwidth
\settowidth\digitwidth{9}
\def\divrule#1#2{
\noalign{\moveright#1\digitwidth
\vbox{\hrule width#2\digitwidth}}}
























%%This is to help with formatting on future title pages.
\newenvironment{sectionOutcomes}{}{}


\title{Reverse}

\begin{document}

\begin{abstract}
values to angles
\end{abstract}
\maketitle










The sine function associates every real number to a number in the interval $[-1,1]$.  This association comes form the unit circle.


Given an angle, interpret this as a counterclockwise angle measurement from the positive $x$-axis. There is exactly one point on the unit circle positioned at that angle. The value of sine is the $y$-coordinate of that point.

We can go the other way.



\textbf{\textcolor{blue!55!black}{$\blacktriangleright$ Reverse:}} Given a number between $-1$ and $1$, we can find interpret this number as a $y$ coordinate of a point on the unit circle.  There is an angle associated with this point on the unit circle.  That angle has a counterclockwise measurement from the positive $x$-axis.   In fact it has an infinity such angles.


Given a number in the interval $[-1,1]$, there are an infinite number of angles whose sine is that number.





\textbf{\textcolor{blue!55!black}{$\blacktriangleright$}} This means the reverse is not a function.  It breaks the one and only function rule.


That doesn't mean it isn't useful.  It is very useful!  For one thing, this helps us solve trigonometric equations, which means zeros of functions.


\begin{example} Zeros


What are all of the zeros of $f(x) = 2 \sin(3x - \pi) + 1 = 0$


\textbf{\textcolor{red!75!green}{explanation}} 




\[
f(x) = 2 \sin(3x - \pi) + 1 = 0
\]



\[
\sin(3x - \pi) = -\frac{1}{2}
\]



$3x - \pi$ is an angle whose sine is $-\frac{1}{2}$.

We have encountered such angles.  Since the value of sine is negative here, we are looking for ``$\frac{\pi}{6}$'' angles in quadrants III and IV.






\begin{image}
\begin{tikzpicture}
  \begin{axis}[
            xmin=-1.1,xmax=1.1,ymin=-1.1,ymax=1.1,
            axis lines=center,
            width=4in,
            xtick={-1,1},
            ytick={-1,1},
            clip=false,
            unit vector ratio*=1 1 1,
            xlabel=$x$, ylabel=$y$,
            every axis y label/.style={at=(current axis.above origin),anchor=south},
            every axis x label/.style={at=(current axis.right of origin),anchor=west},
          ]        
          \addplot [smooth, domain=(0:360)] ({cos(x)},{sin(x)}); %% unit circle

          \addplot [textColor] plot coordinates {(0,0) (0.866,-0.5)}; %% 40 degrees

          %\addplot [ultra thick,penColor] plot coordinates {(0.766,0) (0.766,0.643)}; %% 40 degrees
          %\addplot [ultra thick,penColor2] plot coordinates {(0,0) (0.766,0)}; %% 40 degrees
          
          %\addplot [ultra thick,penColor3] plot coordinates {(1,0) (1,.839)}; %% 40 degrees          

          \addplot [textColor,smooth, domain=(0:40)] ({0.15*cos(x)},{-0.15*sin(x)});
          %\addplot [very thick,penColor] plot coordinates {(0,0) (.766,.643)}; %% sector
          %\addplot [very thick,penColor] plot coordinates {(0,0) (1,0)}; %% sector
          %\addplot [very thick, penColor, smooth, domain=(0:40)] ({cos(x)},{sin(x)}); %% sector
          \node at (axis cs:0.15,-0.07) [anchor=west] {$\theta$};
          %\node[penColor, rotate=-90] at (axis cs:0.84,0.322) {$\sin(\theta)$};
          \node[penColor] at (axis cs:0.866,-0.5) [anchor=west] {$(x,y))$};
          %\node[penColor3, rotate=-90] at (axis cs:1.06,.322) {$\tan(\theta)$};

          \addplot[color=black,fill=black,only marks,mark=*] coordinates{(0.866,-0.5)};


        \end{axis}
\end{tikzpicture}
\end{image}

























\end{example}




















































\begin{center}
\textbf{\textcolor{green!50!black}{ooooo-=-=-=-ooOoo-=-=-=-ooooo}} \\

more examples can be found by following this link\\ \link[More Examples of Unit Circle Functions]{https://ximera.osu.edu/csccmathematics/precalculus/precalculus/theUnitCircle/examples/exampleList}

\end{center}




\end{document}

