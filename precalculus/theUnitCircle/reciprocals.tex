\documentclass{ximera}


\graphicspath{
  {./}
  {ximeraTutorial/}
  {basicPhilosophy/}
}

\newcommand{\mooculus}{\textsf{\textbf{MOOC}\textnormal{\textsf{ULUS}}}}


\usepackage{tkz-euclide}\usepackage{tikz}
\usepackage{tikz-cd}
\usetikzlibrary{arrows}
\tikzset{>=stealth,commutative diagrams/.cd,
  arrow style=tikz,diagrams={>=stealth}} %% cool arrow head
\tikzset{shorten <>/.style={ shorten >=#1, shorten <=#1 } } %% allows shorter vectors

\usetikzlibrary{backgrounds} %% for boxes around graphs
\usetikzlibrary{shapes,positioning}  %% Clouds and stars
\usetikzlibrary{matrix} %% for matrix
\usepgfplotslibrary{polar} %% for polar plots
\usepgfplotslibrary{fillbetween} %% to shade area between curves in TikZ
\usetkzobj{all}
\usepackage[makeroom]{cancel} %% for strike outs
%\usepackage{mathtools} %% for pretty underbrace % Breaks Ximera
%\usepackage{multicol}
\usepackage{pgffor} %% required for integral for loops



%% http://tex.stackexchange.com/questions/66490/drawing-a-tikz-arc-specifying-the-center
%% Draws beach ball
\tikzset{pics/carc/.style args={#1:#2:#3}{code={\draw[pic actions] (#1:#3) arc(#1:#2:#3);}}}



\usepackage{array}
\setlength{\extrarowheight}{+.1cm}
\newdimen\digitwidth
\settowidth\digitwidth{9}
\def\divrule#1#2{
\noalign{\moveright#1\digitwidth
\vbox{\hrule width#2\digitwidth}}}
























%%This is to help with formatting on future title pages.
\newenvironment{sectionOutcomes}{}{}


\title{Reciprocals}

\begin{document}

\begin{abstract}
zeros to singularities
\end{abstract}
\maketitle



\begin{idea} \textbf{\textcolor{blue!55!black}{Reciprocals}}


If $f(x)$ is a function, then its reciprocal is $\frac{1}{f(x)}$.

Reciprocal functions have ``opposite'' characteristics.

\begin{itemize}
\item zeros of $f(x)$ become singularities of $\frac{1}{f(x)}$.
\item singularities of $f(x)$ become zeros of $\frac{1}{f(x)}$.
\end{itemize}


The sign of $\frac{1}{f(x)}$ is the same the sign of $f(x)$.


\begin{itemize}
\item where $f(x)$ is increasing, $\frac{1}{f(x)}$ is decreasing.
\item where $f(x)$ is decreasing, $\frac{1}{f(x)}$ is increasing.
\end{itemize}


\end{idea}






\subsection*{$\frac{1}{\sin(\theta)}$}


The reciprocal of sine is called cosecant, which is abbreviated as $\csc$.


\[
\csc(\theta) = \frac{1}{\sin(\theta)}
\]


Since $\sin(\theta)$ is periodic with a period of $2 \pi$, $\csc(\theta)$ is also periodic with a period of $2 \pi$.



\begin{itemize}
\item zeros of $\sin(\theta)$ become singularities of $\csc(\theta)$.
\item $\sin(\theta)$ has no singularities, so $\csc(\theta)$ has no zeros.
\end{itemize}




\begin{itemize}
\item $\sin(\theta)$ is increasing on $\left( 0,\frac{\pi}{2} \right)$ and $\left( \frac{3\pi}{2}, 2 \pi \right)$, which means that $\csc(\theta)$ is decreasing on $\left( 0,\frac{\pi}{2} \right)$ and $\left( \frac{3\pi}{2}, 2 \pi \right)$.
\item $\sin(\theta)$ is decreasing on $\left( \frac{\pi}{2}, \frac{3\pi}{2} \right)$, which means that $\csc(\theta)$ is increasing on $\left( \frac{\pi}{2}, \frac{3\pi}{2} \right)$.
\end{itemize}



The behavior tells us that $\csc(\theta)$ has a local minimum at $\frac{\pi}{2}$ and a local maximum at $\frac{3\pi}{2}$





Since $\lim\limits_{\theta \to 0} \sin(\theta) = 0$, we know that $\csc(\theta)$ is unbounded near $0$.  $0$ is a singularity of $\csc(\theta)$.

We also know that $\sin(\theta) > 0$ when $\theta$ is slightly positive.

That tells us that

\[
\lim\limits_{\theta \to 0^+} \csc(\theta) = \infty
\]



We also know that $\sin(\theta) < 0$ when $\theta$ is slightly negative.

That tells us that

\[
\lim\limits_{\theta \to 0^-} \csc(\theta) = -\infty
\]








\textbf{\textcolor{blue!55!black}{$\blacktriangleright$ desmos graph}} 
\begin{center}
\desmos{ithaiyrlee}{400}{300}
\end{center}






















\subsection*{$\frac{1}{\cos(\theta)}$}


The reciprocal of cosine is called secant, which is abbreviated as $\sec$.


\[
\sec(\theta) = \frac{1}{\cos(\theta)}
\]


Since $\cos(\theta)$ is periodic with a period of $2 \pi$, $\sec(\theta)$ is also periodic with a period of $2 \pi$.



\begin{itemize}
\item zeros of $\cos(\theta)$ become singularities of $\sec(\theta)$.
\item $\cos(\theta)$ has no singularities, so $\sec(\theta)$ has no zeros.
\end{itemize}




\begin{itemize}
\item $\cos(\theta)$ is decreasing on $(0, \pi)$, which means that $\sec(\theta)$ is increasing on $(0, \pi)$.
\item $\cos(\theta)$ is increasing on $(\pi, 2 \pi)$, which means that $\sec(\theta)$ is decreasing on $(\pi, 2 \pi)$.
\end{itemize}



The behavior tells us that $\sec(\theta)$ has a local minimum at $0$ and a local maximum at $\pi$.





Since $\lim\limits_{\theta \to \tfrac{\pi}{2}} \cos(\theta) = 0$, we know that $\sec(\theta)$ is unbounded near $\frac{\pi}{2}$.  $\frac{\pi}{2}$ is a singularity of $\csc(\theta)$.

We also know that $\cos(\theta) > 0$ when $\theta$ is slightly less than $\frac{\pi}{2}$.

That tells us that

\[
\lim\limits_{\theta \to \frac{\pi}{2}^-} \sec(\theta) = \infty
\]



We also know that $\cos(\theta) < 0$ when $\theta$ is slightly greater than $\frac{\pi}{2}$.

That tells us that

\[
\lim\limits_{\theta \to \frac{\pi}{2}^+} \sec(\theta) = -\infty
\]












\textbf{\textcolor{blue!55!black}{$\blacktriangleright$ desmos graph}} 
\begin{center}
\desmos{rdruru0ayb}{400}{300}
\end{center}

























\subsection*{$\frac{1}{\tan(\theta)}$}


The reciprocal of tangent is called cotangent, which is abbreviated as $\cot$.


\[
\cot(\theta) = \frac{1}{\tan(\theta)}
\]


Since $\tan(\theta)$ is periodic with a period of $\pi$, $\cot(\theta)$ is also periodic with a period of $\pi$.



\begin{itemize}
\item zeros of $\tan(\theta)$ become singularities of $\cot(\theta)$.
\item singularities of $\tan(\theta)$ become zeros of $\cot(\theta)$.
\end{itemize}




\begin{itemize}
\item $\tan(\theta)$ is an increasing function.
\item $\cot(\theta)$ is a decreasing function.
\end{itemize}


$\tan(\theta)$ has no maximums or minimums, so neither does $\cot(\theta)$.



Since $\cot(\theta)$ is a decreasing function we know that

\[
\lim\limits_{\theta \to 0^+} \sec(\theta) = \infty
\]




\[
\lim\limits_{\theta \to 0^-} \sec(\theta) = -\infty
\]












\textbf{\textcolor{blue!55!black}{$\blacktriangleright$ desmos graph}} 
\begin{center}
\desmos{vp4b0dqz9b}{400}{300}
\end{center}















algebra











































\begin{center}
\textbf{\textcolor{green!50!black}{ooooo-=-=-=-ooOoo-=-=-=-ooooo}} \\

more examples can be found by following this link\\ \link[More Examples of Unit Circle Functions]{https://ximera.osu.edu/csccmathematics/precalculus/precalculus/theUnitCircle/examples/exampleList}

\end{center}



\end{document}

