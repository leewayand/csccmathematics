\documentclass{ximera}


\graphicspath{
  {./}
  {ximeraTutorial/}
  {basicPhilosophy/}
}

\newcommand{\mooculus}{\textsf{\textbf{MOOC}\textnormal{\textsf{ULUS}}}}


\usepackage{tkz-euclide}\usepackage{tikz}
\usepackage{tikz-cd}
\usetikzlibrary{arrows}
\tikzset{>=stealth,commutative diagrams/.cd,
  arrow style=tikz,diagrams={>=stealth}} %% cool arrow head
\tikzset{shorten <>/.style={ shorten >=#1, shorten <=#1 } } %% allows shorter vectors

\usetikzlibrary{backgrounds} %% for boxes around graphs
\usetikzlibrary{shapes,positioning}  %% Clouds and stars
\usetikzlibrary{matrix} %% for matrix
\usepgfplotslibrary{polar} %% for polar plots
\usepgfplotslibrary{fillbetween} %% to shade area between curves in TikZ
\usetkzobj{all}
\usepackage[makeroom]{cancel} %% for strike outs
%\usepackage{mathtools} %% for pretty underbrace % Breaks Ximera
%\usepackage{multicol}
\usepackage{pgffor} %% required for integral for loops



%% http://tex.stackexchange.com/questions/66490/drawing-a-tikz-arc-specifying-the-center
%% Draws beach ball
\tikzset{pics/carc/.style args={#1:#2:#3}{code={\draw[pic actions] (#1:#3) arc(#1:#2:#3);}}}



\usepackage{array}
\setlength{\extrarowheight}{+.1cm}
\newdimen\digitwidth
\settowidth\digitwidth{9}
\def\divrule#1#2{
\noalign{\moveright#1\digitwidth
\vbox{\hrule width#2\digitwidth}}}
























%%This is to help with formatting on future title pages.
\newenvironment{sectionOutcomes}{}{}


\title{Algebraic Geometry}

\begin{document}

\begin{abstract}
%%%
\end{abstract}
\maketitle





The real number line had just one dimension, which made the idea of direction simple - two of them: left and right.  These were commonly known as the positive and negative directions.  Each number was described with two pieces of information.  Each number came with a direction and a distance.  We usually give the direction information first and then the distance.


\begin{itemize}
\item For example $-3$.  The hyphen out front means the left (negative) direction and the $3$ gives the distance.
\item For example $+5$.  The plus sign out front means the right (positive) direction and the $5$ gives the distance.
\end{itemize}
\textbf{Note:} We usually do not write the $+$, $+3 = 3$.

We represent the direction first.  A hyphen appears to the left of the distance number for a negative number. Or, the a hyphen is absent for a positive number.

The Complex Numbers have many more directions, but the idea is the same.  Each complex number can be described with a direction and a distance.  This type of representation is known as \textbf{polar form}.














\begin{center}
\textbf{\textcolor{red!70!black}{Every complex number can be written as a scalar multiple of a unit direction vector.}}
\end{center}






Direction vectors or unit vectors are vectors of length $1$, which makes the unit circle the key to understanding the Complex Numbers. \\ 



We can describe every complex number by providing a complex number on the unit circle to give the direction and then a distance to move in that direction.  \\ 

We have already described points on the unit circle with sine and cosine: $(\cos(\theta), \sin(\theta))$. 

Therefore, we have already described complex numbers on the unit circle with sine and cosine: $\cos(\theta) + \sin(\theta) \, i$.  


All of this came from investigating the right triangle formed from points on the unit circle. \\



\begin{center}
\textbf{\textcolor{purple!85!blue}{Complex Numbers, the Unit Circle, Right Triangles, and Sine and Cosine all give the same information}}
\end{center}







\subsection*{Learning Outcomes}



\begin{sectionOutcomes}
In this section, students will 

\begin{itemize}
\item represent real numbers on the 2D number line in polar form.
\item examine the unit circle via right triangles.
\item decompose vectors into perpendicular components.
\item describe components with sine and cosine.
\item describe complex numbers with this information
\end{itemize}
\end{sectionOutcomes}












\begin{center}
\textbf{\textcolor{green!50!black}{ooooo-=-=-=-ooOoo-=-=-=-ooooo}} \\

more examples can be found by following this link\\ \link[More Examples of Polar Form of Complex Numbers]{https://ximera.osu.edu/csccmathematics/precalculus/precalculus/algebraicGeometry/examples/exampleList}

\end{center}








\end{document}
