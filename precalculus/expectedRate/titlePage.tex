\documentclass{ximera}


\graphicspath{
  {./}
  {ximeraTutorial/}
  {basicPhilosophy/}
}

\newcommand{\mooculus}{\textsf{\textbf{MOOC}\textnormal{\textsf{ULUS}}}}


\usepackage{tkz-euclide}\usepackage{tikz}
\usepackage{tikz-cd}
\usetikzlibrary{arrows}
\tikzset{>=stealth,commutative diagrams/.cd,
  arrow style=tikz,diagrams={>=stealth}} %% cool arrow head
\tikzset{shorten <>/.style={ shorten >=#1, shorten <=#1 } } %% allows shorter vectors

\usetikzlibrary{backgrounds} %% for boxes around graphs
\usetikzlibrary{shapes,positioning}  %% Clouds and stars
\usetikzlibrary{matrix} %% for matrix
\usepgfplotslibrary{polar} %% for polar plots
\usepgfplotslibrary{fillbetween} %% to shade area between curves in TikZ
\usetkzobj{all}
\usepackage[makeroom]{cancel} %% for strike outs
%\usepackage{mathtools} %% for pretty underbrace % Breaks Ximera
%\usepackage{multicol}
\usepackage{pgffor} %% required for integral for loops



%% http://tex.stackexchange.com/questions/66490/drawing-a-tikz-arc-specifying-the-center
%% Draws beach ball
\tikzset{pics/carc/.style args={#1:#2:#3}{code={\draw[pic actions] (#1:#3) arc(#1:#2:#3);}}}



\usepackage{array}
\setlength{\extrarowheight}{+.1cm}
\newdimen\digitwidth
\settowidth\digitwidth{9}
\def\divrule#1#2{
\noalign{\moveright#1\digitwidth
\vbox{\hrule width#2\digitwidth}}}
























%%This is to help with formatting on future title pages.
\newenvironment{sectionOutcomes}{}{}


\title{Expected Rate}

\begin{document}

\begin{abstract}
%Stuff can go here later if we want!
\end{abstract}
\maketitle



We have a well-known method for calculating a rate of change of a funciton over an interval.



The rate of change of $f$ over the interval $[a,b]$ is given by

\[
\frac{f(b) - f(a)}{b-a}
\]



Obviously $a$ and $b$ need to be different numbers for this to work. Otherwsie, the rate would be $\frac{0}{0}$, which makes no sense.

We want to make some sense out of it.




\subsection*{Making Sense}

We will use limits to make sense out of 


\[
\frac{f(a) - f(a)}{a-a}
\]




Limits give us expected values. We can use limits to give us an expected value for $\frac{f(a) - f(a)}{a-a}$.



Our plan is to calculate $\frac{f(b) - f(c)}{b-c}$ for intervals continaing $a$, $a \in (b,c)$.

We'll let this interval shrink down closer and closer to $a$.

As the interval shrink, we'll watch the value of $\frac{f(b) - f(c)}{b-c}$.

From this we can determine an expected value.


We will call this value \textbf{the derivative of f at a : $f'(a)$}










\subsection*{Learning Outcomes}



\begin{sectionOutcomes}
In this section, students will 

\begin{itemize}
\item examine tangent lines.
\item create a new function called the derivative.
\item analyze funcitons using the derivative
\end{itemize}
\end{sectionOutcomes}
















\pdfOnly{
\begin{center}
more examples can be found in the online edition.
\end{center}
}




\begin{onlineOnly}
\begin{center}
\textbf{\textcolor{green!50!black}{ooooo-=-=-=-ooOoo-=-=-=-ooooo}} \\

more examples can be found by following this link\\ \link[More Examples of Expected Rate]{https://ximera.osu.edu/csccmathematics/precalculus/precalculus/expectedRate/examples/exampleList}

\end{center}
\end{onlineOnly}






\end{document}
