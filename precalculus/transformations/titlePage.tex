\documentclass{ximera}


\graphicspath{
  {./}
  {ximeraTutorial/}
  {basicPhilosophy/}
}

\newcommand{\mooculus}{\textsf{\textbf{MOOC}\textnormal{\textsf{ULUS}}}}


\usepackage{tkz-euclide}\usepackage{tikz}
\usepackage{tikz-cd}
\usetikzlibrary{arrows}
\tikzset{>=stealth,commutative diagrams/.cd,
  arrow style=tikz,diagrams={>=stealth}} %% cool arrow head
\tikzset{shorten <>/.style={ shorten >=#1, shorten <=#1 } } %% allows shorter vectors

\usetikzlibrary{backgrounds} %% for boxes around graphs
\usetikzlibrary{shapes,positioning}  %% Clouds and stars
\usetikzlibrary{matrix} %% for matrix
\usepgfplotslibrary{polar} %% for polar plots
\usepgfplotslibrary{fillbetween} %% to shade area between curves in TikZ
\usetkzobj{all}
\usepackage[makeroom]{cancel} %% for strike outs
%\usepackage{mathtools} %% for pretty underbrace % Breaks Ximera
%\usepackage{multicol}
\usepackage{pgffor} %% required for integral for loops



%% http://tex.stackexchange.com/questions/66490/drawing-a-tikz-arc-specifying-the-center
%% Draws beach ball
\tikzset{pics/carc/.style args={#1:#2:#3}{code={\draw[pic actions] (#1:#3) arc(#1:#2:#3);}}}



\usepackage{array}
\setlength{\extrarowheight}{+.1cm}
\newdimen\digitwidth
\settowidth\digitwidth{9}
\def\divrule#1#2{
\noalign{\moveright#1\digitwidth
\vbox{\hrule width#2\digitwidth}}}
























%%This is to help with formatting on future title pages.
\newenvironment{sectionOutcomes}{}{}


\title{Transforming}

\begin{document}

\begin{abstract}
%Stuff can go here later if we want!
\end{abstract}
\maketitle





\subsection*{Composition}


Composition is an operation on functions.  It takes two functions and produces a third function.


\[  (Outside \circ Inside)(d) = Outside(Inside(d))    \]


We will examine composition in more detail throughout the course, but right now we are emphasizing composition with linear functions.



For our transformation story, we begin with a function, $f(x)$.


We can compose with a linear function in two ways:



\[
f(linear)
\] 


\[
linear(f)
\]


How does each of these affect $f(x)$?


\textbf{\textcolor{blue!55!black}{Inside:}} $f(a \, x + b)$  


We can view this as a composition:


\[  \text{Let } \, L(t) = a \, t + b   \]


\[   f(a \, x + b)  = (f \circ L)(x)\]



This affects the domain of $f$, which is illustrated as horizontal graphical transformations.





\textbf{\textcolor{blue!55!black}{Outside:}} $a \, f(x) + b$  


We can view this as a composition:


\[  \text{Let } \, L(t) = a \, t + b   \]


\[   a \, f(x) + b  = (L \circ f)(x)\]


This affects the range of $f$, which is illustrated as vertical graphical transformations.









\subsection*{Learning Outcomes}


\begin{sectionOutcomes}
In this section, students will 

\begin{itemize}
\item compose functions with linear functions.
\end{itemize}
\end{sectionOutcomes}











\begin{center}
\textbf{\textcolor{green!50!black}{ooooo-=-=-=-ooOoo-=-=-=-ooooo}} \\

more examples can be found by following this link\\ \link[More Examples of Transforming the Inside]{https://ximera.osu.edu/csccmathematics/precalculus/precalculus/transformations/examples/exampleList}

\end{center}




\end{document}
