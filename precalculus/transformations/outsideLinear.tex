\documentclass{ximera}


\graphicspath{
  {./}
  {ximeraTutorial/}
  {basicPhilosophy/}
}

\newcommand{\mooculus}{\textsf{\textbf{MOOC}\textnormal{\textsf{ULUS}}}}


\usepackage{tkz-euclide}\usepackage{tikz}
\usepackage{tikz-cd}
\usetikzlibrary{arrows}
\tikzset{>=stealth,commutative diagrams/.cd,
  arrow style=tikz,diagrams={>=stealth}} %% cool arrow head
\tikzset{shorten <>/.style={ shorten >=#1, shorten <=#1 } } %% allows shorter vectors

\usetikzlibrary{backgrounds} %% for boxes around graphs
\usetikzlibrary{shapes,positioning}  %% Clouds and stars
\usetikzlibrary{matrix} %% for matrix
\usepgfplotslibrary{polar} %% for polar plots
\usepgfplotslibrary{fillbetween} %% to shade area between curves in TikZ
\usetkzobj{all}
\usepackage[makeroom]{cancel} %% for strike outs
%\usepackage{mathtools} %% for pretty underbrace % Breaks Ximera
%\usepackage{multicol}
\usepackage{pgffor} %% required for integral for loops



%% http://tex.stackexchange.com/questions/66490/drawing-a-tikz-arc-specifying-the-center
%% Draws beach ball
\tikzset{pics/carc/.style args={#1:#2:#3}{code={\draw[pic actions] (#1:#3) arc(#1:#2:#3);}}}



\usepackage{array}
\setlength{\extrarowheight}{+.1cm}
\newdimen\digitwidth
\settowidth\digitwidth{9}
\def\divrule#1#2{
\noalign{\moveright#1\digitwidth
\vbox{\hrule width#2\digitwidth}}}
























%%This is to help with formatting on future title pages.
\newenvironment{sectionOutcomes}{}{}


\title{Outside}

\begin{document}

\begin{abstract}
range
\end{abstract}
\maketitle






\textbf{\textcolor{blue!55!black}{Outside : Range : Vertical}}



We have already seen a couple of versions of composition. 


$\blacktriangleright$ \textbf{\textcolor{purple!85!blue}{Pointwise}} composition was seen via individual numbers: 

\[ (F \circ G)(a) = F(G(a)) \]

The number, $a$, in the domain of $G$ was connected to its range partner, $G(a)$.  $G$ was evaluated at $a$ and the function value, $G(a)$, was then viewed as a member of the domain of $F$.  As a member of the domain of $F$, $F$ can be evaluated at $G(a)$ to get $F(G(a))$. 




$\blacktriangleright$ \textbf{\textcolor{purple!85!blue}{Linear}} composition between two linear functions produced a whole new function - a linear function.  Instead of thinking of domain numbers individually, this composition was viewed as an operation on linear functions.

\[    (L_o \circ L_i)(x) = L_o(L_i(x))  \]

There is an outside function, $L_o(x)$, and an inside function $L_i(x)$.  The composition operation, $\circ$, is applied and a new linear function is created.  



Our symbol for this function is $(L_o \circ L_i)$ or $L_o \circ L_i$.  The parentheses are used to clear up communication.

In our investigations, we have discovered that $(L_o \circ L_i)$ ``is'' $L_o$, just shifted, stretched, and reflected horizontally.


We would like to extend this idea of a function operation beyond linear functions.





\section*{Composition}










\begin{definition} \textbf{\textcolor{green!50!black}{Composition}}  


Given two functions, $Outside$ and $Inside$, the composition, $Outside \circ Inside$ is defined by

\[      (Outside \circ Inside)(a) = Outside(Inside(a))        \]

For $ a \in \{  x \in Dom_{Inside} \, | \,    Inside(x) \in Dom_{Outside}  \}$

\end{definition}



This time, we would like to focus on the outside function as a linear function.




Let $f(x)$ be any function. \\
Let $L(y)$ be any linear function. \\


Form the composition $L(f(z))$. \\

$\blacktriangleright$ How does $L$ affect $f$?




The main issue here is the range of $f$ intersecting the domain of $L$.  However, since the natural domain of a linear function is all real numbers, there shouldn't be a problem.\\






$\star$ \textbf{\textcolor{purple}{Outside = Linear}}


In this section, our $Outside$ function will always be a linear function.



\[      (L \circ Inside)(z) = L(Inside(z))        \]


where $L(x) = a \, x + b$, with $a$ and $b$ real numbers and $a \ne 0$.



Let's consider a quadratic function: $Q(h) = (h+1)(h-4)$ as the $inside$ function.











\begin{image}
\begin{tikzpicture}
  \begin{axis}[
            domain=-10:10, ymax=10, xmax=10, ymin=-10, xmin=-10,
            axis lines =center, xlabel=$h$, ylabel={$y=Q(h)$}, grid = major, grid style={dashed},
            ytick={-10,-8,-6,-4,-2,2,4,6,8,10},
            xtick={-10,-8,-6,-4,-2,2,4,6,8,10},
            yticklabels={$-10$,$-8$,$-6$,$-4$,$-2$,$2$,$4$,$6$,$8$,$10$}, 
            xticklabels={$-10$,$-8$,$-6$,$-4$,$-2$,$2$,$4$,$6$,$8$,$10$},
            ticklabel style={font=\scriptsize},
            every axis y label/.style={at=(current axis.above origin),anchor=south},
            every axis x label/.style={at=(current axis.right of origin),anchor=west},
            axis on top
          ]
          
            
      		\addplot [line width=2, penColor, smooth,samples=200,domain=(-2.5:5.5),<->] {(x+1)*(x-4)};








  \end{axis}
\end{tikzpicture}
\end{image}



From left to right, the range values, or function values, for $Q$ begin very big and positive.  These values decrease to $0$ and continue negative until they reach a value of $-6.25$. Then, they increase to $0$ again and continue to very big and positive values.

Now we will transform these function values with a linear function.


Let $L(t) = -\frac{1}{4} t + 3$ with domain $\mathbb{R}$.


\begin{itemize}
\item $L$ will take a function value from $Q$ and compress it by a factor of $\frac{1}{4}$. Our parabola will be squished vertically a bit. 
\item Then $L$ will negate the values.  This will vertically reflect the parabola over the horizontal axis.
\item Then $L$ will add $3$ to all of the values.  This will shift the parabola up by $3$.
\end{itemize}












\begin{image}
\begin{tikzpicture}
  \begin{axis}[
            domain=-10:10, ymax=10, xmax=10, ymin=-10, xmin=-10,
            axis lines =center, xlabel=$h$, ylabel={$y=(L \circ Q)(h)$}, grid = major, grid style={dashed},
            ytick={-10,-8,-6,-4,-2,2,4,6,8,10},
            xtick={-10,-8,-6,-4,-2,2,4,6,8,10},
            yticklabels={$-10$,$-8$,$-6$,$-4$,$-2$,$2$,$4$,$6$,$8$,$10$}, 
            xticklabels={$-10$,$-8$,$-6$,$-4$,$-2$,$2$,$4$,$6$,$8$,$10$},
            ticklabel style={font=\scriptsize},
            every axis y label/.style={at=(current axis.above origin),anchor=south},
            every axis x label/.style={at=(current axis.right of origin),anchor=west},
            axis on top
          ]
          
            
      		\addplot [line width=2, penColor, smooth,samples=200,domain=(-6:9),<->] {-0.25*(x+1)*(x-4) + 3};








  \end{axis}
\end{tikzpicture}
\end{image}

The vertical measurements have all been processed linearly, which means the shape doesn't change.  It is still a parabola and all of its features are relatively in the same place.



The minimum of $Q$ is $-6.25$ and this occurrs at $1.5$. We are flipping vertically, so the maximum of the composition is still at $1.5$. There were no horizontal transformations.  The maximum is $L(-6.25) = 4.5625$.




The zeros of $L \circ Q$ occur when $L(t) = 0$, which is at $t = 12$.  Therefore, the zeros of $L \circ Q$ occur when $Q(h) = 12$.





$Q(h) = (h+1)(h-4)$



\begin{align*}
Q(h) = (h+1)(h-4)     &  = 12  \\
h^2 - 3h - 4      & = 12   \\
h^2 - 3h - 16 & = 0  \\
\end{align*}

This does not factor easily.  We'll use the quadratic formula.


\[  h = \frac{-(-3) \pm \sqrt{(-3)^2 - 4 \cdot 1 \cdot (-16)}}{2 \cdot 1}  = \frac{3 \pm \sqrt{73}}{2}     \]


We have two zeros: $\frac{3 + \sqrt{73}}{2}  \approx 5.77$ and $\frac{3 - \sqrt{73}}{2} \approx -2.77$, which agrees with our graph.



\begin{claim} Check These are Zeros


\[   Q(h) = (h+1)(h-4)   \]


\[   L(Q(z)) = -\frac{1}{4} (z+1)(z-4) + 3   \]


Let's verify that $\frac{3 + \sqrt{73}}{2}$ is a zero of $L(Q(z))$.



\[   L \left( Q \left( \frac{3 + \sqrt{73}}{2} \right) \right) = -\frac{1}{4} \left( \frac{3 + \sqrt{73}}{2}+1 \right) \left( \frac{3 + \sqrt{73}}{2}-4 \right) + 3    \]



\[  = -\frac{1}{4} \left( \frac{9 + 6 \sqrt{73} + 73}{4} - 3 \cdot \frac{3 + \sqrt{73}}{2} - 4 \right) + 3    \]


\[  = -\frac{1}{4} \left( \frac{82 + 6 \sqrt{73}}{4} - \frac{9 + 3 \sqrt{73}}{2} - 4 \right) + 3    \]


\[  = -\frac{1}{4} \left( \frac{82 + 6 \sqrt{73}}{4} - \frac{18 + 6 \sqrt{73}}{4} - 4 \right) + 3    \]


\[  = -\frac{1}{4} \left( \answer{\frac{48}{4}} \right) + 3    \]


\[  = -3 + 3   = 0 \]


\end{claim}














\begin{claim} Check These are Zeros


\[   Q(h) = (h+1)(h-4)   \]


\[   L(Q(z)) = -\frac{1}{4} (z+1)(z-4) + 3   \]


Let's verify that $\frac{3 - \sqrt{73}}{2}$ is a zero of $L(Q(z))$.



\[   L \left( Q \left( \frac{3 - \sqrt{73}}{2} \right) \right) = -\frac{1}{4} \left( \frac{3 - \sqrt{73}}{2}+1 \right) \left( \frac{3 - \sqrt{73}}{2}-4 \right) + 3    \]



\[  = -\frac{1}{4} \left( \frac{\answer{9 - 6 \sqrt{73} + 73}}{4} - 3 \cdot \frac{3 - \sqrt{73}}{2} - 4 \right) + 3    \]


\[  = -\frac{1}{4} \left( \frac{82 - 6 \sqrt{73}}{4} - \frac{9 - 3 \sqrt{73}}{2} - 4 \right) + 3    \]


\[  = -\frac{1}{4} \left( \frac{82 - 6 \sqrt{73}}{4} - \frac{18 - \answer{6 \sqrt{73}}}{4} - 4 \right) + 3    \]


\[  = -\frac{1}{4} \left( \frac{48}{4} \right) + 3    \]


\[  = -3 + 3   = 0 \]


\end{claim}











































Let $L(t) = -(t+3) - 2$.


Let $K(x)$ be a piecewise defined function defined by 


\[
K(x) = 
\begin{cases}
  2|x+6| - 4         &    \,     \text{ on } \,   [-8,-3)    \\
  -6               &    \,     \text{ on } \,    [-3,1)      \\
  -\frac{3}{2}(x-4)(x-8)    &   \,     \text{ on } \,    (4,8]
\end{cases}
\]



Let $C(z) = (L \circ K)(z) = L(K(z))$ with the implied domain.



\begin{image}
\begin{tikzpicture}
  \begin{axis}[
            domain=-10:10, ymax=10, xmax=10, ymin=-10, xmin=-10,
            axis lines =center, xlabel=$x$, ylabel={$y=K(x)$}, grid = major,
            ytick={-10,-8,-6,-4,-2,2,4,6,8,10},
            xtick={-10,-8,-6,-4,-2,2,4,6,8,10},
            yticklabels={$-10$,$-8$,$-6$,$-4$,$-2$,$2$,$4$,$6$,$8$,$10$}, 
            xticklabels={$-10$,$-8$,$-6$,$-4$,$-2$,$2$,$4$,$6$,$8$,$10$},
            ticklabel style={font=\scriptsize},
            every axis y label/.style={at=(current axis.above origin),anchor=south},
            every axis x label/.style={at=(current axis.right of origin),anchor=west},
            axis on top
          ]
          
          %\addplot [line width=2, penColor2, smooth,samples=100,domain=(-6:2)] {-2*x-3};
            \addplot [line width=2, penColor, smooth,samples=100,domain=(-8:-3)] {2*abs(x+6)-4};
            \addplot[color=penColor,fill=penColor,only marks,mark=*] coordinates{(-8,0)};
            \addplot[color=penColor,fill=white,only marks,mark=*] coordinates{(-3,2)};

            \addplot [line width=2, penColor, smooth,samples=100,domain=(-3:1)] {-6};
            \addplot[color=penColor,fill=penColor,only marks,mark=*] coordinates{(-3,-6)};
            \addplot[color=penColor,fill=white,only marks,mark=*] coordinates{(1,-6)};

            \addplot [line width=2, penColor, smooth,samples=100,domain=(4:8)] {-1.5*(x-4)*(x-8)};
            \addplot[color=penColor,fill=white,only marks,mark=*] coordinates{(4,0)};
            \addplot[color=penColor,fill=penColor,only marks,mark=*] coordinates{(8,0)};



  \end{axis}
\end{tikzpicture}
\end{image}




What is $L$ going to do to this graph?

$L$ is a linear function, which means it is not going to change the shapes.  There will be a ``Vee''.  There will be a horizontal line segment.  There will be a parabola.  $L$ is the outside function, which means it cannot affect horizontal features.  The ``Vee'' will be on the left. The parabola will be on the right.  The horizontal line segment will be in the middle.

$L$ is going to affect the vertical measurements. \\


\begin{itemize}
\item The leading coefficient for $L$ is negative.  The graph is going to be reflected vertically. \\
The ``Vee'' will open down.  The parabola will open up.
\item This reflecting will happen after the graph is move $3$ up.
\item Finally, the graph will be shifted down $2$.

\end{itemize}






Graph of $ y = (L \circ K)(m)$.







\begin{image}
\begin{tikzpicture}
  \begin{axis}[
            domain=-10:10, ymax=10, xmax=10, ymin=-11, xmin=-10,
            axis lines =center, xlabel=$m$, ylabel=$y$, grid = major,
            ytick={-10,-8,-6,-4,-2,2,4,6,8,10},
            xtick={-10,-8,-6,-4,-2,2,4,6,8,10},
            yticklabels={$-10$,$-8$,$-6$,$-4$,$-2$,$2$,$4$,$6$,$8$,$10$}, 
            xticklabels={$-10$,$-8$,$-6$,$-4$,$-2$,$2$,$4$,$6$,$8$,$10$},
            ticklabel style={font=\scriptsize},
            every axis y label/.style={at=(current axis.above origin),anchor=south},
            every axis x label/.style={at=(current axis.right of origin),anchor=west},
            axis on top
          ]
          
          %\addplot [line width=2, penColor2, smooth,samples=100,domain=(-6:2)] {-2*x-3};
            \addplot [line width=2, penColor, smooth,samples=100,domain=(-8:-3)] {-(2*abs(x+6)-4+3)-2};
            \addplot[color=penColor,fill=penColor,only marks,mark=*] coordinates{(-8,-5)};
            \addplot[color=penColor,fill=white,only marks,mark=*] coordinates{(-3,-7)};

            \addplot [line width=2, penColor, smooth,samples=100,domain=(-3:1)] {-(-6+3)-2};
            \addplot[color=penColor,fill=penColor,only marks,mark=*] coordinates{(-3,1)};
            \addplot[color=penColor,fill=white,only marks,mark=*] coordinates{(1,1)};

            \addplot [line width=2, penColor, smooth,samples=100,domain=(4:8)] {-(-1.5*(x-4)*(x-8)+3)-2};
            \addplot[color=penColor,fill=white,only marks,mark=*] coordinates{(4,-5)};
            \addplot[color=penColor,fill=penColor,only marks,mark=*] coordinates{(8,-5)};



  \end{axis}
\end{tikzpicture}
\end{image}





\textbf{Note:} All of the endpoints are the same type.  

\begin{itemize}
\item The short arm of the ``Vee'' is a solid dot on both graphs.
\item The long arm of the ``Vee'' is a hollow dot on both graphs.
\item The end of the horizontal line segment nearest the ``Vee'' is a solid dot on both graphs.
\item The end of the horizontal line segment nearest the parabola is a hollow dot on both graphs.
\item The inside endpoint on the parabola is hollow on both graphs.
\item The outside endpoint on the parabola is solid on both graphs.
\end{itemize}



We can also think of $(L \circ K)(m)$ algebraically:


\[
(L \circ K)(m) = -(K(m)+3) - 2
\]


Let $a$ be in the domain of $L \circ K$. We can think of an existing value of $K(a)$.  $3$ is added to this value. The result is negated and then $2$ is subtracted. \\

Graphically, the coordinates of a point begin as $(a, K(a))$.  Then they go through the same steps: 


\begin{itemize}
\item $(a, K(a))$
\item $(a, K(a) + 3)$
\item $(a, -(K(a) + 3))$
\item $(a, -(K(a) + 3) - 2)$
\end{itemize}



\begin{procedure}
For example, the endpoint $(-3, -6)$ goes through these transformational steps.

\begin{itemize}
\item $(-3, -6)$
\item $\left( \answer{-3},  \answer{-3} \right)$
\item $\left( \answer{-3},  \answer{3} \right)$
\item $\left( \answer{-3},  \answer{1} \right)$
\end{itemize}

The solid endpoint of the horziontal line segment becomes $(-3, 1)$.
\end{procedure}




\begin{procedure}
For example, the corner point $(-6, -4)$ goes through these transformational steps.

\begin{itemize}
\item $(-6, -4)$
\item $\left( \answer{-6},  \answer{-1} \right)$
\item $\left( \answer{-6},  \answer{1} \right)$
\item $\left( \answer{-6},  \answer{-1} \right)$
\end{itemize}

The solid endpoint of the horziontal line segment becomes $(-6, -1)$.
\end{procedure}



































\begin{center}
\textbf{\textcolor{green!50!black}{ooooo-=-=-=-ooOoo-=-=-=-ooooo}} \\

more examples can be found by following this link\\ \link[More Examples of Transforming the Outside]{https://ximera.osu.edu/csccmathematics/precalculus/precalculus/transformations/examples/exampleList}

\end{center}




\end{document}
