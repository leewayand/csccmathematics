\documentclass{ximera}


\graphicspath{
  {./}
  {ximeraTutorial/}
  {basicPhilosophy/}
}

\newcommand{\mooculus}{\textsf{\textbf{MOOC}\textnormal{\textsf{ULUS}}}}


\usepackage{tkz-euclide}\usepackage{tikz}
\usepackage{tikz-cd}
\usetikzlibrary{arrows}
\tikzset{>=stealth,commutative diagrams/.cd,
  arrow style=tikz,diagrams={>=stealth}} %% cool arrow head
\tikzset{shorten <>/.style={ shorten >=#1, shorten <=#1 } } %% allows shorter vectors

\usetikzlibrary{backgrounds} %% for boxes around graphs
\usetikzlibrary{shapes,positioning}  %% Clouds and stars
\usetikzlibrary{matrix} %% for matrix
\usepgfplotslibrary{polar} %% for polar plots
\usepgfplotslibrary{fillbetween} %% to shade area between curves in TikZ
\usetkzobj{all}
\usepackage[makeroom]{cancel} %% for strike outs
%\usepackage{mathtools} %% for pretty underbrace % Breaks Ximera
%\usepackage{multicol}
\usepackage{pgffor} %% required for integral for loops



%% http://tex.stackexchange.com/questions/66490/drawing-a-tikz-arc-specifying-the-center
%% Draws beach ball
\tikzset{pics/carc/.style args={#1:#2:#3}{code={\draw[pic actions] (#1:#3) arc(#1:#2:#3);}}}



\usepackage{array}
\setlength{\extrarowheight}{+.1cm}
\newdimen\digitwidth
\settowidth\digitwidth{9}
\def\divrule#1#2{
\noalign{\moveright#1\digitwidth
\vbox{\hrule width#2\digitwidth}}}
























%%This is to help with formatting on future title pages.
\newenvironment{sectionOutcomes}{}{}


\title{Parentheses}


\begin{document}

\begin{abstract}
grouping
\end{abstract}
\maketitle



We use parentheses. A lot!. \\

We use parentheses so much, that we use them when we aren't using them. \\

We use parentheses so much, that we use them by not using them. \\

Sometimes we use parentheses with arithmetic and there are arithmetic rules available. \\

Sometimes we use parentheses and there is no arithmetic implied. \\


It can be confusing. \\ 









\begin{notation}  \textbf{\textcolor{blue!55!black}{Parentheses}}  \\

Mathematics overuses symbols and allows the context in which the symbol is used to affect their meaning (just like any language).
  \begin{itemize}
    \item \textbf{Grouping:} $(3+5)(6+7)$ is a product.  Grouping expressions might signal multiplication with the multiplication sign omitted.
    \item \textbf{Ordered Pairs:} $(4, 5)$ is an ordered pair and might represent a function pair.
    \item \textbf{Coordinates:} $(4, 5)$ might represent the coordinates of a point in the Cartesian Plane.
    \item \textbf{Intervals:} $(4, 5)$ might repesent the open interval of real numbers from $4$ to $5$.
    \item \textbf{Function Notation:} $Double(5)$ symbolizes a range element that is paired with $5$ from the domain in the \textit{Double} function. It is not multiplication.
  \end{itemize}


The context in which parentheses are used helps the reader interpret the parentheses.  And, these contexts can be intertwined.

\[  ((1+4)(7-6), Double(5))    \]
\end{notation}



There is more to using parentheses than just using parentheses. \\



\subsection*{Boundaries}


Parentheses are mathematical fences.  They identify designated areas, that each keep to themselves. \\

\begin{itemize}
\item You think inside the parentheses. 
\item You think outside the parentheses. 
\end{itemize}



The inside and the outside do not interact, except when there are explcit rules detailing how they interaqct.  The Distributive Property is one of those rules, provided the parentheses are describing multiplication. \\










\subsection*{Distributive Property}


If the inside is a sum or a difference and there is a number immediately to the left of the left parenthesis (or right of the right parenthesis), then we have an arithmetic option.


\begin{notation}  \textbf{\textcolor{blue!55!black}{Distributive Property}} \\


If $A$, $B$, and $C$ are real numbers then,

\[
A \cdot (B + C) = A \cdot B + A \cdot C
\]


\[
A (B + C) = A  B + A  C
\]


\[
A \cdot (B - C) = A \cdot B - A \cdot C
\]


\[
A (B - C) = A  B - A  C
\]




\end{notation}


The distributive property tells us that multiplication \textbf{distributes} over addition.  You can distribute the factor across the addition. \\





\begin{warning}

The distributive property tells us how multiplication distributes over addition (or subtraction). \\


There are no other distributive properties! \\

\begin{itemize}
    \item Exponents \textbf{\textcolor{red!70!black}{DO NOT}} distribute over addition.
    \item Exponents \textbf{\textcolor{red!70!black}{DO NOT}} distribute over subtraction.
    \item Roots \textbf{\textcolor{red!70!black}{DO NOT}} distribute over addition.
    \item Roots \textbf{\textcolor{red!70!black}{DO NOT}} distribute over subtraction.
    \item Addition \textbf{\textcolor{red!70!black}{DO NOT}} distribute over multiplication.
\end{itemize}


It is very easy to see an arrangment of symbols involving parentheses and naturally distribute the symbols throughout the parentheses...VERY EASY. \\

\begin{center}

\textbf{\textcolor{red!70!black}{Stop Doing That !!!}}

\end{center}



If you cannot quote a direct algebra rule that tells you that you can do something, then you can't.  

A feeling is not enough.  There must be a rule that you can quote.


\end{warning}














\subsection*{Function Notation}


If there is a function name immediately to the left of the left parenthesis, then the parentheses are surrounding a domain number.   \\


The function name and the parentheses are actually one symbol, not two. \\


\begin{warning}

Function notation is \textbf{\textcolor{red!70!black}{NOT}} multiplication. \\

You \textbf{\textcolor{red!70!black}{CANNOT}} distribute a function through parentheses.


\end{warning}

You cannot distribute function evaluation. You can distribute number multiplication. \\


The following two examples will illustrate how confusing it can be to react to written symbols without taking into account the context.  











\begin{example}


Let $f$ be a function.  Then, 


\[
f(2 + 3) \ne f(2) + f(3)
\]


\end{example}





\begin{example}


Let $f$ be a real number.  Then, 


\[
f(2 + 3) = f(2) + f(3)
\]


\end{example}


It isn't just the symbols.  It is how they are being used. 


Context is everything.


If you are just reacting to written symbols regardless of what they represent, then you are heading down a frustrating road.





\begin{onlineOnly}
\begin{center}
\textbf{\textcolor{green!50!black}{ooooo-=-=-=-ooOoo-=-=-=-ooooo}} \\

more examples can be found by following this link\\ \link[More Examples of Formulas]{https://ximera.osu.edu/csccmathematics/precalculus/precalculus/formulas/examples/exampleList}

\end{center}
\end{onlineOnly}







\end{document}
