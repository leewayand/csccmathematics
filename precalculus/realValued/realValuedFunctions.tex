\documentclass{ximera}

%\usepackage{todonotes}

\newcommand{\todo}{}

\usepackage{esint} % for \oiint
\ifxake%%https://math.meta.stackexchange.com/questions/9973/how-do-you-render-a-closed-surface-double-integral
\renewcommand{\oiint}{{\large\bigcirc}\kern-1.56em\iint}
\fi


\graphicspath{
  {./}
  {ximeraTutorial/}
  {basicPhilosophy/}
  {functionsOfSeveralVariables/}
  {normalVectors/}
  {lagrangeMultipliers/}
  {vectorFields/}
  {greensTheorem/}
  {shapeOfThingsToCome/}
  {dotProducts/}
  {partialDerivativesAndTheGradientVector/}
  {../productAndQuotientRules/exercises/}
  {../normalVectors/exercisesParametricPlots/}
  {../continuityOfFunctionsOfSeveralVariables/exercises/}
  {../partialDerivativesAndTheGradientVector/exercises/}
  {../directionalDerivativeAndChainRule/exercises/}
  {../commonCoordinates/exercisesCylindricalCoordinates/}
  {../commonCoordinates/exercisesSphericalCoordinates/}
  {../greensTheorem/exercisesCurlAndLineIntegrals/}
  {../greensTheorem/exercisesDivergenceAndLineIntegrals/}
  {../shapeOfThingsToCome/exercisesDivergenceTheorem/}
  {../greensTheorem/}
  {../shapeOfThingsToCome/}
  {../separableDifferentialEquations/exercises/}
  {vectorFields/}
}

\newcommand{\mooculus}{\textsf{\textbf{MOOC}\textnormal{\textsf{ULUS}}}}

\usepackage{tkz-euclide}
\usepackage{tikz}
\usepackage{tikz-cd}
\usetikzlibrary{arrows}
\tikzset{>=stealth,commutative diagrams/.cd,
  arrow style=tikz,diagrams={>=stealth}} %% cool arrow head
\tikzset{shorten <>/.style={ shorten >=#1, shorten <=#1 } } %% allows shorter vectors

\usetikzlibrary{backgrounds} %% for boxes around graphs
\usetikzlibrary{shapes,positioning}  %% Clouds and stars
\usetikzlibrary{matrix} %% for matrix
\usepgfplotslibrary{polar} %% for polar plots
\usepgfplotslibrary{fillbetween} %% to shade area between curves in TikZ
%\usetkzobj{all}
\usepackage[makeroom]{cancel} %% for strike outs
%\usepackage{mathtools} %% for pretty underbrace % Breaks Ximera
%\usepackage{multicol}
\usepackage{pgffor} %% required for integral for loops



%% http://tex.stackexchange.com/questions/66490/drawing-a-tikz-arc-specifying-the-center
%% Draws beach ball
\tikzset{pics/carc/.style args={#1:#2:#3}{code={\draw[pic actions] (#1:#3) arc(#1:#2:#3);}}}



\usepackage{array}
\setlength{\extrarowheight}{+.1cm}
\newdimen\digitwidth
\settowidth\digitwidth{9}
\def\divrule#1#2{
\noalign{\moveright#1\digitwidth
\vbox{\hrule width#2\digitwidth}}}




% \newcommand{\RR}{\mathbb R}
% \newcommand{\R}{\mathbb R}
% \newcommand{\N}{\mathbb N}
% \newcommand{\Z}{\mathbb Z}

\newcommand{\sagemath}{\textsf{SageMath}}


%\renewcommand{\d}{\,d\!}
%\renewcommand{\d}{\mathop{}\!d}
%\newcommand{\dd}[2][]{\frac{\d #1}{\d #2}}
%\newcommand{\pp}[2][]{\frac{\partial #1}{\partial #2}}
% \renewcommand{\l}{\ell}
%\newcommand{\ddx}{\frac{d}{\d x}}

% \newcommand{\zeroOverZero}{\ensuremath{\boldsymbol{\tfrac{0}{0}}}}
%\newcommand{\inftyOverInfty}{\ensuremath{\boldsymbol{\tfrac{\infty}{\infty}}}}
%\newcommand{\zeroOverInfty}{\ensuremath{\boldsymbol{\tfrac{0}{\infty}}}}
%\newcommand{\zeroTimesInfty}{\ensuremath{\small\boldsymbol{0\cdot \infty}}}
%\newcommand{\inftyMinusInfty}{\ensuremath{\small\boldsymbol{\infty - \infty}}}
%\newcommand{\oneToInfty}{\ensuremath{\boldsymbol{1^\infty}}}
%\newcommand{\zeroToZero}{\ensuremath{\boldsymbol{0^0}}}
%\newcommand{\inftyToZero}{\ensuremath{\boldsymbol{\infty^0}}}



% \newcommand{\numOverZero}{\ensuremath{\boldsymbol{\tfrac{\#}{0}}}}
% \newcommand{\dfn}{\textbf}
% \newcommand{\unit}{\,\mathrm}
% \newcommand{\unit}{\mathop{}\!\mathrm}
% \newcommand{\eval}[1]{\bigg[ #1 \bigg]}
% \newcommand{\seq}[1]{\left( #1 \right)}
% \renewcommand{\epsilon}{\varepsilon}
% \renewcommand{\phi}{\varphi}


% \renewcommand{\iff}{\Leftrightarrow}

% \DeclareMathOperator{\arccot}{arccot}
% \DeclareMathOperator{\arcsec}{arcsec}
% \DeclareMathOperator{\arccsc}{arccsc}
% \DeclareMathOperator{\si}{Si}
% \DeclareMathOperator{\scal}{scal}
% \DeclareMathOperator{\sign}{sign}


%% \newcommand{\tightoverset}[2]{% for arrow vec
%%   \mathop{#2}\limits^{\vbox to -.5ex{\kern-0.75ex\hbox{$#1$}\vss}}}
% \newcommand{\arrowvec}[1]{{\overset{\rightharpoonup}{#1}}}
% \renewcommand{\vec}[1]{\arrowvec{\mathbf{#1}}}
% \renewcommand{\vec}[1]{{\overset{\boldsymbol{\rightharpoonup}}{\mathbf{#1}}}}

% \newcommand{\point}[1]{\left(#1\right)} %this allows \vector{ to be changed to \vector{ with a quick find and replace
% \newcommand{\pt}[1]{\mathbf{#1}} %this allows \vec{ to be changed to \vec{ with a quick find and replace
% \newcommand{\Lim}[2]{\lim_{\point{#1} \to \point{#2}}} %Bart, I changed this to point since I want to use it.  It runs through both of the exercise and exerciseE files in limits section, which is why it was in each document to start with.

% \DeclareMathOperator{\proj}{\mathbf{proj}}
% \newcommand{\veci}{{\boldsymbol{\hat{\imath}}}}
% \newcommand{\vecj}{{\boldsymbol{\hat{\jmath}}}}
% \newcommand{\veck}{{\boldsymbol{\hat{k}}}}
% \newcommand{\vecl}{\vec{\boldsymbol{\l}}}
% \newcommand{\uvec}[1]{\mathbf{\hat{#1}}}
% \newcommand{\utan}{\mathbf{\hat{t}}}
% \newcommand{\unormal}{\mathbf{\hat{n}}}
% \newcommand{\ubinormal}{\mathbf{\hat{b}}}

% \newcommand{\dotp}{\bullet}
% \newcommand{\cross}{\boldsymbol\times}
% \newcommand{\grad}{\boldsymbol\nabla}
% \newcommand{\divergence}{\grad\dotp}
% \newcommand{\curl}{\grad\cross}
%\DeclareMathOperator{\divergence}{divergence}
%\DeclareMathOperator{\curl}[1]{\grad\cross #1}
% \newcommand{\lto}{\mathop{\longrightarrow\,}\limits}

% \renewcommand{\bar}{\overline}

\colorlet{textColor}{black}
\colorlet{background}{white}
\colorlet{penColor}{blue!50!black} % Color of a curve in a plot
\colorlet{penColor2}{red!50!black}% Color of a curve in a plot
\colorlet{penColor3}{red!50!blue} % Color of a curve in a plot
\colorlet{penColor4}{green!50!black} % Color of a curve in a plot
\colorlet{penColor5}{orange!80!black} % Color of a curve in a plot
\colorlet{penColor6}{yellow!70!black} % Color of a curve in a plot
\colorlet{fill1}{penColor!20} % Color of fill in a plot
\colorlet{fill2}{penColor2!20} % Color of fill in a plot
\colorlet{fillp}{fill1} % Color of positive area
\colorlet{filln}{penColor2!20} % Color of negative area
\colorlet{fill3}{penColor3!20} % Fill
\colorlet{fill4}{penColor4!20} % Fill
\colorlet{fill5}{penColor5!20} % Fill
\colorlet{gridColor}{gray!50} % Color of grid in a plot

\newcommand{\surfaceColor}{violet}
\newcommand{\surfaceColorTwo}{redyellow}
\newcommand{\sliceColor}{greenyellow}




\pgfmathdeclarefunction{gauss}{2}{% gives gaussian
  \pgfmathparse{1/(#2*sqrt(2*pi))*exp(-((x-#1)^2)/(2*#2^2))}%
}


%%%%%%%%%%%%%
%% Vectors
%%%%%%%%%%%%%

%% Simple horiz vectors
\renewcommand{\vector}[1]{\left\langle #1\right\rangle}


%% %% Complex Horiz Vectors with angle brackets
%% \makeatletter
%% \renewcommand{\vector}[2][ , ]{\left\langle%
%%   \def\nextitem{\def\nextitem{#1}}%
%%   \@for \el:=#2\do{\nextitem\el}\right\rangle%
%% }
%% \makeatother

%% %% Vertical Vectors
%% \def\vector#1{\begin{bmatrix}\vecListA#1,,\end{bmatrix}}
%% \def\vecListA#1,{\if,#1,\else #1\cr \expandafter \vecListA \fi}

%%%%%%%%%%%%%
%% End of vectors
%%%%%%%%%%%%%

%\newcommand{\fullwidth}{}
%\newcommand{\normalwidth}{}



%% makes a snazzy t-chart for evaluating functions
%\newenvironment{tchart}{\rowcolors{2}{}{background!90!textColor}\array}{\endarray}

%%This is to help with formatting on future title pages.
\newenvironment{sectionOutcomes}{}{}



%% Flowchart stuff
%\tikzstyle{startstop} = [rectangle, rounded corners, minimum width=3cm, minimum height=1cm,text centered, draw=black]
%\tikzstyle{question} = [rectangle, minimum width=3cm, minimum height=1cm, text centered, draw=black]
%\tikzstyle{decision} = [trapezium, trapezium left angle=70, trapezium right angle=110, minimum width=3cm, minimum height=1cm, text centered, draw=black]
%\tikzstyle{question} = [rectangle, rounded corners, minimum width=3cm, minimum height=1cm,text centered, draw=black]
%\tikzstyle{process} = [rectangle, minimum width=3cm, minimum height=1cm, text centered, draw=black]
%\tikzstyle{decision} = [trapezium, trapezium left angle=70, trapezium right angle=110, minimum width=3cm, minimum height=1cm, text centered, draw=black]


\title{Real-Valued Functions}

\begin{document}

\begin{abstract}
connecting reals
\end{abstract}
\maketitle




\section*{Real-Valued Functions}

For the most part, our attention in this course is focused on real-valued functions.




\begin{definition} \textbf{\textcolor{green!50!black}{Real-Valued Function}} \\

A \textbf{real-valued function} is one whose range is a subset of the real numbers.

\end{definition}
The values of a real-valued function are real numbers.







\begin{example} Squaring Function \\

Let the function $SQ$ be defined as follows.


\begin{itemize}
\item Domain of $SQ$ is $(-3, 5]$.
\item Codomain of $SQ$ is $[-30, 30)$.
\item $SQ$ pairs a domain number with its square.
\end{itemize}


First, this function is well-defined, since the square of any number in $(-3, 5]$ will be in $[-30, 30)$ and each domain number has exactly one square.


\textbf{Shorthand Notation: } $SQ: (-3, 5] \mapsto [-30, 30)$.

\begin{question}
Evaluate the following:

\begin{itemize}
	\item $SQ(-2) = \answer{4}$
	\item $SQ(0) = \answer{0}$
	\item $SQ(1.1) = \answer{1.21}$
	\item $SQ\left(\frac{7}{5}\right) = \answer{\frac{49}{25}}$
\end{itemize}

\end{question}




\begin{question} Range \\

The range of $SQ$ is $\left[ \answer{0}, \answer{25} \right]$.

\end{question}

$SQ$ is not an onto function.  For instance, $-1 \in [-30,30)$, yet $-1$ is not in the range, since squares of real numbers cannot be negative. \\


$SQ$ is not a one-to-one function, since $SQ(-1)=SQ(1)$.
\end{example}











\begin{example} Collatz \\

Let the function $C$ be defined as follows.


\begin{itemize}
\item Domain of $C$ is the positive integers: $\mathbb{N}$.
\item Codomain of $C$ is the positive integers: $\mathbb{N}$.
\item $C$ pairs a domain number with a range number according to the following rule:
	\begin{itemize}
			\item If the domain number is even, then $C$ pairs it with half the domain number:
			\item If the domain number is odd, then $C$ pairs it with one more than three times the domain number.
	\end{itemize}
\end{itemize}


First, this function is well-defined, since the calculations can only produce one result.


\textbf{Shorthand Notation: } $C: \mathbb{N} \mapsto \mathbb{N}$.

\begin{question}
Evaluate the following:

\begin{itemize}
	\item $C(8) = \answer{4}$
	\item $C(7) = \answer{22}$
	\item $C(1) = \answer{4}$
	\item $C(28) = \answer{14}$
\end{itemize}

\end{question}



$C$ is not a one-to-one function since $C(8) = C(1)$.


\end{example}









\begin{example} Identity Function \\

Let the function $Id$ be defined as follows.


\begin{itemize}
\item Domain of $Id$ is all real numbers: $\mathbb{R}$.
\item Codomain of $Id$ is all real numbers: $\mathbb{R}$.
\item $Id$ pairs a domain number with itself.
\end{itemize}


First, this function is well-defined.


\textbf{Shorthand Notation: } $Id: \mathbb{R} \mapsto \mathbb{R}$.

\begin{question}
Evaluate the following:

\begin{itemize}
	\item $Id(\pi) = \answer{\pi}$
	\item $Id(\sqrt{5}) = \answer{\sqrt{5}}$
	\item $Id\left(\frac{13}{27}\right) = \answer{\frac{13}{27}}$
	\item $Id(0) = \answer{0}$
\end{itemize}

\end{question}





The Identity function is an onto function.  If $r \in \mathbb{R}$, the range, then $Id(r) = r$. \\


The Identity function is a one-to-one function, since if $Id(r)=Id(s)$, then $r = s$.  If two $Id$ values are equal, then the domain numbers are equal. They were not different domain numbers.





\end{example}













\begin{example} Remainder \\

Let the function $Remainder$ be defined as follows.


\begin{itemize}
\item Domain of $Remainder$ is all natural numbers: $\mathbb{N}$.
\item Codomain of $Remainder$ is $\{ 0, 1, 2, 3, 4, 5, 6, 7, 8, 9 \}$.
\item $Remainder$ pairs a natural number with the remainder when divided by $10$.
\end{itemize}


First, this function is well-defined. Dividing by $10$ can only have one remainder.


\textbf{Shorthand Notation: } $Remainder: \mathbb{N} \mapsto \{ 0, 1, 2, 3, 4, 5, 6, 7, 8, 9 \}$.



\begin{example}

\begin{itemize}
	\item $Remainder(12) = 2$
	\item $Remainder(25) = 5$
	\item $Remainder(77) = 7$
	\item $Remainder(12 + 11) = 3$
	\item $Remainder(7 \cdot 7) = 9$
\end{itemize}

\end{example}







\begin{question}
Evaluate the following:

\begin{itemize}
	\item $Remainder(101) = \answer{1}$
	\item $Remainder(3) = \answer{3}$
	\item $Remainder(1652435) = \answer{5}$
	\item $Remainder(94) = \answer{4}$
\end{itemize}

\end{question}



\end{example}










\begin{example} Linear \\

Let the function $L$ be defined as follows.


\begin{itemize}
\item Domain of $L$ is $[-3, 4)$.
\item Codomain of $L$ is $\mathbb{R}$.
\item $L$ pairs a number with twice the number.
\end{itemize}


First, this function is well-defined. 


\textbf{Shorthand Notation: } $L: [-3, 4) \mapsto \mathbb{R}$.

\begin{question}

The codomain of $L$ is $\mathbb{R}$ but the range is not.  

The range is \wordChoice{\choice[correct]{[}\choice{(}} \wordChoice{\choice{-8}\choice[correct]{-6},\choice{6}\choice{8}} , \wordChoice{\choice{-8}\choice{-6},\choice{6}\choice[correct]{8}} \wordChoice{\choice{]}\choice[correct]{)}}.

\end{question}



\end{example}
...more communication. \\






\subsection*{Communication Summary}







The range is also called the \textbf{image} of the function.   

Sometimes the range partner of a domain number is called the \textbf{image} of the domain number.

$f(a)$ is called the ``value of $f$ at $a$'' or ``the image of $a$ under $f$''.


$f(a)$ is pronounced ``$f$ of $a$''.


And, we have the reverse direction.

The preimage of a subset of the codomain consists of the domain members whose function values are inside the given subset.

The \textbf{preimage} of $S$ is the set

\[  f^{-1}(S) = \{ d \in D \, | \, f(d) \in S  \}    \]

\textbf{Notation:}  The $-1$ exponent does not mean reciprocal.  Instead, it is conveying an ``opposite'' direction.


The preimage of the range is the domain. \\

Sometimes when the preimage is a single domain member, then we drop the idea of a set and just quote that one domain member. \\


\textbf{Note:} The preimage of a codomain number, which is not in the range, is the empty set, $\emptyset$. \\























\begin{onlineOnly}
\begin{center}
\textbf{\textcolor{green!50!black}{ooooo-=-=-=-ooOoo-=-=-=-ooooo}} \\

more examples can be found by following this link\\ \link[More Examples of Real-Valued Functions]{https://ximera.osu.edu/csccmathematics/precalculus/precalculus/realValued/examples/exampleList}

\end{center}
\end{onlineOnly}





\end{document}
