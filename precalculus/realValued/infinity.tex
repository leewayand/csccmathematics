\documentclass{ximera}

%\usepackage{todonotes}

\newcommand{\todo}{}

\usepackage{esint} % for \oiint
\ifxake%%https://math.meta.stackexchange.com/questions/9973/how-do-you-render-a-closed-surface-double-integral
\renewcommand{\oiint}{{\large\bigcirc}\kern-1.56em\iint}
\fi


\graphicspath{
  {./}
  {ximeraTutorial/}
  {basicPhilosophy/}
  {functionsOfSeveralVariables/}
  {normalVectors/}
  {lagrangeMultipliers/}
  {vectorFields/}
  {greensTheorem/}
  {shapeOfThingsToCome/}
  {dotProducts/}
  {partialDerivativesAndTheGradientVector/}
  {../productAndQuotientRules/exercises/}
  {../normalVectors/exercisesParametricPlots/}
  {../continuityOfFunctionsOfSeveralVariables/exercises/}
  {../partialDerivativesAndTheGradientVector/exercises/}
  {../directionalDerivativeAndChainRule/exercises/}
  {../commonCoordinates/exercisesCylindricalCoordinates/}
  {../commonCoordinates/exercisesSphericalCoordinates/}
  {../greensTheorem/exercisesCurlAndLineIntegrals/}
  {../greensTheorem/exercisesDivergenceAndLineIntegrals/}
  {../shapeOfThingsToCome/exercisesDivergenceTheorem/}
  {../greensTheorem/}
  {../shapeOfThingsToCome/}
  {../separableDifferentialEquations/exercises/}
  {vectorFields/}
}

\newcommand{\mooculus}{\textsf{\textbf{MOOC}\textnormal{\textsf{ULUS}}}}

\usepackage{tkz-euclide}
\usepackage{tikz}
\usepackage{tikz-cd}
\usetikzlibrary{arrows}
\tikzset{>=stealth,commutative diagrams/.cd,
  arrow style=tikz,diagrams={>=stealth}} %% cool arrow head
\tikzset{shorten <>/.style={ shorten >=#1, shorten <=#1 } } %% allows shorter vectors

\usetikzlibrary{backgrounds} %% for boxes around graphs
\usetikzlibrary{shapes,positioning}  %% Clouds and stars
\usetikzlibrary{matrix} %% for matrix
\usepgfplotslibrary{polar} %% for polar plots
\usepgfplotslibrary{fillbetween} %% to shade area between curves in TikZ
%\usetkzobj{all}
\usepackage[makeroom]{cancel} %% for strike outs
%\usepackage{mathtools} %% for pretty underbrace % Breaks Ximera
%\usepackage{multicol}
\usepackage{pgffor} %% required for integral for loops



%% http://tex.stackexchange.com/questions/66490/drawing-a-tikz-arc-specifying-the-center
%% Draws beach ball
\tikzset{pics/carc/.style args={#1:#2:#3}{code={\draw[pic actions] (#1:#3) arc(#1:#2:#3);}}}



\usepackage{array}
\setlength{\extrarowheight}{+.1cm}
\newdimen\digitwidth
\settowidth\digitwidth{9}
\def\divrule#1#2{
\noalign{\moveright#1\digitwidth
\vbox{\hrule width#2\digitwidth}}}




% \newcommand{\RR}{\mathbb R}
% \newcommand{\R}{\mathbb R}
% \newcommand{\N}{\mathbb N}
% \newcommand{\Z}{\mathbb Z}

\newcommand{\sagemath}{\textsf{SageMath}}


%\renewcommand{\d}{\,d\!}
%\renewcommand{\d}{\mathop{}\!d}
%\newcommand{\dd}[2][]{\frac{\d #1}{\d #2}}
%\newcommand{\pp}[2][]{\frac{\partial #1}{\partial #2}}
% \renewcommand{\l}{\ell}
%\newcommand{\ddx}{\frac{d}{\d x}}

% \newcommand{\zeroOverZero}{\ensuremath{\boldsymbol{\tfrac{0}{0}}}}
%\newcommand{\inftyOverInfty}{\ensuremath{\boldsymbol{\tfrac{\infty}{\infty}}}}
%\newcommand{\zeroOverInfty}{\ensuremath{\boldsymbol{\tfrac{0}{\infty}}}}
%\newcommand{\zeroTimesInfty}{\ensuremath{\small\boldsymbol{0\cdot \infty}}}
%\newcommand{\inftyMinusInfty}{\ensuremath{\small\boldsymbol{\infty - \infty}}}
%\newcommand{\oneToInfty}{\ensuremath{\boldsymbol{1^\infty}}}
%\newcommand{\zeroToZero}{\ensuremath{\boldsymbol{0^0}}}
%\newcommand{\inftyToZero}{\ensuremath{\boldsymbol{\infty^0}}}



% \newcommand{\numOverZero}{\ensuremath{\boldsymbol{\tfrac{\#}{0}}}}
% \newcommand{\dfn}{\textbf}
% \newcommand{\unit}{\,\mathrm}
% \newcommand{\unit}{\mathop{}\!\mathrm}
% \newcommand{\eval}[1]{\bigg[ #1 \bigg]}
% \newcommand{\seq}[1]{\left( #1 \right)}
% \renewcommand{\epsilon}{\varepsilon}
% \renewcommand{\phi}{\varphi}


% \renewcommand{\iff}{\Leftrightarrow}

% \DeclareMathOperator{\arccot}{arccot}
% \DeclareMathOperator{\arcsec}{arcsec}
% \DeclareMathOperator{\arccsc}{arccsc}
% \DeclareMathOperator{\si}{Si}
% \DeclareMathOperator{\scal}{scal}
% \DeclareMathOperator{\sign}{sign}


%% \newcommand{\tightoverset}[2]{% for arrow vec
%%   \mathop{#2}\limits^{\vbox to -.5ex{\kern-0.75ex\hbox{$#1$}\vss}}}
% \newcommand{\arrowvec}[1]{{\overset{\rightharpoonup}{#1}}}
% \renewcommand{\vec}[1]{\arrowvec{\mathbf{#1}}}
% \renewcommand{\vec}[1]{{\overset{\boldsymbol{\rightharpoonup}}{\mathbf{#1}}}}

% \newcommand{\point}[1]{\left(#1\right)} %this allows \vector{ to be changed to \vector{ with a quick find and replace
% \newcommand{\pt}[1]{\mathbf{#1}} %this allows \vec{ to be changed to \vec{ with a quick find and replace
% \newcommand{\Lim}[2]{\lim_{\point{#1} \to \point{#2}}} %Bart, I changed this to point since I want to use it.  It runs through both of the exercise and exerciseE files in limits section, which is why it was in each document to start with.

% \DeclareMathOperator{\proj}{\mathbf{proj}}
% \newcommand{\veci}{{\boldsymbol{\hat{\imath}}}}
% \newcommand{\vecj}{{\boldsymbol{\hat{\jmath}}}}
% \newcommand{\veck}{{\boldsymbol{\hat{k}}}}
% \newcommand{\vecl}{\vec{\boldsymbol{\l}}}
% \newcommand{\uvec}[1]{\mathbf{\hat{#1}}}
% \newcommand{\utan}{\mathbf{\hat{t}}}
% \newcommand{\unormal}{\mathbf{\hat{n}}}
% \newcommand{\ubinormal}{\mathbf{\hat{b}}}

% \newcommand{\dotp}{\bullet}
% \newcommand{\cross}{\boldsymbol\times}
% \newcommand{\grad}{\boldsymbol\nabla}
% \newcommand{\divergence}{\grad\dotp}
% \newcommand{\curl}{\grad\cross}
%\DeclareMathOperator{\divergence}{divergence}
%\DeclareMathOperator{\curl}[1]{\grad\cross #1}
% \newcommand{\lto}{\mathop{\longrightarrow\,}\limits}

% \renewcommand{\bar}{\overline}

\colorlet{textColor}{black}
\colorlet{background}{white}
\colorlet{penColor}{blue!50!black} % Color of a curve in a plot
\colorlet{penColor2}{red!50!black}% Color of a curve in a plot
\colorlet{penColor3}{red!50!blue} % Color of a curve in a plot
\colorlet{penColor4}{green!50!black} % Color of a curve in a plot
\colorlet{penColor5}{orange!80!black} % Color of a curve in a plot
\colorlet{penColor6}{yellow!70!black} % Color of a curve in a plot
\colorlet{fill1}{penColor!20} % Color of fill in a plot
\colorlet{fill2}{penColor2!20} % Color of fill in a plot
\colorlet{fillp}{fill1} % Color of positive area
\colorlet{filln}{penColor2!20} % Color of negative area
\colorlet{fill3}{penColor3!20} % Fill
\colorlet{fill4}{penColor4!20} % Fill
\colorlet{fill5}{penColor5!20} % Fill
\colorlet{gridColor}{gray!50} % Color of grid in a plot

\newcommand{\surfaceColor}{violet}
\newcommand{\surfaceColorTwo}{redyellow}
\newcommand{\sliceColor}{greenyellow}




\pgfmathdeclarefunction{gauss}{2}{% gives gaussian
  \pgfmathparse{1/(#2*sqrt(2*pi))*exp(-((x-#1)^2)/(2*#2^2))}%
}


%%%%%%%%%%%%%
%% Vectors
%%%%%%%%%%%%%

%% Simple horiz vectors
\renewcommand{\vector}[1]{\left\langle #1\right\rangle}


%% %% Complex Horiz Vectors with angle brackets
%% \makeatletter
%% \renewcommand{\vector}[2][ , ]{\left\langle%
%%   \def\nextitem{\def\nextitem{#1}}%
%%   \@for \el:=#2\do{\nextitem\el}\right\rangle%
%% }
%% \makeatother

%% %% Vertical Vectors
%% \def\vector#1{\begin{bmatrix}\vecListA#1,,\end{bmatrix}}
%% \def\vecListA#1,{\if,#1,\else #1\cr \expandafter \vecListA \fi}

%%%%%%%%%%%%%
%% End of vectors
%%%%%%%%%%%%%

%\newcommand{\fullwidth}{}
%\newcommand{\normalwidth}{}



%% makes a snazzy t-chart for evaluating functions
%\newenvironment{tchart}{\rowcolors{2}{}{background!90!textColor}\array}{\endarray}

%%This is to help with formatting on future title pages.
\newenvironment{sectionOutcomes}{}{}



%% Flowchart stuff
%\tikzstyle{startstop} = [rectangle, rounded corners, minimum width=3cm, minimum height=1cm,text centered, draw=black]
%\tikzstyle{question} = [rectangle, minimum width=3cm, minimum height=1cm, text centered, draw=black]
%\tikzstyle{decision} = [trapezium, trapezium left angle=70, trapezium right angle=110, minimum width=3cm, minimum height=1cm, text centered, draw=black]
%\tikzstyle{question} = [rectangle, rounded corners, minimum width=3cm, minimum height=1cm,text centered, draw=black]
%\tikzstyle{process} = [rectangle, minimum width=3cm, minimum height=1cm, text centered, draw=black]
%\tikzstyle{decision} = [trapezium, trapezium left angle=70, trapezium right angle=110, minimum width=3cm, minimum height=1cm, text centered, draw=black]


\title{$\infty$}

\begin{document}

\begin{abstract}
not a number
\end{abstract}
\maketitle




\subsection*{Infinity: $\infty$}


\begin{center}
\textbf{\textcolor{red!80!black}{Infinity is not a real number.}}
\end{center}




This course presents a study of real numbers and real-valued functions. That doesn't include $\infty$.




\textbf{\textcolor{red!90!darkgray}{$\blacktriangleright$ }} You \textbf{\textcolor{red!80!black}{CANNOT}} perform arithmetic with $\infty$.

Addition, subtraction, multiplication, and division are operations for real numbers.  Since $\infty$ is not a real number, it CANNOT be involved with any of these operations


\begin{example}


\begin{itemize}
\item $\infty + \infty \ne \infty$
\item $\infty - \infty \ne 0$
\item $\infty \cdot \infty \ne \infty$
\item $\infty \div \infty \ne 1$
\item $\infty^{\infty} \ne \infty$
\end{itemize}


\end{example}

Our operations are strictly for real numbers, only.









\textbf{\textcolor{red!90!darkgray}{$\blacktriangleright$ }} You \textbf{\textcolor{red!80!black}{CANNOT}} form fractions with $\infty$.



\begin{example}


\begin{itemize}
\item $\frac{\infty}{\infty} \ne \infty$
\item $\frac{\infty}{\infty} \ne 1$
\item $\frac{\infty}{\infty} \ne 0$
\item $\frac{1}{\infty} \ne 0$
\item $\frac{\infty}{1} \ne \infty$
\end{itemize}


\end{example}








\textbf{\textcolor{red!90!darkgray}{$\blacktriangleright$ }} You \textbf{\textcolor{red!80!black}{CANNOT}} evaluate functions at $\infty$.



\begin{example}


Let $f(x) = 3x^2 + 5x + 9$, then $f(\infty) \ne \infty$ \\

Let $g(x) = \frac{6x + 5}{3x - 1}$, then $g(\infty) \ne 2$ \\

Let $h(x) = \frac{4}{7x + 2}$, then $h(\infty) \ne 0$ \\

\end{example}











\textbf{\textcolor{red!90!darkgray}{$\blacktriangleright$ }} $\infty$ \textbf{\textcolor{red!80!black}{CANNOT}} be a function value.



\begin{example}



Let $k(x) = \frac{8}{3x + 6}$, then $k(-2) \ne \infty$ \\

\end{example}

$\infty$ is not a real number and cannot be treated like a real number in any way.








\subsection*{Infinity: What is it?}


If $\infty$ is not a real number, then what is it?


The problem here is the question itself.  The question presupposes that $\infty$ is a mathematical object.  For us, it isn't.   

For us, $\infty$ is simply shorthand communication.  It is the shorthand symbol we use to let readers know that we have encountered an unbounded situation. Unbounded meaning there is no number greater than all of the values we are examining.


If you keep studying mathematics, especially logic, then $\infty$ might become an object, perhaps with its own operations.

However, for us, it is shorthand communication that describes the \textbf{SIZE} of a set of values we are analyzing.







\begin{example}


\begin{itemize}
\item The interval $(5, \infty)$ describes the set of all real numbers greater than $5$.  $\infty$ tells us that there is no number greater than all of the numbers in this set.



\item The interval $(-\infty, 4)$ describes the set of all real numbers less than $4$.  $-\infty$ tells us that there is no number less than all of the numbers in this set.
\end{itemize}




\end{example}
In interval notation, $\infty$ gets a parenthesis because it is not a real number and cannot be included in any set of numbers.






\begin{question}


\[  \infty \in \mathbb{R}  \]


\begin{multipleChoice}
\choice {True}
\choice [correct]{False}
\end{multipleChoice}


\end{question}





\begin{question}


\[  \infty \notin \mathbb{R}  \]


\begin{multipleChoice}
\choice [correct]{True}
\choice {False}
\end{multipleChoice}


\end{question}











\begin{question}


\[  \infty + \infty = \infty \]


\begin{multipleChoice}
\choice {True}
\choice [correct]{False}
\end{multipleChoice}


\end{question}










\begin{question}


\[  \infty \cdot \infty = \infty \]


\begin{multipleChoice}
\choice {True}
\choice [correct]{False}
\end{multipleChoice}


\end{question}









\begin{question}


\[  \infty^2 = \infty \]


\begin{multipleChoice}
\choice {True}
\choice [correct]{False}
\end{multipleChoice}


\end{question}










\begin{question}


\[  \frac{1}{\infty} = 0 \]


\begin{multipleChoice}
\choice {True}
\choice [correct]{False}
\end{multipleChoice}


\end{question}











\item \textbf{\textcolor{purple!85!blue}{A Peek Ahead}} 


With functions, we will investigate the relationships between sets of information (domain and range).  We will be particularly interested in how the values in the range change compared to how domain values change. \\

One aspect of this is how the function values change as the domain values become unbounded and ``approach'' $\infty$. \\

We will need language to talk algebaically and rigorously about $\infty$. We have such language.  \\


Our language is called \textbf{limits} and we will use limits extensively.\\











\begin{onlineOnly}
\begin{center}
\textbf{\textcolor{green!50!black}{ooooo-=-=-=-ooOoo-=-=-=-ooooo}} \\

more examples can be found by following this link\\ \link[More Examples of Real-Valued Functions]{https://ximera.osu.edu/csccmathematics/precalculus/precalculus/realValued/examples/exampleList}

\end{center}
\end{onlineOnly}












\end{document}

