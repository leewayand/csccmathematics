\documentclass{ximera}


\graphicspath{
  {./}
  {ximeraTutorial/}
  {basicPhilosophy/}
}

\newcommand{\mooculus}{\textsf{\textbf{MOOC}\textnormal{\textsf{ULUS}}}}


\usepackage{tkz-euclide}\usepackage{tikz}
\usepackage{tikz-cd}
\usetikzlibrary{arrows}
\tikzset{>=stealth,commutative diagrams/.cd,
  arrow style=tikz,diagrams={>=stealth}} %% cool arrow head
\tikzset{shorten <>/.style={ shorten >=#1, shorten <=#1 } } %% allows shorter vectors

\usetikzlibrary{backgrounds} %% for boxes around graphs
\usetikzlibrary{shapes,positioning}  %% Clouds and stars
\usetikzlibrary{matrix} %% for matrix
\usepgfplotslibrary{polar} %% for polar plots
\usepgfplotslibrary{fillbetween} %% to shade area between curves in TikZ
\usetkzobj{all}
\usepackage[makeroom]{cancel} %% for strike outs
%\usepackage{mathtools} %% for pretty underbrace % Breaks Ximera
%\usepackage{multicol}
\usepackage{pgffor} %% required for integral for loops



%% http://tex.stackexchange.com/questions/66490/drawing-a-tikz-arc-specifying-the-center
%% Draws beach ball
\tikzset{pics/carc/.style args={#1:#2:#3}{code={\draw[pic actions] (#1:#3) arc(#1:#2:#3);}}}



\usepackage{array}
\setlength{\extrarowheight}{+.1cm}
\newdimen\digitwidth
\settowidth\digitwidth{9}
\def\divrule#1#2{
\noalign{\moveright#1\digitwidth
\vbox{\hrule width#2\digitwidth}}}
























%%This is to help with formatting on future title pages.
\newenvironment{sectionOutcomes}{}{}


\title{Fractions}

\begin{document}

\begin{abstract}
packages
\end{abstract}
\maketitle



\begin{definition}  \textbf{\textcolor{green!50!black}{Fractions}} 


\textbf{Fractions} are packages. 


Fractions are vertically arranged packages.  There is a horizontal bar separating a top position called the \textbf{numerator} and a bottom position called the \textbf{denominator}.



\[
\frac{numerator}{denominator}
\]


There is only one rule:

\begin{center}
\textbf{\textcolor{red!70!black}{The denominator of a fraction cannot equal $0$.}}
\end{center}


\end{definition}




\begin{warning}

We do not use ``typewriter fractions''.  We don't write things like 3/4.  We write $\frac{3}{4}$.


The reason is that typewriter fractions cause confusion.

\[
(x+1)(x+3)/2 e^x (x-4) = ???
\]


If you are going by the Order of Operations, then this says


\[
\frac{(x+1)(x+3)}{2} \, e^x (x-4) = ???
\]


But the author could mean any one of several fractions.  We can't tell.


This confusion often shows up in the middle of algebraic calculations and everything goes off the rails.

Use a horizontal bar !!!


\[
\frac{numerator}{denominator}
\]


\end{warning}


Fractions are tools we use to represent many mathematical objects.



\subsection*{Numbers} 



We use fractions to represent numbers.  Fractions give us an unlimited supply of representations for every number.

The number four can be represented with the following fractions:

\[
4 = \frac{12}{3} = \frac{-24}{-6} = \frac{4}{1} = \frac{1}{\tfrac{1}{4}} = \frac{\tfrac{1}{5}}{\tfrac{1}{20}} = \frac{100\pi}{25\pi}
\]








\textbf{\textcolor{purple!85!blue}{$\blacktriangleright$ 1}} 



Having an endless supply of representations is very helpful when thinking about numbers, especially the number $1$. \\


\[
1 = \frac{2}{2} = \frac{-7}{-7} = \frac{\pi}{\pi} = \frac{\sqrt{2}}{\sqrt{2}} 
\]


All of these options for the number $1$, help us find alternatives for another numbers.


\[
8 = \frac{8}{1} = \frac{8}{1} \cdot 1 = \frac{8}{1} \cdot \frac{3}{3} = \frac{24}{3}
\]




\begin{idea}

The number $1$ can be represented by any fraction of the form $\frac{N}{N}$, where $N$ is any number, except $0$.


\end{idea}







Multiplication by the number $1$ is one of our most important ways to compare numbers.




\begin{example} Comparing Numbers

How do we compare $\frac{17}{23}$ and $\frac{27}{37}$ ? 


\textbf{\textcolor{red!75!green}{explanation}} 



We multiply both by $1$.


\[
\frac{17}{23} = \frac{17}{23} \cdot 1 = \frac{17}{23} \cdot \frac{37}{37} = \frac{629}{851}
\]


\[
\frac{27}{37} = \frac{27}{37} \cdot 1 = \frac{27}{37} \cdot \frac{23}{23} = \frac{621}{851}
\]



\[
\frac{17}{23} > \frac{27}{37}
\]


\end{example}


All of the representations for $1$ follow the same pattern.  Both the numerator and the denominator are the same number, except for $0$. 



\begin{warning} 

$0$ can be ``in'' a denominator.

The denominator of a fraction cannot EQUAL $0$.

\end{warning}












\textbf{\textcolor{purple!85!blue}{$\blacktriangleright$ 0}} 


Similar to the number $1$, All of our fractional representations of $0$ follow a pattern.






\begin{idea}

The number $0$ can be represented by any fraction of the form $\frac{0}{N}$, where $N$ is any number, except $0$.


\end{idea}

\textbf{Note:} the reason $1$ and $0$ cannot be represented with a fraction whose denominator equals $0$ is because fractions cannot have denominators equal to $0$.






\begin{warning}


$\frac{N}{0}$ is NOT a fraction, no matter what $N$ is.



\end{warning}




\begin{example} $0$


For what values of $A$ is  $\frac{(A-3)(A-5)}{(A+2)(A-5)} = 0$ ?


\textbf{\textcolor{red!75!green}{explanation}} 

$\frac{(A-3)(A-5)}{(A+2)(A-5)} = 0$ when $A = 3$. \\


If $A = 5$, then $\frac{(A-3)(A-5)}{(A+2)(A-5)} = \frac{2\cdot0}{7\cdot0} = \frac{0}{0}$, which isn't a fraction.






\end{example}


\newpage

\subsection*{Ratios and Rates}

\hfill\break


We use fractions to represent ratios and rates between measurements. 




The rate  ``$24$ hours per day'' can be represented with the fraction $\frac{24 \, hours}{1 \, day}$. 



In this ``dimensional analysis'' context, $\frac{24 \, hours}{1 \, day} = 1$, since $24 \, hours = 1 \, day$. 


We use fractions to represent rates when thinking about dimensional analysis. 





\[
4 \, days = 4 \, days \cdot \frac{24 \, hours}{1 \, day} \cdot \frac{60 \, mins}{1 \, hour} \cdot \frac{60 \, secs}{1 \, min} = 345600 \, seconds
\]






\subsection*{Quotient Functions}

\hfill\break


We use fractions to represent quotient functions, which we will study in this course. 




\[
\frac{x+1}{\sqrt{x}}
\]


\[
\frac{e^{2t}}{\cos(t)}
\]


\[
\frac{\ln(2k+1)-5}{7 - |3k+6|}
\]



\subsection*{Arithmetic}

\hfill\break

No matter what you are representing with fractions, they all follow the same arithmetic. 



\begin{formula}

\[
\frac{A}{B} + \frac{C}{B} = \frac{A + C}{B}
\]

\end{formula}







\begin{formula}

\[
\frac{A}{B} \cdot \frac{C}{D} = \frac{A \cdot C}{B \cdot D}
\]

\end{formula}






\begin{example} Addition


Create a single fraction equivalent to the sum $\frac{4x+1}{x-2} + \frac{1}{x}$. 

\textbf{\textcolor{red!75!green}{explanation}} 



\[
\frac{4x+1}{x-2} + \frac{3}{x}
\]


\[
\frac{4x+1}{x-2} \cdot 1 + \frac{3}{x} \cdot 1
\]


\[
\frac{4x+1}{x-2} \cdot \frac{x}{x} + \frac{3}{x} \cdot \frac{x-2}{x-2}
\]



\[
\frac{(4x+1)x}{(x-2)x} + \frac{3(x-2)}{x(x-2)}
\]






\[
\frac{(4x+1)x + 3(x-2)}{(x-2)x} 
\]








\end{example}


















\subsection*{Comparing Numbers}

We compare numbers in two distinct ways: size and position.


\textbf{\textcolor{blue!55!black}{$\blacktriangleright$ Size}} 


Size means big and small.  Big means far away from $0$.  Small means very close to $0$.


A number can be big and negative or big and positive. 


A number can be small and negative or small and positive. 




\textbf{\textcolor{blue!55!black}{$\blacktriangleright$ Position}} 


Position refers to the number line.




\textbf{Less Than:} If $A$ is to the left of $B$ on the number line, then we use the phrase ``less than''. 


\begin{itemize}
\item $-6$ is less than $-2$.
\item $-3$ is less than $1$.
\item $4$ is less than $5$.
\end{itemize}


$<$ is our symbol for ``less than'.



\begin{itemize}
\item $-6 < -2$.
\item $-3 < 1$.
\item $4 < 5$.
\end{itemize}









\textbf{Greater Than:} If $A$ is to the right of $B$ on the number line, then we use the phrase ``greater than''. 


\begin{itemize}
\item $-2$ is greater than $-6$.
\item $1$ is greater than $-3$.
\item $5$ is greater than $4$.
\end{itemize}


$>$ is our symbol for ``greater than'.



\begin{itemize}
\item $-2 > -6$.
\item $1 > -3$.
\item $5 > 4$.
\end{itemize}





Size and position are separate characteristics.

\begin{warning}  \textbf{\textcolor{red!70!black}{Size and Position}}

A number can be less than and bigger than another number.


$-10$ is less than $2$, since $-10$ is to the left of $2$ on the number line.


$-10$ is bigger than $2$, since it is farther away from $0$.

\end{warning}


\begin{warning}  \textbf{\textcolor{red!70!black}{Size and Position}}

A number can be greater than and smaller than another number.


$3$ is greater than $-5$, since $2$ is to the right of $-5$ on the number line.


$3$ is smaller than $-5$, since it is closer to $0$.

\end{warning}



\begin{definition} \textbf{\textcolor{green!50!black}{Absolute Value}} 


Big and small refer to the distance from $0$.  We have a symbol for the ``distance from $0$''. It is two vertical bars.

$| r |$ represents the distance between $0$ and $r$ on the real number line.

$| r |$ is called the \textbf{absolute value of $r$}.


$| r |$ is postive or $0$. It is never negative, since distance is never negative.

\end{definition}





We often think of subtraction has giving the distance between two numbers. However, this only works if the difference has a positive value, because distance cannot be negative.


\begin{itemize}
\item $7-3$ represents the distance between $7$ and $3$, 
\item whereas $3-7$ does not.
\end{itemize}

Of course $3-7$ is just the negative of $7-3$. 


$7-3$ and $3-7$ are two numbers that are the same distance from $0$. 


 Therefore, the distance of either $7-3$ or $3-7$ from $0$ is the also the distance between $3$ and $7$.  


\begin{idea} \textbf{\textcolor{purple}{Distance}} 

The distance between $A$ and $B$ is given by $| A - B |$ or $| B - A |$.


\end{idea}







\subsection*{Size}

We use fractions extensively to represent numbers and our ideas of big and small can often be deduced from the sizes of the numerator and denominator of the fraction.



$\frac{small}{big} = small$



$\frac{big}{small} = big$



Not always though. 


$\frac{small}{small} = anything$


$\frac{big}{big} = anything$



\begin{example} Size



$10000000000000000$ and $10000000$ are both big numbers.



\[
\frac{10000000000000000}{10000000} = big
\]


\[
\frac{10000000}{10000000000000000} = small
\]


\end{example}



Since the fractions of the form $\frac{small}{small}$  or  $\frac{big}{big}$ could represent big or small numbers, we know that we cannot immediately determine the overall size of the fractions.  We use the word ``indeterminate'' for such expressions.




















\subsection*{Cancelling}


\begin{warning} \textbf{\textcolor{red!70!black}{Cancelling}}



\begin{center}
\textbf{\textcolor{red!70!black}{There is no such thing as cancelling !!!}}
\end{center}


Cancellation is a figment of your imagination.  

There is no such thing in mathematics as ``slash things in the numerator and denominator that look the same''.

There is no need to jump out of mathematics to get the right answer.

In fact, jumping out of mathematics and performing nonsense procedures virtually assure you that you will get the wrong answer.


Stay inside mathematics.



\textbf{\textcolor{blue!55!black}{Step 1)}}  Factor the numerator

\textbf{\textcolor{blue!55!black}{Step 2)}}  Factor the denominator

\textbf{\textcolor{blue!55!black}{Step 3)}} If the numerator and denominator have a common factor, then you can factor the fraction into two fractions, where one of the factors represents $1$.

\textbf{\textcolor{blue!55!black}{Step 4)}} Anything times $1$ is itself.



That is how you stay inside mathematics and achieve the same goal.

Factoring out fractions that represent $1$ will work every time and avoid all of the trouble and frustration that comes with cancelling.




\end{warning}








\begin{onlineOnly}
\begin{center}
\textbf{\textcolor{green!50!black}{ooooo-=-=-=-ooOoo-=-=-=-ooooo}} \\

more examples can be found by following this link\\ \link[More Examples of Real-Valued Functions]{https://ximera.osu.edu/csccmathematics/precalculus/precalculus/realValued/examples/exampleList}

\end{center}
\end{onlineOnly}











\end{document}

