\documentclass{ximera}

%\usepackage{todonotes}

\newcommand{\todo}{}

\usepackage{esint} % for \oiint
\ifxake%%https://math.meta.stackexchange.com/questions/9973/how-do-you-render-a-closed-surface-double-integral
\renewcommand{\oiint}{{\large\bigcirc}\kern-1.56em\iint}
\fi


\graphicspath{
  {./}
  {ximeraTutorial/}
  {basicPhilosophy/}
  {functionsOfSeveralVariables/}
  {normalVectors/}
  {lagrangeMultipliers/}
  {vectorFields/}
  {greensTheorem/}
  {shapeOfThingsToCome/}
  {dotProducts/}
  {partialDerivativesAndTheGradientVector/}
  {../productAndQuotientRules/exercises/}
  {../normalVectors/exercisesParametricPlots/}
  {../continuityOfFunctionsOfSeveralVariables/exercises/}
  {../partialDerivativesAndTheGradientVector/exercises/}
  {../directionalDerivativeAndChainRule/exercises/}
  {../commonCoordinates/exercisesCylindricalCoordinates/}
  {../commonCoordinates/exercisesSphericalCoordinates/}
  {../greensTheorem/exercisesCurlAndLineIntegrals/}
  {../greensTheorem/exercisesDivergenceAndLineIntegrals/}
  {../shapeOfThingsToCome/exercisesDivergenceTheorem/}
  {../greensTheorem/}
  {../shapeOfThingsToCome/}
  {../separableDifferentialEquations/exercises/}
  {vectorFields/}
}

\newcommand{\mooculus}{\textsf{\textbf{MOOC}\textnormal{\textsf{ULUS}}}}

\usepackage{tkz-euclide}
\usepackage{tikz}
\usepackage{tikz-cd}
\usetikzlibrary{arrows}
\tikzset{>=stealth,commutative diagrams/.cd,
  arrow style=tikz,diagrams={>=stealth}} %% cool arrow head
\tikzset{shorten <>/.style={ shorten >=#1, shorten <=#1 } } %% allows shorter vectors

\usetikzlibrary{backgrounds} %% for boxes around graphs
\usetikzlibrary{shapes,positioning}  %% Clouds and stars
\usetikzlibrary{matrix} %% for matrix
\usepgfplotslibrary{polar} %% for polar plots
\usepgfplotslibrary{fillbetween} %% to shade area between curves in TikZ
%\usetkzobj{all}
\usepackage[makeroom]{cancel} %% for strike outs
%\usepackage{mathtools} %% for pretty underbrace % Breaks Ximera
%\usepackage{multicol}
\usepackage{pgffor} %% required for integral for loops



%% http://tex.stackexchange.com/questions/66490/drawing-a-tikz-arc-specifying-the-center
%% Draws beach ball
\tikzset{pics/carc/.style args={#1:#2:#3}{code={\draw[pic actions] (#1:#3) arc(#1:#2:#3);}}}



\usepackage{array}
\setlength{\extrarowheight}{+.1cm}
\newdimen\digitwidth
\settowidth\digitwidth{9}
\def\divrule#1#2{
\noalign{\moveright#1\digitwidth
\vbox{\hrule width#2\digitwidth}}}




% \newcommand{\RR}{\mathbb R}
% \newcommand{\R}{\mathbb R}
% \newcommand{\N}{\mathbb N}
% \newcommand{\Z}{\mathbb Z}

\newcommand{\sagemath}{\textsf{SageMath}}


%\renewcommand{\d}{\,d\!}
%\renewcommand{\d}{\mathop{}\!d}
%\newcommand{\dd}[2][]{\frac{\d #1}{\d #2}}
%\newcommand{\pp}[2][]{\frac{\partial #1}{\partial #2}}
% \renewcommand{\l}{\ell}
%\newcommand{\ddx}{\frac{d}{\d x}}

% \newcommand{\zeroOverZero}{\ensuremath{\boldsymbol{\tfrac{0}{0}}}}
%\newcommand{\inftyOverInfty}{\ensuremath{\boldsymbol{\tfrac{\infty}{\infty}}}}
%\newcommand{\zeroOverInfty}{\ensuremath{\boldsymbol{\tfrac{0}{\infty}}}}
%\newcommand{\zeroTimesInfty}{\ensuremath{\small\boldsymbol{0\cdot \infty}}}
%\newcommand{\inftyMinusInfty}{\ensuremath{\small\boldsymbol{\infty - \infty}}}
%\newcommand{\oneToInfty}{\ensuremath{\boldsymbol{1^\infty}}}
%\newcommand{\zeroToZero}{\ensuremath{\boldsymbol{0^0}}}
%\newcommand{\inftyToZero}{\ensuremath{\boldsymbol{\infty^0}}}



% \newcommand{\numOverZero}{\ensuremath{\boldsymbol{\tfrac{\#}{0}}}}
% \newcommand{\dfn}{\textbf}
% \newcommand{\unit}{\,\mathrm}
% \newcommand{\unit}{\mathop{}\!\mathrm}
% \newcommand{\eval}[1]{\bigg[ #1 \bigg]}
% \newcommand{\seq}[1]{\left( #1 \right)}
% \renewcommand{\epsilon}{\varepsilon}
% \renewcommand{\phi}{\varphi}


% \renewcommand{\iff}{\Leftrightarrow}

% \DeclareMathOperator{\arccot}{arccot}
% \DeclareMathOperator{\arcsec}{arcsec}
% \DeclareMathOperator{\arccsc}{arccsc}
% \DeclareMathOperator{\si}{Si}
% \DeclareMathOperator{\scal}{scal}
% \DeclareMathOperator{\sign}{sign}


%% \newcommand{\tightoverset}[2]{% for arrow vec
%%   \mathop{#2}\limits^{\vbox to -.5ex{\kern-0.75ex\hbox{$#1$}\vss}}}
% \newcommand{\arrowvec}[1]{{\overset{\rightharpoonup}{#1}}}
% \renewcommand{\vec}[1]{\arrowvec{\mathbf{#1}}}
% \renewcommand{\vec}[1]{{\overset{\boldsymbol{\rightharpoonup}}{\mathbf{#1}}}}

% \newcommand{\point}[1]{\left(#1\right)} %this allows \vector{ to be changed to \vector{ with a quick find and replace
% \newcommand{\pt}[1]{\mathbf{#1}} %this allows \vec{ to be changed to \vec{ with a quick find and replace
% \newcommand{\Lim}[2]{\lim_{\point{#1} \to \point{#2}}} %Bart, I changed this to point since I want to use it.  It runs through both of the exercise and exerciseE files in limits section, which is why it was in each document to start with.

% \DeclareMathOperator{\proj}{\mathbf{proj}}
% \newcommand{\veci}{{\boldsymbol{\hat{\imath}}}}
% \newcommand{\vecj}{{\boldsymbol{\hat{\jmath}}}}
% \newcommand{\veck}{{\boldsymbol{\hat{k}}}}
% \newcommand{\vecl}{\vec{\boldsymbol{\l}}}
% \newcommand{\uvec}[1]{\mathbf{\hat{#1}}}
% \newcommand{\utan}{\mathbf{\hat{t}}}
% \newcommand{\unormal}{\mathbf{\hat{n}}}
% \newcommand{\ubinormal}{\mathbf{\hat{b}}}

% \newcommand{\dotp}{\bullet}
% \newcommand{\cross}{\boldsymbol\times}
% \newcommand{\grad}{\boldsymbol\nabla}
% \newcommand{\divergence}{\grad\dotp}
% \newcommand{\curl}{\grad\cross}
%\DeclareMathOperator{\divergence}{divergence}
%\DeclareMathOperator{\curl}[1]{\grad\cross #1}
% \newcommand{\lto}{\mathop{\longrightarrow\,}\limits}

% \renewcommand{\bar}{\overline}

\colorlet{textColor}{black}
\colorlet{background}{white}
\colorlet{penColor}{blue!50!black} % Color of a curve in a plot
\colorlet{penColor2}{red!50!black}% Color of a curve in a plot
\colorlet{penColor3}{red!50!blue} % Color of a curve in a plot
\colorlet{penColor4}{green!50!black} % Color of a curve in a plot
\colorlet{penColor5}{orange!80!black} % Color of a curve in a plot
\colorlet{penColor6}{yellow!70!black} % Color of a curve in a plot
\colorlet{fill1}{penColor!20} % Color of fill in a plot
\colorlet{fill2}{penColor2!20} % Color of fill in a plot
\colorlet{fillp}{fill1} % Color of positive area
\colorlet{filln}{penColor2!20} % Color of negative area
\colorlet{fill3}{penColor3!20} % Fill
\colorlet{fill4}{penColor4!20} % Fill
\colorlet{fill5}{penColor5!20} % Fill
\colorlet{gridColor}{gray!50} % Color of grid in a plot

\newcommand{\surfaceColor}{violet}
\newcommand{\surfaceColorTwo}{redyellow}
\newcommand{\sliceColor}{greenyellow}




\pgfmathdeclarefunction{gauss}{2}{% gives gaussian
  \pgfmathparse{1/(#2*sqrt(2*pi))*exp(-((x-#1)^2)/(2*#2^2))}%
}


%%%%%%%%%%%%%
%% Vectors
%%%%%%%%%%%%%

%% Simple horiz vectors
\renewcommand{\vector}[1]{\left\langle #1\right\rangle}


%% %% Complex Horiz Vectors with angle brackets
%% \makeatletter
%% \renewcommand{\vector}[2][ , ]{\left\langle%
%%   \def\nextitem{\def\nextitem{#1}}%
%%   \@for \el:=#2\do{\nextitem\el}\right\rangle%
%% }
%% \makeatother

%% %% Vertical Vectors
%% \def\vector#1{\begin{bmatrix}\vecListA#1,,\end{bmatrix}}
%% \def\vecListA#1,{\if,#1,\else #1\cr \expandafter \vecListA \fi}

%%%%%%%%%%%%%
%% End of vectors
%%%%%%%%%%%%%

%\newcommand{\fullwidth}{}
%\newcommand{\normalwidth}{}



%% makes a snazzy t-chart for evaluating functions
%\newenvironment{tchart}{\rowcolors{2}{}{background!90!textColor}\array}{\endarray}

%%This is to help with formatting on future title pages.
\newenvironment{sectionOutcomes}{}{}



%% Flowchart stuff
%\tikzstyle{startstop} = [rectangle, rounded corners, minimum width=3cm, minimum height=1cm,text centered, draw=black]
%\tikzstyle{question} = [rectangle, minimum width=3cm, minimum height=1cm, text centered, draw=black]
%\tikzstyle{decision} = [trapezium, trapezium left angle=70, trapezium right angle=110, minimum width=3cm, minimum height=1cm, text centered, draw=black]
%\tikzstyle{question} = [rectangle, rounded corners, minimum width=3cm, minimum height=1cm,text centered, draw=black]
%\tikzstyle{process} = [rectangle, minimum width=3cm, minimum height=1cm, text centered, draw=black]
%\tikzstyle{decision} = [trapezium, trapezium left angle=70, trapezium right angle=110, minimum width=3cm, minimum height=1cm, text centered, draw=black]


\title{Set Symbols}

\begin{document}

\begin{abstract}
vocabulary
\end{abstract}
\maketitle




Mathematics is a language.  It uses symbols and vocabulary and language and pronunciation. \\




\subsubsection*{Sets}



We will be investigating functions that are defined on sets. Naturally, we have symbols for sets. \\



\textbf{\textcolor{blue!55!black}{Membership:}}  The symbol $\in$ means ``is a member of''.


\[
7 \in \{ 4, 5, 7, 8 \}
\]

\begin{center}

$7$ is a member of the set $\{ 4, 5, 7, 8 \}$.

\end{center}

People will also say ``$7$ is an element of'' $\{ 4, 5, 7, 8 \}$. \\

People will also say ``$7$ is in'' $\{ 4, 5, 7, 8 \}$. \\



\textbf{\textcolor{blue!55!black}{Subset:}}  The symbol $\subset$ means ``is a proper subset of''.


\[
\{ 4, 5 \} \subset \{ 4, 5, 7, 8 \}
\]

\begin{center}

$\{ 4, 5 \}$ is a subset of the set $\{ 4, 5, 7, 8 \}$.

\end{center}


Every member of $\{ 4, 5 \}$ is also a member of $\{ 4, 5, 7, 8 \}$. \\


\textbf{Proper} means that the subset is not equal to the larger set. \\


The symbol $\subseteq$ means ``is a subset of''.  This allows the possibility of the subset being equal to the larger set.

\[
\{ 4, 5, 7, 8 \} \subseteq \{ 4, 5, 7, 8 \}
\]









\textbf{\textcolor{blue!55!black}{Union:}}  The symbol $\cup$ stands for ``union''.


The union of two sets is another set.  The union contains all of the members of the two original sets.

\[
\{ 1, 2, 3 \} \cup \{ 4, 5, 7, 8 \} = \{ 1, 2, 3, 4, 5, 7, 8 \}
\]








\textbf{\textcolor{blue!55!black}{Intersection:}}  The symbol $\cap$ stands for ``intersection''.

The intersection of two sets is another set.  The intersection contains all of the members shared by the two original sets.

\[
\{ 1, 2, 3, 4, 5 \} \cap \{ 4, 5, 7, 8 \} = \{ 4, 5 \}
\]








\textbf{\textcolor{blue!55!black}{Empty Set:}}  The symbol $\emptyset$ stands for ``the empty set''.

The empty set is a set.  It just contains no members.














\subsection*{Numbers}

We have some standard sets of numbers and they have special symbols. \\



\begin{itemize}

\item $\mathbb{N}$ : the natural numbers
\item $\mathbb{Z}$ : the integers
\item $\mathbb{Q}$ : the rational numbers
\item $\mathbb{R}$ : the real numbers
\item $\mathbb{C}$ : the complex numbers
\end{itemize}

















\subsection*{Small}



As we move toward Calculus, our attention will focus on ``close''... a lot! \\


The word \textbf{instantaneous} will describe our ``close'' measurements. \\


So, we will use our symbols, notation, and language to help us talk about ``close''. \\


We have two symbols from the Greek language that we traditionally use to mean ``a small positive number''. \\



\begin{notation} \textbf{\textcolor{red!80!black}{Small}} 


\textbf{\textcolor{blue!55!black}{epsilon:}}  $\epsilon$ \\

\textbf{\textcolor{blue!55!black}{delta:}}  $\delta$ \\


\end{notation}

$\epsilon$ and $\delta$ are used to mean ``a very small positive number''. \\  


$\epsilon$ and $\delta$ are how we talk algebraically about ``close''. \\





\begin{warning} \textbf{\textcolor{red!80!black}{Not Specific Numbers}} 


$\epsilon$ and $\delta$ are not specific numbers with specific numeric values, like $\pi$. \\


$\epsilon$ and $\delta$ are used to represent very small positive numbers, but not specific small positive numbers. \\


So, $\epsilon$ and $\delta$ are used as constants - small positive constants.


When we use $\epsilon$ and $\delta$, we are talking about numbers smaller than $0.000000000001$. 


\end{warning}


\newpage

\begin{idea} \textbf{\textcolor{blue!55!black}{Big and Small}} 

\textbf{\textcolor{blue!55!black}{$\blacktriangleright$}}  ``small'' means ``close'' to $0$. \\


A number can be small and negative or a number can be small and positive. Small just means really close to $0$. \\


\textbf{\textcolor{blue!55!black}{$\blacktriangleright$}}  ``big'' or ``large'' means ``far way from'' to $0$. \\


A number can be big and negative or a number can be big and positive. Big just means far awy from $0$. \\


\end{idea}


``Big'' and ``small'' are size words.  They describe how close a number is to $0$. \\

Separate from this are ``greater than'' and ``less than''.  These are position words. \\

\begin{idea} \textbf{\textcolor{blue!55!black}{Left and Right}} 


\textbf{\textcolor{blue!55!black}{$\blacktriangleright$}} ``lesser'' or ``less than'' mean to the left of on the number line. These words describe realtive position.\\


\textbf{\textcolor{blue!55!black}{$\blacktriangleright$}} ``Greater'' or ``less than'' mean to the right of on the number line. These words describe realtive position.\\




\end{idea}








\begin{example}


The number $-7$ is less than the number $3$, but $3$ is smaller than $-7$.



\end{example}



\begin{example}


The number $-9$ is less than the number $2$, but $-9$ is bigger than $2$.



\end{example}






Numbers get bigger as they move away from $0$ in either direction. \\




We will have many situations where numbers approach $-\infty$.  That is they get bigger negatively.  They don't get smaller.  They head to the left on the number line. They get lesser, just negatively.  They get lesser and bigger. \\



Small and less are synonyms if you are only talking about positive numbers. With negative numbers, the story is much more interesting.











\begin{onlineOnly}
\begin{center}
\textbf{\textcolor{green!50!black}{ooooo-=-=-=-ooOoo-=-=-=-ooooo}} \\

more examples can be found by following this link\\ \link[More Examples of Real-Valued Functions]{https://ximera.osu.edu/csccmathematics/precalculus/precalculus/realValued/examples/exampleList}

\end{center}
\end{onlineOnly}





\end{document}
